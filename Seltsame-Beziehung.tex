% Sollen die Kommentare des Autors im PDF enthalten sein? Dann unten entsprechen aktivieren/deaktivieren

% Sollen die Ab achtzehn-Sachen im PDF enthalten sein?
\newcommand{\fsk}{18}

% ---------------
% Ab hier ändern
% ---------------


% damit Texniccenter die Dateien in der Übersicht öffnen kann, muss der Inhalt den \mainpath in den input-Befehlen ersetzen
\IfFileExists{./includes/IncludedPackages.tex}
{\newcommand{\mainpath}{.}}%{true-branch}
{\newcommand{\mainpath}{../../..}}%{false-branch}

\newcommand{\hogwartslogo}{gray}
\newcommand{\pdfcreationdef}{D:20080212190426}
\newcommand{\pdfkeywordsdef}{Frauen, Hermine, Cho, Ginny, Pansy, Luna}
\newcommand{\titledef}{Eine seltsame Beziehung}
\newcommand{\authordef}{Daniel Brunner}
\newcommand{\yeardef}{2008-2024} % 2008-2014 geschrieben
\newcommand{\published}{2. Ausgabe 2024} % 1. Ausgabe 2015
\newcommand{\partdef}{Teil eins der Elber-Reihe}
\newcommand{\firstpublished}{Zuerst veröffentlicht auf 
\href{https://www.fanfiktion.de/s/52cab44f000207a162539d8/1/Eine-seltsame-Beziehung}{Fanfiktion.de}
\footnote{https://www.fanfiktion.de/s/52cab44f000207a162539d8/1/Eine-seltsame-Beziehung} von \authordef}

\newcommand{\additionalsdef}
{
	\spoiler{Die ersten fünf originalen Bücher.}

	\summary{In Harrys sechstem Jahr macht er eine Entdeckung, die sein Liebesleben auf den Kopf stellt. Harry fühlt sich plötzlich zu Luna hingezogen und durchlebt eine aufregende Zeit mit ihr. Ihre Vereinigung ist mit nichts zu vergleichen. Doch auch die anderen Mädchen Hogwarts entdecken Harrys Reize; sogar die Lehrerinnen. Doch irgendwann entdeckt er, dass sein Herz an Ginny hängt. Und noch einer mischt Harrys Leben auf. Einer seiner Vorfahren hat Verbindung mit ihm aufgenommen. Und dieser hat keinen guten Ruf. Außerdem hat er wieder einen neuen Lehrer in Verteidigung gegen die dunklen Künste. Dieser Lehrer am Anfang sympathisch wird mit der Zeit Harry leicht unheimlich. Was hat er zu verbergen?}

    \disclaimerStd{}
}

% \hfuzz=1pt
% \vbadness=10000
% \clubpenalty = 500
\raggedbottom

% ---------------
% Bis hier ändern
% ---------------

\input{\mainpath/includes/IncludedPackages.tex}
\input{\mainpath/includes/Harry-Potter/IncludedHPPackages.tex}

\providecommand{\fsk}{0}
\ifthenelse{\equal{\fsk}{18}}
{
    % uncomment to include stuff in standard comment-environment
    \includecomment{abAchtzehn}
    \excludecomment{safedivide}
}{
    % define a abAchtzehn env which content is excluded
    \excludecomment{abAchtzehn}
    \includecomment{safedivide}
}

\hyphenation{Durm-strang-Di-rek-tor Quid-ditch-Welt-meis-ter-schaft Sly-the-rin Sly-the-rin-Tisch min-der-jährige Ers-tens Zwei-tens Mac-Cor-na-hiew"-Mannschaft}

\listfiles
\begin{document}
\input{\mainpath/includes/Titel/Harry-Potter/HP-Titelseite.tex}
\input{\mainpath/includes/Harry-Potter/DocumentStart.tex}

\chapter{Eine Squib}


Nachdem der Hogwarts-Express eingetroffen war, Harry sich von seinen Mitschülern verabschiedet hatte und von seinem Onkel abgeholt wurde, hievte er seinen Koffer in das Auto und schloss den Kofferraum. Er setzte sich hinten in das Auto auf den Rücksitz neben Hedwig, die in ihrem Käfig saß. Sein Onkel fuhr los und brachte ihn nach Hause. Er konnte nicht erwarten, dass sein Onkel viel mit ihm redete, wenn überhaupt. Nicht nachdem er ihn frisch von der Schule abgeholt hatte. Er wusste, dass ihn sein Onkel und seine Tante nicht besonders mochten. Und selbst Dudley, der in seinem Alter war, mied ihn, als hätte er eine ansteckende Krankheit.

\enquote{Ist mit Tante Petunia und Dudley irgendetwas?}, fragte Harry.

\enquote{Wieso? Was soll mit ihnen sein?}, sagte Onkel Vernon schnippisch. 

\enquote{Na ja, sonst sind sie und Dudley immer mitgekommen}, sagte Harry.

\enquote{Sie hat zu Hause zu tun. Dudley braucht neue Schulsachen}, antwortete Onkel Vernon ihm.

Harry hatte das Gefühl, jetzt nicht weiter mit Onkel Vernon reden zu können, also schwieg er. Zu Hause angekommen lud er wieder seine Schulsachen aus dem Auto, schleppte seinen Koffer in den Hausgang und holte Hedwig in ihrem Käfig nach. Onkel Vernon maulte ihn an, dass er seinen Koffer in sein Zimmer räumen sollte und verschwand in der Küche. Harry holte schnell einen Aufkleber aus dem Koffer, klebte ihn auf und verschloss den Koffer wieder sorgsam. Der Aufkleber sorgte dafür, dass der Koffer leichter wurde und nur einen Millimeter über dem Boden schwebte, denn Harry war noch keine siebzehn und dufte daher außerhalb der Schule nicht zaubern. Dann ließ er ihn in sein Zimmer schweben. Harry lächelte, als er seinen Koffer ohne Anstrengung aufgeräumt hatte. Er stellte Hedwigs Käfig auf dem Tisch ab und öffnete die Käfigtür und sein Zimmerfenster. Hedwig flog etwas im Zimmer umher und landete auf seinem Zimmertisch. Eigenartigerweise auf einer Vorrichtung, die genauso aussah, als wäre sie für Eulen gemacht worden. Harry nahm sich vor, es Vernon und Petunia gegenüber nicht zu erwähnen. Er vermutete, dass es eine Art Geburtstagsgeschenk seines Onkels und seiner Tante sei, sein erstes überhaupt, mal abgesehen von ein Paar Socken oder einem Kleiderbügel, aber er wollte nicht riskieren es wieder zu verlieren. Sie würden bestimmt sagen, sie hätten es vergessen. Sein Geburtstag war zwar erst in ein paar Wochen, aber sein Onkel und seine Tante nahmen es wohl nicht so genau.

Diese waren in den ersten Tagen wesentlich entspannter als in den Jahren zuvor, fiel Harry auf. Aber er machte sich darüber keine Gedanken. Er richtete wie gewöhnlich das Abendessen, spülte ab und ging dann früh zu Bett. Vielleicht lag es auch nur daran, dass sie ihn nur noch diesen Sommer und nächsten Sommer bis zu seinem Geburtstag ertragen mussten. Denn dann wird er siebzehn und kann in der Zaubererwelt tun, was er will. Er darf dann auch außerhalb der Schule zaubern. Doch wenn er das zu Hause tat, würde er bestimmt von seinem Onkel hinaus geworfen werden.

In der zweiten Woche, als Harry mit seinem Cousin, seinem Onkel und seiner Tante am Frühstückstisch saß, flatterte eine Eule herein und setzte sich auf den Fensterrahmen des offenen Küchenfensters. Onkel Vernon erschrak wie jedes Mal und schnauzte Harry an.

\enquote{Was will der blöde Vogel schon wieder?}

Harry stand auf, ohne seine Frage zu beantworten. Er war es leid, ihm jedes Mal sagen zu müssen, dass er es nicht wusste. Er nahm den Brief und staunte, als er nicht an ihn, sondern an Tante Petunia adressiert war.

\begin{brief}
Petunia Dursley,

Ligusterweg 4

Little Whinging

Surrey
\end{brief}

\enquote{Tante Petunia, der ist für dich.} Harry überreichte ihr den Brief.

Unsicher nahm sie ihn entgegen und stieß einen Schrei aus, als sie auf die Rückseite sah. Am Wappen hatte Harry erkannt, dass er von Hogwarts kam. 

\gedanke{Er musste wohl von Dumbledore sein}, dachte er. \gedanke{Bestimmt war das wieder ein Brief so wie letztes Mal.} 

Als Onkel Vernon fragte: \enquote{Was ist denn los?} und nach dem Brief greifen wollte, wich sie zurück und zog sich schnell ins Wohnzimmer zurück. Onkel Vernon stand auf und wollte sich zu ihr setzen.

\enquote{Bleib, wo du bist, Vernon, gehe keinen Schritt weiter}, rief Tante Petunia mit zitternder Stimme. Sie öffnete den Brief und begann zu lesen. Als sie fertig war, sagte sie zu Harry, wie immer betont ärgerlich, aber trotzdem mit einem Zittern in der Stimme: \enquote{Die ganzen sechs Wochen, die du bei uns bist, entfernst du dich nicht weiter als} und sie blickte kurz auf ihren Brief zurück, \enquote{als 1 Kilometer.} Tränen begannen über ihr Gesicht zu laufen, als sie das Wohnzimmer verließ und die Treppen nach oben stürmte.

\enquote{Da siehst du, was du wieder angerichtet hast. Nicht nur, dass du dauernd Briefe bekommst von deinen abnormen Freunden, nein, jetzt bekommt auch noch deine Tante einen und bricht in Tränen aus.} Er stürmte den Gang hinaus und die Treppe hoch. Er wollte wohl seine Frau beruhigen.

Dudley unterdessen schien das ganze wenig zu stören, denn er aß immer noch. Harry nahm sich einen Streifen Schinken und gab ihn der Eule. Die verschlang ihn genüsslich, drehte sich um und flog weg. Gerade in dem Moment kam eine zweite Eule angeflogen. Harry nahm ihr den Brief aus dem Schnabel und überreichte auch ihr einen Streifen Schinken, dieses Mal jedoch von Onkel Vernons Teller. Er schaute auf den Brief. Dieser war an ihn adressiert. Vorsichtig schaute er über seine Schulter, aber Dudley mampfte immer noch sein Frühstück. Harry drehte sich wieder zum Fenster und sah die Eule davonfliegen. Er öffnete seinen Brief und las.

\begin{brief}
Lieber Harry,

ich habe das Zaubereiministerium gebeten, dir die Erlaubnis zu geben auch in den Ferien zu zaubern (natürlich nur, ohne dass es ein Muggel erfährt) und zu trainieren. Nachdem Voldemort so ein immens großes Interesse an dir zeigt und die Dementoren dich und deinen Cousin letzten Sommer überfallen hatten, mache ich mir ernsthafte Sorgen.

Besonders die Tatsache, dass eine gewisse Dame euch letztes Jahr nicht ausreichend unterrichtet hat, hat das Ministerium nachdenklich gemacht.

Es überdenkt zurzeit meinen Vorschlag. Du wirst zu gegebener Zeit eine Eule mit einem blauen Umschlag und einem Siegel des Ministeriums erhalten. Sollte das Ministerium dem zustimmen, darfst du während der Ferien nicht nur deine gelernten Zauber zur Verteidigung einsetzen, sondern du bereitest dich besser vor und wirst kräftig üben, wann immer sich dir die Gelegenheit bietet. Da ich annehme, dass Voldemort oder einer seiner Todesser deine Briefe abfangen könnte, nimm bitte den blauen Umschlag vom Ministerium, um mit Ron und Hermine zu kommunizieren. Entsprechende Briefe und Umschläge für Ron und Hermine liegen bereit und warten nur noch auf das OK vom Ministerium. Lege einen in einem Umschlag verschlossenen und adressierten Brief in den blauen Umschlag vom Ministerium, klappe ihn zu und berühre das Siegel des Zaubereiministeriums mit deinem Zauberstab. Der Brief wird Ron oder Hermine dann zugestellt. Warte aber mit dem Üben, bis du vom Ministerium das OK bekommen hast.

Ron und Hermine wissen nichts von deiner Ausnahmegenehmigung des Ministeriums. Sie wissen nur von den blauen Umschlägen als Eulenersatz.
\signumspace
Albus Dumbledore
\signumspace
PS: Wir werden nächstes Schuljahr wohl kräftig üben müssen.
\end{brief}

Harry grinste und schob seinen Brief in die Hosentasche. Er fühle kurz an seinem Rücken, ob sein Zauberstab noch da war. Er trug ihn seit seinem ersten Ferientag ständig bei sich. Nur für den Fall.

Onkel Vernon kam mit hochrotem Kopf wieder durch die Küchentür und Harry kam es fast so vor, als ob kleine Rauchwolken seinen Ohren entstiegen. 

\enquote{Komm Dudley, gehen wir Einkaufen. Wir brauchen noch neue Schulbücher für Smeltings. Und du Freak räumst den Tisch und spülst ab.}

Onkel Vernon nahm Dudley mit und stieg in den Wagen ein. Seine Tante saß schon und sie fuhren davon. Jetzt fiel Harrys Blick in das Wohnzimmer. Petunia hatte ihren Brief fallen lassen. Er ging hin und schaute vorsichtshalber noch einmal aus dem Fenster. Nachdem er ihn aufgehoben hatte, begann er zu lesen.

\begin{brief}
Liebe Petunia Dursley,

Ich brauche ihnen wohl nicht zu sagen, wie sehr wir uns alle um Harry sorgen. Nach dem Angriff letzten Sommer auf ihn und seinen Cousin Dudley, hat sich hier einiges geändert. Harry schwebt in höchster Gefahr, wenn er sich zu sehr von seinem Haus entfernt. Sie wissen es sicher noch, dass bei Ihnen Harry ein sicheres Zuhause hat. In Ihrem Haus und der näheren Umgebung ist er während seiner Ferienzeit sicher. Die Todesser werden alles versuchen, ihren Neffen zu bekommen. Beschützen Sie ihn so gut es geht. Bleiben Sie weiterhin stark.

Der Bannkreis für Apparitionen wurde auf 2 Kilometer erweitert (Plötzliches magisches Auftauchen an einem Ort). Bitte geben Sie gut auf ihren Neffen Acht und passen sie auf ihn auf. Sie werden ihn nicht mehr lange sehen, falls Voldemort noch länger unter uns weilt.
\end{brief}

Am Freitag danach, nachdem alle mit dem Frühstück fertig waren, meinte Onkel Vernon: \enquote{Vorhin hat der Schneider angerufen. Wir sollen mit Dudley doch mal vorbeikommen, um zu sehen, ob seine neue Uniform denn passt. Und du, du weißt, was deine Aufgaben sind.}

\enquote{Ja Onkel Vernon}, antwortete Harry und begann den Tisch abzuräumen. Seine Gedanken kreisten immer noch um den Brief. Seine Tante hatte ihm gesagt: \gedanke{Die ganzen sechs Wochen, die du bei uns bist, entfernst du dich nicht weiter als 1 Kilometer}. Wieso war sie so besorgt um ihn? Wusste sie etwas, was sie ihm nicht sagte, oder sagen wollte? Hatte sie Onkel Vernon je etwas davon erzählt?

Onkel Vernon verschwand mit Tante Petunia und Dudley durch die Tür nach draußen. Er hörte, wie sie einstiegen und startete den Motor. Als er in Gedanken versunken durchs Fenster hinaus in den Garten blickte, sah er wie eine Eule heranflog. Sie hatte einen blauen Umschlag im Schnabel. Er nahm ihr den Umschlag ab, nachdem er ihr seinen Arm hinhielt und sie darauf landete und drehte ihn um. Auf der Rückseite war das Siegel des Ministeriums zu sehen. Er gab der Eule noch etwas zu fressen und schaute ihr zu, wie sie wieder davon flog. Er öffnete den Umschlag und fand im Inneren einen kleinen Zettel.

\begin{brief}
Ausnahmegenehmigung für Harry Potter

Sehr geehrter Mister Potter,

ihnen wurde auf Bitten ihres Schulleiters Professor Dumbledore und einer ausgiebigen und langwierigen Prüfung, eine Ausnahmegenehmigung erteilt, in den sonst für minderjährige Zauberer schulfreien Zeit ihre Künste auszuüben und zu verfeinern.

Entsprechende zu ihrer Verfügung stehende Zauber sind auf beigelegtem Faltblatt aufgeführt, beziehungsweise deren Sparten, falls ein ganzer Unterbereich darunter fallen sollte.

Ich möchte Sie zudem darauf hinweisen, dass diese Genehmigung jederzeit widerrufen werden kann. Ein entsprechendes Schreiben wird ihnen dann persönlich überreicht werden und bekommt damit Gültigkeit.

Seien sie außerdem gewarnt, dass Sie trotz dieser Genehmigung weiterhin unter Beobachtung durch das Ministerium stehen, sodass ein entsprechender Regelverstoß laut §12a des MZSGB\footnote{Minderjährigen Zauberer Strafgesetzbuch} eine sofortige Rücknahme der Genehmigung und ein entsprechendes Verfahren nach sich ziehen wird.

In der Hoffnung, dass es nicht dazu kommen wird
\signumspace
gez. Mafalda Hopfkirch,

Abteilung für unbefugte Zauberei,

Ministerium für die Vernunft gemäße Einschränkung der Zauberei Minderjähriger.
\end{brief}

Harry konnte sich ein Grinsen nicht verkneifen. \gedanke{Mafalda Hopfkirch, das war doch die Dame, die mich fast aus der Schule geworfen hätte}, dachte er.  Es lag noch eine Kopie der Regeln bei, die er kurz überflog und danach einsteckte. Vorsichtig sah er sich um und begann die restlichen Teller in die Spüle zu stellen. Er zog seinen Zauberstab und zeigte mit einer schwingenden Bewegung auf die Teller und die Spüle. Die Teller begannen sich in die Luft zu erheben; das Wasser fing an aus dem Hahn zu plätschern und die Spülbürste wurde von dem Wasserstrahl umspült. Die Teller näherten sich und auf jeden einzelnen tropfte eine kleine Menge Spülmittel, die aus einer ebenfalls schwebenden Flasche kam. Die Handtücher flogen von ihren Haken und trockneten die sauberen Teller ab. Nachdem das ganze Geschirr sauber gewaschen war, verstummte das Wasser aus dem Hahn; das Spülmittel und die Bürste nahmen wieder ihren Platz ein. Harry schwang erneut seinen Zauberstab und die Teller und Tassen, die Pfannen und das Besteck begannen sich erneut anzuheben. Die Schränke und Türen öffneten sich und die Dinge schwebten an ihren Platz. Die Schränke und Türchen schlossen sich mit einem satten \geraeusch{Bupp} und die Küche blitzte und funkelte, wie schon lange nicht mehr. Er schob seinen Zauberstab wieder ein und dachte: \gedanke{Das ist die einzig wahre Art Hausarbeit zu verrichten.}

Zufrieden mit seiner Arbeit ging er in sein Zimmer und packte erst einmal seinen Koffer aus. Danach setzte er sich an seinen Tisch, schlug ein Buch auf, nahm Tinte und Pergament und machte sich an die Hausaufgaben, die er zu lösen hatte. Durch das offene Fenster kam die beginnende vormittägliche Wärme in sein Zimmer und er spürte einen warmen Luftzug. Nach einer Weile hatte er einen Teil seiner Hausaufgaben für Professor Snape fertig. Das waren seiner Meinung nach die unangenehmsten und so hatte er sie als erstes erledigt. Zumindest den theoretischen Teil. Für den praktischen müsste er sich noch etwas überlegen, denn Zaubertränke brauen konnte er wohl schlecht in der Küche oder seinem Zimmer. Als das Auto seines Onkels die Einfahrt hereinfuhr, klappte er sein Buch zu und räumte die anderen Schulsachen weg. Er schaute aus seinem Fenster und bemerkte, wie Tante Petunia ein Prospekt und einige wichtige Dokumente auf ihrem Arm trug.

Er machte sich auf den Weg nach unten. Als er auf halbem Weg die Treppe heruntergegangen war, öffnete Onkel Vernon die Tür und ließ seine Frau und seinen Sohn herein. 

Onkel Vernon grinste Harry an. 

\gedanke{Das bedeutet nichts Gutes}, dachte Harry. Er ging die restlichen Stufen hinunter und schloss die Haustür, um dann auch in die Küche zu gehen. 

Mit zugekniffenen Augen und einem höchst befriedigtem Gesichtsausdruck sagte Vernon dann: \enquote{Wir fahren in den Urlaub.}

Harry wusste genau, wen er mit \accentuate{wir} meinte. 

\enquote{Und du}, fuhr sein Onkel fort, \enquote{wirst diese Zeit bei Miss Figg verbringen.}

Harry wollte schon anfangen zu grinsen, zwang dann aber doch seine Mundwinkel nach unten und senkte den Kopf.

\enquote{Du wirst dich anständig bei ihr aufführen. Und mach keinen Unsinn. Stell mir bloß nichts an}, sprach Onkel Vernon. Mit \accentuate{mach keinen Unsinn} meinte er Zaubern. Aber Harry hatte damit kein Problem. Glücklicherweise war es ihm ja erlaubt worden und Miss Figg war keine Muggel, sondern eine Squib. Er konnte ihr also ruhig im Haushalt helfen und dabei zaubern. \gedanke{Das werden lustige Ferien}, dachte Harry.

\enquote{Wir fahren nächsten Montag, also in drei Tagen. Du meldest dich derweil bei Miss Figg an und \gst sei mir ja anständig.}

Harry bejahte und verließ das Haus. Die Sonne fing an den Boden und die Luft weiter zu erwärmen. \gedanke{Vielleicht sollte ich doch anfangen Sport zu betreiben}, dachte er. Er machte sich auf den Weg zu Miss Figgs Wohnung. Seit dem Vorfall mit den Dementoren hatte er sie nicht mehr gesehen. Ihr Haus lag nur etwa 50~Meter von dem der Dursleys entfernt. Das sollte also noch innerhalb seines Bannkreises liegen, den Dumbledore in dem Brief an seine Tante erwähnt hatte. Bei Miss Figg angekommen klingelte er und nach kurzer Zeit öffnete Miss Figg ihre Haustüre.

\enquote{Harry}, sprach sie, \enquote{schön, dass du mich besuchen kommst. Komm herein.}

\enquote{Nein, nein Miss Figg. Ich habe keine Zeit. Ich wollte sie nur fragen, ob ich die paar Wochen, die mein Onkel und meine Tante mit Dudley im Urlaub sind, bei ihnen verbringen könnte.}

\enquote{Aber ja, Harry.}

Harry entwich nur ein: \enquote{Danke Miss Figg}, bevor er sich umdrehte und gehen wollte. Da fiel ihm ein, dass sie noch gar nicht wusste, wann er denn zu ihr kommen würde. Er fügte noch hinzu: \enquote{Montag früh.}

Wieder zu Hause bei seinem Onkel und zurück in der Küche sah Harry wie Tante Petunia ihre blitzblank geputzte Küche betrachtete.

\enquote{Harry, ausnahmsweise muss ich dich mal loben}, entfuhr es ihr.

Als er den erstaunten Blick seines Onkels auffing, musste er sich ein Lächeln verkneifen.

\enquote{Danke Tante Petunia, ich hatte auch genügend Zeit dazu.} 

Tante Petunia sah ihn an, als ob sie ihm nicht ganz glauben würde.

Am Montagmorgen, nachdem das Frühstück bereits um sechs Uhr stattgefunden hatte, räumte Harry gerade das Geschirr in die Spüle, als Onkel Vernon mit ein paar Koffern die Treppe herunterkam. \gedanke{Vermutlich lädt er sie in das Auto}, dachte Harry. Er war gerade dabei, mit einem feuchten Tuch den Tisch abzuputzen, als Onkel Vernon wieder hereinkam.

\enquote{Wir fahren jetzt los. Du räumst hier alles auf und machst dich dann auf den Weg zu Miss Figg.} Er drückte ihm einen Hausschlüssel in die Hand. \enquote{Den gibst du bei Miss Figg ab.} Harry bejahte.

Onkel Vernon drehte sich um, blieb kurz stehen und meinte dann: \enquote{Pass bloß auf, dass Miss Figg deine Eule nicht zu Gesicht bekommt.} Er verließ die Küche und ging durch den Flur. 

Dudley war bereits draußen, als seine Tante noch zu ihm sagte: \enquote{Nimm die Eulenstange mit.} Dann verließ auch sie die Wohnung und stieg ins Auto ein.

Harry ging ins Wohnzimmer und beobachtete wie Onkel Vernon in sein Auto einstieg, den Motor anließ und mit Tante Petunia und Dudley davon fuhr.

Harry grinste, zog seinen Zauberstab hervor und ließ den Abwasch sich selbst erledigen. Danach ging er in sein Zimmer, packte seinen Koffer und ließ ihn danach in den Flur schweben. Er zauberte kleine Räder an ein Ende des Koffers und machte ihn leichter, damit er nicht so schwer zu ziehen war.

Zurück in der Küche war das Geschirr bereits gespült. Harry räumte es, wie auch die Tage zuvor als er alleine abspülen musste, auf und machte sich auf den Weg zu Miss Figg.

\trenn

Harry bezog das Zimmer, das ihm Miss Figg für gewöhnlich gab, wenn er bei ihr über längere Zeit übernachtete. Sie trug Hedwig in ihrem Käfig die Stufen hinauf und Harry folgte ihr mit seinem Koffer, der leicht über dem Boden schwebte, denn er brauchte in Miss Figgs Haus keine Räder mehr. 

Sie öffnete Harrys Zimmerfenster und Hedwigs Käfigtür. Dann sagte sie: \enquote{Wenn du fertig bist, komm gleich zu mir. Wir müssen reden.}

Harry nickte und richtete sein Zimmer ein. Er öffnete seinen Koffer und stellte Hedwigs neue Stange auf dem kleinen Tisch in seinem Zimmer ab. Es war ein wenig kleiner als das, welches er bei den Dursleys bewohnte. Und dort war es das kleinste Zimmer im ganzen Haus. Er hatte es noch gut in Erinnerung. Damals durfte er bei Miss Figg nur wenig heraus. Aber seit letztem Sommer hatte sich die Einstellung Miss Figgs ihm gegenüber wohl verändert, denn die Einladung, nachher zu ihr in das Wohnzimmer zu kommen, sofern man es so nennen mochte, war ausgesprochen freundlich. Er verließ sein Zimmer und ging die Treppen hinab, um in das Wohnzimmer zu gelangen.

Dort saß sie bereits und wartete auf ihn.

\enquote{Harry, \gst es gibt ein paar wichtige Dinge, die du dir merken solltest}, sprach sie. \enquote{Erstens \gst vergiss nie deinen Zauberstab. Egal wo du auch hingehst. Zweitens \gst bleib immer in einem Umkreis von 1,5 Kilometern um das Haus deines Onkels und deiner Tante. Gehe nicht weiter, das wäre gefährlich. \gst Ach und übrigens, ich nehme an, du musst noch Hausaufgaben machen. Die kannst du hier im Wohnzimmer machen. Ich helfe dir auch gerne, wenn du willst. Und nenne mich in Zukunft Arabella.}

Harry wusste nicht, was er sagen sollte. So hatte er Miss Figg, oder Arabella, wie er sie jetzt nennen durfte, noch nie gesehen. Bisher war sie immer recht grob zu ihm gewesen und gab ihm das Gefühl, nicht so sehr willkommen zu sein. \gedanke{Das lag wohl daran, dass sie nicht wollte, dass die Dursleys glauben, ich würde mich hier wohlfühlen und mich nicht mehr zu ihr gehen lassen}, dachte Harry.

\enquote{Aber ich denke, sie sind eine Squib und können nicht zaubern?}, antwortete Harry.

\enquote{Das ist richtig}, lächelte sie, \enquote{aber ich habe ein Händchen für Tränke aller Art. Ob zur Abwehr von Krankheiten oder um die Sinne zu vernebeln.}

\enquote{Die habe ich schon fertig}, antwortete Harry leicht indigniert\footnote{empört, grimmig, entrüstet, unbeherrscht, ärgerlich, ungehalten}. \enquote{Aber Sie können sie gerne durchsehen. Vielleicht hilft mir das bei Snape.}

\enquote{Ach! Lehrt der immer noch?}, fragte Arabella.

\enquote{Ja! Sie kennen ihn?}, fragte Harry erstaunt.

\enquote{Du weißt doch, dass ich ihn kenne. Hast du schon vergessen, dass ich auch im Orden bin? Genau wie Severus!}

Harry kam sich blöd vor. Er hatte sie mehrere Male bei den Treffen gesehen, wo sie die neuesten Gerüchte aus der Muggelwelt erzählte. Er lief nach oben und holte seine Zauberbücher, seine Tinte und einige Rollen Pergament. Wieder unten angekommen, gab er Arabella seine Hausaufgaben von Snape und Sprout und machte sich an die anderen, die er noch zu erledigen hatte.

Gegen Mittag entschloss sich Harry nun endlich etwas für seine Figur zu tun und begann sich umzuziehen. Er sagte Arabella Bescheid, dass er noch ein wenig joggen ging und betonte dabei, dass er in der Nähe bleiben würde. Er verließ das Haus, kontrollierte, ob er noch seinen Zauberstab dabei hatte und fing an sich aufzuwärmen. Danach lief er den Block hinunter und zählte seine Schritte, damit er wusste, wann er wieder umkehren musste. Schweißgebadet kam er zurück und stellte sich unter die Dusche. \gedanke{Ab Morgen jogge ich nur noch frühmorgens, wenn es noch kühler ist}, dachte sich Harry. Er trocknete sich ab, zog sich wieder an und kam in das Esszimmer, wo Arabella gerade das Essen fertig hatte und es ihm auf seinen Platz stellte.

\enquote{Dein Aufsatz für Snape ist nicht schlecht, aber ich habe mir die Freiheit genommen, einige Anmerkungen dazuzuschreiben. Du findest sie auf dem Blatt darunter}, sagte sie.

Harry nickte und begann erst einmal zu essen. Während des Essens überkam ihm der Gedanke wie er denn nun in Form bleiben und seine Künste üben sollte. In Arabellas Haus ist nicht viel Platz. \gedanke{Darüber kann ich Morgen auch noch nachdenken}, dachte sich Harry und aß zu Ende. Arabella war eine gute Köchin und sie verwöhnte ihn mit den tollsten Dingen, die sich Harry vorstellen, oder manchmal auch nicht vorstellen, konnte.

Die erste Woche verlief relativ ruhig und Arabella hatte ihm inzwischen das du angeboten.

Harry stand schon bald auf und zog sich seine üblichen Laufsachen an. Es war Dienstag. Obwohl er keinen Trainingsanzug hatte, oder etwas, das als Sportkleidung durchging, hatte er eine normale kurze Hose und ein T-Shirt an. Er ging die Treppe hinunter und zur Tür hinaus, wärmte sich auf und begann einen anderen Weg zu laufen. Er kam an gepflegten Gärten vorbei, die schöner waren, als die seiner Tante. \gedanke{Vielleicht sollte ich in unserem Garten auch ein wenig graben}, dachte sich Harry. \gedanke{Aber nur nachts, wenn es keiner sieht, dass ich dazu keinen Spaten benutze.} Wieder bei Arabella angekommen, duschte sich Harry bevor er zum Frühstück gerufen wurde. Er zog sich schnell an und ging in die Küche um Arabella beim Decken des Tisches zu helfen.

\trenn

Als Harry am nächsten Morgen die Treppen heruntergegangen war, um noch was zu Trinken bevor er wieder loslief, fand er in der Küche einen kleinen Zettel auf dem Tisch.

\begin{brief}
Lieber Harry,

Ich bin kurz Einkaufen. Wir brauchen neue Lebensmittel. Tue nichts Unüberlegtes und bleib in der Nähe. Solltest du angegriffen werden, komm zurück zum Haus, oder das der Dursleys. Dort bist du sicher.
\signumspace
Grüße Arabella
\end{brief}

Harry schmunzelte. Er verließ das Haus, wärmte sich auf und machte sich auf den Weg. Vor lauter Aufregung über den überstürzten Aufbruch seines Onkels und seiner Tante nebst seines Neffen in den Urlaub, hatte er fast vergessen, dass er Morgen Geburtstag hatte. \gedanke{Geburtstag}, dachte Harry. \gedanke{Ich werde sechzehn Jahre alt.} Er lief mehrere Male um denselben Block, bevor er vor Arabellas Haus ankam und sich erst einmal duschte. Der Tag verlief wie immer recht ruhig und Harry erledigte wieder ein paar seiner Hausaufgaben. Er setzte sich ins Wohnzimmer und öffnete eines seiner Bücher um mit den Hausaufgaben von Professor Trelawney zu beginnen. Dieses Mal mussten sie etwas aus dem Rauch von Räucherstäbchen lesen. 

Als Arabella wieder nach Hause kam, roch das ganze Wohnzimmer und Teile der Küche nach Räucherstäbchen. Sie musste erst einmal ein Fenster öffnen, bevor sie überhaupt zu Wort kam. Wieder zurück im Wohnzimmer entdeckte sie Harry, der wie hypnotisiert in den Rauch der Stäbchen starrte. Sie gab ihm einen Schubs und weckte ihn so aus seiner Trance.

\enquote{Was hast du denn hier angestellt? Das halbe Haus riecht nach diesen verdammten Dingern.}

Harry wurde erst jetzt so richtig wach und entschuldigte sich. \enquote{Tut mir leid Arabella, aber das war die Hausaufgabe von Professor Trelawney. Wahrsagen. Ich hätte das wohl besser im Garten draußen gemacht.}

\enquote{Allerdings}, entgegnete ihm Arabella. \enquote{Das kriegen wir die nächsten Tage nicht wieder raus. Mach mir so etwas ja nicht noch einmal.}

Harry schluckte und senkte seinen Kopf. \enquote{Tut mir leid, kommt nicht wieder vor.} Dann fiel ihm etwas ein. Er zog seinen Zauberstab, murmelte etwas und lief im Wohnzimmer herum. Kleine gelbe Funken kamen aus seiner Spitze. Schnell war der Geruch verschwunden. \gedanke{Doch gut, dass Hermine darauf bestanden hat, mir ein paar Haushaltszauber beizubringen}, dachte er.

Nach dem Mittagessen setzte sich Harry an das nächste Fach. Kräuter- und Pflanzenkunde. Denn Arabella hatte bemerkt, dass er einen wichtigen Teil seiner Hausaufgaben vergessen hatte. Das war recht einfach und Arabella schaute ihm interessiert dabei zu. Sie verbesserte einige kleinere Fehler, die sie fand, bevor sie aufstand und das Abendessen herrichtete. Harry half ihr wie immer dabei. 

Nachdem es draußen dunkel geworden war, ging Harry nochmal raus, um zu dem Haus seines Onkels und seiner Tante zu gehen. Einige Zeit stand er vor dem Haus und sah es an. Die Luft war noch immer warm von den Sonnenstrahlen, die den ganzen Tag herab geschienen hatten. Nur der Schein der Straßenlaternen erhellte die Nacht. Die Sterne am Himmel funkelten durch die aufsteigende Wärme und Harry hatte fast den Eindruck, sie würden am Himmel tanzen. Er legte sich in Tante Petunias Garten und schaute einige Zeit in den Himmel. Eine einzelne Sternschnuppe flog hoch oben am Himmel vorbei und Harry wünschte sich etwas. Immerhin hatte er Morgen Geburtstag und er nahm nicht an, dass Arabella das wusste. Man könnte es ihr zwar gesagt haben, aber da war er sich nicht sicher. Da das Gras doch inzwischen etwas unangenehm war, entschied sich Harry nach einem prüfenden Blick in alle Richtungen es auf magische Art abzuschneiden. Er zog seinen Zauberstab und murmelte etwas. Viele kleine Funken sprühten aus seiner Spitze hervor und überdeckten die gesamte Rasenfläche bis auf den kleinsten Grashalm in der hintersten Ecke. Fast hatte es den Anschein, dass das Gras rückwärts wachsen würde. Als es die richtige Länge hatte, stoppte Harry den Zauber und legte sich wieder hin. Nachdem er etwa eine dreiviertel Stunde dort gelegen hatte, überkam ihn die Müdigkeit. Er entschloss sich aufzustehen und zurück zu Arabellas Haus zu laufen. Dort würde er ein gemütliches Bett vorfinden, in dem er schlafen konnte.

In seinem Bett, unter die Decke gekuschelt, schlief Harry friedlich ein und überlegte, was ihm wohl der Morgen bringen würde. \accentuate{Nicht viel}, dachte Harry. Nach dem üblichen morgendlichen Joggen und dem darauffolgenden Frühstück werden vielleicht ein paar Geburtstagsgrüße eintreffen. Ansonsten konnte er ja immer noch Hausaufgaben machen.

Am nächsten Morgen stand Harry wieder sehr früh auf, um draußen joggen zu gehen. Er zog sich an, ging die Treppen hinunter, öffnete die Haustüre und fing nach seinen üblichen Aufwärmübungen an zu laufen. Harry dachte nicht groß über seinen Geburtstag nach. \gedanke{Ein paar Glückwunschkarten von Ron und Hermine, vielleicht ein kleines Geschenk}, dachte sich Harry. \gedanke{Da ich nichts zu erwarten habe, laufe ich ein bisschen länger.} Wieder im Haus sagte er Arabella Bescheid und ging gleich die Treppe hoch um sich zu duschen. Er zog sich frische Sachen an und kam die Treppe herunter. Gerade wollte er den Gang entlang in die Küche laufen, als ihm Arabella entgegenkam und meinte.

\enquote{Harry, schau lieber mal nach deiner Eule. Als ich vorhin in deinem Zimmer war, schaute sie gar nicht gut aus.}

Harry drehte sich wieder um und ging die Treppen hoch. Arabella folgte ihm. Als er sein Zimmer erreichte, Hedwig intensiv angeschaut hatte und nichts Außergewöhnliches bemerkt hatte, drehte er sich zu Arabella um. 

\enquote{Ich sehe nichts Ungewöhnliches}, meinte er.

\enquote{Komisch, vorhin hatte sie richtig schlapp und Müde ausgesehen. Aber vielleicht habe ich mir das auch nur eingebildet. Ich habe hier nicht so oft Eulen. Tut mir leid}, sagte sie.

\enquote{Macht nichts. Hätte ja sein können.}

Harry ging wieder die Treppen hinunter und Arabella folgte ihm. Er ging in die Küche und sein Gesicht versteinerte.

\trenn

\accentuate{Einen Tag zuvor, an einer ganz anderen Stelle in England.}

Ein kleines Wohnzimmer, mit alten Möbeln aus Nussbaum war die Hauptkulisse der folgenden Szenen. Die Tapete an den Wänden war schon leicht vergilbt, da sie jahrzehntelang nicht gewechselt worden war. Seit der Geburt ihres Enkels hatte sie diesen Raum nicht mehr verändert. Die Möbel spiegelten die vergangenen Tage wider; alt aber wertvoll. Im hinteren Teil war ein Fenster, vor dem ein grobmaschiger, weißer Vorhang hing. In den Schränken mit Glasfront waren Gläser zu sehen: Champagnergläser, Cognacgläser, Weingläser und andere Gläser, die Muggel als eigentümlich bezeichnen würden. Sektflöten ohne Standteller, gläserne Pfeifen, die mit Flüssigkeit gefüllt werden konnten und beim Ziehen am Mundstück die Flüssigkeit als feinen Nebel abgaben um die Lungen zu reinigen. Direkt vor dem Fenster stand ein Sofa, welches mit einer Decke auf der Sitzfläche gegen durch Sitzen geschont wurde.

Im vorderen Teil hingegen waren helle und modernere Möbel zu sehen. Wäre eine Mauer durch den Raum gezogen, würde es nicht so Kontrastreich ausfallen. Der Tisch aus Buche, sowie die sechs Stühle drumherum, dienten als Esstisch für sämtliche Mahlzeiten.

Doch heute stand außer der üblichen Deko, einige schwebende Kugeln, frische Blumen in einer Vase, einem kleinen schmalen Läufer über dem Tisch, auch noch eine Geburtstagstorte mit Kerzen. Diese begannen gerade Feuer zu fangen, als von draußen Stimmen und Geräusche zu hören waren.

Die Haustür wurde geöffnet und man hörte zwei Paar Füße, die sich auf dem Schuhabstreifer abputzten.

\enquote{Augen zu}, sagte eine weibliche Stimme.

\enquote{Aber Grandma, bin ich dafür nicht schon ein wenig zu alt?}

\enquote{Nein, nun mach schon.} Kurz darauf stand der junge Mann vor der Tür und die alte Dame direkt hinter ihm.  Die Tür ging auf und beide traten hinein. \enquote{Augen auf.}

Neville öffnete seine Augen. Vor ihm stand eine Torte. Wie jedes Jahr. Und ein kleines verpacktes Geschenk. Doch kaum hatte sich die Tür hinter ihnen geschlossen und Neville die Kerzen aus gepustet, klopfte es.

\enquote{Machst du mal auf?}, bat ihn seine Oma. 

Neville nickte und öffnete die Tür. Erstaunt blieb er stehen und bewegte sich nicht. 

Dean, Seamus und Luna standen davor. \enquote{Alles Gute zum Geburtstag, Neville}, sagten die drei. \enquote{Dürfen wir hereinkommen?}

\enquote{Aber \gst klar doch, kommt rein.} Neville trat zur Seite und lies seine Freunde ein.

\enquote{Ron und Hermine haben mir ein Geschenk mitgegeben. Sie stecken noch in den Vorbereitungen für Harrys Feier}, sagte Luna und überreichte Neville ein kleines Geschenk.

Dankend nahm er es an und bat seine Gäste sich zu setzen. Dann sah er fragend seine Oma an.

\enquote{Du wirst nur einmal sechzehn, Neville}, sagte sie und verschwand. \enquote{Viel Spaß ihr vier.} Dann schloss sie die Tür hinter sich und verzog sich in die Küche. Während der Feier tauchte sie nur noch ein paar Mal auf um Nachschub an Getränken oder Nahrung zu liefern.

Nachdem alle ein Stück Kuchen gegessen hatten, fragte Dean vorsichtig nach. \enquote{Hast du deine Eltern heute schon besucht?}

\enquote{Ja}, antwortete Neville niedergeschlagen. \enquote{Keine Änderung. Sie erkennen niemanden.}

Luna legte eine Hand auf seinen Arm, was ihm ein Lächeln entlockte. Dean und Seasmus wechselten nur ein paar Blicke. Dann widmete sich Neville seinen Geschenken. Dean und Seasmus schenkten ihm ein Buch über seltene Pflanzen. Von seiner Oma bekam er dieses Jahr die Erlaubnis, sich um den Garten zu kümmern. Er konnte ihn dabei so gestalten wie er es wollte. Neville war begeistert. Nun konnte er mit Pflanzen und anderen Gewächsen experimentieren. Als er das Geschenk von Luna in Händen hielt, zögerte er kurz, öffnete es dann doch. Er wusste nicht, was ihn erwarten würde. Umso erstaunter war er, als er einen kleinen Pflanzenzögling fand.

\enquote{Das sind Lenkpflaumen. Dad hat mir einen Ableger gegeben. Wenn du dich um sie kümmerst, sind sie dir treu. Du kannst sie vor dein Haus pflanzen. Dann halten sie Feinde ab, die dir etwas Böses wollen. Zumindest warnen sie dich.}

Neville bedankte sich. Gerade wollte er fragen, was sie denn geplant hatten, als über dem Tisch eine kleine, leuchtende Kugel auftauchte. Diese wurde größer und verdampfte dann in einem Nebel. An seiner Stelle erschien ein Päckchen. Es senkte sich langsam auf den Tisch und bewegte sich auf Neville zu. Das Paket war in Geschenkpapier eingewickelt. Unter der Schleife war ein Brief.

Sorgfältig überprüfte Neville das Geschenk mit sämtlichen ihm bekannten Zaubern. Er achtete sorgfältig darauf, dass seine Großmutter davon nichts mitbekam. Er sollte zwar seinen Zauberstab immer bei sich tragen, aber das Zaubern, das hatte sie ihm nahe gelegt, sollte er sein lassen. Es sei denn es gäbe einen triftigen Grund dafür. Dann erst griff er nach dem Geschenk und zog den Brief heraus. Er entfaltete ihn und las halblaut vor:

\begin{brief}
Sehr geehrter Mister Longbottom,

vor etwa fünfzehn Jahren habe ich ihren Eltern ein Versprechen gegeben, ihnen zu ihrem sechzehnten Geburtstag dieses Geschenk zu überreichen. Ihre Eltern empfanden es für wichtig genug, es mir und nicht ihrer Großmutter anzuvertrauen. Ich selber habe keine Ahnung, worum es sich dabei handelt, da ich das Paket niemals geöffnet habe. Ein Brief im Inneren soll sie näher informieren, nachdem sie die Anweisungen ihrer Eltern, die ich hier wiedergeben möchte, gelesen haben.
\end{brief}

Dann wechselte die Schrift und er erkannte die Schrift seiner Mutter. Zumindest stimmte sie mit alten Briefen überein, die ihm seine Großmutter gezeigt hatte. 

\begin{brief}
Lieber Neville,

Es tut uns leid, dass wir dir dein Geschenk nicht persönlich überreichen können, aber unser Beruf als Auror ist nun mal gefährlich. Wir wollten aber, dass du es auf jeden Fall bekommst und auch nutzen kannst. Solltest du diese Zeilen lesen, sind wir vermutlich Tod. Packe dein Geschenk erst einmal aus und hole es heraus. Dann betrachte es mit deinen Freunden und ratet, was es sein könnte. Dann erst bitte liest du deinen Brief, der darunter (unter der Uhr) im Paket liegt.

Außerdem ist noch eine kleine Ausgabe davon eine Lage darunter. Diese ist für deine Hosentasche gedacht.
Alles Gute zum Geburtstag,
\signumspace
Mum und Dad
\end{brief}

Dann fand er noch zwei Unterschriften, die nur Mum und Dad lauteten. Er legte gerade den Brief zur Seite, als seine Oma hereinkam.

\enquote{Oh, bist du noch gar nicht fertig mit auspacken?}, fragte sie.

Neville schüttelte nur stumm den Kopf und sah auf das Päckchen, das vor ihm lag. Seine Großmutter setzte sich neben ihn hin und betrachtete es nun auch. Ihre Augen flogen über den Brief. Endlich nahm Neville es in die Hand und entfernte die Schleife. Dann entfernte er das Geschenkpapier. Heraus kam ein komplett schwarzes Pappkästchen, in dem z.B. Uhren verpackt waren. Vorsichtig drehte er es herum und besah sich die Verpackung. Dann begriff er, dass er nur den Deckel anheben musste, der bis zum Boden reichte.

Als der den Deckel hob und alle sahen, was darin lag, presste seine Großmutter ihre Hand vor den Mund. \enquote{Das habe ich seit Jahren vermisst.}

\enquote{Wie meinst du das?}

\enquote{Das hat Frank gehört. Eines Tages war es verschwunden. Ich wollte ihn noch fragen, wo er es hin hatte, aber dann war es auch schon zu Spät.}

Neville nahm die Sanduhr heraus. Das Glas wurde durch ein Gestänge aus grünem Glas an den Längsseiten und oben und unten durch einen Messingring gehalten. Die drei Glasstreben waren oben und unten eingedreht. Neville hatte das Gefühl, dass die Sanduhr selbst nichts wog. Er wollte sie gerade in der Hand wiegen, da bemerkte er, dass sie von selbst in der Luft stehen blieb. Waagerecht lag sie nun in der Luft. Der Sand im Inneren war auf beide Seiten der Uhr gleichmäßig verteilt. Doch er lag nicht auf dem Boden, sondern war an den Glasböden der Sanduhr. So, als ob es im Inneren der Uhr eine andere Schwerkraft geben würde, als außerhalb. So langsam fing der Sand an sich zu bewegen. Direkt auf die Verengung zu und, nachdem er durch war auf der anderen Seite auf den Sand zu.

\enquote{Ich würde sagen, das ist eine Sanduhr}, sagte Dean.

\enquote{Das ist zu offensichtlich, Dean. Hast du nicht gehört, was Neville vorgelesen hat?}, fragte Luna. \enquote{Wir sollen raten, was es ist. Eine Sanduhr kennt wohl jeder. Es muss eine besondere sein.}

\enquote{Hast du eine Ahnung, Granny?}

\enquote{Du hast den Brief doch vorgelesen. Du wirst schon dahinter kommen.} Dann stand sie auf und ging.

Nach einer Weile erfolglosen Ratens, gaben die vier auf und Neville nahm den Brief, der unter der Uhr lag heraus und begann ihn zu lesen.

\begin{brief}
Mein Sohn,

Diese Uhr wurde mir von meinem Vater vererbt und dieser hat sie von seinem Vater bekommen. Ich hoffe, dass auch du eines Tages einen Sohn oder eine Tochter hast, dem oder der du diese Uhr vererben kannst.

Diese Uhr hat mehrere Funktionen. Je nachdem, welche du gerade brauchst. Einmal zeigt sie dir die Zeit an. Jeder Person, auf die die Uhr geprägt ist, kann die aktuelle Zeit ablesen. Frag mich nicht, wie es funktioniert, aber wenn du die Uhr siehst, dann weißt du, wie spät es ist. Als weitere Funktion zeigt dir die Uhr an, ob ein Gespräch auch einen Wahrheitsgehalt besitzt, oder nicht. Je nachdem wie langsam oder schnell der Sand verrinnt. Je langsamer, desto mehr der Wahrheit entspricht das Gespräch.

Außerdem gibt es noch eine kleine Ausgabe. Hole sie jetzt bitte heraus.
\end{brief}

Neville kam dem nach und suchte in der Schachtel eine kleine Ausgabe der Eieruhr. Er nahm sie in die Hand und wog auch diese. Diese jedoch war etwas schwerer als die große Ausgabe. Da diese aber nichts wog, war das keine Kunst. Er schloss seine Faust und las weiter.

\begin{brief}
Trage diese Uhr immer bei dir. Sie wird dir mehr als nützlich sein. Sie hat ein paar außergewöhnliche Eigenschaften. Eine ist, dass nur du sie in deiner Tasche spüren kannst und sonst kein anderer. Man kann sie dir nicht wegnehmen. Zudem hast du immer die richtige Zeit, wenn du sie brauchst. Aber das Beste ist, man kann dir keinen fremden Willen aufzwingen. Ja, du verstehst mich richtig. Diese Uhr schützt dich vor einem Fluch, den man Imperius nennt. Damit können Todesser (ich hoffe, sie sind bereits seit langer Zeit Geschichte) einem den eigenen Willen aufzwingen. Das ist bei dir dann nicht mehr möglich.
\end{brief}

Neville staunte, sah sich die kleine Uhr noch einmal an und steckte sie dann in seine Tasche. Er fühlte zwar keine Veränderung, wusste aber sofort, wie spät es war. Neville lächelte leicht in sich hinein und verbrachte den Rest des Nachmittags mit seinen Gästen. Dean und Seamus verabschiedeten sich kurz nach dem Abendessen und machten sich auf die Heimreise. Luna blieb eine Stunde länger und unterhielt sich mit Neville über ihr Geschenk. Sie besprachen, wie er sie am besten zu Pflegen habe und wie sie zu vermehren seien. Neville erfuhr, dass sie eine der seltensten Pflanzen seien, die die Zaubererwelt kennt. Nur die wenigsten Pflanzenjäger würden sie allerdings als nützlich oder sinnvoll ansehen.

Mit einer herzlichen Umarmung verabschiedete sich Luna von Neville und machte sich ebenfalls auf die Heimreise.

\trenn

Während Neville und seine Gäste noch feierten, saßen die Todesser in einer großen, leeren und grauen Halle zusammen an einem langen Tisch. Voldemort sah in die Runde und sprach dann: \enquote{Wir sollten die Drachen auf unsere Seite ziehen. Sie sind mächtige Wesen, die dem Ministerium und Hogwarts eine Menge Schaden können.}

\enquote{Wenn ich darf, mein Lord?}, fragte Snape nach und fuhr nach einem Nicken Voldemorts fort. \enquote{Ich rate euch davon ab, die Drachen für eure Zwecke einzusetzen. Dies könnte zu Problemen führen, da einer der Drachen innerhalb unserer Reichweite vom Wildhüter Hagrid aufgezogen und dieser auf ihn geprägt ist. Außerdem ist einer der Drachenbändiger Charlie Weasley. Ein Freund von Hogwarts. Das könnte ein Problem darstellen.}

\enquote{Dann müssen wir diesen Wildhüter und Weasley aus dem Weg schaffen.}

\enquote{Das wird nicht einfach. Der Riese ist schwer zu bezwingen. Umbridge hat es letztes Jahr versucht und dieser Weasley kann sich sehr gut verteidigen. Das habe ich schon während seiner Schulzeit erlebt. Zudem hat er es mit Drachen zu tun, das stärkt auch.}

\enquote{Was schlägst du vor?}, fragte Voldemort.

\enquote{Wenn ich darf?}, fragte Yaxley. Voldemort nickte nur. \enquote{Es gibt Gerüchte über ein Artefakt, das Drachen schaden kann. Es dürfte in der Lage sein, sie zu vernichten. Diese Gerüchte besagen, dass dieser Gegenstand an einer Klippe sein soll. Ich meine auch mich zu erinnern, dass es in einem Buch mehr Informationen darüber gibt. Ich kann mich auf die Suche machen, mein Lord.}

\enquote{Tu das}, sagte Voldemort und dachte weiterhin nach. \enquote{Als Ausweichplan. \gst Nott, Macneir, ihr werdet mal sondieren, wie man Drachen schaden kann. Begebt euch nach Rumänien und sondiert die Lage. Berichtet mir dann.}

Die beiden nickten, standen auf und verließen den Raum. 




\begin{kommentar}
Tante Petunia bekommt einen Brief per Eule. Das ist schon mal per se ungewöhnlich. Aber ihre Reaktion darauf ist noch ungewöhnlicher. Hier habe ich bereits die Idee aufgegriffen, dass Petunia kein normaler Muggel ist, sondern eine Hexe und zusammen mit ihrer Schwester Lily Hogwarts besucht hat. Petunia hat sich später bereit erklärt, ihren Neffen zu schützen, auf die Magie weitgehend zu verzichten und ihn nicht besonders gut zu behandeln. Dies hatte allerdings zur Folge, dass sich keiner in der magischen Welt mehr daran erinnern konnte, dass Petunia in Hogwarts zur Schule gegangen war. Wie Dumbledore schon gesagt hat: Es gibt Magie, die nur durch Taten gewirkt wird. Dabei bezog er sich auf den Schutz Harrys vor Voldemort. Die Magie, die Harrys Tante wirkt, verstärkt den Schutz, den Harry durch seine Mutter bekommen hat zusätzlich.
\end{kommentar}

\begin{kommentar}
Etwas später, kurz bevor Harry zu Miss Figg geht, meint seine Tante, dass er die Eulenstange mitnehmen soll. Ein weiterer Hinweis, dass Petunia scheinbar weiß, dass Miss Figg eine Squib ist.
\end{kommentar}

\begin{kommentar}
Nevilles Geburtstag: Neville bekommt an seinem sechzehnten Geburtstag ein mysteriöses Päckchen. Dieses hatte Frederick Elber, von dem noch keiner weiß, dass er später auf Hogwarts unterrichten wird, aufbewahrt, damit es Neville rechtzeitig bekommt.
\end{kommentar}

\begin{kommentar}
Am Ende des Kapitels will Voldemort noch die Drachen auf seine Seite ziehen oder sie aus dem Verkehr bringen. Dabei soll ihm der Drachenstein helfen, der erst später im weiteren Verlauf genannt wird.
\end{kommentar}

\chapter{Geburtstagsfeier und Numensobligat}


\enquote{Überraschung}, kam es Harry entgegen.

Dieser kam gerade von oben aus seinem Zimmer, wo er von Arabella hin gelockt wurde, da in der Küche die Vorbereitungen noch nicht ganz abgeschlossen waren. Er konnte gar nichts mehr sagen. Da standen Ron, Hermine und Ginny die ihn alle anlächelten. Hermine und Ginny stürmten auf den verdutzt drein schauenden Harry zu und umarmten ihn. Dann gaben ihm beide gleichzeitig einen Kuss auf die Backe und drückten ihn nochmals.

\enquote{Alles Gute zum Geburtstag}, sagten Hermine und Ginny; und Ron fügte ebenfalls ein: \enquote{Alles Gute zum Geburtstag}, hinzu.

Harry fing an zu grinsen, denn so einen Geburtstag hatte er bisher noch nie erlebt. Seine besten Freunde Ron und Hermine waren da und auch Ginny, von der er wusste sie wollte unbedingt dabei sein, denn sie hatte immer noch dieses gewisse Leuchten in ihren Augen, wenn sie ihn sah. Harry würde sich für ihren Geburtstag etwas ganz Besonderes überlegen, dachte er. Als die Mädels ihn endlich wieder losließen, kam Ron auf ihn zu und umarmte ihn brüderlich. Harry drehte sich um und sah zu Arabella, die ihn anlächelte und sagte, \enquote{Alles Gute zum Geburtstag Harry.}

Er fühlte sich glücklich wie schon lange nicht mehr. Dann drehte er sich wieder zu seinen Freunden um. Hermine nahm seine beiden Hände und sagte zu ihm: \enquote{Wir haben noch eine Überraschung für dich.} Ginny, die jetzt hinter ihm stand, schlang von hinten ein Band um seine Augen, sodass er nichts mehr sehen konnte. Harry wusste nicht, was passierte, aber ließ es einfach geschehen. Hermine führte ihn nun an einer Hand zur Küche heraus und in das Wohnzimmer. Dort angekommen gab sie ihm zu verstehen, er solle sich doch setzen. Etwas unsicher und nach Halt suchend, ließ er sich schließlich in einen Sessel fallen. Er spürte einen Mund nahe seinem Ohr, konnte aber nicht sagen, ob es Ginny oder Hermine war. Seine Gedanken begannen zu kreisen und er bekam eine leichte Gänsehaut. Ginny und Hermine fingen leicht an zu kichern, denn er hatte nur ein kurzes Hemd an, und so konnte man seine Unterarme gut erkennen, die ebenfalls leichte Gänsehaut anzeigten. Plötzlich spürte er etwas, dass an seinen Ohren entlang fuhr. Es dauerte etwas bis er realisierte, dass Ginny und Hermine mit ihren Zungen seine Ohren umspielten. Er musste schlucken. Dann, so als hätten sie lange geübt, begannen sie jeweils ein Wort abwechselnd in seine Ohren zu sprechen.

\enquote{Das war nicht die Überraschung, Harry. Mach deine Augen auf.} Das Band wurde ihm vom Kopf gezogen und er erblickte den Wohnzimmertisch, auf dem vier Geschenke lagen. Harrys Augen weiteten sich.

\enquote{Für mich?}, fragte er erstaunt.

\enquote{Ja!}, antwortete Ginny.

Er hatte zwar schon von allen Geschenke erhalten, aber sie waren noch nie dabei gewesen, wie er sie ausgepackt hatte, denn Harry hatte seinen Geburtstag immer in den Ferien und so konnte er keinen einladen. Das hätten sein Onkel und seine Tante niemals zugelassen. Er gab Ginny und Hermine jeweils einen Kuss auf die Backe und meinte dann zu Ron: \enquote{Du bekommst aber keinen von mir.}

\enquote{Das macht nichts}, sagte Ginny, \enquote{ich mache das.} Als Ginny zu ihrem Bruder gehen wollte, hielt sie Hermine zurück und flüsterte ihr etwas ins Ohr. Sie begann zu kichern und Hermine ging zu Ron.

\enquote{Damit du nicht meinst, du würdest gar nichts bekommen} und gab ihm einen Kuss. Harry musste sich beherrschen, um nicht zu lachen, denn der Ausdruck auf Rons Gesicht als ihn Hermine küsste, blieb noch eine ganze Weile auf seinem Gesicht stehen. \gedanke{Bahnt sich zwischen den beiden was an?}, fragte sich Harry, aber er hatte Geschenke auszupacken und so lenkte er sich damit ab.

Zuerst packte er Hermines Geschenk aus. Er ahnte bereits, dass es irgendein Buch war. Zu seiner Überraschung fand er eine Schachtel darin. Er öffnete sie und staunte, als er etwas, das wie eine schwarze Scheibe aussah, betrachtete. Sie war etwas größer als eine Galleone.

\enquote{Was ist das?}, fragte er.

\enquote{Das ist ein magisches Tagebuch. Du kannst wichtige Termine eintragen und wirst automatisch daran erinnert. Funktioniert auf Sprachbasis}, sagte Hermine. \enquote{Lege die Medaille einfach auf das mitgelieferte Buch und der Text wird automatisch übernommen.} Harry zog das Schutzpapier beiseite und entdeckte in der Schachtel ein kleines Buch.

Harry zog beide Augenbrauen hoch, stand auf und bedankte sich bei Hermine mit einem kleinen Kuss auf ihre Wange nahe ihrem Mundwinkel. Er setzte sich wieder und hatte den Eindruck, als ob Hermine leicht errötete. Als Nächstes öffnete er Rons Päckchen. Von Ron bekam er silberne Erweiterungs-Medaillen für sein Tagebuch inklusive Beschreibung und bedankte sich. \gedanke{Die Textlänge ist wohl begrenzt!} Er wollte gerade das große weiche Päckchen öffnen, als Ginny ihm ihres in seine Hand drückte. Es war das kleinste Päckchen von allen. Harry hatte keine Ahnung was es sein konnte. Als er es öffnete versteinerte sein Gesicht. Er war richtig sprachlos. Seine Augen begannen zu leuchten als er es berührte. Seine Hand umschloss sanft seine Konturen. Es war ein kleiner Anhänger in Form eines Basilisken. Er hatte eine schwarze Oberfläche und die Augen glitzerten grün wie Smaragde. Er zog sich den Anhänger um und lächelte Ginny an. Leicht unsicher wendete sie ihren Blick ab, doch Harry öffnete seine Arme und ging auf Ginny zu. Er zog sie an sich und flüsterte in ihr Ohr.

\enquote{Danke Ginny. Ich liebe dich \gst wie eine Schwester, die ich nie hatte} und küsste sie auf den Mund. Ron verschlug es die Sprache und Ginny begann rot zu werden. Als eine einzelne Träne begann ihr Gesicht herunterzulaufen, fing Harry an \enquote{Tut mir leid Ginny, ich hätte dich nicht küssen sollen.}

\enquote{Nein, das ist es nicht Harry}, antwortete sie. \enquote{Es ist das, was du mir gesagt hast. Der Teil mit der Schwester die du nie hattest. Bei dir fühle ich mich richtig geborgen, Harry.} Sie drehte sich zu Ron um und sagte: \enquote{Nicht, dass ich mich bei dir nicht geborgen fühle Ron, aber Harry ist wie ein Bruder zu mir, der nicht immer und überall um seine kleinere Schwester besorgt ist.}

Ron senkte leicht seinen Blick, begann aber nach kurzer Zeit zu grinsen. \enquote{Du weißt ja, wie große Brüder manchmal sind. Vielleicht habe ich es auch nur übertrieben.}

Alle fingen an zu lachen. Harry setzte sich wieder in seinen Sessel und betrachtete seine Geschenke. Da fiel ihm das vierte auf, welches noch immer auf dem Tisch lag. Harry wunderte sich, von wem das denn sein könnte. Er nahm es und fand keine Karte, die auf den Absender schließen ließ.

\enquote{Das ist von mir}, hörte Harry von hinten. Er drehte sich um und sah Arabella.

\enquote{Von dir? Ja aber\abs}

\enquote{Mach es einfach auf}, sagte sie.

Harry öffnete das Päckchen. Besser gesagt er riss es auf und fand ein T-Shirt, eine Hose und ein Paar Schuhe.

\enquote{Nachdem du gerne joggen gehst, dachte ich mir, du könntest das brauchen. Sie sind bereits gewaschen.}

\enquote{Danke Arabella}, sagte Harry, stand auf und umarmte auch sie. \enquote{Gleich morgen früh probiere ich die Sachen aus.}

\enquote{So, jetzt wird erst einmal gefrühstückt}, sagte Arabella und verschwand in die Küche.

Die anderen folgten ihr und Hermine und Ginny begannen den Tisch zu decken, während Ron Harry zu seinem Stuhl delegierte und meinte: \enquote{Du hast heute Geburtstag, wir machen das.}

\enquote{Wie kommt ihr hierher? Woher wisst ihr, dass ich nicht bei meinem Onkel und meiner Tante bin?}

Ron antwortete: \enquote{Wir haben Eulen von Miss Figg bekommen. Sie bat uns heute herzukommen. Wir sind schon sehr früh da gewesen und halfen ihr beim Frühstück machen. Sie hatte Hedwig geschickt.}

Harry staunte nur und bekam seinen Teller serviert. Genüsslich biss er in seinen Schinken und sein Toastbrot, verschlang seine gebackenen Bohnen und blickte zu Ginny.

\enquote{Ist es wegen der Kammer?}, fragte Harry und deutete auf sein Amulett um den Hals.

\enquote{Ja}, antwortete sie und senkte ihren Blick. \enquote{Als ich es in einem Schaufenster sah, wusste ich einfach was ich zu tun hatte. Ich verspürte diesen Drang es dir zu schenken.} Harry lächelte und bedankte sich nochmals. Er wusste, dass es immer noch an ihr nagte.

\enquote{Ich werde es immer bei mir tragen.}

Ginny schaute ihm wieder in die Augen. Sie begannen zu leuchten.

\enquote{Du hast wunderbare Augen Ginny}, sagte Harry. Sie errötete wieder.

\enquote{Harry}, hörte er plötzlich Hermine sagen. \enquote{Wenn du so weitermachst, wird man in Hogwarts denken, sie hätte einen Sonnenbrand.} Jetzt begannen alle zu lachen. Selbst Ginny, die ihre Röte wieder verlor.

Harry fing an zu grinsen und sagte dann: \enquote{Vielleicht will ich ja, dass sie errötet.}

Ron gab ihm einen Knuff in seine rechte Seite und sagte dann: \enquote{Wir reden hier immerhin über meine Schwester.}

Wieder brach Gelächter aus.

Nach dem Frühstück führte Arabella dann alle in ihr Wohnzimmer. Sie deutete ihnen an, kurz zu warten und verschwand aus dem Zimmer. Nach ein paar Minuten kam sie wieder herein und hatte ein kleines Brett in der Hand, das etwa so groß wie der Wohnzimmertisch war. Sie stellt es auf dem Tisch ab, und meinte zu Harry, er solle doch mit seinem Zauberstab darauf tippen. Er zog ihn heraus und tippt auf das Spielfeld. Plötzlich fingen an drei Stangen auf den kurzen Seiten herauszuwachsen. Oben an den Stangen begannen sich Ringe auszubilden, und etwas, das wie Zuschauertribünen aussah, fing an ebenfalls aus dem Brett heraus zu wachsen. Als sich alles wieder beruhigt hatte, sah man ein Quidditch-Feld in Miniaturausgabe auf dem Wohnzimmertisch stehen.

\enquote{Wahnsinn, so etwas habe ich noch nie gesehen}, sagten alle fast gleichzeitig.

Arabella setzte sich und meinte: \enquote{Diese Art von Spielzeug wird auch nicht mehr hergestellt. Er ist viele Jahrzehnte alt. Schon mein Großvater hatte damit gespielt. Jeder von euch sucht sich eine Mannschaft und eine Position aus und sagt diese laut und deutlich. Dann könnt ihr mit euren Gedanken die Figuren steuern. Ich habe früher gerne damit gespielt. Nur alleine macht das keinen Spaß. Zwar werden die anderen Figuren von selbst ihren Weg finden, aber wenn man keinen hat, gegen den man spielen kann, verliert das Ganze seinen Reiz.}

Harry fand Gefallen daran. \enquote{Welche Mannschaften \abs}, aber dann entdeckte er die Schriftzüge auf seiner Seite und führte seinen Satz nicht fort. Er verstand und sagte dem Spielfeld laut und deutlich.

\enquote{MacCornahiew, Hüter.} Eine Figur, die Harry sehr ähnlich sah und die Robe der MacCornahiew-Mannschaft trug, erschien auf dem Spielfeld. \gedanke{Mal sehen, ob ich als Hüter auch gut bin}, dachte Harry. Ginny nahm sich die Position als Sucherin in Harrys Mannschaft und Ron und Hermine spielten in der anderen Mannschaft, der \accentuate{Panther von Loch Lumen}. Hermine nahm sich die Position des Treibers und Ron wählte ebenfalls die Position des Hüters aus.

Arabella Figg spielte den Schiedsrichter und sagte dann: \enquote{Spiel beginnen.}

Die verschiedenen Bälle und restlichen Figuren kamen zum Vorschein. Der Schnatz wurde losgelassen und die Klatscher fingen an sich in die Lüfte zu erheben, um auf den Moment des ersten Ballkontaktes zu warten. Der Quaffel schnellte hoch und die Jäger versuchten ihn zu erreichen. Das war für die Klatscher das Startsignal und sie schossen auf dem Spielfeld umher, wurden von einem Treiber zum andern geschlagen die versuchten die Spieler der gegnerischen Mannschaft zu treffen und von ihren Besen zu werfen.

Als Harrys Spielfigur, die von einem Klatscher getroffen wurde, abstürzte und alle ihr hinterherschauten, stand diese vom Boden wieder auf, wischte sich den Staub von ihrer Robe, drehte sich zu Harry um und begann ihn zu schimpfen. \enquote{Was fällt dir ein, mich einfach so fallen zu lassen. Den hättest du doch kommen sehen sollen. Du bist schließlich größer als ich.} Merklich empört, stieg die kleine Figur wieder auf ihren Besen und begann ihren Platz vor den Ringen einzunehmen.

Sie verbrachten den ganzen Vormittag und den halben Nachmittag damit, Mini-Quidditch zu spielen; nur durch das Mittagessen unterbrochen. Harrys Mannschaft verlor vier zu zwei. \enquote{Das nächste Mal spiele ich den Sucher}, bekräftigte Harry. Alle lachten.

Arabella sagte: \enquote{Spiel beendet}, und die Figuren verschwanden vom Spielfeld. Die Tribünen und die Ringtore verschwanden und es blieb nur noch das Brett übrig. \enquote{Ihr könnt Morgen weiterspielen, wir müssen noch das Abendessen herrichten.}

\enquote{Ihr bleibt bis Morgen?}, fragte Harry.

\enquote{Ja, kurz nach dem Mittagessen gehen wir wieder}, sprach Hermine und verschwand mit Arabella in der Küche.

Harry setzte sich auf das Sofa neben Ginny, lehnte sich zurück und ließ seine Gedanken schweifen. Ron verließ ebenfalls den Raum und ließ Harry mit seiner Schwester alleine.

Nach einigen Minuten sagte Harry: \enquote{Tut mir leid, wenn ich dich heute Morgen mit dem Kuss überrascht habe, aber ich war so glücklich über dein Geschenk. Es überkam mich einfach.}

\enquote{Das macht nichts, Harry. Ich war zwar zuerst überrascht, aber es war nicht unangenehm.}

Harry lächelte und fragte dann: \enquote{Sollen wir es wiederholen?}

Ginny drehte sich zu ihm um und hob eine Augenbraue. Während sie ihn kitzelte, meinte sie: \enquote{So war das auch nicht gemeint.} Harry fing an zu lachen. Nicht weil Ginny das gesagt hatte, sondern weil er  kitzlig war. Ginny schien das zu gefallen und machte weiter. Harry versuchte sich zu wehren, und an Ginny heranzukommen, um sie ebenfalls zu kitzeln. Ein paar Mal schaffte er es sogar, sie von ihm abzubringen. Aber Ginny war sehr geschickt darin seinen Attacken auszuweichen. Als sich beide eng ineinander umschlungen auf dem Boden wälzten, kam Hermine rein, die das Lachen gehört hatte und einfach mal nachschauen wollte. Zuerst blieb sie schockiert im Türrahmen stehen, als sie die beiden so sah. Harry lag am Boden und Ginny saß auf ihm. Hermine fing an ihre Hand vor ihren Mund zu halten, als Ginny zu Harry sagte. \enquote{Gibst du nun auf} und sie stupste wieder in paar Mal in seine Seite, \enquote{oder soll ich weiter machen?}

\enquote{Ich geb’ auf, ich geb’ auf}, konnte sich Harry mit Müh und Not sagen hören. Nachdem er sich beruhigt hatte, strich er Ginny über ihre Wange und sagte leise zu ihr: \enquote{Ich hab dich wirklich sehr gerne Ginny. Du bist mehr als nur eine Schwester für mich. Ich kann nur nicht sagen was genau.}

Ginny lächelte ihn an, stieg von Harry herab und drehte sich um, um in die Küche zu gehen, als sie Hermine sah.

\enquote{Wie lange stehst du schon da?}, fragte sie erschrocken. Harry richtete sich auf und lachte immer noch.

\enquote{Lange genug, um festzustellen, dass ihr nicht das gemacht habe, was ich zuerst dachte, als ich hereingekommen bin}, sagte Hermine.

Hermine drehte sich lächelnd um und Ginny folgte ihr in die Küche. Harry konnte Hermine nur noch sagen hören: \enquote{Jetzt weiß ich ja, was ich mit Harry machen muss, wenn er wieder nicht das tut, was ich von ihm will.}

Noch leicht außer Atem stand Harry auf und ging ebenfalls in die Küche. \gedanke{Das ist der schönste Geburtstag, den ich je hatte}, dachte er.

Nach dem Abendessen und einer langen Unterhaltung mit seinen Freunden ging Harry früh schlafen.

Er lag im Bett und konnte nicht einschlafen. Die Aufregung war immer noch da. Er umfasste Ginnys Amulett und schloss seine Augen.

\begin{traum}
Er war wieder im Raum der Wünsche, als er letztes Schuljahr seine Mitschüler in Verteidigung gegen die dunklen Künste unterrichtete. Er stand hinter Ginny und sah ihr dabei zu, wie sie ihren ersten gestaltlichen Patronus herbei schwor. Dieser hatte die Form eines Pferdes. Harry lief weiter, an Neville vorbei und ermutigte ihn, es gleich wieder zu versuchen, nachdem sein Versuch missglückt war. Er sprach der ganzen Gruppe Mut zu und sagte ihnen, dass ein gestaltlicher Patronus schwierig sei, aber auch abstrakte Formen gut seien. Er lief an Ron vorbei, dessen Hund-Patronus Neville umwarf. Dann sah er wie Hermines Otter-Patronus umherflog und sie dabei leicht kitzelte. Er ging weiter zu Luna und sagte, sie möge es versuchen. Sie erzeugte einen kleinen Hasen, der in der Luft umhersprang. Sie drehte sich zu ihm um und lächelte ihn an. Harry lächelte zurück.
\end{traum}

Dann verschwamm das Bild und Harry sah nur das Innere seiner Augenlider. Er ließ das Amulett wieder los, drehte sich um und schlief ein.

Am nächsten Morgen hatte Harry seine neuen Jogging-Klamotten an und ging in die Küche, um wie jeden Tag ein Glas Wasser zu trinken, bevor er loslief. Als er die Küche betrat, saß Ginny da.

\enquote{Darf ich dich begleiten?}, fragte sie.

Harry konnte ihr diesen Wunsch nicht ausschlagen, nicht nachdem sie bereits selbst in Jogging-Sachen da saß.

\enquote{Gerne Ginny.} Er trank sein Glas leer und die beiden machten sich auf.

Sie verließen das Haus und Harry zeigte Ginny, wie sie sich aufzuwärmen habe. Er hielt ihre Beine während sie Sit-ups machte und danach hielt sie seine. Dann begannen sie zu joggen. Rund um den Block, danach die Straße hoch und wieder zurück. Auf dem Rückweg liefen sie am Haus seines Onkels und seiner Tante vorbei. Harry blieb stehen und sagte ihr, dass er hier normalerweise wohnte.

\enquote{Dies ist das Haus meines Onkels und meiner Tante. Schrecklich. Sie mögen mich nicht besonders.}

Er lief wieder los und Ginny folgte ihm.

\enquote{Mein Onkel ist ganz schlimm. Er hat eine Art Eulen-Phobie. Immer wenn eine Eule Post bringt, ist er ganz aufgeregt. Er hasst alles, was mit mir oder der Zauberei zu tun hat.}

Ginny schaute ihn mit großen Augen an.

\enquote{Sind sie wirklich so schlimm?}

\enquote{Du kannst ja mal vorbeikommen und sagen, du kennst mich aus der Schule. Und dann wartest du ab wie sie reagieren.}

Ginny schüttelte den Kopf und meinte.

\enquote{Lieber nicht.}

Harry lächelte sie an und Ginny lächelte zurück. Obwohl sie nicht in ihn verliebt war, so dachte er zumindest, hatte sie immer dieses Leuchten in ihren Augen, wenn sie ihn ansah. Sie gingen wieder in Arabellas Haus und Harry ließ Ginny den Vortritt in der Dusche. Nachdem beide mit Duschen fertig waren, gingen sie in die Küche, wo schon Ron, Hermine und Arabella beim Frühstücken waren.

\enquote{Wo wart ihr?}, wollte Hermine wissen.

\enquote{Draußen. Joggen}, sagte Ginny. \enquote{War angenehm. Daran könnte ich mich gewöhnen.}

Sie lächelte Harry zu. \enquote{Machst du in der Schule weiter? Nimmst du mich wieder mit?}

\enquote{Ja}, sagte Harry, \enquote{ich mache weiter und du kannst gerne mitkommen.}

Ginny schien glücklich zu sein.

Nach einer weiteren Runde Mini-Quidditch, in der Harry nun Sucher spielte und die anderen auch eine andere Position einnahmen, kam auch schon die Zeit für das Mittagessen. Dieses Mal gewannen Harry und Ginny. \gedanke{Als Sucher bin ich wohl doch besser, im Gegensatz zur Position des Hüters.} Arabella hatte wie immer ausgezeichnet gekocht und Harry verabschiedete sich von seinen Freunden nach dem Mittagessen.

\enquote{Wir sehen uns dann im Zug}, rief er ihnen hinterher, als sie in einen Wagen einstiegen, der verdächtig wie einer der Ministeriumswagen aussah, die Rons Vater sonst organisiert hatte.

Die nächste Woche verlief relativ ruhig. Er joggte jeden Morgen und half Arabella bei ihrer Hausarbeit. Er nahm ihr durch seine Zauberei viel Arbeit ab und hinterließ ihr ein paar nützliche Gegenstände, die ihr das Arbeiten erleichterten. Mitten in der Woche kam eine Postkarte, die besagte, dass am Montag die Dursleys wieder ankommen würden. Harry packte also am Sonntag seinen Koffer und brachte ihn zurück in sein Zimmer im Haus seines Onkels und seiner Tante. Er wollte nicht, dass sie erfuhren, dass er alles bei Arabella dabei gehabt hatte.

Wieder zurück gab Harry Arabella den Wohnungsschlüssel, den sie an sich nahm und in ihrer Tasche verstaute.

\enquote{Ich nehme an, dass ich dich wieder sieze, wenn mein Onkel mich abholt.}

\enquote{Ja Harry, das wäre passend. Wenn wir uns ihnen gegenüber zu gut verstehen, kannst du nächstes Jahr nicht wieder kommen, falls sie mal wieder in den Urlaub wollen.}

Harry nickte und lächelte. Ihm hatten die drei Wochen bei Arabella gefallen. Und für die nächsten drei Wochen war sie wieder Miss Figg. Als Onkel Vernon ihn am nächsten Tag abholte, saß Harry in der Küche. Es klingelte und Miss Figg öffnete die Haustüre. An der Stimme konnte Harry Onkel Vernon erkennen, der sich mit Miss Figg zu unterhalten schien. Sie sagte ihm, dass er ab und zu mal etwas kaputt gemacht hatte, aber die Sachen eigentlich eh schon alt gewesen waren. Im Großen und Ganzen sei er anständig gewesen. Onkel Vernon bedankte sich bei Miss Figg.

\enquote{Harry, komm her, dein Onkel holt dich wieder ab}, herrschte sie ihn an. Harry verkniff sich ein Grinsen und kam aus der Küche in den Flur. Er lief auf Onkel Vernon zu und hatte dabei leicht seinen Kopf gesenkt.

\enquote{Komm mit}, raunte ihm sein Onkel zu. Harry ging hinterher und musste die nächsten drei Wochen bei seinem Onkel und seiner Tante verbringen.

Auf dem nach-Hause-Weg ließ er den vorigen Abend noch einmal Revue passieren. Gestern Abend hatte er mit einem Zauber in einer Ecke des Gartens ein paar Petunien gepflanzt und dafür gesorgt, dass sie zur richtigen Zeit blühen würden.

\begin{rueckblick}
\enquote{Professor Sprout? Haben sie einen Moment Zeit?}

\enquote{Aber sicher doch Mister Potter. Was gibt es?}

\enquote{Ich hätte gerne gewusst, ob es einen Zauber gibt, mit dem man Blumen pflanzen kann. Mit dem man dafür sorgen kann, dass Blumen zu einem bestimmten Zeitpunkt blühen.}

\enquote{Ja, so etwas gibt es. Wieso?}

\enquote{Ich würde diese Zauber gerne lernen.}

\enquote{Erhoffen sie sich davon mehr Erfolg bei den Mädchen?}

\enquote{Auch. Aber vor allem möchte ich etwas für meine Tante und meine Mutter tun. Ich dachte da an Lilien und Petunien.}

\enquote{Oh, das sind schöne Blumen.}

\enquote{Werden sie mir helfen?}

\enquote{Ja, kommen sie mit, Mister Potter. Die notwendigen Zauber sind nicht besonders schwer. Sie brauchen nur etwas Vorbereitung.}
\end{rueckblick}

Kaum war er wieder zu Hause, flog eine Eule durch das offene Küchenfenster. Sie warf einen hellen, oliv-grünen Umschlag auf den Boden und verschwand, ohne etwas zu erwarten. Sein Onkel war gerade damit beschäftigt aus dem Wohnzimmerfenster hinauszusehen und sich wieder über die Nachbarin mit ihrem Hund aufzuregen.

\enquote{Diese aufgetakelte Schnepfe. Ständig läuft sie mit ihrer Promenadenmischung vor unserer Einfahrt herum und lässt diesen Köter an unseren Zaun\abs}

\enquote{Vernon!}, ermahnte ihn seine Frau.

\enquote{Ja, Petunia, du hast recht.}

Harry sah auf den Umschlag, der an ihn adressiert war, und drehte ihn um. Der Absender war ein Notar.

\begin{brief}
Plaustein \& Söhne

Notariat
\end{brief}

Harry wunderte sich, was ein Notar der Zauberergemeinschaft von ihm wollte. Er öffnete den Umschlag und begann, nachdem er sich auf einen Küchenstuhl gesetzt hatte, den Brief zu lesen.

\begin{brief}
Sehr geehrter Mister Harry James Potter,

als Nachlassverwalter und Familiennotar der Familie Black, setzen wir sie gemäß den Bestimmungen der Zaubereibehörde und den Wünschen des verstorbenen Sirius Black in Kenntnis. Bitte treffen Sie pünktlich am kommenden Donnerstag in unserem Notariat ein.

Wir wurden gebeten, sie zur Eröffnung des Numensobligats einzuladen. Des Weiteren sind noch folgende Personen geladen: Albus Percival Brian Wulfric Dumbledore, Remus John Lupin, Nymphodora Tonks.

Bitte teilen Sie uns mit, falls sie nicht erscheinen können, oder eine Person benennen, die sie vertreten soll.
\signumspace
Hochachtungsvoll

Nymphodora Plaustein
\end{brief}

Harry ließ seine Hände sinken. \gedanke{Eröffnung eines Numensobligats}, ging ihm durch den Kopf. Ihm wurde wieder schwer ums Herz. Tränen begannen sich in seinen Augen zu bilden. Er faltete den Brief zusammen und steckte ihn in seine Hosentasche. Danach wischte er sich mit seinem Ärmel über die Augen, um seine Tränen wegzuwischen. \gedanke{Dumbledore, Remus und Tonks}, ging ihm durch den Kopf. \gedanke{Sie sind auch geladen worden. \gst Eingeladen worden}, korrigierte er sich.

Traurig und die Momente im Ministerium immer wieder in seinem Geist durchspielend, saß er auf seinem Stuhl in der Küche und blickte ins Leere. Seine Verwandten ignorierten ihn, da sie andere Sachen zu tun hatten. Sie packten noch ihre Koffer aus.

Plötzlich klingelte es an der Haustür und Harry stand auf, um sie zu öffnen. Er sah durch das Guckloch an der Haustür nach draußen. Der Mann sah aus wie Remus Lupin. Doch Harry war vorsichtig geworden. Seinen Zauberstab ziehend, stand er hinter der Tür.

\enquote{Wer da?}, fragte Harry hinter der Tür.

\enquote{Remus John Lupin. Freund und ehemaliger Hogwarts-Lehrer.}

Harry öffnete die Tür und hielt seinen Zauberstab auf Lupin.

\enquote{Du bist vorsichtig geworden}, sagte Remus.

\enquote{Bist du, du?}, fragte Harry bewusst.

\enquote{Nein, ich bin jemand anderes. Ich bin unser Freund Voldi.}

Harry bat ihn herein, hielt seinen Zauberstab aber immer noch auf Remus. Er hatte schon eine Weile darüber nachgedacht, wie er herausfinden konnte, ob Remus wirklich der Richtige ist. Plötzlich sah er Bildfetzen vor sich. Sie schienen aus Remus herauszukommen und durch den Raum zu schweben. Er sah den Biss, der Remus zum Werwolf machte.

\enquote{Wie genau bist du zum Werwolf geworden?}, fragte Harry. \enquote{Du bist von einem Werwolf ins Genick gebissen worden, richtig?}

\enquote{Woher weißt du das? Das habe ich dir nie erzählt.}

\enquote{Erzähl weiter. Was ist danach passiert}, forderte er.

\enquote{Na ja, ich habe geblutet und den Werwolf davonlaufen sehen. Meine Mutter hat mich damals gefunden. Sie hat mir auch geholfen, damit klarzukommen.}

\enquote{Was hatte sie an?}, fragte Harry.

\enquote{Warum willst du das wissen? Was soll das?}

\enquote{Was hatte sie an?}, forderte Harry mit Nachdruck.

\enquote{Ein gelbes Kleid}, antwortete Remus.

\enquote{Mit Blümchenmuster}, nickte Harry.

\enquote{Nein, es war nicht gemustert.}

\enquote{Ok, Remus.} Harry senkte seinen Zauberstab und bat ihn in die Küche.

\enquote{Wie kannst du davon wissen?}, fragte er Harry.

\enquote{Weiß nicht genau. Ich habe Bildfetzen gesehen, die aus dir herauskamen. Ich habe mir nämlich die Frage gestellt, wie ich am besten überprüfen kann, ob du auch du selber bist. Da ist das dann passiert.}  Dann wurde Harrys Gesichtsfeld für ein paar Sekunden eingeschränkt. Er sah schwarze Ränder und sein Gesicht erblasste etwas.

\enquote{Alles in Ordnung?}, fragte Remus.

Als sich Harry wieder beruhigt hatte, sagte er: \enquote{Geht schon wieder. War nur kurz benommen. Aber weswegen bist du hier? Geht es um diesen Numensobligat?}

\enquote{Ja, Dumbledore bat mich, dich übermorgen abzuholen. Ich bin deshalb vorher vorbeigekommen, damit du Bescheid weißt.}

Harry nickte. \enquote{Willst du was zu trinken?}

\enquote{Gerne.}

Harry schenkte ihm ein halbes Glas Wasser ein und füllte mit Saft auf.

\enquote{Ich werde dann übermorgen vorbeikommen und dich mitnehmen. Wir müssen allerdings vom Haus weg. Sonst können wir nicht apparieren.}

\enquote{Ich weiß, zwei Kilometer.}

\enquote{Du weißt wie groß der Bannkreis ist?}

\enquote{Warum nicht? Man muss seine Grenzen kennen!}, antwortete Harry mit einem etwas süffisanten Lächeln.

Remus schüttelte nur ungläubig den Kopf. Er wunderte sich nicht mehr über Harrys Möglichkeiten und Wege, etwas in Erfahrung zu bringen.

\enquote{Was ist denn ein Numensobligat?}

\enquote{So etwas wie ein Testament}, antwortete Remus, bevor er wieder ging.

Zwei Tage später kam Remus vorbei, um Harry abzuholen. Harry hatte in der Zwischenzeit seinem Onkel und seiner Tante erklärt, dass er einen wichtigen Amtstermin habe. Z-Sachen erklärte er ihnen, da Onkel Vernon immer einen Anfall bekam, wenn er zaubern sagte. Die kleine Unterhaltung zwischen Vernon, Petunia und Remus, die stattfand, als Remus Harry abholen wollte, war entsprechend knapp, aber dennoch höflich gewesen.

\enquote{Und Harry, wie sind die Ferien so?}, wollte Remus wissen, als sie zu Fuß unterwegs waren.

\enquote{Na ja, ging so. Als ich bei unserer Nachbarin untergebracht war, während meine Verwandtschaft Urlaub gemacht hat, hatte ich eine Menge Spaß. Ich habe mich mit ihr gut verstanden.}

Dann wurde es für eine Weile still. Harry genoss es einfach mal nur so neben Remus herzulaufen. Einfach nur Gesellschaft zu haben, ohne ständig reden und sich erklären zu müssen. Dann waren sie außerhalb der Apparitionsgrenze. Remus nahm Harry an der Hand und zog ihn mit sich. Beide apparierten in einer ruhigen Ecke. Remus zog, für Harry unbemerkt, eine Tüte aus seiner Tasche heraus. Harry würgte einmal kurz, musste aber nicht brechen. Etwas blass, sah er Remus an.

\enquote{Beeindruckend Harry. Die meisten müssen sich beim ersten Mal übergeben. Oder auch noch nach ein paar Mal danach.}

\enquote{Komisch, warum nur?}, fragt Harry sarkastisch nach. Mit verzogenen Gesicht schaute er Remus an und fragte schließlich: \enquote{Wohin?}

\enquote{Komm mit.}

Sie liefen ein paar Meter die Gasse entlang, als Harry entdeckte, wo sie waren. \enquote{Nokturngasse?}, fragte er.

\enquote{Ja. Nokturngasse, Ecke Winkelgasse. Hier entlang}, sagte Remus, als er sich sorgsam umsah und Harry dann hinter sich her zog.

Nach wenigen Metern standen sie vor der Tür. Daneben war ein grün angelaufenes Kupferschild mit glänzender, gravierter Schrift angebracht. Dort stand: \accentuate{Plaustein \& Söhne}

Harry trat vor Remus durch die Tür in das Treppenhaus und folgte den Hinweisschildern in den ersten Stock. Er trat durch die Glastür und auf die Dame am Empfang zu.

\enquote{Die Herren Potter und Lupin}, begann er.

\enquote{Dritte Tür Links. Sie sind bei der Chefin persönlich}, sagte die Dame, ohne aufzusehen.

Harry sah zu Remus und zog eine Augenbraue hoch. Sie gingen vor die Tür, welche die Dame genannt hatte, und klopften. Nach einigen Sekunden kam ein \enquote{Herein!}, durch die Tür.

Harry öffnete sie und er und Remus traten ein.

\enquote{Ah Mister Potter, Mister Lupin. Sie sind etwas zu früh.}

Harry sah auf seine Uhr. Fünf Minuten zu früh, fand er. \enquote{Fünf Minuten}, sagte er.

\enquote{Ich vergaß}, entschuldigte sich die Notarin. \enquote{Muggel nehmen es mit der Pünktlichkeit nicht so genau.}

Harry zog beide Augenbrauen hoch.

\enquote{Hexen und Zauber ziehen es vor höchstens eine Minute zu früh, oder zu spät zu kommen. In manchen Kreisen ist sogar eine Differenz von zehn Sekunden unhöflich.}

\enquote{Verstehe, sollen wir so lange draußen warten?}

\enquote{Nein, nein, bitte \gst setzen Sie sich}, bot sie den beiden einen Sessel an. \enquote{Mein Name ist Nymphodora Plaustein, Inhaberin dieses Notariats mit angeschlossener Kanzlei.}

Harry und Remus nahmen dankend an und setzten sich.

\enquote{Möchten Sie einen Tee? Kürbissaft? Gebäck?}

\enquote{Einen Tee, danke}, sagte Harry.

\enquote{Für mich nichts}, antwortete Remus.

Nachdem der Tee für Harry auf dem Tischchen neben seinem Sessel stand, kamen auch schon Dumbledore und Tonks herein. Die Begrüßung war kurz, aber herzlich. Nachdem alle saßen, begann die Notarin die Testamentseröffnung.

\enquote{Vielen Dank, dass Sie alle hier sind. Ich habe sie eingeladen, weil sie im Numensobligat erwähnt wurden\abs}

\enquote{Verzeihung Ma’am, aber was ist ein Numensobligat?}, fragte Harry noch einmal sicherheitshalber nach.

\enquote{Wie? Sie wissen nicht was\abs? Oh natürlich, Verzeihung. Ich hätte mich anders ausdrücken müssen bei Ihnen. Das ist ein Testament. Wir sind hier um den letzten Willen Sirius Blacks zu verlesen.}

\enquote{Oh, danke, ich ahnte schon etwas in der Richtung, wollte aber Gewissheit haben}, sagte Harry.

Die Notarin begann: \enquote{Numensobligat \gst Verzeihung \gst Testament von Sirius Black.} Es folgte eine kurze Pause. \enquote{Ich, Sirius Black, durch die Magie im Vollbesitz meiner geistigen Fähigkeiten bestätigt und durch den Zauber bestätigt, verfüge als meinen letzten Willen folgende Punkte. Zu meiner Testamentseröffnung (diesen Begriff habe ich wegen Harry gewählt) möchte ich folgende Personen hier haben: Albus Percival Wulfric Brian Dumbledore, Remus John Lupin, Nymphodora Tonks (verzeih mir Tonks) und Harry James Potter.} Erneut pausierte sie kurz. \enquote{Ich will nicht viele Worte verlieren, also nur das wichtigste. Jeder von euch bekommt etwas. Nymphodora \gst wenigstens jetzt kannst du mir nichts mehr anhaben \gst dir gebe ich ein besonderes Familienschmuckstück. Es schützt dich vor vielen Zaubern. Du wirst es schon noch herausfinden.}

Mrs Plaustein öffnete ihre Schublade, nahm ein kleines hölzernes Kästchen heraus und übergab es Tonks.

\enquote{Remus, mein treuer Freund. Dir vermache ich ebenfalls ein besonderes Schmuckstück aus meiner Familie.} Sie nahm ein weiteres Kästchen aus ihrer Schublade und überreichte es Remus.

Dann las sie weiter: \enquote{Bitte, meine Freunde. Öffnet es.}

Tonks und Remus öffneten ihre Kästchen und staunten über die Ketten, die darin waren. Sie waren leicht rosafarben. Ihr Wert war nicht genau zu erkennen, aber beide schätzten, dass Kupfer drinnen sein musste, oder die Ketten Rotgold enthielten.

\enquote{Diese Ketten}, fuhr Mrs Plaustein fort, \enquote{enthalten kein Kupfer oder Rotgold, wie ihr vielleicht annehmen werdet. Diese Ketten \gst und bitte behaltet sie \gst}

Dann sagte sie: \enquote{Fassen sie die Ketten bitte an}, forderte die Notarin.

Remus und Tonks taten dies, worauf die Ketten kurz zu Leuchten begannen.

\enquote{\gst sind aus einer Rhodiumlegierung und enthalten fast hundert Prozent dieses Metalls. Es ist ein wunderbarer Katalysator und wird euch auf eurem Weg gute Dienste leisten. Den wahren Wert, der über dem materiellen liegt, wird euch erst später bewusst werden.}

Remus und Tonks wollten die Ketten schon ablehnen, da hoben sie aus ihren Kästchen ab und legten sich um ihre neuen Träger. Das Wissen, das die Ketten mitbrachten, sickerte in die Köpfe der beiden, sodass sie ihren Widerstand aufgaben.

\enquote{Albus, dir gebe ich etwas, das du schon immer haben wolltest. Meine Sammlung an alten Heften, meine Schoko"-frosch-Samm"-lung und ein besonderes Rätsel.}

Mrs Plaustein übergab Dumbledore einen Würfel, der in sämtlichen Richtungen dreimal unterteilt schien. Harry erkannte ihn als Rubik-Würfel.

\enquote{Dieser Würfel wird dir sicher viel Spaß machen. Die anderen Sachen werden dir nach Hause geliefert.}

Dumbledore drehte begeistert den Würfel und besah ihn sich von allen Seiten.

\enquote{Zu guter Letzt, Harry, mein Patensohn. Dir vermache ich den ganzen Rest. Mein Verlies in Gringotts, mein Haus samt Kreacher. Leider liegt ein Zauber auf dem Haus, der verhindert, dass zu viele Leute die Erste Zeit sich im Hause aufhalten. Unsere Kartengruppe muss sich also die nächste Zeit einen anderen Ort suchen.}

Mrs Plaustein legte das Testament auf ihren Schreibtisch und überreicht den Schlüsselbund mit den beiden Schlüsseln an Harry. Zusätzlich gab sie ihm noch einen Brief in die Hand.

Harry öffnete ihn und las:

\begin{brief}
Lieber Harry,

bitte nimm dein Erbe an, sonst fällt Kreacher meiner Cousine zu und er wird ihr freudestrahlend alles erzählen, was wir in mühevoller Kleinarbeit alles erreicht haben.
\end{brief}

Dann löste sich der Brief auf. Harry verstand. Obwohl er noch nicht genau wusste, wie, nahm er sich vor, Kreacher zu resozialisieren und in die Gemeinschaft zurückzuführen. Er musste vom Zwang seiner alten Familie gelöst werden.

\enquote{Damit wäre der Numensobligat beendet. Ich danke für ihr Erscheinen.}

Die vier standen auf, gaben der Notarin die Hand und verließen ihr Büro.

\enquote{Ich bringe Harry dann heim, Remus.}

Remus nickte und verschwand mit Tonks, nachdem sie aus dem Gebäude getreten waren. Harry und Dumbledore liefen noch eine Weile nebeneinander in der Winkelgasse her.

\enquote{Hier war ich schon lange nicht mehr}, sagte Dumbledore plötzlich.

\enquote{Haben Sie auch bemerkte, dass zwischen Tonks und Lupin etwas läuft?}, fragte Harry.

\enquote{Du hast das bemerkt?}

\enquote{Ja, war nicht zu übersehen. Die Ketten passten zueinander und leuchteten kurz, als sie sie umlegten \gst Äh ja. Es zogen sich feine Linien zueinander hin. Es würde mich nicht wundern, wenn diese Ketten ihnen zu einer engeren Bindung und Beziehung verhelfen würden.}

\enquote{Wie kommst du darauf?}, fragte Dumbledore.

\enquote{Weiß nicht genau. Ich habe so ein Gefühl.}

Dumbledore streckte Harry einen Arm hin, worauf dieser ihn ergriff und beide disapparierten. Dieses Mal ohne Gesichtsverfärbung aber mit noch leichtem Aufstoßen. \enquote{Du scheinst das Apparieren recht gut zu vertragen.}

\enquote{Na ja, es gibt schlimmeres. Aber es ist schon recht unangenehm. Beim ersten Mal hätte ich mich fast übergeben.}

Dann liefen sie still nebeneinander her und genossen die Stille. Es war schön, mit Dumbledore einfach zu spazieren.

Kurz vor der Tür fragte Harry: \enquote{Wie sieht es mit einem Lehrer für \VgddK aus?}

\enquote{Na ja, ich bin dran. Ich habe jemand; fast. Er ist gut geeignet. Scheinbar ist er nicht sonderlich davon angetan, dass ihr letztes Jahr nichts gelernt habt.}

\enquote{Nichts ist nicht ganz richtig. Das, was wir selber gelernt haben, war schon viel.}

\enquote{Das galt aber nicht für alle.}

\enquote{Und die Theorie war zumindest ein verschwindend geringer Teil}, sagte Harry und grinste Dumbledore an.

\enquote{Ich habe mir sagen lassen, dass du die praktische Prüfung, ebenso wie ein Großteil aus deiner Jahrgangsstufe außer den Slytherins, hervorragend geschafft hast. Dolores hat getobt, habe ich gehört, als die Ergebnisse bekannt wurden.} Dann grinste er.

\enquote{Professor? Kommen Sie noch kurz mit rein? Dann sollten sie sich noch umziehen, sonst bekommt mein Onkel wieder einen Anfall.}

\enquote{Ich glaube nicht. Ich würde trotzdem auffallen.} Kurz vor der Tür verabschiedeten sich die beiden voneinander. \enquote{Mach’s gut Harry.}

\enquote{Bis zum September Professor}, antwortete Harry und ging ins Haus.

Dumbledore ging die Straße weiter und schaute noch auf einen Tee bei Arabella Figg vorbei.

\trenn

In der letzten Woche vor Schulbeginn flatterte am Dienstagmorgen eine Eule zum offenen Fenster herein. Onkel Vernon war gerade nicht da, und Dudley vergnügte sich anscheinend mit seinen Kumpels. In der Küche stand gerade Harry, der seiner Tante beim Abspülen half, als die Eule auf sich aufmerksam machte. Sie hatte einen Brief an ihrem Fuß, den Harry abmachte. Er gab ihr noch etwas von den Resten des Essens und die Eule flog davon. Harry öffnete seinen Brief.

\enquote{Das ist meine Einkaufsliste von \gst äh \gst Schulsachen, Tante Petunia}, sagte er seiner Tante, die ihn mit erhobener Augenbraue und zusammengekniffenen Lippen anschaute. Nervös drehte sie sich um, sah zu ihm zurück und sagte dann. \enquote{Was geht mich das an? Am Freitag habe ich Besseres zu tun. Du wirst mit mir einkaufen gehen. Ich habe ein paar Besorgungen zu machen. Schwere Sachen, die ich Dudley nicht heben lassen kann.}

Harry schluckte. Wie sollte er seine Schulsachen besorgen, wenn ihn keiner abholte? Er dachte kaum, dass seine Tante oder sein Onkel ihn zur Winkelgasse oder zumindest in die Nähe bringen würden.

Am Freitag darauf stand Harry auf und saß wie gewöhnlich am Frühstückstisch, als seine Tante zu sprechen anfing.

\enquote{Vernon, ich brauche heute den Wagen. Ich gehe mit Harry ein paar Besorgungen machen. Und Dudley kann sich dabei verletzen.}

Onkel Vernon entwich nur ein kurzes \enquote{Hm, ja.} Er aß unbekümmert weiter. Ihn störte es nicht, wenn sich Harry danach nicht mehr bewegen konnte.

Harry stand auf und legte seinen Teller in die Spüle. Tante Petunia lief ihm hinterher und ging dann durch den Flur zur Tür hinaus. Harry folgte ihr. Im Auto angekommen setzte sich Harry hinein und fuhr mit seiner Tante los. Er war erstaunt, dass sie nicht wie üblich in eines der Kaufhäuser fuhr, die sie sonst immer benutzte. Dieses Mal benutzte sie die Schnellstraße und fuhr Richtung London. \gedanke{Das kann ja heiter werden; London, und keine Möglichkeit Schulsachen zu kaufen.} Seine Tante fuhr durch die halbe Stadt, an vielen Kaufhäusern vorbei, von denen Harry dachte, da müsste er jetzt rein. Zu seinem großen Erstaunen fuhr sie auf einen Parkplatz in der Nähe des tropfenden Kessels. Wenn er dort nur für eine halbe Stunde verschwinden könnte, dachte er. Als sie anhielt, wollte er gerade aussteigen, als ihn seine Tante am Arm festhielt. Sie zog einen Zettel aus ihrer Tasche und reichte ihn Harry. Er entfaltete das Pergament und entdeckte seine Einkaufsliste.

\enquote{Du machst deine Besorgungen und ich mache meine. In einer Stunde treffen wir uns wieder hier. Und \gst kein Wort zu deinem Onkel oder zu Dudley. Verstanden?}

Harrys Augen weiteten sich und sein Mund stand offen. Er war sprachlos. Als er sich wieder fasste, brachte er nur ein \enquote{Ja, Tante Petunia} heraus. Er stieg aus dem Wagen aus und machte sich auf den Weg zum tropfenden Kessel. Seine Gedanken schwirrten umher. \gedanke{Wusste Tante Petunia, wo sich die Winkelgasse befand? War sie nur zufällig hier, weil sie in der Gegend einkaufen wollte?} Harry konnte es gar nicht so recht glauben. Ihm war es nur recht, dass er eine Gelegenheit hatte seine Sachen einzukaufen.

Nachdem er bei Gringotts wieder etwas Geld abgehoben hatte, ging er einkaufen. \gedanke{Hier dürfte mir nichts passieren}, dachte er. Er kaufte seine Bücher bei Florish \& Blotts, neue Tinte und reichlich Pergamentrollen. Außerdem frische Eulenkekse für Hedwig, die während der gesamten Ferienzeit keinen einzigen Brief für ihn zu transportieren brauchte und so recht entspannt wirkte. Lediglich Arabella hatte ein paar wenige verschickt. Nachdem er seine Einkäufe erledigt hatte, sprach ihn von hinten jemand an.

\enquote{Hallo Harry, auch beim Einkaufen?}

Er drehte sich um und sah Luna, die ihn anlächelte.

\enquote{Ja. Bin gerade fertig geworden.}

Neben Luna stand ein älter aussehender Mann.

\enquote{Das ist mein Vater}, sagte sie.

Er schüttelte Harry die Hand und meinte nur: \enquote{Schön Sie mal persönlich kennenzulernen Mister Potter. Meine Tochter sagte, Sie seien letztes Schuljahr ein ausgezeichneter Lehrer gewesen, als Sie gegen Miss Umbridge agierten. Und das Interview ließ die Auflage meines Magazins in die Höhe schnellen.}

Harry empfand Mister Lovegood als durchaus sympathisch.

\enquote{Danke Mister Lovegood. Das hat auch Spaß gemacht.}

\enquote{Ach nennen Sie mich doch Xenophilius.}

\enquote{Gerne, dann müssen Sie mich aber auch Harry nennen.}

\enquote{Geht klar.}

Die beiden verabschiedeten sich und Luna winkte ihm zum Abschied nochmals zu. Harry schaute auf seine Uhr, die er gekauft hatte und bemerkte, dass er noch etwas Zeit hatte. Langsam schlenderte er die Winkelgasse entlang, als sein Blick in das Schaufenster von Ollivanders fiel. Er betrat den Laden und Mister Ollivander, kam kurz darauf um die Ecke.

\enquote{Ah, Mister Potter. Schön Sie wiederzusehen. Wie geht es Ihnen? Ist mit Ihrem Zauberstab alles in Ordnung?}

\enquote{Ja}, meinte Harry. \enquote{Mir geht es gut. Aber was ich Sie fragen wollte. Wenn ich meinen Zauberstab verliere, gibt es dann irgendeinen Ersatz? Ist der dann auch so gut wie mein alter?}

Mister Ollivander schaute ihn an. Er atmete tief ein und dann wieder aus.

\enquote{Normalerweise ist ein Ersatz kein Problem, Mister Potter. Aber bei ihrem \gst nun ja. Wissen Sie, es ist so. Phönixfedern sind extrem selten. Ich habe keine mehr auf Lager. Sie müssten mir also eine Beschaffen, wenn Sie einen neuen oder einen Ersatz wollen.}

Harry schaute Mister Ollivander nur an. Er wusste nicht, was er sagen sollte. Dann fiel ihm etwas ein.

\enquote{Wissen Sie von welchem Phönix meine Feder stammt?}

\enquote{Ja. Aber Sie müssten ihn überreden, ihnen eine abzugeben. Anders als bei Einhörnern oder Drachen lassen sich nur freiwillig gegebene Phönixfedern in Zauberstäben verwenden.}

\enquote{Und welcher war es?}

Mister Ollivander Augen begannen leicht zu leuchten.

\enquote{Fawkes}, sagte er.

Harry fing an zu grinsen.

\enquote{Wieso grinsen sie Mister Potter?}, fragte Mister Ollivander nach.

\enquote{Nun ja}, sagte Harry, \enquote{ich kenne Fawkes. Vielleicht kann ich ihn irgendwann dazu überreden, falls ich mal Bedarf dazu habe.}

Harry schaute wieder auf seine Uhr. Er bemerkte, dass er langsam gehen musste.

\enquote{Ich muss leider gehen Mister Ollivander, meine Tante wartet sonst auf mich.}

Er ging durch die Tür, zurück in den Tropfenden Kessel und hinaus in die Welt der Muggel. Er fand seinen Weg zurück zum Auto, wo Tante Petunia gerade ihre Sachen einlud. Er verstaute ebenfalls seine Einkäufe und sie fuhren zurück. Zu Hause packte Harry seine Sachen schnell in sein Zimmer und half seiner Tante ihre Einkäufe auszuladen. Onkel Vernon kam in die Küche und staunte.

\enquote{Du hast ja kaum was dabei Petunia}, sagte Onkel Vernon.

\enquote{Sie haben die Sachen bedauerlicherweise nicht mehr da gehabt. Das müssen sie erst bestellen.}

Harry war sich nicht sicher, ob Tante Petunia log und ihn nur seine Schulsachen einkaufen ließ. Er war sich nicht einmal sicher, was sie wusste und was sie verdrängte. \gedanke{Irgendwie hatte sie der Brief von Dumbledore verändert}, bemerkte Harry. \gedanke{Aber was soll’s. Ich hab meine Schulsachen und Tante Petunia ihre Einkäufe.}

Plötzlich flatterte wieder eine Eule herein. Onkel Vernon zuckte zusammen und wollte schon wieder etwas sagen, hielt sich aber zurück. Harry nahm den Brief und las ihn.

\begin{brief}
Lieber Harry,

Ich schicke dir Sonntag um 9 Uhr einen Ministeriumswagen vorbei.

Halte deine Sachen gepackt.
\signumspace
Arthur Weasley
\end{brief}

\enquote{Ich werde am Sonntag abgeholt}, sagte Harry trocken und ließ den Brief auf den Tisch fallen. Er ging nach oben, kontrollierte, ob er auch alle Hausaufgaben gemacht hatte und verstaute seine Sachen in seinem Koffer.

Am Sonntag dann stand er auf und ging in die Küche um zu Frühstücken. Es läutete und Onkel Vernon öffnete die Tür. Harry saß am Frühstückstisch und konnte die Unterhaltung mit anhören.

\enquote{Guten Tag, ich komme Harry abholen}, hörte Harry eine Männerstimme sagen.

\enquote{Junge, komm her, hol deine Sachen und verschwinde}, sagte Onkel Vernon.

Harry stand auf, ging hoch in sein Zimmer, machte seinen Koffer wieder leichter, nahm ihn und Hedwigs Käfig und ging die Treppen hinunter. Der Mann nahm Harrys Koffer an sich und schmunzelte, als er bemerkte, dass er sehr leicht war. Er verstaute ihn in seinem Kofferraum und Harry verabschiedete sich von seinem Onkel und seiner Tante. Am Wagen angekommen, öffnete er ihn und sah, dass bereits Hermine im Inneren saß.

\enquote{Hallo Hermine.}

\enquote{Hallo Harry.}

Nach einer angenehmen Fahrt zum Londoner Bahnhof lud der freundliche Herr die Gepäckstücke aus und fuhr davon.

Im Zug sitzend warteten beide auf Ron, der sich kurze Zeit später blicken ließ. Während der ganzen Fahrt über unterhielten sich die drei über die vergangenen Ferienwochen. Nach einer Weile fasste Harry wieder sein Amulett an und schloss die Augen.

\begin{traum}
Er war wieder im Raum der Wünsche, als er letztes Schuljahr seine Mitschüler in Verteidigung gegen die dunklen Künste unterrichtete. Er stand hinter Ginny und sah ihr dabei zu, wie sie ihren ersten gestaltlichen Patronus herbei schwor. Dieser hatte die Form eines Pferdes. Harry lief weiter, an Neville vorbei und ermutigte ihn, es gleich wieder zu versuchen, nachdem sein Versuch missglückt war. Er sprach der ganzen Gruppe Mut zu und sagte ihnen, dass ein gestaltlicher Patronus schwierig sei, aber auch abstrakte Formen gut seien. Er lief an Ron vorbei, dessen Hund-Patronus Neville umwarf. Dann sah er wie Hermines Otter-Patronus umherflog und sie dabei leicht kitzelte. Er ging weiter zu Luna und sagte, sie möge es versuchen. Sie erzeugte einen kleinen Hasen, der in der Luft umhersprang. Sie drehte sich zu ihm um und lächelte ihn an. Harry lächelte zurück.
\end{traum}


Er öffnete seine Augen und dachte sich: \gedanke{Wieder das Gleiche. Jedes Mal sehe ich Lunas Gesicht. Aber Luna ist recht nett. Ich mag sie. Sogar sehr gerne.}




\begin{kommentar}
Die Idee zum Basiliskenanhänger kam mir, als ich eine andere Geschichte (Die übersinnliche Schlange) las. Dort konnte Harry Ginny sehen, wenn er den Anhänger in seiner Hand hielt. Hier stellt sich heraus, dass der Anhänger wirklich von Salazar Slytherin ist und einen kleinen Teil von Salazar selbst beinhaltet. Damit hat Harry Zugriff auf Teile Salazars Magie und auch Salazars Wissen selbst.
\end{kommentar}

\begin{kommentar}
Während Harrys Geburtstagsfeier bei Arabella hält diese ein kleines Brett heraus. Ein Mini-Quidditch-Spiel. Die Idee zu solch einem Spiel kam mir, als ich den Film >Zathura - Ein Abenteuer im Weltraum< gesehen hatte. Es ist eine Mischung zwischen diesem Spiel und >Jumanji<.
\end{kommentar}

\begin{kommentar}
Harry pflanzt im Garten seiner Tante Petunien. Eine schöne Anspielung auf ihren Namen.
\end{kommentar}

\begin{kommentar}
Wieder zurück bei den Dursleys bekommt Harry einen Brief und kurz darauf erscheint Remus Lupin. Harry fliegen Bildfetzen zu, von etwas, was Harry nicht über seinen ehemaligen Lehrer wissen kann. Hier greift zum ersten Mal die Magie des Amuletts, das Harry zum Geburtstag geschenkt bekommen hat.
\end{kommentar}

\begin{kommentar}
Der Notarin Mrs Plaustein habe ich Tonks Vornamen gegeben, weil ich es als nett empfand, jemanden zu haben, der diesen Vornamen auch trägt und ihn nicht so ablehnt, wie Tonks es tut.
\end{kommentar}

\begin{kommentar}
Später, als Harry seine Liste mit Schulsachen bekommt, fährt ihn seine Tante in die Nähe der Winkelgasse, da er sonst keine Gelegenheit bekommen hätte, seine Schulsachen zu kaufen. Wieder ein kleiner Hinweis, dass seine Tante mehr weiß, als sie zugibt.
\end{kommentar}

\chapter{Ankunft in Hogwarts}


Als Harry den Saal betrat, fiel ihm auf, dass Professor McGonagall auf dem Stuhl des Schulleiters saß und Professor Snape fehlte. Auch Ron und Hermine fiel das auf und die drei fragten sich, was wohl mit Dumbledore passiert sei. \enquote{Wir werden es wohl bald erfahren}, sagte Hermine.

Der Stuhl mit dem alten Hut stand bereits da. Es dauerte nicht lang und die große Flügeltür ging auf. Die Erstklässler betraten den Raum und alle schauten ihnen zu. Zu Harrys erstaunen begleitete Snape die jungen Schüler. Während dessen lief Hagrid außen um die Tische herum, um zu seinem Platz am Lehrertisch zu gelangen. Am Ende angekommen zog Snape ein Pergament aus seiner Tasche und sprach: \enquote{Ich werde Ihre Namen aufrufen und sie werden den Hut auf diesem Stuhl dort aufheben, sich auf den Stuhl setzen und danach werden sie sich den Hut aufsetzen. Er wird sie in ihre Häuser einteilen. Doch zuvor hören wir noch das Lied des Hutes.} Und der Hut hob von seinem Stuhl ab und begann schwebend zu singen.

\begin{lied}
Vor langer Zeit, was keiner weiß,\\
Hogwarts gegründet zu einer Zeit,\\
nach Krieg und Schrecken in jener Nacht\\
dies Schloss hier war ausgedacht.\\
Von den vier größten Zauberern damals sie waren,\\
Godric Gryffindor, Rowena Ravenclaw,\\
Helga Hufflepuff, und Salazar Slytherin.\\
Im Traume entdeckten sie dieses Schloss,\\
das ihnen als Schule dienen sollte.\\
Der Hausherr ganz zügig überließ es ihnen,\\
damit sie der Welt Vermächtnis erhalten.\\
Drum haltet zusammen, wie damals jetzt auch,\\
das Böse nur immer wartet darauf,\\
dass schwach wir werden und einsam dazu,\\
bekämpfen wir es ja klar nur zu.
\end{lied}

Der Hut sank wieder auf den Stuhl und Snape begann die Schüler aufzurufen.

\enquote{Was heißt hier: \inner{Der Hausherr ganz zügig überließ es ihnen,\abs}} fragte Hermine.

\enquote{Pst}, sagte Ron, \enquote{die Auswahlzeremonie.}

Allman Phillip, ein kleiner Junge, den Harry nicht genau sehen konnte, aber er vermutete, dass er auf seinem Gesicht neben seinem einzigen Auge auch noch Narben erkennen konnte, wurde zu den Hufflepuffs geschickt. Snape rief die anderen Nacheinander auf und sie wurden vom Hut in ihre Häuser eingeteilt und durch tosenden Applaus der jeweiligen Schüler auf ihren Platz begleitet. Harry ließ seinen Blick durch die noch verbleibenden Schüler schweifen und sein Blick blieb wieder bei einem kleinen blonden Mädchen hängen. Er erinnerte sich daran, dass er sie schon auf dem Bahnsteig gesehen hatte.

\begin{rueckblick}
Harry und Hermine passierten gerade die Absperrung und bogen ab. Fast wären sie in ein kleines Mädchen gelaufen.

\enquote{Oh Entschuldigung}, sagte sie. Dann erkannte sie ihn. \enquote{Du bist Harry Potter}, bemerkte sie. \enquote{Ich heiße Tamara. Mein Bruder hat mir von dir erzählt.} Dann drehte sie sich um und lief in den Zug.

Verdutzt sah er Hermine an. Er fragte sich, zu wem sie wohl gehören mochte.
\end{rueckblick}

Dann rief Professor Snape \enquote{Malfoy, Tamara.}

Harry drehte sich um und sah zu Malfoy hinüber. Er hatte also eine Schwester. \gedanke{Das könnte interessant werden}, dachte er sich. Professor Snape setzte ihr den Hut auf, nachdem sie sich gesetzt hatte. Der Hut bewegte sich und eine Weile passierte nichts. Harry fühlte sich an seine Auswahl erinnert.

\begin{rueckblick}
Ihm wurde der Hut aufgesetzt und er hörte den Hut in seinem Geiste. In ihm war viel Talent und der Drang, sich zu beweisen. Aber auch Mut steckte ihn ihm. Doch er wollte nicht nach Slytherin und wünschte sich nicht dorthin zu kommen. Also schickte ihn der Hut nach Gryffindor.
\end{rueckblick}

Der Hut brauchte immer noch um eine Entscheidung zu treffen und Malfoys Schwester sah sich in der Großen Halle um. Ihr Blick fiel auf ihren Bruder, der sie anlächelte, danach über die anderen Tische zu Harry. Sie lächelte ihn an. Schließlich war der Hut mit seiner Entscheidung fertig: \enquote{Wenn du dir sicher bist: \extase{Gryffindor!}} Der gesamte Tisch jubelte ihr zu und hieß sie herzlich willkommen. Harry sah wieder zu Malfoy hinüber und sah ihn lächeln. Er freute sich für seine Schwester. Warum, konnte er sich nicht erklären, aber das Lächeln stand ihm. Ron war leider alles andere als begeistert ihn so zu sehen. In all dem Jubel konnte er Ron nicht richtig verstehen, aber das war Harry nur recht. Sie hatten eine Malfoy, die in Gryffindor war. Vielleicht die Erste; vielleicht auch nach langer Zeit wieder eine. 

Pestrow, Cornelia schickte der Hut nach Ravenclaw und als vorletztes rief Snape Slystorp, Harold auf.

Harry kannte seine größere Schwester Anna, sie war im siebten Jahr und ihre Familie war seit mehreren Generationen in Slytherin. Er würde bestimmt auch dorthin kommen, dachte Harry. Aber zu seinem Erstaunen, und scheinbar zum Erstaunen vieler, kam er nach Gryffindor. Harry blickte schnell zu seiner Schwester, die ein entsetztes Gesicht machte. Harry musste grinsen. Es hatte für ihn den Anschein, dass Harold ganz froh darüber war. Snape rief noch Zabini Ludger auf und dieser wurde vom Hut nach Slytherin geschickt.

Nachdem alle Schüler ihren Häusern zugeordnet waren und Snape den Stuhl samt Hut nun mitgenommen hatte, stand Professor McGonagall auf und sprach zu den Schülern.

\enquote{Ich freue mich, dieses Jahr viele neue Schüler begrüßen zu dürfen und möchte so viele junge und enthusiastische Neuankömmlinge herzlich willkommen heißen. Der Schulleiter Professor Dumbledore hat heute leider keine Zeit dem Fest und der Auswahlzeremonie beizuwohnen, da er mit anderen wichtigen Dingen beschäftigt ist. Ihren neuen Lehrer in \VgddK werden sie morgen beim Frühstück kennenlernen. Seien sie also pünktlich. Den Neuankömmlingen möchte ich sagen, dass ihr Professor Dumbledore morgen früh sehen werdet und er euch noch einmal persönlich begrüßen möchte.}

Sie machte eine kurze Pause und fuhr dann fort. \enquote{Mister Filch, unser Hausmeister}, sie zeigt auf ihn und die neuen drehten sich um, \enquote{bat mich, ihnen mitzuteilen, dass das Ausüben von Zaubern oder von Flüchen jedweder Art auf den Korridoren untersagt ist und dass der verbotene Wald aus gutem Grund verbotener Wald heißt, weil es nicht erlaubt ist diesen zu betreten. Möge das Fest nun beginnen}, sie holte ihren Zauberstab heraus und machte eine schwingende Bewegung. Sofort erscheinen auf den langen Haustischen die unterschiedlichsten Speisen und Getränke.

\enquote{Ob sie ihren Zauberstab dazu wirklich braucht?}, fragte Ron Hermine.

\enquote{Weiß nicht}, antwortete sie, \enquote{vielleicht wollte sie es auch nur etwas effektvoller aussehen lassen.}

\enquote{Aber was wohl Dumbledore gerade macht? Sucht er einen neuen Vgddk-Lehrer und hofft, dass er morgen gleich mitkommt?}, fragte Harry.

Nicht weiter darüber nachdenkend, begannen sie zu essen. Ron und Hermine begleiteten nach dem Essen die Neuankömmlinge in ihre Quartiere und standen noch eine Weile ihren Fragen zur Verfügung, Ron den Buben und Hermine den Mädchen. Ebenso die anderen Vertrauensschüler, die die kleine Gruppe begleitete.

\trenn

In dieser Nacht hatte Harry einen seltsamen Traum.

\begin{traum}
Er stand in einem gewölbeartigen Raum. Um ihn herum lagen Goldmünzen und Bestecke, sowie Teller und andere Gegenstände aus Gold auf Tischen und auf dem Boden verteilt. Harry sah sich im Raum um. Auf einem der dunklen, fast schon schwarzen Regale, die an der Wand standen, entdeckte er einen goldenen Trinkpokal. Auf ihm war ein Dachs abgebildet. Er sah sich weiter in dem Raum um, doch immer wieder zog es seinen Blick auf den Pokal auf dem Regal zurück.

Hinter sich sah er eine Tür. Er überlegte eine Weile, während er immer wieder auf den Pokal sah. Dann traf ihn eine Erkenntnis. Er stand wohl in einem Verlies in Gringotts. Aber wie kam er hierher? Gerade als er sich noch über diese Frage den Kopf zerbrach, hörte er mechanische Geräusche. Panikartig suchte er nach einem Versteck. Doch nichts kam in Betracht. Die Tür wurde geöffnet und Bellatrix Lestrange betrat das Verlies.

Harry war dabei seinen Zauberstab zu ziehen, doch er hatte ihn nicht dabei. Die irre Hexe trat direkt auf ihn zu und blieb vor ihm stehen. Schweiß lief von seiner Stirn. Doch Bellatrix sah sich nur um, nahm einige Münzen und schob sie in ihren Umhang. Harry stand wie paralysiert da. Es lief ihm kalt den Rücken hinunter. \gedanke{Hat sie mich gesehen und jagt mir kurz bevor sie geht einen Fluch hinterher? Oder hat sie mich nicht bemerkt? Nein, sie muss mich gesehen haben}, dachte er sich.

Dann traf ihn der sprichwörtliche Blitz der Erkenntnis. Er war am Träumen. \gedanke{Das würde auch erklären, dass sie mich nicht gesehen hatte und dass ich meinen Zauberstab nicht dabei habe}, dachte er sich.

Als Bellatrix ihr Verlies wieder verließ, folgte er ihr. Doch sein Blick zog es ein letztes Mal zu diesem Pokal mit dem Dachs. Seinen Blick rückwärts gerichtet, verließ er das Gewölbe. Sein Blick fiel auf die Nummer des Verlieses: 431. Unbewusst registrierte er die Zahl. Dann wechselte die Szene auf eine Wiese und die Erkenntnis, dass er träumte, verlor sich in den über ihm schwebenden Wolken.
\end{traum}

Harry wachte auf und stellte fest, dass er alleine im Schlafsaal war. Er hörte ein leises: \enquote{Guten Morgen Harry.} Es war Myrte.

\enquote{Was machst du hier Myrte?}, fragte er, ehe er ein: \enquote{Guten Morgen übrigens}, hinterher gab, da er nicht unhöflich sein wollte und da ihm Myrte das sicherlich wieder vorwerfen würde.

\enquote{Ich wollte dich mal wieder besuchen. Du warst schon lange nicht mehr bei mir, Harry}, sagte sie.

Harry schwang die Füße aus seinem Bett und sah Myrte nun direkt an.

Interessiert betrachtete sie sein Basilisken-Amulett. \enquote{Oh, woher hast du denn das?}, fragte sie.

\enquote{Das war ein Geschenk von Ginny}, antwortete Harry.

Myrte schwebte näher. \enquote{Leuchtet es leicht, wenn du es anfasst?}, fragte sie ihn.

Harry staunte und sagte dann: \enquote{Ich weiß es nicht.} Dann umfasst er das Amulett und hielt es kurz darauf in der offenen Hand, da er kaum etwas sehen konnte, wenn es in seiner Hand eingeschlossen war. Und tatsächlich, die Augen leuchteten auf.

Myrte ließ einen glücklichen Seufzer los. \enquote{Das ist schön}, antwortete sie. Sie kam Harry näher und drückte ihm einen Kuss auf seine Wange.

Harry spürte eine leichte, aber kurze Kälte, ehe sich Myrte wieder löste, ihm zuwinkte und durch die Wand verschwand.

Ron und Hermine saßen bereits am Frühstückstisch, als Harry hereinkam und setzte sich. Als die Erstklässler alle zur gleichen Zeit hereinkamen und sich setzten, stand Dumbledore auf und begann mit seiner Rede.

\enquote{Es freut mich, dass auch dieses Jahr wieder viele Erstklässler dabei sind, in die Geheimnisse der Zauberei eingewiesen zu werden. Ich nehme an, Professor McGonagall hat ihnen bereits erzählt, was es mit dem verbotenen Wald auf sich hat und dass das Zaubern auf den Gängen untersagt ist. Ich möchte\abs}

Doch Dumbledore kam nicht mehr dazu den Rest zu sagen, denn ein lauter Knall war durch die ganze Halle zu hören. Eine Schrecksekunde später kam ein blau schimmernder Vorhang zur linken Seite der Großen Halle herein und bahnte sich seinen Weg durch die Selbige. Innerhalb einer Sekunde war alles vorbei. Harry fröstelte es ein bisschen. Dann konnte er einen weiteren Knall hören, und noch einen und noch einen. Plötzlich fiel ein Etwas durch die Decke und plumpste auf den Boden. \gedanke{Es sah aus wie ein Geist}, dachte Harry. Nach ein paar Sekunden begann er im Boden zu versinken. Harry schaute zu Professor Dumbledore und fragte sich, was denn nun passiert sei. Lautes Gemurmel erfüllte plötzlich die Halle. Keiner wusste genau was passiert war. Harry sah, wie Professor Dumbledore Professor McGonagall etwas ins Ohr flüsterte, um den Lehrertisch herum lief und zwischen den Reihen die Große Halle verließ. Er hatte es sehr eilig und beachtete niemanden.

Harry schaute zurück zu Ron und Hermine und dachte bereits an Fred und George. \enquote{Meinst du Fred und George haben etwas damit zu tun?}, fragte er Ron.

Ron antwortete: \enquote{Ich glaube nicht, die sind nicht so fahrlässig und würden es riskieren jemanden zu verletzen, oder einen Geist durch das halbe Schloss schleudern. Und außerdem sind sie nicht mehr da.}

\enquote{Was immer es auch war}, sagte Hermine, \enquote{die Explosion muss von außerhalb des Schlosses gekommen sein.}

Harry und Ron sahen sie nur ungläubig an.

\enquote{Wie kommst du darauf?}, fragte Ron.

\enquote{Na ja, schau dir doch mal an von wo die blaue Welle kam.} Hermine zeigte auf die Wand. \enquote{Und dahinter ist nicht mehr viel. Also muss die Explosion von außerhalb des Schlosses gekommen sein.}

Ron schaute sie mit leichtem Entsetzten im Gesicht an.

Nach dem Essen machten sich die drei wieder auf den Weg zu ihren Schlafsälen, wo ihre Stundenpläne in der Zwischenzeit auf ihre Betten gelegt worden waren. Harry freute sich, als er montags in der ersten Stunde \fach{Pflege magischer Geschöpfe} bei Hagrid hatte. \gedanke{Mal sehen was Hagrid wohl wieder neues dran bringt}, dachte Harry. Er schnappte sich sein Schulbuch und machte sich auf den Weg zu Hagrids Hütte. Dieses Jahr hatte Hagrid wieder ein normales Buch und kein beißendes wie vor drei Jahren. Damals hatte Hagrid unter anderem auch Hippogreife durchgenommen. \gedanke{Wie es jetzt wohl Seidenschnabel geht?}, fragte er sich.

Vor Hagrids Hütte angekommen führte er sie in Richtung eines Geheges, das Leer zu sein scheint. Vor dem Gehege angekommen sagte Hagrid: \enquote{Heute fangen wir mit Clestinern an. Clestiner sind schwer zu sehen. Es sei denn man singt ihnen etwas vor. Dann können sie nicht widerstehen und zeigen sich uns.} Harry betrachtete das Gehege und meinte etwas zu sehen. Etwas, das wie fahle, kaum sichtbare Umrisse aussah. Etwas, das ihn an einen Elch erinnerte. Hagrid öffnete das Gehege und deutete seinen Schülern an, hereinzukommen. Nachdem alle drinnen waren, schloss er das Gatter wieder und holte eine kleine Flöte aus einer seiner Taschen.

\enquote{So, jetzt passt mal auf}, sagte er und begann zu spielen. Viele Schüler hielten sich die Ohren zu, denn Hagrid konnte nicht besonders gut auf der Flöte spielen. Jetzt zeigten sich, zwar noch durchlässig, aber man konnte sie schon sehen, die Clestiner. Geschöpfe, die ein dunkelgrünes und an manchen Stellen Moos bewachsenes Fell hatten. Ihre Schaufeln waren weiß und die Hufe glänzten in einem Purpur schimmernden rot. Hagrid hörte sein Spielen auf der Flöte auf, worüber Harry mehr als nur froh war.

\enquote{Ich weiß, das war nicht besonders gut, aber so habt'er jedenfalls eine Vorstellung, von dem, was euch erwartet. Jeder nimmt sich ein' Beutel aus dem Kasten hinter mir. Dann verteilt euch im Gehege und setzt euch auf'n Boden, öffnet den Beutel und legt den Inhalt vor euch auf'n Boden. Wenner merkt, dass euch ein Clestiner den Inhalt aufisst, singt 'r ihm etwas vor, damit 'r 'n sehen könnt. Wenn 'r 'n vollständig seht, gebt ihm schnell einen Namen. Akzeptiert er ihn, wird er sich euch nähern und ihr müsst 'n streicheln. Ab da könnt 'r 'n dann immer sehen. Natürlich nur euren eigenen.}

Die Schüler drängten sich nun um den Kasten hinter Hagrid und jeder nahm sich einen Beutel heraus. Dann verteilten sie sich im Gehege und setzten sich wie geheißen auf den Boden. Es hörte sich eigenartig an, als alle ein Liedchen in die unsichtbare Morgenluft hinein trällerten.

Harry wollte seinem Clestiner den Namen \accentuate{Mike} geben als er merkte, dass es sich um ein Weibchen handelte. So entschied er sich für \accentuate{Rosalie}. Anscheinend gefiel ihr der Name und sie kam näher, um ihn abzuschnuppern. Harry streichelte ihr grünes Fell und das Weibchen begann sein Gesicht zu lecken.

\enquote{Wer fertig ist, steht auf und versucht sich auf sein' Clestiner zu setzen.}

Harry hatte etwas Mühe, da Clestiner doch recht große Tiere waren und man auf sie nicht so leicht wie auf ein Pony steigen konnte. Rosalie hatte anscheinend Mitleid mit ihm, als es ihm auch nach seinem fünften Versuch nicht gelang, auf ihren Rücken zu steigen. Sie senkte ihren Vorderkörper, indem sie die Vorderbeine einknickte, und gab Harry so die Möglichkeit aufzusteigen. Hagrid grinste, als alle ihre Tiere bestiegen hatten.

\gedanke{Es muss für Hagrid ein ungewöhnlicher Anblick sein}, dachte Harry.

\enquote{So, wenn alle fertig sind, können wir ja weiter machen.} Er schlug die Hände gegeneinander, was einige Clestiner dazu brachte zusammen zu zucken. Einige Schüler mussten sich festhalten, um nicht herunterzufallen. \enquote{Wenn euch eure Schulkameraden die Namen ihrer Clestiner verraten, ihr sie ruft und die dann zu euch kommen, könnt ihr auch die eurer Schulkameraden sehen. Aber das macht ihr am besten nach Gutdünken aus. Die nächsten vier Wochen werden wir uns um die Aufzucht und die Pflege eurer Tiere kümmern. Ihr werdet alles darüber Lernen, wie sie euch helfen können. Wie sie Verletzungen heilen können, wie sie euch in einem Kampf einen Überraschungsmoment verschaffen können und wie ihr euch mit ihnen unterhalten könnt. Nach den vier Wochen gibt es eine kleine Prüfung und den Rest des Schuljahres müsst ihr sie nur ab und an besuchen, damit sie an euch gewöhnt bleiben. Wer will, kann sie am Ende des Schuljahres mit nach Hause nehmen.}

Harry musste bei dem Gedanken grinsen, denn er stellte sich gerade Dudleys, oder Onkel Vernons Gesichter vor, wie Rosalie durch den Garten lief, Spuren hinterließ, sie aber nichts sehen konnten. Aber sie würden bestimmt ihn im Verdacht haben und ihm deshalb wieder Ärger verursachen. Der Rest der Stunde verlief recht angenehm und jeder kümmerte sich um seinen Clestiner.

\gedanke{Dieses Mal scheint Hagrid kein gefährliches Tier mit im Unterricht zu haben. Zumindest noch nicht.}

Die Stunde bei Professor McGonagall bestand heute aus Wiederholungen und dem Besprechen des diesjährigen Schulstoffes. Es war also recht anspruchslos. Zumindest dachte Harry das.

Ebenso erging es der Klasse heute in Kräuterkunde. Professor Sprout ließ sie wieder Alraunen umtopfen, da es diese bitter nötig hatten und die Vorgängerklasse \gst das zweite Schuljahr \gst es nicht schaffte. Danach wechselten sie noch das Gewächshaus und besahen sich die Pflanzen, die sie ab dem nächsten Mal durchnehmen wollten. Es waren rankenartige Fleischfresser. Zwar standen auf der Speisekarte der Pflanzen maximal Mäuse, aber vor solchen Pflanzen musste man sich trotzdem in Acht nehmen, da sie sonst zubissen.

Die erste richtige Herausforderung kam aber jetzt auf der Krankenstation.

Dieses Jahr hatte Harry montags Unterricht bei Madame Pomfrey, welche sich mal wieder beim Schulleiter Professor Dumbledore beschwert hatte und Erste-Hilfe-Kurse für die Schüler für sinnvoll befand. Professor Dumbledore gab dem nach und so hatten sie am Montag immer Heilkunde.

Heute nahm sie Madame Pomfrey richtig ran. Sie mussten heute den richtigen Umgang mit dem Zauberstab zu Diagnosezwecken lernen. Etwas, was jedem Heiler und jeder Krankenschwester beigebracht wird. Die Kunst, den Stab richtig zu führen zu erlernen, dauerte meist ein paar Stunden, hält aber \gst auch bei seltener Anwendung \gst jahrelang an und bedarf nur gelegentlicher Auffrischung.

Die Klasse schnaufte, da Madame Pomfrey bereits zu jedem mehrmals kam, um die Haltung des Zauberstabes zu korrigieren.

\enquote{Warum brauchen wir das eigentlich? Bei unseren anderen Zaubern achten wir auch nicht auf die richtige Handhaltung.}

\enquote{Da hängt auch nicht das Leben ihres Patienten davon ab. Die Handhaltung ist für diese präzisen Zauber sehr wichtig. Sie beeinflusst die Magie, die sie wirken. Deshalb bin ich so streng in diesem Punkt. Eine falsche Handhaltung und die Diagnose verändert sich.} Sie suchte sich einen Sitzplatz und erzählte dann weiter. \enquote{In früheren Zeiten wurde so manche Fehldiagnose gestellt, die die Leute auch in den Selbstmord trieb. Erst als Heiler Antyos und seine Frau aus Versehen den gleichen Patienten untersuchten und verschiedene Diagnosen erstellten, begannen sie zu forschen. Sie beobachteten einander, als sie ihre Patienten untersuchten, und stellten Differenzen in der Art der Zauberstabhaltung fest. Sie experimentierten mit verschiedenen Haltungen und entwickelten die Grundlagen der modernen Diagnostik.}

\gedanke{Deshalb triezt sie uns so}, dachte Harry.

\trenn

Es war eine ruhige Nacht, ein Uhr vierundzwanzig und der Wind blies nur langsam und ab und an. Eine dunkle, ganz in schwarz gekleidete Gestalt trat aus dem Zwielicht heraus auf den Weg. Langsam bewegte sich die menschliche Gestalt auf das verschlossene Tor mit den geflügelten Ebern zu, die auf den Steinsäulen links und rechts das Tor flankierten. Die Gestalt blieb davor stehen und sah durch das schmiedeeiserne Flügeltor auf den Weg, der zum Schloss führt. Dann sah sie zu den Ebern hoch und betrachtete beide eine Weile. Schließlich zog sie ihren Zauberstab und fuhr den schmalen Spalt zwischen den beiden Torflügeln entlang. Das Tor öffnete sich geräuschlos und gab den Weg frei. Als die Gestalt hindurchgeschritten war, erhoben sich die Eber auf ihre Hinterbeine und schlugen mit den Flügeln. Die Gestalt winkte ab und hinter ihr schloss sich das Tor wieder und sie lief den Weg zum Schloss entlang hoch.

Das wenige Licht der Nacht brach durch die Bäume und Äste und fiel auf den Weg. Der Umhang der Gestalt leuchtete leicht bräunlich im fahlen Licht, welches auf den Gehweg schien. Kurz nachdem das Schloss wieder in Sichtweite gekommen war, blieb die Gestalt stehen und fuhr sich mit einer Handschuh-verpackten Hand übers Gesicht. An Stelle der Augen waren nun kaum zu sehende, schwache, grünliche Lichter zu sehen. Aufmerksam beobachtete die Gestalt das Schloss und es schien fast so, als ob sie durch die Wände in das Innere schauen würde.

Dann hörte das Leuchten der Augen auf und die Gestalt lief den Weg weiter zum Schloss. Vor der großen Flügeltür blieb sie wieder stehen und besah sich das Holz der Tür. Wieder griff die Gestalt in ihren Umhang und zog erneut den Zauberstab heraus. In einer waagerechten Bewegung fuhr sie vor sich auf dem Holz entlang. Die Maserung wurde undeutlicher und leichte Wellenbewegungen zeichneten sich ab, wie bei einer Wasseroberfläche über die leicht der Wind strich. Die Gestalt schritt hindurch und die Oberfläche zeichnete den Umriss der Person nach. Dann wurde die Maserung wieder deutlicher und das Holz war nicht mehr durchdringbar. Zwei dunkle Glockenschläge erklangen synchron, aber die zweite Glocke schlug um einen Halbton höher als die Erste und so konnte man doch deutlich den Glockenschlag, der zur Ankunft des seltsamen Gastes erklang, von dem, der zur halben Stunde schlug, unterscheiden.

Im Inneren des Schlosses vollzog die Gestalt dieselbe Bewegung wie im Schatten der Bäume, um das Schloss zu beobachten. Mit erneut grün leuchtenden Augen sah sich die Gestalt sorgfältig um, ohne jedoch den Platz knapp hinter der Tür zu verlassen. Als die Augen wieder mit dem Dunkel verschmolzen, nahm sie ihren Zauberstab und ging ein paar Schritte am Holzportal entlang und suchte mit einer Hand ein kleines Loch in der Wand neben dem Scharnier der Tür. Sie steckte ihren Zauberstab bis zum Griff in das Loch, worauf hin der Griff leicht heller wurde aber nicht selbst zu Leuchten begann. Das Licht im Raum wurde etwas heller und man sah nun den Umhang in dunklem Braun mit grauen Schlieren darin, welcher perfekt mit jeder dunklen Umgebung verschmolz. Doch selbst in der leicht erhellten Umgebung war zu wenig Licht, um ganz zu erkennen, welche Gestalt dort stand. Sie drückte ihre Hand an die Wand, behielt aber immer den restlichen Körper und den Blick in die Hallenmitte gerichtet.

Mit leiser und krächzender Stimme begann sie einen Monolog, in dem das Schloss durch Töne antwortete.

\enquote{Hogwarts. Ich werde in Zukunft öfters hier im Schloss auftauchen. Bitte mach bis auf Weiteres keine große Sache daraus. Behandle mich wie jeden Gast hier in deinen Mauern. Keine Extrabehandlungen \gst Hast du mich verstanden?} Ein leiser hoher Glockenschlag erklang wie zur Bestätigung. \enquote{Ich danke dir}, sagte die Gestalt und verneigte sich. Sie nahm ihre Hand von der Wand, worauf die Beleuchtung des Raumes zurückging. Danach zog sie ihren Zauberstab aus der Wand und berührte damit, ohne sich umzudrehen, die Tür hinter sich.

Abermals zerfloss die Struktur des Holzes und die Gestalt trat rückwärts aus dem Schloss heraus und sofort begann sich die Struktur wieder zu verfestigen. Nach einem Kreuzschritt und einer geschmeidigen Drehung schritt die Gestalt den Weg zum Tor hinunter. Das gusseiserne Tor öffnete sich automatisch und die geflügelten Eber gaben nur ein kurzes Grunzen vor sich. Ein kleines und leises Kichern ging von der Gestalt aus, bevor sie während des Laufens lautlos verschwand. Eine einzelne Windböe verwischte die restlichen Spuren im Kies und den durch Dreck, Sand und Laub bedeckten Stellen des Weges.

\trenn

Am nächsten Tag ging jeder der Gryffindors nach der Stunde im Gewächshaus 6 in Richtung Krankenstation. Sie verabschiedeten sich von den Hufflepuffs, mit denen sie immer Kräuterkunde hatten und gingen zur nächsten Stunde.

In der Krankenstation angekommen wartete bereits Madame Pomfrey und die Ravenclaws auf sie. Sie deutete ihnen an, sich in einem Halbkreis aufzustellen. \enquote{Das Wichtigste, das sie lernen werden, ist, verschiedene Verletzungen zu erkennen und herauszufinden, ob der Verletzte transportiert werden kann. Warten Sie hier kurz.} Sie drehte sich um und verschwand in ihrem Büro. Kurz darauf kam sie mit einer Puppe zurück, die so groß wie ein ausgewachsener Mensch war. \enquote{An ihm werden sie in den folgenden paar Wochen lernen, zu erkennen, welche Art Verletzung die betroffene Person hat. Ab dem nächsten Mal werden ihnen mehrere Avatare zur Verfügung stehen.} Sie nahm ihren Zauberstab aus ihrer Schürze und fuhr dem Avatar auf und ab. \enquote{Mister Longbottom, würden Sie bitte ihren Zauberstab nehmen, ihn über den Avatar halten und dann folgendes sagen: \zauber{Scindescenti}}

% ital. 	scintille incandescenti
Neville nahm seinen Zauberstab und sprach: \zauber{Scindescenti}, worauf Funken aus der Spitze seines Stabes sprühten und den Avatar bedeckten.

Neville erschrak kurz, doch Madame Pomfrey fing an weiterzuerzählen. \enquote{Sie machen das sehr gut, Mister Longbottom. Das klappt hervorragend. \gst An der Art, Form und Farbe der Funken können sie erkennen, um welche Art von Verletzung oder Krankheit es sich handelt. \gst Betrachten Sie bitte die Funken. Hier handelt es sich um gelbliche Kugelfunken mit einem leichten bläulichen Einstich. \gst Sie können jetzt aufhören.} Neville nahm seinen Zauberstab zurück und stellte sich wieder in die Reihe. \enquote{Schlagen Sie bitte in ihren Büchern nach und suchen sie nach den entsprechenden Funken. Jeder von ihnen schreibt mir dann seine Meinung dazu auf einen kleinen Zettel \gst ein paar Sätze genügen. Gebt ihn dann bei mir ab. Die Zeit läuft.}

\enquote{Wie, jetzt?}, fragten ein paar Schüler.

\enquote{Na klar jetzt, der Verletzte wartet nicht, bis sie Zeit haben, sich um ihn zu kümmern und sich wohlgenährt an die Arbeit machen.} Sie räumte den Avatar zur Seite und setzte sich entspannt auf ein leeres Bett in der Nähe. Auf der Krankenstation herrschte reges Treiben. Überall saßen oder standen Schüler und blätterten in ihren Büchern.

Madame Pomfrey wartete geduldig auf die Annahmen und Vermutungen. Bereits nach wenigen Minuten kamen die ersten Schüler und gaben ihre Zettel ab. Madame Pomfrey nahm sie entgegen und las sie aufmerksam durch. \enquote{Interessante Bemerkungen, das muss ich schon sagen.} Ihre Augen weiteten sich etwas. \enquote{Und bei einigen ist der Patient bereits dem Tod nahe.}

Sie stand wieder auf und meinte dann: \enquote{Wir werden uns jetzt erst mal um den richtigen Umgang mit dem Zauberstab kümmern. Mir scheint, sie brauchen eine Auffrischung. Nehmen sie ihn mal alle hervor.} Sie wartete, bis alle so weit waren und bewegte ihren Zauberstab in der Luft. \enquote{Scindescenti}. Viele weiße sternförmige Funken sprühten ohne Unterlass aus der Spitze ihres Zauberstabes. \enquote{Als Erstes lassen Sie bitte alle diese netten Analysefunken aus ihren Zauberstäben heraussprudeln. Da es nichts gibt, worauf sie sich niederlassen könnten, das einer Heilung bedarf, bleiben die Funken weiß und ihre Form wird sich auch nicht verändern.} Jeder schwang nun seinen Zauberstab und schon sprudelten die ersten Funken heraus.

\enquote{Sehr schön. Sie können dann, wenn ihre Funken weiß und sternförmig sind, ihren Zauberstab gegen ihren eigenen Körper richten. Dort dürften sie, wenn sie gesund sind, keinerlei Änderung in Form oder Farbe feststellen.}

Harrys Funken sprühten aus seinem Zauberstab heraus und er beobachtete sie, wie sie Richtung Boden fielen. Kurz davor begannen sie sich aufzulösen. Dann richtete er seinen Zauberstab gegen sich selbst und schaute, ob sich die Funken veränderten.

Als die Funken seinen Körper berührten leuchteten sie nur kurz auf und verschwanden dann. Es hatte nicht den Anschein, dass Harry krank war. Er fühlte sich zudem kerngesund. Harry schaute sich um. Es waren nur vereinzelt leichte Färbungen bei anderen zu erkennen. \gedanke{Nichts Schlimmes}, dachte er sich. Madame Pomfrey löste noch die Krankheit des Dummys auf, es war eine mittelschwere Bronchitis. Nachdem die Stunde beendet war, ging er wie gewöhnlich zum Abendessen in die Große Halle.

Es war Zeit zum Abendessen und als Harry die Große Halle betrat, stand Dumbledore geduldig hinter seinem Podest, auf dem er immer Ansprachen hielt. Als alle zum Abendessen gekommen waren und die Türen der Großen Halle geschlossen wurden, begann Dumbledore. \enquote{Nachdem ich heute Morgen unterbrochen worden bin, machen wir eben jetzt weiter. Viele von Ihnen wissen noch, was letztes Jahr passiert ist, also brauche ich mich nicht zu wiederholen. Umso erfreuter bin ich, dass ich jetzt einige Preise vergeben darf. Ich bitte alle Mitglieder des ehemaligen Inquisitionskommandos zu mir vor. Sie haben sich alle den \accentuate{Salazar Slytherin}-Orden für Hinterlist und Spionage verdient.} Eine Unruhe durchdrang nun die Große Halle und nacheinander standen die Inquisitionsmitglieder auf und gingen nach vorne. \enquote{Die Mädchen bitte links von mir, die Jungen bitte Rechts.}

Professor McGonagall stand auf und lief um den Tisch herum zu den Jungs. Als alle standen, nahm Professor Dumbledore von Professor McGonagall ein kleines Holzkästchen entgegen und begann, den Mädchen die einzelnen Orden anzuheften und ihnen zu gratulieren.

\enquote{Bleiben Sie noch kurz stehen}, sagte Dumbledore zu den stolzen Empfängern der Orden. Er schritt zurück zu seinem Podest und Professor McGonagall verschwand währenddessen in einen kleinen Nebenraum, um kurz darauf mit noch ein paar Holzkästchen wiederzukommen. Sie heftete auch den Jungs die Orden an. \enquote{Dieser Schulorden macht ihnen alle Ehre. Sie dürfen sich setzen.} Die Ordensträger gingen jetzt auf ihre alten Plätze, doch Dumbledore stand immer noch hinter dem Podest. Professor McGonagall stand an seiner Seite mit ihrem Holzkästchen in der Hand. Als sich die Slytherins gesetzt hatten, fing Dumbledore wieder an.

\enquote{Ich möchte nun folgende Personen zu mir bitten: Hannah Abbott, Katie Bell, Lavender Brown, Susan Bones, Terry Boot, Cho Chang, Michael Corner, Colin Creevey, Dennis Creevey, Justin Finch-Fletchley, Hermine Granger, Anthony Goldstein, Lee Jordan, Neville Longbottom, Luna Lovegood, Ernie Macmillan, Parvati Patil, Padma Patil, Harry Potter, Zacharias Smith, Alicia Spinnet, Dean Thomas, Fred Weasley, George Weasley, Ginny Weasley und Ronald Weasley.}

Harry wunderte sich, was denn jetzt noch kommen würde. Er stand wie die anderen auf und ging nach vorne. \enquote{Alles Mitglieder der DA} hörte er Hermine sagen.

Harry war erstaunt. \gedanke{Werden wir auch einen Schulpreis erhalten?}, fragte er sich.

Vorne angekommen, stand jetzt eine bunt gemischte Reihe. Dieses Mal hatte  Dumbledore die Schüler nicht nach Geschlecht sortiert. \enquote{Diese Schüler hier}, erklärte Dumbledore, \enquote{haben mehr getan, als jeder andere, letztes Schuljahr. Sie haben sich gegen Verleumdung und Verhetzung gestellt und sich sogar selber ausgebildet, als sie in einem bestimmten Fach nur die Theorie erlernen sollten.} Dumbledore machte eine kurze Pause, damit auch allen klar war, wen er damit meinte.

\gedanke{Umbridge}, kam es Harry in den Sinn.

\enquote{Das Kollegium ist der Meinung, dass dieses Verhalten belohnt werden soll. Ich möchte nun folgende Personen bitten einen Schritt vor zu treten: Hannah Abbott, Katie Bell, Lavender Brown, Susan Bones, Terry Boot, Cho Chang, Michael Corner, Colin Creevey, Dennis Creevey, Justin Finch-Fletchley, Anthony Goldstein, Angelina Johnson, Lee Jordan, Ernie Macmillan, Parvati Patil, Padma Patil, Zacharias Smith, Alicia Spinnet und Dean Thomas.}

\enquote{Sie alle erhalten den Orden der Schule zweiter Klasse.} Ein leises Raunen ging durch die Große Halle. Währenddessen ging Professor McGonagall um die Schüler herum und heftete ihnen jeweils einen Orden an. Die Holzkästchen neben ihr schwebend und das Oberste geöffnet, entnahm sie jeweils einen Orden. Nachdem alle ihre Orden erhalten hatten, machte Dumbledore weiter. \enquote{Den Orden der Schule erster Klasse erhalten folgende Schüler: Hermine Granger, Neville Longbottom, Luna Lovegood, Harry Potter, Ginny Weasley und Ronald Weasley.} Wieder heftete Professor McGonagall die Orden an, dieses Mal aus der unteren Schachtel, die ihre Position nach oben wechselte.

\enquote{Als Letztes möchte ich noch zwei Urkunden vergeben. Eine für die grandiose Idee, diese Gruppe zu gründen und für Verwendung eines Protheus-Zaubers. Hermine Granger, bitte kommen Sie zu mir.} Er zog ein Pergament aus seiner Tasche und überreichte es Hermine. \enquote{Und eine weitere an Harry Potter, für die Leistung ihren Mitschülern die Anwendung der Zauber beizubringen. Vor allem aber den Patronus-Zauber, den sie vielen ihrer Mitschüler beibrachten, obwohl er schwer ist. Viele erwachsene Hexen und Zauberer können das nicht.} Harry ging zu Dumbledore. Dieser überreichte ihm die Urkunde und Harry kehrte dann auf seinen Platz zurück. \enquote{Nachdem nun die Verleihungen beendet sind, wünsche ich allen ein angenehmes Abendessen. \gst Noch eines: Die Schüler, welche die Schule bereits verlassen haben, bekommen die Orden natürlich nachgeliefert.} Er klatschte wieder in die Hände und augenblicklich erschienen auf den Tischen die Speisen und Getränke.

Harry belud seinen Teller und betrachtete, während er aß, ausgiebig seinen Stundenplan. \gedanke{Warum ist mir das nicht schon heute Morgen aufgefallen?} \enquote{Hermine, ich habe auf meinem Plan kein \fach{Verteidigung gegen die dunklen Künste} mehr}.

Hermine schreckte hoch. \enquote{Das kann nicht sein.} Sie zog ihren Stundenplan aus ihrer Tasche und überflog ihn. \enquote{Bei mir fehlt es auch.} In diesem Moment lief Professor McGonagall am Tisch vorbei. \enquote{Professor}, fragte Hermine. \enquote{Warum fehlt bei Harry und mir \fach{Verteidigung gegen die dunklen Künste}? Wir haben dieses Fach doch belegt. Ist das ein Versehen?}

\enquote{Oh nein, Miss Granger. Sie haben dieses Fach jetzt noch nicht. Erst in ein paar Tagen. Es ist etwas dazwischen gekommen.}

Hermines Augen weiteten sich. \enquote{In ein paar Tagen? Und was machen wir so lange?}

Professor McGonagall zog die Schultern hoch. \enquote{Ich bin mir sicher, dass \accentuate{ihnen} etwas einfällt.}

Hermine starrte auf ihren Stundenplan, während Professor McGonagall sich entfernte.

\trenn

Harry stand mal wieder im Wald. Er betrachtete die Thestrale \gst wie schon so oft \gst wie auch in den Tagen an den Wochenenden zuvor. Seine Gedanken glitten dahin und vorsichtig, als ob sie ihn nicht erschrecken wollte, näherte sich ihm ein Thestralweibchen um ihn abzuschnuppern. Thestrale waren eigenartige Tiere, sie konnten nur von Personen gesehen werden, die dem Tod ins Auge geblickt hatten. Harry schmunzelte bei dem Gedanken wie die anderen letztes Jahr sich wohlfühlen mussten, als sie auf einem für sie unsichtbaren Tier nach London flogen. Denn außer für Luna und ihn waren sie für fast alle anderen Schüler nicht sichtbar. Harry fragte sich, ob es eine Auszeichnung sei. Mann musste dem Tod ins Auge schauen, sehen, wie jemand stirbt. Ein grässlicher Gedanke. Doch andererseits wird man mit Thestralen belohnt, sanftmütige Kreaturen, die wohl nur deshalb versteckt waren, weil die meisten sich vor ihnen fürchten. Das Thestralweibchen schnupperte noch immer an ihm und auch die anderen schienen sich für ihn zu interessieren, denn langsam kamen einige von ihnen näher. Harry spürte, dass sich von hinten etwas näherte, das kein Thestral zu sein schien.

\enquote{Hallo Luna}, sagte er, wohl wissend, dass nur sie es sein konnte.

\enquote{Hallo Harry}, kam es ihm verträumt entgegen.

Sie kam näher und stellte sich neben ihn. Ihre Hand berührte leicht die seine. Harry war ganz in Gedanken und bemerkte es zuerst nicht. Es war ein lauer Samstagmorgen, die Sonne schien und erhellte das Waldstück. Es war ein toller Anblick, wie die Lichtstrahlen sich um die Bäume und Büsche herum ihren Weg auf den Boden suchten, um ihn langsam zu erwärmen und die golden glänzenden Lichtstrahlen das Moos und die Blätter auf dem Waldboden ausleuchteten. Nach ein paar Minuten waren sie von einigen Thestralen umkreist, die die beiden beobachteten und beschnupperten. So langsam bemerkte Harry Lunas Hand und sah sie an.

\enquote{Weißt du, Luna}, sagte Harry, \enquote{ich mag dich.}

\enquote{Ich mag dich auch, Harry}, antwortete Luna, und ihre Hand berührte noch immer die seine.

Plötzlich drehte sie sich zu Harry, legte ihre Arme um seine Taille und klammerte sich an ihn, mit ihrem Kopf auf seiner Schulter. Harry konnte nicht anders: Er hielt sie fest. Einige Minuten standen sie so da, von Thestralen umringt, und hielten sich fest. Mit einer unruhigen Stimme begann sie zu erzählen.

\enquote{Ich hab das bisher noch nie jemandem erzählt, Harry. Aber bei diesem Versuch, bei dem meine Mutter starb, habe ich auch meine sechs Monate alte Schwester verloren.} Tränen kullerten ihre Augen herunter. Er hielt sie weiter fest. \enquote{Es schmerzt, immer noch.} Nach einer Weile wischte sie sich ihre Tränen weg. \enquote{Weißt du, Harry \gst ich träume noch heute davon. Es geht mir immer noch nach. Immer noch, nach all den Jahren.} Dann löste sie sich etwas von ihm und ließ ihn los. Beide standen sich nun gegenüber. Noch immer waren ihre Augen feucht und Harry nahm ein Taschentuch und trocknete sie. \enquote{Danke Harry}, sprach Luna.

\enquote{Keine Ursache}, antwortete Harry.

Gerade als er sich wieder den Thestralen zuwenden wollte, nahm Luna seine Hand und zog ihn vorsichtig zu sich. Sie legte ihren Kopf auf seine Schulter und betrachtete die Thestrale. Nach einigen Minuten gab sie ihm einen schamhaft flüchtigen Kuss auf die Backe.

\enquote{Ich gehe jetzt besser zurück}, sagte sie in ihrer verträumten Art. \enquote{Kommst du mit?}

Harry stand da und war vollkommen durcheinander. \gedanke{Will Luna mit mir anbandeln?}, fragte er sich. Doch er verwarf den Gedanken wieder. Einer der Thestrale stupste ihn leicht mit der Schnauze an, in der Hoffnung, das, was in Harrys Tasche war, zu bekommen. Denn Harry hatte sich angewöhnt, immer, wenn er zu den Thestralen ging, ihnen etwas mitzubringen. Er ging in die Hocke und öffnete seine Tasche, die auf dem Boden stand. Er nahm ein kleines papierumschlagenes Paket heraus und wickelte ein paar Brocken rohes Fleisch aus, sowie etwas Gemüse aus seiner Tasche. Er verteilte es auf dem Boden \gst das heißt, er warf es etwa zwei bis drei Meter entfernt von sich auf den Boden, da er die fressenden Thestrale nicht in seiner unmittelbaren Umgebung haben wollte. Das letzte Stückchen behielt er jedoch in der Hand und reichte es dem Thestralweibchen, dass noch immer vor ihm stand. Sie nahm es ihm mit dem Maul aus der Hand und schlang es hinunter. Nachdem sie damit fertig war, leckte sie ihm die nach Fleisch schmeckenden Hände ab. Es war ein ungewöhnliches Gefühl. Er hatte zwar schon mehrere Zungen verschiedener Tiere gespürt, aber keine war vergleichbar mit denen der Thestrale. Es war so, als ob ein pelziger Belag die Zunge umhüllte, obwohl man nichts davon erkennen konnte. Dann drehte er sich um und folgte Luna, die während des gesamten Weges zurück Harrys Hand hielt. \gedanke{Luna will tatsächlich mit mir anbandeln}, dachte Harry und schmunzelte. Noch vor einem Jahr hätte er es nicht für möglich gehalten; als er sie zum ersten Mal in den Schulkutschen traf, die, wie er inzwischen weiß, von Thestralen gezogen werden. Sie hatte damals eine Zeitschrift in der Hand, die sie verkehrt herum las, erinnerte sich Harry.




\begin{kommentar}
Im Lied des Hutes gleich zu Beginn des Kapitels ist ein erster Hinweis darauf, dass das Schloss bereits besteht, als die vier Schulgründer es zur Schule umbauen. Nur sagt der Hut in dem Punkt mit dem Traum nicht ganz die Wahrheit. Aber er ist nur ein Hut und weiß es halt nicht besser.
\end{kommentar}

\begin{kommentar}
Später hat Harry einen seltsamen Traum. Er steht in Bellatrix’ Verlies in Gringotts. Dort entdeckt er den Pokal von Hufflepuff. Ein weiterer Hinweis auf die Magie des Amuletts, die bereits wirkt. Nur weiß Harry da noch nicht, dass der Pokal ein Horkrux ist.
\end{kommentar}

\begin{kommentar}
Als Dumbledore seine erste Rede hält, kam eine blaue Welle durch das Schloss. Leider weiß ich nicht mehr, was ich mir als Ursache dabei gedacht habe. Es war etwas mit Frederick Elber, aber den ursprünglichen Grund habe ich nach der langen Zeit vergessen. Bedauerlicherweise.
\end{kommentar}

\begin{kommentar}
Da aber Elber erst einige Tage später zu unterrichten beginnen kann, muss er wohl bei einem Versuch verletzt worden sein und als Geist (oder ähnliches) durch die Decke gefallen sein.
\end{kommentar}

\begin{kommentar}
In einer windigen Nacht darauf schleicht sich eine Gestalt durch das Schlosstor und in das Schloss. Sie teilt dem Schloss mit, dass es ganz normal reagieren soll. Es ist Frederick Elber, der nicht will, dass man seine Beziehung zum Schloss erfährt, denn er hatte es damals gebaut.
\end{kommentar}

\begin{kommentar}
Kurz danach, nach einem Trenner, kommt ein Satz, der später als kleine Anspielung nochmal auftaucht: >Harry stand mal wieder im Wald.<
\end{kommentar}

\begin{kommentar}
Nach seiner ersten Nacht mit Luna im Gemeinschaftsraum der Paare trifft Harry zum ersten Mal auf Frederick Elber. Dieser erklärt ihm, was es mit dem Patronus-Zauber auf sich hat und dass er manche Gefühle und Erinnerungen für sich behalten sollte.
\end{kommentar}

\chapter{Luna}


Zurück im Schloss ging Harry erst mal ins Badezimmer, um danach mit Ron und Hermine zu Mittag zu essen. Während Ron duschte, sprach Harry einige Worte in seine schwarze Tagebuch-Scheibe. Dann legte er sie auf sein Tagebuch und betrachtete nach einen kurzen aufleuchten was dort geschrieben stand. Er musste morgen noch zu Snape. Der hatte ihm eine Strafarbeit gegeben, weil er wieder einen Trank versaut hatte. Das sollte ihn dazu bringen sich mehr anzustrengen. Harry war stinksauer, doch Hermine lenkte ihn auf dem Weg zur Großen Halle ab.

Während Ron, wie immer sehr begierig aufs Essen, und Hermine, wie immer in einem Buch vertieft, neben Harry saßen, dachte dieser über seine Begegnung mit Luna nach. \gedanke{Ob ich wohl Ron und Hermine etwas davon erzählen sollte?}, dachte er. Nein, das wäre noch zu früh und außerdem wusste er nicht, was sich zwischen ihm und Luna entwickeln würde oder ob sie nur jemanden zum Reden brauchte. Nachdem er mit dem Essen fertig war, gab er Ron und Hermine Bescheid. Er ging zu seinem Gemeinschaftsraum, in dem wenig los war, dann die Treppe hoch und in sein Zimmer, wo er seinen Besen aufbewahrte. Er nahm seine Quidditch-Sachen aus dem Schrank und begab sich hinunter auf das Feld, zog sich um und schwang sich auf seinen Besen. Dann ließ er den Schnatz aus seiner Tasche los, wartete ein paar Sekunden und machte sich auf die Suche. Harry fühlte sich wundervoll; wieder auf seinem Besen zu reiten und den Schnatz zu jagen. Es war herrlich, die sonnige Luft zu berühren, den Wind zu spüren, der ihm ins Gesicht blies. Er zog Schleifen und Kurven, stieg hoch und stürzte sich hinab um den Schnatz zu suchen und zu jagen. Nachdem er ihn mehrmals gefangen hatte und es langsam Abend wurde, packte er seine Sachen zusammen, duschte in den Umkleideräumen auf dem Quidditch-Feld und machte sich auf den Weg zurück ins Schloss. Vor dem Porträt der fetten Dame sagte er routinemäßig das Passwort. Bald würde das Quidditch-Training wieder beginnen.

Das Porträt schwenkte zur Seite und gab den Weg zum Gemeinschaftsraum der Gryffindors frei. Anders als am Morgen war nun viel Betrieb. Die Erstklässler saßen zusammen und einige von ihnen lasen Bücher. Er beschloss erst einmal seinen Besen zu verstauen. Nachdem er in seinem Zimmer angekommen war, den Besen in die Ecke gestellt und seine verschwitzen Quidditch-Klamotten in den Wäschekorb geschmissen hatte, zog er sich für das Abendessen um. Währenddessen herrschte im Gemeinschaftsraum der Gryffindors immer noch geschäftiges Treiben und ein Erstklässler spielte mit einem Sechstklässler gerade Schach.

Harry schaute ihnen eine Weile zu, um sich so die Zeit zu vertreiben bis Ron und Hermine kamen. Nachdem Harry eine viertel Stunde den beiden zugesehen hatte, kamen sie auch schon. Beide hatten ein seltsames Glänzen in den Augen. Sie nahmen ihn mit und gingen wie jeden Tag in die Große Halle zum Abendessen. Dort warf er Luna einen flüchtigen Blick zu, den sie lächelnd erwiderte. Harry setzte sich gegenüber von Hermine, die neben Ron saß. Irgendwie hatte er ein eigenartiges Gefühl während des gesamten Essens. Ron hatte sich noch nie alleine so dicht neben Hermine gesetzt. Näher als all die anderen Male, in denen sie Hermine in ihre Mitte nahmen.

Harry ließ sich nichts anmerken und begann zu essen. Heimlich schaute er immer wieder zu Luna, die seine Blicke erwiderte. Als er gerade auf dem Weg zu seinem Gemeinschaftsraum war, sprach ihn Luna von hinten an.

\enquote{Harry?}

Harry drehte sich um und sagte: \enquote{Ja Luna.}

Luna kam noch ein paar Schritte näher und nahm seine beiden Hände.

\enquote{Ich danke dir, dass du mich heute Morgen gehalten und getröstet hast.}

\enquote{Keine Ursache Luna, das habe ich gerne getan.}

\enquote{Harry Potter, Sir}, kam es plötzlich aus dem Dunklen. Harry erschrak und löste sich sofort von Luna. Still standen beide da und schauten in die Dunkelheit. Dann bewegte sich etwas und ein kleiner grau-häutiger Hauself erschien.

\enquote{Dobby \gst was machst du denn hier?}, fing Harry an, dem es sichtbar unangenehm war mit Luna erwischt worden zu sein.

\enquote{Dobby war gerade am Sauber machen}, sagte Dobby, \enquote{als Dobby Stimmen hörte. Würde Harry Potter und seine Schulkameradin bitte mitkommen. Dobby möchte ihnen etwas zeigen.}

Dobby dreht sich um und verschwand wieder im Dunkel der Gänge. Luna folgte ihm mit ihrem üblichen verträumten Gesichtsausdruck, ohne eine Sekunde darüber nachzudenken, was der kleine Hauself denn nun vorhatte. Nur Harry blieb stehen, nicht wissend was er denn tun solle.

Luna drehte sich kurz um und meinte, \enquote{Komm schon Harry, wird bestimmt lustig werden.}

Also trottete Harry etwas lustlos hinterher. Aber mit jedem Schritt stieg seine Spannung. Sie liefen durch das Treppenhaus und Harry sah zum ersten Mal, wie Dobby die Treppen hinauf lief. Für gewöhnlich apparierte er. Für einen Hauselfen waren die Stufen nämlich entschieden zu groß.

Als Dobby näher an die Stufen herantrat, passierte etwas, was Harry noch nie gesehen hatte. Die Stufen fingen auf der rechten Seite, dort wo Dobby sich näherte, an sich zu teilen. Aus jeder einzelnen Stufe wuchs eine kleine Stufe heraus, sodass sich die Stufen, wo der Hauself lief zwar nur noch halb so breit, aber auch nur noch halb so hoch waren.

Im dritten Stock im Westflügel angekommen, lief Dobby in den Gang hinein und bog nach links ab. Harry erinnerte sich an sein erstes Jahr in Hogwarts, wo im dritten Stock der Zugang zum Stein der Weisen bewacht worden war. Er war nicht oft dort, da es in diesem Teil des Schlosses scheinbar nichts Interessantes gab. Dobby bog rechts um eine Ecke, die in einer Sackgasse endete, und blieb vor einem Porträt stehen. Er dreht sich um und wartete bis beide nah genug waren.

\enquote{Dobby möchte ihnen beiden danken, dass sie sich entschlossen haben Dobby zu folgen.}

\enquote{Keine Ursache}, sagte Harry.

Luna blickte sich nur um. Sie schien sich wohl zu fühlen, soweit man das sagen konnte.

\enquote{Was Harry Potter und seine \gst äh\abs}

\enquote{Luna, Luna Lovegood}, antwortete Harry.

\enquote{Was Harry Potter und Luna Lovegood versprechen sollten, ist, niemandem etwas hiervon zu erzählen.}

\enquote{Ich verspreche es}, antworteten Harry und Luna fast gleichzeitig.

\enquote{Das hier ist der fünfte Gemeinschaftsraum}, sagte Dobby.

\enquote{Kommt}, er drehte sich um und sprach zu dem Porträt: \zauber{Aqua Neros} und das Porträt öffnete sich.

Dobby schritt voran und winkte die beiden herein. Luna und Harry betraten den Raum und das Porträt hinter ihnen versperrte wieder den Eingang. Nach ein paar dunklen Metern und einer Biegung gelangten sie in einen Raum, der um einiges größer als der Gemeinschaftsraum der Gryffindors aussah, war. Luna schaute sich nur verträumt um und lief umher.

\enquote{Was ist das hier, Dobby?}, fragte Harry.

\enquote{Das}, Dobby drehte sich zu Harry um, \enquote{ist der Gemeinschaftsraum der Paare}, antwortete Dobby.

Harry verschlug es die Sprache. Für ein paar Sekunden konnte er gar nichts mehr sagen. Dachte Dobby vielleicht Luna und er wären ein Paar?

\enquote{Aber Luna und ich sind kein\abs}, doch Dobby unterbrach ihn.

\enquote{Paar? Vielleicht nicht in dem Sinne wie Sie denken, Sir. Aber Dobby hat zwischen Ihnen und Miss Luna eine Verbindung gespürt, keine Liebesbeziehung wie zwischen den anderen Pärchen, die Dobby ab und an beobachtet hat.}

\enquote{Du spionierst uns allen hinterher?}, fragte Harry erstaunt.

\enquote{Nein}, antwortete Dobby, \enquote{ich treffe sie nur zufällig bei meiner Arbeit. Halte mich aber immer versteckt, sodass sie mich nicht sehen können.}

\enquote{Ja aber was sollen wir in dem Gemeinschaftsraum für Paare, wer weiß noch davon, \abs}

Dobby unterbrach ihn abermals. \enquote{Setzen Sie sich Harry Potter, Sir, und ich werde es ihnen erklären.}

Harry setzte sich und auch Luna nahm auf dem Sofa neben ihm Platz. Immer noch mit ihrem leicht verträumten Gesichtsausdruck.

\enquote{Harry und Luna müssen wissen, dass außer den Hauselfen und Ihnen beiden niemand hier im Schloss etwas über diesen Raum weiß. Nicht einmal die Lehrer hier, oder der Hausmeister. Dobby hat ihn bei seiner Einweisung durch die anderen Hauselfen hier gezeigt bekommen. Aber da hier nie einer ist, braucht er auch nicht sauber gemacht zu werden. Doch Dobby hat zwischen Ihnen und Miss Luna etwas gespürt und sich entschlossen, Ihnen diesen Ort zu zeigen.}

Dobby senkte den Kopf und sprach weiter. \enquote{Es wäre Dobby eine Ehre, wenn Sie und Miss Luna diesen Ort hier benutzen würden.}

Harry war perplex, so etwas hatte er nicht erwartet.

\enquote{Wie \gst benutzen? Und was ist mit den anderen Pärchen?}

Dobby sah wieder hoch zu Harry, schaute Luna an und sprach dann.

\enquote{Noch nicht. Für die anderen ist es zu früh. Dobby lässt es Harry Potter wissen, wenn er es für richtig hält} und er fügte hinzu, \enquote{sofern Harry Potter, Sir, nichts dagegen hat.}

\enquote{Nein Dobby}, antwortete Harry.

Dobby sprach weiter: \enquote{Vor vielen Jahrzehnten wurde dieser Ort von zwei Schülern hier errichtet} und er deutete auf das Bild über dem Kamin hinter ihm. \enquote{Er diente ihnen als Liebesnest. Doch nach ihrem siebten Jahr vergaßen sie, es den anderen Pärchen mitzuteilen.} Dobby nahm wieder seine Hand herunter und blickte nun zu Harry. \enquote{Nur Sechst- und Siebtklässler dürfen diesen Raum betreten. Fünftklässlern ist der Zutritt nur gestattet, wenn sie einen Sechst- oder Siebtklässler als Partner haben.}

\enquote{Und woher wissen dann die Hauselfen davon?}, fragte ihn Harry.

\enquote{Dazu wollte Dobby gleich kommen. Da sich die Hauselfen hier um alles kümmern, haben die beiden beschlossen ihnen die Aufgabe des Reinigens zu überlassen. Sie ließen ihnen immer dann eine Nachricht zukommen, wenn sie es für nötig hielten, dass der Raum gereinigt werden sollte.} Und Dobby fügte hinzu: \enquote{Sie und Miss Lovegood sind die ersten nach einer sehr langen Zeit, die diesen Raum wieder benutzen dürfen.}

\enquote{Danke Dobby}, sagte Harry \gst Luna lächelte Dobby nur an und sagte dann: \enquote{Ja, danke Dobby} und sah sich weiter um.

\enquote{Dobby muss Sie jetzt verlassen Harry Potter, Sir, er hat noch andere Arbeit}, sagte der Elf und verließ den Raum.

Harry und Luna schauten sich im Raum um als plötzlich Luna sagte: \enquote{Schau mal, hier ist ja ein Schaukelstuhl. Der sieht genau so aus wie die in unserem Gemeinschaftsraum.}

Harry fiel auf, dass er noch nie im Gemeinschaftsraum der Ravenclaws oder der Hufflepuffs gewesen war. Nur einmal im zweiten Schuljahr war er bei den Slytherins gewesen. Nun schaute er sich genauer um und bemerkte, dass wohl von jedem Gemeinschaftsraum etwas hier sein musste. Luna blickte in der Zwischenzeit zur Decke und sah, dass dort die vier Hausfahnen hingen.

\enquote{Schau mal da oben}, sagte sie zu Harry ohne ihn dabei anzusehen. Harry blickte nach oben und sah die Fahnen jetzt auch. \gedanke{Wahnsinn}, dachte er.

Nach einer Weile im Raum fiel sein Blick auf eine Art Ankündigungstafel. Er näherte sich und las laut vor.

\begin{brief}
Gemeinschaftsraum der Paare erbaut und erdacht im Jahre 1875

von Sardak Slyhoot (Slytherin) und Selvine Vertap (Hufflepuff)

letzte Benutzung im selbigen Jahr

Reaktivierung durch Harry Potter (Gryffindor)

und Luna Lovegood (Ravenclaw) im Jahre 1996
\end{brief}

Harry bekam ein komisches Gefühl seinen und Lunas Namen auf der Tafel zu sehen. Er entschloss sich zu gehen, als er einen schwebenden Brief im Ausgang bemerkte. Er war an ihn adressiert. Er drehte ihn um und las den Namen Dobby. Nachdem er ihn geöffnet hatte, las er Folgendes:

\begin{brief}
Hallo Harry Potter, Sir.

Ich vergaß ihnen mitzuteilen, dass sie bitte mit Miss Lovegood die heutige Nacht hier verbringen sollen, im gleichen Bett. Leider hatte Dobby keine Zeit mehr um umzukehren, so schicke ich ihnen den Brief auf diesem Weg.
\signumspace
Stets zu Diensten

Hauself Dobby
\end{brief}

Harry verschlug es die Sprache. \gedanke{Dachte Dobby, ich sollte mit Luna schlafen? Oder meinte er nur, wir sollen die Nacht miteinander verbringen; im selben Bett, schlafend?} Er hatte nicht bemerkt wie sich Luna ihm näherte und den Brief mitlas.

\enquote{Schön} sagte sie, \enquote{dann verbringen wir die Nacht eben gemeinsam hier}. Sie drehte sich wieder um und ging an einen Tisch mit Schachbrettmuster auf der Platte, setzte sich in einen Stuhl und fragte, \enquote{Harry, hast du Lust auf eine Runde Schach?}

Das riss Harry wieder los, er drehte sich um und fragte Luna \enquote{Was? Du kannst Schach spielen? Hab ich gar nicht gewusst.}

\enquote{Du weißt vieles von mir noch nicht, Harry}, sagte sie und bot ihm den Platz gegenüber an.

Harry machte sich auf den Weg zu ihr. In der Zwischenzeit holte Luna ihren Zauberstab aus der Tasche und tippte das Schachbrettmuster mit der Spitze ihres Zauberstabes an. Als die Figuren auf dem Brett erschienen, setzte sich Harry und Luna nahm einen weißen und einen schwarzen Bauern vom Spielbrett. Sie verschränkte hinter ihrem Rücken die Arme und tauschte die Figuren ein paar mal hin und her. Dann nahm sie ihre Arme wieder nach vorne und zeigte ihm beide Hände ausgestreckt zur Wahl der Farbe. Harry entschied sich für die linke Hand und durfte beginnen. Luna setzte die zwei fehlenden Bauern auf ihre Positionen. Da Luna die weißen Figuren bei sich hatte, fing das Spielbrett an leicht abzuheben und sich um 180 Grad zu drehen.

Nach einem 3:1 für Luna war es auch schon recht spät und die beiden entschieden sich ein Zimmer zu suchen. Harry holte nochmals Dobbys Brief aus der Tasche. Nur um sicherzugehen, dass sie auch wirklich das gleiche Bett benutzen sollen. Ihm war leicht flau. Sie nahmen das erstbeste Zimmer und öffneten die Tür. Harry ließ Luna den Vortritt.

An der rechten Wand im Zimmer stand ein großes doppeltes Himmelbett mit Vorhängen, die man um das Bett zuziehen konnte. Die Farbe der Vorhänge bestand aus einem leichten Creme-Ton und durch das Fenster an der gegenüberliegenden Seite schimmerte der Mond herein. Es war nicht besonders hell, aber man konnte genug erkennen. Gegenüber der Tür war ein Schrank. \gedanke{Darin verbergen sich wohl Anziehsachen}, dachte Harry. Er lief um das Bett herum und schlug die Bettdecke zurück. Darunter kamen Lunas und sein Schlafzeug zum Vorschein. Dobby hatte wie immer, wie alle Hauselfen, perfekte Arbeit geleistet. Er drehte sich um, setzte sich auf die Bettkante und begann seine Robe aufzuknöpfen. Luna tat es ihm gleich, nachdem sie sich auf die andere Seite des Bettes gesetzt hatte. Nachdem beide ihre Schlafsachen angehabt hatten, krabbelte Harry ins Bett, wo Luna schon lag. Beide lagen nun nebeneinander und betrachteten die verzauberte Decke des Himmelbettes. Sterne schimmerten und ab und an kam eine Sternschnuppe vorbei. Luna griff nach Harrys Hand, legte den Kopf zur Seite um Harry anzuschauen und sagte mit ihrem üblichen Gesichtsausdruck.

\enquote{Gute Nacht Harry.}

\enquote{Gute Nacht Luna}, antwortete Harry und machte die Augen zu. Er dachte an Dobby: \gedanke{Ich kann ihm vertrauen. Auch wenn ich noch nicht weiß, was er will}, dachte sich Harry und schlief ein.

Als es am nächsten Morgen langsam heller wurde und die ersten Sonnenstrahlen den Raum anfingen zu erhellen, waren beide mit ihren Körpern zueinander gedreht, immer noch die Hände ineinander gelegt. Luna öffnete ihre Augen und sagte: \enquote{Guten Morgen Harry.}

Harry war noch etwas schläfrig und öffnete seine Augen. Da lag sie, Luna Lovegood. Doch etwas an ihr war anders als sonst. Ihr verträumter Blick war verschwunden und sie hatte einen durchdringenden und musternden Ausdruck auf ihrem Gesicht. Nicht dass sie ihn anstarren würde, aber ihr verträumter Blick war verschwunden. Sie schaute sich jeden Zentimeter seines Gesichtes genau an. Dann passierte etwas, was Harry einen wohligen Schauer über seinen Rücken laufen ließ. Ihr Gesicht kam seinem näher und sie küsste ihn. Nicht auf die Backe wie am Morgen zuvor bei den Thestralen. Nein, sie küsste ihn direkt auf den Mund. Harry war erstaunt, aber nicht überrascht, irgendwie hatte er gespürt, dass sie ihn küssen würde. Es war ein eigenartiges Gefühl. Sie ließ seine Hand los und sagte: \enquote{Ich hoffe, es gibt wieder Toastbrot und etwas Speck.}

Und da war er wieder, der leicht verträumte Gesichtsausdruck den sie die ganze Zeit hatte. Sie stand auf und zog sich um. Harry dachte an Lunas Kuss. Es war so ein komisches Gefühl. Nicht zu vergleichen mit dem Kuss von Cho und den Schmetterlingen, die er in seinem Bauch gefühlt hatte, als er sie in seinem fünften Jahr ansah oder küsste. Etwas anderes war da. Etwas was ihn glücklich machte, ohne verliebt zu sein. Er konnte es nicht beschreiben. Er stand auf und begann sich umzuziehen. Luna war in der Zwischenzeit schon wieder im Gemeinschaftsraum der Paare und wartete.

Als Harry hereinkam, fragte sie ihn: \enquote{Sehen wir uns wieder? Hier? Vielleicht in zwei Wochen? Zur selben Zeit am selben Tag, an dem uns Dobby hierhergeführt hat?}

\enquote{Ja}, antwortete Harry knapp und fügte noch ein: \enquote{Gerne Luna}, hinzu.

Sie marschierten wieder Richtung Ausgang und bogen um die Ecke, als sie bemerkten, dass man, anders als bei den anderen Gemeinschaftsräumen, sehen konnte, was draußen passiert, ohne dass man hereinschauen konnte. Sie verließen den Raum und liefen noch ein Stück gemeinsam.

\enquote{Das behalten wir aber für uns, Luna}, sagte Harry.

\enquote{Ja natürlich. Mir würde das sowieso keiner glauben}, meinte Luna.

Ihre Wege trennten sich und jeder machte sich auf den Weg zum jeweiligen Gemeinschaftsraum. Als Harry bei seinem ankam, waren schon einige seiner Mitschüler auf. Ron und Hermine spielten gerade eine Partie Schach, um Hermines Spiel zu verbessern. Harry betrat den Raum und lief Richtung Jungenschlafsäle, um sich zu duschen.

\enquote{Wo warst du?}, fragte Ron.

\enquote{Draußen. Den Sonnenaufgang anschauen}, antwortete Harry.

Er traute sich nicht über die Nacht mit Luna zu reden.

\enquote{Jetzt? Sonntagmorgen? Du spinnst}, meinte Ron.

Harry zuckte mit den Schultern und begab sich nach oben.

\enquote{Manchmal ist er schon merkwürdig}, sagte Ron zu Hermine.

Harry kam in der Zwischenzeit oben an, zog seine Sachen aus und duschte wie schon am Morgen zuvor. Gerade als er das klare kühle Wasser über sein Gesicht fließen spürte, fiel ihm der eigenartige Zauberer ein, der ihm bei einem seiner morgendlichen Spaziergänge vorige Woche begegnet war.

\begin{rueckblick}
Er lief ein Stück neben ihm her, als er sagte: \enquote{Dementoren \gst Harry} und der Fremde schaute ihn an und fuhrt fort: \enquote{Dementoren kann man, wie sie sicherlich wissen, mit dem Patronus Zauber vertreiben. Und ich habe gehört, dass sie ihn schon erfolgreich angewendet haben.}

\enquote{Ja}, antwortete Harry wahrheitsgemäß.

\enquote{Sie müssen wissen, dass die effektivsten Gedanken die positivsten sind. Aber \gst und das wissen nicht viele, die positiven Gedanken, die man für sich behält und keinem anderen sagt, das sind die Besten.}

\enquote{Woher wollen sie das wissen?}, fragte Harry und schaute den fremden Mann an.

Der Fremde betrachtete Harry, dann den See hinter ihm, und zeigte auf den See.

\enquote{Da, schauen sie}, meinte er.

Harry drehte sich um, doch da war nichts. Als er den Fremden fragen wollte, was er denn gesehen habe, und sich zu ihm umdrehte, war dieser verschwunden. \gedanke{Wo ist er hin?}, wunderte sich Harry.
\end{rueckblick}

Mittlerweile war er mit Duschen fertig und lief mit einem Handtuch um seine Hüfte in seinen Schlafsaal, um sich umzuziehen. Nachdem er fertig war und in den Gemeinschaftsraum der Gryffindors zurückgekehrt war, nahm er Ron und Hermine mit, um mit ihnen zu Frühstücken. Auf dem Weg dorthin begegneten sie wieder Luna, die ihn nur verträumt anlächelte und weiter lief. Hermine schaute Harry etwas komisch an, sagte aber nichts. Das war Harry nur Recht. Er wollte sie nicht belügen. Aber er wollte auch nichts über seine gemeinsame Nacht mit Luna erzählen.

In der Großen Halle angekommen, waren schon einige beim Frühstücken und der Rest der Schule würde auch bald folgen. Harry nahm sich heute mal etwas Butter und Marmelade auf den Toast. Er wollte das schon immer mal versuchen, statt des sonst üblichen Speck mit Bohnen und Toast, dass er sonst hatte. Er fand, dass es nicht besonders übel schmeckte, und dachte sich: \gedanke{Das nimmst du jetzt jede zweite Woche zu dir. Und zwar immer sonntags, nachdem du mit Luna zusammen warst.} Während er sein Frühstück kaute und seinen Kürbissaft trank, fiel ihm die Bibliothek ein. Nach seinem Frühstück verabschiedete er sich von Ron und Hermine und sagte ihnen, sie würden ihn in der Bibliothek finden, falls sie ihn brauchen sollten.

Dort angekommen fragte er Madame Pince, wo er denn etwas über Dementoren und positive Gedanken zu deren Abwehr finden würde, da er vermutete, die gesuchte Lektüre würde sich im abgesperrten Teil der Bibliothek befinden. Madame Pince gab ihm durch ihren durchdringenden Blick zu verstehen, dass er das hätte besser nicht fragen sollen. Nach einigen Sekunden, die Harry viel zu lange vorkamen, besserte sich schließlich ihr Blick und sie sagte: \enquote{Ausnahmsweise Mister Potter, folgen sie mir}. Harry war erstaunt über so viel entgegenkommen. Madame Pince war normalerweise nicht so.

\enquote{Wissen Sie}, sagte sie, als sie mit Harry im Schlepp durch die Bibliothek lief, \enquote{da sie schon öfters gegen Dementoren gekämpft haben und die sie anscheinend für ein besonders lohnendes Opfer halten, gestatte ich ihnen einen kurzen Blick in das entsprechende Kapitel. Wenn sie mehr wissen wollen, fragen sie einen Lehrer und bringen mir eine schriftliche Erlaubnis.} Harry bejahte, und Madame Pince, bereits angekommen, öffnete den Zugang zum abgesperrten Bereich der Bibliothek. Sie schritt hindurch, dicht gefolgt von Harry, und bog ein- zweimal ab, um vor einem Regal haltzumachen. Sie griff eines der Bücher heraus und reichte es ihm. \enquote{Sie haben fünf Minuten und ich bleibe hier bei ihnen stehen.}

Harry nickte abermals und schlug das Buch auf. Im Inhaltsverzeichnis fand er unter der Rubrik Abwehr von Dementoren den Eintrag über positive Gedanken. Er schlug die entsprechende Seite auf und las. Nachdem er wieder zum Inhaltsverzeichnis zurückgekehrt war, bemerkte er ein Kapitel über die Fortpflanzung und die Aufzucht von Dementoren. Leider waren die fünf Minuten schon abgelaufen und Madame Pince forderte das Buch zurück. Harry wusste nicht, ob er froh oder niedergeschlagen sein sollte. Einerseits hätte es ihn schon interessiert, aber andererseits hatte er nie so richtig darüber nachgedacht, wie Dementoren ihre Jungen aufziehen, bzw. ob sie sich wirklich vermehren würden. Madame Pince begleitete Harry wieder in den normalen Teil der Bibliothek zurück, schloss ab und widmete sich wieder ihrer Arbeit.

Auf dem Rückweg zu ihrem Schreibtisch am Eingang der Bibliothek, schaute sie begierig in die einzelnen Reihen um irgendwelche Verstöße festzustellen. Denn sie hasste es, wenn in ihren Büchern Flecken oder Krümel zu finden waren. Zwar gehörten die Bücher der Schule und nicht Madame Pince, aber sie benahm sich so als seien es ihre. Und den Professoren schien das nichts auszumachen, denn sie konnten sich immer darauf verlassen, wenn sie mal ein Buch benötigten, dass es flecken- und krümmelfrei sei. Madame Pince machte zwar immer ein riesiges Theater, wenn sie wieder jemanden erwischte, der über ihren Büchern seine Brotzeit oder sein Getränk ausgebreitet hatte, aber Harry fand in seinem sechsten Jahr, dass sie gar nicht so übel sei; wenn man sich mit ihr arrangieren konnte. Sie schnauzte die Lehrer genauso an wie die Schüler, wenn die Bücher verschmutzt zurückgegeben wurden, und das machte sie in Harrys Augen noch sympathischer.

Harry musste grinsen, als er daran dachte, wie Madame Pince einmal Dumbledore zusammengeschrien hatte, weil auf dem Buchdeckel ein runder Getränkefleck zu sehen war. Zwar benötigte sie nur einen Schwenk mit ihrem Zauberstab, um das Buch zu säubern, aber sie erachtete das immer als unnötig und nicht ihre Aufgabe. Ihre Devise war: Die Bücher müssen sorgfältig behandelt werden und ordnungsgemäß abgeliefert werden.

Harry machte sich auf die Suche nach einem Buch über Zaubertränke, denn er musste für Snape noch einen Aufsatz schreiben. Er machte sich nicht allzu viele Hoffnungen, da er bei Snape sowieso keine guten Noten bekommen würde. Aber dieses Thema, das sie gerade durchnahmen, interessierte ihn doch und er entschied sich, sich dieses Mal mehr anzustrengen. \gedanke{Aber nicht um Professor Snape einen Gefallen zu tun}, dachte er sich, \gedanke{sondern weil ich es will.} Als er nur noch wenig Zeilen zu schreiben hatte, tauchten Ron und Hermine mit Ginny im Schlepptau auf.

\enquote{Hi Harry} sagte Ginny und setzte sich gegenüber Harry, nachdem sie ihn beim Frühstück nicht gesehen hatte. \enquote{Hi Ginny} sagte Harry, schaute auf, und für einige Zeit konnte er seinen Blick nicht von ihr lassen.

Zu allem Überfluss saß Hermine genau neben ihr und schaute ihn an. In diesem Licht, das durch die Bibliothek hereinschien, und in dieser neuen Robe hatte er Ginny noch nie gesehen. Er kannte sie zwar seit seinem ersten Schuljahr und hatte auch sonst regen Kontakt mit ihr, aber so hatte er sie noch nie gesehen. Und direkt daneben Hermine. Harry wusste nicht, was er machen oder denken sollte. Zwei bezaubernde junge Mädchen, die ihm gegenüber saßen und sein jugendliches Blut in Wallung brachten.

\enquote{Wie zwei Engel}, brach es aus ihm heraus. Als er merkte, dass er das nicht nur dachte, sondern den beiden direkt ins Gesicht sagte, senkte er beschämt seinen Blick. Ginny wurde sofort rot und drehte sich mit einem breiten Schmunzeln im Gesicht weg. Doch auch Hermine ging es nicht besser. Auch sie errötete, etwas weniger als Ginny aber dennoch. Peinlich berührt drehte er sich zur Seite. Leider in die Richtung, in der Ron saß, der ihn nur mit offenem Mund anstarrte. So etwas hätte er von Harry nie erwartet. Harry entschied, es sei das Beste seinen Aufsatz zu Ende zu Schreiben und dann seine Sachen in sein Zimmer zu bringen.

Später am Abend war er auf dem Weg zu den Kerkern. Es musste bei Snape nachsitzen. Er wusste, dass es nicht gerade angenehmen werden würde. Er klopfte an die Bürotür und sie schwang auf. In Professor Snapes Büro brannte im Kamin ein kleines Feuer. Snape saß hinter seinem Schreibtisch und korrigierte Hausaufgaben. \enquote{Setzen Sie sich, Potter} sagte er, ohne aufzuschauen. Harry lief zum Tisch und setzte sich auf den Stuhl. Dann wartete er. Snape korrigierte diese Hausaufgabe zu Ende und widmete sich dann Harry.

\enquote{Wie laufen ihre Okklumentik-Übungen?}, fragte er Harry. Harrys Gesicht versteinerte. \enquote{Nun?}, fragte Snape und nahm sich die nächste Hausaufgabe vor.

\enquote{Na ja, ich hatte mich in letzter Zeit nicht mehr darum bemüht. Nicht nachdem sie den Unterricht beendet hatten.}

Professor Snape sah auf. \enquote{Bitte? Ich dachte, sie führen sie trotzdem weiter!?} Jetzt war Harry vollkommen perplex. Professor Snape unterbrach seine Korrekturarbeiten, als Harry nicht mehr reagierte. \enquote{Ich war wütend und sauer. Ich wollte nicht, dass sie das sehen. Aber jetzt habe ich erkannt, dass sie es mehr als nur notwendig haben.} Snape legte die Hausaufgabe wieder auf den unerledigten Stapel und die Feder beiseite. Er ging um den Tisch herum und setzte sich Harry gegenüber. \enquote{Schließen Sie ihre Augen.} Widerwillig tat Harry, was von ihm verlangt wurde. \enquote{Was sehen sie?}, fragte Snape.

\enquote{Nichts. Es ist dunkel.}

\enquote{Gut. Stellen Sie sich vor, dass sie vor einer schwarzen Wand stehen. Stellen Sie sich ferner vor, dass hinter ihnen ebenfalls eine schwarze Wand ist. Sie fühlen sie in ihrem Rücken. Die Wand vor ihnen ist nur wenige Zentimeter von ihnen entfernt. Links und rechts ist frei, aber nur Dunkelheit. Stellen Sie sich vor, dass sie Müde werden. Sie sehen nur die schwarze Wand und stellen sich vor, dass sie Müde werden.}

\enquote{Verlassen Sie jetzt ihren Körper und sehen sie sich an, wie sie Müde werden. Ihre Gedanken verlassen sie.} Harry spürte, wie er langsam in einen leichten Dämmerschlaf glitt. Snapes Stimme schien eine beruhigende Wirkung auf ihn zu haben. Er hatte ihn mit keinerlei Aggressivität oder Hass angesprochen. Er sah sich, wie er zwischen zwei Wänden stand. Vollkommen entspannt. Sein Geist begann sich zu leeren.

\enquote{Öffnen Sie jetzt wieder ihre Augen.} Harry öffnete seine Augen und sah Snape direkt an. \enquote{Was denken sie?}

\enquote{Wenig. Ich dachte an\abs} Plötzlich kamen immer mehr Gedanken wieder in seinen Sinn.

\enquote{Professor, es kommen wieder mehr Gedanken in meinen Geist.}

\enquote{Gut, sehr gut, für das Erste mal. Machen sie weiter, wiederholen sie diese Übung.}

Harry nickte und schloss seine Augen. Er hörte noch wie Snape aufstand und sich hinter sein Pult setze, ein Pergament nahm und darauf herumkratzte.

Als er seine Augen wieder aufmachte, begann er nach einigen Sekunden kratzende Geräusche wahrzunehmen. Er drehte seinen Kopf und sagte: \enquote{Das ist eigenartig. Ich habe ein paar Sekunden keine Geräusche mehr wahrgenommen. Anders als beim ersten Mal.}

Snape nickte und sah auf seine Uhr. \enquote{Ihr Nachsitzen ist für heute vorbei. Üben Sie bis zum nächsten Termin. Ich werde ihnen schon Bescheid sagen. Dann werde ich versuchen wieder in ihren Geist einzudringen. Es ist also besser, sie üben abzuschalten.}

Dann wandte sich Professor Snape wieder seinen Hausaufgaben zu und beachtete Harry nicht mehr. Als Harry die Tür geöffnet hatte, hörte er noch: \enquote{Und kein Wort! Zu niemandem!} Harry verstand und nickte. \enquote{Verstanden Professor!} Auf dem Weg zu seinem Bett schwirrte sein Kopf. \gedanke{Warum tat Snape das?}

Aber die spannendere Frage war: Würde er es wiederholen, oder war das eine einmalige Sache?

\trenn

Kurz nach dem Mittagessen war es wieder so weit, dass sich das Gryffindor-Quidditch-Team, um die aktuelle Lage zu besprechen und zu trainieren, auf dem Quidditch-Feld traf. Nachdem alle umgezogen und mit ihren Besen bereit waren, begann das Training. Auf Katies Zeichen saßen alle auf und flogen hinaus auf das Feld. Katie holte ihren Zauberstab heraus und zeigte auf die Truhe unten auf dem Feld. Sie öffnete sich und der Schnatz, sowie die beiden Klatscher sprangen heraus. \zauber{Accio Quaffel} sprach Katie und steckte ihren Zauberstab sofort weg.

Der Quaffel bewegte sich auf sie zu und wurde sofort von den Treibern abgefangen. Diese versuchten nun den Ball in eine der drei runden Tore zu bekommen.

Nach dem Training und einer Dusche in den Duschräumen des Quid\-ditch-Team\-rau\-mes, ging Harry zurück in sein Zimmer, schaute auf seinen Stundenplan und entdeckte eine Änderung.

\enquote{Was?}, rief er, \enquote{Mittwochabend eine Doppelstunde Verteidigung gegen die dunklen Künste und donnerstagmorgens ebenso?}

Harry starrte seinen Stundenplan an. Warum hatte er erst in seiner vierten Woche Verteidigung gegen die dunklen Künste? Hatte Dumbledore jetzt doch jemanden gefunden? Oder konnte der Professor vorher nicht? Harry rannte hinunter in den Gemeinschaftsraum, um es Ron und Hermine zu erzählen. Doch die hatten auch ihre Stundenpläne in der Hand und schauten erstaunt.

\enquote{Harry, hast du es schon gesehen?}, fragte Hermine.

\enquote{Ja, gerade eben}, antwortete Harry.

\enquote{Was hat das zu bedeuten?}, fragte Hermine. \enquote{Erst gar kein Unterricht, nicht einmal eine Vertretung durch Snape wie im dritten Jahr, und jetzt gleich zwei Tage hintereinander Doppelstunden.}

\gedanke{Ich werde es noch bald genug herausfinden}, dachte Harry und machte sich daran in die Große Halle zum Abendessen zu gehen. Beim Hineingehen fiel ihm ein Mann auf, den er nur von hinten sah. Er war in der Hocke und lehnte seine Arme auf den Lehrertisch; direkt gegenüber von Dumbledore. Er schien sich mit ihm zu unterhalten. \gedanke{Irgendwie kommt er mir bekannt vor}, dachte Harry und begann sein Abendbrot einzunehmen. Als er gelegentlich wieder zum Lehrertisch herüberschaute, entdeckte er wie der unbekannte Mann aufstand und sich an den freien Platz an der Stirnseite des Lehrertisches aufseiten des Slytherin-Haustisches setzte. Es war ein Mann mittleren Altern, schätzungsweise 35. Er hatte einen Umhang an, der schmutzig aussah und nur vereinzelte Stellen eines goldenen Stoffes aufwies. Seine Hosen waren in dunklem grün gehalten und sein Oberteil war Kastanienbraun. Harry erkannte ihn wieder und dachte an den fremden Mann, den er einmal während einer seiner wenigen morgendlichen Spaziergänge getroffen hatte, als er mal nicht joggte. \gedanke{Komisch}, dachte er sich. \gedanke{Ist das der neue Professor in Verteidigung gegen die dunklen Künste?}

\trenn

Am Montag-Morgen hatte Harry, wie dieses Jahr üblich, Unterricht bei Professor McGonagall im Fach Verwandlung. So langsam trafen alle ein, aber McGonagall war nicht da. Harry schaute sich um, aber er fand sie nicht. Auch die Katzengestalt, die sie immer mal wieder annahm, sah er nirgends. \enquote{Das sieht ihr gar nicht ähnlich}, sagte Hermine, die genau hinter ihm stand. Sie setzten sich und als alle Schüler da waren, trat Professor McGonagall ein. Sie hatte den fremden Mann hinter sich, der ihr folgte. Nachdem sie ihr Pult erreicht hatte und sich dahinter gestellt hatte, blieb der Mann neben ihr stehen.

\enquote{Dies}, so sprach sie, \enquote{ist Professor Elber. Er wird sie in Verteidigung gegen die dunklen Künste unterrichten. Er ist hier, weil Professor Dumbledore meint, dass es an der Zeit ist, etwas in diesem Fach zu tun. Mehr als sie die vergangenen Jahre getan haben. Er hat noch nie unterrichtet und hat mich daher gebeten, einmal zuschauen zu dürfen. Sie werden ihn vor Mittwochabend in einigen anderen Klassenräumen antreffen.} Sie drehte sich zu ihm um und fragte ihn: \enquote{Möchtest du noch was sagen?} Harry hörte zum ersten Mal, dass Professor McGonagall einen anderen Lehrer duzte. Es war eigenartig.

\enquote{Ja, das würde ich gerne.} Er dreht sich zur Klasse und sprach weiter \enquote{Meinen Namen kennen sie ja bereits alle und außer der Tatsache, dass sie alle pünktlich zu meinem Unterricht erscheinen werden, gibt es momentan nichts zu sagen. \gst Ach ja, noch eines. Wenn sich mein Fach auf ihrem Stundenplan rot färbt, dann werfen sie einen Blick auf die Rückseite und lesen Sie sie bitte. Durch Antippen verschwindet die Schrift und die rote Markierung in meinem Fach}. Er dreht sich zu Professor McGonagall um und sagte zu ihr: \enquote{Du kannst anfangen, Minerva, ich bin fertig}.

Professor McGonagall hielt ihren Unterricht wie immer und man merkte, dass, anders als letztes Jahr, sie sich zwar genauso wenig anmerken ließ, dass jemand dabei war, der sie beobachtet, aber sie war wesentlich entspannter. \enquote{Heute}, so Professor McGonagall, \enquote{nehmen wir die Verwandlung von Gold in Lebewesen vor. Ein schwieriges Unterfangen. Zu diesem Zweck bekommen sie von mir je zwei Galeonen Wichtel-Gold, das nach dem Ende der Stunde verschwinden wird.}

\gedanke{Wichtel-Gold}, dachte Harry. \gedanke{So eines hat mir mal Ron gegeben. Doch nach kurzer Zeit war es verschwunden.}

Professor McGonagall öffnete eine ihrer Schubladen und nahm ein Säckchen heraus. Auf jeden Platz legte sie zwei Münzen. Als sie fertig war, nahm sie eine Münze aus dem Säckchen und warf es auf ihr Pult. Danach legte sie die Münze ebenfalls auf ihr Pult und fing an. \zauber{Aurumomorph chelys} Sie schwang ihren Zauberstab in einer eleganten Bewegung und das Goldstück verwandelte sich in eine kleine Schildkröte. Sie drehte sich wieder zur Klasse und sprach weiter.

\enquote{Auf diese Art und Weise, können sie ihr Geld verstecken, oder auch andere täuschen. Bitte versuchen Sie es alle.} Harry schaute zu Hermine, die sich sofort darüber hermachte. Funken sprühten aus ihrem Zauberstab und die Münze begannt sich zu verformen. Leider schaute das Ergebnis nicht sehr schön aus. Es war zwar eine Schildkröte, aber ihr Panzer sah immer noch wie eine Seite einer Münze aus. \enquote{Für diesen Zauber, wie für alle in meinem Unterricht, brauchen sie viel Konzentration. Fahren Sie fort.}

Die Stunde verlieft relativ ruhig und nach der Hälfte der Zeit hatte es Harry geschafft seine Münze in ein Tier zu verwandeln. Der Panzer war zwar nicht grün oder gräulich wie der einer echten Schildkröte, sondern hatte noch immer eine goldene Farbe, aber man konnte die Konturen und Muster auf dem Panzer erkennen. Ron machte sich auch erstaunlich gut, obwohl das Muster des Panzers nur leicht zu erkennen war. Hermine hatte es schon fast so weit, dass die Schildkröte nicht von einer echten zu unterscheiden war. Auch Parvati war recht gut. Nachdem Harry in Richtung Neville geblickt hatte, wurde er wieder ruhiger. Neville saß noch immer vor seiner Münze und versuchte sie umzuwandeln. Aber außer einem Kopf und einem Schwanz, war nicht viel zu sehen. McGonagall schaute ihn etwas mitleidig an und blickte danach zu Professor Elber. Dieser stand auf und ging zu Harrys erstaunen zu Neville. Als er an seinem Tisch angekommen war, machte er eine Geste die Neville andeutete, er hätte gerne seinen Zauberstab. Professor McGonagall schnürte ihre Lippen zusammen und ihre Augen weiteten sich, denn das hatte auch sie noch nicht gesehen. Neville gab Professor Elber seinen Zauberstab. Dieser tippte die Münze einmal an und sie verwandelte sich in eine perfekte Schildkröte. Er tippte die Schildkröte erneut an und an ihrer Stelle war wieder die Münze zu sehen. Dann ging er leicht in die Hocke, flüsterte Neville etwas ins Ohr und gab ihm seinen Zauberstab zurück. Neville nickte nur kurz und versuchte es noch einmal. Jetzt kamen wieder Funken aus Nevilles Zauberstab und die Münze begann sich zu verwandeln. Dieses Mal jedoch so weit, wie Harry bei seinem dritten Versuch war. Man konnte die Schildkröte erkennen, jedoch war sie vollkommen goldfarben und das Prägemuster der Münze war über ihren ganzen Körperteilen zu erkennen. Neville atmete sichtbar erleichtert auf, und Professor Elber und Professor McGonagall grinsten. Am Ende der Stunde waren fast alle so weit, dass sie ihre Münzen perfekt umwandeln konnten.

Nach der Stunde ging Harry noch zu Professor McGonagall, um sie wegen einer Berechtigung für die abgesperrte Sektion der Bibliothek zu fragen.

\enquote{Professor, kann ich sie was fragen?}

\enquote{Natürlich Mister Potter}, antwortete sie. Sie waren bereits alleine im Zimmer, als Harry fragte.

\enquote{Ich hätte gerne eine Berechtigung für die abgesperrte Sektion. Es geht um ein Buch über Dementoren, Madame Pince hat mich kurz lesen lassen\abs}

\enquote{Auf keinen Fall}, gab Professor McGonagall zurück und räumte ihre Sachen zusammen.

\enquote{Aber Professor}, unterbracht sie Harry.

\enquote{Nein Potter}, antwortete sie.

\trenn

Nachdem der restliche Tag einigermaßen ruhig verlief und Neville anscheinend jetzt bei vielen Fächern besser geworden zu sein schien, kam der Abend. Harry las noch einmal seinen Aufsatz durch, verbesserte noch ein paar Fehler und schrieb ihn noch einmal in Reinform ab. Dann ging er zu Bett. Der nächste Tag versprach nichts Gutes, da er kurz nach dem Frühstück Snape hatte. Wie die anderen bezog auch er seinen Platz im Kerker. Snape kam herein und sammelte die Aufsätze ein. Er tippte mit seinem Zauberstab auf die Tafel und die Zutaten für den nächsten Zaubertrank erschienen. Harry schrieb sie sich ab und fing an seinen Trank zu brauen. Währenddessen korrigierte Snape diverse Hausaufgaben. Immer wieder schaute er auf, suchte nach irgendjemanden, blickte zurück und strich ein paar Zeilen durch. Harrys Trank brodelte so vor sich hin, als er wieder aufblickte und Snapes Blick auffing.

Er fixierte ihn kurz und Snape blickte wieder auf seine Hausaufgaben zurück. \gedanke{Oh!}, dachte Harry. Das ist wohl meiner, aber außer ein paar scheinbar kurzen Bemerkungen und der Benotung passierte nichts. Harry befürchtete schon wieder ein D für Durchgefallen zu bekommen. Am Ende der Stunde füllte Harry seinen Trank in ein Glas ab, hängte einen Zettel mit seinem Namen dran, stellte es wie die anderen auf Snapes Tisch und nahm seine Hausaufgaben mit. \gedanke{Snape war heute einigermaßen ruhig}, dachte Harry. Als er den Kerker verlassen hatte, schaute er sich seine Hausaufgaben an. Er konnte nicht glauben, was er da las. Ein A - zum ersten Mal ein A, das hieß er hatte bei Snape eine Hausaufgabe bestanden, sie war annehmbar. Er schaute unten auf die Kommentare von Snape, die da lauteten. \accentuate{Es scheint, als ob sie doch lernfähig sind Potter. Zwar knapp aber doch noch ein A. Snape.}

Der Rest des Tages, dachte Harry, schien gerettet zu sein, denn die Stunden bei Professor Flitwick waren immer angenehm. Harrys Gedanken waren schon bei morgen Abend, wenn sie zum ersten mal Professor Elber hatten. Er wusste nichts über ihn und auch Hermine konnte nichts dazu sagen. Als er Hagrid fragte, meinte der nur: \enquote{Keine Ahnung, Professor Dumbledore hat ihn irgen'wo aufgetrieb'n. Ich hab' ihn sonst noch nie g'seh'n.} Er beschloss also zu warten bis es so weit war, da anscheinend niemand etwas über diesen ominösen Professor wusste.

Mittwochmorgens war mal wieder Wahrsagen bei Firenze, der sich seit diesem Jahr die Stunden mit Professor Trelawney zu teilen schien. Das waren jedes Mal eigenartige Stunden, mittwochs bei Firenze, in einem Zimmer, das wie ein Wald aussah, und freitags bei Professor Trelawney, bei der es immer nach Weihrauch und Räucherstäbchen roch und von der jeder genau wusste, dass sie nicht hellsehen konnte. Der Vormittag verlief genauso ruhig wie am Tag zuvor die Stunden bei Professor Flitwick. Endlich läutete die Schulglocke das Ende der Stunde ein und die Schüler packten ihre Sachen, um zum Mittagessen in die Große Halle zu gehen. Am Mittagstisch sitzend fragte Ron \enquote{Und, gippt es schm ws neus}, den Mund voller Essen. Und als er herunterschluckte \enquote{über diesen Professor Elber?}

\enquote{Nein}, antwortete Hermine, \enquote{aber das wirst du nachher schon erfahren}.

\trenn

\enquote{Herzlich willkommen zu einem neuen Schuljahr. Holen Sie bitte alle ihre Feder hervor}, sagte Professor Flitwick am Anfang der Stunde. Er wartete, bis alle Schüler dies getan hatten, und meinte dann: \enquote{Also, wir beginnen heute mit dem Wutschen und Wedeln.}

\enquote{Aber Professor}, sagte Hermine, \enquote{das hatten wir doch schon im ersten Schuljahr.}

Professor Flitwick drehte sich herum und meinte: \enquote{Dann können sie mir das sicher zeigen, Miss Granger.}

Hermine schluckte kurz, vollzog dann aber die Bewegung und sagte: \zauber{Wingardium Leviosa.} Ihre Feder begann zu schweben und flog durch den Raum.

\enquote{Gut, Miss Granger. Und jetzt wieder zurück.} Hermine lies ihre Feder sinken und auf ihren Platz zurück schweben.

\enquote{Ja ja, das war ganz eindrucksvoll, Miss Granger, aber das hatten sie ja bereits in ihrem ersten Jahr. Ja, sehr eindrucksvoll.} Er drehte sich grinsend weg und als er auf seinem Podest Platz genommen hat, sagte er: \enquote{Und jetzt Miss Granger, versuchen sie es ohne Worte.}

\enquote{Aber\abs} stammelte Hermine.

\enquote{Nur zu, Miss Granger.}

Hermine sah ihn unsicher an. Dann sah sie auf ihre Feder. Sie konzentrierte sich und dachte: \spruch{Wingardium Leviosa.} Doch nichts geschah.

Professor Flitwick grinste sie an. \enquote{Verzeihung, Miss Granger.} Nun lachte er lauter. \enquote{Das konnte nicht funktionieren. Ich hatte einen kleinen Zauber über das Klassenzimmer gelegt.} Er schwang seinen Zauberstab und meinte dann erneut. \enquote{Versuchen Sie es jetzt.}

Sie konzentrierte sich abermals und dachte: \spruch{Wingardium Leviosa.} Dieses Mal jedoch klappte es. Die Feder schwebte zaghaft und zittrig nach oben. Es kostete Hermine eine Menge Kraft.

\enquote{Sehr schön. Bitte versuchen Sie es jetzt alle.}

Kurz darauf schwebten ein paar Federn zittrig in die Höhe. Der Rest hatte damit noch Probleme.

\enquote{Wir werden die nächsten paar Male noch mit der Feder üben. Und damit auch dem letzten von ihnen klar ist, was wir hier machen, wir üben ungesagte Zauber.}

Den Rest der Stunde wurden die Federn immer und immer wieder durch den Raum schweben gelassen. Professor Flitwick erklärte immer wieder, worauf man achten musste, wenn man ungesagte Zauber ausführte, und so wurden die Schüler immer sicherer.

\trenn

\enquote{Der hat doch ’nen Knall}, sagte Adrian, ein Siebtklässler, als er den Gemeinschaftsraum betrat.

Harry drehte seinen Kopf zu ihm, um zu hören, wen oder was er meinte.

\enquote{Bringt uns \accentuate{lebendiges Feuer} bei. Gleich in der ersten Stunde}, fuhr er an seinen Kumpel, zwei Klassen unter ihm, fort. \enquote{Erst hat es ja noch Spaß gemacht, aber dann\abs ganz gewöhnliches Dämonenfeuer ist das gewesen.}

Harrys Kopf fuhr weiter herum, damit er beide gut sehen konnte. Hermine ließ ihr Buch fallen und Ron verschluckte sich fast an seinem Getränk.

\enquote{Erzählt uns was vom Pferd\abs}, maulte er weiter. \enquote{\inner{Am Montag werden wir uns ansehen, wann es nützlich ist und wann man es als Waffe einsetzen kann. Sie müssen wissen, wie sie diese Art von Feuer unter Kontrolle halten können, es selbst erzeugen und gegen ihre Gegner zurückwerfen können. Ich rede hier nicht vom Angriff}, sprach er. Dann hat er so eine Flamme erzeugt und uns alle daran fühlen lassen.}

\enquote{Wie hat es sich angefühlt?}, wollte Thomas, sein Kumpel wissen.

\enquote{Wie ein kleiner Herzschlag}, sagte Adrian ruhig. Er hatte sich in der Zwischenzeit beruhigt.

\enquote{Wieso bringt er euch so etwas bei?}, fragte Hermine.

\enquote{Weil er vielleicht ein verkappter Todesser, oder ein Sympathisant von Du-weißt-schon-wem ist?}

\enquote{Dann müsste sich Dumbledore aber schwer täuschen}, warf sie ein. \enquote{Wieso bringt er euch das bei? Dämonen- oder Teufelsfeuer ist gefährlich.}

\enquote{Eben nicht}, antwortete er. \enquote{Er hielt es in seiner Hand und manipulierte es, ohne Mühe. Er zeigte uns, dass auch das gefährliche Feuer zwei Seiten haben kann. Dass Magie nicht in zwei Seiten geteilt werden kann, sondern es verschiedene, mannigfaltige Abstufungen gibt.}

Mittlerweile lauschten alle Schüler im Raum Adrian zu.

\enquote{Er hat irgendwas Unheimliches an sich}, fuhr er fort. \enquote{Mal sehen, ob mich mein Gefühl täuscht. Denn der Unterricht selber war nicht schlecht, nur das Thema hat mich irritiert.}

\trenn

Harry saß in Dumbledores Büro.

\enquote{Nun Harry, was willst du von mir wissen?}

Harry senkte kurz seinen Kopf, sah dann in Dumbledores Augen und fing an, \enquote{Professor, \gst ich habe immer, wenn ich mein Amulett anfasse} und er zeigte es ihm kurz, \enquote{diese Vision, wie ich letztes Jahr den anderen den Patronus-Zauber lehre. Sie haben aber mal zu mir gesagt, dass das sehr fortgeschrittenen Magie sei, und die meisten Erwachsenen das nicht können. Und doch haben es viele von ihnen geschafft. Was hat das zu bedeuten Professor? Ist es, weil wir noch sehr jung sind? Liegt es an mir, weil die anderen Wissen, dass es möglich ist und sie dadurch auch hoffen, es zu können, oder liegt es vielleicht daran, wie ich es ihnen gezeigt habe?}

Dumbledore stellte seine Finger gegenüber und führte sie zu seinem Mund und seiner Nase. Er schaute Harry über seine halbmondförmige Brille an und atmete tief ein. Dann nahm er seine Hände herunter und stand auf. Er lief ein paar Schritte im Raum umher und drehte sich wieder zu Harry, der ihn die ganze Zeit über mit seinem Blick verfolgte. \enquote{Davon habe ich gehört. Eine erstaunliche Leistung, die dir da gelungen ist. Lass mich versuchen es dir zu erklären \gst Warum kannst du es?}

\enquote{Weil ich es schon einmal getan habe \gst Ich erinnere mich wieder an unser Gespräch darüber.}

\enquote{Ihr habt doch gelernt, dass Magie keine feststehende Größe ist. Sie ist ein lebendiges Wesen, das uns alle umhüllt, das uns durchdringt, uns mit jedem Baum, jedem Stein, jedem Lebewesen und anderen Dingen verbindet. Darum sind auch einige Zauberer in der Lage sich mit Tieren zu unterhalten.}

Harry staunte und erinnerte sich an sein zweites Schuljahr, als er die Schlange von Justin abhielt, indem er mit ihr sprach. Jetzt erinnerte er sich auch wieder, dass er Parsel sprechen konnte.

\enquote{Du musst wissen, dass ein Patronus, wie man dir schon gesagt hatte, nur von jemandem erzeugt werden kann, der sehr intensive positive Erinnerungen hat.}

Harry dachte an seine Nächte mit Luna, schmunzelte und nickte dann.

\enquote{Ich sehe, du hast einige sehr schöne Erinnerungen. Behalte sie für dich. Sie sind umso effektiver, wenn du sie mit niemandem, oder mit so wenig wie möglich teilst.}

Harry nickte abermals und wusste nun, dass er sehr wenig über Luna und sich erzählen würde.

\enquote{Aber, und das wissen nicht viele}, Professor Dumbledore setzte sich wieder und legte seine Hände auf die Armlehnen seines Sessels, \enquote{nur derjenige, oder diejenige, die auch traurige Erinnerungen, schreckliche Erinnerungen mit sich trägt, kann einen wirklich effektiven Patronus heraufbeschwören, der hunderte von Dementoren abhält. Und das ist nur, weil diejenigen am besten Wissen, wie sich Schmerz anfühlt. Wenn man nie einen so großen Schmerz gefühlt hat, hat man keine Ahnung davon hat, was einem ein Dementor alles anhaben kann.}

Harry begann zu verstehen. \enquote{Sie meinen, weil ich mit ansehen musste wie meine Mutter und Cedric Diggory starben und andere schreckliche Erlebnisse hatte, bin ich dazu in der Lage.}

\enquote{Ja}, sprach Dumbledore \enquote{und weil du weißt, dass du es kannst. Du hast ein intuitives Verständnis für die Macht entwickelt, die einem die Magie verleiht. Und ich habe bemerkt, dass du in dieser Schule nicht der einzige bist. Andere in deinem Jahrgang beginnen langsam zu verstehen, wie die ganzen Zusammenhänge funktionieren. Es gibt viele Zauberer und Hexen, die ihre Magie anwenden, ohne sich im Klaren über die Zusammenhänge zu sein. Um ehrlich zu sein, das ist ein ziemlich großer Teil.}

\trenn

Nachdem die Glocke den Beginn der Verteidigung gegen die dunklen Künste Stunde angekündigt hatte, standen Harry, Ron und Hermine auf, um ihr übliches Klassenzimmer aufzusuchen. Dort angekommen setzten sie sich und warteten bis der Professor aus seinem Büro kam. Es läutete ein zweites Mal, um die Schüler daran zu erinnern, dass in zwei Minuten der Unterricht begann. Es war still im Raum, denn die Tür zum Büro des Professors, die über eine Treppe zu erreichen war und direkt vor der Klasse in etwa 2 Meter 50 Höhe war, blieb verschlossen. Die Glocke läutete ein drittes Mal und die Tür zum Klassenzimmer schloss sich. Alle drehten sich um, in der Hoffnung der Professor würde auftauchen, aber nichts geschah. Langsam wurde es unruhig und noch immer fehlten zwei Schüler. Als dann die Tür aufging und die fehlenden Schüler zur Tür hereinstürmten, schrien diese kurz auf, und die Tür hinter ihnen fiel wieder zu. Sie rieben sich ihre Arme und setzten sich. Jetzt ging die Tür zum Büro des Professors auf, und ein gut gelaunter Professor Elber kam herein. Er schloss hinter sich die Tür und kam die Treppe herunter.

\enquote{Ich sagte ihnen schon zu Beginn der Woche bei Professor McGonagall, dass sie zu meinem Unterricht pünktlich erscheinen werden. Aber sie brauchen keine Angst zu haben, denn wie mir berichtet wurde, haben sie es ja bereits durchgenommen, wie man sich gegen virtuelle Schmerzen wehrt}. Dann sprach er weiter. \enquote{Dieses Jahr fangen wir mit den wirklich wichtigen Dingen an. Nachdem sie erst jetzt dieses Fach haben}, machte er weiter, \enquote{konnten sie keine Bücher mehr beschaffen. Deshalb werden sie von mir welche bekommen, die sie am Ende des Schuljahres hier wieder abliefern werden. Oder sie kaufen sie.} Er zeigte auf einen Bereich an der Seite des Zimmers auf dem \accentuate{Bücherrückgabe} stand. Er schwenkte seinen Zauberstab und wie aus dem Nichts tauchte an jedem Platz ein Buch auf. Pechschwarzer Einband und in silbernen Lettern stand darauf: \accentuate{Dunkle Künste für Anfänger} und darunter \accentuate{Band 1}.

\enquote{Diese Bücher enthalten die Grundlagen derjenigen Künste, die sie lernen, zu bekämpfen im Begriff sind. Es wird nur das jeweilige Kapitel, an dem wir arbeiten, und alle früheren sichtbar sein, um zu verhindern, dass sie sich durch Ausprobieren von Flüchen höherer Kapitel verletzen.}

Hermine hob ihre Hand und fragte: \enquote{Professor, lernen wir hier wirklich die dunklen Künste?}

Die ganze Klasse hielt den Atem an.

Professor Elber blickte sie an und sagte dann: \enquote{Ja, \gst was die Theorie anbelangt vorerst schon. Wir werden diese Zaubersprüche und Flüche erst später wirklich praktisch anwenden, denn nur wenn sie wissen, was der Gegner auf sie loslässt, wenn sie es an seinen Bewegungen und den Funken seines Zauberstabes erkennen, sind sie effektiv genug, sich dagegen zu wehren. Sie haben noch eine Menge Grundlagen zu lernen.} Er wanderte dabei durch die Klasse und fuhr mit seiner Erzählung fort. \enquote{Nur die Verteidigungszauber zu kennen, kann einem Probleme verursachen. Und lassen Sie mich eines klarstellen. Es gibt keine dunkle Magie. Genauso wenig gibt es eine weiße, oder helle Magie. Magie hat keine Farbe. Sie ist neutral, farblos. Nur zu welchem Zweck ich einen Zauber einsetze bestimmt, ob ich böse bin oder nicht.}

Er lief zu seinem Schreibtisch zurück und hob ein Tuch hoch, das zwei Pflanzen verdeckte. Die eine konnte man als eine Art Tulpe bezeichnen, die andere sah dürr und mager aus, mit kleinen Kapseln an den Enden der Zweige. \enquote{Miss Granger, kommen Sie bitte her}, sagte Professor Elber mit einem leicht forschen Ton. Hermine schluckte kurz, stand dann aber auf und ging zu Professor Elber. \enquote{Ich möchte, dass sie einen Feuerstrahl auf diese Pflanze halten.} Hermine schaute ihn etwas eigenartig an, holte dann aber ihren Zauberstab aus der Innentasche ihres Umhangs und brannte die Pflanze nieder. Außer einem Häufchen Asche blieb nichts mehr übrig. Die Pflanze war tot. \enquote{Und jetzt, nachdem sie alle gesehen haben, was dieser Zauber bewirkt, wenden sie ihn bei dieser Pflanze an, Miss Granger.}

Der Professor zeigte auf die andere Pflanze. Hermine tat wie geheißen und es ergoss sich ein Feuerstrahl über die andere Pflanze. Diese hier schien widerstandsfähiger zu sein. Doch plötzlich passierte etwas Unerwartetes. Die kleinen Kapseln sprangen auf und viele kleine Samen kamen heraus. Viele fielen auf den Boden und nur wenige wieder in die Erde. Hermine stoppte ihren Feuerregen und steckte ihren Zauberstab wieder ein. Die Pflanze war nun auch zu Asche geworden, aber im Gegensatz zur Tulpen-ähnlichen, begannen aus den Samen neue kleine Pflanzen zu sprießen. Zuerst kamen ein paar kleine grüne Blätter, die immer größer wurden, dann fing in der Mitte an ein Stängel zu wachsen, der am Ende in einer Blüte endete, die schöner als alle anderen Blüten war, die sie jemals gesehen hat. \enquote{Vielen Dank Miss Granger} sagte Professor Elber und deutete Hermine auf ihren Platz.

\enquote{Was sie gerade gesehen haben}, sagte er, als Hermine wieder zu ihrem Platz lief, um sich zu setzen, \enquote{war zweimal der exakt selbe Zauberspruch. Nur, einmal hat er geschadet und einmal hat er genutzt. Wir können also nicht sagen, dass dieser oder jene Zauber böse oder schlecht bzw. gut ist. Der Zauber war immer der gleiche. Es kommt nur darauf an, wie man ihn anwendet und einsetzt}, sprach Professor Elber. \enquote{Zwar gibt es viele Zauber, die unseren moralischen Vorstellungen nach als schlecht eingestuft werden, aber das sagt nichts über die Magie selber aus. Sie ist lediglich ein Werkzeug dessen wir uns bedienen. Denken Sie zum Beispiel an einem Hammer. Damit kann man sowohl gutes als auch böses machen.}

\enquote{Lest nun das erste Kapitel in eurem Buch. Ihr müsst euch erst die Grundlagen in diesem Kapitel aneignen, bevor wir weitermachen können.} Er ging hinter seinen Tisch und setzte sich in den Stuhl, der dort stand. Er betrachtete den Tisch und sein Blick fiel auf die Schublade, die er öffnete und der er ein Buch entnahm. Er legte es auf den Tisch und schlug es auf. Nach einem kurzen Blättern trübte sich seine Miene.

\chapter{Da die DA}


\enquote{Sagt mal, habt ihr in diesem Fach überhaupt etwas getan?}, stellte Professor Elber die Frage an die Klasse. \enquote{Im ersten Jahr habt ihr nur ca. 70\% dessen erreicht, was im Lehrplan steht. Im zweiten Jahr habt ihr fast nichts gemacht. Das Klassenbuch hier liest sich so, als ob ihr keinerlei Unterstützung von eurer Lehrkraft bekommen habt.} Sein Blick streifte durch die Klasse \gst betretenes Schweigen \gst und er fuhr fort. \enquote{Wenigstens habt ihr im dritten Jahr was gelernt. Da habt ihr euer Soll sogar überboten. Im vierten Jahr habt ihr einen schönen falschen Professor gehabt, aber wenigstens hat er euch einen Einblick in \accentuate{die richtigen Bereiche} gegeben. Aber den Vogel hat ja wohl eure Lehrerin im fünften Schuljahr abgeschossen. \enquote{Theoretische Verteidigung} \gst blabla Mumpitz. Draußen geistert Voldemort herum und sie macht nur Theorie. Pah.} Wieder streifte sein Blick durch die Klasse und einige Schüler zuckten, als er Voldemorts Namen nannte.

\enquote{Eines kann ich euch jetzt schon mal sagen. Ich bin stinksauer. Da opfert man seine Zeit, tut Dumbledore einen Gefallen, gibt sein ruhiges Leben auf und nimmt sich Zeit, junge Leute in die Geheimnisse der gehobenen Magie einzuweihen, und dann stellt man fest, dass euch die Grundlagen fehlen. Ich glaube, ich muss mein ganzes Konzept auf den Kopf stellen. Ich muss mich nachher mal mit eurem Direktor unterhalten und ihm in den Hintern\abs Wir werden wohl mehr Praxis machen als Theorie. Die ersten Wochen werdet ihr wohl viel in eurer Freizeit lesen müssen, denn ihr habt einiges nachzuholen. Ich schätze mal so 14 Tage lang mindestens eine Stunde jeden Tag.} Betrübt blickte er auf sein Klassenbuch, nahm die Feder aus dem bereitstehenden Tintenfass und machte einen Eintrag. Nachdem er fertig war, schlug er das Buch zu und steckte die Feder in das Tintenfass zurück. Er stand auf und lief Richtung Tür. \enquote{Ihr könnt hier weiterlesen oder in euren Gemeinschaftsräumen. Ich gehe mal zu Dumbledore.} Mit diesen Worten verließ er das Klassenzimmer und ließ die Schüler mit einem Staunen im Gesicht zurück.

Es war spät, als Harry im Gemeinschaftsraum saß und an seinen Hausaufgaben arbeitete. Eine Viertklässlerin war die letzte, die sich von ihm verabschiedete. \enquote{Gute Nacht Harry}, sagte sie.

\enquote{Gute Nacht Emilie}, gab Harry zurück.

Als er mit seinen Hausaufgaben fertig war, nahm er sich noch das Fachbuch und setzte sich auf eine Couch. Er wollte noch etwas nachlesen, bevor er ins Bett ging. Doch nach wenigen Minuten schlug er sein Buch wieder zu, nahm seine Sachen mit und wollte gerade ins Bett gehen, als er Schritt von den Mädchenschlafsälen hörte. Gespannt wartete er, wer so spät abends noch herunterkam. Es war Tamara, Malfoys kleine Schwester.

\enquote{Harry?}, fragte sie. \enquote{Kann ich kurz mit dir reden?}

Harry legte seine Sachen wieder auf dem Tisch ab, setzte sich in ein Sofa, schlug mit der Hand neben ihm ein paar Mal auf und sagte dann: \enquote{Was hast du auf dem Herzen?} Fast hätte er noch \accentuate{Kleine} gesagt.

Sie setzte sich neben ihn, schlüpfte aus ihren Pantoffeln und drehte sich so, dass ihr Kopf auf einer Armlehne lag und ihre Beine angewinkelt nach oben zeigen. Ihre Füße berührten Harry nicht. Dann saß sie schweigend da.

\enquote{Draco hat mir erzählt\abs}, dann stockte sie wieder eine Weile. \enquote{Draco hat mir erzählt, dass du es ihm nicht gerade leicht machst.} Sie erzählte es in einer Ruhe, dass Harry nicht wagte, sich darüber zu beschweren. Also hörte er ihr weiter zu. \enquote{Trotzdem bewundert er dich. Irgendwie.}

Jetzt verschlug es Harry endgültig die Sprache. \gedanke{Malfoy bewundert mich?}

\enquote{Er hat es mir nie direkt gesagt, aber\abs} Wieder pausierte sie kurz. \enquote{Es klang immer wieder durch, wenn er über dich sprach. Ich glaube, er hätte nie den Mut dazu gehabt, das zu tun, was du getan hast.}

Dann stand sie auf und ging nach oben. Ohne sich umzudrehen, sagte sie noch: \enquote{Gute Nacht, Harry.}

Harry blieb verwirrt eine Zeit lang sitzen. \gedanke{Warum hat sie mir das erzählt? Ist sie Schlaf gewandelt und kann sich morgen nicht mehr daran erinnern?} Dann stand er auf, nahm seine Sachen vom Tisch und ging nach oben. Nachdenklich schlief er ein.

Am nächsten Morgen hatte er wieder Vgddk.

Nach der Stunde sprach ihn Dean an. \enquote{Du Harry, machst du wieder die DA? Nachdem Umbridge nicht mehr da ist, können wir sie doch wieder aufleben lassen. Zudem können wir dann gleich für den Unterricht hier üben.}

\enquote{Ich weiß nicht, Dean. Das muss ich mir erst einmal durch den Kopf gehen lassen. Vielleicht läuft das unserem Unterricht hier auch entgegen. Es wäre also kontraproduktiv.}

\enquote{Überlege aber nicht zu lange. Wir warten auf deine Antwort.}

\enquote{Wer ist wir?}, fragte Harry skeptisch nach.

\enquote{Gut zwei Drittel der alten Truppe. Die anderen konnte ich noch nicht fragen.}

\enquote{Wie, du hast alle gefragt?}

\enquote{Nicht alle. Aber ja, ich wollte erst mal nachfragen, ob noch Bedarf besteht, bevor ich dich damit überfalle.}

Harry sah zu seinem Lehrer, der interessiert dem Gespräch folgte.

\enquote{Mal sehen}, sagte Harry, als er Dean wieder ansah.

\enquote{Haben Sie noch eine Frage, Harry?}, fragte ihn Professor Elber, als schon alle gegangen waren und Harry alleine im Raum stand und seinen Gedanken hinterher hing.

\enquote{Wie?}

\enquote{Ob sie noch eine Frage haben, wollte ich wissen!}

\enquote{Nein. Oder doch. Können wir auch Lerngruppen machen, um den Stoff aufzuholen und auch praktisch üben?}

\enquote{Sehr gerne. \gst Das, was ich gerade über eine DA mitbekommen habe\abs Deswegen hat mich Dumbledore schon vorgewarnt. Er konnte mir nicht viel sagen, außer, dass sie einer kleinen Gruppe Zaubersprüche aus diesem Fach beigebracht hatten.} Harry nickte. \enquote{Falls sie Hilfe zu einem Zauber brauchen, wir Lehrer stehen zur Verfügung. Suchen Sie einfach einen von uns auf und fragen nach. Ich habe nichts dagegen. Führen Sie die Gruppe nur fort.}

Harry war erleichtert und sagte: \enquote{Danke.}

\enquote{Wofür? Für die Genehmigung, dass sie üben dürfen? Das hätten sie auch ohne mich gemacht.} Bevor Harry den Raum verließ, meinte er noch: \enquote{Wir zwei werden demnächst anfangen zu üben. Ich werde ihnen Sachen beibringen, die sie in der Schule nicht lernen werde, die ihnen aber hilfreich sein werden. Unter anderem Tarnen und Aufspüren. Dann Angriffs- und Verteidigungszauber. Dann kommen noch Element- und Gedankenzauber dazu. Wenn sie diese Gruppe leiten werden, dann werde ich auch Sachen einbauen, die sie den anderen beibringen können. Ansonsten gilt: \inner{Kein Wort zu niemandem darüber, was sie lernen.} Wir werden diesen oder nächsten Samstag anfangen.}

Harry nickte. Er war froh, etwas zu haben, um die DA weiterhin ausbilden zu können.

\trenn

\enquote{Willkommen zurück}, begrüßte sie  Professor Sprout im Gewächshaus. \enquote{Heute werden wir mit einer neuen Pflanze arbeiten und dabei etwas über sie lernen. Ich habe neue Pflanzen aus Japan erhalten, über die noch wenig bekannt ist. Wir werden uns die nächsten Wochen mit diesen Pflanzen auseinandersetzen. Auf den Tischen vor ihnen sehen sie die Exemplare. Sie werden sich zu viert um die Pflanzen kümmern. Die Liste hängt am Eingang aus.}

Die Schüler gingen zu den Listen und schauten, auf welchen Platz sie gehen mussten. Harry wurde mit Susan Bones und Alice McBrite aus Hufflepuff, sowie mit Neville zusammengebracht. Alice wollte gerade die Pflanze anfassen, um sie zu untersuchen, doch Neville hielt ihren Arm zurück.

\enquote{Langsam Alice. Wir wissen nichts über diese Pflanze. Es könnte eine giftige sein, oder eine fleischfressende, die sich getarnt hat.} Dann zog er seine Drachenhandschuhe an und griff nach der Pflanze. Als er sie berührte, fing sie an ein giftiges Sekret über die Handschuhe laufen zu lassen und kleine Tentakel versuchten den Handschuh zu packen. Dies verhinderte Neville aber, indem er seine Hand rechtzeitig zurückzog. \enquote{Seht ihr?}

\enquote{Und warum hast du keinen Gesichtsschutz?}, fragte Susan nach.

\enquote{Wie kommst du da drauf? Ich lege generell einen Zauber über mein Gesicht, bevor ich eines der Gewächshäuser betrete. Nur für den Fall der Fälle.}

Die beiden Hufflepuffs und Harry waren erstaunt darüber, wie professionell Neville mit der neuen Pflanze umging. Er führte praktisch die kleine Gruppe an und sagte ihnen, wie man vorgehen solle. Harry war richtig beeindruckt. Nach einer viertel Stunde kam Professor Sprout zu ihnen und stand nun hinter Neville. Keiner der vier bemerkte sie, da sie in ihre Arbeit vertieft waren. Neville war in seinem Element.

\enquote{Ich muss ihnen gratulieren, Mister Longbottom. Beeindruckend, wie sie ihren Mitschülern das wichtigste über die Vorgehensweise bei unbekannten Pflanzen beibringen. Sie haben sicher gesehen, dass es manch anderer Gruppe nicht gut erging. Sie mussten zu Poppy\abs Verzeihung, Madame Pomfrey. Da haben sie Glück gehabt, dass sie mit Mister Longbottom zusammen gekommen sind.}

Neville wurde rot.

\enquote{Sie haben uns doch mit ihm in eine Gruppe gesteckt}, meinte Susan.

\enquote{Nein, nein. Die Liste ist per Zufall entstanden. Ich habe eine Namensliste genommen und dann einen Zauber gesprochen, der die Nummern zufällig verteilt hat. Ich bin unschuldig.} Dann ging sie zu einer anderen Gruppe.

Harry sah ihr hinterher. Sie ging zu einer Gruppe, in der Hermine stand. Einer aus ihrer Gruppe verdrehte die Augen. Dann sah er zu Harry, der ihn schmunzelnd ansah. Er formte das Wort \accentuate{Redeschwall}, worauf sein gegenüber stumm nickte und sich wieder der Pflanze widmete.

Am Ende der Stunde auf dem Rückweg zum Schloss hatte er einen Einfall. Er nahm seine verzauberte Münze aus der Tasche und setzte ein neues Datum. Es war der kommende Samstag. \gedanke{Mal sehen, wie viele kommen werden}, dachte er noch, als er die Münze wieder in seine Tasche schob.

Gleich nach dem Essen zogen Ron und Hermine Harry beiseite.

\enquote{Was hast du vor, Harry?}, fragte Ron. \enquote{Willst du die DA wieder aufleben lassen?}

\enquote{Wie kommst du\abs} Ron hielt ihm eine Münze hin. \enquote{Ja. Dean hat mich gefragt. Ich möchte vorfühlen, wer kommt, ob Bedarf besteht und wie der aktuelle Stand ist.}

\enquote{Glaubst du, das brauchen wir jetzt noch?}, wollte Hermine wissen.

\enquote{Laut Dean schon. Er hat schon einige gefragt und ist selber schon gefragt worden.}

\enquote{Und was willst du ihnen dann beibringen?}

Harry überlegte kurz. \enquote{Zuerst machen wir den Patronus nochmal. Dann eine Wiederholung der bisherigen Sachen. Die Übungen für \VgddK und dann noch einige andere Sachen. Da lasst euch überraschen. \gst Kommt jetzt, gehen wir in unseren Gemeinschaftsraum. Ich möchte diesen lästigen Aufsatz endlich hinter mich bekommen.}

Ron und Hermine schauten sich an, nickten kurz und nahmen Harry zwischen sich mit.

Ein paar Tage später war es so weit. Erstaunlicherweise hatte sich der Raum der Wünsche selbst repariert. Oder waren es die Elfen gewesen? Keiner wusste es so genau. Harry wartete mit Ron und Hermine bereits, als die ersten Schüler eintrafen. Zehn Minuten später waren alle der alten DA da. Selbst Marietta Edgecombe konnte er ausmachen. Sie schien irgendwie abseits zu stehen.

\enquote{Es freut mich, dass alle kommen konnten. Ich habe mir gedacht, wir machen da weiter, wo wir das letzte Mal aufgehört haben. Wir üben erst einmal den Patronus.} Harry erschuf seinen und der Hirsch lief langsam durch den Raum, während er weiter erzählte. \enquote{Denkt an ein sehr schönes und intensives Erlebnis. Dann sprecht ihr: \spruch{Expecto Patronum} und ruft euren Beschützer herbei. Ob als Figur oder als Nebelschwade. Hauptsache er schützt euch.}

Während die Gruppen zu üben begannen, ging Harry auf Marietta zu, nahm sie zur Seite und setzte sich neben sie auf den Boden.

\enquote{Magst du mir sagen, was dich dazu gebracht hat?}, fragte er. Sanft aber traurig. Seine Wut und sein Zorn waren in der langen Zeit seit ihrem Verrat verraucht.

Noch immer hatte sie ihre Haare über ihre Stirn gezogen um das Wort \accentuate{Petze}, das sie zierte zu verbergen. \enquote{Damals, als Umbridge noch hier war\abs}, dabei schüttelte sie sich leicht, \enquote{\aabs Meine Mutter sagte damals zu mir, dass ich aufpassen solle, mit wem ich mich zeigen soll. Da sah ich es als meine Pflicht an\abs} Erste Tränen liefen ihre Wangen hinab. Harry holte ein Taschentuch heraus und gab es ihr. \enquote{Als meine Mum davon erfahren hat \gst Umbridge hat es ihr am selben Tag noch gesagt, was ich für eine vorbildliche Tochter wäre \gst gab es nach meiner Heimkehr erst einmal eine kräftige Standpauke. Dann hat sie mich in Ruhe gelassen, damit ich nachdenken konnte. \gst Erst langsam, dann immer schneller wurde mir klar, dass mich meine Mutter gewarnt hatte, offen zu dir zu stehen. Ich sollte dich warnen. Ich hatte sie komplett falsch verstanden.}

Harry legte einen Arm um sie und zog sie ein Stück zu sich heran. Dann strich er ihr die Haare aus der Stirn. \enquote{So siehst du schon viel besser aus.}

\enquote{Bitte nicht. Ich will nicht, dass man das sieht.}

\enquote{Du willst nicht, dass man deine glatte Stirn sieht?}

\enquote{Glatte Stirn? Wo lebst du denn\abs} Sie stutzte, da sie über ihre Stirn fuhr, um ihre Haare wieder runterzuziehen. \enquote{Sie ist glatt!}, staunte sie. \enquote{Was hast du gemacht?}

\enquote{Ich? Gar nichts. Wahrscheinlich wurde der Zauber gebrochen.}

\enquote{Aber durch was?}

\enquote{Dein Eingeständnis, deine Reue und meine Bereitschaft, dir zu Verzeihen. \gst Weißt du, Dumbledore hat mir mal erzählt, dass es Arten von Magie gibt, die durch Taten oder Gefühle ausgelöst werden. Dazu braucht es keinen Zauberspruch.}

Mittlerweile hatten die anderen aufgehört zu üben und hörten ihnen zu. Jeder bekam mit, wie sich Marietta an die Stirn fasste. Jetzt gehörte sie wohl für alle wieder ganz zur Gruppe und wurde nicht nur geduldet. Herzlich wurde sie in der Gruppe aufgenommen. Für heute war die Stunde beendet. Harry übte noch eine viertel Stunde mit ihr alleine.

Die nächsten zwei Wochen vergingen wie im Flug und es war der Abend, an dem er sich wieder mit Luna treffen wollte. Harry saß gerade in der Großen Halle gegenüber Ron.

\enquote{Eulen-Pudding}, hörte Harry nur und begann daraufhin Ron danach zu fragen.

\enquote{Was ist das?}, fragte Harry.

\enquote{Was?}, fragte Ron, \enquote{Eulen-Pudding?}

\enquote{Ja genau.}

\enquote{Nun, das ist eine Art Pudding, der, wenn du ihn isst, deinen Hals in den einer Eule verwandelt. Das heißt du kannst deinen Hals um fast 360 Grad drehen, ohne dich umzudrehen.}

\enquote{Du hättest mal McGonagalls Gesicht sehen sollen, als sie uns von hinten antippte und wir nur unseren Kopf drehten}, warf Seamus ein. Harry prustete fast sein Essen heraus, das er gerade wieder in sich hinein stopfte, und konnte sich nur schwer beherrschen.

\enquote{Das heißt ihr hab ihn an euch ausprobiert und an McGonagall getestet?}, fragte Harry immer noch entsetzt und dabei leicht grinsend.

\enquote{Ja}, meinte Seamus. \enquote{Hat Spaß gemacht. Aber was heißt getestet. Fred und George haben die gemacht. Das war ein Probeexemplar.}

Harry schluckte seinen Bissen herunter, schaute auf seine Uhr und verließ die Große Halle. \enquote{Wo gehst du hin, Harry?}, fragte Ron.

\enquote{Noch was in der Bibliothek nachschlagen}, log Harry, genau wissend, dass er sich gleich mit Luna treffen wollte. Er verließ die Große Halle und machte sich auf den Weg zum Gemeinschaftsraum der Paare. Auf dem Weg dorthin traf Harry auf Peeves. Jener kleine Taugenichts-Geist, der nur Unsinn im Kopf hatte. Er bemerkte Harry zwar, grinste ihn frech an, ließ ihn aber in Ruhe. \gedanke{Scheinbar hat er etwas Größeres vor}, dachte Harry und lief weiter in Richtung Westflügel.

\enquote{Was haben sie da in der Hand, Frederick?}, fragte Professor Flitwick, der mit Professor Elber und Professor Sprout unterwegs war.

\enquote{Einen Plan des Schlosses}, sagte Professor Elber.

Harry horchte auf und folgte ihnen in sicherer Entfernung.

\enquote{Wie meinen sie das?}, fragte Professor Sprout.

\enquote{Das ist ein Bauplan. Alle Räume, Gänge und Innenhöfe, sowie Türme und andere Anbauten sind darin verzeichnet. Ich habe darauf einen Hof gesehen. Er wird als Rosenhof bezeichnet. Jetzt möchte ich ihn mir mal ansehen, da an der Stelle wo der Durchgang sein sollte, ein Teppich hängt.}

\enquote{Und warum sollen wir dann mitkommen?}, fragte Professor Flitwick.

\enquote{Weil ich sie vermutlich brauchen werde. Die Pflanzen im Garten werden überwuchert sein. Deswegen sie, Pomona. Und vermutlich sind sie auch aggressiv. Es sind laut diversen Aufzeichnungen \accentuate{Rosa clavicula aggressiva}\footnote{Aggressive Rankrosen}. Deswegen auch sie, Filius.}

\enquote{Rechnen Sie mit Problemen, Frederick?}, fragte der kleine Zauberkünstler.

\enquote{Könnte sein.}

Sie kamen vor der besagten Stelle zum Stehen. Harry stand um die Ecke etwa zehn Meter entfernt und beobachtete die Gruppe. Professor Elber zog seinen Zauberstab, sah noch einmal auf dem Plan nach und tippt dann den Teppich an verschiedenen Stellen an. Der Teppich rollte sich nach oben und Professor Elber stutzte. Er sah erneut auf die Karte und tippte danach auf einen Stein neben der hinter dem Teppich erscheinenden Tür. Der Teppich rollte sich wieder ab und Professor Elber tippte den Teppich an anderer Stelle an und er verschwand. Er klappte die Karte zusammen und steckte sie ein.

Er trat zur Seite und deutete seiner Begleitung an, es ebenfalls zu tun. Die beiden drückten sich ebenso wie Professor Elber an die Wand und dieser tippte mit seinem Zauberstab die Tür an. Diese verschwand und sofort fielen dornige Ranken auf den Gang und schlichen am Boden entlang. Auswüchse, aussehend wie eine Venus-Fliegenfalle suchten Opfer und kamen den drei Zauberern immer näher.

\enquote{Deswegen habe ich sie mitgenommen}, sagte Professor Elber.

Professor Sprout schwang ihren Zauberstab und versuchte die Ranken zurückzudrängen. Zu dritt kamen sie langsam voran und drängten die Pflanzen durch die Tür.

\enquote{Wir müssen sie knapp über dem Boden kappen und danach entwurzeln}, sagte Professor Elber.

\enquote{Nein}, antwortete Professor Sprout. \enquote{Es reicht sie zu kappen. Ich habe einen Zauber, der sie vor dem Nachwachsen schützt. Diese Pflanzen sind zu wertvoll, um sie zu verlieren.}

Die beiden Zauberer nickten und verschwanden nach Professor Sprout im Hof.

Harry trat hinter der Ecke hervor und schaute nach einigen Metern neugierig in den Hof. Die letzten Sträucher wurden gerade gekappt und danach begann die Verbrennung der riesigen Mengen an Schnittgut.

Harry ging weiter Richtung Westflügel, während die drei Professoren sich nach getaner Arbeit auf eine Bank setzten, die Frau in die Mitte, und sich begannen zu unterhalten.

\enquote{Woher haben sie den Plan eigentlich?}, fragte Professor Flitwick neugierig.

\enquote{Der ist schon lange in unserer Familie weitergegeben worden.} Wie vom Donner gerührt sahen ihn die beiden an. \enquote{Ich kann ihnen nicht sagen, woher er stammt, oder wie alt er ist. Als ich von Dumbledore eingestellt wurde, habe ich ihn mitgebracht. So konnte ich mich zumindest auf die Örtlichkeit anpassen.}

Oben angekommen lief Harry Richtung Porträt, blieb vor ihm stehen und sagte: \zauber{Aqua Neros!} Doch nichts geschah. Harry versucht es abermals. \enquote{Aqua Neros!}, doch noch immer geschah nichts. Plötzlich hörte er Schritte. Er drehte sich um und sah, dass Luna auf ihn zukam.

\enquote{Hallo Harry}, sagte sie mit dem gleichen, verträumten Gesichtsausdruck wie immer. \enquote{Was ist los?}

\enquote{Ich komme nicht rein}, sagte Harry.

\zauber{Aqua Neros!}, sagte Luna, als sie sich dem Porträt zuwandte. Doch es öffnete sich nicht. Bange Sekunden verstrichen, als Luna plötzlich sagte, \enquote{Ich glaube, wir sollten es gemeinsam versuchen. Das ist doch der Gemeinschaftsraum der Paare.}

Harry nickte und fast zeitgleich sagten beide: \zauber{Aqua Neros!}

Das Porträt öffnete sich und beide gingen hinein. Da es noch etwas früh für das zu Bett gehen war, schaute sich Harry genauer um. Er nahm in einem, wie Luna sagte, Schaukelstuhl der Ravenclaws Platz und betrachtete die Decke. Dort hingen die vier Hausfahnen von Gryffindor, Ravenclaw, Hufflepuff und Slytherin. Ihm kam es vor, als ob die Decke in diesem Raum etwas höher war. Er ließ seinen Blick schweifen, und stellte fest, dass der Rauchabzug des Kamins ihm gegenüber mit dem Hauswappen von Hogwarts besetzt war. An beiden hervorstehenden Säulen, die vom Rauchabzug bis an den Boden gingen, waren die Wappentiere der vier Häuser in Stein gemeißelt. Harry überlegte, warum hier alle vier Häuser vertreten waren, wo doch nur die beiden Gründer dieser Räumlichkeiten darin unterwegs gewesen waren. Vielleicht lag es daran, dass sie vorhatten, nachdem sie gegangen waren anderen Pärchen hiervon zu erzählen, aber keine Zeit mehr dazu hatten oder es aus anderen Gründen nicht mehr konnten.

Der ganze Raum strahle eine Gemütlichkeit aus, die Harry gefiel. Also schloss er die Augen und schaukelte ein wenig. Währenddessen schaute sich Luna an der Wand hinter ihm um und entdeckte hinter einem Vorhang das Ende eines Regals. Sie schob den Vorhang beiseite und fand ein Paar Bücher. Nur eines davon hatte auf dem Rücken etwas stehen. Auf dem smaragd-roten Buchrücken war in schwarzen Lettern zu lesen: \enquote{Geschichte des Gemeinschaftsraumes der Paare von 1875\agst}

\enquote{Schau mal, Harry}, sagte Luna und zog das Buch heraus. Harry öffnete die Augen und drehte sich zu Luna um. Sie zeigte ihm das Buch, setzte sich in einen Stuhl neben ihm und blätterte darin. Es waren kaum Seiten beschriftet, aber als Harry den letzten Eintrag sah, versteinerte seine Mimik. \accentuate{September, 1996\\ Harry Potter, Luna Lovegood haben hier genächtigt.}

\enquote{Das scheint eine Art Tagebuch über die Aufenthalte hier zu sein}, sagte Luna. Sie schlug das Buch zu und ging an den Tisch, an dem sie das letzte Mal Schach gespielt hatten. Sie winkte ihn herüber und zauberte die Figuren wieder her. Eigentlich hatte Harry keine Lust zu spielen, aber wenn er Ron mal wieder schlagen wollte, brauchte er ein wenig Training. Also stand er auf und lief zu Luna, um mit ihr zu spielen.

Nach einigen viel zu kurzen Runden, in denen Luna immer wieder gewonnen hatte, häufiger als er, machten sie sich auf den Weg zu den Schlafräumen, denn hier gab es mehr als nur einen. Er zog seinen Zauberstab aus der Innentasche seiner Robe und ließ die Figuren verschwinden. Anscheinend hatten die beiden Erbauer vorausgedacht und hier so manche Zeit verbracht. Harry wunderte sich immer noch, wie die beiden an die ganzen Sachen hier gekommen waren. Er machte sich auf den Weg in den Schlafraum wo Luna ihn schon erwartete. Sie war bereits umgezogen und unter die Decke geschlüpft. Harry zog sich ebenfalls um, d.h. er zog sich bis auf seine Unterhose aus und schlüpfte ebenfalls ins Bett. Beide drehten ihre Gesichter einander zu und schauten sich an. Da lag sie wieder, mit ihren weißen Haaren, ihren blauen Augen und einem Gesichtsausdruck, der jedem sagte, sie sei nicht bei der Sache, wenn er sie anschaute. Aber Luna hatte einen sehr scharfen Verstand, wie Harry letztes Jahr schon bemerkt hatte. Sie hatte kaum Freunde, weil alle sie für \gst \accentuate{wie hatte sie sich ausgedrückt} \gst speziell hielten. Luna griff nach Harrys Hand und zog sie etwas zu sich. Da es im Zimmer warm war, hatten beide die Decke nur bis zu den Hüften hochgezogen und die Hände einander haltend über dem Stoff liegen. Langsam schloss Luna die Augen und Harry wünschte ihr eine gute Nacht. Luna lächelte und bald schliefen sie ein und träumten. Unbemerkt für beide begann sich in der Nacht die Decke zu bewegen. Sie bewegte sich erst Richtung Beine, um kurz, nachdem die immer noch gefassten Hände auf das Leinentuch gefallen waren, die Richtung zu ändern. Die Decke begann an beiden entlang hoch zu wandern, um sie vor dem Auskühlen zu schützen.

Am nächsten Morgen lagen beide nebeneinander auf der Seite. Harrys Oberkörper schmiegte sich an Lunas Rücken und die gefassten Hände der Nacht zuvor waren geöffnet. Harry öffnete verschlafen die Augen und roch an Lunas Haar. Ihm stieg ein Duft die Nase hoch wie er ihn an ihr noch nie wahrgenommen hatte. \gedanke{Um ernst zu bleiben}, dachte Harry, \gedanke{du hast noch nie an ihrem Haar gerochen, oder an irgendeinem anderen Körperteil.} Er setzte sich etwas auf und gab ihr einen Kuss auf die Backe, sodass sie erwachte.

Sie drehte sich auf den Rücken und rieb sich die Augen. Da war er wieder, dieser eigenartige Blick in ihrem Gesicht. Nicht verträumt, nicht abwesend, sondern kristallklar. Er gab ihr einen Gute-Morgen-Kuss auf den Mund den sie erwiderte. Dieses Mal dauerte er etwas länger als beim letzten Mal, was sie aber nicht zu stören schien. Harry war wieder glücklich und nahm sich vor, nächstes Mal von ihr geweckt zu werden. Sie standen beide wieder auf und zogen sich ihre Schülerroben an.

Heute war Sonntag und Harry musste noch Hausaufgaben machen. Professor Sprout wollte etwas über Giftpilze wissen, und wie man das Gift aus ihnen herausbekommt, um einen wirksamen Heiltrank zu brauen. Beide liefen noch Händchen haltend den Gemeinschaftsraum entlang, bis sie zum Portraitloch kamen. Hinaus schauend entdeckten sie Filch, den Hausmeister, der mit einem Mob durch die Gänge patrouillierte. Als er um die Ecke gebogen war, warteten beide noch ein paar Sekunden um sich dann zurück in ihre Gemeinschaftsräume zu schleichen. Als Harry seinen Gemeinschaftsraum betrat, war noch keiner da. Also beschloss er erst einmal zu duschen. Er schlich sich langsam nach oben und kurze Zeit später lief das klare Nass an ihm herunter.

Mit einem Handtuch um die Hüften verließ er die Dusche wieder und zog sich in seinem Zimmer um. Er ging wieder nach unten und setzte sich im noch immer leeren Gemeinschaftsraum in einen Sessel mit Armlehnen. Das Feuer im Kamin war aus und Harry betrachtete die Decke. Nach ein paar Minuten anstarren sah er sich in seinem Raum um und versuchte Gemeinsamkeiten zu erkennen; zwischen diesem und dem anderen Raum. Er versuchte sich auch an den Gemeinschaftsraum der Slytherins zu erinnern, in dem er in seinem zweiten Schuljahr einmal gewesen war. Er erinnerte sich an ein paar schwarze Sessel und ein schwarzes Sofa. Ein Kamin mit Schlangen die ihn einfassten und einen Erker, der in der Mitte einen Tisch und außen eine an die Wand gebaute Bank hatte. Harry hörte wie eine Tür aufging, war aber mit seinen Gedanken so beschäftigt, dass er es gar nicht realisierte, wie sich der Raum langsam mit Leben erfüllte. Er dachte immer noch an die Nacht mit Luna, den Kuss am nächsten Morgen und versuchte sich die Gemeinschaftsräume der Ravenclaws und der Hufflepuffs vorzustellen.

Harry realisierte erst wieder wo er war, als ihn Ron mit einem Knuff gegen den Arm aus seinen Träumen riss.

\enquote{Morgen Harry}, sagte er.

\enquote{Morgen Ron.}

\enquote{Gut geschlafen?}

\enquote{Ja.}

\enquote{Komm, lass uns Frühstücken. Hermine braucht noch ein bisschen und kommt dann mit Ginny nach.}

\enquote{Ja, geht in Ordnung.}

Noch leicht verträumt folgte er Ron durch das Porträt und ging Richtung Große Halle. Als Ron Dean sah, entschuldigte er sich bei Harry und meinte. \enquote{Ich geh' mal kurz zu Dean. Den muss ich noch was fragen}, und sprang davon. Kurz darauf bog Luna um eine Ecke und lief das restliche Stück neben Harry her. Harry war noch immer in Gedanken über die Einrichtung im Gemeinschaftsraum der Paare versunken und Luna hatte ihren üblichen Ausdruck in den Augen.

Als ein Gryffindor-Fünftklässler die beiden Seite an Seite sah, konnte er sich nicht beherrschen. Er stupste Ron an, zeigte auf die beiden und sprach dann: \enquote{Das schönste Liebespaar, das Hogwarts seit mehr als hundert Jahren gesehen hatte.}

Dies sagte er so laut, dass alle, die hinter den dreien standen, sich abrupt umdrehten und zu kichern anfingen. Fast zeitgleich kam Colin, der Haus- und Hof-Fotograf der Gryffindors und der restlichen Schule, um ein Foto von den beiden zu machen. Erst jetzt kam Harry wieder zu klarem Verstand und verspürte einen Drang, seinem Mitschüler eines auszuwischen. Er zwinkerte Colin zu, der sofort wieder seine Kamera vor sein Auge hob, und drehte sich zu Luna um, die ihn nun anlächelte und ihn in den Arm nahm. Beide sahen zu Colin und lächelten ihn an. Der machte ein Foto von den beiden. Harry löste sich von Luna und ging zu Colin, um ihm etwas ins Ohr zu flüstern. Dieser nickte und verschwand, da er scheinbar mit dem Frühstück schon fertig war. Harry stellte sich jetzt zu Dean und Ron und meinte nur: \enquote{Kommt, lasst uns frühstücken} und fügte mit einem leicht schnippischen Grinsen zu seinem Mitschüler, \enquote{Klatschpresse von Hogwarts}, hinzu. Luna war schon an ihrem Platz und setze sich hin, um zu frühstücken, während Harry neben Ron in die Große Halle eintrat und auf einen freien Platz zu lief. Er und Ron setzen sich und Lavender und Sally saßen ihnen gegenüber.

\enquote{Was hast du zu Colin gesagt?}, wollte Dean wissen. \enquote{Hast ihm gedroht ihn zu verhexen, wenn er die Bilder nicht vernichtet? So wie der gerannt ist} und drehte sich grinsend Sally und Lavender zu.

\enquote{Das werdet ihr schon sehen}, antwortete Harry, in dessen Kopf sich ein Plan zu formen begann. Ein Bild hatte er vor, im Tagebuch im Gemeinschaftsraum der Paare zu verewigen, und das andere mit einem netten Kommentar leicht modifiziert im Gryffindor-Gemeinschaftsraum aufzuhängen.

Dann machte Professor Dumbledore eine Ankündigung: \enquote{Wie ich soeben mitgeteilt bekommen habe, haben wir einen weiteren Innenhof zur Verfügung. Einen sogenannten \accentuate{Rosenhof}. Es dauert noch ein paar Tage, bis er in altem Glanz erstrahlt, da er Jahrhunderte lang verschlossen war und entsprechend verwahrlost aussah. In diesem Innenhof werden viele Bänke zum Entspannen aufgestellt und viele duftende Rosen werden den Hof zieren. \gst Aber! Gehen Sie nicht zu nah ran. Diese Rosengewächse sind sehr selten und entstammen der Gattung \accentuate{Rosa clavicula aggressiva}. Diese Rosen-Art ist eine Kreuzung zwischen normalen Rosen und der Venus-Fliegenfalle. Halten sie also gebührenden Abstand. \gst Noch ein Hinweis: Spezielle Schutzzauber verhindern Schlimmeres. Sie können sich an diesen Rosen durchaus verletzen, aber es werden keine lebensbedrohlichen Verletzungen sein. Sie werden höchstens gezwickt werden oder bekommen einen Ausschlag, den sie dann bei Madame Pomfrey kurieren müssen. Gehen Sie also nicht zu nah ran.}

\trenn

Diesen Donnerstag hatte Harry wieder Pflanzenkunde bei Professor Sprout und er musste seinen Aufsatz abgeben. Sie mischten einen wirksamen Dünger zusammen, welcher die Pflanzen schneller wachsen lassen sollte. Zu ihrem Erstaunen hatte Harrys Wachstumstrank eine besonders schnell wachsende Wirkung, wofür er mit zwanzig Punkten belohnt wurde und Professor Sprout ihn nach der Stunde fragte, wie er das fertiggebracht habe. Er meinte, dass er zum Umrühren seinen Zauberstab verwendete und einen kleinen Wachstumszauber gesprochen habe. Dann schaute er noch in der Bibliothek vorbei, bevor er in dem Gemeinschaftsraum ging.

Er setzte sich neben Malfoys Schwester Tamara. \enquote{Hi Tamara. Ich hatte keine Zeit mehr dich zu fragen, was du meintest, als du sagtest: \enquote{Ich glaube, er hätte nie den Mut dazu gehabt, das zu tun, was du getan hast.} Du hast deinen Bruder gemeint.}

Tamara schaute ihn verständnislos an. \enquote{Daran kann ich mich nicht erinnern.}

\enquote{Ach komm schon, ich hatte meine Hausaufgaben gemacht und wollte noch etwas lesen, dann bin ich aufgestanden, als du in den Gemeinschaftsraum gekommen bist und dich mit mir unterhalten wolltest.}

Tamara sah ihn erstaunt an. In ihr Gesicht schienen sich einerseits Erkenntnis, andererseits aber auch Unkenntnis zu spiegeln.

\enquote{Oh}, sagte sie schließlich. \enquote{Weißt du}, und sie senkte ihren Blick, \enquote{ich Schlafwandle manchmal, Harry.} Harry zog eine Augenbraue hoch. \enquote{Und am nächsten Tag weiß ich nicht mehr, was ich danach getan habe.}

\enquote{Geh' zu Madame Pomfrey und lass dir was gegen Schlafwandeln geben.}

\enquote{Äh}, antwortete Tamara. \enquote{Wo lang?}

\enquote{Nicht wo lang. Zu Madame Pomfrey unserer Krankenhexe.}

\enquote{Oh}, entgegnete Tamara.

\enquote{Warte kurz.} Harry schrieb seinen Satz zu Ende, klappte seine Bücher zu und schob sie von der Tischkante. Sie fielen leicht ab und bewegten sich dann auf ihre Plätze zu. Dann stand er auf und sagte: \enquote{Gehen wir.}

Freudig nickte sie und packte schnell ihre Tasche. Das mit den Büchern versuchte sie auch gleich. Sie schob sie von der Tischkante, doch ihre Bücher fielen herunter. Harry konnte sich ein Schmunzeln nicht verkneifen.

Tamara schaute ihn mit Erstaunen an. \enquote{Da hast du mich aber schön hereingelegt}, antwortete sie ihm.

\enquote{Nicht wirklich}, meinte Harry. Er hob die Bücher auf und legte sie auf den Tisch. \enquote{Überleg’ mal wie ich die Bücher vom Tisch herunter geschoben habe und wie du die Bücher geschoben hast.}

Er konnte an ihrem Gesichtsausdruck erkennen, dass sie angestrengt nachdachte. Dann schüttelte sie den Kopf und meinte: \enquote{Ich sehe keinen Unterschied.}

Harry lächelte. \enquote{Pass auf.} Er schob das erste Buch mit der flachen Hand vom Tisch, worauf es herunterfiel. \enquote{Gesehen?} Tamara nickte nur. Dann nahm er ein zweites Buch und schob es mit dem Handrücken vom Tisch. Es fiel leicht ab und begann danach seinen Platz zu suchen. \enquote{Gesehen?} Tamara nickte. \enquote{Nun versuchst du es mal.}

Tamara schob jetzt auch mit dem Handrücken die Bücher vom Tisch. Sie fielen etwas weiter dem Boden entgegen, fingen sich aber schließlich und suchten sich ebenfalls den Weg zu den Regalplätzen. Dann hob sie das heruntergefallene Buch auf und fragte Harry: \enquote{Und was machen wir mit dem? Nochmal auf den Tisch legen und es herunterschubsen?}

Harry grinste sie an. \enquote{Heb’ es in die Luft, halte es an der Unterseite und lass es los.} Tamara schaute ihn leicht skeptisch an, tat aber, was Harry ihr sagte. Sie nahm das Buch an der Unterseite, hob es in die Luft und ließ es los. Langsam und sanft schwebte es an seinen Platz im Regal.

Dann sagte er: \enquote{Komm} und verließ die Bibliothek. Er glaubte noch aus den Augenwinkeln heraus einen blonden Jungen zu erkennen. Auf dem Wege zum Krankenflügel nahm Tamara seine Hand und lief neben ihm her. Er hatte das Gefühl, dass er verfolgt wurde, wollte sich aber nicht ständig umdrehen. Eine innere Stimme sagte ihm, dass es Draco Malfoy sein musste. Er lachte innerlich, obwohl er nicht wusste, wie er darauf reagieren würde.

Die Türen der Krankenstation öffneten sich und Harry betrat mit Tamara den Krankenflügel. Madame Pomfrey kam ihnen entgegen und fragte die beiden, was sie denn benötigen würden. Harry sah zu Tamara, welche seinen Blick auffing. Wortlos gab er ihr durch eine Kopfbewegung zu verstehen, dass es an ihr lag Madame Pomfrey zu sagen, was sie wollte.

\enquote{Ich brauche etwas gegen mein Schlafwandeln, Miss}, sagte sie.

Madame Pomfrey nickte und meinte: \enquote{Setzen sie sich. Das dauert höchstens zehn Minuten.} Sie verschwand in ihr Büro.

\enquote{Kann ich dich alleine lassen?}, fragte Harry.

Tamara musterte ihn. So, als wolle sie ihm sagen, sie sei bereits groß genug. \enquote{Ich komme zurecht. Danke Harry}, sagte sie.

Harry nickte und verließ die Krankenstation. In der Tür traf er auf Draco Malfoy.

\enquote{Malfoy.}

\enquote{Potter.}

Er hörte nur noch wie Malfoy zu seiner Schwester sagte: \enquote{Und kleine\abs äh Tamara, wie hast du dich bislang eingewöhnt?}

Im Gemeinschaftsraum angekommen, wartete Colin auf Harry. Er überreichte ihm einen Umschlag und nickte Harry zu. Der nahm ihn an sich und erblickte William, den Viertklässler, der Harry und Luna diesen peinlichen Moment beschert hatte, in der anderen Ecke des Raumes. Harry grinste William an. Es war ein fieses Grinsen. Dann suchte er sich eine freie Stelle im Gemeinschaftsraum und drückte den Umschlag samt Inhalt gegen die Wand. Er nahm seinen Zauberstab heraus und murmelte einen Zauberspruch. Der Umschlag blieb an der Wand hängen und Harry entfernte sich, seinen Zauberstab einsteckend. Kurz nachdem Harry die ersten Stufen zu den Schlafräumen hochgegangen war, rannte William an die Stelle, an der Harry den Umschlag an die Wand gepinnt hatte. In der Zwischenzeit löste dieser sich auf und Harry drehte sich um, um das Schauspiel zu bewundern. Vor dem Bild angekommen, sah William Luna und Harry Händchen haltend und atmete erleichtert auf. Dann jedoch passierte es. Luna verschwand und an ihrer Stelle tauchte Hermine auf. Nach ein paar Sekunden verschwand Harry und an seiner Stelle tauchte Ron auf. Jeweils die Hand des anderen haltend. Hermine verschwand und es tauchte Marlice auf, Williams Freundin. Seinen Blick konnte Harry nicht sehen, aber er nahm an, dass es nicht unbedingt erfreulich für ihn sein musste. Schließlich verschwand Ron und an seiner Stelle erschien William. Jetzt war Harry gespannt auf die nächste Paarung und bereitete sich schon einmal darauf vor, zu verschwinden, denn jetzt taucht an Marlices Stelle wieder Luna auf, die mit William Händchen haltend dastand. In diesem Moment kam Marlice durch das Porträt und für William unbemerkt schlich sie sich an ihn heran. Ihn eigentlich überraschen wollend, hörte er nur noch einen Schrei. Er drehte sich um und Harry verschwand in seinem Schlafraum.

Diese Standpauke wollte er wirklich nicht miterleben. \gedanke{Colin hatte gute Arbeit geleistet}, dachte Harry. \gedanke{Nicht nur mit dem Bild, sondern auch mit dem Timing, als er Marlice zum Gemeinschaftsraum lotste.} Harry hatte seine Rache und wollte erst einmal duschen. Nachdem er sich umgezogen hatte, ging er wieder in den Gemeinschaftsraum. William und Marlice waren verschwunden.

Ron grinste ihn an und Dean tätschelte ihm auf die Schulter. \enquote{Das hätte ich dir nicht zugetraut}, sagte er.

Harry grinste zurück und lud Ron zu einem Spiel Schach ein. Nach ein paar verlorenen Runden legte er seinen Kopf in die Rückenlehne des Stuhles und schloss die Augen. Langsam und sachte formte sich ein Raum vor seinen Augen. Er enthielt Elemente aus dem Gemeinschaftsraum der Paare, die Harry gesehen hatte. Jetzt war das Bild ganz klar in seinem Kopf. Er hatte das Gefühl als stünde er mitten im Raum und war vollkommen abwesend, hörte die Stimmen der anderen Ravenclaws und fand alle wieder, an die er sich erinnern konnte. Harry hatte nicht das Gefühl seine Beine zu bewegen und doch schien er zu einem Stuhl zu laufen, sich zu setzen und eine Zeitschrift in die Hand zu nehmen. Es war schwer etwas zu erkennen, denn die Zeitung hielt er scheinbar über Kopf. Plötzlich wurde er wieder wach und Ron und Hermine standen vor ihm mit einem besorgten Gesichtsausdruck.

\enquote{Geht es dir gut, Harry?}, fragte Hermine.

\enquote{Ja}, antwortete er, \enquote{ich bin nur etwas eingenickt.}

\enquote{Du hast im Schlaf geredet Harry}, fügte Ron ein.

Harrys Augen weiteten sich. \enquote{Was hab ich denn gesagt?}

\enquote{Du hast irgendein Rätsel gelöst}, sagte ihm Hermine.

Harry fiel wieder ein, wie er vom Gemeinschaftsraum der Ravenclaws geträumt hatte, da hatte auch eine Stimme laut ein Rätsel gelöst. So langsam fühlte er sich ein wenig ertappt, denn es kam ihm in den Sinn, dass die Zeitschrift der Wortklauber sein musste. Und den las nur Luna. \enquote{Hab ich es denn geschafft?}, fragte Harry mit einer unschuldig aussehenden Miene im Gesicht.

\enquote{Weiß nicht. Ich glaube nicht}, sagte Ron.

\enquote{Ich glaube, ich geh' dann mal schlafen}, antwortete Harry und machte sich erneut auf den Weg zu seinem Schlafsaal. Er musste unbedingt mit Luna darüber reden. Aber darüber konnte er auch Morgen noch nachdenken.

Mitten in der Nacht wachte Harry auf und ging aufs Klo. Nachdem er fertig war, fühlte er sich unruhig und dachte sich: \gedanke{Lese ich noch ein bisschen was, Morgen ist wieder Wahrsagen bei Trelawney und die Lektüre, die sie uns immer gibt, ist so herrlich einschläfernd.} Er schlich vorsichtig in sein Zimmer, holte seinen Morgenmantel, ging die Treppen zum Gemeinschaftsraum hinunter und setzte sich in den Sessel. Unten fiel ihm ein, dass er kein Buch mitgenommen hatte und stand wieder auf. Plötzlich fühlte er eine Präsenz. Erschrocken drehte er sich um, konnte aber niemanden entdecken. Er blieb noch ein paar Minuten stehen und das eigenartige Gefühl verlor an Intensität. Er begab sich wieder nach oben, zog seinen Morgenmantel aus und schlief bis zum nächsten Morgen durch.

Nach dem Frühstück bemerkte er, wie Luna die Große Halle verließ. Er sprang auf und ging ihr hinterher. \enquote{Hast du ein paar Minuten Zeit, Luna?}, fragte er sie.

Luna drehte sich um und sah ihn an. \enquote{Ja Harry. Gerne.}

\enquote{Laufen wir ein paar Schritte}, sagte Harry und begann den Flur entlangzulaufen; mit Luna neben sich. \enquote{Gestern hatte ich einen seltsamen Traum von eurem Gemeinschaftsraum}, sagte Harry.

Luna sah in an und fragte ihn dann: \enquote{Welcher Art?}

\enquote{Nun ja}, sagte Harry, \enquote{Ich konnte ihn genau sehen. Ich konnte die Schachtische sehen, den Kamin auf der linken Seite mit den Raben als Stützen, die Teppiche an der Wand und die Marmorstatue von Rowena Ravenclaw.}

Luna schaute ihn musternd an und fragte ihn: \enquote{Sag mal Harry, warst du schon mal in unserem Gemeinschaftsraum?}

\enquote{Nein,} sagte Harry. \enquote{Ich hatte aber nicht das Gefühl, dass ich das selbst war. Ich hatte mehr das Gefühl, dass ich durch die Augen eines anderen schaue und durch die Ohren eines anderen höre.} Luna öffnete ihren Mund, sagte aber nichts und Harry fuhr fort. \enquote{Ich lief vom Porträt weg und setzte mich in einen Stuhl neben Ravenclaws Statue aus Marmor, nahm eine Zeitschrift und las sie \gst verkehrt herum, löste ein Rätsel und dann wurde ich aufgeweckt.}

\enquote{Verrückt}, meinte Luna. \enquote{Genau das habe ich gestern Abend gemacht, nachdem ich aus der Bibliothek gekommen bin. Und du hast unseren Gemeinschaftsraum recht gut beschrieben.} Luna schaute wieder auf den Boden und sagte mit einem Zittern in ihrer Stimme: \enquote{Mir ist gestern auch was passiert}, dann sah sie ihm direkt in die Augen: \enquote{Ich hatte das Gefühl, dass ich während der Zeit beobachtet wurde, dass jemand in mir steckt und erlebt, was ich erlebe.}

Harry schluckte, \enquote{Genau das Gefühl hatte ich gestern Abend, kurz nach Mitternacht, als ich in unserem Gemeinschaftsraum saß.}

\enquote{Der mit unserem Bild an der Wand?}, fragte Luna.

Harry fiel der Mund herunter und blieb offen stehen. Er drehte sich zu Luna und fragte: \enquote{Was? Wer hat dir das erzählt?}

\enquote{Keiner}, antwortete sie. \enquote{Ich hatte es einfach gesehen. Ich habe geträumt ich säße in einem Stuhl der Gryffindors. Dann bin ich aufgestanden und habe mich umgesehen \gst im Körper eines anderen.}

Harry begann langsam zu begreifen und er zog Luna in einen leeren Klassenraum und verschloss hinter sich die Tür. \enquote{Ist dir klar, was das bedeutet, Luna?}, fragte er sie.

\enquote{Ja, wir haben durch unsere gemeinsamen Nächte in diesen Räumlichkeiten irgendeine Art Verbindung zueinander aufgebaut.}

Harrys Kinn fiel wieder herunter. Diese Antwort hatte er nicht erwartet.

Die Schulglocke läutete die erste Stunde ein und Harry verließ mit Luna das Klassenzimmer. Während der gesamten Stunde bei Professor Flitwick musste ihn Hermine mehrmals in seine Seite kneifen, da er mit seinen Gedanken immer wieder abschweifte und die Konzentration verlor. Er dachte immer wieder über das nach, was Luna ihm gesagt hatte. Über die Verbindung zwischen ihnen und über ihre seelische Verbundenheit. Er war sich nicht sicher, ob er sich wieder mit ihr dort treffen sollte, wusste aber instinktiv, dass es das richtige war. Oder er glaubte es zumindest.

\trenn

\enquote{Dieser Lehrer ist mir irgendwie unheimlich}, sagte Hermine. \enquote{Hast du das gesehen?}, fragte sie.

\enquote{Was denn?}, fragte Ron und auch Harry sah sie verständnislos an.

\enquote{Die Tiere reagieren anders auf ihn. Schaut mal}, sagte sie. Sie zeigte auf eine entfernte Stelle und auf einen Baum, in dessen Nähe gerade ein Schüler auf einem Pfad entlang ging. Vögel flogen aus dem Baum heraus und ließen sich kurz darauf wieder auf dessen Zweige und Äste nieder, als die vermeidliche Gefahr vorüber war.

\enquote{Was ist daran so besonders?}, fragte Ron, der nicht wusste, worauf Hermine hinaus wollte.

\enquote{Warte es ab und schau weiter.}

Wieder ging jemand an dem Baum vorbei und wieder flogen die Vögel.

\enquote{Was soll daran schon toll sein?}, fragte Ron schon halb genervt.

\enquote{Warte doch einfach mal ab Ron, sei nicht immer so ungeduldig.}

Entnervt gab Ron auf und schaute weiterhin auf den Baum.

\enquote{Und nun, was soll jetzt so besonders daran gewesen sein?}, fragte er, als wieder jemand am Baum vorbeiging.

\enquote{Das war er, aber die Vögel sind nicht aufgeflogen. Nichts hat sich gerührt.}

\enquote{Sie werden nicht mehr dort sein}, schloss Ron. Doch sein Argument wurde sofort widerlegt, als eine weitere Person am Baum vorbeilief und die Vögel erneut davon flatterten.

\enquote{Und woher willst du wissen, dass er es ist?}, versuchte Ron einen letzten Versuch.

\enquote{Sein Gang, sein Auftreten\abs}

Ron plusterte. \enquote{Soweit ist es schon? Erinnerst du dich noch an dein zweites Jahr?}

\enquote{Ronald Weasley, das ist keine Schwärmerei. Ich bin nicht in ihn verliebt oder so, ich bin skeptisch, was ihn betrifft. Schau doch mal, was er uns beibringt.}

Ron dachte nach.




\begin{kommentar}
Schon der Titel des Kapitels ist eine nette Anspielung auf das Morsealphabet. Und gleich ein paar Zeilen später, nachdem Elber festgestellt hat, dass den Schülern viel fehlt, will er Dumbledore in den Hintern treten. Ein schönes Bild, wenn man sich das so vorstellt. Kurz darauf spricht Elber Harry an und möchte ihn ausbilden. Er scheint zu wissen, dass Harry Voldemort gegenübertreten muss.
\end{kommentar}

\begin{kommentar}
Elber geht mit Sprout und Flitwick zu einem Raum im Schloss, dem Rosenhof. Wenn man weiß, dass er das Schloss erbaut hat, wirkt es lustig, dass er sagt, er habe auf einem Plan einen Raum gesehen, dessen Zugang durch einen Teppich verhängt wurde.
\end{kommentar}

\chapter{Kälte oder Wärme}


Nachdem Harry wieder eine Nacht mit Luna verbracht hatte, wachte er am nächsten Tag auf. Es war wieder ein Sonntag, wie auch die vielen Male zuvor. Harry schaute zu Luna hinüber, streichelte langsam ihr Haar und legte sich wieder hin, als er merkte, dass sie noch schlief. Er erinnerte sich daran, dass er sich vorgenommen hatte, sich von ihr wecken zu lassen. \gedanke{Mal warten was passiert}, dachte er sich. Er schloss die Augen und entspannte. Nach einer Weile bemerkte er wie Luna erwachte. Unruhig drehte sie sich auf den Rücken, dachte er. Und er meinte, dass er sogar das schwach kratzende Geräusch des Sandes hörte, als sie sich die Augen rieb. Sie drehte sich zu Harry und fing unter der Bettdecke an sich umzudrehen. Harry musste sich beherrschen und tat so, als ob er schlafen würde. Er fühlte, wie Luna ihr rechtes Bein auf seine linke Seite legte, nur um kurz darauf direkt über ihm zu sitzen, die Arme an seiner Seite aufgestützt. Er fühlte ihre körperliche Nähe, obwohl er nicht das Gefühl hatte sie zu berühren. Langsam spürte er ihren warmen Atem seinem Gesicht näher kommen. Er musste sich anstrengen, nicht enttäuscht auszusehen, als er bemerkte, dass ihr Atem seine Nase wärmte. \gedanke{Verdammt}, dachte Harry. \gedanke{Will sie mich nur auf die Nase küssen? Hatte sie das Interesse an einem Kuss verloren?} Erleichtert stellte er jedoch fest, dass seine Nase langsam abkühlte und ein leichter warmer Luftstrom seinen Mund umspülte. Er gab einen befriedigenden Laut von sich. Fast so, als ob er jeden Moment mit einem guten Traum in Erinnerung aufwachen würde. Das war wohl der Moment, auf den Luna gewartet hatte. Er spürte ihre Lippen auf den seinen. Stärker, intensiver, fordernder als zuvor. Er öffnete seine Augen und Luna sah ihn an. Es war der gleiche Blick wie die beiden Male zuvor, als sie aufwachten und sich küssten.

Sie ließ von seinen Lippen ab und ein fröhliches: \enquote{Guten Morgen, Harry}, kam es ihm sanft entgegen.

Sanfter als er es zuvor erlebt hatte. Ihr Mund war nur wenige Zentimeter von seinem entfernt. Ihr Blick verriet ihm, sie wollte nun auch von ihm geküsst werden.

\enquote{Guten Morgen, Luna}, sagte Harry und erwiderte ihren Kuss. Er spürte wieder diese Zufriedenheit, diese Geborgenheit in sich. Es war anders, anders als bei Cho, als sie sich geküsst hatten, anders, als ihn Hermine in einer Phase überschäumenden Glückes Ende des vierten Schuljahres auf seinen Mund geküsst hatte. Dieser Kuss war anders. Nicht schwesterlich, sondern verlangend, begehrend, aber dennoch zart und wild entschlossen. Langsam löste er sich von ihr und dachte sich: \gedanke{Lass dich fallen Luna, lass dich einfach auf mich fallen. Ich möchte deine Wärme spüren, deinen Körper, einfach alles.}

\stimme{Ja Harry}, hörte er, als er in Lunas Gesicht blickte. Er hörte ihre Stimme klar und deutlich vor sich, als hätte sie selber gesprochen. Aber ihre Lippen bewegten sich nicht.

Jetzt wurde ihm klar, dass er ihre Gedanken gehört hatte. Sein Gesichtsausdruck nahm eine leicht erstaunte Form an und Luna ließ sich langsam nieder. Er spürte ihren Körper auf sich, ihre Wärme, ihre Nähe. Sie war mit nichts außer einem dünnen Nachthemd bekleidet. Ihre Backe lag dicht neben seiner. Ihre Hände begannen sanft seinen Hals an beiden Seiten zu umspielen. Harry genoss es, als er plötzlich merkte, dass er es scheinbar zu sehr genoss.

Jetzt hörte er wieder Lunas Stimme. \stimme{Mach dir keine Sorgen, mir geht es genauso, auch ich bin erregt.} Sie streichelte weiter seinen Hals und er konnte spüren, wie ihr Körper zu zittern begann, als auch er endlich aktiv wurde und seine Hände ihren Rücken berührten.

Er fing mit langsam kreisenden Bewegungen an ihrer Hüfte an und arbeitete sich langsam, fast quälend langsam, in der Mitte ihrer Wirbelsäule zu ihrem Hals entlang hoch. Befriedigend nahm er ihre Gedanken wahr.

\begin{abAchtzehn}

\stimme{Mach schneller Harry. Nicht so langsam. Ich halt's nicht mehr aus. Berühr’ meinen Hals.}

Innerlich lachend fühlte er, wie ihre Erregung in ihr hoch stieß. Ihr Verlangen nach seinen Bewegungen immer stärker wurde. Er dachte: \gedanke{Auch wenn ich nicht vorhabe mit dir zu schlafen, es ist ein wahnsinniges Gefühl.} Er biss sich beinahe auf die Zunge, als er diese Gedanken hatte, genau wissend und doch nicht hoffend, dass sie ihn hören würde. Als er aber ihre Gedanken empfing: \stimme{Das möchte ich auch nicht, noch nicht}, wurde er wieder ruhiger.

Mittlerweile war er an ihrem Hals angekommen und streichelte ihn ganz sanft. Er genoss jede ihrer Berührungen, wie auch sie jede der seinen genoss. Jetzt fing er an, sich in ihren langen schneeweißen Haaren zu vergraben. Er dachte daran ihr Gesicht zu umfassen und sie wieder zu küssen. Als er begann ihr Gesicht hochzuheben, kam sie ihm entgegen. Sie wollte es genauso wie er und beide versanken in einem langen Kuss. Er öffnete leicht seinen Mund und spürte bereits ihre Zunge, die anfing mit seinen Zähnen zu spielen.  Er wollte ewig mit ihr hier liegen, sich berührend, die Gedanken des anderen lesend und nichts weiter als sich zu streicheln, zu küssen und den anderen zu verwöhnen. Er hatte das Gefühl, eine Einheit mit ihr zu bilden. Er fing an, den Kuss kurz zu unterbrechen, nur um ihr Nachthemd auszuziehen, warf es aus dem Bett und drehte sich auf die Seite, bis er auf ihr lag. Seine Finger immer noch in ihren Haaren, begann er das Spiel mit der Zunge fortzufahren und ihr Gesicht mit vielen Küssen und dem leichten Spiel mit seiner Zunge zu verwöhnen. Er spürte sie und hörte ihre Gedanken, die ihm nur eines vermittelten \gst Mach weiter.

Als er an ihrem Hals angekommen war, löste er seinen Mund von ihrem und schaute ihr in die Augen. Er fand keine Spur mehr von ihrem sonst so verträumten Blick. Sie war voll konzentriert, entspannt und auf ihn fixiert. Er bewegte wieder seinen Mund in ihr Gesicht, aber nur um sie kurz zu küssen und dann wieder loszulassen. Er spürte wie sich ihre erregten Nippel gegen seine Brust drückten und genoss dieses Gefühl. Er wusste genau, was sie wollte, wie sie es wollte und wann sie es wollte. Er begab sich auf Entdeckungstour. Keinen Zentimeter ihres Körpers wollte er auslassen, dachte er sich. Sein Mund wanderte, ihre Haut ganz zart berührend, an ihrer Kehle entlang. Nur um dort kurz zu verweilen, einen Kuss zu hinterlassen, und sich dann weiter ihrem Körper zu widmen. Er umspielte ihre rechte Brust mit seinem Mund und seiner Zunge und hörte zum ersten Mal ein wohliges Raunen in seinen Ohren. Sie hatte es nicht nur gedacht. Mit seiner rechten Hand umspielte er ihre andere Brust. Sie genoss es sichtlich, denn als er einmal aufschaute, sah er ihre geschlossenen Augen. Er hörte sie wieder in seinen Gedanken.

\gedanke{Mach weiter, Harry. Anschauen kannst du mich später noch.}

Er machte weiter, bis er an ihrem Fußknöchel angekommen war. Für einen kurzen Moment hielt er inne. Setzte sich auf, sah an ihr herauf und war sprachlos. Sie lag nun vor ihm. Vollkommen nackt und schön. Sie öffnete ihre Augen und sagte zu ihm.

\enquote{Jetzt bist aber du dran. Ich habe mir für dich was Besonderes überlegt} wohl wissend, dass er ihre Gedanken lesen konnte.

Sie setzte sich auf und warf ihn mit leichtem Druck auf das Bett zurück. Sie kam wieder seinem Gesicht näher und küsste ihn, löste den Kuss und begann das gleiche Programm, das er ihr zu Teil hatte werden lassen. Denn jetzt hatte er nicht nur das Gefühl, ihre Gedanken lesen zu können, sondern auch ihre Gefühle zu teilen. Das war überwältigend. Er spürte ihre Haut auf seiner, ihren Mund seinen Körper entlang spielend, mit der Zunge seine Brustwarzen umspielen. Dabei durchdrang ihn das wohlige Gefühl wie sie es genoss. Das hatte er noch nie erlebt! Er hatte sich schon zuvor vorgestellt, wie es wäre, mit einem Mädchen intim zu werden. Aber das übertraf seine kühnsten Erwartungen. Er fühlte seine und Lunas Gefühle. Es war schwer sie auseinander zu halten und dabei so schön. Ihn überkam wieder das Gefühl mit ihr eine Einheit zu bilden. Zwei Wesen im selben Körper zu sein, oder ein Wesen in zwei Körpern zu sein. Er konnte sich fast vorstellen, wie sie ihn anblickte. Fast war es ihm so, also ob er auf sich hinab durch ihre Augen sehen konnte. Als Luna fertig war, setzte er sich auf und gab ihr einen langen Kuss. Er hatte das Gefühl, als würde ihn das Glück nie wieder verlassen.

\end{abAchtzehn}

\begin{safedivide}
\fskdivider
\end{safedivide}

Nebeneinander sitzend, voll von Schweiß und Speichel des anderen, lagen sie sich in den Armen. Er hatte seinen Kopf auf ihre Schulter gelegt und atmete wie nach einem Marathonlauf. Er wusste nicht, was er Hermine und Ron erzählen sollte, ober er ihnen überhaupt etwas erzählen sollte, denn es fiel ihm wieder Professor Elber ein, der ihn bei einem seiner morgendlichen Spaziergänge etwas über die Dementoren erzählt hatte. Er verwarf den Gedanken und war froh, dass Luna scheinbar zu erschöpft war, seine Gedanken zu lesen. Er stand auf und war auf dem Weg zur Tür.

\enquote{Was hast du vor}, hörte er Luna.

\enquote{Eine Dusche suchen}, gab er zur Antwort.

\gedanke{Warte, ich komme mit}, kam es ihm entgegen. Dieses Mal hatte sie es nur ge\-dacht und Harry grins\-te.

Ein paar Meter weiter fand er eine Aufschrift mit der Bezeichnung Dusch- und Baderäume. Er öffnete die Tür und sah sich um. Luna stand dicht gedrängt hinter ihm und lehnte ihren Kopf auf seine Schulter. Harry ging in eine der freien Kabinen und drehte sich um, nur um Luna die Hand hinzuhalten.

\gedanke{Nimm sie schon}, dachte er. \enquote{Komm Luna, duschen wir gemeinsam.}

Luna kam auf ihn zu und durch Harry lief ein wohliger Schauer. Sie nahm seine Hand und stieg mit ihm in die Dusche. Er öffnete die Hähne und ließ das Wasser über seinen Körper laufen. Luna drängte sich hinter ihn, fuhr mit ihrer Hand zwischen seinem Körper und seinem Arm durch und griff sich die Seife. Mit der anderen Hand tat sie dasselbe und rieb unter einem Wasserstrahl die Seife zwischen ihren Händen. Ihren Körper dicht an ihn gedrückt. Nachdem sich genügend Schaum gebildet hatte, legte sie die Seife zurück, nahm ihn einen Schritt mit nach Hinten, außer Reichweite des Wassers und begann seinen Vorderkörper einzuseifen. Er spürte ihre Brüste an seinem Rücken und ihm entstieg ein Seufzer. Nachdem sie seinen Vorderkörper und seinen Rücken gewaschen hatte, drehte sie sich um und signalisierte ihm, es ihr gleichzutun. Er wollte sich die Spannung etwas bewahren und fing mit ihrem Rücken an, da er sich nach dem Einseifen seiner Hände umdrehen musste. Er rieb in kleinen Kreisen ihren Rücken entlang, seifte sie von oben bis unten ein, und fuhr schließlich zwischen ihren Körper und ihre Arme. Er seifte ihre Vorderseite ein und umspielte zum Schluss ihre Brüste.

Ihr Kopf war auf seiner Schulter zu liegen gekommen und sie sagte neckisch zu Harry: \enquote{Hast dir das Beste wohl zum Schluss aufgehoben?}

\enquote{So weit sind wir noch lange nicht}, meinte Harry, als er einen Seufzer von ihr hörte.

Nachdem sie sich die anderen Körperteile selber eingeseift und von der Seife wieder befreit hatten, stieg Luna zuerst aus der Dusche, um sich ein Handtuch zu nehmen, es sich um den Körper zu wickeln und mit einem weiteren die Haare zu trocknen. Harry drehte in der Zwischenzeit das Wasser ab, stieg ebenfalls aus der Wanne und wickelte sich auch ein Handtuch um seine Taille. Er schaute sich um und entdeckte Spiegel, die über mehreren Waschbecken angebracht waren. Plötzlich fiel ihm auf, dass Luna verschwunden war; aber sie stand schon wieder im Türrahmen mit ihrem Zauberstab in der Hand. Harry stutzte. Sie stellte sich vor einen Spiegel, nahm ihr Haartuch ab und murmelte etwas, das Harry nicht verstand. Mit ihrem Zauberstab fuhr Luna nun über ihr Haar. Harry hörte ein leichtes Surren. Ihn im Spiegel beobachtend und einen etwas ratlosen Gesichtsausdruck zeigend, sagte ihm Luna, dass das ein Haar trocknender Zauberspruch sei. Als sie mit ihren Haaren fertig war, kam sie zu Harry, stellte sich hinter ihn und begann das Gleiche mit seinen Haaren. Im Nu waren sie trocken und Harry hatte sich vorgenommen, sie nach dem Zauberspruch zu fragen.

\zauber{Haraare desert}, hörte er plötzlich.

Sie musste wieder seine Gedanken aufgenommen haben. Trockenen Haares gingen beide wieder in ihr Zimmer und Harry fiel zum ersten Mal auf, dass an der Tür etwas stand. Er war sich sicher, dass dort nie ein Schild gehangen hatte. Auf dem Schild stand \accentuate{Privatraum von Luna Lovegood und Harry Potter.} Harry war sich nicht sicher, was er davon halten sollte und ging zum Schrank auf der gegenüberliegenden Seite des Raumes, um sich etwas zum Anziehen zu holen. Vor dem Schrank stehend drehte er sich noch einmal um. Er sah wie Luna ihr Handtuch abnahm und es aufs Bett warf.

\enquote{Du siehst bezaubernd aus Luna}, sagte er.

Ihre langen Haare lagen über ihren Schultern, ein paar lagen auf ihrem Oberkörper und fielen herab. Ihre Brüste sahen straff und doch zierlich aus. Sein Blick wanderte tiefer und er musste feststellen, jedes Haar an ihr war schneeweiß. Harry war sich der Tatsache bewusst, dass Luna ihn ebenfalls sehr genau beobachtete. Er konnte es fühlen, konnte es hören. Er drehte sich wieder um, als sich Luna auf den Weg zum Schrank machte und zog seine durch Dobby sorgsam zusammengelegte Kleidung heraus, um sich anzuziehen.

Luna stand neben ihm und bekleidete sich auch. Nachdem sich beide wieder angezogen hatten, hörte er wie Luna ein leises und unsicheres: \enquote{Sind wir jetzt ein Paar?}, abgab.

Harry antwortete: \enquote{Ich bin mir nicht sicher Luna, ich glaube schon. Aber noch ist es zu früh, dass anderen von uns zu erfahren. Warten wir noch etwas ab.}

Sie nickte und ging durch die Tür nach unten. Harry drehte sich wieder zum Schrank, schloss ihn und schaute sich ein letztes Mal, im Türrahmen stehend, um, nur um sich zu vergewissern nichts vergessen zu haben. Im Gemeinschaftsraum der Paare wieder angekommen, stand Luna vor dem Bücherregal und hatte eines der unbeschrifteten Bücher in der Hand.

Harry näherte sich und konnte lesen:

\begin{brief}
Dieser Gemeinschaftsraum der Paare wurde von uns als ein Zuflucht- und Rückzugsort in Hogwarts geschaffen. Verschiedene Schutzzauber verhindern die zufällige Entdeckung und das Aufspüren dieses Ortes. Leider hat einer der Zauber, den wir nicht mehr rückgängig machen konnten, einen kleinen Nebeneffekt. Wenn sich zwei seelenverwandte Personen hier einfinden, entwickeln sie eine so enge Beziehung zueinander, dass sie Gedanken und Gefühle des anderen lesen und deuten können; zu jeder Zeit, an jedem Ort. Dieses kann man durch Training abblocken. Außerdem kann es in \gst und sei die Wahrscheinlichkeit auch noch so klein \gst wenig Fällen dazu führen, dass man durch die Augen des anderen sehen kann, hören, was der andere hört, fühlen, was der andere fühlt. So als seien es die eigenen Empfindungen. Aber normalerweise haben die Paare nur eine intuitive Empfindung, wo sie ihren Partner finden und was er fühlt.
\end{brief}

\enquote{Das erklärt also, was mit uns passiert ist}, sagte Harry. \enquote{Steht noch mehr drin?}

Luna blätterte um, doch die nachfolgenden Seiten waren leer. \enquote{Vielleicht sind wir noch nicht so weit}, sagte Luna, als sie sich Harry zuwandte.

Sie stellte das Buch wieder in das Regal und die Beiden verschwanden, um in der Großen Halle zu frühstücken. Sie war noch leer, als die beiden ankamen. Beide beschlossen, sich an ihre jeweiligen Tische zu setzen und schon mal mit dem Frühstück zu beginnen. Harry hatte noch etwas Zeit und ließ sich, während er seinen Kürbissaft, den er dieses Mal mit Orangensaft mischte, etwas einfallen, was er Ron und Hermine sagen konnte. Schon halb mit dem Frühstück fertig, füllte sich die Große Halle langsam und auch Hermine trat mit Ginny ein. Sie setzten sich gegenüber von Harry.

Hermine fragte ihn: \enquote{Wo warst du letzte Nacht? Neville hat gesagt, du warst nicht in deinem Bett.}

\enquote{Ich war beim Lesen eingeschlafen.}

\enquote{Beim Lesen eingeschlafen? Du warst nicht einmal im Gemeinschaftsraum}, schnauzte ihn Hermine an.

Harry antwortete nur etwas von Strafarbeiten und Nachsitzen. Hermine und Ginny schüttelten, sich gegenseitig anschauend, nur die Köpfe und fingen mit ihrem Frühstück an.

Als er mit Ron und Hermine die Große Halle verlassen hatte, um zusammen Hausaufgaben zu machen, sagte Hermine plötzlich: \enquote{Sagt mal, ist euch aufgefallen, dass Luna Lovegood irgendwie anders aussieht?}

Harry wusste natürlich, was sie meinte, verkniff sich aber irgendwas zu sagen und antwortete nur lapidar. \enquote{Was meinst du?}

\enquote{Na, ihren Gesichtsausdruck. Sie schaut nicht mehr so aus, als ob sie den ganzen Tag träumen würde.}
Harry grinste nur.

\trenn

\enquote{Ein interessantes Buch haben sie da, Professor}, meinte Hermine zu Dumbledore, als er es auf dem Küchentisch des Grimmauldplatz 12 aufgeschlagen hatte.

\enquote{Das habe ich mir ausgeliehen}, sagte Dumbledore. Er blätterte durch das Inhaltsverzeichnis und schlug danach eine bestimmte Seite auf. Dann suchte er in den nachfolgenden Seiten einen bestimmten Zauberspruch. \enquote{Ah, hier haben wir ihn.} Er zog seinen Zauberstab und ging in den schmalen Flur vor das mit Stoffen verhangene Bild von Mrs Black. Dumbledore murmelte einige Zauber und das Bild fiel von der Wand.

Der Stoff viel herunter und sofort begann Miss Black zu fluchen und zu schreien. Bis sie merkte, dass ihr Bild am Boden stand. \enquote{Was habt ihr mit meinem Bild gemacht? Blutverräter. Abschaum}, brüllte sie. Sofort legte Harry wieder ein Tuch über das Bild, während er Dumbledore hinterherging und die Schreie und Flüche verstummten. Währenddessen blätterte Hermine weiter in dem Buch und sah sich diverse Zaubersprüche an.

Es war gerade Samstag und Harry und Hermine hatten die Erlaubnis, unter Aufsicht Sirius' altes Haus zu entrümpeln. Die Treffen des Ordens wurden verlagert, da keiner nach Sirius’ Tod mehr hier die wichtigen Themen besprechen wollte. Andererseits wusste niemand so genau, welche Auswirkungen es hatte, wenn die Person starb, welche den Zauber aussprach, aber nicht der Geheimniswahrer war. Doch trotz Dumbledores Bemühungen und der Zusage, dass es keine negativen Auswirkungen hätte, wollte keiner mehr an den Ort zurück, der das Zuhause eines ihrer gefallenen Kameraden war.

Harry hatte Kreachers Küchenschrank geöffnet und die gehamsterten Sachen auf den Küchentisch gelegt. Nachdem er den Abfall weggeworfen hatte (verschimmeltes Brot, verdreckte Küchentücher, welken Salat und andere unnütze Sachen), zauberte er eine Schachtel herbei, in die er die Reste legte, welche er Kreacher bringen würde. Dumbledore lief unterdessen durch das Haus und entfernte die Dauerklebezauber der Dinge, die man nicht mehr brauchte.

\enquote{Harry?}, sagte Dumbledore plötzlich. Harry sah zu ihm und er sprach weiter: \enquote{Nenn’ mich in Zukunft, wenn wir alleine sind ruhig Albus. \gst Aber eigentlich wollte ich dir etwas anderes sagen.} Er pausierte kurz. \enquote{Ich möchte, dass dir dein \VgddK-Lehrer Einzelunterricht gibt. Das heißt eigentlich \gst ich denke, dass er ihn dir sowieso gibt, auch ohne meine Erlaubnis. \gst Ich will dir damit eigentlich nur sagen, dass du dich darauf einstellen solltest, dass er dir private Stunden geben wird. \gst Er sieht es genauso wie ich, dass dir Gefahren drohen, für die ich keine Zeit habe dir Gegenmaßnahmen beizubringen\abs nicht mehr.}

\enquote{Professor}, sagte Harry. Dumbledore hob seine Hand und schaute ihn streng an. \enquote{Albus, was bedeutet das, was wird er mir beibringen?}, fragte Harry ganz überrascht.

\enquote{Ich schätze erweiterte Magie. Etwas, was man normalerweise nicht in der Schule lernt. Eventuell auch einen Einblick in die schwarzen Künste, damit du weißt, womit du es zu tun bekommst, da Voldemort ein großes Interesse an dir hat.}

\enquote{Da haben wir während des Unterrichtes schon etwas angefangen}, antwortete Harry vorsichtig.

Dumbledore sah ihn an, als hätte er zum ersten Mal einen Geist gesehen. \enquote{Er hat was?}, fragte er nach.

\enquote{Er weist uns gerade in die dunklen Künste ein}, sagte Hermine, die gerade in den Flur kam. \enquote{Nur leichte Sachen, damit wir wissen, warum und wie wir uns verteidigen müssen.} Dumbledore war sprachlos, aber auch leicht verlegen. Anscheinend hatte er es auf diese Weise noch nicht gesehen. \enquote{Wissen Sie, Professor}, erzählte Hermine weiter, \enquote{ich habe den Eindruck, dass er es als ganz normal empfindet, darüber zu sprechen.} Dann ging sie mit dem Bild von Mrs Black auf den Dachboden hinauf.

Dumbledore sah Harry nachdenklich an. \fluestern{Sollte ich mich in ihm getäuscht haben?}, fragte er sich selbst.

Um das Thema zu wechseln, sprach er: \enquote{Gilt das mit Albus, auch wenn Hermine oder Ron dabei sind, Albus?}, fragte er leicht grinsend.

Dumbledore antwortete: \enquote{Ja.} Dann begann sein Magen zu knurren. \enquote{Wird Zeit, dass wir was essen. Wenn Hermine wieder da ist, dann gehen wir in ein kleines Muggellokal. Ich lade euch ein. Komm, wir ziehen uns schon einmal unsere Jacken an.}

Beide gingen nach draußen und warteten mit ihren Jacken auf Hermine, die kurz darauf herunterkam und sie erstaunt ansah.

\enquote{Komm Hermine, Albus lädt uns zum Essen ein. Wir gehen in ein Muggellokal.}

\enquote{Albus?}, fragte sie ganz ungläubig.

\enquote{Solange wir alleine sind, oder maximal Minerva dabei ist, dann gilt das}, sagte Albus und reichte Hermine ihre Jacke.

Sie verließen das Haus und Harry sicherte es mit ein paar Zaubern, die er in einem Buch heute Morgen gefunden hatte. In diesem stand, wie man das Haus der Blacks sichern konnte und die Zauber, die auf ihm lagen, aktiviert wurden. Nur der Hausherr konnte diese Zauber aktivieren und sie für die Familienmitglieder freigeben.

\gedanke{Sirius musste davon nichts gewusst haben}, überlegte Harry.

Der Weg zum Lokal war kurz und für die Zeit etwas windig. Aber der Wind, der ihnen entgegenblies, war noch warm.

Im Lokal angekommen, es war eher eine gemütliche Wirtschaft, setzten sich Harry und Albus über Eck an die Wand hinter einen Tisch. Hermine setzte sich neben Harry, sodass alle drei den Schankraum gut im Blick hatten. Die Bedienung kam sofort und nahm die Getränke auf.

\enquote{Ihr seid meine Enkel, falls euch jemand fragt. Wir sind auf der Durchreise von Bath und wollen an die Ostseite, um Verwandte zu besuchen}, sagte Albus zu den Zweien.

Hermine und Harry nickten und sahen sich weiter um, während die Getränke gebracht wurden. Am Stammtisch fiel ihnen ein Mann auf, der scheinbar anzugeben schien.

\enquote{Ekelhaft, wie der angibt}, meinte Hermine. Harry nickte nur.

\enquote{Albus? Warum habe ich keinen vom\abs na ja, keinen im Haus gesehen?}

Albus druckste etwas herum. \enquote{Nach Sirius’ Tod wollten die anderen nicht mehr so gern ins Haus. Es war ihnen schon damals nicht ganz geheuer. Außerdem brauchen wir deine Erlaubnis als neuem Hausherrn.}

\enquote{Von mir aus gern}, antwortete Harry. \enquote{Ich bin eh die meiste Zeit über in Hogwarts und in den Ferien muss ich ja zu meinen Verwandten.} Dabei zog er eine Miene. \enquote{Was ist eigentlich mit meiner Tante los? Weißt du was? Ich bin der Meinung, dass sie die Zeit, in der ich bei ihr war, sich anders verhalten hat. Ich habe Begriffe von ihr gehört, von denen ich eigentlich sicher war, dass sie sie nicht kannte. Oder zumindest nicht in den Mund nehmen würde.}

Das Essen wurde gebracht und sie begannen den Fisch, das Rindfleisch und den Salat zu essen.

\enquote{Darauf kann ich dir leider keine zufriedenstellende Antwort geben}, sagte Albus.

Harry musste diese Antwort akzeptieren. Er wusste, dass Albus ihm die richtige Antwort nicht geben würde, sollte er sie wissen. Andererseits wollte er vielleicht auch nicht zugeben, dass er es nicht wusste. Es war recht kurzweilig. Nach dem Essen zahlte Albus und die beiden dankten ihm, als die Bedienung wieder gegangen war. Der Mann am Stammtisch gab immer noch an. Jetzt hörten die drei es genauer.

\enquote{Ich saufe noch immer jeden unter den Tisch, der mich herausfordert.}

Albus, der den Mann die ganze Zeit mit einer Abneigung angesehen hatte, kam zu dem Entschluss, dem Aufschneider eine Lektion zu erteilen.

\enquote{Wartet mal, oder trinkt noch einen Tee, das kann jetzt etwas dauern.} Er ging auf den Mann zu und sprach ihn an. \enquote{Nehmen Sie es auch mit mir auf?}, fragte er den Mann höflich.

\enquote{Ich trinke doch nicht gegen dich, Alterchen}, spottete der Mann.

\enquote{Ahh, große Klappe, aber wenn sie einer herausfordern will, dann ziehen sie den Schwanz ein?}

\enquote{Ich will dich nur vor Peinlichkeiten bewahren, Opa.}

\enquote{Und an Respekt mangelt es ihnen auch, wie ich sehe.} Dumbledore setzte sich und fragte weiter: \enquote{Was sind die Bedingungen des Wetttrinkens?}

Der Aufschneider wollte gerade seinen Mund wieder öffnen, doch einer seiner Kumpel unterbrach ihn. \enquote{Beide trinken gleichzeitig einen Schnaps nach dem anderen, bis einer keinen mehr trinken kann, oder will. Der Verlierer zahlt alle getrunkenen Schnäpse. Also seine eigenen und die des Gegners.}

\enquote{Das ist annehmbar. Ich akzeptiere.} Dumbledore schnippte mit den Fingern und sah dabei Richtung Theke.

Der Wirt nickte ihm zu und kam mit zwei Schnapsgläsern und zwei Schnapsflaschen auf einem Tablett zu den beiden Kontrahenten. Die Schnapsgläser stellte er vor Dumbledore und dem Mann ab, die Flaschen hingegen vor seinem Kumpel, der die Regeln erklärte, damit dieser sich vom ordnungsgemäßem Zustand des Schnapses überzeugen und die Gläser einschenken konnte.

Der Freund öffnete die Flasche und roch daran. Dann schenkte er beiden ein. Dumbledore nahm das Glas zwischen zwei Fingern und drehte es auf dem Tisch und eine viertel Umdrehung. Dann nahm er das Glas hoch und wartete, bis sein Gegenüber ebenfalls so weit war. Zeitgleich kippten sie sich die Schnäpse hinunter.

So ging das viele Gläser weiter. Dumbledore drehte jedes Mal sein Schnapsglas auf dem Tisch und trank es dann erst leer.

\enquote{Wie schafft er so viel Alkohol?}, fragte Hermine Harry, nach einem Schluck Tee. \enquote{Er müsste doch schon längst besoffen sein.}

\enquote{Schau dir doch mal an, was er mit dem Schnapsglas macht}, hauchte der in ihr Ohr. \enquote{Er dreht es, also wird er wohl irgendwas damit machen.}

Hermine spürte eine Gänsehaut an ihrem Nacken herunterlaufen. Harry entging dies nicht, also ging er einen kleinen Schritt weiter und küsste ihr Ohr und biss kurz darauf ganz sanft in ihr Ohr. Sofort ließ er wieder von ihr ab und nippte an seinem eigenen Tee.

Hermine war noch für eine Sekunde ganz benommen, dann fing sie sich wieder, schlug Harry leicht auf den Arm und meinte nur: \enquote{Harry.}

\enquote{Es hat doch funktioniert, oder?}, fragte Harry ganz scheinheilig.

\enquote{Was funktioniert?}, fragte Hermine nach.

\enquote{Es lenkt deine Gedanken von dem Wetttrinken und deinem Ekel davon ab}, grinste Harry.

Überraschend gab sie ihm einen Kuss auf die Wange und meinte dann: \enquote{Da hast du recht. Es hat geholfen, danke.}

Dann hörten sie ein Grölen und einen \geraeusch{Plumps}. Der Mann lag bewusstlos am Boden, während Dumbledore seinen letzten Schnaps austrank, dann aufstand und leicht zu wanken schien. Mit glasigem Blick verabschiedete er sich und verließ das Lokal. Harry warf ein paar Pfund auf den Tisch, nickte dem Wirt kurz zu und verließ mit Hermine das Lokal. Um die Ecke fing sich Dumbledore wieder und lief nüchtern und gerade zwischen Hermine und Harry zurück zum Grimmauldplatz.

\enquote{Netter Trick mit dem Schnaps-zu-Wasser. Aber wie?}

Dumbledore grinste. \enquote{Ein einfacher Zauber. \spruch{Decoholo}.}

\enquote{Den muss ich mir merken. Ist sicherlich hilfreich, falls man auf einer Party eingeladen ist.}

Hermine sah ihn nur entgeistert an. \enquote{Heißt das, dass sie\abs keinen Tropfen Alkohol getrunken haben, P\abs Albus?}

Albus grinste sie an: \enquote{Genau, Hermine. Nur Wasser.}

\enquote{Und ihr Wasserbauch?}

\enquote{Der war etwas schwerer. Das Wasser muss noch in der Speiseröhre verdunsten und als feiner, nicht sichtbarer Nebel durch Mund und Nase entweichen. Das hat nicht immer geklappt}, grinste er. \enquote{Die nächsten Stunden brauche ich keine Flüssigkeit mehr. Und jetzt muss ich erst mal, wenn wir zurück sind.}

Im Verlauf des weiteren Tages verbrachten sie die meiste Zeit damit, Räume zu putzen. Harry und Hermine machte es Spaß, die Räume aufzuräumen und mit Zaubern vor erneutem Einstauben zu sichern. Mithilfe des Buches war es leicht, die Doxys mitsamt den Eiern zu entfernen, die beißenden Teppiche stillzulegen und andere Artefakte zu entzaubern.

Am Ende des Tages hatten sie etwa ein Viertel der Räume von diversen lästigen Sachen befreit. Vom Staub wischen ganz abgesehen. Denn der Staub schien ebenso schnell wieder zurückzukommen, wie sie ihn entfernten. Doch es gelang ihnen schließlich mit speziellen Zaubern aus dem Buch. Dumbledore gab Hermine das Buch in die Hand. Dann verließen sie das Haus und stiegen auf die unterste Stufe der Treppe. Hermine und Harry fassten je einen Arm von Professor Dumbledore und er disapparierte mit ihnen.

Zurück auf Hogwarts überreichte Hermine Dumbledore das Buch und ging mit Harry und seiner Schachtel Richtung Gryf\-fin\-dor-Ge\-mein\-schafts\-raum. Den Inhalt würde er Kreacher morgen geben.

Hermine war bei der Hinreise erstaunt, wie es Dumbledore schaffte, durch die Schutzzauber des Schlosses zu kommen. Aber als er ihr sagte, dass er als Direktor die Möglichkeit hatte, sich einen speziellen Ort auszusuchen, von dem das klappte, und es nur ihm, sowie mit ihm verbundenen Begleitern, klappte, akzeptiert sie es. Bei nächster Gelegenheit würde sie in der Bibliothek nach diesem Phänomen suchen.

Nach dem Frühstück ging Harry in den Gemeinschaftsraum zurück, um seine Schachtel zu holen. Dann ging er hinunter Richtung Kerker und bog in den Gang ab, der in die Küche führte. Er stellte seine Schachtel ab und kitzelte die Birne, drehte danach den Knopf und öffnete die Tür, nahm seine Schachtel und betrat die Küche. Die Tür hinter ihm schloss sich.

Die Elfen räumten die Teller und das Besteck, sowie die Trinkbecher, weg, die auf den Tischen erschienen. Harry war der Meinung, dass die Elfen heute etwas langsamer wären als sonst. Er sah sich um und entdeckte Kreacher, der gerade Teller reinigte. Sie schwebten durch das Spülbecken und kamen sauber heraus. Danach flogen sie in die offenen Türen der Küchenschränke.

Harry lief auf ihn zu. \enquote{Kreacher?}

Der alte Elf drehte sich um. \enquote{Ja Meister}, sagte er mit öliger Stimme.

\enquote{Ich habe da etwas für dich.} Harry überreichte ihm die Sachen aus dem Küchenschrank, in dem er sonst schlief. \enquote{Das schlechte Essen und die dreckigen Sachen habe ich weggeworfen. Aber der Rest ist wohl deines}, sagte Harry.

Kreacher nahm die Schachtel an sich und sah sich jedes Stück genau an. Harry nahm an einer langen Bank Platz. Er machte sich keine Gedanken darüber, aber er saß an einem Tisch, der direkt unter dem Tisch der Slytherins stand. Kreacher verbeugte sich tief und lies ein leises \enquote{Danke Meister}, erklingen. Dann verschwand er.

\gedanke{Er muss wohl seine Schätze in Sicherheit bringen}, dachte Harry. Als Kreacher danach wieder kam, hatte er für Harry ein Glas Kürbissaft in der Hand. Harry nahm es dankend entgegen und trank es aus. Er wollte nicht unhöflich sein. Dann verabschiedete er sich von Kreacher, um sich seinen Hausaufgaben zu widmen.

\trenn

Als am Freitagabend Katie in den Gemeinschaftsraum kam, war sie ganz aufgelöst. Sie kam gerade von der Krankenstation, wo sie sich von Madame Pomfrey eine Verletzung an der Hand kurieren lassen wollte. Sie setzte sich schluchzend in einen Sessel und merkte nicht, dass dort bereits jemand saß.

\enquote{Au}, meckerte Ron, auf den sich Katie setzte. Sie schien ihn nicht zu beachten. Ron nahm Katie unter den Schultern hoch, um aufzustehen. Dann setzte er sich kurz auf eine Armlehne, um Katie herunterzulassen. Fast hätte er das Gleichgewicht verloren, wenn ihn Hermine nicht gehalten hätte. \enquote{Danke}, sagte er schnell. Er setzte sich neben Harry und betrachtete Katie, die bittere Tränen heulte. Ihre Hände hatte sie unter ihrer Robe versteckt.

\enquote{Was ist los Katie?}, fragte Hermine.

\enquote{Ich\abs} stammelte Katie. Doch ehe sie weiter machen konnte, um ihre Tränen aus dem Gesicht zu wischen, rutschten ihr, als sie ihre Hände nach oben nahm, ihre beiden Ärmel herunter und zeigten den Grund für Katies Heulattacke. Sie hatte nur noch ihre linke Hand. Die Rechte war ab und der Arm hörte kurz bevor das Handgelenk anfing auf. Sie hatte nur noch einen Stumpf.

Hastig schlug sie ihre Ärmel wieder hoch, damit es keiner sehen würde. Katie schluchzte noch mehr und saß kurz darauf auf Rons Schoss und begann ihn zu umarmen. Dieser riss die Augen ungläubig auf und wollte Katie schon abweisen, als Hermine seine Hände ergriff und um Katie legte. Sie flüsterte ihm leise etwas ins Ohr, worauf hin Ron sich doch noch entspannte und nun Katie festhielt.

Sie weinte minutenlang in Rons Schulter und machte seinen Umhang ganz nass. Harry rutschte näher an sie heran und legte jetzt ebenfalls einen Arm um sie.

Als Lavender die Treppe herunterkam und \enquote{Wird das ein Dreier?}, fragte, sahen sie alle mehr als böse an.

Hermine zog Lavender wütend nach oben und schimpfte unentwegt auf sie ein. \enquote{Du unsensibles Miststück. Du\abs} Doch mehr hörte man nicht mehr.

Als die beiden nach mehreren Minuten wieder herunterkamen, entschuldigte sich Lavender bei ihnen und umarmte Katie zusätzlich. Diese beruhigte sich und sah in Rons Augen. Sie nahm seine Hand in die ihre und gab ihm einen kurzen Kuss auf den Mund. \enquote{Danke}, hauchte sie ihm entgegen. Ron sah unsicher zu Hermine, die mit ihrem Fuß auf dem Boden wippte. Katie löste sich wieder von Ron und setzte sich in einen Stuhl ihnen gegenüber. Hermine nahm kurz auf Rons Schoß Platz und rutschte dann neben ihm auf den schmalen freien Platz zwischen ihm und Harry und sah mit gemischten Gefühlen zu Katie. Lavender hatte sich vor Katies Sessel niedergelassen und sah sie traurig an.

Dann zog sie einen Ärmel hoch und schüttelte den anderen zurück. \enquote{Madame Pomfrey konnte nichts mehr tun. Der Giftstachel war zu tief eingedrungen und meine Allergie gegen dieses Gift verschlimmerte es noch zusätzlich.}

Wieder flossen vereinzelte Tränen Katies Gesicht hinunter. Harry trocknete ihre Tränen mit einem Taschentuch. Dann gab er es Lavender, die näher an Katie saß.

Die nächsten Tage war die Stimmung durch Katies Verlust getrübt. Keiner hatte mehr Elan und war schlecht gelaunt.

\trenn

\enquote{Morgen werden wir uns dem Patronus-Zauber widmen. Schreiben Sie zunächst die ersten Absätze des Kapitels über Patroni ab.}

Die Klasse raschelte eifrig in ihren Taschen und nahm Buch und Pergament heraus. Lavender, die neben Katie saß, legte ihr Buch in die Mitte. Mit unsicherer Hand nahm Katie die Feder in ihrer linken Hand und versuchte den Abschnitt abzuschreiben.

Professor Elber ging durch die Reihen und beobachtete die Schüler. Als er an Katie vorbeiging, meinte er: \enquote{Seit wann schreiben sie mit der linken Hand?}

Harry konnte ihn kaum verstehen. Er stand sehr nah bei ihr und beugte sich zu ihr herunter. Harry konnte nur ein Rascheln hören.

\enquote{Deshalb Professor}, sagte sie. Sie musste ihm ihren Stumpf gezeigt haben.

\enquote{Bleiben Sie nach der Stunde noch hier. Ihre Freundin wird sicher ihre Tasche mitnehmen.} Dann ging er weiter durch die Reihen und setzte sich auf seinen Stuhl. Die Arme auf den Lehnen gelegt, wartete er, bis alle fertig waren.

\enquote{Welche Gestalt wird ihr Patronus wohl annehmen, wenn sie ihn heraufbeschwören? Falls er überhaupt eine Gestalt annimmt.}

Harry hob als Erstes seine Hand.

\enquote{Ja?}

\enquote{Ein Hirsch.}

\enquote{Sie nehmen an, dass\abs}

\enquote{Nein. Mein Patronus ist ein Hirsch.}

\enquote{Das werden wir dann sehen}, sagte er. \enquote{Weitere Meinungen?}

\enquote{Hund.}

\enquote{Otter.}

\enquote{Fuchs}, kam ihm entgegen.

\enquote{Gut, die Patroni werden wir dann alle morgen früh sehen. Kommen Sie bitte zu Hagrids Hütte. Wir werden morgen im Freien sein. Wir treffen uns dort und gehen dann zum Treffpunkt. Für heute ist die Stunde beendet. Gehen Sie jetzt Essen.}

Die Klasse packte zusammen. \enquote{Professor?}, fragte Hermine. \enquote{Was ist ihr Patronus?}

Professor Elber saß noch immer in seinem Stuhl. Nun verschränkte er seine Finger und legte die Daumen auf seine Lippen. Er sah jetzt genauso aus wie Dumbledore. \enquote{Lassen Sie sich überraschen.} Dann wandte er sich zu Katie. \enquote{Sie bleiben noch hier.}

Harry aß begeistert an einem Maiskolben herum, als Katie wieder in die Große Halle kam. \enquote{Und?}, fragte sie Lavender.

Katie ignorierte sie und legte ihre Ärmel auf den Tisch. Sie goss sich mit der linken Hand etwas Kürbissaft ein und verdünnte ihn mit Wasser. Dann nahm sie mit der Gabel einen Hühnchenschenkel und aß daran. Harry stellte gerade seinen Trinkbecher wieder ab und schluckte herunter, als Katie mit der anderen Hand nach ihrem Trinkkelch griff. Sie sah ihm in die Augen und trank dann daraus. Harry bekam große Augen. Zum Glück hatte er gerade heruntergeschluckt.

Harry sammelte sich kurz und biss beiläufig in seinen Maiskolben. \enquote{Was ist mit deiner Hand?}, fragte er so normal wie möglich.

Sie grinste ihn an. Dann sahen die anderen, dass Katies rechte Hand wieder vollkommen in Ordnung war Sie hatte nun wieder zwei Hände.

\enquote{Wie?}

\enquote{Was?}

\enquote{Woher?}, fragten die um sitzenden Katie plötzlich.

\enquote{Nachher, im Gemeinschaftsraum}, kam daraufhin von Katie.

Als sie die Große Halle verließen, hatte Katie wieder ihre Hände in den Ärmeln verborgen. Die gehässigen Sprüche einiger Slytherin überging sie. Sie würde nicht überreagieren und ihre zweite Hand herausholen, nur um es den Slytherin zu zeigen.

Im Gemeinschaftsraum angelangt, setzte sich Katie in einen Sessel. Tamara saß bereits auf einem Sofa, das gegenüber stand. Harry setzte sich auf den Boden und sah zu Katie auf. Er mochte sie seit sie zusammen Quidditch spielten. Dann fing Katie an zu erzählen.

\enquote{Nachdem ihr gegangen seid, nahm mich Professor Elber mit in den Krankenflügel. Er bat mich, mich auf einen Stuhl zu setzen, damit ich nachher meine Arme auf das Bett legen könnte. Dann ging er zu Madame Pomfrey in ihr Büro und kam kurz darauf mit einer Flasche Skele-Wachs und einer Salbe heraus. Madame Pomfrey folgte ihm. Dann legte er die beiden Sachen auf das Bett. Er bat mich, meine beiden Arme auf das Bett zu legen. Ich tat, was er mir sagte. Dann streifte er mir beide Ärmel hoch und besah sie sich. Ich fragte ihn, was er vorhabe. Er antwortete nur, dass er sich um mich kümmern würde und ich meine Hand zurückerhalten würde. Madame Pomfrey sah ihn ungläubig an und beschwerte sich, aber er unterbrach sie und bot ihr einen Stuhl an. Dann erklärte er ihr, dass es eigentlich nicht schwer sei, eine abgetrennte Hand wieder komplett wachsen zu lassen. Falls es sie interessieren würde, könne sie gern mitschreiben und ihn alles fragen.}

Alle, die Katie zuhörten, bekamen große Augen. Keiner wagte etwas zu fragen, bevor Katie nicht fertig mit ihrer Erzählung war. Harry fuhr leicht über ihre neue Hand. Sie zuckte, zog sie aber nicht zurück. Dann lauschten alle weiter gespannt.

\enquote{Dann gab er mir eine kleine Holzrolle und meinte, dass er jetzt leider, nachdem der Stumpf schon geheilt und nicht mehr offen war, vorne ganz knapp abschneiden müsse. Er würde zwar den Schmerz betäuben, aber es würde doch weh tun. Wenn er ihn ganz lindern würde, dann müsste ich mehrere Tage lang meine Hand in einer Art starre halten. Also stimmte ich zu und nahm die Holzrolle in den Mund. Madame Pomfrey sah mich mit feuchten Augen an. Dann sprach Professor Elber einen seltsamen Zauber. Ich glaube, er hieß: \enquote{Sectum sempra.} Meinen Stumpf schnitt er ganz vorne ab. Ich schrie in meinen Holzklotz, aber der Schmerz war nur von kurzer Dauer. Er nahm mir die Rolle wieder aus dem Mund und berührte mich dabei\abs Dann war mein Arm wieder offen und blutete. Die Blutung wurde sofort wieder gestoppt und ich musste einen Becher Skele-Wachs austrinken. Währenddessen sprach Professor Elber mit seinem Zauberstab auf meinen offenen Arm gerichtet Beschwörungsformeln. Ich habe noch nie zuvor gesehen, wie meine Knochen gewachsen sind. Schließlich waren alle vollständig. Dann sagte er mir, dass ich meine Hand nun ruhig halten müsse. Obwohl es kribbeln würde, müsse ich sie ganz still halten. Er strich mit seinem Zauberstab über meine Knochen und sagte Madame Pomfrey den passenden Spruch. Meine Sehnen bildeten sich an den Knochen entlang aus. Danach waren meine Muskeln dran. Und als letztes meine Hautschichten. Ich musste meine Hand noch eine Weile ruhig halten. Danach konnte ich sie bewegen. Er nahm meine neue Hand zwischen seine Finger und fuhr jeden einzelnen Finger ab. Jeden Zentimeter fuhr er mit seinen zarten Fingern ab. Als ich ihn fragte, warum er das machte, sagte er nur, dass er die Sensorik überprüfen müsse. Wenn eine Stelle taub sei, dann müsse ich ihm das sagen. Ich spüre jetzt noch die zarten Bewegungen und Berührungen auf meiner Haut. Dann konnte ich gehen.}

Katie fügte noch hinzu: \enquote{Als ich dann ging, hörte ich nur noch wie Madame Pomfrey fragte, warum das nicht bekannt sei. Er antwortete darauf hin nur: \enquote{Viele würden dazu schwarze Magie sagen, aber Magie hat keine Farbe.} Da fiel mir die erste Stunde ein, als du Hermine,} sie sah Hermine an \enquote{die beiden Pflanzen abbrennen musstest.}

\begin{rueckblick}
\enquote{Aber warum ist das schwarze Magie?}

\enquote{Wissen Sie, Poppy. Mit diesen Zaubern kann man die Kontrolle über die neu gewachsene Hand erlangen. Man muss beim Zaubern aufpassen, dass die Kontrolle beim Körper bleibt, bzw. ihm übertragen wird}, erklärte er ihr.
\end{rueckblick}

\enquote{War Madame Pomfrey sehr sauer?}, fragte Tamara.

\enquote{Eher entsetzt, dass man so etwas an einer Schülerin anwenden würde. Aber ich glaube, dass ihre Hingabe an diesen Beruf sie darüber hinwegsehen lässt und sie das in Zukunft auch anwenden wird.} Sie grinste. \enquote{Ich muss jetzt noch Hausaufgaben machen}, sagte Katie und ging nach oben, um ihre Tasche zu holen. Als sie wieder kam, setzte sie sich an einen Tisch und fing an.

Harry richtete sich nach einer Weile, in der er über seinen Lehrer nachdachte, fürs Bett her. Er legte die Hände hinter seinen Kopf und dachte nach. Er stellte sich die Szenen im Krankenflügel bildlich vor und in seinem Geist bildete sich der Ablauf. Er dachte noch an die Stunde, in der sie Patroni üben würden. \gedanke{Die DA dürfte damit kein Problem haben.} Dann schlief er müde ein.

\trenn

\enquote{Einen wunderschönen guten Morgen}, sprach Professor Elber die Klasse an. \enquote{Schön, dass sie es einrichten konnten, heute hier zu sein.} Die gesamte Klasse war anwesend. Gryffindor ebenso wie Slytherin. \enquote{Ich sagte ihnen bereits gestern, dass wir uns heute um den Patronus-Zauber kümmern werden. Sie kennen bereits den entsprechenden Zauber und die Bewegung des Zauberstabes. Sie haben sie ja gestern abgeschrieben. Nun folgen Sie mir bitte, wir gehen ein Stück Richtung Wald.} Er drehte sich um und die Klasse folgte ihm.

Hagrid, der gerade seine Hütte verlassen hatte, winkte Harry, Ron und Hermine zu. Die drei winkten zurück.

Kurz vor einem Waldstück blieb Professor Elber stehen. Es wurde bereits merklich kühler. Er zog seinen Zauberstab und sprach: \zauber{Expecto Patronum.} Eine dicke Nebelschwade breitete sich aus. Sie verdichtete sich zu einer kleinen Wolke und zog dann durch die Schüler hinweg.

Harry konnte vereinzeltes Kichern und Getuschel hören.

\enquote{Will uns einen Patronus beibringen\abs}

\enquote{Kann keinen gestaltlichen\abs}

Harry hatte das Gefühl, das da noch mehr kommen würde. Sobald er seinen Gedanken beendet hatte, wandelte sich der Nebel. Er bildete viele kleine Kügelchen aus. Die Kügelchen formten sich zu Bienen, Hummeln und Wespen, sowie Hornissen. Jetzt flogen viele kleine Patronus-Insekten um die Schüler herum und die aufsteigende Wärme, die der Nebel bereits brachte, verstärkte sich.

Sie folgten weiter ihrem Professor in den Wald. Dann sah Harry fünf Gestalten. Schwebende Figuren, die mit Stoffen behangen waren.

\enquote{Dementoren}, hörte er.

Er konnte es nicht fassen. Vor der Klasse schwebten tatsächlich fünf Dementoren.

Professor Elber drehte sich um und begann nun. \enquote{Diejenigen von ihnen, die gestern bereits ihren Patronus genannt hatten, treten zwei Schritte vor.}

Die gesamte DA trat jetzt aus der Menge hervor.

\enquote{Diejenigen unter ihnen, die schon einmal einen Patronus erzeugt haben, treten abermals einen Schritt vor.}

Wieder trat die gesamte DA einen Schritt vor.

Professor Elber zeigte sich erstaunt. \enquote{Ok, dann treten Sie bitte einzeln vor, aus dem schützenden Patronus-Feld heraus und stellt euch einem Dementor.}

Wie auf ein unsichtbares Zeichen arrangierten sich die Dementoren um und ein einzelner schwebte einige Meter vor, um den ersten Schüler zu erwarten.

\enquote{Na los Harry}, hörte er seine Mitschüler sagen.

Harry trat vor und ließ seinen Patronus erscheinen. \zauber{Expecto Patronum.} Der Patronus stellte sich dem Dementor in den Weg, der begeistert die Energie aufnahm. Als der Dementor sich wieder zurückzog, gab Professor Elber Harry ein Zeichen, der daraufhin seinen Patronus verschwinden ließ. Nun waren die anderen der DA dran. Alle absolvierten sie ihren Patronus mehr oder weniger überzeugend. Neville brachte ihn erst beim dritten Mal heraus. Und zum ersten mal sah Harry Nevilles Geier-Patronus.

Da bereits alle Gryffindors ihre Patroni zeigten und somit fertig waren, kamen jetzt die Slytherins an die Reihe. Harry grinste in sich hinein, als die Slytherins ihre Patroni aufzeigen sollten, doch er staunte, dass zumindest Blaise Zabini und Pansy Parkinson  ein Nebelschwaden-ähnliches Etwas zum Vorschein brachten, aber der Rest versagte kläglich beim ersten Mal.

\enquote{Pansy, Blaise, sie können sich zu den anderen stellen. Für heute haben sie ihr Soll mehr als erfüllt. Der Rest muss noch etwas üben.}

So verlief der Rest der Stunde. Die restlichen Slytherin versuchten sich an ihren Patroni und brachten zumindest am Ende der Stunde etwas zustande, was man als Nebel durchgehen lassen würde.

\enquote{Diejenigen unter ihnen, die bisher nichts, oder nur einen schwachen nebelartigen Patronus zustande gebracht haben, werden die nächste Zeit hier noch üben}, sagte Professor Elber.

Zehn Minuten vor Ende der Stunde meinte Professor Elber: \enquote{Die fünf Dementoren werden diesen Samstag zwischen Acht und Acht Uhr zum Üben bereitstehen. Also von morgens bis abends.} Er gab den Dementoren einen Fingerzeig und diese entfernten sich ein paar Meter. \enquote{Sie dürfen zurückgehen, damit sie für die nächste Stunde nicht zu spät kommen.}

\enquote{Professor?}, fragte Ron ihn, als ein großer Teil der Schüler schon gegangen war. \enquote{Werden wir auch lernen mit Patroni Nachrichten zu übermitteln?}

\enquote{Wenn sie möchten, dann machen wir das.}

Die restliche Klasse nickte.

\enquote{Also gut, aber nicht beim nächsten Mal. Wir werden das zu einem späteren Zeitpunkt durchnehmen. In diesem Fall werden sie allerdings alle diesen Samstag nochmals zum Üben kommen, mit echten Dementoren. Sie werden auf ihre Parallelklasse treffen. Also Hufflepuff und Ravenclaw. Sie sollten alle den Stand des gestaltlichen Patronus erreichen.}

Die Klasse verabschiedete sich und Professor Elber drehte sich um. Mit seinem Zauberstab zog er ein einzelnes Insekt an und löste den Rest auf. Das Insekt landete auf seinem Zauberstab und er flüsterte ihm etwas zu. Dann vervielfältigte es sich und je eines der Kreaturen flog auf die mundförmige Öffnung der Dementoren zu. Diese sogen sie auf und verschwanden danach im Inneren des Waldes.

Harry drehte sich wieder um und lief schnellen Schrittes seinen Freunden hinterher.

\enquote{Der wird mir so langsam unheimlich}, meinte Hermine, als Harry sie eingeholt hatte.

Harry und Ron nickten nur stumm.

\enquote{Ich meine, er ist ein guter Lehrer, soweit man das bisher sagen kann, aber trotzdem. Der steht da einfach mit Dementoren und referiert und lässt uns Patroni herauf beschwören. Und die tun auch noch genau das, was er ihnen sagt. Oder besser gesagt, das was er will. Denn er hat mit ihnen ja nicht wirklich geredet}, fuhr sie fort.

\enquote{Aber er hat mit ihnen kommuniziert. Als ihr schon weg wart, da hat er ein einzelnes Insekt auf seinen Zauberstab geholt und ihm etwas zugeflüstert. Dann hat es sich vermehrt und jeder der Dementoren hat eines aufgesogen. Dann sind sie davon geflogen}, meinte Harry.

\enquote{Guten Morgen}, kam es den Schülern von Professor Sprout entgegen. \enquote{Wir werden uns heute um fleischfressende Pflanzen kümmern, die dazu noch giftig für Menschen sind. Einige der Pflanzen, die sie die nächsten vier Unterrichtseinheiten kennenlernen, mögen sie vermutlich für intelligent halten. Sind sie aber nicht. Sie haben heute ausnahmsweise mit den Fünftklässlern Ravenclaw und nicht wie üblich mit den Slytherin zusammen diese Stunde.}

Harry stand neben Luna und lächelte sie an. \enquote{Müsst ihr diesen Samstag auch Patroni üben?}, flüsterte er Luna zu.

\enquote{Nein}, flüsterte sie zurück, während sie mit dem anderen Ohr Professor Sprout zuhörte.

\enquote{Was unterscheidet die drei Arten, Miss Lovegood?}, fragte Professor Sprout.

\enquote{Vier Arten, Professor}, antwortete Luna. \enquote{Sie sehen recht ähnlich aus, riechen aber grundverschieden. Leider ist es extrem schwer, an sie heranzukommen, da sie nach jedem greifen, der sich ihnen nähert.}

Professor Sprout hob eine Augenbraue. \enquote{Fünf Punkte für Ravenclaw. \gst Mister Potter, wie riechen die fünf Arten?}

\enquote{Wie Miss Lovegood soeben sagte, sind es vier Arten. Zwei riechen nach verwesendem Fleisch, um Fliegen anzuziehen. Sie kann man durch die Intensität unterscheiden. Eine Art riecht süßlich und hat zudem ultraviolette Lichtabstrahlung und die vierte Art riecht für uns Menschen gar nicht}, antwortete Harry.

\enquote{Sehr gut. Ebenfalls fünf Punkte für Gryffindor.} Harry war froh, dass er heute Morgen noch kurz in sein Buch geschaut hatte, als er nochmals auf Hermine warten musste. \gedanke{Professor Sprout dachte wohl, wir würden nicht aufpassen}, grinste er in sich hinein und lies Luna an seinen Gedanken teilhaben.

Den Rest der Stunde verbrachten sie damit, den Pflanzen auszuweichen und sie mit ihren Drachen"-haut-Hand"-schuhen umzutopfen. Luna war scheinbar die Einzige, die von ihrer Pflanze nur halbherzig angegriffen wurde. \enquote{Du musst mit ihnen reden}, sagte sie, als sie Harrys fragende Blicke spürte.




\begin{kommentar}
Als Elber Katies Hand wieder nachwachsen ließ, musste er das bei offener Wunde tun. Den Armstumpf hatte er dazu mit dem Sectumsempra-Spruch abgeschnitten, den Snape entwickelt hatte und von dem Harry im sechsten Band der Original-Reihe erfahren hatte. Ich fand es eine passende Gelegenheit, hier ein kleines Detail aus dem Original einzuflechten.
\end{kommentar}

\begin{kommentar}
Elber erzeugt als seinen Patronus viele kleine Insekten. Die Idee kam mir beim Lesen einer anderen Geschichte. Sie zeigt auch sehr schön, dass es nicht nur eine Figur sein muss und deutet schon an, was Elber später zu Harry sagen wird. »Übe mit deinem Patronus.« Er impliziert hierbei, dass man die Gestalt des Patronus doch ändern kann. Man muss nur genug wollen.
\end{kommentar}

\chapter{Erste Annäherung, oder doch nur eine Strafe?}


Harry ging zum Unterricht. Auf dem Weg dorthin kam ihm Pansy entgegen und stellte sich ihm in den Weg. Er hatte keine Chance vorbeizukommen. Also entschied er sich zu einem ungewöhnlichen Schritt. Er schritt auf Pansy zu und bewegte seinen Kopf zu ihrem Ohr. Leider kam in diesem Moment Professor Snape um die Ecke. \enquote{Potter. Nachsitzen wegen Belästigen von Schülern. Melden Sie sich nach dem Unterricht bei Madame Pince.} Sauer trat Harry an Pansy vorbei und setzte sich schwer schnaufend neben Ron, nachdem er das Klassenzimmer betreten hatte. \enquote{Man könnte meinen, dass sich Parkinson und Snape absprechen, um mir eins reinzuwürgen.}

\enquote{Wir werden uns heute dem Euphemos-Trank widmen. Schlagen sie ihr Buch auf und suchen sie nach dem Trank. Dann bereiten sie ihn zu. Die fehlenden Zutaten holen sie aus dem Schrank für allgemeinen Bedarf. Wenn sie mich brauchen, ich bin in meinem Büro.} Harry maß die Flüssigkeiten ab, während Ron die anderen Zutaten, wie Kräuter, klein schnippelte. Snape kam hin und wieder aus seinem Büro, die Tür hatte er immer noch offen, und begutachtete die Tränke der Schüler.

Harry wusste, dass ihr Trank perfekt war, und so füllte er am Ende der Stunde ein Glas ab, um es zur Bewertung abzugeben. Nach dem Essen ging er in die Bibliothek, um sich bei Madame Pince für sein Nachsitzen zu melden.

\enquote{Mister Potter, Professor Snape hat sich bereits bei mir gemeldet. Folgen Sie mir.} Sie schritt durch die Gänge der Bibliothek und inspizierte die Reihen. Am Ziel angekommen, lagen vor ihm Bücher auf den Tischen und auf den Boden.

\enquote{Ihren Zauberstab bitte}, sagte Madame Pince und streckte ihm ihre Hand entgegen.

\enquote{Wie?}, fragte Harry ganz erstaunt.

\enquote{Sie werden die Bücher von Hand sortieren. Dort liegt die Liste.} Sie zeigte auf einen Stapel Pergament.

Harry übergab ihr seinen Zauberstab und blickte wütend auf die Pergamente. Seinen Zorn auf Snape auslebend, sortierte Harry die Bücher in der Bibliothek. Seinen Zauberstab hatte er abgegeben und würde ihn erst wieder zurückerhalte, wenn er mit seiner heutigen Arbeit fertig war. \gedanke{Das wird drei Tage á 4 Stunden dauern}, dachte sich Harry. Also machte er sich an die Arbeit. Er ging die Bücherliste durch und fing an, die ersten Bücher in das Regal zu stellen. Hin und wieder warf er einen Blick in ein Buch, dessen Titel sich interessant anhörte. So fand er unter anderem heraus, wie man aufgrund von Körperflüssigkeiten oder Hautzellen den zugehörigen Träger feststellen konnte. Die Probe musste erst mit einem Zauber behandelt werden, um sie danach in einem Trank aufzulösen. Danach musste wieder ein Zauber angewandt werden, um ein plastisches Bildnis zu erhalten. Die restliche halbe Stunde für heute sortierte er die anderen Bücher. Unter anderem einige, die Schwebezauber behandelten. Harry hatte dieses bereits in seinem ersten Jahr gehabt, es war also vollkommen uninteressant. Doch eines erregte seine Aufmerksamkeit, er würde es sich ausleihen. Nach getaner Arbeit nahm er seinen Zauberstab zurück, lieh das Buch aus und ging in sein Bett, während Madame Pince seine Arbeit noch kurz begutachtete und danach die Bibliothek verschloss.

Hundemüde schleppte er sich zuerst ins Bad und nach getaner Abendtoilette ins Bett. Harry lag etwa eine viertel Stunde im Bett, als er ein kleines \geraeusch{Plopp} hörte. Harry öffnete \gst immer noch liegend \gst einen Vorhang und sah nach draußen. Vor seinem Bett stand ein junger männlicher Elf und sah sich ängstlich um.

\enquote{Wer bist du denn?}, fragte Harry den jungen Elf.

Dieser drehte sich etwas und sah ihn an. \enquote{Frodo kalt}, sagte er.

Harry bemerkte, dass der Elf nur eine kleine Serviette anhatte, die das nötigste verdeckte. Klar, dass ihm kalt war. Und er, Harry, lag in einem warmen Bett. Harry dachte darüber nach, ihm einen warmen Schlafplatz anzubieten. Er überlegte, aber außer seinem Bett hatte er nichts. \gedanke{Das muss einer der Elfen aus der Küche sein}, ging ihm durch den Kopf.

Harry hob seine Decke an und bot dem jungen Elf einen Schlafplatz an, doch dieser sah ihn nur an. Harry wurde es langsam zu viel. Da er in Reichweite stand, rückte er etwas auf seinem Bett herum und griff dann vorsichtig mit seinen Händen nach dem Elf und hob ihn hoch. Dann zog er ihn zu sich und steckte ihn unter die Decke. Beide Köpfe schauten nun unter der Decke hervor. Jetzt schloss Harry wieder den Vorhang und legte sich auf die Seite, den kleinen Elf vor sich mit dem Rücken zu ihm.

\enquote{Frodo warm}, meinte der Elf. Dann sanken seine Ohren auf sein Gesicht und er schlief ein.

Auch Harry versank in einen erholsamen Schlaf. Diesen hatte er bitter notwendig. Doch der Schlaf hielt nicht lange. Er wurde wieder durch einen \geraeusch{Plopp} geweckt. Dann hörte er leise Rufe in einer ihm unbekannten Sprache. Vorsichtig hob er seinen Vorhang an und spähte unten durch. Vor seinem Bett bewegte sich etwas.

Dieses etwas drehte sich und sah ihn aus großen Augen an. \enquote{Verzeihen Sie die Störung. Ich suche jemand. Bitte schlafen Sie weiter}, antwortete der Elf.

Harry hob den Vorhang etwas weiter an und fragte dann: \enquote{Wenn suchen sie Sir?}

Der Elf bekam erst große Augen und meinte dann unsicher: \enquote{Meinen Sohn.}

\enquote{Wie heißt er?}, fragte Harry.

\enquote{Frodo.}

Harry ließ den Vorhang wieder herunter und schob ihn kurz darauf beiseite. Dann zog er seine Bettdecke etwas herunter. \enquote{Ist er das, Sir?}

Der Elfenvater bekam große Augen und nickte nur, da er scheinbar aufgrund des Schocks sprachlos war.

Harry hob seine Bettdecke an und hob ihn vorsichtig aus dem Bett und reichte ihn seinem Vater. Dieser nahm ihn dankbar an. Dann sah er Harry ehrfürchtig an.

\enquote{Danke Mister Potter. Wie hat sich mein Sohn verhalten?}

\enquote{Sie kennen mich?}, fragte Harry nach.

\enquote{Fast alle Elfen kennen sie, Mister Potter.}

Harry nickte. \enquote{Nun, er war sehr brav. Tauchte hier auf und schaute mich an. Dann meinte er, ihm sei kalt. Also habe ich ihm einen warmen Schlafplatz angeboten. Morgen hätte ich ihn in der Küche abgegeben, damit sich jemand seiner Art um ihn kümmern kann, während seine Familie gesucht wird.}

\enquote{Hat er sich gewehrt, ins Bett zu kommen?}

\enquote{Er hat gar nichts gesagt. Nur \enquote{Frodo warm}, als er unter der Decke lag.}

\enquote{Sie haben gut daran getan, ihn bei sich zu haben.} Dann verschwand er mit seinem Jungen auf dem Arm und einem leisen \geraeusch{Plopp}.

Harry war zu Müde, um weiter darüber nachzudenken. Er schloss seinen Vorhang, deckte sich wieder zu und schlief ein. Als er am nächsten Morgen aufstand, hatte dieser Vorfall für ihn keine Bedeutung mehr.

Während des Frühstücks sagte er Ron und Hermine, was er in der Nacht erlebt hatte. Danach ging er mit Ron und Hermine Richtung Wald, um mit den Dementoren zu üben. Kaum hatte er das Schloss verlassen und war auf dem Weg zum Wald, traf er Professor Dumbledore, der mit Professor Flitwick und Professor Sprout vor ihnen her liefen.

\enquote{Guten Morgen, die Professoren}, grüßten die drei.

\enquote{Guten Morgen, Harry, Hermine, Ron}, grüßte Dumbledore und auch die anderen Professoren sagten: \enquote{Guten morgen, Mister Potter. Miss Granger, Mister Weasley.}

\enquote{Wohin geht ihr?}, fragte Dumbledore, der nun zwischen den dreien lief, die anderen Lehrer vor ihnen, aber ihre Ohren nach hinten gerichtet.

\enquote{Richtung Wald, für den Unterricht üben. Es ist kurz nach acht, also wird es für uns Zeit.}

\enquote{Ich halte das für gefährlich. Laut meinen Informationen sollen sich hier Dementoren aufhalten.}

\enquote{Wo?}, fragte Harry scheinheilig nach.

\enquote{In diesem Waldstück}, sagte Dumbledore und zeigte auf den Waldrand, der knappe fünfzehn Meter entfernt war.

\enquote{Du musst deine Bewegung präziser machen}, hörte man aus dem Wald. \enquote{Lass mich mal.} Es folgten einige Sekunden der Stille. \zauber{Expecto Patronum}, sagte eine Stimme, die Harry eindeutig Blaise Zabini zuordnen konnte. \enquote{Siehst du, so geht das. Die Bewegung nicht so unsauber, dann klappt es auch.}

Harry fing an zu grinsen, was Dumbledore nicht entging, da er zu Harry sah und wissen wollte, wie er auf die Nachricht mit den Dementoren reagieren würde. Sie hatten den Rand des Waldes erreicht, als die anderen Lehrer bereits ihre Zauberstäbe zogen.

\enquote{Das wird nicht nötig sein, die sind hier um mit uns zu üben}, warf Hermine ein.

\enquote{Wie, zum Üben hier?}, fragte Professor Flitwick.

\enquote{Na ja, wir sollten heute nochmals mit den Dementoren üben, wenn wir demnächst beigebracht bekommen, wie wir mit Patroni Nachrichten übermitteln wollen.}

\enquote{Sie haben schon mal mit Dementoren\abs?}, fragte Professor Flitwick nach.

\enquote{Sicher}, antwortete Ron. \enquote{Vor ein paar Tagen.}  Dann lief er mit Harry und Hermine in den Wald hinein.

\enquote{Was wollt ihr denn hier?}, fragte Zabini nach.

\enquote{Üben}, antwortete Harry.

\enquote{Aber nicht hier.}

\enquote{Von mir aus. Dann nehmen wir einen Dementoren etwas abseits mit.}

\enquote{Geht bloß weiter.}

Die drei winkten einen Dementoren zu sich her und wanderten ein paar Schritte nach nebenan. Dumbledore, Flitwick und Sprout standen erst einmal perplex da. Die Zauberstäbe noch immer in der Hand und abwartend, um ihre Schüler zu schützen. Doch scheinbar hielten sich die Dementoren zurück und griffen nur an, wenn die Schüler bereit waren, und hörten auf, sobald der Patronus zu schwach wurde, oder abebbte.

Nach einigen Minuten kamen weitere Schüler schwatzend hinzu und begrüßten die Professoren. Sie suchten sich einen freien Dementoren, oder schlossen sich einer Gruppe an und übten ihre Patroni.

\enquote{Solch zahme Dementoren habe ich noch nie gesehen}, meinte Professor Sprout nachdenklich.

\enquote{Aber wie\abs?}, fragte sich Professor Dumbledore.

Nach einer Weile meinte Professor Flitwick: \enquote{Also ich gehe jetzt. Falls noch jemand dableiben möchte. \gst Soll ich jemand etwas mitbringen?}

Gedanken versunken schüttelten die beiden die Köpfe. Nach einer Weile, es waren bereits mehrere Gruppen gekommen und gegangen, kam auch Professor Elber vorbei und schaute dem Treiben zu.

\enquote{Erstaunlich. Diese Dementoren}, meinte Dumbledore.

\enquote{Ja, finde ich auch}, entgegnete Elber. \enquote{Eine neue Sorte?}

Dumbledore sah ihn mit hochgezogener Augenbraue an. Elber lächelte ihn kurz an, drehte sich um und ging dann wieder.

Später, nach dem Essen, schwang das Porträt zurück und ein ziemlich fertiger Harry kam herein.

\enquote{Was ist passiert?}, fragte Hermine erstaunt.

\enquote{Ich bin mausetot. Gestern war es nur geistig, aber heute\abs wie die verrückten haben sie auf mich eingezaubert.}

\enquote{Die?}, fragte sie nach.

\enquote{Professor Flitwick war heute auch dabei. Wir haben alle Kombinationen durchgemacht. Als Letztes haben sich beide gegen mich gestellt und mich ordentlich schwitzen lassen. \gst Ich kann mich keinen Millimeter mehr bewegen.}

\begin{rueckblick}
\enquote{Hallo Harry}, sagte Professor Elber. \enquote{Ich habe heute Professor Flitwick mitgebracht. Zusammen werden wir Duelle üben. Das ist etwas, wovon ihre Gruppe profitieren wird.} Harry nickte. \enquote{Wir beginnen ganz klassisch. Zuerst schauen sie zu. Dann wird sich einer von uns gegen sie stellen und sich mit ihnen alleine duellieren. Dann kommt der andere dazu. \gst Verstanden?}

Harry nickte erneut.

\enquote{Also, Mister Potter}, sagte Professor Flitwick. \enquote{Sehen Sie zu und lernen Sie.}

Harry wusste, das Professor Flitwick ein guter Duellant war. Schon während seiner Schulzeit hatte er fast alle Duelle gewonnen. Er sah den beiden zu, wie sie sich gegenüber aufstellten und verbeugten. Er fühlte sich an sein zweites Schuljahr erinnert, in dem er von Lockhart einmal solche Stunden genossen hatte. Es schüttelte ihn innerlich, als er an diesen Blender dachte.

Beide Kontrahenten umkreisten nun einander. Nacheinander warfen sie sich Zauber entgegen. Mal war es ein blauer, dann wieder ein gelber, oder ein purpurner. Doch Harry wusste, dass er sich von den Farben nicht faszinieren lassen durfte. Er zog vorsichtig seinen Zauberstab und hielt ihn in der Hand. Er war darauf vorbereitet, angegriffen zu werden.

Immer mehr und stärker warfen sie sich die Zauber um die Ohren, bis plötzlich Professor Flitwick einen in Richtung Harry warf. Er konnte ihn gerade noch mit einem Schild-Zauber abblocken und zurückwerfen. Nun musste er sich mit dem kleinen Zauberkünstler duellieren. Dieser war flink, das konnte Harry nicht leugnen. Wie ein Wiesel sprang er umher, machte Saltos in der Luft und landete hinter Harry, nachdem er versucht hatte einen Zauber von oben auf Harry loszulassen.

Harry hatte alle Hände voll zu tun, die Zauber abzuwehren, selber anzugreifen und auszuweichen. Er spürte ein eigenartiges Kribbeln hinter sich und konnte gerade noch zur Seite springen. Aber der Zauber, den ihm Professor Elber gegen seinen Rücken warf, streifte noch seinen Arm. Erst langsam und nur wenig, dann aber immer schneller und heftiger spürte er viele kleine Krabbler auf seinem Körper. Er hatte das Gefühl, dass hunderte von Ameisen unter seiner Kleidung waren. Durch Harrys heftige Reaktion und wildes um sich schlagen, wurden die beiden Professoren nachlässig.

Harry bemerkte dies rechtzeitig und simulierte noch eine Weile, nachdem er sich des Zaubers entledigt hatte. In schneller Folge warf er je einen Klammerzauber auf beide Professoren, die sie mehr oder weniger gut wegsteckten. Professor Flitwick konnte noch einen Arm bewegen und sich aus der Umklammerung befreien und Professor Elber hatte es nur eine Körperhälfte gelähmt.

Als Harry stutze, meinte er: \enquote{Ein kleiner, einfacher Trennzauber, damit man nicht vollkommen getroffen wird. Er hilft nur bei einfachen Flüchen.}

Dann griffen beide wieder an. Harry kam ganz schön ins Schwitzen. Er selber konnte auch einige Treffer landen. Als er ziemlich fertig  aussah, beendete Professor Flitwick die Trainingseinheit. Dann bedankte er sich und verließ mit Harry das Gelände, eine einfache Wiese. Professor Elber machte sich in die andere Richtung davon.
\end{rueckblick}

Langsam fing er an, seine Schuhe mit den Füßen auszuziehen. Tamara saß in einem Sessel ihm gegenüber und sah ihn nur über ihre Zeitung hinweg an. Sein Kopf fiel fast automatisch nach hinten und seine Augen schlossen sich. Doch er war immer noch wach, dachte er.

Als er wieder zu sich kam, spürte er ein Kribbeln an seinen Füßen. Seine Beine wurden gerade durch geknetet und seine Schultern massiert. Ein wohliges Gefühl sickerte durch seinen Körper in seine Gedanken. Dann wurde ihm schwarz vor Augen. Zumindest hatte er das Gefühl, da er seine Augen immer noch geschlossen hielt. Nach wenigen Minuten wurde er wieder klar im Kopf. Er öffnete seine Augen und sah nach oben. Parvati lächelte ihn an. Er lächelte zurück und schloss kurz seine Augen. Dann hob er seinen Kopf und sah Ginny und Hermine, die sich um seine Beine kümmerten. Jetzt war er bei vollem Verstand.

Ruckartig zog er sich zusammen. \enquote{Mädels, hört auf.} Doch noch reagierten sie nicht. \enquote{Mädels bitte}, sagte er aufgewühlt.

Etwas hatte sich verändert, spürte er. Er versuchte sie abzuschütteln. Zeitgleich schlugen die Flammen im Kamin höher. Dadurch aufgeschreckt, ließen sie von ihm ab.

Tamara grinste nur. Harry grinste zurück. \enquote{Ist das bei allen Jungs so?}, fragte sie keck.

\enquote{Was meinst du?}

\enquote{Dass die Mädchen einen massieren, wenn man körperlich fertig ist.}

Er wusste, dass er sich für die folgende Aussage armknuffe einfangen würde, aber das war es ihm wert. \enquote{Bei mir ist es jedenfalls so. Mir kann halt kein Mädchen widerstehen.}

\enquote{Was bist du wieder eingebildet}, antwortete Ginny.

Hermine stand auf, schnappte sich ein Buch und verließ den Raum. Ron folgte ihr kurz darauf.

Und Parvati reagiert, wie es Harry nicht erwartet hätte. Sie beugte sich vorne über und kam seinem Gesicht näher. \enquote{Du siehst auch zum Anbeißen aus.}

Harry schreckte zur Seite und fielt vom Sofa herunter. Parvati begann schallend zu lachen. \enquote{Erst einen auf Macho machen und wenn es ernst wird, den Schwanz einziehen.} Dann ging sie zu Lavender, um mit ihr Hausaufgaben zu machen.

Tamara sah ihn immer noch an. \enquote{Hilfst du mir bei meinen Hausaufgaben? Ich habe da ein Problem.}

\enquote{Nur, wenn ich nichts tun muss, sondern nur denken.} Dann ließ er einfach seinen Oberkörper auf den Teppich sinken und schloss seine Augen. \enquote{Dann frag mich was. Vielleicht kann ich dir helfen.}

\trenn

Harry aß gedankenverloren an einem Apfel, blätterte gerade durch das Buch aus der Bibliothek und schaute immer mal wieder auf den großen See am Fuße Hogwarts. Plötzlich stutze er. Er sah jemanden auf der Wasseroberfläche stehen. Zumindest dachte er das. Das Buch musste warten. Er ging durch die Flure des Schlosses, um der Sache auf den Grund zu gehen.

Kurz vor dem Ausgang des Schlosses traf er auf Ron.

\enquote{Hi Harry. Wohin gehst du?}

\enquote{Ich gehe zum See. Ich habe etwas Komisches gesehen: Jemand der auf dem Wasser läuft.}

\enquote{Du willst mich verarschen!}, sagte Ron.

\enquote{Nein. Ich habe es gesehen. Ich weiß nicht, ob ich mir das nur eingebildet habe oder nicht. Deshalb möchte ich dem nachgehen. Und ein Spaziergang wäre nicht verkehrt.}

\enquote{Ich begleite dich, Harry.}

Und so liefen Harry und Ron den schmalen Weg zum See hinunter.

\enquote{Hey, das sieht gerade so aus, als ob jemand auf dem Wasser steht und sich bewegt.}

\enquote{Genau das habe ich gemeint.}

Noch immer waren sie nicht nahe genug, um etwas zu sehen. Erst als sie vor dem See standen, konnten sie die Gestalt durch ein Loch im umgebenden Buschwerk erkennen. Sie stand tatsächlich auf dem Wasser. Der Kleidung nach war es ihr Lehrer in VgddK. Er hielt in einer Hand einen langen Stab, an dessen einem Ende eine Art schmale Schaufel befestigt war und an dessen anderem Ende eine Art Elektroschocker war. Zumindest sah es so aus. Er bewegte sich wie ein Martial Arts-Kämpfer mit seinem Stab und schien jemanden anzugreifen. Ron und Harry konnten aber aufgrund der Wasserpflanzen den angegriffenen nicht sehen, aber durch einen schmalen Spalt in der Wasserbepflanzung, mit wem er kämpfte. Plötzlich bewegten sich die Wasserpflanzen und bogen ihre Triebe zur Seite um Platz zu machen. Erstaunt sah Harry zu Ron in der Erwartung, dass er ebenso erstaunt sei. Doch Ron grinste leicht und steckte seinen Zauberstab wieder ein.

Jetzt hatten sie eine bessere Sicht. Beide waren erstaunt, dass sich ihr Professor scheinbar mit Parvati und Padma Patil duellierte. Beide hatten ebenfalls so einen Stab. Professor Elber schleuderte mit der Schaufel immer wieder eine Menge Wasser auf die beiden Zwillinge. Durch das Wasser leicht abgelenkt konnte er mit einer Vorwärtsbewegung seiner Hand, die beiden zurückdrängten. Es war so, als ob sie ein Luftstoß zurück blies. Plötzlich kam von der Rückseite ein weiterer Wasserstoß auf sie zu. Professor Snape kam ins Bild und griff nun ebenfalls an. Harry und Ron hatten ihn vorher nicht gesehen. Umso überraschter waren sie, dass sie jetzt zu viert kämpften.

Verwirrt schauten sich die beiden kurz an und gingen dann zum Gegenangriff über. Abwechselnd schleuderten sie eine Menge Wasser auf ihre beiden Angreifer. Professor Elber und Professor Snape mussten die Wassermassen mit ihren Schaufeln abwehren. Während dessen schleuderten die beiden Patil-Zwillinge Stromstöße von der anderen Seite der Stäbe. Professor Elber warf es zurück und er kam mit dem Rücken zuerst auf der Wasseroberfläche auf. Nach wenigen Sekunden, in denen die Zwillinge einen zufriedenen Eindruck machten, sank er in den See ein. Es dauerte eine Weile bis den beiden mulmig wurde und sie sich aufmachten ihn zu suchen. Doch er kam hinter ihnen wieder hoch und schleuderte ihnen je einen Stromstoß in den Rücken. Jetzt standen beide Angreifer vor ihnen. Nach ein paar Sekunden schickte er wieder ein paar Stöße, die die Zwillinge aber gekonnt mit der Spitze ihrer Kampfstäbe abwehrten. Professor Snape schickte mit einer Drehung eine Menge an Wasser hinterher, was die beiden Zwillinge mit einem Schild-Zauber parierten.

Professor Elber stieß mit seinem Stab kurz auf der Wasseroberfläche auf, wodurch sich eine kreisförmige Welle ausbreitete und vom Schild-Zauber abgehalten wurde. Kurz danach begann sich das Innere des Schildes von unten her mit Wasser zu füllen. Unaufhaltsam stieg das Wasser im Inneren des Schildes hoch. Professor Elber richtete die spratzelnde Seite seines Kampfstabes auf die beiden Zwillinge. Er schien nur darauf zu warten, dass sie den Zauber fallen ließen, weil sie Angst hatten zu ertrinken. Als es nicht mehr ging, warfen die Patil-Zwillinge ihre Stäbe nach außen und sofort begann der Wasserstand sich innerhalb des Schildes zu verringern, bis er die Seeoberfläche wieder erreicht hatte.

Der Schild löste sich auf und die beiden Zwillinge hoben ihre Stäbe wieder auf. Danach gingen alle zum Rand des Sees, wo Harry und Ron standen.

\enquote{Was war das denn?}, fragte Ron.

\enquote{Magi-Fu}, antwortete Professor Elber. \enquote{Eine spezielle magische Kampfsportart. Ich habe sie entwickelt, als ich einige Zeit in Asien unterwegs war und mich mit den dortigen Kampfsportarten auseinandergesetzt habe.}

\enquote{Können wir das auch lernen?}, fragte Ron begeistert.

\enquote{Wir?} Harry war entsetzt.

\enquote{Nur, wenn sie sich mit asiatischen Kampfsportarten auskennen. Die beiden Zwillinge hier beherrschen sie, deswegen habe ich auch angefangen sie zu unterrichten.} Und er fügte leise für Ron und Harry hinzu. \enquote{Und um Übungspartner zu haben.}

Dann schaute Harry fragend zu Professor Snape. \enquote{Mein Vater war begeisterter Martial-Arts Kämpfer}, war das Einzige, was Professor Snape sagte, bevor er an ihnen vorbeiging.

\trenn

Schweigend übergab Harry am Abend seinen Zauberstab der geierartigen Hexe in der Bibliothek und ging wortlos zu seinem Regal. Wieder sortierte er die Bücher ein. Und wieder fand er interessante Bücher, die er durchblätterte und einige Sachen fand, die er später einmal durchlesen wollte. Nachdem er für heute seine Strafe abgearbeitet hatte, nahm er eines der Bücher zu Madame Pince mit vor, um es sich auszuleihen. Als sie den Titel sah, zog sie nur eine Augenbraue hoch, sagte aber nichts.

Nach dem Essen nahm Harry Ron und Hermine mit in ein leeres Klassenzimmer. Auf dem Weg dorthin hörten sie mehrmals etwas dumpf aufschlagen und nach jedem dritten Mal eine schnelle, hohe, aber melodiöse Tonfolge. Die drei folgten den Geräuschen. Sie schlichen sich an eine Ecke ran, da sie den Geräuschen immer näher kamen. In einem kleinen Gang, der in einer Sackgasse endete und nur etwa fünf Meter lang war, standen der Schulleiter und ein Lehrer und warfen Pfeile auf eine Scheibe mit farbigen Ringen und einem Metallgitter darüber, welche auf Augenhöhe angebracht war. In etwa zwei Metern Höhe waren hölzerne Plaketten angebracht, auf denen der Name des Spielers und darunter die Punkte standen. Harry las: \accentuate{Dumbledore} und \accentuate{Elber}.

Auf Dumbledores Schild waren 140 Punkte und auf dem Schild von Elber 150 Punkte.

\gedanke{Anscheinend führt Elber}, dachte Harry.

Dumbledore warf einen Pfeil, traf und die Anzeige zog sofort die Punkte ab. Jetzt standen nur noch 60 Punkte auf Dumbledores Schild.

\gedanke{Ups, die werden abgezogen}, dachte sich Harry und sah kurz zu Ron und Hermine, welche gebannt zusahen.

Dumbledore warf die restlichen Pfeile, worauf noch zehn Punkte übrig blieben. Er ging nach vorne und zog die Pfeile aus der runden Scheibe und ging dann zurück hinter die gemalte Linie.

Dann war Elber dran. Er warf eine 100, und zweimal eine 25. Dann standen auf seinem Schild Null Punkte und er gewann.

Er ging nach vorne und zog seine Pfeile heraus. Als er wieder hinter der Linie stand, begann eine neue Runde und Dumbledore begann zu werfen. Währenddessen entwickelte sich eine Unterhaltung zwischen den beiden.

\enquote{Sagen Sie mal Frederick\abs Ich habe gehört, dass sie die Schüler auch teilweise in die Geheimnisse der dunklen Künste einweihen.}

\geraeusch{Klonk}, machte es und ein Pfeil von Professor Elber flog daneben. Leicht erregt und mit zusammen gepressten Zähnen sagte er: \enquote{Es gibt keine dunkle Magie. Es gibt nur die Intention dessen, der sie anwendet \gst Herrschaft.}

\enquote{Ich habe ein ungutes Gefühl dabei.}

\enquote{Wieso? Was ist daran schlimm? Man muss wissen, womit man es zu tun hat. Die Schüler sollen sich wehren können.} Er warf eine 80, danach eine 320.

\enquote{Die Schüler haben Angst vor ihnen.} Es klingelte und für Dumbledore wurden 240 Punkte abgezogen.

\enquote{Guter Wurf, Albus.}

\enquote{Danke.}

\enquote{Wieso haben die Schüler Angst vor mir?}, fragte Professor Elber weiter.

\enquote{Weil sie die dunklen\abs Sie wissen, was ich meine. Sie bringen ihnen Magie bei, die Voldemort und seine Todesser verwenden. Viele glauben, sie sind ein Todesser. Das glauben auch viele meiner Kollegen.}

\geraeusch{Klonk}. Wieder ging ein Wurf daneben. Zornig warf er beide Pfeile auf die Scheibe und die Anzeige verlöschte und es gab, erschien: \accentuate{ungültiger Wurf}, samt passenden Geräuschen.

Tief und langsam durchatmend sah er Dumbledore an. \enquote{Ich werde mal mit den Kollegen reden müssen. Vielleicht war ich nicht deutlich genug. \gst Aber dass die Schüler es nicht begriffen haben? Ich habe ihnen doch in der ersten Stunde alles erklärt.}

\enquote{Offenbar nicht deutlich genug. Aber mir wäre es lieber, wenn sie nochmal vor allen\abs}

\enquote{Vor der gesamten Schule? Sie wissen doch, wie ungern ich das mache.}

\enquote{Das sah letztes Mal aber anders aus.}

\enquote{Da musste ich. Und dann soll es nicht aussehen, als hätte ich Angst.}

\enquote{Dann los, bald ist Abendessen.}

\enquote{Heute? Ausgeschlossen, dienstags habe ich immer einen wichtigen Termin nach dem Essen. Was glaubst du, warum ich immer so schnell weg bin? Morgen, Albus.}

\enquote{Also gut.}

Dumbledore kam an die Wand nach vorne und steckte seine Pfeile in den kleinen Vorsprung, während Professor Elber sie aus der Scheibe zog und ebenfalls ablegte. Dann drehten sich beide herum und kamen aus dem Gang. Hermine, Harry und Ron schreckten zurück und liefen dann normal nach vorne. Die Scheibe nicht beachtend grüßten sie ihre Lehrer und gingen normal weiter.

\enquote{Haben Sie bemerkt, dass sie uns belauscht haben?}, hörte Harry Dumbledore sagen. Dann waren sie zu weit weg, um noch etwas zu verstehen.

Nachdenklich gingen die drei in den Gemeinschaftsraum, wo sie darüber sinnierten. Etwas später holte Harry sein Buch und begann endlich zu lesen.

\begin{buch}
Neben vielen anderen Orten, wo Wissen gesammelt wird, gibt es einen, der das magische Wissen unserer Gesellschaft enthält. Die Mondbibliothek. In diesem Werk habe ich alle Informationen gesammelt, die ich finden und herausfinden konnte. Leider sieht man schon an der Dicke des Buches, dass es nicht allzu viele Informationen sind, die ich habe. Selbst den Namen herauszufinden hatte mich eine Weile beschäftigt. Bedauerlicherweise konnte, oder wollte, mir keiner sagen, wo sich diese geheimnisvolle Bibliothek befindet. Wenn ich nur an alle die Bücher, sofern es welche gibt, denke, dann läuft mir schon das Wasser\abs aber ich schweife ab. Ich habe ein Rätsel gefunden, dass ich bislang nicht lösen konnte. Es lautet: \enquote{Um dorthin zu gelangen, nimm den direkten Weg. Wähle den richtigen Zeitpunkt und du wirst dein Ziel erreichen. Übe das Reisen ohne Zeit und du wirst erkennen, wann und wohin du musst. Das Wissen wartet auf dich und wird ständig wachsen. Ein Leben reicht nicht aus, um alle Geheimnisse zu ergründen. Aber hüte dich vor dem Wächter. Wenn du ihn gegen dich hast, hilft auch keine Flucht mehr. Dann hilft nur noch beten und der Übergang in das Danach.}
\end{buch}

Auf den nachfolgenden Seiten fand Harry denselben Text in mehreren Sprachen. Es waren auch Bilder vorhanden. Die Bibliothek, wie sie sich der Autor wohl vorstellt. Es waren Kupferstiche. Einer von vielen Gängen mit einer endlos scheinenden Buchreihe war zu sehen, denn es schien, als laufe man durch die Bücherreihen hindurch. Ein anderes Bild zeigte eine leuchtende Kugel inmitten eines Raumes.

Dann las er noch unterhalb des Bildes:

\begin{buch}
Nur, wer den richtigen Blick auf die Bibliothek hat, wird sie finden. Das Wissen zu erlangen kann mühselig oder leicht sein. Suche nicht, finde sie.
\end{buch}

Da Hermine immer wieder sehr interessiert schaute, was Harry las, gab er ihr das Buch, nachdem er fertig war.

Nun dachte er nach. Er schloss die Augen und merkte nicht, wie sich jemand neben ihn setzte. Irgendwann spürte er nur eine Berührung. \gedanke{Mondbibliothek. Vermutlich wegen der hellen Gänge. Das Holz der Regale in den Reihen sah auf dem Bild sehr hell aus. Nein, das war vielleicht nur die Vorstellung des Künstlers. Mondbibliothek. Der Mond ist unser Trabant. Etwa vierhundert-tausend Kilometer entfernt. Er bremst die Erde. Er sorgt für Ebbe und Flut. Er verdunkelt alle wie viele Jahre die Sonne? Der richtige Zeitpunkt. Eine Sonnenfinsternis? Warum Mond?}

Er öffnete die Augen und sah aus dem Augenwinkel heraus jemand neben ihm sitzen. Es war Ginny. Sie sah ihn an und lächelte. Dann wandte sie ihren Blick ab. Harry nutze die Gelegenheit und gab ihr einen Kuss auf die Wange. Überrascht sah sie ihn an.  \enquote{Danke}, war das einzige, was Harry sagte.

\enquote{Wofür?}, fragte sie nach.

\enquote{Denk nach} und an Hermine gewandt: \enquote{Bist du fertig mit dem Buch?}

Diese nickte und gab ihm das Buch zurück. Harry kopierte die Seite und die Bilder auf beide Seiten eines Pergaments und schob es dann in seine Tasche.

\enquote{Ich gehe dann mal nach oben}, sagte er und verschwand.

\enquote{Für das Geburtstagsgeschenk?}, rief ihm Ginny noch nach.

\enquote{Nein.}

Auf seinem Weg nach oben grinste er in sich hinein. \gedanke{Mal sehen, wann Ginny darauf kommt.}

Nachdem er sich schlafen gelegt hatte, begann er zu träumen. Er träumte von letztem Jahr, als er bei Umbridge gerade Nachsitzen hatte.

\begin{traum}
In ihrem hässlichen Rosa stand sie da und starrte aus dem Fenster, während sich Harry die Hand wund schrieb. Plötzlich kam Hagrid mit seinem rosa Schirm herein. Harrys Blick fiel auf ihn. \gedanke{Rosa}, dachte er.
\end{traum}

Dann schreckte er hoch. Irgendwo musste er sich das ganze notieren, damit er es nicht vergaß. Sein neues Tagebuch-Set kam ihm gerade recht. Er führte zwar kein Tagebuch, aber als Notizbuch wusste er es zu schätzen. Interessante oder auch komische Träume hielt er fest. Er schrieb auch auf, was Hagrid sagte. \gedanke{Manchmal muss man seine Träume auch wörtlich nehmen. Hagrids Rosa Schirm. Er hat eine Bedeutung. Wenn mir nur einfallen würde, welche!} Dieser Traum war vollkommen daneben, aber in einem Traum ist alles möglich. Er kann durchaus eine Bedeutung haben.

Er schlief wieder ein und erwachte erst am nächsten Morgen. Er ging seine Notizen noch einmal durch und überlegte, was es mit dem rosa Schirm auf sich hatte. \gedanke{Hagrid hat einen. Darin hat er seinen Zauberstab. Er hat ihn geklebt. Illegal.} Das war es. Sein Unterbewusstsein wollte ihm sagen, dass er Hagrid helfen sollte. Aber wie? Der einzige Beweis, den er je hatte, war Voldemorts Tagebuch. Selbst, wenn es noch intakt wäre; Voldemorts jüngeres selbst würde keinem anderen zeigen, was passiert war. Wie also Hagrid helfen?

Doch er wurde von Ron unterbrochen und ging mit ihm, um zu frühstücken.

\trenn

Harry befand sich im Krankenflügel der Schule, nachdem ihn aufgrund seiner Unaufmerksamkeit eine Pflanze gebissen hatte. Seine linke Hand bis hinauf zu seiner Beuge schwoll auf das Doppelte an.

\enquote{Es ist doch jedes Jahr dasselbe. Immer sind ein oder zwei dabei, die es schaffen sich beißen zu lassen}, sagte Madame Pomfrey die Augen rollend.

\enquote{Es ist nicht so, dass ich darauf gewartet habe oder mich absichtlich habe beißen lassen, Madame Pomfrey}, antwortete Harry.

Er hatte nun das erste Mal das Gefühl, dass sie lächelte. Sonst sah er sie immer nur sehr ernst. Sie verschwand kurz in ihrem Büro. Währenddessen setzte sich Harry auf ein freies Bett und entfernte den Rest seiner schon reißenden, spannenden Kleidung, da sein Arm geschwollen war, mit seinem Zauberstab.

Zu allem Überfluss hatten sie heute auch noch Heilkunde, zusammen mit den Ravenclaws. Die Glocke läutete erneut und somit begann der Unterricht. Die Klasse war vollständig versammelt, als Madame Pomfrey wieder hereinkam. \enquote{Ah ja}, sagte sie. \enquote{Gut gut.} Sie lief zu Harry und nahm ihn gleich als Demonstrationsobjekt her.

\enquote{Kommen Sie ruhig näher. Als Heiler dürfen sie keine Angst vor dem Patienten haben. In diesem Fall ist es nur eine simple Schwellung einer Todarm-Wurzel. Wenn sie nicht rechtzeitig behandelt wird, dann stirb der Arm ab. Aber das dauert. Sie können mit bis zu einem Monat Verzögerung die Heilsalbe aufbringen. Nun gut, wenn sie bis dahin die ständig wachsenden Schmerzen aushalten.} Sie öffnete die Salbendose und nahm einen ordentlichen Schwung heraus. Dann strich sie Harrys Hand damit ein.

Harry war überrascht, dass die Salbe nicht kalt war. Aber trotzdem lief ihm ein Schauer über den Rücken. Er genoss die Berührung von Madame Pomfrey. Und das irritierte ihn.

\enquote{Wer will auch mal?}, fragte sie in die Runde. Harry riss seine Augen auf. Eine einzelne Ravenclaw trat hervor und Madame Pomfrey reichte ihr die Salbe.

\enquote{Nehmen sie nur eine ordentliche Portion auf ihre Hand und streichen sie damit den Arm ein. Bedecken Sie ihn richtig damit. Es sollte schon ein halber Millimeter sein. Das zieht recht schnell in die Haut ein. Sehen Sie Miss Elfwood, die Hand ist schon wieder frei von der Salbe.}
% Linda Elfwood

Sie nickte und nahm eine ordentliche Portion heraus. Dann verteilte sie die Salbe auf Harrys Arm. \enquote{Etwas weniger}, korrigierte sie Madame Pomfrey. Sie verteilte die Menge, die sie genommen hatte nun über den ganzen Arm, anstatt noch einmal nachzufassen. Binnen zehn Minuten war die Schwellung komplett verschwunden.

Dann dachte Harry, er höre nicht richtig, als Madame Pomfrey sagte: \enquote{Und wenn sie zu ihrem Patienten besonders nett sind und ihn auf die Wange küssen, dann wirkt es noch schneller.} Sie tat wie geheißen und küsste Harry auf die Wange. Danach drehte sie sich zu Madame Pomfrey um, um ihre Meinung zu hören, ob es so richtig gewesen sei.

Madame Pomfrey zog ihre Augenbraue hoch und sagte mit gewisser Heiterkeit. \enquote{Das war eigentlich als Scherz gedacht. Ich hatte nicht erwartet, dass sie das auch wirklich tun.} Die Schülerin, Linda hieß sie, wie ihm Luna mal gesagt hatte, wurde sofort rot.

Madame Pomfrey fuhr fort. \enquote{Es gibt nur zwei mir bekannte Verletzungen, deren Heilung durch einen Kuss beschleunigt werden, aber diese gehört nachweislich nicht dazu. \gst Trösten Sie sich Miss Elfwood. Mit mir hat man während meiner Ausbildung denselben Scherz getrieben.} Dann wandte sie sich zur Klasse und meinte: \enquote{Lassen sie sich also nicht veralbern, falls sie einmal den Beruf des Heilers antreten wollen.}

\trenn

Harry hatte seine zweite Strafarbeit hinter sich und machte sich auf den Weg zum dritten Stock im Westflügel. Er klopfte an das Porträt und Luna ließ ihn kurz darauf herein.

\enquote{Wie bist du\abs?}

\enquote{Shh!}, gab Luna zurück und küsste ihn. Dann zog sie ihn zu sich und nahm ihn mit in ihren Raum. Nach einem gemütlichen Gespräch im  Bett, begannen sich beide umzudrehen und einzuschlafen.

Mitten in der Nacht wachte Harry auf und hob seinen Kopf. Am Bettende sah er einen kleinen Elfen herumlaufen und aufräumen. Harry schaute ihm eine kleine Weile zu und erkannte Kreacher. Er legte seinen Kopf wieder auf das Kissen, schloss die Augen und horchte in die Nacht hinein. \gedanke{Kreacher scheint mich doch irgendwie zu mögen, sonst würde er hier nicht aufräumen.} Als Kreacher endlich fertig war, stand Harry auf um auf die Toilette zu gehen. Als er sich wieder ins Bett legte und zu Luna drehte, drehte sie sich ebenfalls in seine Richtung. Ihr Atem war ruhig und gleichmäßig. Sie schien zu schlafen. Vorsichtig reckte er seinen Kopf nach vorne und gab ihr einen Kuss auf die Stirn. Sie ließ einen wohligen Schnaufer erklingen und kam ihm entgegen. Er lächelte leicht und gab ihr nun einen richtigen Kuss. Wieder gab sie ein wohliges Geräusch von sich. Harry grinste und drehte sich herum. Langsam tastete sie sich, scheinbar immer noch schlafend, an ihn heran und legte einen Arm um seine Hüfte. Ihr warmer Atem ließ seine Nackenhärchen aufstehen. In ihm breitete sich ein langsamer erholsamer Schlaf aus.

Am Morgen nach seiner vierten gemeinsamen Nacht mit Luna, welche ähnlich ablief wie die davor, hatte es Harry geschafft, sich noch ein bisschen hinzulegen, bevor es zum Frühstück ging und so eine weitere Unterredung mit Hermine oder Ron vermieden. Abermals in der Großen Halle angekommen, traf sein Blick den von Luna und er wusste genau was sie dachte. Mitten während des Frühstückes war es wieder so weit und die Posteulen kamen. Harry schaute zu ihnen hinauf und ihm fielen zwei seltsam klein wirkende, gänzlich grau aussehende Eulen auf. Es waren Schuleulen, die sonst sehr selten Post brachten. Eine flog auf Harry zu und landete beinahe auf seinem Teller. Sie hatte einen Brief in ihrem Schnabel und auf dem ungewöhnlich grau aussehenden Umschlag stand in krakeliger Handschrift:

\begin{brief}
Harry Potter,\\
Gryffindor\\
Große Halle
\end{brief}

Harry nahm der Eule den Brief ab, gab ihr etwas Schinken vom Frühstückstisch und ließ sie wieder fliegen. Er drehte den Brief um, um nach dem Absender zu schauen. \accentuate{Dobby} stand dort. Harry drehte sich um und schaute zu Luna, die scheinbar auch einen Brief bekommen hatte. Er steckte den Brief in seine Tasche und frühstückte weiter. Die Fragen seiner Banknachbarn ignorierte er und verschwand nach dem Essen. Er erzählte Ron und Hermine, dass er noch etwas Quidditch-Training benötige und danach im Gemeinschaftsraum seine Hausaufgaben machen wollte. Luna wusste bereits, was er vorhatte und stand auf, um in den dritten Stock zu gehen. Dort traf sie auch schon auf Harry, den sie die letzten Schritte begleitete. Im Gemeinschaftsraum der Paare öffneten Harry und Luna ihre Briefe und Harry las seinen vor.

\begin{brief}
Hallo Harry Potter, Sir.

Dobby freut sich ihnen mitzuteilen, dass er nunmehr bereit ist, mit den anderen Hauselfen die anderen Paare in den fünften Gemeinschaftsraum zu lassen. Entsprechende Briefe, sie hier nächsten Sonntag zu treffen, werden morgen rausgehen. Bitte weisen Sie und Miss Luna die anderen Paare ein, so wie Dobby es mit ihnen gemacht hat.
\end{brief}

Luna las ihren Brief mit und sagte Harry, dass bei ihr dasselbe drinnen stand. Harry blickte auf, lächelte Luna leicht gequält an, worauf sie aufstand, sich in seinen Schoß setzten und ihm zu spüren gab, dass es schon klappen würde. Harry dachte immer wieder daran, dass er ihre Gedanken lesen konnte, wenn er sie nicht unterdrückte und bemerkte seit einigen Tagen, dass er, wenn er sich besonders anstrengte, auch durch ihre Augen sehen und durch ihre Ohren hören konnte. Luna ging es nicht anders und sie fragte Harry, ob die nächste Partie Schach nicht sie gegen Ron spielen dürfte. Das vertrieb die sorgenvolle Miene von Harrys Gesicht und er gab ihr einen Kuss, den sie erwiderte. Harry ging wie Luna  mit gemischten Gefühlen dem Treffen entgegen, trafen sie dort doch Leute, die in ihren eigenen Häusern waren. Doch die Tatsache, dass dieser Raum ihnen allen eine Zufluchtsstätte bot, gab ihm Trost, sie den Fängen von Mister Filch und den anderen patrouillierenden Lehrern zu entreißen, gab ihm Mut das Ganze durchzustehen. Nach den üblichen Sticheleien Malfoys den Rest der Woche, dem langweiligen Unterricht von Professor Binns und dem übermäßigen Enthusiasmus von Professor Flitwick, kam so langsam aber sicher der Freitag.

Während der Stunde bei Professor Trelawney, erinnerte er sich wie Dean einen grauen Brief von Dobby bekommen hatte. Dean öffnete ihn, stutzte und zeigte mit seinem Zauberstab, seinen Namen nennend, auf das Stückchen Papier. Er schaute erstaunt, als er Dobbys Zeilen las, schaute zu Harry, grinste frech und tippte seinen Brief wieder an, welcher sich begann aufzulösen und mit einem lauten \geraeusch{Plopp} verschwand. Harry hörte an diesem Tag mehrere Plopp-Geräusche, die das übliche Grundrauschen während des Frühstücks übertönten. Harry war gar nicht wohl gewesen, als Dean seinen Brief bekommen hatte. Bekam aber merkwürdigerweise nichts zu hören. Keiner schien ihn darauf anzusprechen, oder schief anzuschauen. Harry fiel erst jetzt auf, dass er keine Ahnung hatte, was Dobby in den Briefen erwähnte und ob er überhaupt seinen oder Lunas Namen nannte. Plötzlich stieß ihn Ron mit seinem Ellenbogen und Harry war gerade wieder bei der Sache, als Professor Trelawney vor ihnen erschien und Harry aufforderte, Ron die Tarot-Karten zu legen. Er ließ Ron die Karten mischen und begann auszuteilen. Er murmelte irgendwas von einer Veränderung in der Liebe und dem Umstand, dass er bald Vater werden würde. Professor Trelawney beglückwünschte ihn auf eine Art und Weise, dass es die ganze Klasse mitbekam. Ron verzog sein Gesicht und blickte Harry böse an. So als ob es seine Schuld gewesen sei.

Die Stunde nach dem Mittagessen bei Professor Snape war auch nicht gerade die Beste, obwohl sich Harry bei seinen Hausaufgaben zu bessern schien. Snape behandelte ihn immer noch äußerst ungerecht und zog ihm bei jeder Kleinigkeit Punkte ab, die er aber scheinbar im Mittel immer wieder durch seine Hausaufgaben ausgleichen konnte. Harry strengte sich zwar etwas mehr an als sonst, aber dass sich das dermaßen auf seine Noten bei Snape auswirken würde hatte er nicht gedacht. Vielleicht liegt es auch nur daran, was er alles durchgemacht hatte, während der letzten Schuljahre (Harry, nicht Snape). Er hatte gegen Voldemort im ersten Schuljahr gekämpft. Im zweiten einen Teil von Voldemorts Seele und einen Basilisken zur Strecke gebracht. Im dritten seinem Paten zur Flucht verholfen. In seinem vierten den Trimagischen~Pokal gewonnen und war knapp Voldemort entkommen und im fünften seinen Paten verloren. Ron weckte ihn wieder aus seinen Gedanken. Gerade noch rechtzeitig, damit Harry seinem Trank noch ein paar Kräuter hinzufügen konnte. Am Ende der Stunde füllte er eine Probe seines Trankes in ein Glas und gab es Snape zur Überprüfung.

\enquote{Potter}, hörte er Snape sagen, als er sich gerade umgedreht hatte. \enquote{Sie haben vergessen ihren Namen draufzuschreiben. Sie haben drei Sekunden Zeit.}

Harry wusste genau, dass er es nicht schaffen würde, seinen Namen auf ein Blatt Papier zu schreiben, um es mit einem Stückchen Faden an sein Glas zu binden. Er drehte sich blitzschnell um, so schnell, dass Snape erschrak, zeigte mit seinem Zauberstab auf das Glas und sprach:

\enquote{Gravure, Harry Potter.}

Er schaute Snape in die Augen während er seinen Zauberstab einsteckte. Dann drehte er sich um und verließ den Raum. Snape nahm sich das Glas und schaute es missmutig an. In das Glas waren die Worte \accentuate{Harry Potter} eingraviert und Snape rief ihm wütend nach: \enquote{Fünf Punkte Abzug für Gryffindor wegen Beschädigen von Schuleigentums.} Harry schmunzelte im Hinausgehen Ron und Hermine zu, denn er war sich sicher, dass er die Punkte mit seinen Hausaufgaben wieder ausgleichen konnte, Snape vielleicht sogar beeindruckt hatte. Er hatte ihn verblüfft! Dabei hatte er den Zauberspruch nur zufällig gefunden, als er in der Bibliothek lustlos in einem Buch blätterte.

In der Zwischenzeit hatte er bereits Professor Sprout und Professor Snape nach einer Erlaubnis für die abgesperrte Abteilung gefragt, war aber immer abgeblitzt. Sogar Professor Dumbledore hielt nichts davon.

\trenn

Als Harry im Gemeinschaftsraum der Gryffindors ankam, war dieser leer, wie ausgestorben. Harry legte seinen Besen vom Quidditch-Training unter sein Bett und ihm kam ein Einfall. Er öffnete seinen Koffer und suchte nach der Karte des Rumtreibers. Als er sie gefunden hatte, ging er zu seiner Tür, verschloss sie mit einem Zauberspruch und setzte sich auf sein Bett. Er tippte die Karte an und sprach: \enquote{Ich schwöre feierlich, ich bin ein Tunichtgut.} Die Karte begann Hogwarts zu zeigen und Harry klappte sie auf. Er suchte nach dem dritten Stock im Westflügel, doch da war nichts. Plötzlich fiel Harry an der Stelle wo das Porträt an der Wand hing, ein kleiner Punkt auf, der sich kaum merkbar von der Karte abhob. Harry stutzte und versuchte ihm mit seinem Zauberstab anzutippen, doch der Punkt färbte sich kurz rot, nur um dann wieder seine ursprüngliche Farbe anzunehmen. Harry überlegte kurz, tippte den Punkt abermals an und sprach dann: \zauber{Aqua Neros!}

Er sah, wie sich aus der Karte ein weiterer Abschnitt erhob und den Gemeinschaftsraum der Paare abbildete. Er sah kleine Fußspuren darin umherlaufen. Kleinere als sonst. Und eine Schrift daneben. Es waren Dobby und Kreacher. \gedanke{Anscheinend räumen die beiden gerade auf.} Da fiel ihm ein, dass er noch nie die Küche auf der Karte gesehen hatte. Er tippte wieder auf das neue Kartenstück, welches verschwand und schaute an die Stelle, wo die Küche war. Da war auch ein kleiner Punkt, den Harry sofort antippte. Das Porträt, welches sonst den Zugang zur Küche versperrte, wurde auf einem weiteren Kartenstück sichtbar. Harry rieb an der Stelle der Birne auf dem Papier herum und die Karte verformte sich. Es erschien ein genauer Umriss der Küche und Harry sah viele kleine Füße umherlaufen. Neben jeder Fußspur ein kleiner Name. Viele davon kannte er nicht aber der Name Winky fiel ihm auf. Harry tippte wieder auf die Karte und das Papier, welches die Küche darstellte, verschwand wieder. Es war so, also wurde das Papier von der Karte aufgesogen. Jetzt wurde Harrys Neugier geweckt und er suchte auf der Karte nach weiteren Punkten. Als er einen fand, nahm er seine Karte von Hogwarts \gst er musste immer noch schmunzelnd, als er daran dachte, wie er sie bekommen hatte \gst und verzeichnete den Punkt. So verfuhr er bei jedem Punkt, den er fand. Sorgfältig trug er sie ein. Ebenso schrieb er die Positionen der Punkte auf ein Pergament, denn die Karte konnte er nicht immer mitnehmen.
Als er alle gefunden hatte, war die Freistunde auch schon wieder vorbei. Harry tippte die Karte mit seinem Zauberstab an und sagte: \enquote{Missetat begangen.}

\begin{rueckblick}
Harry bemerkte, dass er seine Karte des Rumtreibers verloren hatte. Sofort drehte er sich um und ging den Weg zurück. Als er in seinem Klassenzimmer ankam, hatte die Professor Elber bereits in der Hand und saß hinter seinem Schreibpult im Klassenzimmer.

Mit leicht zusammengekniffenen Augen und skeptischem Blick fuhr er mit einer Hand über das dicke Pergament. Dabei murmelte er immer wieder. \enquote{Interessante Schutzzauber\abs keine herkömmlichen\abs was ist das für\abs? Komisch\abs}

Langsam näherte sich Harry seinem Lehrer.

Dieser sah auf und sah ihn fragend an. \enquote{Ja bitte?}

Harry druckste etwas herum und sah immer wieder verstohlen auf das Pergament.

Darauf hin wechselte Professor Elbers Blick ebenfalls zwischen Harry und dem Pergament hin und her. \enquote{Ihres?}, fragte er.

Harry nickte.

\enquote{Beweisen Sie es}, forderte Professor Elber mit leichtem schmunzeln im Gesicht.

Harrys Gesicht versteinerte. \enquote{Ich\abs äh\abs}

Professor Elber gab ihm die Karte.

\enquote{Danke}, sagte Harry und wandte sich ab.

\enquote{Versprechen sie mir etwas!}, forderte er. Harry drehte sich noch einmal um. \enquote{Wenn dieses Dokument schwarze Magie enthält, verwenden Sie es nicht. Ich werde es merken. \gst Und noch etwas. Ich habe keinerlei Kenntnis vom Inhalt.}

Harry nickte und ging. Kurz vor der Tür drehte er sich noch einmal um. \enquote{Sagen Sie mal\abs Ich habe sie vor einiger Zeit mit einer Karte von Hogwarts gesehen, wie sie mit Professor Flitwick und Professor Sprout einen Innenhof freigelegt hatten. Haben Sie da noch eine?}, fragte er schüchtern.

Professor Elber nickte und stand auf. Er verschwand in seinem Büro und kam nach einer guten Minute wieder zurück. In der Hand hatte er eine große Rolle Pergament. Er gab sie an Harry. Dieser bedankte sich artig und ging.
\end{rueckblick}

Die Karte faltete sich wieder zusammen und die Schrift verschwand. Harry legte die Karte zurück in seinen Koffer und faltete sein neues Pergament zusammen, um es in die Tasche zu stecken. Er nahm sich sein Schreibzeug und beschloss, in die Bibliothek zu gehen, um seine Hausaufgaben zu machen. In einigen Tagen konnte er beginnen, sich auf die Suche nach den geheimen, ominösen Punkten der Karte machen.

Auf dem Weg dorthin traf er Professor Elber, der verwirrt um sich schaute, ansonsten war der Gang leer.

\enquote{Sehen Sie auch alles grau?}, fragte er Harry.

\enquote{Die Steinwände sind doch immer grau Professor.}

\enquote{Ich meine nicht die Steinwände.} Er lief zu einem Fenster und sah nach draußen. \enquote{Wenn ich mich umsehe}, begann er und Harry stellte sich neben ihn, \enquote{dann sehe ich grau; graues Gras, grauer Himmel, graue Bäume, alles grau. Selbst sie kommen mir vor, als seinen sie aus einem alten Schwarz-Weiß-Film.}

Harry sah ihn an, als wüsste er nicht, ob er versuchte ihn zu veralbern, oder ob er verrückt geworden war. Oder er sah wirklich nur noch grau-stufig.

\enquote{Ich sehe keinen Unterschied zu früher}, gab Harry zur Antwort.

\enquote{Na ja, danke Harry, ich werde dann mal zu Poppy gehen. Vielleicht weiß sie etwas.} Er drehte sich um und ging.

Harry sah noch ein paar Minuten verwirrt drein, bevor er sich wieder Richtung Bibliothek absetzte.

Auf der Suche nach einem Buch für seine Hausaufgaben sah er eines mit dem Titel: \buchtitel{Grauschleier in praktischer und theoretischer Natur.} Jetzt war sein Interesse geweckt und er nahm es heraus. Darin fand er eine Beschreibung und ein paar Zauber, die diverse Wirkungen hatten. Ein Zauber entfernte die Farbe eines unbewegten Gegenstandes. Ein anderer von lebendem Gewebe, wieder einer von toten Lebewesen. Ein anderer nahm einem die farbige Sicht\abs Das war es doch, was er suchte. Es gab diese Zauber mit zeitlicher Begrenzung, aber auch dauerhafte. Er blätterte weiter und fand noch passende Gegenzauber.

Dann packte er seine Sachen zusammen und nahm das Buch mit auf die Krankenstation. Dort zeigte er es Madame Pomfrey, die sofort einen passenden Gegenzauber versuchte. Scheinbar schlug er fehl.

\enquote{Ich sehe immer noch alles in Grau}, antwortete Professor Elber.

Madame Pomfrey runzelte ihre Stirn und sah erneut in das Buch. Dann fand sie bei den Nebenwirkungen, dass es unter besonderen Umständen bis zu einer Stunde dauern könnte, bis die Wirkung wieder nachließ.

Professor Elber sah sich das Buch an und meinte: \enquote{Ein einfacher Grau-Zauber also\abs} Er musste noch eine Weile warten, bis er die Krankenstation verlassen durfte, also verließ Harry mit dem Buch in der Hand die Krankenstation und machte sich wieder auf den Weg in die Bibliothek. Er stellte das Buch zurück und widmete sich wieder seinen Hausaufgaben.




\begin{kommentar}
Als Harry seine Strafarbeit von Professor Snape in der Bibliothek absitzen muss, liest er zum ersten Mal etwas über Zauber, mit denen man aufgrund von Körperflüssigkeiten oder Hautzellen den Träger identifizieren kann. Das sind die ersten Schritte hin zu forensischen Methoden, die im nächsten Teil der Geschichte wieder aufgegriffen werden, als Harry Unterricht bei Elber erhält.
\end{kommentar}

\begin{kommentar}
Als Harry dann in seinem Bett liegt, erscheint ein kleiner Elf. Frodo. Harry nimmt ihn mit in sein Bett, damit er nicht frieren muss. Damit hat er unbewusst etwas mit seiner Magie angestellt und er beginnt, die Magie der Elfen in sich aufzunehmen. Dadurch hat er nun eine dritte Quelle der Magie. Neben seiner eigenen und der durch Voldemorts Seelenteil auch noch die der Elfen.
\end{kommentar}

\begin{kommentar}
Etwas später duelliert sich Harry mit Flitwick zum ersten Mal. Der kleine Zauberer ist erstaunlich flink und wuselt und springt umher. Eine kleine Hommage an Yoda und sein Duell mit Count Dooku.
\end{kommentar}

\begin{kommentar}
Als ich die Magi-Fu-Szenen auf dem Wasser geschrieben habe, dachte ich an Kung-Fu. Ich wollte es etwas spektakulärer machen, als der klassische Kampfsport. Und auf dem Wasser hat man zusätzliche Möglichkeiten.
\end{kommentar}

\begin{kommentar}
Und wieder hat Harry eine Strafarbeit in der Bibliothek. Er muss Bücher manuell sortieren. Eine schöne versteckte Gelegenheit, dass Snape ihm einige Bücher vorschlagen kann, die Harry interessieren dürften.
\end{kommentar}

\begin{kommentar}
Um noch leichte Unsicherheit zu schüren, fand ich es durchaus angemessen, dass Elber sich mit Malfoy Senior trifft. Einmal in der Woche spielen sie Schach. Sogar während Malfoy in Askaban saß.
\end{kommentar}

\chapter{Die Paare treffen ein}


Nach einer Stunde kam Professor Elber in die Bibliothek und setzte sich zu Harry. Nach einem kurzen Blick auf Harrys Hausaufgaben stand er wieder auf und verschwand zwischen den Regalen. Knappe zwei Minuten später kam er wieder und setzte sich mit drei Büchern wieder neben Harry. Er blätterte im ersten und suchte etwas. Dann legte er das Buch vorsichtig vor Harry ab. Während er im nächsten blätterte, sah Harry auf die Seiten und erkannte, dass ihn das ein gutes Stück weiter brachte. Das zweite Buch legte Professor Elber vor sich ab und suchte auch im dritten die richtige Seite heraus. Dann stand er auf und ging.

Er drehte sich noch einmal kurz und meinte: \enquote{Ein kleines Dankeschön dafür, dass sie das Buch gefunden und mir geholfen haben. \gst Und noch ein Hinweis. Beobachten Sie die Bücher, wenn sie sie aufräumen lassen.} Dann verschwand er.

Mitten in die Arbeit vertieft kamen Ron und Hermine, setzten sich neben ihn und begannen ebenfalls mit ihren Hausaufgaben. Mittlerweile war er recht gut darin Lunas Gedanken zu blockieren, die er sonst immer aufgefangen hatte, außer, sie wollte ihm etwas mitteilen, das hörte er dann klar und deutlich. Nachdem es fast Zeit fürs Abendessen wurde, entschieden sich die drei ihre Sachen im Gemeinschaftsraum zu lagern und danach in die Große Halle zu gehen. Harry erzählte ihnen vom Grauzauber und den Büchern, die Elber ihm herausgesucht hatte. Während des Essens erzählte Harry Ron und Hermine was er neues über die Karte herausgefunden hatte, natürlich ohne etwas über den Gemeinschaftsraum der Paare zu erzählen, oder wie genau er die Punkte gefunden hatte. Er deutete auf seine Tasche und versprach, ihnen einen Klon der Liste anzulegen, damit jeder, der Zeit hatte, sich auf die Suche und die Erkundung der Stellen machen konnte.

Plötzlich kam ein Vogel hereingeflogen, ein kleiner grüner Wellensittich. Professor McGonagall lief ihm hinterher. \enquote{Fiorin, Fiorin, komm her. Komm zu Mama.}

Als der Vogel an Harry vorbeiflog, versuchte er ihn zu fangen. Doch er griff daneben.

Nach langem Versuchen schaffte es schließlich Luna den kleinen Vogel zu fangen. Sie gab ihn Professor McGonagall. Nachdem sie den Wellensittich vorsichtig in den Käfig zurückgesetzt hatte, sprach Luna zu ihr: \enquote{Jetzt weiß ich, dass sie gut zu Vögeln sind, Professor.}

Harry drehte seinen Kopf zu Luna und sah sie entsetzt an.

\enquote{Miss Lovegood, denken sie an ihre Ausdrucksweise} maßregelte Professor McGonagall sie.

Harry hatte das Gefühl, ihr helfen zu müssen. Er sagte zu Professor McGonagall: \enquote{Wer zweideutig denkt, hat eindeutig mehr vom Leben.}

Professor McGonagall dreht sich zu Harry. Er hatte das Gefühl, dass nun sie es war, die rot wurde.

Sie nahm den Vogel im Käfig mit und verließ die Große Halle. Erst jetzt bemerkte Harry, dass alle umstehenden Schülerinnen und Schüler sie anstarrten und krampfhaft ein Grinsen unterdrückten. Es dauerte es ein paar Minuten bis die Große Halle vollständig gefüllt war, da Dumbledore angekündigt hatte, eine Erklärung abzugeben. Jeder Schüler und jede Lehrkraft war jetzt anwesend. Selbst Madame Pomfrey.

Professor Elber baute aus drei Gabeln, die er sich von Nachbartellern holte, eine kleine Skulptur, indem er die Zinken der Gabeln gegeneinander stellte und so versuchte ein Dreieck auf dem Teller zu legen. Die Seiten der Gabeln mit den Spitzen zeigten nach oben. Als das Essen verschwand, stand ein paar Sekunden danach ein Elf neben ihm und sah ihn abwartend an. Er lächelte ihm zu und sprach mit ihm. Der Elf nickte und wartete.

Dumbledore stand auf und begann gegen sein Glas zu klopfen. Alles verstummte. \enquote{Wie ich gestern bereits angekündigt habe, wird Professor Elber heute noch einmal versuchen, euch zu erklären, warum er euch das beibringt, was viele als dunkle Magie ansehen. Ich möchte heute noch einmal betonen, dass Professor Elber nichts mit Voldemort\abs} ein Raunen und Zucken lief wieder einmal durch die Halle. \enquote{\aabs zu tun hat. Er ist kein Todesser und auch kein Sympathisant. Er wird jetzt noch einmal versuchen euch zu erklären, warum er es für wichtig hält, dies zu unterrichten.} Dann setzte er sich wieder.

Die Tische begannen sich zu bewegen und nach oben an die Decke zu schweben. Dann folgten die Bänke. Die Schüler schreckten hoch und standen in der leeren Halle.

\enquote{Bitte bilden Sie alle einen großen Halbkreis.}

Die Schüler folgten und machten einen großen Halbkreis in mehreren Reihen. Eine Abwärts-Geste machte ihnen  klar, dass sie sich setzen sollen. Unter jeder Schülerin und jedem Schüler bildete sich ein Kissen, bevor er den Boden berührte und mit seinem Hintern auf den harten Boden aufkam. In der Mitte erschienen viele kleine Kissen auf denen Elfen erschienen.

In den hinteren Reihen kam es zu Gemurmel und Tumulten.

\enquote{Die werden später gebraucht, damit sie nicht verletzt werden. Sie werden sie schützen}, sprach Professor Elber, stand auf und lief in die Mitte der kleinen Empore. Er setzte sich auf den Boden und stellte die Füße auf den steinernen Boden darunter. Die Hände auf seinen Oberschenkeln atmete er einmal kurz durch und fing dann an zu erzählen.

\enquote{Ich war verletzt, als ich hörte, sie haben vor mir Angst. Es war nicht meine Absicht sie zu ängstigen. Es war vielmehr meine Absicht, ihnen klarzumachen, dass es notwendig ist, den Todessern etwas entgegensetzen zu können. Und das besteht nun mal darin, deren Methoden zu kennen. Sie müssen wissen, was sie erwartet. Wenn man die Angriffsmethoden und -taktiken seiner Gegner kennt, kann man effektiver zuschlagen.}

Er sah wieder auf und wartete kurz.

\enquote{Ich habe Ihnen in ihrer ersten Stunde erzählt, dass es keine dunkle oder helle, schwarze oder weiße Magie gibt. Ich habe ihnen erzählt, dass Magie keine Farbe hat. Haben sie das so weit verstanden?}

Keine Antwort.

\enquote{Dann lassen sie es mich wiederholen. Ich nehme an, jeder kennt einen Hammer.} Nicken quer durch alle Häuser und Geschlechter, sowie Jahrgangsstufen. \enquote{Und sie wissen, was man mit einem Hammer alles machen kann? Mit einem Hammer kann man Nägel in die Wand einschlagen. Aber man kann auch den Hammer jemanden auf den Kopf hauen und ihn somit umbringen. Bedeutet das jetzt, dass ein Hammer böse oder dunkel ist?}

Die Schüler schüttelten die Köpfe.

\enquote{Eben. Und genau so verhält es sich mit der Magie. Die Magie ist nur ein Werkzeug. Man verwendet sie. Soweit noch alles klar?}

Wieder kam ein Nicken von den Schülern.

\enquote{Den siebten Klassen habe ich es schon gezeigt und bin auch ins Detail gegangen.} Er erzeugte eine kleine Flamme auf seiner Hand. \enquote{Diese Flamme hier wird von vielen Dämonen- oder auch Teufelsfeuer genannt. Aber warum? Weil viele, die es kennen es für zerstörerische Zwecke einsetzen. Dabei kann man mit diesem Feuer viel mehr machen.}

Er ließ es fallen und sofort schlängelte es sich in die Mitte der Halle und bildete einen Ring inmitten des Halbkreises.

\enquote{Diesem lebendigen Feuer kann man seinen Verwendungszweck mitgeben. Wie sie hier alle sehen, greift es niemanden an, sondern brennt nur vor sich hin und wärmt. \gst Wer von ihnen kennt denn das ewige Feuer? Diese netten blauen Flammen, die man überall hin in einem Glas tragen kann?}

Ein paar Hände gingen in die Höhe.

\enquote{Würden sie diese Flammen der dunklen Magie zuordnen?}

Diejenigen, die gezeigt hatten, schüttelten ihre Köpfe.

\enquote{Warum nicht? Dieses Flammen unterscheiden sich vom lebendigen Feuer unwesentlich. Es ist dieselbe Art von Magie, die dahinter steckt. Ein paar kleine Änderungen und die Farbe der Flammen wird blau und es breitet sich nicht automatisch aus. Aber prinzipiell ist das dasselbe.}

Die Augen der Schüler und Lehrer wurden größer.

\enquote{Lassen sie mich noch etwas sagen. Lebendiges Feuer kann für viele Zwecke eingesetzt werden. Nehmen Sie zum Beispiel einen Waldbrand. Einen großen Waldbrand wie in Amerika. Diesen müsste man mit hunderten Zauberern bekämpfen und in Schach halten. Wenn diese aber das lebendige Feuer verwenden, dann reichen etwa ein Dutzend. Man gibt dem Feuer mit, dass es eine Brandschneise in den Wald brennen soll. Das Feuer schlägt nicht auf benachbarte Bäume über und verlöscht nach getaner Arbeit.}

Er stand auf und lief in die Mitte des Feuers. Dieses bildete nun einen Ring um ihn, nachdem es ihn eingelassen hatte. Dann fing eine kleine Schlangengestalt an aus dem Feuer heraus zu kommen; immer noch mit dem Feuer verbunden schlängelte sie sich an seinem Bein entlang hoch und legte dann ihren Kopf auf seine Schulter.

\enquote{Sehen Sie, keine Gefahr.}

Das Feuer löste sich auf. Jetzt traten die Elfen in Aktion. Sie bauten vor sich eine Energiekuppel auf, in der nur noch Professor Elber und der Elf stand, der schon die ganze Zeit in seiner Nähe war.

\enquote{Um ihnen zu zeigen, dass man sich gegen \accentuate{dunkle} Magie sehr wohl wehren kann, durch Abwehr oder auch Gegenangriffe, haben Henry und ich eine kleine Demonstration vorbereitet. Die Elfen vor ihnen werden sie vor den Zaubern schützen, die sicher kommen werden.}

Dann stellten sich beide voreinander auf, nickten einander zu und griffen an. Die nächste viertel Stunde kam Zauber um Zauber, der den Gegner verwirren sollte, ihm zusätzliche Gliedmaßen wachsen lassen, oder ihn entstellen sollte. Doch jeder der Zauber wurde abgewehrt. Dann griff der Elf in seine Trickkiste und warf durch eine Bodenverschiebung Professor Elber um. Dieser ließ sich fallen und fiel weich auf den Boden. Der Elf sprang in die Luft und wollte von oben auf ihn drauf fallen. Doch Professor Elber baute einen Schild auf, worauf der Elf abprallte und in die Luft geschleudert wurde. Er blieb förmlich an der Decke stehen und sah nach unten. Dann legte er sich hin und fing im Liegen an, Zauber nach unten zu werfen. Professor Elber warf Zauber nach oben. Als der Elf wieder den Boden berührte, brachen beide ihre Zauber ab und die Elfen lösten die Energiebarriere und verschwanden. Der Elf und Professor Elber verbeugten sich voreinander und der Elf verschwand auch.

Professor Elber setzte sich wieder auf seinen Platz an der Empore und fragte: \enquote{Sind Fragen da?}

\enquote{Wie sieht es mit den Unverzeihlichen aus?}, fragte ein Schüler.

\enquote{Bevor ich ihnen darauf antworte\abs Wie sehen sie das?}

\enquote{Na ja, ich würde sie der dunklen Magie\abs Sie wissen wie ich das meine}, sagte er.

\enquote{Sie würden also sagen: \enquote{Dunkle Magie.}}

\enquote{Ja}, antwortete der Schüler.

\enquote{Nehmen wir an, diese drei Flüche seien erlaubt, also legal.} Das verwunderte Kopfschütteln, oder das Gemurre einiger ignorierte er. \enquote{Nehmen wir weiterhin an, dass eine Person, die ihnen sehr nahe steht \gst zum Beispiel ein guter Freund, aber kein Partner \gst gerade aus einer Beziehung kommt und mit den Nerven vollkommen fertig ist.} Die Schüler nickten zu Bestätigung. \enquote{Nehmen wir weiterhin an, dieser Freund steht auf einer Klippe. Über ihm der Himmel, unter ihm der Boden und einen Schritt weiter ein Abgrund von zwanzig Metern Tiefe, der in einer stürmischen Brandung mit Felsen endet, sodass er bei einem Sprung aufprallt und stirbt. Er sagt ihnen, dass sie keinen Schritt weiter gehen sollen, dann würde er springen. Sie können sich ihm also nicht nähern. \gst Wie würden sie ihn retten? \gst Denken sie daran, was ich gerade noch am Anfang gesagt habe.}

Dann war es eine Weile still. Eine Schülerin hob die Hand und Professor Elber nickte ihr zu.

\enquote{Ich würde es mit einem Schwebezauber versuchen.}

\enquote{Dann ergreift er Gegenmaßnahmen. Er belegt sie mit einem Kitzelfluch und stürzt ab. Sie rennen lachend an den Rand und sehen vielleicht gerade noch, wie ihr Freund stirbt. Und sie lachen, da der Fluch noch wirkt.}

Betretene Gesichter.

\enquote{Andere Vorschläge?}

Wieder wurde ein Schüler dran genommen. \enquote{Ich würde dann den Imperius-Fluch nehmen, da wir ja annehmen er sei in unserem Beispiel legal. Der Fluch kann zwar mit einem starken Willen überwunden werden, aber wenn einer emotional dermaßen aufgelöst ist, wird es ein leichtes sein, ihn zu halten. Dann hole ich ihn zu mir, beruhige ihn und nehme den Fluch von ihm.}

\enquote{Guter Einfall. Wie sehen sie den Fluch jetzt?}

\enquote{Er kann auch nützlich sein. \gst Ich glaube, ich verstehe. Mit einem Messer zum Beispiel kann man sein Essen zubereiten, Seile zerschneiden, aber auch jemanden umbringen. Es kommt darauf an, was man damit macht.}

\enquote{Warum, denken sie, sind Messer dann nicht verboten? Man kann damit schließlich jemanden töten?}

Wieder dauerte es eine Weile, bis sich jemand meldete.

\enquote{Ich nehme an, dass es weniger \accentuate{missbräuchliche} Anwendungen für Messer gibt, als für den Imperius-Fluch.}

\enquote{Gut gesagt. Wissen sie, was mich bei unserer Gesetzgebung im Falle dieses Fluches stört? \gst Die Tatsache, dass man per se einen Aufenthalt in Askaban gewinnt und nicht per Fall unterschieden wird. \gst Nächster Fluch, der Tötungsfluch.}

Dieses Mal dauerte es etwas länger, bis sich jemand meldete. Professor Elber nickte und der Schüler begann.

\enquote{Die einzige sinnvolle Anwendung, die ich mir vorstellen könnte, wäre jemanden wie \accentuate{Sie-wissen-schon-wen} seinem Schicksal zuzuführen.}

\enquote{Also würden sie den Zauber nicht ganz der \accentuate{dunklen} Seite zusprechen?}

\enquote{Nein.}

\enquote{Gut. Was ist mit dem letzten Zauber? Dem Folterzauber?} Nach einer Weile löste Professor Elber auf. \enquote{Dieser Zauber ist wohl eine der wenigen Ausnahmen, die ich tatsächlich der \gst wie sie so schön sagen \gst \accentuate{dunklen Seite} zuordne.}

\enquote{Was sind die anderen Zauber, die sie dazu zählen?}

\enquote{Die werde ich ihnen nicht nennen. Dagegen ist der Cruciatus-Fluch ein erholsames Bad. Wobei nicht alle mit körperlichen Schmerzen zu tun haben.} Nach einer ganzen Weile fragte er: \enquote{Noch Fragen?}

Einhelliges Kopfschütteln folgte. Die Tische und Bänke kamen langsam von der Decke und nahmen ihre Plätze ein.

Die Türe öffnete sich und Professor Elber stand auf. \enquote{Es tut mir leid}, sagte er, \enquote{dass ich sie geängstigt habe. Das war nicht meine Absicht. Ich werde mir etwas überlegen, um sie dafür zu entschädigen.} Dann ging er Richtung Ausgang.

Dumbledore kam auf ihn zu und fing ihn ab. \enquote{Auf ein Wort, Frederick, in meinem Büro.}

In Dumbledores Büro angekommen setzen sich beide auf ein Sofa hinter Dumbledores Bürotisch. Es war ein kleines Separee.

\enquote{Warum haben sie den Siebtklässler das Dämonenfeuer beigebracht?}

\enquote{Lebendiges Feuer}, korrigierte ihn Elber.

\enquote{Meinetwegen, aber warum haben sie es den Schülern beigebracht?}

\enquote{Weil es prüfungsrelevant sein wird.}

\enquote{Wie?}

\enquote{Ich stelle mir das so vor. Es gibt draußen auf der Wiese eine große Kuppel. Milchig und mit einem Tor als Eingang. Dahinter ist es dunkel. Der Raum wird magisch erweitert. Jeder Schüler geht mit zehn Sekunden Verzögerung rein. Dort bekommt er verschiedene Aufgaben, welche zunehmend schwerer werden. Irgendwann steht er vor einem Problem, das er nicht lösen kann. Dann kommt der Ausgang und die Punkte werden analysiert. Wenn er den Schulstoff beherrscht, dann gibt es ein Annehmbar, hat er darüber hinaus eigene Ideen, gibt es ein \note{Erwartungen übertroffen} und wenn man den ganzen Parcours fast ohne Fehler geschafft hat, dann gibt es ein Ohnegleichen. Je schlechter man allerdings gegenüber dem Schulstoff wird, das heißt, je weniger man weiß, desto niedriger wird die Note.}

\enquote{Aber warum das Dämo\aabs lebendige Feuer?}

\enquote{Jeder, der es so weit schafft, kommt in eine Zone, die sehr kalt ist. Kein Zauber wirkt. Nur das lebendige Feuer wird ihn oder sie wärmen. Deshalb habe ich es dran genommen. Die Schüler müssen lernen, Zauber nicht nach Gut und Böse einzuteilen, sondern nach Notwendigkeit und Anwendungszweck. Wenn sie das Feuer einsetzen, um sich selber zu wärmen, oder die Kälte zu bekämpfen, ist es ein positiver Einsatz.}

\enquote{Schön und gut, soweit bin ich d'accord\footnote{So gut wie einverstanden, Ok, abgemacht}, aber wie wollen Sie sie überwachen? Mit einem Zauber? Diese kann man auch austricksen. Und die Schüler sind sehr gewieft.}

\enquote{Hauselfen, dachte ich mir. Ich habe schon bei den Elfen in der Küche angefragt, ob sie Lust darauf hätten. Das war vor drei Tagen. Wenn ich eine positive Antwort erhalte, dann wäre ich zu Ihnen gekommen und wollte das Ganze mit Ihnen durchgehen. Aber wenn Sie wollen, dann können wir das auch jetzt machen.}

Während des Gesprächs, warte Ron auf Harry im Gemeinschaftsraum, da dieser noch sein ausgeliehenes Buch zurückgab.

\enquote{Gibt es sonst noch irgendwo Bücher, welche die Mondbibliothek betreffen?}, fragte er Madame Pince.

\enquote{Leider nein, Mister Potter. Aber ich kann gerne einmal nachschauen.} Sie schwang ihren Zauberstab und kurz darauf erschien ein Pergament in der Luft. Sie schaute was darauf stand und zeigte das Pergament schließlich Harry. \enquote{Sonst gibt es nichts.}

Auf dem Pergament waren lediglich zwei Bücher aufgelistet. Eines davon hatte sich Harry bereits ausgeliehen und zurückgebracht. Das andere hatte er in der Bibliothek gelesen.

\enquote{Danke Madame Pince.} Dann ging er.

Nach dem Essen und zurück im Gemeinschaftsraum fragte Harry nach einer Partie Schach. Ron willigte ein, weil er wusste, dass er Harry mit Sicherheit schlagen würde. Doch Harry wusste, was er zu tun hatte. Er schloss kurz die Augen, nahm mit Luna Verbindung auf und spürte ihre Anwesenheit. Er spürte wie sie auf einem Stuhl im Gemeinschaftsraum der Ravenclaws saß und durch seine Augen und Ohren wahrnahm, was Harry sah und hörte. Ron begann das Spiel, und Harry entspannte sich, sodass er das Gefühl hatte, ein Fremder in seinem Körper zu sein. Er sah zu, wie er seine Hand hob und die einzelnen Züge fuhr. Nachdem er, oder besser Luna, Ron zweimal geschlagen hatte, wollte er sein Glück selber versuchen und glitt wieder in seinen Körper zurück. Luna war noch immer präsent, um ihn zu beobachten oder Tipps zu geben. Harry gewann knapp, nachdem er sich von Luna ein paar Züge abgekupfert hatte. Ron war sprachlos.

\enquote{Wie hast du das geschafft?}, fragte Ron.

\enquote{Übung}, entgegnete ihm Harry, \enquote{einfach nur Übung.}

\enquote{Ja, aber wie, wer?}

\enquote{Glaubst du ernsthaft, dass ich dir meine Quellen oder Übungspartner verrate?}

Ron grinste. \enquote{Jetzt weiß ich, warum du dich immer mal wieder nachts herumgetrieben hast und nicht im Bett warst. Du hast heimlich geübt.}

\enquote{Vielleicht}, antwortete ihm Harry.

Ron grinste und meinte dann nur: \enquote{Du hast dich verbessert. Dann muss ich jetzt wohl aufpassen, wenn ich mit dir spiele.}

\trenn

Der Samstag verlief angenehm ruhig und wurde nur durch die Hausaufgaben hinausgezögert. Kurz nach dem Abendessen verschwand Harry aus der Großen Halle und Ron vermutete, er würde wieder zum Schach üben gehen. Kurz darauf verschwand auch Luna, was aber nicht sonderlich auffiel. Einige Zeit später machten sich, für die anderen unbemerkt, die einzelnen Pärchen zum dritten Stock im Westflügel auf. Harry und Luna standen bereit und warteten auf die anderen. Als Dean und Amanda auftauchten, bekam Harry einen Kloß im Hals, da er immer noch nicht wusste was Dobby ihnen geschrieben hatte. Er hatte zwar einen weiteren Brief von Dobby erhalten, in dem er ihm die Anzahl der Pärchen mitteilte, aber nicht was sie wussten.

\enquote{Oh Harry}, sagte Dean erstaunt, als er Harry sah. \enquote{Hast du auch einen Brief bekommen. Und}, dann erst erblickte er Luna und sein Gesicht blieb stehen. \enquote{Du und Luna? \gst Seit wann seid ihr?}

\enquote{Ein paar Wochen schon}, entgegnete ihm Luna.

Langsam füllte sich der Gang und als Harry feststellte, dass fast alle Paare da waren, merkte er, dass wohl keiner wusste, dass er auf sie wartete. Langsam wurden die Paare unruhig, also ergriff Harry das Wort.

\enquote{Ich freue mich, dass ihr alle hier seit, obwohl ich nicht weiß, was in euren Briefen stand, möchte ich euch begrüßen und}, begann Harry als Dean ihn unterbrach.

\enquote{Du weißt nicht, was da drinnen stand? Du hast doch auch einen bekommen.}

\enquote{Na ja, Luna und ich wissen schon etwas länger, dass\abs}

\enquote{Was?}, sagte Dean erstaunt, \enquote{warum bist du dann hier?}

\enquote{Luna und ich werden euch einweisen.}

Den anderen fiel die Kinnlade herunter.

\enquote{Na ja}, machte Dean weiter, \enquote{es stand nur drin, dass, wenn wir einen Platz für uns Pärchen haben wollen, wir uns heute hier melden sollen. Ich wusste nicht, dass du\abs}

Doch Harry antwortete: \enquote{Und ich dachte schon, als du mich danach angeschaut hattest, du wüsstest, dass ich das sein werde.}

\enquote{Nein}, sagte Dean.

Plötzlich blieb Harrys Gesicht stehen, als er Draco Malfoy um die Ecke schauen sah.

\enquote{Malfoy?}, rief er. Die anderen drehten sich herum und schluckten ebenfalls; so als hätten sie etwas Verbotenes gemacht. Doch Malfoy kam um die Ecke dicht gefolgt von Maria Mayquen. \enquote{Was machst du hier?}, fragte Harry.

\enquote{Was machst du hier, Potter?}, antwortete Malfoy spöttisch.

\enquote{Ich unterhalte mich mit meinen Kollegen.}

\enquote{Alles Pärchen, wie?}

\enquote{Du bist aber auch nicht allein hier, wie? Hast wohl auch einen grauen Brief bekommen.}

In diesem Moment viel Dracos Kinnlade runter. Er wollte gerade anfangen wieder etwas zu sagen, als ihn Harry unterbrach.

\enquote{Alles, was ihr hier und jetzt erfahren werdet, dürft ihr niemandem erzählen}, sagte Harry.

Die ganze Meute sagte fast gleichzeitig: \enquote{Verstanden.}

\enquote{Und wieso kommst du auf die Idee, ich würde meinen Mund halten?}, fragte Malfoy.

\enquote{Nun, ich denke, dass du genauso froh darüber bist einen Zufluchtsort vor Mister Filch und den anderen Lehrern zu haben, dass du garantiert hier niemanden verpfeifen wirst.}

Luna drehte sich herum und sprach gleichzeitig mit Harry, der sich immer noch Malfoy zu wandte: \zauber{Aqua Neros.} Das Porträt öffnete sich und Luna verschwand. Harry hielt seine Hand hin, als wollte er die ganzen Pärchen hinein delegieren. Er ging als Letztes durch das Loch hinter dem Porträt, worauf es sich schloss. Dann kämpfte er sich durch die doch jetzt große Menge an Personen und stellte sich neben Luna. Gerade so als hätten sie es eingeübt, ergänzten sich ihre Sätze passend zu dem des anderen.

\enquote{Wir freuen uns, euch hier und heute zu sehen}, sagte Harry.

\enquote{Es ist selbstverständlich, dass hiervon keiner außer den hier anwesenden wissen darf.}

\enquote{Was ist mit den Lehrern? Wissen die hiervon?}, fragte ein Siebtklässler aus Hufflepuff, den Harry nur anhand der Uniform einordnen konnte.

\enquote{Nein}, entgegnete Luna. \enquote{Nur die Hauselfen und wir wissen von diesem Ort.}

\enquote{Die Hauselfen}, fuhr Harry fort, \enquote{werden ab heute jeden Tag hier sein und wenn nötig die entsprechenden Aufräumarbeiten vornehmen.}

\enquote{Hier hinter uns findet ihr eure Schlafzimmer.}

\enquote{Alles Doppelbetten.}

\enquote{Aber passt auf, der erste Raum\abs}

\enquote{\aabs gehört schon uns.}

\enquote{Da steht unser Name drauf.}

\enquote{Wie habt ihr von diesem Ort erfahren?}, fragte Amanda, Deans Freundin.

\enquote{Durch Dobby}, antwortete Harry.

\enquote{Der ehemalige Hauself der Malfoys?}, fragte Dean nach.

\enquote{Ja}, antwortete Harry. \enquote{Der hat uns in unserer dritten Woche diesen Ort gezeigt.}

\enquote{Und da kommst du erst jetzt?}, fragte Draco.

\enquote{Dobby hat uns gesagt, es sei noch zu früh für die anderen. Und ich musste ihm recht geben.} Er wurde leicht rot als Luna frei heraussagte: \enquote{Wir mussten ihn erst ausprobieren und verbrachten einige Nächte hier.}

Den anderen blieb das Gesicht stehen. Einige ließen ihre Kinnlade herunterfallen.

Als Dean seine Fassung wiedergefunden hatte, fragte er Harry \enquote{Ihr beide habt miteinander?}

\enquote{Nein}, antworteten Harry und Luna. Und Luna fügte hinzu: \enquote{Aber nahe dran. Glaubst du, wir haben keine Kontrolle über unsere Körper?}

Harry wurde rot. Doch seine Röte verblasste gleich wieder. \enquote{Schaut euch hier ruhig etwas um}, fügte Luna hinzu. Sie nahm Harrys Arm und zog ihn zu dem Schachtisch hinüber, drückte ihn auf den Stuhl und setzte sich auf seinen Schoß. Danach legte sie ihren Arm um seinen Hals und ihren Kopf auf seiner Schulter ab. Dean und Amanda beobachteten sie dabei. Dean konnte sich ein Grinsen nicht verkneifen, schaute zu Amanda um küsste sie.

Währenddessen machten sich Ron und Hermine im Gemeinschaftsraum Sorgen um Harry.

\begin{rueckblick}
Gerade als Luna und Harry wieder im Gemeinschaftsraum der Paare waren und es sich in einem Stuhl gemütlich gemacht hatten, schwebte ein Buch hinter dem Wandteppich hervor, direkt auf Luna und Harry zu. Es blieb wenige Zentimeter vor ihnen stehen und begann sich zu öffnen. Es blätterte auf eine bestimmte Seite und begann zu leuchten, um auf sich aufmerksam zu machen. Harry und Luna ließen voneinander ab und Harry las vor:

\enquote{Jedes Paar, egal ob es gerade diesen Raum hier benutzt, oder früher einmal benutzt hat, kann keinem Fremden gegenüber etwas über diese Räume erzählen. Zum Teil gibt es Schutzzauber, welche die Erinnerung blockieren oder verändern, zum anderen können auch schwere Schmerzen auftreten. Dies hängt vom Charakter des Individuums ab, das versucht diese Räumlichkeiten zu verraten.} Harry grinste erleichtert. Jetzt war er sich absolut sicher, dass ihn Malfoy nicht verraten würde.
\end{rueckblick}

Am anderen Morgen stand Harry auf und war gerade auf dem Weg zur Dusche, als ihn Luna am Handgelenk fasste. Sie stand auf, küsste ihn und meinte: \enquote{Willst du etwa ohne mich duschen?}

Harry lächelte sie an und sagte: \enquote{Ich wollte dich nicht aufwecken.} Er löste ihren Griff, nahm sie bei der Hand und führte sie zu den Duschen. Nachdem Harry die Tür geöffnet hatte und ein paar Schritte mit Luna hineingegangen war, bemerkte er Dean und Amanda in einer Duschkabine. Die beiden erschraken, als sie Luna und Harry sahen, lächelten aber dann, als sie merkten, dass sie genauso nackt waren wie Harry und Luna.

Dean und Amanda widmeten sich wieder sich selber und Luna zog Harry in eine weitere Duschkabine. Gerade als sie ihn eingeseift hatte, kam Dean ums Eck und meinte: \enquote{Hier ist es ja traumhaft.}

\enquote{Deswegen seid ihr ja auch hier}, antwortete Luna.

Dean fing an zu lachen und auch Amanda, die hinzugekommen war, lächelte.

\enquote{Lasst uns abtrocknen, anziehen und dann frühstücken gehen}, schlug Harry vor.

\enquote{Ok}, antworteten die anderen.

\enquote{Wir sollten aber nicht immer nachts hier sein. Sonst fällt das nur auf.}

Nach einem gemütlichen Sonntag und ein paar Runden Schach gegen Hermine, ging Harry ins Bett.

Am Montag darauf hatte Harry wieder Unterricht bei Hagrid. Als Harry Richtung Hagrids Hütte lief, bemerkte er, dass ein großes Areal mit Schnee bedeckt war. Es war klar abgegrenzt, sodass Harry vermutete, es sei herbeigezaubert worden. Außerdem meinte Harry, dass sich da etwas bewegen würde, denn es waren viele Berge voll von Schnee zu sehen. Jeder hatte eine andere Größe und Form.

\enquote{Heut nehm mer Schneewalker durch}, sprach Hagrid.

\enquote{Schneewalker sin ganz friedliche Geschöpfe und ihre, naja, des was hinten rauskommt, is auch ganz nützlich. Wir werdn uns heut etwas um sie kümmern, sie füttern und warten, bis’ mit dem Verdauen fertig sind. Dann sammelt des Ganze auf und bringt’s in’n Krankenflügel.}

Sämtliche Schüler verzogen das Gesicht bei dem Gedanken, schon mal etwas von Schnee\-wal\-ker-Ex\-kre\-men\-ten bekommen zu haben, als sie im Krankenflügel gewesen waren. Sie gingen mit Hagrid zwischen die Schneehaufen. Als sie mitten im Schnee standen, drehten sie sich um und sahen ein paar von den eigenartig aussehenden Tiere. Etwas, das wie ein Hirsch mit Flügeln und kleinem Geweih aussah, näherte sich ihnen. Es war ganz weiß und näherte sich Malfoy von hinten. Als er sich umdrehte, erschrak er. Das Schneewalker-Männchen kam einen Schritt näher und schnupperte an ihm.

Malfoy zitterte leicht, doch er blieb stehen.

Plötzlich wirbelte Harry herum, da er eine Zunge an seinem Haaransatz spürte. Vor ihm stand ein weißes Etwas. Dann erinnerte er sich, dass sie Schneewalker durchnahmen. Es war ihm für kurze Zeit entfallen. Jetzt spürte er allerdings dieselbe Zunge im Gesicht. Der Schneewalker hatte anscheinend an ihm gefallen gefunden. Oder er hatte Salzmangel und leckte ihn deshalb ab. Harry besah sich das Tier genauer und merkte, dass es ein Männchen war. Er strich ihm über das Fell und war erstaunt darüber, wie weich es war. Doch trotz alle Bemühungen durch reden und füttern sowie streicheln, hatte Harry heute kein Glück. Das Männchen, das er betreute, machte keinen Haufen, den Hagrid aufsammeln konnte. Aber trotzdem musste er sich das Männchen und seine Fellzeichnung einprägen, damit er ihn in der nächsten Stunde wieder finden konnte. Da das Tier aber weiß war, war das mit der Fellzeichnung gar nicht so einfach. Man musste sehr genau hinsehen, damit man eine leichte Schattierung im Fell wahrnehmen konnte.

\trenn

Donnerstagnachmittag hatte Harry wieder Geschichte der Zauberei bei Professor Binns. Es hatte bereits geläutet und alle bereiteten sich darauf vor, dass Professor Binns durch die Tafel schwebte, doch nichts passierte. Plötzlich öffnete sich die Tür zum Klassenzimmer und Professor Elber kam herein.

\enquote{Einen schönen Nachmittag, Klasse}, sagte Professor Elber, als er durch die Reihen zum Lehrerpult durchlief.

Etwas irritiert kam ein \enquote{Guten Nachmittag Professor Elber} zurück.

\enquote{Professor Binns ist leider für etwa eine Woche weg. Er sprach etwas von einem vierhundertjährigen Geburtstag und sieht sich daher außer Stande zu unterrichten.}

Ein Murmeln ging durch die Klasse. \enquote{Also}, sprach Professor Elber weiter, \enquote{da ich nicht weiß, was Professor Binns bisher durchgenommen hat und noch unterrichten möchte, er hatte es nämlich ziemlich eilig, dachte ich, machen wir eine Fragestunde. Was wollt ihr wissen?} Stille herrschte im Klassenzimmer und ein paar Schüler weckten ihre bereits schlafenden Nachbarn. Professor Elber grinste als er dies sah. \enquote{Professor Binns scheint wohl eine einschläfernde Wirkung zu haben?}

Hermine war die erste, die wieder die Hand hob, und Professor Elber fragte. \enquote{Professor, wann kamen eigentlich die ersten Magier und Hexen auf die Welt?}

Professor Elber schaute sie ernst an und trat dann vor den Reihen auf und ab, seinen Blick auf den Boden gesenkt. Dann blieb er stehen und schaute Hermine an. Ihr Gesichtsausdruck zeigte einen Hauch von Panik. Harry hatte den Eindruck, dass sie bereute, die Frage gestellt zu haben.

\enquote{Nun, vor über sechstausend Jahren gab es noch keine Magier. Eines Tages dann wurde eine Person geboren, die schon sehr früh feststellte, dass sie über besondere Fähigkeiten verfügte. Diese könnte man als den ersten Magier bezeichnen.} Er machte eine Pause und lief wieder umher. \enquote{Irgendwann begann er zu heiraten und Nachwuchs zu zeugen. Seine Kinder erbten seine Fähigkeiten und er begann sie auszubilden. Damals dauerte es wesentlich länger, da die Entwicklung der Zauberei noch nicht so weit fortgeschritten war. Er merkte, dass seine Frau und seine Kinder alterten, spürte selber aber nichts dergleichen. Er entschloss sich sein aussehen dem seiner Frau anzupassen und als es an der Zeit war zu verschwinden, ging er. Damals war es nichts Besonderes, wenn jemand von einer großen Reise nicht zurückkam. Er verjüngte wieder sein Aussehen und entschloss sich, die Population der Magier und Hexen zu vergrößern. Ob er damals an Heirat untereinander dachte, ist nicht bekannt, aber man geht allgemein davon aus. Irgendwann wurde es ihm zu mühselig, immer wieder neue Magier und Hexen heranzubilden, und er entschloss sich, die Grundlagen der heutigen Magie zu legen. Dann rief er alle lebenden Zauberer und Hexen zusammen.} Er wanderte wieder durch die Reihen und lehnte sich an das Lehrerpult, verschränkte die Arme und erzählte weiter.

\enquote{Er brachte ihnen die neue Form der Magie bei und erteilte ihnen den Auftrag, diese weiterzugeben.} Er drehte sich um und ging zur Tafel. \enquote{So wie wir heute zaubern} und er malte vier Blöcke untereinander, \enquote{bauen wir auf der obersten Schicht auf. Wir benutzen unseren Zauberstab, um Magie auszuführen. Dies erleichtert uns das Zaubern.} Er füllte den oberen Kasten mit grün aus und fuhr fort. \enquote{Nur bei einigen Sachen können wir keinen Zauberstab verwenden. Denken wir mal an die Verwandlung eines Animagus. Oder habt ihr schon mal einen Animagus gesehen, der sich mit einem Zauberstab antippte?}

Allgemeines Kopfschütteln. Er füllte den zweiten Block in Blau aus. \enquote{Über die anderen Blöcke ist heute so gut wie nichts bekannt. Also fragt lieber nicht.} Professor Elber stellte sich wieder vor das Lehrerpult und lehnte sich an.

\enquote{Aber wieder zurück zu unserem Magier. Irgendwann entschloss er sich, sich zur Ruhe zu setzen und nicht mehr zu unterrichten. Er baute ein Schloss und verbrachte viel Zeit darin. Dann kam die Zeit der großen Kriege, an denen er vielfach beteiligt war, und am Ende hatte er genug und überließ sein Schloss den vier größten Magiern ihrer Zeit und verschwand. Keiner weiß wohin. Er stand natürlich auf der anderen Seite.}

Er stieß sich vom Pult ab und fragte erneut: \enquote{Gibt es sonst noch etwas, was ihr wissen wollt?}

\enquote{Ja}, fragte Draco Malfoy. \enquote{Wer waren die vier?}

\enquote{Ich glaube, ihr kennt sie}, sagte Professor Elber vergnügt. \enquote{Es waren die Gründungsmitglieder von Hogwarts.}

\enquote{Was?}, klang es durchs Klassenzimmer.

\enquote{Das kann unmöglich sein Professor Elber}, sprach Hermine. \enquote{Im Buch Geschichte von Hogwarts\abs}

\enquote{Halt}, unterbrach sie Professor Elber. \enquote{Das wurde nie im Buch erwähnt. Er wollte nicht, dass sein Name mit diesem Gemäuer in Verbindung gebracht wurde. Sonst wärt ihr alle jetzt nicht hier.}

\enquote{Wie hieß er denn?}, fragte Ron.

\enquote{Sein Name war Friedward Alejious Elberon.} Ein Raunen ging durch die Klasse. \enquote{Er war Zeitweise unter dem Namen \enquote{Schlächter von Nervot} bekannt und wo wir schon beim Thema sind, gibt es noch einen Fehler in der Geschichte Hogwarts}, fuhr Professor Elber fort. \enquote{Betrachten Sie doch einmal das Wappen.}

Er drehte sich zur Tafel und mit dem Schwung seines Zauberstabes erschien das Wappen auf der Tafel. Er drehte sich zur anderen Seite und zauberte vier weitere Tafeln herbei und zog jedes einzelne der vier Wappenteile auf die vier Tafeln, um sie dort vergrößert anzuzeigen. Links das der Slytherins, rechts davon das der Hufflepuffs, daneben das der Ravenclaws und ganz rechts das von Gryffindor.

\enquote{Nun}, fuhr er fort, \enquote{irgendjemand eine Idee?} Er lief wieder durch die Reihen. Totenstille. Kein Laut war zu hören.

Hermine machte nur große Augen. \gedanke{Ein Fehler in der Geschichte Hogwarts? Wieso habe ich das nicht bemerkt?}

Als Professor Elber am Ende des Raumes angelangt war, lehnte er sich gegen die Wand und zeigt mit seinem Zauberstab auf die Tafel mit dem Wappen von Slytherin. Das Logo begann sich herauszulösen und schwebte als dreidimensionales Gebilde in der Luft.

\enquote{Irgendjemand eine Ahnung?}, wiederholte Professor Elber.

Hermine streckte die Hand und sprach: \enquote{Sollte die Schlange vielleicht ein Basilisk sein?}

\enquote{Wie kommen sie darauf, Miss Granger?}

\enquote{Nun ja, sie haben gerade dieses Logo ausgewählt. Also dachte ich, es müsse etwas mit einer Schlange zu tun haben. Und das ähnlichste einer Schlange ist eben ein Basilisk.}

\enquote{Interessanter Ansatz}, sagte Professor Elber. \enquote{Aber ich habe einfach irgendwo angefangen. \gst Nein, das ist kein Basilisk.} Einige Minuten verstrichen, während der alle ratlos waren und ihre Köpfe anstrengten.

Er schwang wieder seinen Zauberstab und das Logo kehrte zurück zu seiner Tafel. Das nächste Logo kam von der Tafel und schwebte in den Raum hinein.

\enquote{Wie sieht es hier aus?}

Das Hufflepuff-Logo schwebte im Raum und zeigt einen Dachs. Betretenes Schweigen herrschte. Nach einer Weile schwang Professor Elber wieder seinen Zauberstab und der schwebende Dachs kehrte zurück und der Adler von Ravenclaw schwebte im Raum. Immer noch schwiegen die Schüler. Professor Elber schwang ein letztes Mal seinen Zauberstab und die schwebende Figur kehrte wieder zurück auf die Tafel und der Löwe der Gryffindors begann sich von der Tafel zu lösen und im Raum zu schweben. Wieder sagte keiner etwas.

Professor Elber tippte ungeduldig mit seinem Zauberstab umher und fing wieder an, durch den Raum zu laufen. Er drehte sich um und zeigte auf die ersten drei Tafeln, sodass nun alle Figuren im Raum schwebten. \enquote{Nun, was ist? Irgendjemand \gst eine Ahnung.} Immer noch schwiegen alle. Professor Elber sprach jetzt leicht sauer: \enquote{Also, wenn ihr nicht mitarbeitet, dann kann das hier nichts werden. Ich glaube, ich muss Professor Binns mal erzählen, wie in seiner Klasse die Arbeitsmoral ist \gst wenn ihr keine Ahnung habt, dann ratet wenigstens. Schaut euch die Figuren an. Gibt es irgendwelche Tiere, die ähnlich aussehen?}

Die Klasse zuckte zusammen. Keiner von ihnen hatte Professor Elber derartig zornig gesehen. Immer war er bisher freundlich gewesen und noch nie hatte er einen Ton drauf, der einen zusammen zucken ließ. Außer vielleicht dem berechtigten Aufbrausen über ihre bisherigen VgddK-Lehrer.

Jetzt meldete sich ein Hufflepuff. \enquote{Es könnte vielleicht ein Greif sein? Ein goldener Greif. Eine Mischung zwischen einem Greifen und einem Löwen.}

Professor Elbers Miene zeigte jetzt wieder einen fröhlicheren Ausdruck. \enquote{Wie kommen sie darauf?}

\enquote{Na ja, ein goldener Greif sieht einem Löwen ähnlich. Und das Haus heißt schließ\-lich auch Gryf\-fin\-dor.}

Professor Elber lief wieder durch die Klasse und meinte: \enquote{Zehn Punkte für Hufflepuff. Das war eine kluge Schlussfolgerung. Eigentlich sollte ich die zehn Punkte Gryffindor abziehen, weil keiner von ihnen darauf gekommen ist.} Am Pult vorne angekommen drehte er sich wieder um. \enquote{Es ist in der Tat so, dass hier der Name Gryffindor nicht nur für den Namen des Hausgründers steht, sondern auch der Greif das Wappentier ist. Genauer gesagt ein goldener Greif, so wie man die Löwen-ähnlichen Wesen auch nennt.}

\trenn

Als Harry am nächsten Freitag zum Frühstücken kam und die Treppen zur Großen Halle hinunterging, lagen dort Fred und George, sich die Hände vor den Bauch haltend und lachend, auf dem Boden. Ihr Lachen schallte durch die ganze Große Halle. Er war mit Ron und Hermine zum Frühstücken unterwegs, und die drei blieben vor Fred und George stehen.

\enquote{Was macht ihr da?}, fragte Ron die beiden.

Fred und George standen auf und fingen an zu erzählen.

\enquote{Also}, erzählte George, \enquote{wir wollten gerade frühstücken, da lief uns Dumbledore über den Weg.}

\enquote{Er murmelte etwas von Verstopfung und Bauchschmerzen}, fügte Fred hinzu.

\enquote{Also haben wir ihm eines unserer Toffees angeboten.}

\enquote{Unsere neueste Kreation, Montezumas-Kicher-Toffees.}

\enquote{Was ist das denn?}, fragte Harry.

Und George antwortete ihm: \enquote{Eine erweiterte Form der Montezumas Rache Toffees.}

Harry, Ron und Hermine blieb das Gesicht stehen.

\enquote{Ihr habt doch nicht etwa unserem Schulleiter ein Abführmittel gegeben?}, fuhr Hermine entsetzt dazwischen.

\enquote{Oh doch}, antwortete George, der sich immer noch den Bauch vor lauter Lachen hielt.

Hermine schüttelte nur den Kopf, während Harry und Ron zu grinsen begannen. Sie zog die beiden in die Große Halle und begann zu frühstücken. Bereits nach einer viertel Stunde kam Dumbledore in die Große Halle und lief an Harry vorbei, welcher nur ein paar Worte aufschnappte.

\enquote{Tolle Dinger, diese Toffees. So viel Spaß hatte ich schon lange nicht mehr. Ich muss die Weasley-Zwillinge unbedingt fragen, ob sie mir noch mehr davon haben}, murmelte Dumbledore im Vorbeigehen und als er Harry sah: \enquote{Nur für den Fall der Fälle.}

Man konnte Harry ansehen, dass er genau wusste, was Dumbledore meinte. Und Dumbledore schien auch aufzufallen, dass Fred und George es den dreien erzählt hatten. Fred und George hatten zwar die Schule verlassen, kamen aber immer mal wieder zu Besuch. Sie vertrieben ihre Waren und blieben meist zum Mittagessen. Natürlich wurden sie hinausgeworfen, wenn sie entdeckt wurden, was aber nicht immer passierte. So waren sie einmal im Monat für eine Stunde da.

Diese Woche hatte Harry am Nachmittag, nach Snape, wieder Kräuterkunde bei Professor Sprout. Sie mussten wieder ein paar Pflanzen umtopfen und aus den Blättern einen Sud brauen, der etwaige offene Fleischwunden vor Infektionen schützen konnte. Der Nachmittag verlief ruhig, und Harry konnte am Abend den Großteil seiner Hausaufgaben erledigen. Am Samstagvormittag war Quidditch-Training angesetzt und Harry war erschöpft und hungrig, als er die Große Halle passierte, um zu Mittag zu Essen.

\trenn

Harry war mit Luna auf dem Weg zum dritten Stock, als er kurz vor dem Ziel auf Professor Elber stieß, der ein Bild an der Wand betrachtete.

\enquote{Professor Elber, was machen sie denn hier?}, frage Harry.

\enquote{Ah, Harry, Luna, ich schaue mir ein paar Bilder an. Ich bin auf der Suche nach etwas}, antwortete Professor Elber.

\enquote{Auf der Suche nach etwas Bestimmtem?}

\enquote{Sagen wir eher nach etwas, was da sein muss, von dem ich aber nicht weiß, wo es sich genau befindet, oder wie es genau aussieht.}

\enquote{Können wir ihnen helfen?}, fragte Luna.

Harry bekam plötzlich Herzklopfen, als er das hörte und Luna merkte auch, dass sie das wohl besser nicht gesagt hätte. \gedanke{Wärst du bloß still gewesen}, dachte Harry und hörte von Luna ein: \gedanke{Upps, Entschuldigung.}

\enquote{Sagen euch die Namen Sardak Slyhoot und Selvine Vertap etwas?} Harry bekam große Augen und auch Luna erweckte den Eindruck, die Namen schon einmal gehört zu haben. Professor Elbers Mund zeigte ein leichtes Schmunzeln. \enquote{Wisst ihr, die beiden haben mich mal, es ist jetzt schon einige Zeit her, besucht. Sie waren auf der Suche nach einem ruhigen Plätzchen und ich habe ihnen ein paar Sprüche verraten, die es ihnen ermöglichen sollten, ihr Ziel zu erreichen. Hier ungefähr}, er machte eine ausladende Geste durch den ganzen Flur, \enquote{sollte eine Art Kammer sein, in der die beiden ungestört sein konnten.} Harrys und Lunas Gesicht färbte sich leicht rot. \enquote{Ich nehme an, ihr wisst, von was ich spreche.} Luna und Harry blieben standhaft und verzogen keine Miene. Plötzlich runzelte Professor Elber leicht die Stirn und sein Kopf schwenkte zum Porträt, das an der Wand hing. Er fing an leicht zu grinsen, schloss die Augen und meinte dann kurz darauf: \enquote{Dann will ich euch mal nicht weiter stören. Es scheint, als ob dieser Ort benutzt wird. Ich hätte ihn mir gerne einmal angeschaut. Aber das ist jetzt wohl leider nicht mehr möglich.} Er drehte seinen Kopf wieder zu Harry und Luna und ging Richtung Treppenaufgang, woraufhin er kurz im Gewirr der sich ständig bewegenden Treppen verschwand.

Harry und Luna schauten sich nur fragend an und waren verwirrt. Eine Stille erfüllte plötzlich ihre Köpfe und beide erschraken, als sich ihnen ein Pärchen näherte, das kurz zuvor aus dem Loch hinter dem sich öffnenden Porträt stieg. Beide atmeten erleichtert auf, als sie die beiden sahen und begannen, sich auf das immer noch offen stehen Porträt zu zubewegen. Innen angekommen setzten sich beide und starrten sich mit ausdruckslosen Gesichtern an. Ein Hufflepuff kam vorbei und setzte sich auf den freien Platz neben Harry.

\enquote{Hi Harry}, sagte er fröhlich gelaunt.

\enquote{Hi Donan}, kam es aus Harry heraus und das, obwohl er nicht mal seinen Namen wusste.

\enquote{Woher kennst du meinen Namen?}, fragte Donan.

Harry war erst jetzt wieder voll bei Sinnen und sagte: \enquote{Luna hat ihn mir gesagt.} Das war zum Teil richtig, denn Luna kannte ihn. Aber sie dachte nur seinen Namen.

\enquote{Du schaust bedrückt aus Harry.}

\enquote{Weißt du, es ist Folgendes\abs} und Harry und Luna erzählten von ihrem Zusammentreffen mit Professor Elber.

\enquote{Der wird mir so langsam unheimlich, dieser Professor Elber}, meinte Donan, \enquote{Aber, wenn er wie du sagtest, die beiden getroffen hat und die Angaben in dem Buch stimmen, welches dort hinter dem Wandteppich steht, dann müsste er ja über 128 Jahre alt sein. Und wenn er ihnen die Tricks verraten hat, dann muss er bestimmt schon die Zauberschule abgeschlossen und einige Jahre praktische Erfahrungen gesammelt haben. Das würde dann bedeuten er wäre über 150 Jahre, so grob geschätzt. Und dafür sieht er mir doch noch zu jung aus.}

Harry dachte nach.

\enquote{Also, entweder lügt er uns an, oder er ist wirklich so alt und hat sich nur verplappert. Denn ich glaube kaum, dass er es jemanden auf die Nase binden wollte.}

\enquote{Bei dem, was er uns schon während des Unterrichts erzählt und auch gezeigt hat, glaube ich schon, dass er einige Jahre auf dem Buckel hat}, meinte Harry. \enquote{Aber so alt? Wie alt werden denn Zauberer?}, fragte er nach.

\enquote{Schon so um die 150~Jahre. Die sehen dann aber so aus wie Dumbledore, oder noch faltiger}, sagte Donan.

\enquote{Harry?}, sagte Luna plötzlich, \enquote{hat er nicht etwas über die ersten Magier erzählt?}

\enquote{Ja Luna}, antwortete Harry; unsicher, worauf sie hinaus wollte.

\enquote{Wie wäre es, wenn er dieser Magier ist?}

\enquote{Welcher Magier?}, fragte Donan.

\enquote{Der erste}, antwortete Luna. \enquote{Über siebentausend Jahre alt.}

Das kam den beiden dann doch etwas abwegig vor.

\trenn

Als Harry eines Morgens wieder mit Luna das Schloss durchstreifte, traf er auf Ron und Hermine, die sich mit Professor Elber unterhielten. Er näherte sich ihnen, um ihre Unterhaltung mit anzuhören.

\enquote{Guten Morgen, Harry, Luna}, kam es ihnen entgegen.

Er bemerkte nicht, dass Professor Elber auf seine Brust schaute.

Dann sprach Professor Elber ihn an. \enquote{Ein interessantes Amulett haben sie da.} Harry schaute zuerst seinen Professor an und danach auf seine Brust. \enquote{Ein Erbstück?}, fragte Professor Elber.

\enquote{Nein}, antwortete Harry. \enquote{Ein Geburtstagsgeschenk.}

Professor Elber hob erstaunt seine Augenbrauen und öffnete seine Augen. \enquote{Die Person, die ihnen das geschenkt hat, muss sie sehr, sehr gernhaben.}

\enquote{Wieso?}, fragte Harry. \enquote{Wissen Sie was für ein Amulett das ist?}, fragte er Professor Elber.

Ginny bog gerade um die Ecke und sagte: \enquote{Ich habe ihm das geschenkt.}

\enquote{Wo haben sie das eigentlich her?}, fragte Professor Elber.

\enquote{Von Borgin und Burkes.}

Elbers Augen weiteten sich. \enquote{Wie die da wieder herangekommen sind?}, fragte sich Professor Elber nun.

\enquote{Was hat es mit dem Amulett auf sich?}, fragte Hermine.

\enquote{Es gehörte einst \gst nun ja, gehören ist vielleicht übertrieben, er hat es erschaffen.}

\enquote{Wer?}, bohrte Hermine nach.

\enquote{Salazar Slytherin.} Die fünf keuchten. \enquote{Das Amulett stammt von ihm. Er hat es ursprünglich als Hochzeitsgeschenk erschaffen.} Noch immer schauten die fünf erstaunt. \enquote{Ach Harry, sehen sie was, wenn sie das Amulett in die Hand nehmen und ihre Augen schließen?}

Harry wurde zunehmend unwohl. \enquote{Ja}, antwortete er zögerlich.

\enquote{Dann sind sie in direkter Linie ein Nachfahre von Slytherin. Nur einer seiner Nachfahren ist in der Lage, etwas durch das Amulett zu sehen.}

\enquote{Aber}, machte Harry weiter. \enquote{Als ich in der Kammer Godric Gryffindors Schwert aus dem sprechenden Hut zog, meinte Professor Dumbledore, ich sei ein wahrer Gryffindor.} Jetzt keuchte Professor Elber. Unsicher drehte er seinen Kopf und suchte nach einer Sitzmöglichkeit. Er ging zu einer kleinen Nische in der Wand und setzte sich. Er atmete schwer. Mit einem durchdringenden Blick schaute er Harry an. Harry kannte diesen Blick sonst nur von Dumbledore.

Harry wurde wieder zunehmend unwohl. \enquote{Was hat das zu bedeuten, Professor Elber?}, fragte Harry.

Er schaute ihn lange und intensiv an, bevor er zu erzählen begann. \enquote{Sie sind einerseits ein Ur-Ur-Ur-Enkel von Salazar Slytherin, andererseits von Godric Gryffindor, scheint mir.}

Harry staunte. \enquote{Wollte mich deshalb der Hut nach Slytherin stecken? Weil ich ein Nachfahre von Slytherin bin?}, fragte Harry nach.

\enquote{Das kann gut sein. Aber sie sind auch ein Nachfahre von Gryffindor und nach dem sie in seinem Haus sind, müssen sie den Hut wohl überzeugt haben, sie hier reinzustecken.}

\enquote{Aber warum wollte dann der Hut mich nach Slytherin stecken?}

\enquote{Voldemort.}

\enquote{Voldemort?}

\enquote{Ich habe von Dumbledore gehört, dass er einige seiner Kräfte versehentlich auf sie übertragen hat. Sie sind ein Parselmund. Der Hut hat das erkannt und die Seite Slytherins ist in Ihnen scheinbar etwas stärker ausgeprägt als die Seite Gryffindors.}

\enquote{Nein}, schrie Harry.

\enquote{Sachte, sachte. Salazar war kein schlechter Mensch oder Zauberer. Er hatte seine Ideologie der Rassenreinheit erst nach der Schulgründung entwickelt. Der Hut war zu diesem Zeitpunkt schon erschaffen. Er entscheidet nach den ursprünglichen Eigenschaften von ihm. Vielleicht schlummerten einige seiner Thesen schon in ihm, als er dem Hut etwas von sich überließ. Vielleicht sind deshalb in den letzten tausend Jahren wenig Muggelstämmige oder Halbblüter in Slytherin.}

Harry stand auf. Voller Zorn darüber, dass er ein Nachfahre von Salazar Slytherin sein sollte, riss er sein Amulett vom Hals und warf es in hohem Bogen von sich.

\enquote{Nein!}, rief Professor Elber. Reflexartig sprang er hoch und streckte seine Hand nach dem Amulett aus. Wenige Millimeter vom Boden entfernt hielt es an und schwebte in der Luft. Es dreht sich noch. Doch langsam kam es zum Stillstand. Professor Elber zog seine Hand kurz zurück, worauf das Amulett in seine Hand sprang. Er umschloss das Amulett und sah Harry vorwurfsvoll an. \enquote{Wenn sie es nicht mehr wollen, dann nehme ich das gerne. Aber zerstören Sie es nicht einfach, Harry.}

\enquote{Von mir aus können sie es haben, Professor. Ich bin nicht scharf darauf.} Zornig verließ er die Gruppe und begab sich zum Speisesaal. Seine weiteren Worte konnte keiner mehr vernehmen, da er sich bereits außer Hörweite befand.

\enquote{Wie?}, fragte Hermine nach. \enquote{Sie meinen, es gab in Slytherin keine reinrassigen Zauberer oder Hexen?}

\enquote{Ja, wieso erstaunt sie das?}




\begin{kommentar}
Ein kleiner Vogel kommt während des Abendessens hereingeflogen und flattert in der Großen Halle herum. Er gehört McGonagall. Als er wieder eingefangen wurde, sagt Luna: »Jetzt weiß ich, dass Sie gut zu Vögeln sind, Professor.« Leider kann man dem geschriebenen Satz nicht so leicht die Doppeldeutigkeit entnehmen, die er in gesprochener Form haben sollte. Aber ich hoffe doch mal, dass ihn viele trotzdem entdeckt haben. (Tipp: Vögeln kann man auch kleinschreiben. <vögeln>)
\end{kommentar}

\begin{kommentar}
Als Elber das lebendige Feuer demonstriert und über die unverzeihlichen Flüche referiert, sagt er am Ende, dass er den Schülern eine Entschädigung zu deren Verängstigung anbieten möchte. Da hat er schon Das Magie in Konzert im Hinterkopf, das später stattfinden wird.
\end{kommentar}

\begin{kommentar}
Die erste Vertretungsstunde die Elber in Geschichte der Zauberei gibt, ist zugleich auch ein Hinweis, dass er der erste Magier ist, der auf der Welt entstanden ist und auch, dass er Hogwarts gebaut hatte, um es später an die vier Schulgründer zu übergeben.
\end{kommentar}

\chapter{Eigenarten}


\enquote{Hhhr. Das wirft alles über den Kopf, was wir von Slytherin glauben zu wissen. Wieso sollte es uns also nicht erregen?}

Professor Elber schaute Hermine nun direkt an und meinte: \enquote{Was halten Sie von einer Geschichtsstunde über Hogwarts? Professor Binns wird sicherlich die eine oder andere Stunde mir überlassen und sich vielleicht auch dazu bereit erklären, mir in \VgddK zu helfen. Ihr braucht nämlich noch Übung, um Geister und Gespenster abzuwehren.}

\trenn

In der Nacht wachte Harry plötzlich auf, da er auf die Toilette musste. Er verließ seinen Schlafsaal und erledigte sein Geschäft. Auf dem Rückweg zu seinem Bett hörte er eine Stimme aus dem Gemeinschaftsraum. Er schlich leise in sein Zimmer und holte seinen Zauberstab, bevor er sich auf den Weg nach unten machte. Im Gemeinschaftsraum angekommen sah er eine durchsichtige Gestalt. Doch es war keiner der üblichen Hausgeister. Nicht einmal ein Geist, den er kannte.

\enquote{Hallo}, sagte Harry zögerlich.

Der Geist drehte sich um und begrüßte Harry. \enquote{Hallo Harry.}

Harry stutzte. \enquote{Wer sind Sie?}, fragte er.

\enquote{Später Harry. Setz’ dich erst einmal}, bot ihm der Geist an und platzierte sich in einem Sessel, sodass Harry sich gegenüber setzen konnte. \enquote{Zuerst einmal musst du das Amulett wieder zurückholen. Es ist im Kampf gegen Voldemort sehr wichtig.}

Harry staunte. Ein Geist, der Voldemorts Namen sagte. \enquote{Wer sind Sie?}, fragte Harry erneut.

Doch der Geist schien ihn nicht zu hören. Oder er wollte es nicht. \enquote{Du musst mein Amulett wieder an dich nehmen, Harry.}

Harry durchzuckte ein Blitz. Ihm wurde plötzlich heiß und kalt. Dann brach es aus ihm heraus. \enquote{Slytherin. Sie sind Salazar Slytherin.}

\enquote{Gut bemerkt, mein Junge. Und zudem dein Urahn. Aber zurück zum Amulett.}

\enquote{Aber}, brach es aus Harry heraus, \enquote{ich weigere mich das zu akzeptieren. Sie können nicht mein Urahn sein.}

Doch Slytherin protestierte. \enquote{Erstens, es ist so. Zweitens, es ist so, und drittens, es ist so. \gst Dich stört wahrscheinlich das weit verbreitete Gerücht, dass ich nur reinblütige Zauberer mag und etwas gegen Halbblüter oder Muggelgeborene habe. Aber dem ist nicht so. Zugegeben\abs} und so erzählte Slytherin Harry seine Geschichte. \enquote{Ich habe damals die Theorie vertreten, dass sich die magischen Künste durch die Heirat von Magiern untereinander stärken sollten. Von vielen wurde sie damals begeistert aufgenommen. Aber nach ca. einem Jahrzehnt oder zwei habe ich erkannt, dass diese genetische Inzucht untereinander zu vielen Problemen führt. Zudem brachte sie nicht das erwartete Ergebnis. Also gab ich meine Theorie auf und versuchte sie zu bereinigen. Doch meine Versuche schlugen fehl. Zu stark war inzwischen die Idee der Rassenreinheit geworden. \gst Ich schäme mich noch heute dafür.}

\enquote{Sie schämen sich dafür?}, fragte Harry.

\enquote{Na na, nicht so förmlich, Harry.}

\enquote{Also gut, Salazar.}

\enquote{Schon besser.}

\enquote{Aber warum?}, wollte Harry schon wieder fragen, doch Salazar unterbrach ihn.

\enquote{Nicht jetzt. Das kannst du alles nachlesen. Ich möchte mit dir über dein Amulett sprechen. \gst Doch zuerst, sag mir was du siehst, wenn du es fest in der Hand hältst.}

\enquote{Ich sehe meine Freundin}, antwortete Harry.

\enquote{Ah, sehr gut. Du musst wissen, dass du mit diesem Amulett aber auch andere Personen sehen kannst. Zu jedem Zeitpunkt, an jedem Ort. \gst Ich habe erfahren, dass du eine Verbindung mit jemand anderen hast. Dich und Tom Riddle verbindet etwas.}

\enquote{Ja}, antwortete Harry leicht überrascht.

\enquote{Aber mit dem Amulett kannst du eine besondere Verbindung zu ihm aufnehmen. Du kannst ihn beobachten, ohne dass er etwas davon mitbekommt. Du kannst ihm etwas zuflüstern, ohne dass du in der Nähe bist. Du kannst ihm Bilder in den Kopf setzen.}

Harry fiel sein Kinn herunter. \fluestern{Voldemort Bilder in den Kopf setzen}, murmelte er.

\enquote{Ganz genau. Du musst es dir wieder holen. Gleich morgen früh. Je länger du es trägst, desto besser kannst du Verbindung mit ihm aufnehmen.} Langsam begann Salazar zu verblassen.

\enquote{Meine Zeit ist bald um Harry. Wir sehen uns morgen Abend wieder hier. Gleiche Zeit. Ich muss dir noch was über das Amulett\gst} Doch weiter kam er nicht. Er war verschwunden.

\trenn

\enquote{Herein!}

Harry öffnete die Tür und ging auf Professor Elbers Schreibtisch zu. Sein Büro war anders als die Büros der anderen Lehrer. Seines enthielt kaum einen Gegenstand. Neben dem reich verzierten Schreibtisch aus edlem Holz war dort nur ein kleines Sideboard mit vielen Schubladen, ebenfalls aus Holz. Ansonsten war sein Büro komplett leer, wenn man von den vier Bildern der Hausgründer von Hogwarts einmal absah.

\enquote{Professor?}, fragte Harry.

\enquote{Ja, was gibt es, Harry? Setzen Sie sich doch.}

Harry setzte sich. \enquote{Es ist wegen gestern. Ich habe da vermutlich überreagiert, als ich das Amulett wegwarf. \gst Ich hätte es gerne zurück.}

Professor Elber sah ihm lange in die Augen. \enquote{Und was genau führte zu Ihrem Sinneswandel, wenn ich fragen darf?}

Harry druckste leicht herum und meinte schließlich: \enquote{Die Verwandtschaft zu Salazar. Ich habe eingesehen, dass er doch nicht so war, wie wir alle vermuteten. Das war nicht leicht.}

\enquote{Deshalb sind sie heute Morgen in der Bibliothek gewesen?}

\enquote{Ja}, log Harry.

Professor Elber griff in seine Tasche und nahm das Amulett heraus. Er hatte den Verschluss bereits repariert und legte das Amulett auf seinen Tisch. \enquote{Lassen Sie mich eines klarstellen. Sie baten mich, Ihnen mein Amulett zu schenken, denn als sie es wegwarfen und nichts mehr damit zu tun haben wollten, war es herrenlos. Also habe ich mich seiner angenommen. Momentan gehört es noch mir.} Damit gab er Harry zu verstehen, dass es wieder in sein Eigentum übergehen würde, sollte er es nehmen. Dann dürfte er es aber nicht mehr so einfach wegwerfen.

Harry betrachtete es kurz und nahm es dann in seine Hand. Es lag nun in seiner Innenhand und die Kette lag um seinen Zeigefinger herum und hing zu Boden. Er schloss seine Hand und wollte gerade zu Professor Elber schauen, als er sich plötzlich drehte und in einem dunklen Saal wieder fand.

Harry sah einen langen, Kerzen-erleuchteten Tisch an dessen Kopfende Voldemort saß. Links und rechts am Tisch saßen Todesser. Einige davon kannte Harry. Malfoy, Lestrange, Rowls, Dolohov. Harry bekam große Augen, als Voldemort ihn sah.

\enquote{Harry Potter}, säuselte er.

Alle Todesser drehten sich um. Harry wurde zunehmend unwohl. Voldemort zückte seinen Zauberstab und Harry griff instinktiv in seine Tasche. Doch er hatte keinen dabei.

\enquote{Wo soll Harry Potter sein?}, fragte Bellatrix Lestrange.

\enquote{Ja}, meinte ein Todesser, den Harry nicht kannte.

\enquote{Er steht genau dort, drei Meter vom mittleren Fenster weg}, antwortete Voldemort.

Harry bemerkte, wie sich die Todesser ratlos ansahen. \enquote{Ich weiß nicht, woher du plötzlich kamst Harry Potter, aber diesen Raum verlässt du nicht lebend. \gst \spruch{Avada Kedavra}}, brüllte er.

Ein grüner Blitz verließ seinen Zauberstab, durchdrang Harry und schlug hinter ihm eine Vase und ein kleines Holztischchen entzwei. Verängstigt und ungläubig wegen dessen, was gerade eben passiert war, betastete er seinen Brustkorb, doch da war nichts. Er war noch am Leben und es ging ihm gut. Plötzlich drehte es Harry wieder und die Umgebung wandelte sich.

Er saß wieder in Professor Elbers Büro. Genau in derselben Position, wie er den Raum gedanklich verlassen hatte und hob seinen Kopf, um seine Augen zu sehen. Dann sah er sich irritiert um, als er realisierte, wieder im Büro zu sein. Seine Hand öffnete sich leicht. Das Amulett lag nun wieder lose in seiner Hand.

Professor Elber schaute ihn mit leicht schrägem Kopf an. \enquote{Alles in Ordnung?}, fragte er.

\enquote{Ja}, antwortete Harry. Er bedankte sich, stand auf und ging. Im Türrahmen drehte er sich noch um und sah zu den Porträts. Sein Blick blieb bei Salazar Slytherin stehen. Er blinzelte ihm aus seinem Bild kurz zu. Harry schmunzelte leicht und verließ das Büro. Ihn wunderte nur, dass seine Narbe nicht schmerzte. Voldemort musste doch wütend sein.

Die Tür zum Büro flog zu und Harry hörte eine gedämpfte Stimme. \enquote{Er hat es sich gerade geholt, Salazar.}

Harry fiel noch etwas ein, er drehte auf der Treppe um und klopfte abermals an die Tür. \enquote{Herein!}, vernahm er von drinnen.

\enquote{Was kann ich für sie tun?}, fragte Professor Elber Harry ohne aufzusehen.

\enquote{Ich hätte gerne eine Erlaubnis für die verbotene Abteilung}, sagte Harry.

Professor Elber hob seinen Kopf, deutete ihm an sich zu setzen, nahm seinen Zauberstab und holte ein Formblatt aus dem Nichts herbei.

An der Überschrift konnte er eine Berechtigung für die Bibliothek erkennen. Professor Elber wartete, bis Harry weitermachte, damit er das Formular ausfüllen konnte.

\enquote{Es ist ein Buch über Dementoren}, sagte Harry.

\enquote{Fünfzehn}, gab Professor Elber zur Antwort.

\enquote{Wie bitte?}, fragte Harry nach.

\enquote{Fünfzehn \gst Bücher, die auf diese Beschreibung passen. Eines mehr oder weniger.}

Harry begriff. Er versuchte sich an das Inhaltsverzeichnis oder den Titel zu erinnern. \enquote{Da war was über Abwehr von Dementoren drin.}

\enquote{Neun.}

\enquote{Und über Aufzucht und Vermehrung.}

\enquote{Zwei.}

\enquote{Es hatte einen braunen Einband}, sagte Harry schließlich. Professor Elber wollte schon zur Feder greifen, als er es sich anders überlegte und stattdessen das Formblatt zerriss, worauf es sich in Luft auflöste. Er stand auf und lief um den Tisch herum. Harry bekam es mit der Angst zu tun. \enquote{Entschuldigung Professor}, stammelte er und stand hastig auf, um den Raum zu verlassen. Harry drehte sich um wollte gerade das Büro verlassen, als er hinter sich eine Stimme hörte.

\enquote{Habe ich gesagt, dass sie gehen sollen?} Harry lief es eiskalt den Rücken runter. Wenn er die Stimme nicht erkannt hätte, hätte er schwören können, dass es Severus Snape war, der das sagte. Derselbe Tonfall, dieselbe Betonung. Mulmig blieb er stehen. Professor Elber lief an ihm vorbei und stieg die Treppe zum Klassenzimmer hinunter. \enquote{Kommen Sie mit.}

Harry folgte ihm. Sie standen nun vor einem der vielen Schränke an der Seite des Klassenzimmers. Alle hatten eine gläserne Front mit Holzrahmen. Professor Elber holte einen kleinen Schlüsselbund aus der Hosentasche und öffnete das Schloss, um an die Bücher im Schrank zu gelangen. Er zog eines der Bücher ein paar Zentimeter heraus und schob es nach rechts. Es sah so aus, als ob die Bücher der Reihen ihm folgen würden. Von Links erschienen neue Bücher und Rechts verschwanden sie im Holz. Dann bemerkte Professor Elber seinen Fehler und drückte das Buch in die andere Richtung. In gewissem Abstand tauchte ein neues Buch auf, das ebenfalls ein paar Zentimeter herausgezogen war. Als er das richtige Buch gefunden hatte, zog er es ganz heraus. \enquote{Ist es das hier?}, fragte er Harry und zeigte es ihm. Harry nickte. Professor Elber nahm das Buch mit zum Pult und legte es darauf. Dann schwang er seinen Zauberstab darüber und ein Blatt Pergament erschien aus dem Nichts. Er legte seinen Zauberstab darüber und zog ihn waagerecht von oben nach unten. Ein weiteres Formblatt erschien. Der Titel des Buches war bereits ausgefüllt. Professor Elber winkte Harry zu sich und meinte dann \enquote{Unterschreiben sie bitte hier.}

Harry schritt um den Tisch und sah sich das Blatt an. Dort stand in etwa:

\accentuate{Der Unterzeichnete bestätigt, sich folgendes Buch ausgeliehen zu haben.}

Dann kam der Titel des Buches. Harry kontrollierte ihn, indem er den Zettel anhob und verglich. Dann las er weiter. Das heutige Datum war unter \accentuate{Ausleihdatum} bereits eingetragen und das Feld \accentuate{Rückgabedatum} war leer. Harry unterschrieb neben dem Ausleihdatum und reichte Professor Elber das Pergament. Dieser nahm es entgegen und legte es in den Schrank an die Stelle des Buches. Danach verschloss er den Schrank wieder.

Harry bedankte sich, schob das Buch in seine Tasche und wollte gerade den Raum verlassen. \enquote{Spätestens am Jahresende hätte ich das Buch gerne wieder.} Harry drehte sich kurz um, nickte und verschwand.

\trenn

Auf dem Weg zur Großen Halle traf er auf Luna, die ihn fragte: \enquote{Was wolltest du bei Voldemort?}

Harry stutzte. \enquote{Woher?}

\enquote{Unsere Verbindung. Ich habe es gesehen. Die ganze Runde an Todessern.}

Harry war erstaunt. \enquote{Davon habe ich gar nichts mitbekommen}, meinte Harry.

\enquote{Du warst auch ziemlich beschäftigt}, meinte Luna. \enquote{Ich wäre fast beim Gehen gestolpert. Daran müssen wir üben. Wir müssen lernen, uns gegeneinander abzuschotten.}

Harry verfolgte seine Vision noch das ganze Essen lang. Er erzählte Hermine und Ron, sowie Ginny davon. Diese konnten mit der Art der Vision gar nichts anfangen.

Nach dem Essen zog er sich um und nahm seinen Besen, um auf dem Feld mit seiner Mannschaft zu trainieren. Er suchte das Feld nach dem Schnatz ab und ließ seinen Blick schweifen. Er suchte im Himmel, auf dem Boden und auf den Rängen. Auf einem Turm der Slytherin sah er Dumbledore, der ihnen scheinbar beim Training zusah. Er wunderte sich, warum Dumbledore ihnen zusehen sollte. Aber andererseits kam er auch kurz zu einem Training der Slytherins und der Ravenclaws, sowie der Hufflepuffs, fiel ihm wieder ein.

\gedanke{Er schaut wohl bei allen Gruppen einmal vorbei}, ging Harry durch den Kopf.

Er sah sich weiterhin um und entdeckte den Schnatz am Boden. Sofort sauste er nach unten, um den Schnatz zu fangen. Während seines Fluges durchzog ihn ein Gedanke. \gedanke{Kann ich meinen Besen anweisen, mir hinterherzufliegen und mich zu überholen? Ohne Gewicht sollte er doch schneller sein, da er weniger Masse zum Beschleunigen und Bremsen hat. Unten könnte er mich leicht auffangen.}

Dann musste er seinen Besen bremsen und nach oben ziehen, da der Schnatz zur Seite schwenkte. Er schaffte es nicht ganz und stieg mit einer weniger eleganten Rolle von seinem Besen herab. Doch er verletzte sich nicht. Er stand etwas bedröppelt am Boden und nahm seinen Besen auf. Der Schnatz flog vor seiner Nase vorbei und er musste nur zugreifen. Doch der Schnatz war schnell. Also stieg er wieder auf seinen Besen und jagte ihm hinterher, bis er ihn hatte. Dabei musste er ein paar Klatschern ausweichen, die seine Kameraden ihm immer wieder in den Weg zu schlagen versuchten.

Mit dem Schnatz in der Hand flog er zu den Rängen und blieb vor Dumbledore in der Luft stehen. Er nutzte die Gelegenheit, nachdem er den Schnatz wieder fliegen gelassen hatte, Dumbledore zu fragen, ob diese Aktion mit seinem Besen möglich wäre. Dabei beobachtete er immer wieder den Himmel und den Boden, sowie das gesamte Spielfeld. Er blickte Dumbledore kaum an. Doch auch dieser schaute mehr der Mannschaft zu, als dass er Harry anblickte.

\enquote{Das käme auf einen Versuch an, Harry. Du brächtest auf jeden Fall bei deinen ersten Versuchen jemand am Boden, der deine Aktionen überwacht und deinen Sturz bremst. Es wäre während eines Spiels natürlich eine tolle Aktion, \gst falls es sich anbieten würde.}

Harry nickte in den Raum hinein und bekam die nächsten Worte von Dumbledore nicht mehr mit, da er den Schnatz erspähte und ihm zu folgen versuchte. Doch dieses Mal war der Schnatz schneller. Also kehrte er nach einer viertel Stunde wieder zu Dumbledore zurück und stellte sich in dessen Nähe.

\enquote{Ich war fertig, Harry.}

Harry nickte und wartete, ob er den Schnatz sehen würde. Doch vor Trainingsende kam er nicht mehr in Harrys Blickfeld, obwohl er Kreuz und Quer über das Spielfeld schwebte.

Kurz bevor er an diesem Tag ins Bett stieg, stellte er sich seinen Wecker. Er durfte sein Treffen mit seinem Vorfahren nicht verpassen. Als er geweckt wurde, verließ er sein Bett und schlich sich in den Gemeinschaftsraum. Weil er dort niemanden sah, setzte er sich wieder in denselben Sessel, wie am Abend zuvor und wartete. Harry erschrak, als Salazar Slytherin plötzlich wieder auftauchte.

\enquote{Hallo Harry}, sagte Salazar Slytherin.

\enquote{Hallo Salazar}, gab Harry zurück.

\enquote{Wir haben auch heute nicht viel Zeit}, fuhr Salazar fort. \enquote{Ich wollte dir noch etwas über dein Amulett sagen.}

\enquote{Gleich}, antwortete Harry. \enquote{Ich würde vorher gerne noch wissen, inwieweit ich mit dir und Godric Gryffindor verwandt bin.}

\enquote{Ah ja}, antwortete er. \enquote{Mein Ururur-Enkel hatte die Ururur-Enkelin von Godric Gryffindor geheiratet und du bist der letzte aus dieser Linie.}

\enquote{Und Voldemort?}, fragte Harry.

\enquote{Der ist in direkter Linie mit mir verwandt. Aber fast ausschließlich über die weibliche Linie. Ich schäme mich ein wenig für ihn. \gst Aber nun zurück zum Amulett. Hast du es?}

Harry zog es unter seinem T-Shirt hervor, um es Salazar zu zeigen. \enquote{Ich trage es wieder, ständig.}

\enquote{Sehr gut}, meinte Salazar. \enquote{Du musst wissen, dass dich das Tragen des Amuletts stärkt. Ein Teil meiner Kraft wird somit dir übertragen. Du musst Voldemort aufhalten. \gst Hast du bisher schon mal etwas Eigenartiges mit deinem Amulett erlebt?}, fragte Salazar.

\enquote{Ja}, antwortete Harry. \enquote{Ich hatte mir gerade das Amulett zurückgeholt, als ich mich plötzlich in einer Halle mit einem großen Tisch befand. Voldemort und seine Todesser waren da. Aber nur er konnte mich sehen. Er versuchte sogar, mir den Todesfluch auf den Hals zu hetzen.}

Salazar antwortete: \enquote{Dann hast du bereits eine der Funktionen des Amulettes kennengelernt. Du kannst dich Voldemort zeigen, wann immer du willst, und dich mit ihm unterhalten. \gst Wenn du dich auf ihn konzentrierst, sobald du das Amulett hältst. Oder du kannst ihm Gedanken oder Bilder in seinen Geist und seinen Träumen platzieren.}

\enquote{So wie Legilimentik?}, fragte Harry nach.

\enquote{So in etwa}, antwortete Salazar. Harry grinste etwas. Der Gedanke, Voldemort zu ärgern, gefiel ihm.

\enquote{Bedenke aber, dass er es merken wird. Er ist ein guter ich meine großer Zauberer. Zwar böse, aber er hat viel Macht. Sein Zorn auf dich wird umso größer, je mehr du ihn ärgerst. Also übertreibe es nicht. Oder noch besser. Fange gar nicht damit an.} Salazar fing wieder an zu verblassen.

\enquote{Noch eine Frage, Salazar. Sind die Malfoys auch mit dir verwandt?}

Salazar überlegte eine Weile. \enquote{Nein \gst halte dein Amulett fest und denke an mich. Dann können wir uns unterhalten. Aber nur einmal am Tag können wir uns sehen und dann nur wenige Minuten}, antwortete er und verschwand.

Harry grinste. Das musste er Malfoy unter die Nase halten. Irgendwann. Er stand auf und ging wieder ins Bett. Seine Narbe kribbelte leicht, also begann er wieder mit seinen Okklumentik-Übungen, die er selbst wieder aufgenommen hatte. Er leerte seinen Geist, während er im Bett lag und beruhigte ihn. Dann schlief er mit dem Gedanken daran, in einem schwarzen Raum in einem Bett zu liegen und zu schlafen, ein.

\trenn

\enquote{Würdet ihr mir nachher helfen?}, fragte er Hermine und Ron.

\enquote{Wobei?}, fragte Hermine.

\enquote{Bei einem Experiment.}

\enquote{Welchem?}, fragte Ron nach.

\enquote{Nach dem Essen sage ich es euch. Es ist nicht schwer. Ihr sollt mich nur überwachen und müsst schnell reagieren. Nichts Gefährliches.}

Nach dem Essen schickte er seine zwei Freunde zum Quidditchfeld, während er noch seinen Besen holte. Unten angekommen erklärte er Ron und Hermine was er vorhatte. Er erzähle von seinem Gespräch mit Dumbledore und dass es auf einen Versuch ankäme. Ron war natürlich sofort begeistert, nur Hermine war skeptisch. Aber alleine Ron wollte er sich nicht anvertrauen, da seine Zauberkünste zwar besser wurden, aber er sein Leben einem Zauber von Ron nicht anvertrauen würde, wenn er eine Wahl hätte. Er würde es Ron nie sagen, aber diese Angst hatte er.

Er nahm Hermine etwas beiseite und wagte einen unkonventionellen Versuch sie zu überzeugen. \enquote{Ich werde diesen Versuch machen. Und ich vertraue darauf, dass du über deinen Schatten springst und meinen Arsch rettest, falls es Ron nicht schaffen sollte. Du kennst ihn genauso gut wie ich. Also Hase, spring über deinen Schatten.}

Dann ließ er ihren Arm los und ging zurück zu Ron. Falls sie errötete, merkte er es nicht, da er von ihr weg lief. \enquote{Ich glaube, ich habe sie überzeugt. Bist du bereit?} Ron nickte. \enquote{Arresto Momentum, wenn ich fallen sollte.} Wieder nickte Ron.

Dann stieg er auf seinen Besen, sah noch einmal zu Hermine und stieg nach einen knappen und zögernden Nicken nach oben.

Während des Steigfluges dachte er sich: \gedanke{Ich hätte Hermine nicht Hase nennen sollen. Entweder versteht sie es falsch, oder sie ist sauer auf mich. Ich mag sie zwar, mein Schwesterchen\abs} bei diesem Gedanken musste er lachen. \gedanke{Aber mit seiner Schwester führt man keine Beziehung.}

Oben fiel ihm ein, dass er gar nicht daran dachte, wie er seinen Besen dazu bringen sollte, ihn aufzufangen.

\stimme{Konzentriere dich. Und vertraue auf deine Magie}, hörte er in seinem Kopf.

Er konnte die Stimme nicht zuordnen, aber sie klang vertraut und beruhigend. Also versuche es Harry. Er konzentriere sich und spürte seinen Besen unter sich. Aber nicht nur so, als wenn er ihn berührt. Sondern er spürte ihn in seinem Geiste. Es war eine Art der Kommunikation, die keine Worte benötigt. Worte waren nicht nötig, ja sogar überflüssig. Nicht möglich.

Er teilte seinem Besen mit, dass er abspringen würde und er ihn rechtzeitig auffangen sollte. Doch er konnte es nicht beschreiben, selbst wenn er es müsste. Es war eine Kommunikation, die unbewusst passierte und auf reinem Vertrauen und auf Intuition basierte. Er sprang ins Leere und der Besen folgte ihm. Er spürte, wie er langsamer wurde und er seinen Besen unter sich wieder spürte. Harry hatte das Gefühl, dass Rons Zauber ihn zu früh bremste, denn er hatte noch Zeit, bevor er unten aufschlug.

\enquote{Ron, hast du einen Zauber gewirkt?}, fragte er,

\enquote{Ja, wieso? War das falsch?}, fragte Ron nach.

\enquote{Nein, nein. Nur etwas voreilig. Du hättest noch Zeit gehabt. Das machen wir nochmal. Ich brauche schließlich Sicherheit. Gleiches Szenario wie vorher, nur greifst du später, wenn überhaupt, ein.}

Ron nickte begeistert. Harry stieg erneut auf und bereitete sich vor. Dann sprang er erneut ab. Dieses Mal spürte er keine Bremsung. Allerdings kam er hart auf seinem Besen auf und fiel pustend vom Besen auf den Boden, als er unten ankam. Der Besen hatte ihn hart zwischen den Beinen getroffen und seine Hoden gequetscht. Er brauchte einige Minuten, bis er einen erneuten Versuch wagen wollte, doch Hermine überzeugte ihn, zu Madame Pomfrey zu gehen und sich untersuchen zu lassen.

Es war Harry zwar peinlich, doch er fügte sich, da er sonst nicht zur Ruhe kam. Er machte sich auf den Weg zur Krankenstation. Schweren Ganges ging er durch das Schloss in den Krankenflügel. Sein Besen begleitete ihn den ganzen Weg. Stehend schwebte er neben ihm her, bis sich ihre Wege trennten und er Ron und Hermine folgte. Im Gemeinschaftsraum flog er selbstständig in sein Zimmer und blieb in der Ecke stehen, in der er immer stand.

Doch Harry bekam von all dem nichts mit. Er war zu sehr mit sich selbst beschäftigt.

\enquote{Madame Pomfrey? Könnten sie mich untersuchen?}

Sie war alleine auf der Krankenstation und kam zu ihm heran. \enquote{Was haben sie denn?} Harry wurde rot. \enquote{Oh. Schwere Geschütze.} Sie zog ihren Zauberstab und umrahmte mit Schutzwänden das Bett. Dann legte sie einen Stillezauber auf das Gebiet, um sich vor neugierigen Ohren zu schützen.

\enquote{Ich habe ein Problem mit meinen Hoden. Ich habe sie mir\abs Sagen wir so. Ich habe ein kleines Experiment gemacht und bin quasi auf meinen Besen gefallen. Es war ziemlich hart. Der Schmerz ist zwar kaum noch da, aber sicher ist sicher.} Jetzt fiel ihm eine Last von der Seele. Jetzt, da es raus war.

\enquote{Dann legen sie sich mal hin}, forderte ihn Madame Pomfrey auf. Harry folgte ihrer Aufforderung und sie begann ihn zu untersuchen. Sie fuhr mit dem Zauberstab über seinen Intimbereich und zog eine Augenbraue in die Höhe. \enquote{Ich gebe ihnen eine Salbe mit. Drei Tage lang zweimal täglich eincremen. Den Penis, die Hoden und auch zwischen den Pobacken.} Harry wartete, bis sie wieder kam und ihm die Salbe gab. \enquote{Nehmen Sie nächstes Mal einen Schutz, oder belegen sie ihre Geschlechtsorgane mit einem Zauber}, riet sie ihm.

Harry nickte, bedankte sich bei ihr und verließ den Krankenflügel. Abends cremte er sich ein und ging dann zu Bett.

\trenn

Madame Pomfrey betrat ihre Räume, ging zielstrebig in das Badezimmer und begann sich auszuziehen. Nackt ließ sie sich Badewasser in die Wanne einlaufen und betrachtete sich danach im Spiegel. All die Jahre, die sie schon hinter sich hatte, machten sich in dem freundlichen Gesicht bemerkbar. Das magisch eingelaufene Wasser war bereits hochgestiegen, sodass sie in die Wanne trat und unter wohligem stöhnen sich in das warme Wasser gleiten ließ. Ihren Kopf lehnte sie an die Kante der Wanne und ließ ihre Haare in das Wasser gleiten. Der anstrengende Tag hinter ihr und das warme Wasser um sich ließ sie ihre Gedanken gleiten und ihre Muskeln entspannen. Ihr ging der Narben-geplagte Erstklässler durch den Kopf, der sich Anfang des Schuljahres bei ihr einfinden musste, damit sie ihn untersuchen konnte, um eventuelle Probleme besser abschätzen zu können.

\begin{rueckblick}
\enquote{Hallo, ich bin Madame Pomfrey und Sie sind sicherlich Mister Allman.}

\enquote{Ja Madame.}

\enquote{Sie wissen, weshalb Sie hier sind?}

\enquote{Ja Madame.}

\enquote{Dann setzen sie sich mal auf das Bett hier.}

Der Junge gehorchte und stieg auf das Bett. Madame Pomfrey legte die Krankenakte beiseite und begann mit ihrem Zauberstab den Jungen zu untersuchen. Nach Abschluss der Untersuchungen erschien ein Pergament mit den Ergebnissen neben der Akte. Madame Pomfrey las sich das Ergebnis durch und ordnete das Pergament ein. Dann wandte sie sich wieder ihrem Patienten zu.

\enquote{Das ist vor einem Jahr passiert, richtig?} Der Junge nickte. \enquote{Wie, wenn ich fragen darf?}

\enquote{Ich sprach gerade mit meinem Vater über den Kamin. Ich war gerade fertig und zog meinen Kopf heraus, als die Flammen plötzlich aufloderten und mir mein Gesicht verbrannten. Kurz darauf kam mein Bruder durch den Kamin, sah mich und brachte mich sofort in das Heilerhaus Sankt Mungo. Aufgrund dieses Unfalls hat mein Bruder sofort eine Sicherung für das Kaminnetzwerk erwirkt \gst Dieser arbeitet in der Flohnetzwerk-Überwachung, oder so \gst Jedenfalls kann so etwas nicht mehr passieren. Es war scheinbar das erste Mal, dass es einen Unfall in dieser Art gegeben hat. Es gab zwar eine Entschädigung, aber mein Auge bekomme ich dadurch nicht wieder.}
\end{rueckblick}


In einem leicht dämmernden Zustand drangen nun andere Gedanken in sie ein.

\begin{rueckblick}
\enquote{Das, was sie so abwertend als schwarze Magie bezeichnen, ist nur eine Einordnung von Menschen, die vorwiegend negative Erfahrungen mit dieser Art der Zauberei gemacht haben. Wie ich schon meinen Schülern begreiflich gemacht habe \gst Magie hat keine Farbe. Ich ließ eine Schülerin zwei Pflanzen mit Feuer niederbrennen. Eine ist daran zugrunde begangen. Die andere verbrannte, aber ihr Samen fiel herab, bohrte sich in die Erde und brachte eine wunderbare Blume hervor. Es war also zweimal derselbe Zauber. Nur einmal sahen wir ihn als weiße Magie an, das andere Mal jedoch als schwarze. Es ist also die Intention hinter einer Tat, die gut oder böse ist.}
\end{rueckblick}

Das war das intensive Gespräch, nachdem Katies Hand wieder gewachsen war.

Langsam verbanden sich die beiden Ereignisse in ihrem Kopf zu einem gemeinsamen Gedanken. \gedanke{Wäre es möglich Mister Allman mit dem Einsatz von schwarzer Magie zu helfen?} Dieser Gedanke nahm in ihrem Kopf immer stärkeren Raum ein. Sie öffnete ihre Augen und starrte für eine Weile an die Decke. Dann griff sie zu ihrem Zauberstab und öffnete wortlos die Tür zu ihrem Badezimmer. Außer Sichtweite, aber dennoch in Hörweite, hing ein Bild in ihrem Wohnzimmer, mit dem sie ihre Gedanken teilen wollte und um eine zweite Meinung bat. \enquote{Alain?}, fragte sie in den Raum hinein \gst Stille \gst \enquote{Alain?}, rief sie nun kräftiger.

\enquote{Ja Poppy, ich bin wach. Was gibt es?}

\enquote{Ich brauche eine zweite Meinung.}

\enquote{Zu einem Fall eines Patienten?}

\enquote{Man kann es so nennen. Es ist aber vielmehr ein Gedanke zu einer möglichen Heilung. Ich weiß nicht mal, ob es möglich ist.}

\enquote{Schieß los meine Liebe.}

\enquote{Du erinnerst dich noch an den Schüler, von dem ich dir erzählt habe\abs?}

\enquote{Du hast mir von vielen Schülern erzählt, Poppy.}

\enquote{Lass mich ausreden. Der Erstklässler dieses Jahr. Der mit dem einzelnen Auge und den vielen Narben im Gesicht.}

Es dauerte eine Weile, aber dann kam die Antwort. \enquote{Ja.}

\enquote{Ich konnte ihm nicht helfen. Er muss weiterhin mit einem Auge leben und die Narben im Gesicht tragen.}

\enquote{Ja, den habe ich gesehen, als ich mir eine Birne aus dem Küchengemälde geholt habe.}

Madame Pomfrey musste schmunzeln.

\enquote{Schmunzelst du etwa?}

\enquote{Nein}, kam nicht sehr ernst. \enquote{Wie kommst du darauf?}

\enquote{Das sehe ich bis hierher.}

Ein belustigtes Schnaufen erklang aus dem Bad heraus.

\enquote{Und du erinnerst dich an den Fall \accentuate{abgetrennte Hand}?}

\enquote{Ja.}

\enquote{Könnte man dem Jungen nicht genauso helfen?}

\enquote{Ihm den Kopf abschneiden und nachwachsen zu lassen?}

\enquote{Nein, die Narben durch \accentuate{schwarze Magie} zu entfernen und ein Auge nachwachsen zu lassen\abs Warte mal, wenn wir eine Augenhöhle formen könnten, dann könnte er doch ein magisches Auge bekommen.}

\enquote{So ein Monstrum? Meinst du nicht, dass das für einen jungen Mann hinderlich sein kann?}

\enquote{Nein, denn es gibt mittlerweile kleinere Ausführungen, die nicht größer als ein normales Auge sind. Optisch also keinerlei Beeinträchtigungen.}

\enquote{Hmmm}, erklang aus dem Wohnzimmer. \enquote{Das könnte gehen. Aber ist es auch ungefährlich?}

\enquote{Darüber muss ich mit meinem Kollegen reden. Vielleicht hat er ein paar Hinweise oder Anregungen. Aber ich möchte es auf jeden Fall selber machen.}

\enquote{Du willst dich mit deinem Kollegen beraten, der im Verdacht steht die dunklen Künste zu praktizieren und mit Du-weißt-schon-wem zu sympathisieren?}

Madame Pomfrey stieg aus der Wanne, zog sich einen Bademantel über und stürmte in ihr Wohnzimmer. \enquote{Rede nicht so über meinen Kollegen. Er ist unheimlich nett, zuvorkommend und weiß sehr viel. Man kann sich mit ihm über fast alles unterhalten. Er hat mir sogar ein paar seltene Zutaten aus seinem Garten zu meinem Namenstag geschenkt. Eine Geste, die heute gar nicht mehr zelebriert wird.} Sie war so in Rage, dass sie gar nicht bemerkte, wie ihr Bademantel sich etwas lockerte und sich ihre Brüste bis kurz vor ihre Brustwarten entblößten. Dem älteren Herrn im Bild entging das nicht und so warf er immer wieder einen sehnsüchtigen Blick auf den Bademantel, in der Hoffnung mehr zu sehen. Doch diese Hoffnung blieb unerfüllt, da Madame Pomfrey schon vor Jahren ihren Bademantel entsprechend verzaubert hatte, dass nichts zu sehen war.

\enquote{Ich habe nie behauptet, dass er\abs Ich habe die andern darüber reden hören\abs Ich\abs ach egal. Frag ihn, wenn er dir helfen kann.}

Poppy nickte und ging zurück ins Badezimmer, um sich ihre Schlafsachen anzuziehen. Zurück im Wohnzimmer nahm sie wieder einmal das kleine Schraubglas vom Tisch und setzte sich in einen Stuhl. Mit geschürzten Lippen besah sie sich die seltene Pflanze in dem Glas, bevor sie es abstellte und nach der Abendtoilette zu Bett ging.

\trenn
\onelineback % Anderenfalls werden 2 Leerzeilen gesetzt

\begin{rueckblick}
\enquote{Ich habe gehört, dass sie einen Tarnumhang haben.}

\enquote{Ja}, antwortete Harry.

\enquote{Dann bringen sie ihn zu ihrer nächsten Stunde mit, dann werden sie lernen, wie sie Objekte die durch ihren oder einen Tarnumhang, durch Desillusionierung, oder durch Tarnungstechniken der Muggel verdeckt sind, entdecken können.}
\end{rueckblick}

Harry war auf dem Weg zu seinem heutigen Training mit Professor Elber, als er über seine Karte nachdachte. \gedanke{Wieso zeigt die Karte Punkte an? Die Rumtreiber kannten die doch nicht. Sind die neu hinzugekommen? Und wieso sind mir die nicht früher aufgefallen?}

Harry trat gerade auf das Quidditchfeld, wo eine Sichtschutzwand aufgebaut war. Auf dem Boden davor waren vier Buchstaben angebracht. \accentuate{L, U, G, A.} Was hinter dem Sichtschutz war, konnte Harry nicht sehen. Sein Lehrer verlangte Harrys Tarnumhang, den Harry nur widerwillig hergab. Dann verschwand sein Lehrer kurz hinter der Wand und kam dann wieder hervor.

Er schob die Wand ein Stück beiseite und gab die erste Aufgabe frei. Harry sah einige Buchswedel, die vor einer Pappwand standen. Was dahinter war, konnte er nicht sehen.

\enquote{Was sehen sie?}, fragte ihn sein Lehrer.

\enquote{Ein paar Buchswedel, die vor einer Pappwand stehen.}

\enquote{Was noch?}

\enquote{Nichts.}

\enquote{Was ist hinter der Wand?}

\enquote{Das kann ich nicht sehen.}

\enquote{Dann setzen sie sich und konzentrieren sie sich.}

Harry sah sich um und entdeckte nur den Sand. Er entschied sich ein Kissen herzuzaubern, damit er bequemer sitzen konnte. Dann setzte er sich auf das Kissen und starrte auf die Zweige. Er überlegte, wie er zum Ziel kam. Immer wieder musste er selber darauf kommen. Manchmal gab es Hilfestellungen, mal musste er selber die Lösung finden und manchmal bekam er eine detaillierte Anleitung. Sein Lehrer setzte sich neben ihn auf ein Kissen, das erschien, sobald er dem Boden nahe genug war. Auch er sah nach vorne und schloss die Augen. Harry tat es ihm gleich.

\enquote{Gut}, sagte Elber. \enquote{Versuchen Sie ihre Gedanken auszublenden.}

Harry wandte die Okklumentik-Kenntnisse an, die er bisher erworben hatte, und versuchte, seinen Geist zu leeren, was ihm nicht ganz gelang. Immer wieder drängten sich ihm Gedanken in den Sinn.

Dann passierte etwas, was Harry nicht erwartet hätte. Er begann Farben und Formen zu sehen. Erst verschwommen, dann immer klarer. Nach und nach begann er die einzelnen Zweige vor seinem inneren Auge zu sehen. Dann kam die Pappwand in sein Sichtfeld. Statisch stand sie für mehrere Minuten vor ihm, bis ihn die Konzentration verließ und er wieder in die Wirklichkeit zurückkehrte.

\enquote{Für den ersten Versuch nicht schlecht}, kommentierte Professor Elber.

\enquote{Wie können sie wissen, wie weit ich gekommen bin und ob ich überhaupt etwas gesehen habe?}

\enquote{Wenn sich jemand Zugang zur Magie verschafft, um Objekte magisch zu untersuchen, dann verändert das das magische Feld um einen herum. Da ich neben ihnen sitze, habe ich das bemerkt. Ich weiß nicht genau, wie weit sie gekommen sind, aber ich habe bemerkt, dass sie es zumindest versuchen. Das lernen sie mit der Zeit automatisch, wenn sie diese Technik erlernt haben.}

Harry war erstaunt. Eigentlich hatte er so viel in der Schule und von Dumbledore gesehen und gelernt, dass er sich vornahm, sich nicht mehr zu wundern, aber das hier brachte ihn doch zum Staunen. Dann begann er zu begreifen, dass es weitaus mehr gab, als man sich vorstellen konnte; oder einfacher gesagt, die Magie wirkt so, wie man es sich vorstellt. Die Vorstellung alleine wirkt einen Zauber. Mann muss nur an das glauben, was man bewirken will.

\gedanke{Dumbledore hat es mir schon einmal gesagt. Ich bin ein Magier mit hohen suggestiven Kräften. Wenn ich glaube, dass mir ein Zauber gelingt, dann wird es das auch. Wenn aber ein Zauber gegen meine Natur ist, dann wird er mir auch nicht gelingen, egal wie sehr ich mich bemühe.}

Jetzt hatte er es begriffen. Er schloss wieder seine Augen und begann erneut. Langsam bildeten sich wieder die Formen und die Farben. Dann kam, als alles scharf war, die Pappwand. Harry überlegte, was er nun tun musste.

\enquote{Laufen Sie}, hörte er aus weiter ferne.

Gedanklich lief er auf die Äste und die Wand zu. Knapp hinter der Pappwand sah er unscharf einen Holzbalken. Doch so sehr er sich auch anstrengte, er konnte den Balken nicht scharf bekommen. Er lief wieder zurück und dachte nach. Er drohte schon wieder die Konzentration zu verlieren. Daher beeilte er sich. Er stellte sich vor, wie die Äste verschwinden. Nach wenigen Sekunden begannen diese zu verschwinden. Danach war die Pappwand dran. Dahinter war ein Holzgestell in Form eines \accentuate{T}-s zu sehen. Harry kehrte zurück, bevor seine Konzentration vollständig nachließ.

\enquote{Ein Holzgestell in Form eines \accentuate{T}}, sagte Harry.

\enquote{Gut.}

Mit einem Handstreich schob er die Sichtschutzwand um ein Schulungsobjekt weiter. Doch da war nichts. Harry konnte nichts mehr sehen.

\enquote{Jetzt wird es schwerer. Aber die Möglichkeiten, die sie haben, sind um ein Viertel gesunken. Jetzt stehen nur noch drei mögliche Schutzmaßnahmen zur Auswahl.}

Harry nickte. Er atmete ein paar mal durch, da er sich sammeln musste.

\enquote{Hunger? \gst Durst?}, fragte sein Lehrer.

\enquote{Nein danke. Momentan nicht. Aber ich müsste mal.}

\enquote{Dann los.}

Harry stand auf und ging schnell sein Geschäft erledigen. Der Sichtschutz folgte ihm um die Objekte herum, sodass er keine Möglichkeit hatte, vorab etwas zu sehen. Auf seinem Rückweg dachte er bereits nach, wie er seine Aufgabe lösen konnte. Er konnte nicht so einfach das Objekt davor verschwinden lassen. Oder doch?

Konnte er einfach durch einen Zauber durchsehen? War es überhaupt ein Zauber?

\gedanke{Ja klar. Es wird immer schwerer. Desillusionierungszauber sind schwer und können bei komplizierten Objekten mit schnellen Bewegungen Verzerrungen hervorrufen. Doch mein Tarnum\aabs Was sagte er? \inner{\aabs Objekte die durch ihren oder einen Tarnumhang\abs} richtig. Er hat einen Tarnumhang. Und dann kommt meiner. Ron hat mir doch damals gesagt, dass sie selten sind. Und alle, die er kennt, seien nicht so gut wie meiner. Die meisten verschleißen nach Jahren. Aber meiner gehörte meinem Dad. Er muss also sehr alt sein. Also muss ich gegen einen Zauber angehen. Ok.}

Er setzte sich wieder.

\enquote{Und? Hatten sie eine Erleuchtung?}

\enquote{Ja}, sagte Harry und grinste.

\enquote{Lassen sie mich teilhaben?}

\enquote{Ich glaube, ich weiß die Art, wie das Objekt verborgen wurde.}

\enquote{Und wie kommen sie darauf?}

\enquote{Ich nutze die Hinweise, die sie mir gegeben haben.}

\enquote{Welche wären das?}

\enquote{Sie sagten mir, welche Hindernisse ich überwinden muss. Außerdem beginnen wir immer mit dem leichten, und kommen dann zu den schweren. Die Muggeltarnung habe ich überwunden. Also wird dies hier\abs} Harry zeigte auf das nicht sichtbare Objekt vor ihm. \enquote{\aabs der Desillusionierungszauber sein. Dann kommen die beiden Tarnumhänge.}

\enquote{Gut, dann machen sie.}

Durch diesen Satz kurzzeitig verunsichert, dachte Harry kurz nach und versuchte den Zauber zu überwinden. Er schloss wieder seine Augen und dachte nach. Dieses Mal musste er etwas erscheinen lassen. Er versuchte nicht, ein Objekt vor seinem geistigen Auge zu fixieren, sondern er versuchte einer magischen Signatur zu folgen. Dies fiel ihm mitten in seiner Konzentration ein. Sein Professor sagte ihm, dass sich das magische Feld ändert, wenn man Zauber wirkt. So versuchte Harry magische Signaturen zu finden. Er näherte sich geistig dem Zielobjekt, bis er etwas spürte. Langsam kribbelte es, dann begann sich das Feld aufzulösen und Harry sah ein Metallgestell in Form eines \accentuate{A}-s.

Wieder in der Realität zurück, sah er, dass er den Zauber aufgelöst hatte. Sein Professor sah ihn ungläubig an. Nach einigen Sekunden fing er sich wieder und fing an zu meckern. \enquote{Sie sollten hinter den Zauber schauen und ihn nicht auflösen.}

\enquote{Entschuldigung}, gab Harry kleinlaut zurück. \enquote{Ich habe das Objekt dahinter aber schon vorher erkannt, bevor ich die Augen geöffnet hatte.}

Professor Elber hob eine Augenbraue und sah ihm direkt in die Augen. Nach einigen Momenten meinte er: \enquote{Na gut. Das will ich Ihnen mal vorerst glauben.} Er nahm seinen Zauberstab heraus und legte wieder den Zauber darüber. \enquote{Dieses Mal aber ohne den Zauber zu lösen. Verstanden?}

Harry nickte und begann erneut.

Als wieder das Kribbeln begann, zog sich Harry zurück, bis es nachließ. Erneut versuchte er den Zauber zu brechen, doch mitten drin unterbrach er sich. Dann fing er an, einer neuen Idee nachzugehen. Er versuchte, um den Zauber herum zu sehen. Doch dies hatte auch nicht den gewünschten Erfolg. Zumindest hatte er jetzt die Ausmaße des Feldes erkannt. Schließlich stellte er sich, aufgrund mangelnder anderer Ideen, vor, wie er durch das Feld schauen könnte. Schemenhaft kristallisierte sich ein Metallgestell heraus, das ein \accentuate{A} darstellte.

Als Harry wieder in der Realität war, sah er nichts. Der Zauber schien noch Bestand zu haben.

\enquote{Knapp. Der Zauber hat etwas gewackelt, was ein eventueller Insasse bemerkt hätte}, merkte sein Lehrer an. \enquote{Aber für Ihren ersten Versuch, sehr gut. Sie scheinen für diese Art der Magie ein Händchen zu haben. Respekt. Ich hätte weniger erwartet.}

Durch dieses Lob angespornt, fühlte sich Harry bekräftigt auch die anderen beiden Aufgaben zu lösen. Und wieder wurde ein Testobjekt freigelegt. Jetzt musste er das verborgene Etwas hinter dem \accentuate{G} erkennen.

\enquote{Aha, der normale Tarnumhang}, rutschte es Harry heraus.

Sein Lehrer lächelte. \enquote{Sie scheinen sich bei ihrer Pinkelpause ja richtig Gedanken über die Art des Schutzes und der Reihenfolge gemacht zu haben. Wie kommen sie darauf, dass das ein normaler Tarnumhang sein soll? Und was ist der Unterschied zum nächsten?}

\enquote{Mein Tarnumhang kommt als letzter. Das hier ist ein normaler, wie er überall zu haben ist. Na ja, relativ überall. Sie sind schon so recht selten.}

\enquote{Und der Unterschied?}

\enquote{Laut meinen Informationen halten normale Tarnumhänge nicht sehr lange. Außerdem lassen ihre Eigenschaften mit der Zeit nach. Meiner dagegen hat schon meinem Vater gehört. \gst Wieso erzähle ich Ihnen das eigentlich?}

\enquote{Weil ich ihnen etwas beibringen will.}

\enquote{Richtig}, sagte Harry gedankenverloren. \enquote{Woher wissen Sie \gst wollen Sie wissen, dass mein Tarnumhang anders ist?}

\enquote{Ich hatte eine sehr interessante Unterhaltung mit ihrem Direktor. Außerdem\abs}

\enquote{Was?}

\enquote{Behalten Sie das bitte für sich, ja?}

\enquote{Ok.}

\enquote{Ich habe eine eigenartige Signatur gespürt, als ich nahe an ihrem Zimmer vorbeiging. Es war außerhalb. Also habe ich mich auf die Suche gemacht. Ich habe etwas in ihrem Koffer gespürt. Da es privat war, verließ ich das Zimmer wieder. Erst danach hatte ich diese Unterhaltung mit Dumbledore. Dann wurde mir klar, was ich gespürt hatte. Ich hatte den Tarnumhang gespürt. Dann reifte die Idee, Ihnen dies beizubringen. Ich wollte einerseits Gewissheit, andererseits muss ich Ihnen viel beibringen. Ich bin nicht lange hier, wissen Sie?}

\enquote{Ja, aber was hat es mit meinem Umhang auf sich? Was ist an ihm besonders?}

\enquote{Lösen Sie erst einmal die beiden Aufgaben. Dann reden wir darüber. Eventuell erst in ein paar Tagen.}

Harry nickte schweren Herzens und machte sich an die Arbeit. Durch den Tarnumhang kam er recht schnell. Es war leichter, als beim Desillusionierungszauber. Dieses Mal bildeten Bambusstangen ein \accentuate{B}. Dann verschwand die Wand und das letzte Objekt verlangte nach einer Lösung. Harry brauchte mehrere Minuten, bis er sich wieder konzentrieren konnte. Danach hatte er Schwierigkeiten, überhaupt etwas zu spüren. Er schlich gedanklich den Bereich ab, in dem er das Objekt vermutete, doch da war nichts, was er erfassen konnte. Er brach ab.

\enquote{Ist da wirklich was?}, fragte er entkräftet nach.

\enquote{Wollen Sie ihn fühlen?}, fragte Professor Elber nach. Harry schaute ihn an und überlegte. \enquote{Bin ich nicht vertrauenswürdig genug?}, fragte Elber halb belustigt, halb emotionslos nach.

\enquote{Das ist es nicht. Oder doch? Ich weiß es nicht. Ich bin entmutigt. Dadurch, dass ich nichts gefunden habe. Da ist nichts.}

\enquote{Bohren Sie tiefer. Spüren Sie Signaturen auf. Sie haben den Vorteil, dass Sie wissen, dass da etwas ist. Dies müssen Sie nutzen.}

\gedanke{Tiefer bohren, das sagt der so einfach.}

\stimme{Was hast du unter deinem Umhang immer gedacht?}, hörte er in seinem Kopf.

\gedanke{Dass mich niemand entdeckt.}

\stimme{Wieso sollte es jetzt anders sein? Das Objekt darunter will auch nicht entdeckt werden. Zumindest schützt der Umhang das Objekt.}

\gedanke{Aber, was ist der Unterschied zu anderen Umhängen?}

\stimme{Andere Umhänge wurden nur mit Zaubern belegt und/oder benutzen Haare eines Demiguise.}

Das half Harry nur bedingt weiter. Aber er hatte einen Anhaltspunkt. Er schloss einfach nur die Augen und lies seine Gedanken schweben. Und wieder sickerten so langsam Ideen und Erkenntnisse in seinen Verstand. \gedanke{Das Objekt will nicht gefunden werden. Der Umhang schützt es aktiv. Ich muss es wie eine Person behandeln. Aber wie will ich mit einem Umhang ein Gespräch beginnen?} Die letzte Frage ließ er einfach frei, als ob er sie in die Welt hinaus denken wollte.

\stimme{Da gibt es reichlich Möglichkeiten}, hörte er.

Intuitiv begann er, die Quelle zu orten. \gedanke{Zumindest habe ich jetzt einen Ort}, dachte er. \gedanke{Wieso hast du mir geantwortet?} Keine Antwort. \gedanke{Hallo? Was interessiert dich?}

\stimme{Tut mir leid Harry. Das war ich. Ich konnte nicht anders.}

\gedanke{Salazar? Du hast mich\abs? Hmpf.}

Er war keinen Schritt weiter. Salazar hatte ihn veralbert.

\gedanke{Das Objekt will nicht gefunden werden. Der Umhang schützt es aktiv}, ging ihm wieder durch den Kopf.

Da er keine Kopfschmerzen bekam, wenn er es nur gedanklich vollzog, schwebte Harry in schneller Folge immer wieder durch den gedachten Mittelpunkt in verschiedenen Richtungen. Langsam begann er einen leichten Widerstand zu spüren. Erstaunt öffnete er seine Augen und sah ganz schwach den Buchstaben \accentuate{A} aus Luftschlangen gebildet. Dann waren seine geistigen Kräfte erschöpft. Er sackte zusammen.

\enquote{Kreacher, ihr Herr braucht sie}, sagte Professor Elber.

Kreacher erschien und blickte sich um. Professor Elber nahm gerade Harrys Tarnumhang vom Gestell und reichte es Harrys Hauself. \enquote{Bringen Sie den Umhang bitte in das Zimmer Ihres Herrn. Ich nehme an, Sie wissen, wo er hingehört.} Kreacher nickte und sah danach zu Harry. \enquote{Ich werde ihn in den Krankenflügel bringen.}

\enquote{Wenn Kreacher darf, er kann es schneller.}

\enquote{Wenn ich mit darf?}

Kreacher nickte und nahm beide bei der Hand, sobald Elber in Reichweite war. Kurz darauf waren sie auf der Krankenstation angekommen und Kreacher entschuldigte sich und verschwand.

\enquote{Poppy, Kundschaft.}

Madame Pomfrey kam aus ihrem Büro und schaute Harry an, der gerade auf ein Bett geschwebt wurde. \enquote{Was ist mit ihm?}

\enquote{Geistig erschöpft. Entweder war ich zu nachlässig und hätte ihn eher aufhalten sollen, oder er hat sich zu sehr verausgabt. Suchen Sie sich etwas aus.}

\enquote{Wie soll ich das verstehen?}

\enquote{Wir sind irgendwie beide an seinem Zustand schuld.}

Dann erwachte Harry für einen kurzen Augenblick. Er sah Professor Elber, lächelte kurz und meinte dann: \enquote{Es war ein \accentuate{A} aus Luftschlangen.} Dann dämmerte er wieder ein.

Für die nächsten drei Stunden schlief er ruhig durch. Madame Pomfrey gab ihm während des kurzen wachen Momentes noch einen Schlummertrunk und ließ ihn dann schlafen.

In einem fremden Bett erwachte er. Er schlug die Augen auf und sah an die Decke. Dann setzte er sich auf und verließ das Bett. In seinem Inneren spürte er die Gewissheit, dass er noch immer im Krankenflügel in Hogwarts lag und schlief. Also hatte er wieder eine Art Vision. Er öffnete die einzige Tür im Raum und trat auf den Gang hinaus. Die Tür fiel hinter ihm ins Schloss und verschwand, denn als sich Harry herumdrehte, war sie nicht mehr vorhanden. Er ging die Treppe in das Erdgeschoss hinunter, da er Stimmen von dort vernahm. Er trat in eine Küche ein. Zwei Gestalten saßen an einem Tisch. Harry konnte keine Gesichter ausmachen. Es war eigenartig. Eine der Gestalten hatte vor sich eine Karte liegen. Harry trat näher heran, wurde aber ignoriert, oder nicht wahr genommen.

\enquote{Was machst du da?}, fragte eine weibliche Stimme. Sie klang verzerrt, sodass er sie nicht zuordnen konnte.

\enquote{Ich trage Sachen nach. Auf dieser Karte sind zwar viele Sachen verzeichnet, aber bei weitem nicht alle. Es fehlen noch einige.}

Harry sah nun genauer hin und entdeckte, dass sie der Karte des Rumtreibers zum Verwechseln ähnlich sah.

\enquote{Warum zeichnest du nur Punkte ein?}, fragte die weibliche Stimme erneut.

\enquote{Damit man sich auf die Suche machen muss, um etwas mehr über das Schloss zu erfahren}, sagte die männliche Stimme.

Auch diese konnte Harry nicht zuordnen. Wie die Stimme der Frau klang auch sie verzerrt.




\begin{kommentar}
Harry war mit seinem Tarnumhang auf dem Weg zum Quidditchfeld, wo er lernen sollte, durch Tarntechniken hindurch Objekte zu erkennen. Auf dem Boden sind bereits Buchstaben und weitere verbergen sich hinter den geschützten Objekten. Zusammen ergeben sie das Wort Tabaluga. Ein Drache mit selbem Namen wird später noch vorkommen. Den Namen gibt es wirklich. Es ist ein kleiner Drache, der vom Sänger Peter Maffay erdacht wurde. Es gibt mehrere Musicals über Tabaluga.
\end{kommentar}

\chapter{Vorbereitungen für das Fest}


Wieder einmal war Samstag und Harry hatte frei. Seine Hausaufgaben waren größtenteils erledigt und das Quid"-ditch-Match gegen Ravenclaw hatten sie knapp verloren, mit nur zehn Punkten Rückstand. Er war nicht schnell genug gewesen, um den Schnatz zu greifen, denn Sekunden vorher hatte Cho ihn gegriffen.

\begin{rueckblick}
\enquote{Und weitere Zehn Punkte für Ravenclaw. Was ist heute nur mit den Gryffindors los}, verkündete Stan Lindsay. \enquote{Dreiiissig für Ravenclaw NULL für Gryffindor. Scheinbar haben sie heute keinen guten Tag erwischt.} Harry besah sich das Spiel von oben und warf Cho immer wieder ein mattes Lächeln zu. Doch plötzlich kam eine Wendung in das Spiel. Knappe zwanzig Minuten später lag Gryffindor mit vierzig Punkten im Vorsprung. Es stand jetzt Dreißig zu Siebzig. \enquote{Na also. Es scheint, als seien die Gryffindors endlich aufgewacht}, brüllte Stan, nachdem sie punktgleich mit Ravenclaw waren. Doch als sie 140 Punkte Vorsprung hatten, passierte es. Cho hielt den Schnatz triumphierend in den Händen. Das war ihr nur ein einziges Mal gelungen, seit sie gegen Harry spielte.
\end{rueckblick}

Damals hatten sie das Spiel trotzdem verloren.

Er war heute nicht gut in Form. Er hatte sich den Schnatz vor der Nase vom Ravenclaw-Sucher wegschnappen lassen. Cho war in ihrem letzten Jahr und strahlte nur so, als sie Harry den Schnatz vor seiner Nase wegschnappte. Gryffindors Quidditch-Mannschaft war deswegen zwar nicht besonders niedergeschlagen, da sie mehr Hauspunkte als Ravenclaw hatten, aber sie waren auch nicht gerade gut gelaunt da sie Harry, wenn auch indirekt, die Schuld gaben. Neben einigem unzufriedenem Murmeln kam kein böses Wort über die Lippen seiner Mannschaftskameraden. Sie befanden sich im Gruppenraum der Gryf\-fin\-dor-Quidditch-Mann\-schaft, um das aktuelle Spiel Revue passieren zu lassen. Harry fiel wieder das Spiel ein, das er bei Arabella gespielt hatte. Ob sie es ihm wohl ausleihen würde? Das könnte seiner Mannschaft helfen, die nächsten Züge zu planen.

\gedanke{Ich glaube, ich werde Arabella mal einen Brief schreiben}, dachte sich Harry, als er von seinen Team-Kollegen aus seiner Träumerei wieder zurück in die Realität geholt wurde.

\enquote{Harry, bist du bei der Sache?}, fragte ihn Katie. \enquote{Du wirktest etwas abwesend.}

\enquote{Ach nichts}, meint er. \enquote{Ich hatte nur so eine Idee wie wir unser Spiel verbessern können. Aber darüber muss ich erst mit jemanden Reden, ob ich es auch bekomme.}

\enquote{Was?}, fragten seine Teamkollegen.

\enquote{Ich sage es euch, wenn ich es habe, falls ich es bekomme. Ich möchte euch nicht unnötig enttäuschen. Nicht nach dem heutigen Spiel.}

Nach der Abschlussbesprechung war es auch schon wieder Zeit für das Mittagessen, und so ging Harry, nachdem er im Gruppenraum der Quidditch-Mannschaft geduscht und sich umgezogen hatte, mit dem Rest der Mannschaft zurück zum Schloss. Als er in die Große Halle zum Essen ging, standen Professor Dumbledore und Professor McGonagall da und unterhielten sich mit Professor Elber.

\enquote{Dann ist es abgemacht, sie kümmern sich dieses Jahr darum}, sagte Professor Dumbledore und ging mit Professor McGonagall in die Große Halle. Professor Elber folgte ihnen und setzte sich auf seinen Platz. Harry wunderte sich, was sie wohl zu bereden gehabt hatten. Nach dem Mittagessen schaute Harry in der Bibliothek vorbei, da er noch seine Hausaufgabe bei Professor Sprout machen musste. Harry konnte nicht erkennen, was Professor Elber da las, aber als er sich näherte, um ein anderes Buch aus dem Regal zu holen, schlug Professor Elber sein Buch zu und wartete bis Harry wieder wegging. Harry nahm sein Buch und setzte sich an einen Tisch auf der anderen Seite, an dem schon Ginny und Parvati saßen. Er flüsterte ihnen zu. \enquote{Professor Elber hat wohl irgendetwas vor. Er schlug sein Buch zu, als ich ein Buch aus dem Regal in seiner Nähe geholt habe.} Beide wunderten sich, sagten aber nichts.

Am nächsten Tag sah Harry wie Professor Elber mit diesem Buch durch die Gänge lief und sich im Schloss umschaute. Gerade so, als ob er etwas suchen würde. Er wollte schon nachfragen, als er von Hermine aufgehalten wurde die ihn fragte: \enquote{Hast du schon ein Weihnachtsgeschenk für Ginny?}

Harry verneinte und fragte Hermine: \enquote{Hast du irgendeine Idee? Bleiben sie und Ron überhaupt über Weihnachten?}

\enquote{Ja}, antwortete Hermine. \enquote{Ich bleibe auch. Ich habe mich schon eingetragen. Und du? Fährst du Heim?}

\enquote{Natürlich nicht. Glaubst du, ich will Weihnachten bei den Dursleys verbringen?}

\enquote{Na ja, ich frage nur, weil du dich nicht eingetragen hast.} Harry erschrak und rannte sofort zur Großen Halle, wo die Anschläge hingen. Er trug sich gerade ein, als Hermine um die Ecke kam. Sie war noch vollkommen außer Puste und meinte: \enquote{So eilig musst du es auch nicht haben.}

\enquote{Du kennst meinen Onkel und meine Tante nicht. Wenn ich mich nicht rechtzeitig eintrage, dann fahre ich zurück. Und glaube mir, die wären nicht begeistert.}

Hermine grinste und legte eine Hand auf seine Schulter.

Kurz darauf machte er sich auf den Weg zu Professor Elbers Büro, denn er wollte wissen, was es mit seinem Tarnumhang auf sich hat. Er klopfte, doch es tat sich nichts. Er wollte sich gerade umdrehen und gehen, als die Tür aufging. Entgegen besserem Wissen trat er vorsichtig ein und sah sich um. Er stand noch so, dass die Tür nicht zufallen konnte und lies seinen Blick schweifen. Im Raum waren wenig persönliche Sachen zu finden. Auf einem Sideboard stand ein Bild von Elfen.

\gedanke{Wahrscheinlich seine Elfen, oder die seiner Eltern, die ihn mit aufgezogen haben}, dachte er sich. \gedanke{Habe ich das Bild das letzte Mal übersehen oder steht es erst jetzt da?}

Sein Blick wanderte weiter durch den Raum und blieb über dem Schreibtisch hängen. Dort hingen noch immer die vier Bilder der Gründer von Hogwarts. Rowena Ravenclaw, Salazar Slytherin, Godric Gryffindor und Helga Hufflepuff. Auf der anderen Seite des Raumes, kurz bevor die Tür anfing, sah er ein Bild einer jungen hübschen Frau.

\gedanke{Seine Schwester? Tante? Frau?}, fragte sich Harry.

Plötzlich spürte er eine Präsenz hinter sich. Er war ruhig genug, nicht zu erschrecken. \gedanke{Snape oder Filch können es nicht sein, die hätten schon längst los gebrüllt. Ron und Hermine hätten gefragt, was ich hier mache. \gst Dumbledore? Elber?} Harry dachte nach.

Er versuchte sich zu konzentrieren und hinter sich zu blicken. Als er anfing das verschwommene Bild klarer wahrzunehmen, dämmerte es ihm, dass es nur Elber sein konnte. Ein anderer hätte sich schon längst bemerkbar gemacht. Er öffnete seine Augen und drehte sich während der Ansprache um.

\enquote{Guten Tag, Professor Elber.}

Doch hinter ihm war niemand.

\enquote{Hinter Ihnen}, hörte er.

Erschrocken drehte er sich erneut um. \enquote{Professor? Wie kommen Sie\abs?}

\enquote{Setzen Sie sich.}

Harry setzte sich auf den ihm angebotenen Stuhl und wartete.

\enquote{Wissen Sie, das möchte ich mit einigen von ihnen durchgehen, um am Ende des Schuljahres eine kleine Demonstration zu vorzuführen. Netter Effekt, oder? Man zeigt sein Abbild an anderer Stelle. Das verwirrt den Gegner.} Dann verschwand das Bild und Harry hörte nun wieder hinter sich eine Stimme. \enquote{Sie haben mich richtig erkannt. Ich stehe aber immer noch hinter Ihnen.}

Harry drehte sich auf dem Stuhl um und sah zur Tür. Dort stand Professor Elber und betrat gerade den Raum, um sich auf seinen Platz hinter dem Schreibtisch zu setzen.

\enquote{Sie wollen was von mir?}, fragte er nach.

\enquote{Ja Professor. Sie sagten mir\abs wollten mir etwas über meinen Tarnumhang erzählen.}

\enquote{Haben Sie ihn dabei?}

\enquote{Nein, leider nicht. Aber ich kann ihn holen.}

\enquote{Sie wissen, wie er sich anfühlt?}

\enquote{Ja, sehr genau sogar.} \gedanke{Ups.}

\enquote{Gut. Im linken Sideboard, von Ihnen aus gesehen, in der obersten Schublade ist ein anderer. Schauen Sie ihn sich an, fühlen Sie ihn und legen Sie ihn dann wieder zurück.}

Harry nickt und stand auf. Er öffnete die Schublade und sah den Tarnumhang. Er nahm ihn heraus und fuhr über den Stoff. Dieser fühlte sich anders an.

\enquote{Darunter liegt ein Buch. Holen Sie es heraus, bevor Sie den Umhang hineinlegen und lesen Sie eine der Geschichten.}

Harry legte nach einer Weile den Umhang in die Schublade und zog das Buch darunter hervor. Dann schloss er die Schublade und setzte sich wieder auf den Stuhl. Professor Elber korrigierte unterdessen Schulaufgaben der Drittklässler. Harry erkannte es an dem Irrwicht, über den sie eine Abhandlung schreiben mussten. Einige Zettel lagen wild verteilt auf dem Tisch. Auf einigen war nur wenig Rot zu sehen. Bei den meisten waren nur ein paar Anmerkungen zu finden und zwei hatten sogar mehr rot als schwarz auf ihrem Pergament, vermutete Harry. Er war respektvoll genug nicht zu lesen, was darauf stand.

Er schlug gerade das Buch auf, als es selbstständig weiter blätterte und eine bestimmte Seite aufschlug.

\enquote{Das werden Sie auch noch lernen}, sagte sein Professor zwischen zwei Sätzen, die er schrieb, ohne aufzusehen.

Harry begann zu lesen.

\begin{buch}
Die Geschichte der drei Brüder

Es waren einmal drei Brüder, die in der Abenddämmerung spazieren gingen. Sie kamen an einen Fluss, der\abs
\end{buch}

Als er die Geschichte fertig hatte, schlug sich das Buch automatisch zu und flog zurück an seinen Platz. Harry dachte darüber nach.

\enquote{Was halten Sie davon?}, fragte sein Professor nach einer Weile.

\enquote{Sie wollen mir damit etwas sagen. Aber ich weiß noch nicht genau was. Dieser Umhang des Todes\abs}

\enquote{Vom Tod}, korrigierte ihn sein Lehrer.

\enquote{\aabs vom Tod. Meinen Sie, das könnte mein Umhang sein?}

\enquote{Die Idee ist gut, aber die Richtung ist falsch. Was ist mit den anderen Sachen?}

\enquote{Den Stab hätte man sicher schon gefunden. Ich meine, einen unbesiegbaren Zauberstab?}

\enquote{Warum hätte man ihn finden sollen?}

\enquote{Na ja, wenn sich jemand duelliert und ständig gewinnt?}

\enquote{Und wenn sich die Person nicht duelliert? Wie findet man ihn dann? Wenn die Person schlau genug ist\abs}, er sah auf und Harry an, \enquote{\aabs und nicht durch die Welt reist und überall herumerzählt: \inner{Ich habe den unbesiegbaren Zauberstab, komm und fordere mich heraus.} Was ist dann?} Das hatte Harry nicht bedacht. \enquote{Was ist mit dem Stein?}

\enquote{Der Stein, der lebende zurückholt? Der Stein der Auferstehung?}

\enquote{Holt er sie zurück?  Sind die Geister in Hogwarts alle durch den Stein zurückgeholt worden? Und was ist mit Geistern anders wo auf der Welt?}

Harry dachte lange nach. Professor Elber hatte ihm Rätsel aufgegeben. Dabei wollte er doch nur etwas über seinen Umhang erfahren. Er bemerkte nicht, wie Professor Elber etwas auf zwei Zettel schrieb und diese in die Luft hielt. Dann ließ er sie los, dass sie dort schwebten und verließ sein Büro. Minuten später tauchte Harry aus seiner Denkstarre wieder auf und sah sich irritiert um, da er seinen Professor nicht mehr fand. Dann entdeckte er die beiden Zettel. Es war eine Ausgangserlaubnis, die ihn frei von Ärger halten sollte, falls er auf seinem Rückweg zum Gemeinschaftsraum erwischt werden sollte. Der andere war eine Anweisung an Madame Pince, dass sie Harry ein bestimmtes Buch aushändigen möge. Den Titel konnte er nicht lesen. Er schien kodiert zu sein.

Er nahm beide Zettel, steckte sie in seine Taschen und macht sich auf den Weg Richtung Gemeinschaftsraum. Unterwegs traf er Professor Sprout, der er den Zettel zeigte und die ihn den restlichen Weg begleitete. Sie unterhielten sich über alles Mögliche und Harry redete sich einfach das, was ihn bedrückte, von der Seele.

\enquote{Ja, manchmal ist er schon komisch}, pflichtete ihm Professor Sprout bei. \enquote{Er ist halt doch ein komischer Kauz. Aber nicht immer. Neulich hat er doch im Lehrerzimmer gesagt, als ihn jemand gefragt hat, was er so in seiner Freizeit mache: \inner{Wissen sie, manchmal mache ich einfach, dass Luft schlecht riecht. Es ist ungeheuer entspannend. Zwar muss ich in seltenen Fällen danach lüften, aber das macht nichts. Es gibt ja noch andere Räume.} Ein anderes Mal hat er einfach seine Hobbys aufgezählt, oder anerkennend genickt, wenn jemand über seines erzählt hat und gemeint: \enquote{Ja, das mache ich ab und an. Es ist richtig befriedigend.} Er meinte damit Gartenarbeiten. \gst Oh, wir sind schon da. Dann guten Schlaf Mister Potter.}

\enquote{Ihnen auch Professor Sprout.} Harry gab ihr die Hand und schüttelte sie. Professor Sprout hatte körperliche Nähe gern. Die Hände schütteln, einen Knuff auf die Schulter oder eine Umarmung. Das ließ sie wieder aufleben.

Harry drehte sich einmal um, um sich zu vergewissern, dass sie allein seien. \enquote{Schon schön, das Schloss bei Nacht.}

\enquote{Ja, das ist es.}

Dann umarmte er sie, sagte der Dame das Passwort und stieg durch das Loch in der Wand. Jetzt musste er sich noch weiter gedulden, bis er endlich etwas über seinen Umhang erfuhr.

\trenn

Keiner der beiden hörte die jungen Schritte über den steinernen Boden der Krankenstation laufen.

\enquote{Frederick, weshalb ich Sie habe kommen lassen \gst Sie kennen Philip Allman? Hufflepuff! Narbiges Gesicht und nur ein Auge.}

\enquote{Ja Poppy, kenne ich}, antwortete Elber.

\enquote{Ich habe mir gestern Abend etwas durch den Kopf gehen lassen. Sie haben doch Katie geholfen ihre Hand wiederzubekommen\abs}

Elber hob die Hand und unterbrach sie. \enquote{Ich weiß, worauf Sie hinaus wollen. Ich habe deshalb nichts gesagt, weil \gst na ja, schwarze Magie \gst und mir ist es zu schwierig.}

Plötzlich klopfte es an der Tür.

\enquote{Herein}, sagte Madame Pomfrey.

Die Tür öffnete sich und Philip Allman trat ein.

\enquote{Madame Pomfrey? Ich habe eine\abs Oh entschuldigen Sie, ich habe nicht mit Besuch gerechnet. Ich gehe wieder.}

\enquote{Bleiben Sie ruhig hier und setzen Sie sich.} Madame Pomfrey bot Philip einen Platz an.

\enquote{Danke}, antworte der Junge und setzte sich auf einen Stuhl.

\enquote{Ich bat gerade Professor Elber um Rat, wegen ihres Unfalls, aber er sagte mir gerade, dass da nichts\abs}

\enquote{Poppy, das habe ich nicht.}

\enquote{Aber, Sie haben doch gesagt, dass es\abs}

\enquote{Mir ist es zu schwierig, da ich nicht die notwendigen medizinischen Kenntnisse habe. Die Narben wären nicht das Problem. Aber die Augenhöhle auszubilden. Und wenn das mit Magie nicht funktioniert, dann muss das ein plastischer Chirurg der Muggel machen. Ganz zu schweigen von der Tatsache, dass man auf jeden Fall ein Auge nachwachsen lassen muss. Und zwar in einer Nährlösung mit Unterstützung durch Magie, da die Muggel das nicht können. Aber ein Auge durch Magie nachwachsen zu lassen und es richtig mit dem Gehirn, bzw. einem noch vorhandenen Sehnerv zu verbinden und mit dem Gehirn zu verschalten, übersteigt mein Können bei weitem. \gst Manchmal habe ich das Gefühl, ihr traut mir alles zu. Gut, meine Möglichkeiten mögen vielfältiger sein, als die jedes anderen, aber auch Magie hat ihre Grenzen. Und ohne Unterstützung durch Muggel ist da nichts zu machen.}

\enquote{Dann ist Hilfe möglich?}, fragte Philip zaghaft nach.

Professor Elber kratzte sich kurz am Kinn und antwortete dann: \enquote{Sagen wir mal so. Madame Pomfrey muss erst einmal den Rest ihrer Augenhöhle untersuchen. Ich kann ihr die passenden Bücher geben. Ich selber trau mir das nicht zu. Zu viel könnte bei meinem Eingreifen so nah am Gehirn schiefgehen \gst wenn das erledigt ist, dann kann man die Augenhöhle der intakten angleichen \gst Das wird nicht angenehm; es wird beißen und jucken und sie dürfen nicht kratzen. Das dauert ein paar Tage, bis das Wachstum der Zellen angeregt und in die richtigen Bahnen gelenkt wurde. \gst Dann wird sie Madame Pomfrey zu einem Augenspezialisten der Muggel \gst es gibt ein paar Squibs \gst bringen, der ihnen eine Gewebeprobe entnehmen wird, damit man daraus in einer Nährlösung ein neues Auge mit magischer Unterstützung heranzüchten kann. Beim Augenarzt wird ihnen dann ein Verband auf das Auge gelegt werden und sie werden ihn etwa drei Tage lang auf dem Auge behalten müssen.}

\enquote{Soviel zum Thema: \enquote{Ich habe davon keine Ahnung.}}

\enquote{Ich habe mich etwas eingelesen und mit einem Arzt gesprochen, das gebe ich zu, aber nach deiner Reaktion bei Katie, war ich noch mehr entmutigt, dir etwas in der Richtung vorzuschlagen.}

\enquote{Ich sehe jetzt nichts, was da mit schwarzer Magie zu tun hat.}

\enquote{Das kann ich dir sagen, Poppy. Das Auge, oder ein anderes Körperteil, mithilfe von Magie heranzuzüchten, hat große Ähnlichkeiten mit der Erschaffung eines Inferius. Wer immer das tut, hat die Kontrolle über das Teil und somit dann über ein Auge Philips hier. Durch speziellen Einsatz von Zaubern und der festen Absicht, die Kontrolle in die richtigen Bahnen zu lenken, nämlich die Kontrolle des Eigentümers des Auges, verliert das ganze seine schwarze Seite. Aber das sehen nicht alle so.} Dann drehte er sich zu Philip und sagte: \enquote{Überlegen Sie es sich. Besprechen Sie das Ganze mit ihren Eltern. Schreiben Sie ihnen und erklären es ihnen, oder noch besser, wenn der Schulleiter zustimmt, besuchen Sie sie über das Wochenende. Und Sie, Poppy, denken bitte darüber nach, denn ich werde Ihnen nur assistieren können.}

Beide schluckten schwer nach diesen Ausführungen und hingen schweigend und sich gegenseitig anblickend ihren Gedanken hinterher.

\enquote{Würden sie es machen?}, fragte Philip vorsichtig.

Madame Pomfrey sah ihn immer noch an und nickte schließlich. Glücklich erhob sich der Junge und trat nachdenklich aus dem Raum.

\enquote{Was glauben Sie, wird er es machen}, fragte Madame Pomfrey.

\enquote{Ich weiß es nicht. Ich habe von seiner Familie gehört, dass sie nicht gerade begeistert war, dass er nach Hufflepuff gekommen ist. Und soweit ich weiß, steht kaum jemand den schwarzen Künsten offen gegenüber. Ich habe Tendenzen zu Rassismus in seiner Familie über Gerüchte gehört. Und einige Zweige streben nach reinem Blut. Ich weiß nicht, was sie machen werden. Ich hoffe, seine Eltern sind stark genug. Sollten sie ablehnen, dann können wir immer noch einige Vorbereitungen treffen, die ich als medizinisch notwendig deklarieren würde. Das müssen sie natürlich verantworten und unterstützen. Und wenn er volljährig ist, oder ein entsprechendes Reifezeugnis abgelegt hat, dann kann er seine Eltern überstimmen, was ihn aber seine Familie kosten könnte. Kurzum, ich weiß nicht was passiert, habe keine Ahnung und warte einfach nur ab. Ein Gespräch mit den Eltern wird trotz allem kommen, schätze ich.}

Keiner der beiden merkte, dass der Junge noch draußen vor der Tür stand und sich den Dialog anhörte, bevor er leise aus dem Krankenflügel verschwand, um in sein Zimmer zu gehen und erst einmal zu weinen. Ob aus Trauer oder vor Glück konnte er selbst nicht mehr genau sagen, als er am anderen Tag erwachte. Er schrieb seinen Eltern einen langen Brief und erklärte ihnen alles, was er noch in Erinnerung hatte, und wartete auf deren Antwort.

\trenn

Nachdem Harry wieder einmal seine Hausaufgaben gemacht hatte, nahm er sich die Zeit, Arabella einen Brief zu schreiben.

\begin{brief}
Liebe Arabella,

wie du sicher weißt, bin ich immer noch in der Quidditch-Mannschaft. Zurzeit haben wir aber große Probleme mit unserem Training und auch unsere Taktik ist zur Zeit der, der anderen unterlegen. Ich möchte dich daher bitten mir dein Mini-Quidditch-Spiel auszuleihen, damit wir besser trainieren können.
\end{brief}

Jetzt fiel ihm noch ein, dass er Arabella etwas schenken könnte, da in ein paar Tagen die Weihnachtszeit begann. Er überlegte eine Weile, bis er wieder seine Feder in sein Tintenfass stecken wollte, um seinen Brief abzuschließen, da ihm nichts mehr einfiel. Er war kurz vor seinem Tintenfass, als er plötzlich innehielt und sein Tintenfass anstarrte. \gedanke{Ich habe ja noch meine Spezialtinte, die beim Schreiben und nach dem Trocknen ständig die Farbe ändert.} Er stand auf und ging nach oben, um das Tintenfässchen zu suchen. Nachdem er es gefunden hatte, ging er wieder nach unten und setzte sich an seinen Platz im Gemeinschaftsraum, um seinen Brief weiterzuschreiben.

\begin{brief}
Ich habe dir außerdem noch etwas eingepackt, das du aber erst an Weihnachten öffnen darfst.
\signumspace
Liebe Grüße\\
Harry
\end{brief}

Nachdem er seinen Brief fertig hatte, legte er ihn zum Trocknen beiseite und packte währenddessen sein Geschenk an Arabella ein. Danach rollte er seinen Brief zusammen und band ihn an das Päckchen. Er verließ den Gemeinschaftssaal und machte sich auf den Weg zur Eulerei. Kurz darauf kam er wieder zurück und lief in sein Zimmer. Er hatte den für Hedwig obligatorischen Eulenkeks vergessen. Oben in der Eulerei traf er auf Donan.

\enquote{Hi Donan.}

\enquote{Hi Harry}, antwortete Donan. \enquote{Hast du schon bemerkt, dass zwischen Hermine und Ron aus deinem Haus etwas läuft?}

Harry schaute ihn nur an. \enquote{Bist du dir sicher? Es deutet zwar einiges darauf hin, aber falls da was laufen sollte, verstecken sie es ziemlich gut. Ich versuche die nächsten Tage mal was herauszubekommen.}

\enquote{Gut}, antwortete Donan und verließ die Eulerei.

\enquote{Hallo Hedwig}, begrüßte Harry seine Eule, die er in den letzten Wochen weniger besuchte, da er wenig Zeit hatte. \enquote{Ich habe ein Päckchen für Arabella, bring es ihr bitte.} Er griff in seine Tasche und gab Hedwig den Eulenkeks. Danach streichelte er ihr Gefieder und wünschte ihr einen guten Flug. \enquote{Du kannst dich bei Arabella eine Weile ausruhen und aufwärmen, falls du es möchtest. Es ist zurzeit sehr kalt draußen. Ich werde wohl die nächsten Tage keine Briefe versenden.}

Harry verließ die Eulerei und machte sich auf den Weg zurück in das Schloss. Als er so seinen Blick schweifen ließ, sah er Professor Elber, wie er sich mit Hagrid neben einer Reihe gefällter Bäume unterhielt. \gedanke{Aha, es geht also los mit der Schmückerei}, dachte Harry. \gedanke{Wahrscheinlich kümmert er sich um die weihnachtliche Gestaltung.}

Dann machte er sich auf zur Bibliothek. Dort angekommen suchte er erst einmal Madame Pince, da sie nicht an ihrem Platz war. In einem der hinteren Gänge wurde er fündig.

\enquote{Madame Pince? Ich habe hier einen Anforderungsschein.}

\enquote{Lassen Sie mal sehen.} Sie nahm den Zettel an sich, schaute kurz und fuhr dann mit ihrem Zauberstab darüber. Stöhnend meinte sie: \enquote{Was der immer für ausgefallene Wünsche für seine Schüler hat. \gst Folgen Sie mir, Mister Potter.} Sie lief durch die Gänge Richtung verbotener Abteilung. Kurz vorher bog sie ab. Sie sah die Regale entlang und zog dann ein Buch heraus. \enquote{Dieses hier?}, fragte sie. Als Harry nichts sagte, schlug sie die Augendeckel nieder und atmete einmal ein und wieder aus. \enquote{Sie wissen nicht, was Sie lesen sollen, habe ich recht?}, fragte sie mit geschlossenen Augen.

\enquote{Ja, ich habe schon geschaut, konnte den Text aber nicht entziffern.}

Sie öffnete wieder ihre Augen. \enquote{Und das, wo sie gerade unterrichtet werden, hinter Tarnungen zu sehen.} Dabei hob sie eine Augenbraue. Dann fing sie an, zu ihrem Platz zu laufen, um das Buch auf Harry einzutragen.

\enquote{Woher wissen sie davon?}

\enquote{Wir haben uns unterhalten. Er suchte gestern Abend noch ein Buch. Dabei sind wir ins Gespräch gekommen.}

\enquote{Sie wussten also, dass ich komme und das Buch hole?}

\enquote{Ich ahnte es, als er mir sagte, dass Sie etwas über ihre Vergangenheit erfahren wollen.}

\enquote{Über meine Vergangenheit? Moment mal. Ich will etwas über meinen\abs} Harry stockte. \enquote{Ja, was eigentlich?}

In der Zwischenzeit hatte sie die Ausleihkarte ausgefüllt und gab nun Harry das Buch in die Hand.

\enquote{Danke}, sagte Harry und verließ die Bibliothek.

Zurück in seinem Zimmer überfiel ihn eine Müdigkeit, die ihn zwang ins Bett zu gehen. \gedanke{Scheinbar muss ich mich noch schonen. Das hat mich gestern schon geschlaucht.} Er krabbelte ins Bett und schloss die Augen.

Als er wegdämmerte, zogen sich seine Vorhänge wie von selbst zu und er glitt ins Reich der Träume.

\begin{traum}
Zusammen mit seinen Brüdern Ron und Neville ging er in der Abenddämmerung spazieren. Noch immer trauerten sie um ihre Eltern, die sie nie kennengelernt hatten. Nach einer Weile kamen sie an eine Flussströmung. Um durchzuwaten oder zu schwimmen, war er zu tief und zu schnell. Also zauberten sie sich eine Brücke aus Stämmen, Ästen und Zweigen hervor. Nacheinander gingen sie über die Brücke.

Kaum hatten sie zwei Drittel der Strecke geschafft, als eine Gestalt vor ihnen erschien und stellte sich als der Tod vor. Da sie es geschafft hatten, ihm zu entkommen, denn für gewöhnlich ertranken die Personen im Fluss und er konnte sie sich einverleiben, stellte er jedem von ihnen einen Preis in Aussicht.

Die drei liefen noch über die Brücke auf den sicheren Grund und konnten dem Tod ihre Wünsche vorschlagen. Neville wünschte sich einen Zauberstab, der von keinem andern geschlagen werden konnte. So wollte er sich am Tod seiner Eltern rächen. Also formte der Tod einen Zauberstab aus einem nahe stehenden Wacholderbusch.

Dann kam Ron an die Reihe. Er wünschte sich nur seine Eltern zurück, also verlangte er vom Tod die Möglichkeit, Menschen zurückholen zu können. Der Tod nahm einen Stein vom Flussufer und gab ihn ihm.

Dann kam Harry dran. Er wünschte sich die Möglichkeit, sich vor allem und jedem verbergen zu können. Also gab ihm der Tod ein Stück seines eigenen Umhanges.

Glücklich, dem Tod etwas abgeluchst zu haben, gingen sie weiter.
\end{traum}

Harry schlug seine Augen auf und sah an den Baldachin seines Bettes. Die Vorhänge gingen wieder auf, was Harry aber nicht registrierte. Er wunderte sich über seinen eigenen verzogenen Traum und schüttelte den Kopf. Wie konnte er nur so einen Mist träumen, wunderte er sich. Wieder zog die Geschichte durch seinen Kopf. Die drei Gegenstände \gst Heiligtümer des Todes \gst sah er nun klar vor seinem geistigen Auge. Ein eckiger Stein, seinen Umhang und einen Zauberstab mit größer werdenden kugelförmigen Ausbildungen.

\gedanke{Den Stab habe ich doch schon einmal gesehen}, durchfuhr es Harry. Er überlegte fieberhaft, wo. Immer wieder sah er eine Hand, die den Stab hielt. Um sich abzulenken, durchstöberte er seinen Koffer und nahm seinen Tarnumhang in die Hand. \gedanke{Ja, das ist er. Aber ist er wirklich vom Tod?} Er setzte sich wieder auf sein Bett und legte seine Brille auf seinem Nachtschränkchen ab. Dann fuhr er mit beiden Händen über sein Gesicht. Als er seine Brille wieder aufsetzen wollte, bemerkte er das Buch, das er gerade eben aus der Bibliothek geholt hatte.

Er nahm es mit nach unten und begann es zu lesen. Schon nach wenigen Minuten hatte er zwei Mitleser. Ginny und Hermine schauten ihm zu und fragten ihn, was er denn lese.

\enquote{Etwas über meine Vergangenheit, schätze ich. Ich bin noch nicht sicher. Ich wollte eigentlich etwas anderes erfahren.}

\enquote{Was?}, fragte Ginny nach.

\enquote{Etwas über\abs} Durfte er darüber überhaupt sprechen? Er dachte kurz nach. \enquote{\aabs meinen Tarnumhang}, machte er leise weiter.

\enquote{Den hast du doch von deinem Vater geerbt.}

\enquote{Ja schon, aber woher hat er ihn? Er sieht ziemlich alt aus. Und ehrlich gesagt denke ich, dass er es ist.} Er widmete sich wieder seinem Buch.

Teilweise war es eine Biografie der Familie Peverell, teilweise eine Analyse ihres Schaffens. Es waren sehr begabte Magier und Erfinder. Einer von ihnen war ein Zauberstabmacher. Ein anderer war Auror und musste sich immer wieder Tarnen. Ein dritter war ein Träumer. Er redete immer wieder davon, mit den Geistern der Vergangenheit sprechen zu können.

\gedanke{Das war es doch}, dachte er sich. \gedanke{Die drei Brüder aus dem Märchen könnten die Peverells sein. Und diese haben die Gegenstände erschaffen. \gst Wenn der Tarnumhang immer innerhalb der Familie weitergegeben wurde, dann wäre ich ein Nachfahre\abs} Harry lehnte sich zurück. Er blätterte nun durch das Buch und las gelegentlich wieder einen Abschnitt. Er schwankte zwischen Sicherheit und Unsicherheit. Er müsste wohl noch einmal, oder besser gesagt endlich einmal, mit Professor Elber sprechen.

\enquote{Ich muss noch einmal kurz weg.} Er stand auf und ging zum Porträt. Er blieb stehen und drehte sich erneut um. \enquote{Wenn ich wieder komme, dann werde ich hoffentlich schlauer sein und euch davon erzählen können.} Dann ging er endgültig. Er suchte das Klassenzimmer für Verteidigung auf, doch es war verschlossen. \enquote{Na toll, jetzt ist er nicht da und abgeschlossen hat er auch noch.}

Dann dachte er nach. \gedanke{Lehrerflügel. Es käme auf einen Versuch darauf an.} Harry machte sich also auf den Weg zum Lehrerflügel, in der Hoffnung fündig zu werden.

Als er um die Ecke bog, sah er gerade noch, wie Professor Dumbledore auf den Teppich drückte, der den Zugang versperrte. \enquote{Nimmst du mich mit? Ich muss Professor Elber sprechen. Es ist wichtig.}

Dumbledore sah zu Harry und nickte kurz. Dann bot er ihm den Vortritt an und trat hinter ihm in den Durchgang. Der Teppich versperrte wieder den Weg und beide gingen nebeneinander zur richtigen Tür. Es ging über ein paar Abzweigungen, bis sie dort waren.

\enquote{Was möchtest du denn von ihm, wenn ich fragen darf?}

Harry erinnerte sich wieder, was man ihm mal gesagt hatte, und antwortete deshalb: \enquote{Fragen kannst du mich alles\abs}

\enquote{Nur erhältst du nicht immer eine Antwort}, ergänzte Albus.

Harry grinste. \enquote{Ich möchte ihn etwas über meinen Tarnumhang und meine Vorfahren fragen. Ich habe vor ein paar Tagen geübt, durch Tarnungen zu sehen. Besser gesagt, sie zu erkennen.}

\enquote{Was hat das mit deinem Tarnumhang zu tun?}

\enquote{Das erste Hindernis waren einfach Buchszweige und eine Pappe. Dann kam ein Desillusionierungszauber. Und die letzten beiden waren ein Tarnumhang und dann meiner. Der letzte hat mich umgehauen. Madame Pomfrey gab mir einen \accentuate{Schlaflosen}.}

\enquote{Ah, deshalb warst du auf der Krankenstation.}

Harry sah Dumbledore an und meinte dann: \enquote{Immer bestens informiert, richtig?}

Dumbledore nickte nur und grinste nun ebenfalls. \enquote{Ich erfahre von Poppy nur, wer und wann bei ihr auf der Krankenstation liegt. Aber nicht warum. Das ist meist geheim. Außer sie braucht Hilfe.} Dann waren sie angekommen. Dumbledore zeigte auf die Tür und verabschiedete sich. Er lief den Gang wieder zurück. \enquote{Viel Spaß.}

Harry wollte gerade klopfen, als die Tür aufging und Professor Elber herauskam. Er rannte Harry fast um. Gerade noch konnte er den strauchelnden Harry auffangen und von größerem Schaden bewahren. Seine Bücher, die er in der Hand hielt, blieben neben ihm schweben, nachdem er sie los gelassen hatte, um Harry zu fangen.

\enquote{Was machen Sie denn hier?}

\enquote{Ich wollte Sie sprechen. Es geht um unser Gespräch von kürzlich}, sagte er, da eine andere Lehrerin gerade vorbeilief.

\enquote{Ah Aurora. Hier sind Ihre Bücher. Ich wollte gerade zu Ihnen kommen und sie vorbeibringen.}

\enquote{Hi Frederick, ich habe leider keine Zeit. Ich muss weg. Aber die Bücher dürften doch ihren Weg allein finden, oder?}, sagte sie und zwinkerte ihm zu.

\enquote{Wenn sie dürfen, dann finden sie ihren Weg bis ins Regal. Ansonsten nur bis vor die Tür.}

\enquote{Was muss ich tun?}

\enquote{Es ihnen erlauben. Sagen Sie es ihnen einfach, dann klappt das schon.} Er schob die Bücher in ihre Richtung.

\enquote{Ihr könnt zu mir und euch ins Regal stellen}, sagte sie und die Bücher flogen von dannen. Man hörte, nachdem die Bücher um die Ecke geflogen waren, eine Tür auf und wieder zu gehen. Aurora lief weiter und verabschiedete sich im Laufen.

\enquote{Kommen Sie rein Harry}, sagte Professor Elber.

Harry folgte ihm und stand nun in Professor Elbers Wohnung. Es war ein kleines Zimmer mit zwei Türen, die an den Stirnseiten des Raumes angebracht waren. Zwischen den Türen stand ein Regal mit Büchern. Links war ein Fenster durch das Licht hereinkam. Ein verzaubertes Fenster. Denn dahinter war eine Wand. Auf der rechten Seite waren nur wieder die vier Bilder der Gründer von Hogwarts zu sehen.

\enquote{Bin gleich wieder da}, sagte Professor Elber und verschwand durch eine der Türen.

Harry sah sich im Raum um und danach auf die Bilder. Er lächelte Salazar zu, doch dieser verzog keine Miene. Auch die anderen Bilder zeigten keine Reaktion.

Professor Elber betrat wieder den Raum (Die Spülung hatte Harry nur unbewusst wahr genommen) und trat seitlich neben Harry. \enquote{Die geben nicht jedem eine Antwort.} Harry sah ihn fragend an. \enquote{Die bewegen sich nicht}, sagte er nach einigen Sekunden.

\enquote{Also Muggelbilder?}

\enquote{Kann man so sagen.}

\enquote{Warum haben Sie eigentlich\abs?} Harry brach ab, da sich sein Professor auf die im Raum stehende Sitzgruppe zu bewegte. Harry folgte ihm und setzte sich auch. \enquote{\aabs in Ihrem Büro und hier Bilder der Gründer?} Als er es aussprach, merkte er, dass er wohl übers Ziel hinausgeschossen war und wurde rot.

\enquote{Wissens-hungrig und schamhaft, keine gute Kombination.} Er sah ihn an und dachte nach. \enquote{Sie bedeuten mir recht viel}, war das Einzige, was er dazu sagte. \enquote{Weswegen wollten Sie mich sprechen?}

\enquote{Weswegen? \gst Mein Tarnumhang. Und über die Familie Peverell.}

\enquote{Haben Ihnen die Hinweise, die ich Ihnen gegeben habe, nicht gereicht?}

Damit hatte Harry nicht gerechnet. Aber am meisten wunderte Harry, dass sein Lehrer scheinbar nicht nur über seinen Tarnumhang Bescheid wusste, sondern auch darüber, wer seine Vorfahren waren. Das war ihm unheimlich.

\enquote{Woher weiß der so viel?}, murmelte er halblaut vor sich hin und wurde direkt danach rot, als er merkte, dass er dies laut ausgesprochen hatte. Jedoch zeigte sein Lehrer keine Reaktion. \enquote{Ehrlich gesagt, hatte ich auf einen Beweis gehofft, dass ich wirklich mit einem der Peverells verwandt bin.}

\enquote{Wollen Sie einen Stammbaum haben?}

\enquote{Wäre hilfreich.}

\enquote{Damit kann ich leider nicht aufwarten. Aber ich kann Ihnen vielleicht etwas anderes geben. Wissen Sie um die besonderen Eigenschaften dieses Umhanges?}

\enquote{Ja, er schützt den Träger vor Entdeckungen.}

\enquote{Das machen andere Umhänge auch.}

\enquote{Aber meiner ist schon sehr alt. Er gehörte meinem Vater. Er muss also älter als ich sein. Ich habe mich schlau gemacht. Andere Tarnumhänge lassen mit der Zeit nach, bekommen Risse und sind nicht so fluchsicher. Bei meinem Umhang habe ich das noch nie festgestellt. \gst Er ist einfach perfekt.}

\enquote{Und das reicht Ihnen nicht?}

\enquote{Ich möchte nur wissen, ob ich von einem der Peverell-Brüder abstamme.}

\enquote{Der Umhang hat aber noch eine andere Eigenschaft.} Er stand auf und holte ein schmales Buch mit nur wenigen Seiten und überreichte es Harry. \enquote{Lesen Sie das. Darin steht genau, was sie tun müssen. Der Umhang hat die Eigenschaft, seinen wahren Herrn zu erkennen. Also die eigene Blutlinie. Wenn Sie von einem der Peverell-Brüder abstammen, dann wird der Umhang das anzeigen. Lesen Sie das Buch.}

Harry schaute auf den Umschlag. Er war rot. Keine Beschriftung.

\enquote{Heute haben wir Samstag. Bringen Sie es am Mittwoch zum Unterricht mit und geben Sie mir es dort. \gst Wollen Sie sonst noch etwas wissen?}

Harry schüttelte den Kopf, bedankte sich und stand auf. Dann verließ er den Raum und stand draußen auf dem Gang.

\enquote{Was suchen Sie denn hier Potter?}, hörte er Snapes Stimme.

\enquote{Ich war gerade bei Professor Elber, Professor.} Snape sah ihn nur an. \enquote{Und Sie?}

\enquote{Wo ich war, geht Sie nichts an, aber ich gehe jetzt in mein Büro. Unterricht vorbereiten.}

\enquote{Nehmen Sie mich bis zum Ausgang des Flügels mit?}, fragte Harry höflich, da er nicht sicher war, ob er den Weg allein finden würde.

\enquote{Wenn Sie nicht trödeln}, sagte Snape und lief los.

Harry war ihm dich auf den Fersen.

\trenn

Als Harry in die Große Halle kam, fragte er sich, warum die Hauselfen noch nicht anfingen, sie weihnachtlich zu schmücken. Immerhin war es bereits Dezember und die Gemeinschaftsräume wurden gerade geschmückt. Nur die Große Halle sah so aus wie immer. Professor Elber hatte noch immer dieses eigenartige Buch in der Hand und saß an seinem Platz in der Großen Halle, um zu Essen. Als er fertig war, blätterte er darin herum, ohne dass irgendjemand sehen konnte, was er da las.

Er verließ die Große Halle und bog ab. Harry wunderte sich noch immer, was Professor Elber wohl vorhatte. \gedanke{Hatte er die Aufgabe die Halle zu schmücken? Warum hatte er noch nicht damit angefangen?} Harry schwirrten die Fragen nur so durch seinen Kopf. Aber keiner wusste, was Elber wohl tun würde, sollte er damit beauftragt worden sein.

Dann hörte er eine merkwürdige Unterhaltung.

\enquote{Haben Sie Zeit, Hagrid?}

\enquote{Ja, Professor.}

\enquote{Gut, haben Sie Lust mit mir auf den Dachboden zu gehen und Weihnachtsschmuck zu holen?}

\enquote{Wird der nicht immer herbeigezaubert?}

\enquote{Ich weiß nicht, welchen Schmuck Sie meinen, aber den, den ich meine, kann man nicht so einfach herbeizaubern. Wenn Sie ihn sehen, dann wissen Sie, was ich meine.}

\enquote{Kann ich euch zwei begleiten?}

\enquote{Aurora? Schön dich zu sehen. Gern doch.}

\enquote{Was heißt hier: \enquote{Schön dich zu sehen?} Wir sehen uns doch fast jeden Tag, du Charmeur.}

\enquote{Und jeden Tag wirst du schöner.}

\enquote{Jetzt hör aber auf, ich werde ja gleich rot.}

\enquote{Ähm, läuft da was zwischen euch?}, fragte Hagrid.

\enquote{Ehrlich? Nein. Nur ein bisschen erotisches anzügliches Geplänkel unter sich gut verstehenden allein lebenden und unverheirateten Kollegen. Es gibt dem Leben etwas mehr Würze.}

\enquote{Ich hätte es nicht besser ausdrücken können, Aurora Schätzchen.}

\enquote{Jetzt hör aber auf, Frederick! Du machst mich ja ganz verlegen. Ich wette, das machst du mit anderen Frauen auch.}

\enquote{Ich würde mich nie an unsere stellvertretende\abs!}

\enquote{Die meinte ich ja auch nicht. Wie wäre es mit Septima?}

\enquote{Nicht im Entferntesten so wie mit dir. Sie ist nicht so locker drauf wie du. Sie kann man nicht so einfach mal in den Arm nehmen und trösten, oder an der Hand haltend um den See laufen, wenn kein Schüler zusieht.}

Harry musste grinsen. Das Bild würde er zu gern mal sehen.

\enquote{Also, gehen wir jetzt auf den Dachboden und schauen uns an, was wir dieses Jahr brauchen, oder nicht?}

\enquote{Gut, Hase. Gehen wir.}

\enquote{Ihr seid nicht zusammen?}, fragte Hagrid ungläubig.

\enquote{Ich glaube, egal was ich sage, falls ich dich küssen würde, glaubt der uns das nie, dass wir nicht zusammen sind. \gst Hagrid, wir verstehen uns sehr gut, aber wir sind kein Paar. Ich bin schon anderweitig\abs gebunden}, fügte er nach einer kleinen Pause hinzu.

Dann ging die Gruppe den Gang entlang und eine große geschwungene Treppe hinauf in das oberste Geschoss.

Am Tag vor dem 24. Dezember saß Harry wieder beim Abendessen, aber noch immer war die Große Halle leer. Aber keinen der Lehrer schien das zu stören. Zurück im Gemeinschaftsraum las Harry noch einmal seine Hausaufgaben durch, um sicherzugehen, dass er auch alles hatte und sie in Ordnung waren. Dann ging er zu Bett.

Mitten in der Nacht wachte Harry auf und machte sich auf den Weg zum Klo. Auf dem Rückweg hörte er komische Geräusche aus dem Gemeinschaftsraum und ging vorsichtig die Treppe herunter. Er sah Professor Elber, wie er eigenartige Zeichen auf eine Steinmauer im Gemeinschaftsraum malte und dann eigenartige Worte sprach. Die Zeichen verschwanden und Professor Elber wirkte zufrieden. Er drehte sich um und verließ den Gemeinschaftsraum. Harry ging vorsichtig zur Wand, sah aber nichts. Nach einigen Minuten, die er die Wand anstarrte, ging er wieder ins Bett. \gedanke{Morgen werde ich Ron und Hermine darüber Bescheid geben.} Harry schlief wieder ein und erwachte am nächsten Morgen.

Er zog sich an und ging in den Gemeinschaftsraum. Ron war auch schon da und beide warteten auf Hermine. Als sie endlich auftauchte, ging es ab zum Frühstück. Auf dem Weg dorthin erzählte er von seinen nächtlichen Beobachtungen, doch auch seine beiden Freunde konnten sich keinen Reim darauf machen. Vor der Großen Halle war bereits eine Ansammlung an Personen aller Häuser. Einige versuchten die Türen der Großen Halle zu öffnen, aber nichts rührte sich. Auch der sonst zuverlässige Alohomora-Zauber funktionierte heute nicht. Professor Dumbledore kam gerade um die Ecke, als einer sagte.

\enquote{Wir kommen nicht in die Große Halle Professor.}

\enquote{Schon mal geklopft?}, fragte Dumbledore.

Er klopfte und aus dem Inneren der Halle kam ein: \enquote{Moment noch, in zwei Minuten könnt ihr alle rein.}  Endlich dann, als die zwei Minuten um waren, öffnete sich die Flügeltür zur Großen Halle und Professor Elber stand am anderen Ende und breitete die Arme aus. \enquote{Fröhliche Weihnachten euch allen}, sagte er.

Die Große Halle sah wieder einmal wunderbar aus. In jeder Ecke stand ein geschmückter Weihnachtsbaum mit Figuren dran, die sich dauernd änderten. In einem Moment waren es Elefanten, die silbern das Licht brachen, im nächsten Moment wurden daraus Kamele und im nächsten Moment, wieder andere Figuren. Über jedem Platz, an dem ein Gedeck stand, war ein kleiner Weihnachtsbaum mit vielen Kerzen, feierlich geschmückt. Schnee rieselte von der Decke und erreichte den Boden. Die Wände waren mit Zweigen bestückt und wunderbar dekoriert. Mistelzweige hingen schwebend im Raum und auch über der großen Flügeltür waren Mistelzweige, die aber noch keiner bemerkte.

\enquote{Bitte setzt euch auf die Plätze, an denen euer Name steht}, sprach Dumbledore, \enquote{und keine Scheu, ihr könnt schon hereinkommen.}

Nach und nach betraten die Schüler den Raum. Die Farbe des Schmucks wandelte sich in rot-goldene Töne, grün-silberne, bronze-blaue, oder gelb-schwarze. Zuerst konnte man kein Muster erkennen, doch eine Vermutung ließ Harry die Schüler zählen und nach Häusern einordnen. Dann kam er dahinter. \gedanke{Klever gemacht}, dachte er sich.

Nachdem alle ihre Plätze eingenommen hatten, schlossen sich die Türen der Großen Halle und Dumbledore schnippte mit den Fingern. Ein Hauself erschien und Dumbledore gab ihm ein Zeichen. Der Elf verschwand sofort wieder und Dumbledore begann mit seiner Rede. \enquote{Es dauert leider noch ein paar Minuten, bis ihr zu frühstücken anfangen könnt, da ich noch eine Überraschung für euch habe.}

Harry schaute nur hin und her und fing die fragenden Blicke seiner Mitschüler auf. Die Große Halle sah leer aus, fand er. Wenige waren über die Weihnachtsferien in Hogwarts geblieben. Nur ungefähr fünfzig Schüler waren in der Großen Halle und jeder saß vor seinem Teller.

Nach einigen Minuten, in denen jeder sich umgesehen hatte, schaute Professor Elber auf seine Uhr und meinte: \enquote{Gleich ist es so weit.} Er zählte herunter. \enquote{Fünf - Vier - Drei - Zwei - Eins.} Er klatschte in die Hände und die Türen der Großen Halle gingen mit lautem Knarren auf.

Alle Blicke wanderten zur Tür, in der viele Personen standen. Freudenschreie fielen in der Halle und reflektierten an den steinernen Wänden. Denn im Türrahmen standen die Eltern, Großeltern oder nahe Verwandte der Schüler. Einige Schüler liefen ihnen entgegen und nahmen sie in ihre Arme. Harry machte nur große Augen.

\enquote{Ich freue mich, dass alles geklappt hat}, sagte Professor Elber wieder. \enquote{Bitte nehmen Sie ihre Plätze ein.} Platzkärtchen erschienen und zeigten den Gästen wo sie zu sitzen hatten. Nachdem alle ihre Plätze gefunden und Platz genommen hatten, klatschte Professor Dumbledore in die Hände und das Frühstück erschien auf den gemütlichen, großen Tischen. Alle begannen zu frühstücken und sich zu unterhalten. Mr. und Mrs. Weasley waren da und auch Hermines Eltern saßen da und frühstückten. Überhaupt waren viele Eltern oder Großeltern da. Nur Harry hatte niemanden, denn sein Onkel und seine Tante würden niemals einen Fuß in Hogwarts setzen. Aber um das zu vergessen, saß Mrs. Weasley neben ihm und lenkte ihn so ab.

Einige waren früh mit Essen fertig und führten ihre Besucher die Große Halle hinaus und durch das Schloss. Harry hielt sich nach seinem Frühstück an Hermine und ihre Eltern, da er sie nur einmal kurz gesehen hatte, als sie in \fab waren. Sie verließen zu viert die Große Halle. Harry hätte es wissen müssen, denn als Erstes ging es Richtung Bibliothek. \enquote{Hermines Lieblingsplatz}, sagte Harry zu ihren Eltern.

Sie gab ihm einen Knuff in die Seite und ihre Eltern lachten. In der Bibliothek angekommen, staunten Hermines Eltern, denn so eine große Bibliothek hatten sie noch nie gesehen. Sie führte sie etwas herum und machte sich dann auf den Weg zum Gemeinschaftsraum.
Als sie die Bibliothek verließen, kam ihnen Professor Elber entgegen und fragte: \enquote{Harry, haben Sie mal kurz Zeit? Dauert nur zwei Minuten.} Er nahm ihn zur Seite und fragte ihn dann: \enquote{Ich habe bereits mit den anderen gesprochen, und mit Ihnen wären sie vollzählig. Ich möchte unseren Gästen ein kleines Quidditch-Match bieten und über die Ferien sind mit Ihnen gerade vierzehn Quidditch-Spieler da. Kommen Sie bitte nach dem Mittagessen zum Quidditch-Feld, sie müssen sich zu zwei Demo-Teams zusammenstellen, damit unsere Gäste Morgen etwas zu sehen bekommen.}

Harry bekam große Augen. \enquote{Sie meinen, die bleiben über Nacht?}

\enquote{Ja}, antwortete Professor Elber.

\enquote{Ja, aber wo sollen die schlafen?}

\enquote{Können Sie sich das nicht denken? Sie haben mich doch heute Nacht beobachtet.}

Harrys Augen weiteten sich und er erinnerte sich wieder daran, wie Professor Elber im Gemeinschaftsraum stand und etwas auf die Wand malte. \enquote{Dann haben Sie Gästequartiere in den Häusern geschaffen?}, fragte Harry.

\enquote{Ja}, antwortete Professor Elber.

Harry grinste und fragte \enquote{Wissen die anderen schon, dass ihre Eltern über Nacht bleiben?}

\enquote{Nein, und das sollen sie bis heute Nacht auch nicht erfahren.}

Später dann, als es Zeit für das Abendessen war, gingen Harry, Hermine und ihre Eltern in die Große Halle. Die Tafel war bereits gedeckt und es gab keine Platzkärtchen mehr. Die großen Tische waren verschwunden und es gab viele kleinere Tische, denn es waren nicht viele Personen während der Ferien da. Nachdem alle gegessen hatten, stand Dumbledore auf und fing wieder an.

\enquote{Ich hatte heute Morgen bereits gesagt, dass es noch eine Überraschung geben wird. Unsere Gäste verlassen uns erst am zweiten Weihnachtsfeiertag nach dem Mittagessen.}

Die Augen der Schüler begannen zu leuchten und einige Münder öffneten sich. Viele schauten ihre Eltern an, die sie angrinsten. Harry bemerkte Professor McGonagall, die Professor Elber an seinem Mantel zog. Dieser setzte sich und unterhielt sich mit Professor McGonagall. Fast schien es ihm, als ob sie verärgert sei. Aber nach ein paar Worten besserte sich ihre Miene und Harry hatte den Eindruck, dass Professor McGonagall zu schmunzeln begann. Nach dem Essen begaben sich einige nach draußen, andere hingegen gingen direkt mit ihren Eltern, oder nahen Verwandten, in ihre Gemeinschaftsräume. Nach einigen geselligen Unterhaltungen der Schüler mit den Gästen, gingen viele von ihnen ins Bett. Harry unterhielt sich noch mit Hermine; besser gesagt Hermine erzählte die ganze Zeit und Harry hörte ihr zu. Nach einiger Zeit wurde Harry Müde und er ging ins Bett. Hermine stand ebenfalls auf und gab ihm einen Gute-Nacht-Kuss. Harry rieb sich erstaunt seine Backe und ging die Treppen hoch ins Bett.

Am Tag darauf während des Mittagessens nahm Dumbledore einen Löffel in die Hand und klopfte gegen seinen Trinkkelch. \enquote{Ich möchte alle Schüler bitten, nach dem Mittagessen mit ihren Gästen auf das Quidditch-Feld zu gehen, da dort eine kleine Demonstration stattfinden wird.}

Harry grinste, als Hermine ihn fragend ansah. \enquote{Du wusstest das?}, fragte sie ihn.

Harry grinste. \enquote{Nur seit gestern.}

Er stand auf und ging Richtung Quidditch-Feld, da er sich noch umzuziehen hatte. Es dauerte eine Weile bis alle eingetroffen waren, also ging das Team um Harry noch einmal ihren Plan durch. Als ein Pfiff ertönte, gingen sie an den Rand des Feldes, bestiegen ihre Besen und stiegen hoch in die Lüfte. Es begann leicht zu schneien, als das erste Tor geschossen wurde, weil Harry den Ball durch die Ringe ließ. Heute hatte er die Position als Hüter. Ihm kam dies gelegen, da er bei Arabella schon einmal die Position des Hüters innehatte und daher auf etwas Erfahrung zurückgreifen konnte. Harry erspähte während des Spieles ein paar Mal den Schnatz, aber da er dieses mal nicht der Sucher war, durfte er ihn nicht fangen. Er versuchte verzweifelt, seinen Team-Sucher darauf aufmerksam zu machen. Aber es war unmöglich, da beide Sucher in seiner Nähe waren. Derart abgelenkt, ließ er einen weiteren Ball durch einen der Ringe. Beide Sucher drehten sich nun um und entdeckten den Schnatz. Jetzt entbrannte ein Kampf um den kleinen goldenen Ball. \gedanke{Ich hätte ihn schon längst gehabt}, dachte Harry. \gedanke{Aber Cho ist auch nicht schlecht.} Kurz darauf hatte sie ihn auch schon und das Spiel war beendet. Am Ende gewann Harrys Team mit 160 zu 20 und das Spiel war zu Ende.

Auf dem Weg zurück zum Schloss meinte Hermines Vater: \enquote{Das war ein tolles Spiel. Und als Torwart\abs}

\enquote{Hüter}, unterbrach ihn Hermine.

\enquote{Ja, als Hüter warst du sehr gut. Soweit ich das beurteilen kann.}

\enquote{Danke Mister Granger. Aber normalerweise bin ich der Sucher in unserem Team und fange den goldenen Schnatz. Aber wir sind über die Ferien so wenig, sodass ich heute eine andere Position eingenommen hatte.}

\enquote{Ah ja}, meinte Hermines Vater.




\begin{kommentar}
Harry will etwas über seine Vergangenheit und seine Vorfahren wissen und befragt deshalb Elber. Dort entdeckt er wieder die vier Gründer in Bildern an der Wand. Der andere Ort ist Elbers Büro. Ein weiterer schöner Hinweis darauf, dass diese vier seine Kinder sind.
\end{kommentar}

\begin{kommentar}
Als Harry Elber dann wieder verlässt, trifft er auf Snape. Dieser nimmt ihn als kleine Geste des guten Willens mit. Das Verhältnis der beiden scheint so langsam besser zu werden.
\end{kommentar}

\begin{kommentar}
Kurz darauf unterhält sich Elber mit Hagrid und Auror. Dort sagt Elber, dass er anderweitig vergeben ist. Erst im nächsten Teil kommt heraus, dass er mit der Zwillingspsyche von Bellatrix Lestrange verbandelt ist. Aber hier sind bereit die ersten Andeutungen zu erkennen.
\end{kommentar}

\chapter{Die Weihnachtsüberraschung}


Auf halbem Wege zum Schloss kam ihnen Hagrid entgegen.

\enquote{Oh, hallo Harry, hab dir noch was zu Weihnachten schenken wollen, konnte es aber nicht vorher bekommen}, sagte er, als er vor ihnen stand und Harry sein Weihnachtsgeschenk überreichte.

\enquote{Sie sind aber groß}, meinte Hermines Mutter.

\enquote{Er ist halb Riese und halb Mensch, Mutter}, sagte Hermine.

Die Augen ihrer Mutter weiteten sich und sie fragte: \enquote{Und wie groß ist so ein Riese normalerweise?}

\enquote{Och, nur so acht bis neun Meter groß, aber es gibt noch größere}, meinte Hagrid, als sei es das normalste auf der Welt, so groß zu sein.

Hermines Eltern waren sprachlos. Nicht nur, dass sie soeben einen Halbriesen gesehen hatte, der sich gut mit einem Schulkameraden ihrer Tochter verstand, nein, sie hörten auch noch von Riesen, die so hoch wie ein Haus werden konnten. Harry lachte und griff Hermines Vater unter seinem Arm und nahm ihn mit Richtung Schloss. Hermine nahm ihre Mutter und ging Harry hinterher.

Wieder zurück im Gemeinschaftsraum fingen sich beide so langsam wieder und kehrten zurück zur Normalität. Plötzlich poppte Dobby herbei und wollte aufräumen, da er nicht erwartet hatte, dass sich um diese Zeit jemand im Gemeinschaftsraum aufhalten würde. Die Feiertage bedeutete zusätzliche Arbeit für die Elfen, obwohl Professor Dumbledore Hermine gesagt hatte, dass er für die Aufgaben nur bezahlte Elfen herangezogen hatte und ihnen eine Sonderzulage gewährte.

\enquote{Oh, Verzeihung Harry Potter. Dobby wollte sie nicht stören.} Dann sah er Hermines Eltern an.

\enquote{Du störst uns nicht, Dobby}, meinte Harry. \enquote{Nicht im Geringsten.}

Hermines Eltern standen ihre Gesichter still.

\enquote{Das}, so sagte Hermine, \enquote{ist Dobby. Er ist ein Elf. Er kümmert sich hier um viele Sachen. Er räumt unter anderem auf und kocht.}

\enquote{Freut mich Sie kennenzulernen, Mister Dobby}, sagte Hermines Mutter ganz erstaunt.

\enquote{Das ist meine Mutter}, fügte Hermine hinzu, als sie Dobby fragend anschaute.

\enquote{Ah, Miss Hermines Mutter. Ich bin Dobby, der Hauself.} Er drehte sich wieder um und kümmerte sich um den Gemeinschaftsraum.

Einige Minuten saßen alle vier still im Raum und ließen ihre Gedanken schweifen.

\enquote{Das war wirklich eine tolle Idee von eurem Lehrer. Ich hätte nie gedacht, dass ich hier mal sein werde, geschweige denn Weihnachten hier feiern würde}, sagte Hermines Mutter.

\enquote{Wie kam das eigentlich, dass sie hierherkamen?}, fragte Harry.

\enquote{Na ja}, antwortete Mister Granger. \enquote{Das war so: Wir waren gerade zu Hause und aßen zu Abend, als eine Eule mit einem Brief im Schnabel an das Fenster klopfte. Wir dachten schon, es sei wieder Post von unserer Tochter. Sie schreibt uns immer mal wieder. Aber als wir dann den Namen Professor Elber \gst Lehrer im Fach Verteidigung gegen die dunklen Künste \gst auf der Rückseite lasen, wurde uns etwas mulmig. Wir hatten schon Angst um unsere kleine Miene.}

Harrys Augen wurden größer und er starrte Hermine an.

\enquote{Dad}, sagte Hermine.

\enquote{Miene?}, sagte Harry mit einem fragenden Ausdruck in seinem Gesicht, als er Hermine ansah.

Hermine errötete leicht, fasste sich aber dann recht schnell. \enquote{Wehe du sagst irgendjemanden etwas davon. Ich schwöre dir, ich verhexe dich, dass dich Madame Pomfrey die nächsten acht Wochen nicht mehr aus der Krankenstation entlässt.}

Harry grinst nur und meinte frech. \enquote{Mach’s ruhig. Du hältst es ja ohne mich eh nicht aus und besuchst mich jeden Tag.}

Jetzt lachten Hermines Eltern und auch Hermine konnte sich nicht mehr zurückhalten.

\enquote{Zurück zum Thema}, schlug Hermines Vater vor. \enquote{Wir öffneten also etwas unsicher den Brief und lasen.}

\begin{brief}
Sehr geehrte Familie Granger,

Ihre Tochter hatte sich in Hogwarts eingeschrieben, um die Weihnachtsferien hier zu verbringen. Ich nehme mir vor, dieses Fest für alle dagebliebenen unvergesslich zu gestalten. Zu diesem Zweck möchte ich Sie beide gerne zu einem kurzen Aufenthalt in Hogwarts einladen. Sollten Sie zustimmen, trennen Sie bitte den unteren Teil des Pergaments ab und geben ihn der Eule, die diesen Brief gebracht hat mit. Zu gegebener Zeit erhalten sie einen weiteren Brief mit exakten Informationen, wann und wo sie abgeholt werden.

Reden Sie mit niemandem darüber. Nicht einmal mit Ihrer Tochter.
\signumspace
Mit vielen Grüßen

Professor Frederick Elber

Verteidigung gegen die dunklen Künste
\end{brief}

\enquote{Damit waren unsere Sorgen vergessen und wir freuten uns natürlich, einmal Hogwarts sehen zu können. Wir trennten den unteren Teil des Pergaments ab und antworteten, dass wir kommen werden. Wir gaben den Brief der Eule, die daraufhin verschwand.}

\enquote{Und wie sind Sie schließlich angekommen?}, fragte Harry.

\enquote{Als es Zeit wurde, kam ein weiterer Brief. Er enthielt ein Schreiben mit Instruktionen und eine große ausfaltbare Scheibe. Auf die mussten wir uns zu gegebener Zeit einfach darauf stellen und kurz darauf waren wir in einem Dorf am Fuße Hogwarts. Wir sahen Professor Elber, der uns kurz erklärte, dass wir alle noch etwas Zeit hatten und uns ein wenig umsehen konnten. Glücklicherweise hatten wir noch ein bisschen Zauberergeld und so konnten wir uns ein paar Sachen kaufen. Später dann kamen diese merkwürdigen Kutschen an, wir stiegen ein und wurden hochgefahren, wo wir wie beschrieben vor den verschlossenen Türen der Großen Halle warteten.}

Harry musste nur grinsen und auch Hermine fühlte sich glücklich.

Plötzlich fragte ihre Mutter: \enquote{Und wie lange bist du schon mit deinem Freund zusammen?}, und schaute danach zu Harry.

Hermine errötete wieder. \enquote{Das ist nicht mein Freund.}

\enquote{Aber du hast uns doch erzählt, dass du mit jemandem zusammen bist.} Hermine druckste herum. Sie wollte Harry wohl nicht kränken. \enquote{Ist es Ron?}, fragte Harry.

Hermine schaute ihn nur an und nickte leicht.

\enquote{Ist schon in Ordnung, Hermine. Ich habe auch jemanden.}

Das hätte sie am allerwenigsten von Harry erwartet. \enquote{Wer?}, entfuhr es ihr ganz aufgeregt.

\enquote{Kannst du dir das nicht denken? Wir sind zusammen unter dem Baum im Hof gesessen. Dumbledore war auch dabei.}

Hermines Gesicht blieb buchstäblich stehen. Harry hätte, wer weiß was darum gegeben, es jetzt Fotografieren zu können.

\enquote{Du meinst doch nicht etwa Luna?}, fragte sie ganz erstaunt. Harry grinste nur und schaute sie an. \enquote{Ihr beide? Du und sie? \gst}

Harry grinste weiter. Plötzlich viel ihm der Gemeinschaftsraum ein. Sobald alle Eltern und Verwandten weg sein würden, würde es auch wohl für Ron und Hermine Zeit, in alles eingeweiht zu werden, Platz genug wäre ja noch. Nachdem Hermine wieder ihre Fassung gewonnen hatten, redeten die vier noch eine Weile über Gott und die Welt, bevor es an der Zeit war, zu Bett zu gehen.

Unruhige Träume durchzogen seine Nacht. Er fand sich wieder im Hause der Malfoys. Er wusste noch von Salazar, dass er sich im näheren Umkreis \gst wenige hundert Meter \gst um Voldemort herum frei bewegen konnte. Also begann er sich umzusehen. Er sah wieder den Raum mit dem Kamin und dem Feuer darin. Das große Familiengemälde mit den vier Malfoys, das über dem Kamin an einer sonst kahlen Steinwand hing. Drei davon sahen eigenartig verzerrt aus. Davor der Tisch, an dem die Todesser beratschlagten. Er drehte sich um und sah dieselbe Fensterreihe, mit den Bleiglasfenstern und den Figuren, die sich darin bewegten. Es waren Schlangen, die von einem Fenster in das andere schlängelten. Er sah das noch immer zerstörte Beistelltischchen und, als er sich nach links drehte, eine große, schwere Eichentüre. Er wollte sie öffnen, als er mit seiner Hand durch glitt. Also trat er nach kurzem Zögern einfach hindurch. Er ging eine große geschwungenen Treppe hinunter und sah eine offene Tür, aus der er Stimmen vernahm. Sie schienen sich nicht richtig zu unterhalten, sondern nur Buchstaben und Zahlen zu nennen. Und dann immer wieder: \enquote{Du bist dran!}, gefolgt von einem mechanischen Klacken.

Er wollte gerade durch den Türspalt schauen, als ihm wieder einfiel, er brauche ja nur durch die Tür zu gehen. Er sah als Erstes eine Menge Bücher in Holzregalen stehen. Die Zwischenwände waren reich verziert mit Holzschnitzereien und Intarsien aus Gold, Alabaster und Adamant. Von der Decke herunter hingen handgetriebene Lüster aus Gold und erhellten mit Kerzen den Raum. Er ging durch einen der Gänge auf die Stimmen zu. Vorsichtig blickte er um die Ecke. Er sah Lucius Malfoy vor einem Schachbrett sitzen. Die Person ihm gegenüber konnte er nicht erkennen. Neben ihm saß Draco und schaute den beiden zu. Mister Malfoy hatte die Ellenbogen auf dem wertvollen Tisch, die Hände ineinander verschränkt und legte seinen Kopf darauf ab. Nachdenklich betrachtete er das Brett. Dann nahm er eine Figur und bewegte sie. \enquote{Schach}, sagte er. Dann drückte er die Spieluhr, damit die Zeit für seinen Spielpartner ablief.

Dieser hatte, so sah Harry, nachdem er einige Schritte auf die beiden zugemacht hatte, ebenfalls seine Ellenbogen auf dem Tisch, seine Finger aber auf dem Tisch liegend verschränkt. Irgendwie kamen die Finger Harry bekannt vor. Er wusste nicht, wo er sie hintun sollte. Er war sich unsicher, ob er sich noch weiter bewegen sollte. Bisher war er auf niemanden getroffen und konnte durch Türen gehen. Aber was war, wenn er sich das nur eingebildet hatte und tatsächlich hier war. Im Hause der Malfoys. Ohne Zauberstab. Hier würde er niemals mehr lebend herauskommen.

Draco Malfoy schaute nun in seine Richtung. \enquote{Mum?}, fragte er.

Harry drehte sich erschrocken um. Wenige Millimeter hinter ihm stand Dracos Mutter.

\enquote{Kommst du bitte Draco? Deine Schwester ist im Wohnzimmer und möchte nicht alleine sein, wenn Tante Bellatrix kommt.}

Draco nickte und stand auf. Er ging zu seiner Mutter und verließ den Raum. Die Tür ließ er offen.

\enquote{Schach matt}, hörte Harry jetzt und das abermalige Klicken der Uhr. Diese Stimme kannte er genau. Er drehte sich herum und lief auf ihn zu. Er stellte sich an das Spielbrett und blickte seinem Lehrer in die Augen. Dieser sah Lucius an und meinte: \enquote{Du wirst besser, Lucius. Aber noch hast du mich nicht \gst noch eine Runde, oder lassen wir es für heute gut sein?}

Er sah seinem Lehrer für \VgddK in die Augen. Er konnte es nicht fassen, dass er hier in aller Ruhe saß und mit dem Vater seines schlimmsten Feindes Schach spielte.

\enquote{Lass es gut sein, Frederick. Wir sehen uns nächsten Monat.}

Elber nickte und meinte dann: \enquote{Du kannst dir das Spiel wie immer nochmals anschauen. Letztes Mal war eine Ausnahme, weil ich was versuchen wollte, das du noch nicht mitbekommen solltest. Ich werde es dir nächstes Mal zeigen.} Er wollte gerade aufstehen, als er einen fürchterlichen Schrei hörte.

\enquote{Tamara.} Sofort rannte Professor Elber aus der Bibliothek. Harry ihm hinterher. Als Professor Elber abrupt stehen blieb, rannte Harry förmlich durch ihn durch.

Er zog seinen Zauberstab und entwaffnete Bellatrix, die ihren Zauberstab auf ihre Nichte gerichtet hatte und diese gerade mit dem Cruciatus-Fluch belegte. Nachdem sie ihren Zauberstab verloren hatte, verstummten auch die Schreie der kleinen Tamara.

\enquote{Bist du verrückt geworden, Bellatrix? Deine eigene Nichte mit dem Cruciatus-Fluch zu belegen? Sie gehört zu deiner Familie. So etwas hätte ich dir nicht zugetraut. Zwar vieles, aber nicht das.}

\enquote{Sie ist eine Gryffindor und außerdem mit den Weasleys und diesem Potter befreundet. Sie hat es verdient}, spie sie heraus. Tamaras Mutter und Draco standen ganz verstört neben ihr. Dann ging Draco zu seiner kleinen Schwester hin und tröstete sie. Harry wusste, dass Draco nichts gegen seine Tante ausrichten konnte. Und ihre Mutter hatte wohl zu viel Angst, als dass sie ihrer Schwester Einhalt gebieten könnte.

Bellatrix wollte schon ihren Zauberstab holen, als sie Professor Elber mit einer gefluchten Ohrfeige davon abhielt. \enquote{Draco, nimm deine kleine Schwester und geh mit in ihr Zimmer. Wartet dort auf mich, bis ich komme.}

Draco nickte und trug seine noch immer zitternde und weinende Schwester hinaus. \enquote{Warte draußen, Narcissa. Das jetzt willst du nicht sehen.}

Er blickte ihr kurz aber intensiv in die Augen. Man konnte deutlich Angst in ihrem Blick lesen. Aber nicht die Angst vor ihrer Schwester, sondern Angst vor dem, was mit ihr gleich passieren würde. \enquote{Und}, fügte Elber hinzu, \enquote{leg' einen Schallschutzzauber über diesen Raum.}

Nachdem die Tür geschlossen wurde, steckte Elber seinen Zauberstab wieder ein und ging auf Bellatrix zu. \enquote{Wie kommst du dazu}, fragte er in einem normalen Tonfall, \enquote{deine Nichte zu foltern. Und erzähl mir nicht das Märchen von wegen falscher Umgang, falsches Haus. Warum hast du es wirklich gemacht? Wollte sie nicht dem Dunklen Lord dienen?}

Bei diesen Worten legte sich ein Schauer über Harrys Rücken. Er hatte seinen Lehrer Voldemort noch nie \accentuate{Den Dunklen Lord} nennen hören. \gedanke{War er ein Todesser?}, durchdrang ihn eine Überlegung.

Bellatrix schnappte sich ihren Zauberstab, als Elber gerade durchs Fenster sah und belegte ihn mit einem Fluch. Das hieß, sie versuchte es. Er schrie kurz vor Schmerzen auf, warf ihn aber zurück. So etwas hatte Harry noch nie gesehen. Der Zauberstab flog Bellatrix abermals aus der Hand.

Nun wurde Harry zunehmend unwohl. Denn was er jetzt sah, verschlug ihm die Sprache. Ihm wurde heiß und gleichzeitig kalt. Er wusste nicht mehr, ob er schwitzte, oder fror.

Die Haare seines Lehrers wurden weiß und er konnte in seinen Augen ein kurzes Funkeln erkennen. Seine Iris wurde rot und die Fingernägel wuchsen um mindestens drei Millimeter. Verunsichert ging Bellatrix einen Schritt zurück. Doch das half ihr nicht die Schmerzen, die gleich kommen würden, zu mildern. Elber streckte seine Hände aus und mit leicht offenem Mund und einem genießenden Blick in den Augen zogen blau-violette Blitze aus seinen Fingern heraus und direkt in Bellatrix' Körper. Diese schrie vor Schmerzen, fiel zu Boden und krümmte sich in eine Fötus-Haltung ein. Ihre Augen waren weit aufgerissen und Harry hatte den Eindruck, dass dies weitaus schlimmer als der Cruciatus-Fluch sein musste. Sie wechselte in eine Haltung mit gekrümmten Rücken und wieder zurück. Er hatte zwar keinen Geruchssinn, aber an den aufsteigenden Dämpfen und der Verfärbung ihrer Hände und ihres Gesichtes musste wohl ihr Fleisch an manchen Stellen sehr warm werden und teilweise sogar verbrennen.

Nach etwa zehn Sekunden beendete Elber seinen \gst was auch immer er da gerade getan hatte \gst und ging auf Bellatrix zu. \enquote{Jeder schmerzhafte Fluch, den dir der Dunkle Lord nun schenkt, wird wie ein Kitzeln für dich sein. Das war deine Strafe dafür, dass du deine Nichte gefoltert hast, Bellatrix. Ich sagte dir schon einmal, reize mich nicht.} Er zog wieder seinen Zauberstab und ließ sie wie eine leblose Puppe schweben, dann sich in der Luft drehen und setzte sie anschließend in einem Sessel ab. Dann sprach er mehrere Zauber, die Harry weder kannte, noch verstand. Die Hautrötungen verschwanden und auch der Dampf, welcher immer noch vereinzelt durch ihre Kleidung drang, verschwand. \enquote{Was bleibt, ist die Erinnerung, Bellatrix. Sonst nichts.}

Dann steckte er wieder seinen Zauberstab ein und legte eine Hand auf ihre Stirn. Sie erwachte, hatte aber keinen irren Blick mehr an sich. Sie wirkte irgendwie normal. Er gab ihr einen kurzen Kuss auf die Stirn und meinte: \enquote{Du hast jetzt eine halbe Stunde für dich, vorher wacht Sie nicht auf.} Dann drehte er sich um und verließ den Raum.

Harry sah Bellatrix nur an. Sie saß da, nahm sich ein Buch, das auf dem Tisch neben ihr lag und schlug es auf. Verträumt sah sie nun kurz in Harrys Richtung. Es schien, dass sie ihn betrachtete, aber ihr Blick war scheinbar abwesend. Dann widmete sie sich wieder ihrem Buch.

Ihr Verhalten war so merkwürdig, dass Harry sich keinen Reim darauf machen konnte. Er drehte sich schnell um und verließ mit langen Schritten den Raum, um nach seinem Lehrer Ausschau zu halten. Dieser unterhielt sich im nach oben gehen mit Dracos Mutter. \enquote{Sie braucht jetzt noch Ruhe, gib ihr eine halbe Stunde. Ich habe dafür gesorgt, dass keiner zu ihr kann.}

Dann waren die drei oben angekommen. Elber betrat Tamaras Zimmer und meinte: \enquote{Wir packen jetzt deine und Dracos Sachen, ihr kommt mit zu mir.}

Dracos und Tamaras Mutter lief eine einzelne Träne über ihr Gesicht.

\enquote{Mama!}, sagte Tamara. \enquote{Wir müssen gehen?}

\enquote{Es ist besser für euch. Bei Frederick seid ihr besser aufgehoben als hier. Ich kann euch nicht beschützen. Seht mich an, ich habe es nicht einmal geschafft, mich vor euch zu stellen, als meine Schwester dich\abs} doch Narcissa stockte. Sie konnte es nicht aussprechen.

\enquote{\aabs gefoltert hatte}, beendete ihre Tochter den Satz. Nun liefen noch mehr Tränen an Narcissa herunter.

Elber drehte sich zu Narcissa um, nahm sie kurz in den Arm und sagte dann zu ihr: \enquote{Du kannst jederzeit zu ihnen kommen. Du weißt, wo ich bin. Und du kannst auch ohne Zauberstab apparieren. Und wenn du es nicht mehr bewältigst, dann kommst du ganz. Du weißt, dass sie dich deswegen nicht foltern können?}

Sie nickte stumm und verließ den Raum. Es war alles zu viel für sie.

Harry musste schlucken. Jetzt verstand er Tamara, als sie sagte: \enquote{Draco hat immer weniger Freunde, beziehungsweise keine Freunde, und auch seine Familie macht ihm zu schaffen.}

\enquote{So}, sagte Harrys Lehrer und schlug vergnügt die Hände zusammen. \enquote{Wir packen! Das heißt, ich packe und ihr schaut zu.}

Abgelenkt und dadurch freudig, schaute Tamara ihn an und sprang vom Bett auf ihn zu. \enquote{Danke Frederick.}

\enquote{Aber für meine Patentochter mache ich das doch gerne.} Nachdem sie sich gedrückt hatten, hob Elber die Hände, schlug sie über dem Kopf zusammen, zog seinen Zauberstab und schwang ihn kunstvoll durch die Luft. In der Mitte des Raumes bildete sich eine kleine Kugel und sog das ganze Inventar samt Spielsachen und alles was sich im Raum befand ein. Er nahm die Kugel in die Hand und gab sie Tamara. \enquote{Jetzt machen wir das in Dracos Zimmer auch noch und dann apparieren wir zu mir.}

Vor Harry begann sich die Szene langsam aufzulösen. Er hatte den Eindruck, als würde er einschlafen.

Um drei Uhr morgens wachte er auf. Er erinnerte sich vage an seinen Traum. Er richtete sich im Bett auf, zog die Beine an und schlug seine Arme darum. Dann legte er seinen Kopf auf seine Knie und dachte angespannt nach. Er begann sich nur langsam zu erinnern. Er sah eine Person mit Lucius Malfoy in dessen Haus Schach spielen, sah, wie Draco die beiden beobachtete und hörte einen Schrei. Kurz darauf hatte er auch schon Tamara gesehen, zusammen gekrümmt am Boden. Draco nahm seine Schwester und verschwand. Dann lag Bellatrix, Tamaras Tante, am Boden und hatte Verbrennungen. Dann stand der Mann, der mit Lucius Schach gespielt hatte in Tamaras Zimmer, verließ es mit Draco und ihr und einer kleinen Kugel, die Tamara hielt und ging in Dracos Zimmer. Kurz darauf kamen die drei mit einer weiteren Kugel auf den Gang. Draco und Tamara hielten sich fest und die drei disapparierten.

\gedanke{Wenn ich doch nur das Gesicht wieder sehen könnte}, dachte sich Harry. \gedanke{Irgendwoher kenne ich die Person. Ich weiß nur nicht woher.}

Seine Hoden schmerzten, da sich seine Prellung bemerkbar machte. Er legte sich wieder zurück in sein Bett, zog die Bettdecke zu und wollte noch etwas darüber nachdenken. Doch schon schlummerte er wieder ein und vergaß weitere Details seines Traumes \gst oder war es eine Vision?

Am nächsten Morgen wurde es Zeit die Geschenke auszupacken. In der Mitte des Gemeinschaftsraumes stand ein raumhoher aber schmaler festlich geschmückter Weihnachtsbaum, der wie seine großen Brüder in der Großen Halle geschmückt war. Die restlichen verbliebenen Gryffindors packten ihre Geschenke aus und auch Hermines Eltern hatten für Harry eine Kleinigkeit eingepackt, die sie unten in Hogsmeade erstanden hatten. Es war ein \accentuate{sich Verändernder}, ein kleiner Ring, der sich der Größe seines Benutzers anpasste. Harry steckte ihn an seinen Finger und er verschmolz fast mit ihm. Harry konnte ihn kaum noch sehen und versuchte ihn wieder herunter zu bekommen. Dies gelang ihm auch gleich, worauf hin sich der Ring wieder in seine Ausgangsform zurückverwandelte.

\enquote{Was ist das?}, fragte er.

\enquote{Das}, antwortete Hermines Vater, \enquote{ist ein sogenannter \enquote{sich Verändernder}. Der Ring passt sich seinem Träger an und hat bei jedem eine andere Form und Gestalt. Er warnt einen vor Feinden und anderen komischen Gestalten.}

\enquote{Wir haben für uns auch gleich welche mitgenommen}, fügte Hermines Mutter hinzu. \enquote{Man weiß nie, wann man so etwas gebrauchen kann}, und lachte dabei.

Aus Harry kam nur ein knappes \enquote{Danke} heraus. Er hatte nicht erwartet, von Hermines Eltern etwas zu bekommen. Umso größer war die Überraschung.

Hermine packte noch ein weiteres Buch aus. Harry staunte nicht schlecht, da es über dunkle Künste handelte.

\enquote{Woher habt ihr denn das bekommen?}, fragte Hermine nach. \enquote{Ich glaube nicht, dass es das in Hogsmeade gab, oder bei \fab.}

\enquote{Wir haben es aus der Nokturngasse. Wir wussten nicht genau, was wir dir dieses Jahr schenken sollten, also haben wir deinen Lehrer gefragt. Diesen Professor Elber, als er uns unten in Hogsmeade empfing.}

\enquote{Wie seit ihr dahin\abs ?}, fragte Hermine weiter.

\enquote{Mit Flohpulver in den tropfenden Kessel und zurück.}

Hermines Augen weiteten sich und sie fiel ihren Eltern um den Hals. Auch die anderen hatten in der Zwischenzeit ihre Geschenke ausgepackt und unterhielten sich mit ihren Verwandten.

Plötzlich viel Harry noch ein weiteres Geschenk auf, welches in der Ecke lag. \enquote{Fehlt noch jemand ein Geschenk?}, rief Harry in die Runde.

Doch jeder verneinte und so stand Harry auf, um sich das Geschenk näher anzusehen. Er nahm es von der Wand an der lehnte und drehte es um. Es sah aus, wie ein Tisch ohne Beine. Jetzt entdeckte er ein Namensschild mit seinem Namen. Erstaunt öffnete er sein letztes Geschenk und zum Vorschein kam nur eine hölzerne Tischplatte und ein kleiner Brief. Die umstehenden und herum sitzenden Gryffindors machten nur komische Gesichter und fragten sich, wer Harry wohl so etwas schenken sollte. Harry konnte sich auch keinen Reim darauf machen und so öffnete er den beigelegten Brief.

\begin{brief}
Lieber Harry,

Ich wünsche dir frohe Weihnachten und alles Gute im neuen Jahr. Als kleines, oder besser gesagt großes, Weihnachtsgeschenk, habe ich dir das Mini-Quidditch-Spiel eingepackt. Viel Spaß beim Trainieren und pass' mir gut auf das Spiel auf, es ist wie du weißt sehr alt. Damit gehört es jetzt offiziell dir.
\signumspace
Grüße Arabella.
\end{brief}


\enquote{Es ist von Arabella}, sagte Harry. \enquote{Es ist das Mini-Quidditch-Spiel, welches wir bei meinem Geburtstag gespielt haben.}

Jetzt wurde die Aufmerksamkeit der anderen geweckt. Harry nahm das Spiel aus der aufgerissenen Verpackung heraus und legte es auf den Boden. Er freute sich, denn nun hatte er etwas für seine Mannschaft mit dem sie trainieren und Spielzüge planen konnten.

\enquote{Wie spielt man das?}, fragte Dean, der sich jetzt näherte.

Harry drehte seinen Kopf zu ihm und meinte nur: \enquote{Komm her und setz dich, dann erkläre ich es dir.}

Schnell waren vierzehn Leute für ein Spiel zusammen und Harry begann das Spiel.

Er zog seinen Zauberstab aus seinem Umhang und berührte mit der Spitze seines Zauberstabes das Spielbrett. \enquote{Spiel beginnen}, sagte er laut und deutlich.

Wie schon beim letzten Mal begannen sich zuerst die Stangen und danach die Ringe sowie die Tribünen aus dem Spielbrett herauszubilden.

\enquote{Wahnsinn}, hallte es durch den Gemeinschaftsraum.

Harry fing jetzt an zu erklären. \enquote{Jeder Spieler sucht sich eine Mannschaft und eine Position aus und sagt diese laut und deutlich. Ich führe es euch mal vor.} Er sagte: \enquote{Treiber, Panther von Loch Lumen.} Kurz darauf erschien auf dem Spielbrett eine kleine Figur, welche Harry sehr ähnlich sah. Sie trug die Uniform der Panther von Loch Lumen. \enquote{Ihr steuert eure Figuren mit Hilfe eurer Gedanken. Am Anfang braucht es ein paar Minuten, bis eure Figur auf euch reagiert, aber dann geht es mühelos.}

Alle Spieler sprachen nun durcheinander ihre Position und die Mannschaft. Harry verstand kein Wort, aber das Spielbrett hatte offenbar alles genau verstanden. Harry ließ jetzt seine Figur etwas steigen, um den anderen auf dem Spielbrett etwas mehr Platz zu geben und gleichzeitig zu zeigen, dass die Figur ohne Worte zu steuern war. Nach einigen Minuten bewegten sich alle Figuren in der Luft über dem Spielbrett. Hermine machte den Schiedsrichter und blies in die Pfeife, welche ebenfalls auf dem Spielbrett erschienen war. Jetzt gingen wieder die Klatscher in die Höhe und Harry gab seiner Figur zu verstehen, den Klatscher auf eine Figur der gegnerischen Mannschaft zu treiben. Einige Figuren fielen immer wieder von ihren Besen, oder konnten sich gerade noch so festhalten. Harry fand sich in der Rolle des Treibers gar nicht so schlecht. So konnte er alle Positionen einmal spielen und im Falle eines Falles jemanden ersetzen, bzw. vertreten.

Später, als es wieder Zeit für das Abendessen war, flanierte Harry mit Hermines Vater durch das Schloss. Unterwegs begegnete er dem kleinen Hufflepuff mit den Narben im Gesicht und dem einzelnen Auge. Mittlerweile hatten sich fast alle an diesen Anblick gewöhnt. Harry lief also mit  Mister Granger über das Gelände. Er wollte sich einmal mit ihm alleine Unterhalten. Harry sah immer wieder, wie sich einige Eltern mit Professor Dumbledore oder einem der anderen Lehrer unterhielten. Er musste grinsen und dachte an die Elternsprechtage in seiner Schule, als er noch mit Onkel Vernon und Tante Petunia zum Vorsprechen musste. Es war wohl für viele Eltern normal, die zum ersten Male ein magisch begabtes Kind zur Schule schickten, dass sie sich direkt mit einem Lehrer unterhielten. Im Gegensatz zu den sonst wohl üblichen Unterredungen an einer normalen Schule.

Als am nächsten Tag der Zeitpunkt der Abreise immer näher kam, begleiteten alle Schüler ihre Besucher nach Hogsmeade, um ihnen eine gute Reise zu wünschen. Die Plattformen erschienen wieder und die Besucher stiegen darauf, nur um kurz danach mit einem leisen \geraeusch{Fzzz} zu verschwinden.

Tags darauf war wieder einmal Samstag und Harry lotste Ron und Hermine nach dem Frühstück in den dritten Stock im Westflügel. Dort wartete bereits Luna, die noch an ihrem Nachtisch des Vortages kaute, welchen sie sich mitgenommen haben musste.

Ron bekam große Augen, als er Luna sah, und noch größere als Harry sie zur Begrüßung küsste. \enquote{Was? Ihr beide seid zusammen?}, fragte Ron.

\enquote{Ja}, antworteten beide unisono.

\enquote{Und nachher wird es wohl auch für den Rest der Schule offiziell gemacht werden. Bisher sind das ja nur Vermutungen und Spekulationen}, sagte Harry.

\enquote{Und um uns das zu sagen, mussten wir hierherkommen?}, fragte Ron.

\enquote{Nein}, antwortete Harry. \enquote{Ich weiß seit vorgestern Abend, dass du und Hermine ein Paar seid und \gst}

Doch Ron unterbrach ihn. \enquote{Du hast es ihm erzählt?}

\enquote{Ich konnte nicht anders. Ich war in einer misslichen Situation und \gst}, meinte Hermine.

\enquote{Missliche Situation?}, sagte Ron leicht indigniert. \enquote{Missliche\abs}, doch er wusste nicht mehr weiter.

\enquote{Ron, beruhige dich}, meinte Harry. \enquote{Was ich euch beiden jetzt zeigen werde, lässt dich schnell vergessen, dass Hermine wohl etwas zu früh geplaudert hat.}

\enquote{Na hoffentlich}, sagte Ron barsch.

\enquote{Aber du musst noch eines wissen. Nein, eigentlich sind es doch eher zwei Dinge. Erstens. Kein Wort zu Mister Filch, einem der Lehrer oder Dumbledore. Kein Wort zu irgendjemand außer einem der vielen anderen Pärchen der sechsten oder siebten Stufe. Klar?}, schloss Harry.

Ron sah Harry leicht komisch an: \enquote{Klar!}

\enquote{Und das Zweite! Fast alle Pärchen der sechsten oder siebten Stufe werden ab jetzt davon erfahren.} Leichte Panik stieg in Ron empor. \enquote{Keiner wird sich etwas anmerken lassen. Selbst Mal\aabs}, doch Harry konnte sich gerade noch bremsen.

\enquote{Selbst wer nicht?}, fragte Ron nach.

Doch Harry ignorierte ihn und meinte nur: \enquote{Versprichst du's?}

\enquote{Also gut. \gst Ja ich verspreche es.}

\enquote{Du auch Hermine?}

\enquote{Ja, ich verspreche es.}

\enquote{Na gut. Alles, was ihr ab jetzt zu sehen und zu hören bekommt, ist Top Secret.}

\enquote{Kommt mit}, und Harry lief mit Luna noch um die letzten Ecken bis sie vor das große Porträt kamen.

\zauber{Aqua Neros!}, sagten Harry und Luna unisono und das große Porträt schwenkte zur Seite.

Harry ließ Ron und Hermine den Vortritt und folgte dann Luna in das Loch. Das Porträt schloss sich und kurz darauf standen die vier im Gemeinschaftsraum der Paare.

Harry beeilte sich, um vor Hermine und Ron zu gelangen, und ihren Gesichtsausdruck aufzuschnappen. Er setzte sich auf ein leeres Sofa, Luna nahm neben ihm Platz und legte ihre Hand in seine. Stumm standen die beiden da und Harry musste schmunzeln. Ein ungewohnter Anblick. Doch als Ron Malfoy und Maria sah, fiel ihm die Kinnlade herunter. Draco konnte nicht anders und ihm entwich eine kleine Gehässigkeit.

\enquote{Aha, Granger und Weasley. Die beiden rötesten im ganzen Raum. Und das nicht nur von den Kopfhaaren her.} Und dann zu Harry gewandt: \enquote{Wird wohl Zeit für die beiden endlich zu Schubbern.}

Hier benahm sich Malfoy zumindest anständiger als draußen, denn ihm wurde wohl klar, dass er hier jede Menge Prügel und Flüche einstecken konnte und nachher keine Erklärung haben würde, woher und wie er die sich eingefangen haben konnte. Mit diesen Worten und Maria bei der Hand verließ er den Raum. Ron, immer noch sprachlos, sah nur Harry und Luna an.

\enquote{Dies, ist der Gemeinschaftsraum der Paare}, sagte Luna. \enquote{Nun gehört ihr auch dazu.}

Harry und Luna gaben Ron und Hermine zu verstehen, sie mögen sich doch bitte setzen, und zogen dann das gleiche Programm wie bei den anderen durch. Anschließend zeigten sie den beiden ihren Raum (es stand bereits ihr Name darauf) und zogen sich dann wieder in den großen Gemeinschaftsraum zurück. Es dauerte eine Weile, bis Hermine und Ron zurückkamen. Harry und Luna spielten gerade eine Runde Schach und dem aufmerksamen Zuschauer dürfte nicht entfallen sein, dass sich Harrys Spiel verbessert hatte.

\enquote{Aha, das ist also deine Übungspartnerin}, meinte Ron lapidar und setzte sich.

\enquote{Wieso sind hier überall nur Doppelbetten?}, fragte Hermine.

\enquote{Dies ist doch der Gemeinschaftsraum der Paare}, meinte Luna nur.

\enquote{Es ist ja nicht so, dass ihr unbedingt jedes Mal wenn ihr hier seid miteinander schlafen müsst. Die ersten paar Male lagen wir einfach nur nebeneinander und haben geschlafen.}

Hermine verzog leicht ihren Mund und machte ein Gesicht, das Harry sagte: \gedanke{Und dann habt ihr wohl miteinander geschlafen.}

Harry konnte nicht anders und meinte nur: \enquote{Genau.}

Hermine erschrak, so als hätte Harry ihre Gedanken gelesen.

\enquote{Hast du?}, fragte Hermine.

Harry antwortete nur: \enquote{Was genau? Deine Gedanken gelesen oder mit Luna geschlafen?}

Währenddessen öffnete sich das Porträt und weitere Paare kamen herein.

\enquote{Ach schau an. Mister Weasley und Miss Granger. Hab's doch gewusst. Seit wann sind die denn hier, Harry?}, fragte Donan.

\enquote{Seit etwa einer halben Stunde}, antwortete Harry.

\enquote{Tja, unserem Harry entgeht eben nichts}, grinste er und verschwand mit seiner Freundin um die Ecke, um sich zu duschen. Die beiden sahen nämlich sehr verschwitzt aus.

Die nächsten Tage war in der gesamten Schule nur vom Besuch der Verwandtschaft die Rede. Die Hausaufgaben wurden etwas vernachlässigt, da über den Jahreswechsel schulfrei war. Luna und Harry zeigten sich jetzt offiziell als Paar in der Schule und versteckten sich nicht länger. Die erste Zeit gab es einiges an Getuschel, welches sich aber bald wieder im Schulalltag verlor.

Harry las sich noch schnell die wenigen Seiten des schmalen Buches durch. Dann kopierte er sie, da er das Buch wieder abgeben musste. Er musste endlich den Test mit seinem Umhang machen.

Diese beiden Tage unterhielten sich Philips Eltern mit Professor Elber und Madame Pomfrey immer wieder und auch mit ihrem Sohn. Es dauerte eine Weile, bis sich die zwei für ihren Sohn entschieden, auch wenn sich wohl die restliche Verwandtschaft dagegen stellen würde. Seine Familie musste nicht an Hunger leiden und war gesellschaftlich recht gut angesehen. Man kam überein, dass die beiden es versuchen sollten. Demnächst sollte es losgehen.

Während dieser Zeit machte sich Harry daran, einen Trank zu brauen, der ihm die Gewissheit bringen sollte, ob er wirklich von den Peverells abstammte. Er stand also im Tränkelabor und gab gerade ein Haar zu. Dann schäumte es in seinem Kessel. Mit einer flachen Kelle schöpfte er den Schaum ab und warf ihn in das Waschbecken daneben. Dann, als der Trank sich wieder beruhigt hatte, gab er einen Tropfen Blut hinzu. Nach einigen Minuten des leise köcheln Lassens, siebte er ihn in einen Becher, belegte diesen mit einem Warmhaltezauber und verschloss den Becher selber magisch, damit er ihn nicht aus Versehen verschütten konnte.

In seinem Zimmer angekommen; er hatte seinen Arbeitsplatz sauber verlassen; holte er seinen Tarnumhang hervor und trank einen Schluck aus dem Becher. Den Rest goss er über den auf dem Boden ausgebreiteten Mantel. Nach einer knappen Minuten fingen sein Umhang, sowie er selbst, rot zu leuchten an. Dieses Leuchten hielt eine gute Minute. Dann verblasste es. Er schaute noch einmal in seinen Unterlagen nach, um sicherzugehen.

Jetzt hatte er Gewissheit. Er war einer der Nachfahren der Peverells. \gedanke{Nein}, korrigierte er sich selbst. \gedanke{Ich bin der Nachfahre einer der Peverells. Die beiden anderen Brüder hatten keine Nachkommen.} Er wusste nun etwas mehr über seine Vergangenheit. Aber es sagte ihm noch immer nichts Genaues.

\trenn

Morgens, kurz vor dem Frühstück, Hermine und Harry saßen bereits im Gemeinschaftsraum und warteten auf Ron und Ginny, kam Tamara Malfoy die Treppe herunter und fragte: \enquote{Harry, Hermine. Ich brauche eure Hilfe.}

Ron trat in den Raum und setzte sich auf die andere Seite des Sofas, einen Platz zwischen sich und Harry lassend. \enquote{Wobei?}, fragte er.

\enquote{Ihr müsst mir aber versprechen, nicht sauer oder wütend auf mich zu sein}, fügte sie ängstlich hinzu.

Die drei nickten und so fuhr Tamara mit ihrer Bitte fort. \enquote{Ihr kennt doch meinen Bruder!}

\enquote{Den Arsch}, bemerkte Ron.

Hermine trat ihm gegen sein Schienbein. \enquote{Ron!}

\enquote{Ok, ok.}

\enquote{Schon in Ordnung. Ich weiß ja, wie er auf andere wirkt.} Sie setzte sich zwischen Harry und Ron. Beschämt betrachtete sie ihre Finger. \enquote{Er ist eigentlich ganz lieb. Nur wenn er unter Leuten ist, dann muss er eine Maske aufsetzen. Wegen seines Vaters. Aber sagt das bitte niemandem weiter.} Die drei nickten etwas ungläubig. \enquote{Seit wir nicht mehr Zuhause wohnen, verliert er zunehmend den Halt. Seine eigenen Mitschüler aus seinem Haus distanzieren sich von ihm.}

\enquote{Aber er hat doch noch Crabbe und Goyle.}

\enquote{Die zählen nicht}, sagte Tamara nun zu Harry gewandt. \enquote{Die sind wie Hunde für ihn. Laufen ihm nach, wohin er auch geht. Aber selbst die beiden fangen an, sich von ihm abzukapseln.}

\enquote{Er hat's auch verdient \gst Au \gst Hermine.} Sie war ihm wieder gegen sein Schienbein getreten.

Jetzt stand Ron auf und setzte sich in einen Sessel neben Harry. \enquote{So kannst du mir wenigstens nicht mein Bein kaputt treten.}

Beide wechselten eigenartige Blicke, fiel Harry auf.

\enquote{Harry, ich möchte dich, und auch euch Ron und Hermine, nun um etwas bitten. Aber seid bitte nicht wütend auf mich.} Sie sah Harry mit klimpernden Wimpern an. Harry konnte diesem Blick nicht lange widerstehen, das hatte Tamara recht bald gelernt. Er nahm sie brüderlich in den Arm und wuschelte durch ihr Haar. \enquote{Harry}, sagte sie schüchtern.

\enquote{Ich versprech's.}

Dann sah sie Ron und Hermine an. Gerade als Hermine Ron über Tamara und Harry hinweg in die Seite knuffen wollte, versprach er es ebenfalls und auch Hermine stimmte zu.

\enquote{Ich möchte von euch, dass ihr beginnt, euch mit meinem Bruder anzufreunden.}

Harry musste einen Hustenanfall unterdrücken. Ron und Hermine schauten sie fragend an.

\enquote{Tamara}, sagte Hermine schließlich. \enquote{Wir verstehen uns mit deinem Bruder nicht sonderlich gut. Seit unserem ersten Tag hier, sind wir so etwas wie verfeindet}, sagte Hermine.

\enquote{Und aus diesem Grund sollt ihr anfangen, euch mit ihm anzufreunden. Wenn ihr das macht, dann gewinnt er vielleicht auch wieder Freunde aus seinem Haus}, sagte sie mit leicht feucht werdenden Augen.

\enquote{So einfach wie du dir das vorstellst, ist das aber nicht. Im Leben kann man sich nicht alles wünschen}, antwortete Hermine.

\enquote{Das weiß ich}, gab sie leicht verärgert zurück. \enquote{Nur weil ich eine Malfoy bin, heißt das noch lange nicht, dass ich nichts vom Leben weiß.}

\enquote{Vielleicht liegt es auch nur daran, dass ich ihn gekränkt habe}, antwortete Harry sanft.

\enquote{Wie das?}, fragte sie.

Harry überlegte kurz und sagte dann: \enquote{Ich erzähle dir die Geschichte. Aber sage sie nicht deinem Bruder, verstanden?}

Tamara nickte.

Dann nahm Harry sie auf seinen Schoß. Sie saß nun auf seinen Oberschenkeln, die Beine Richtung Rückenlehne; leicht angewinkelt. Harry hielt sie an ihrer Hüfte fest. Ihre Augen waren leicht feucht. Dahinter konnte er ihre strahlend dunkelgrauen Augen erkennen. Ihr fast schon goldenes Haar fiel über ihre Schulter und hörte in der Mitte ihrer Schulterblätter auf.

Mittlerweile war es im Gemeinschaftsraum leise geworden. Diejenigen, die noch da waren, hatten unterbrochen, was sie taten, um Harry zuzuhören.

\enquote{Vor meinem elften Geburtstag wusste ich nichts von Hexen oder Zauberern. Ich wusste nichts von der magischen Welt, deiner Welt \gst die mittlerweile auch zu meiner Welt geworden ist. Als ich meinen ersten Brief bekam, hatte ich keine Gelegenheit ihn zu lesen. Mein Onkel wunderte sich darüber, wer mir wohl schreiben würde. Gehässig wie immer, warf er ihn in den Kamin, als er das Wappen auf der Rückseite erkannte.} Er schluckte kurz. \enquote{Und jeden weiteren Brief, den die Eulen brachten. Er vernagelte sogar den Briefschlitz in unserer Haustür. Schließlich, als nichts half, sind wir kurz vor meinem Geburtstag\abs} Er unterbrach sich kurz und schaute ihr nun mit leicht schrägem Kopf in die Augen. \enquote{Weißt du, wann ich habe?}, fragte er sie.

Sie nickte. \enquote{Ende Juli}, sagte sie.

\enquote{Am 31.}, antwortete Harry, \enquote{sind wir, das heißt mein Onkel, meine Tante, mein Cousin und ich, in einen alten verlassenen Leuchtturm gezogen, damit die Briefe mich nicht finden würden. Es war am Tag vor meinem Geburtstag, als ich abends eine Torte in den staubigen Boden malte und mir kurz nach Mitternacht was wünschte, die Augen schloss und die imaginären Kerzen ausblies. Es war eine stürmische Nacht und plötzlich schlug etwas mit aller Gewalt gegen die Tür. Sie flog auf und ein großer, bärtiger Mann stand im Türrahmen. Er trat herein und schloss hinter sich die Tür.} Tamara schaute ihn mit entsetztem Blick an. \enquote{Der Riese\abs eigentlich Hagrid\abs kam auf mich zu.} Er spürte, wie sie ausatmete und ihre Anspannung verlor. \enquote{Er erzählte mir etwas über meine Eltern, dass sie nicht, wie ich bis dahin immer geglaubt hatte, bei einem Autounfall ums Leben gekommen seinen, sondern ermordet wurden.} Tamara bekam wieder große Augen. Sie rückte etwas näher an ihn heran. \enquote{Er überreichte mir den Brief von Hogwarts. Einen der Briefe, die ich schon seit langem hätte lesen wollen. Mein Onkel und meine Tante getrauten sich nicht, etwas zu sagen.}

Selbst Ron und Hermine, nebst Ginny, die sich kurz nachdem Harry zu erzählen begonnen hatte, setzte, lauschten gespannt Harrys Worten, denn bisher hatte er auch ihnen nichts Genaues darüber berichtet.

\enquote{Also las ich ihn. Ich bin dann am nächsten Tag mit Hagrid einkaufen gegangen. Du kannst dir das Gefühl gar nicht vorstellen. Doch ich musste noch einen Monat zurück zu meinen Verwandten. Dann, am Tag der Abreise, fuhr mich mein Onkel widerwillig zum Bahnhof und ließ mich zwischen Gleis neun und Gleis zehn stehen und sagte: \inner{Hier Junge, such dein Gleis. Viel Spaß.} Dann drehte er sich um und ging. Tja, nun stand ich da. Verloren und einsam. Wie in einem bösen Traum. Ich ging auf dem Bahnsteig entlang, bis ich ein Wort hörte: \inner{Muggel.} Sofort drehte ich mich zu der Frau hin und lief der Gruppe hinterher, die sie begleitete. Dann sah ich zwei Jungs in einer Mauer verschwinden. Ich ging zu der Frau hin und wollte wissen, wie man zum Gleis kommt. Sie erklärte es mir und ich rannte durch die Mauer und sah den Zug. Erleichtert fiel mir ein Stein vom Herzen.}

\enquote{Wer war die Frau?}, wollte Tamara wissen.

Harry sah zu Ron und antwortete: \enquote{Rons Mutter.}

Tamara drehte sich kurz um zu Ron, danach wieder zu Harry. Gedankenverloren sah er seinen besten Freund an. Bis ihn Tamara stupst und meinte: \enquote{Erzähl weiter.}

Harry sah Tamara wieder in die Augen. \enquote{Ich saß also im Zug in einem leeren Abteil und konnte mein Glück gar nicht fassen. Ein Jahr lang ohne Onkel und Tante und Cousin auf dem Weg nach Hogwarts. Dann ging die Abteiltür auf und Ron fragte mich, ob er sich zu mir setzen könne. Ich bejahte und wir stellten uns vor.} Jetzt lächelte Harry ganz leicht. \enquote{Ron war ziemlich aufgeregt, als ich mich vorgestellt habe.}

Tamara wischte sich das noch feuchte Gesicht ab und Harry fuhr mit seiner Geschichte fort.

\enquote{Ron erzählte mir von den vier Häusern in Hogwarts, von guten und von bösen Zauberern und Hexen und\abs dass fast alle schwarzen Magier aus Slytherin kamen. Ich wusste nun genau, dass ich dahin auf keinen Fall wollte. So langsam freundete ich mich mit Ron an. Die Zugfahrt war schließlich lang. Für eine kurze Weile kam Hermine zu uns, verschwand aber recht bald wieder. Dann fuhr der Zug in den Bahnhof ein und Hagrid brachte uns mit den Booten zum Schloss.}

Wieder pausierte Harry kurz.

\enquote{Als wir dann von Professor McGonagall begrüßt wurden und sie uns nochmals kurz verließ, stellte sich mir dein Bruder vor und bot mir seine Freundschaft an.} Tamara bekam große Augen. \enquote{Ron sagte mir, dass seine Familie seit Generationen in Slytherin ist. Das war wohl ein weiterer Grund, warum ich sagte, was ich zu ihm gesagt habe. Ich sagte in etwa: \enquote{Ich suche mir meine Freunde selber aus.} \gst Damit muss ich ihn wohl gekränkt haben. Es war vielleicht das erste Mal in seinem Leben, dass er nicht das bekam, was er wollte. Dann schritten wir in die Große Halle. Ich hatte Angst ohne Ende. Ich wusste nicht, was mich erwartete. Professor McGonagall setzte uns schließlich den sprechenden Hut auf. Draco kam noch vor mir dran. Er wurde nach Slytherin geschickt. Für mich ein Grund mehr, nichts mit ihm zu tun haben zu wollen.}

\enquote{Nur aufgrund des Hauses?}, fragte Tamara ungläubig. Es standen schon wieder Tränen in ihren Augen. Harry nahm ein Taschentuch heraus und trocknete ihr die Tränen. Mittlerweile hatte er den gesamten Gemeinschaftsraum in seinen Bann gezogen.

\enquote{Ich weiß, das war dumm von mir, aber ich hatte vor wenigen Stunden das erste Mal richtigen Kontakt mit der magischen Gemeinschaft. Ich wusste vorher sonst nichts.} Er senkte betrübt seinen Kopf.

Tamara kroch an ihn heran und legte ihren Kopf auf seine Schulter. Dann begann Harry weiterzuerzählen.

\enquote{Schließlich wurde auch mir der sprechende Hut aufgesetzt. Deinen Bruder hatte der Hut ja nach Slytherin geschickt, kaum dass er ihn aufgesetzt hatte. Bei anderen brauchte er ein bis zwei Sekunden, das ging sogar ziemlich schnell.}

Harry pausierte wieder kurz und sah danach an Ron und Hermine vorbei ins Leere.

\enquote{Dann saß ich auf dem Stuhl, den Hut auf meinem Kopf. Und ich hörte den Hut in meinem Geiste sprechen.}

Tamara setzte sich wieder auf und sah ihm in die Augen. Sie versuchte seinen Blick wiederzuerlangen, scheiterte aber.

Harry fuhr nun leiser fort. \enquote{Der sprechende Hut sagte zu mir: \inner{Oh ja, ich erkenne Mut und den Drang sich zu beweisen. Interessant. Im Köpfchen hast du es auch. Aber wo stecke ich dich hin.} Und ich dachte nur, nicht Slytherin. \inner{Nicht Slytherin}, hörte ich dann in meinem Kopf. \inner{Aber Slytherin könnte dir auf dem Weg zu wahrer Größe verhelfen}, sagte der Hut. Und ich dachte nur immer und immer wieder: \inner{Nicht Slytherin.} Dann sagte der Hut: \inner{Also gut, wenn du dir so sicher bist.} Und dann laut für alle hörbar: \enquote{Gryffindor.}}

Jetzt sah er Tamara wieder an. \enquote{Verstehst du? Der Hut gab mir die Wahl und ich habe mich entschieden. Ich hätte auch nach Slytherin gehen können.} Und dann nach einer kleinen Pause noch leiser zu Tamara: \enquote{Genau wie bei dir. Habe ich recht?}

Tamara drehte sich errötend weg von ihm, doch er nahm ihr Kinn sanft zwischen seine Finger und drehte ihren Kopf zu sich. Sie nickte nur stumm. Harry grinste. Danach küsste er ihre Stirn, stand auf und sagte: \enquote{Wird Zeit zu frühstücken.}

Jetzt war sie wieder bei klarem Verstand. Sie stand auf und sagte: \enquote{Warte Harry, ich muss noch was holen.} So schnell wie sie verschwand, so schnell war sie auch wieder da. Sie drückte ihm eine Rolle Pergament in die Hand und meinte: \enquote{Gib das Draco, gleich wenn du in die Große Halle kommst. Sei nett zu ihm und benutze seinen Vornamen. Sag ihm, dass du es in der Bibliothek gefunden hast und es wohl ihm gehöre.}

Harry und seine beiden Freunde sahen die kleine Tamara verständnislos an. \enquote{Du hast es mir versprochen}, sagte sie mit einem untrüglichen Wimpernschlag. Harry verlor die Kraft, nein zu sagen. \enquote{Gut. Ich mach's}, sagte er.

Tamara verließ den Gemeinschaftsraum durch das Porträtloch. Harry stand noch einige Sekunden da, bevor er sich wieder fasste und das Pergament zu lesen begann. \enquote{Das ist gut, was Mal\gst Draco da schreibt.} Und nach einer Weile fügte er hinzu: \enquote{Das ist sogar sehr gut.}

Hermine stellte sich neben ihn, um überflog ebenfalls den Zaubertrankaufsatz. \enquote{Es scheint}, sagte sie, \enquote{dass Draco bei Snape nicht nur sein Liebling ist, sondern er auch für seine Noten ordentlich arbeitet.}

Harry rollte das Pergament wieder zusammen und zog Ron und Hermine wie durch einen Zauber hinter sich her. Wenige Schritte vor der Großen Halle atmete er einmal tief durch und betrat diese. Er stoppte kurz, um nach Draco Ausschau zu halten, und Schritt dann zielsicher auf ihn zu. Dieser merkte es erst, als Harry fast schon hinter ihm stand. \enquote{Mal\gst Draco. Ich glaube, das ist deines. Du hast es gestern in der Bibliothek vergessen. Es war schon spät, also bringe ich es dir erst jetzt.} Er reichte ihm seine Pergamentrolle.

Dieser nahm sie und schaute kurz nach, was es denn sei. \enquote{Woher weißt du, dass es meine ist?}, fragte er ganz erstaunt und ohne Anflug von Hass oder Ärgernis.

\enquote{Ich kenne nach fünf Jahren Schule deine Handschrift eben.} Dann verließ er den Slytherintisch und setzte sich neben Tamara, um zu frühstücken.

Diese lächelte und winkte ihrem großen Bruder zu, nachdem sie sich kurz umgedreht hatte. Draco lächelte zurück.

\trenn

Bevor er an diesem Abend ins Bett ging, nahm Harry sein Amulett in die Hand und wickelt die Kette um seine Finger, damit er sie nicht verlieren konnte. Dann stieg er ins Bett, lies die Vorhänge offen und legte sich hin. Er fühlte die langsam aufkommende Wärme seines Amulettes und dämmerte nach einer halben Stunde weg. Harry träumte nun.

\begin{traum}
Er stand in Godrics Hollow und sah auf das Haus, welches er so oft in seinen Alpträumen gesehen hatte. Es war ein kleines Steinhaus. Zumindest sah es so aus. Er betrat den Vorgarten durch das Gartentor. Es quietschte leicht, als er es öffnete. \gedanke{Da dürfte etwas Öl angebracht sein}, dachte er sich.

Er lief langsam aber sicher den Weg mit den Steinplatten und dem gepflegten Garten Richtung Haustür. Zu seiner linken Seite sah er einen kleinen Obstbaum, er kannte die Sorte nicht, und darunter ein kleines Blumenbeet. Eine einzelne Figur stand zwischen den Blumen. Sie hatte Ähnlichkeit mit einem Gartenzwerg, obwohl sie ganz aus Stein war und keine Mütze aufhatte. Auf seiner rechten Seite sah er mehrere Tannen und Fichten und einen Weg, der hinter das Haus führte. Durch eine lichte Hecke sah er auf das Nachbargrundstück. Mittlerweile hatte er die Tür erreicht und seine Hand am Griff. Er öffnete die Haustür und ging hinein. Die Eingangshalle war klein, aber geschmackvoll eingerichtet. Dort standen mehrere kleine Tischchen mit Blumenvasen, die Pflanzen der Saison enthielten. Schmucke Verzierungen waren über den beiden Türen zu erkennen. Eine war auf der linken Seite in der Wand, eine andere war Harry gegenüber. Rechts von ihm sah er noch eine weitere. Er ging auf die linke Tür zu, da er Stimmen zu hören meinte.

Er nahm den Griff in die Hand, drückte ihn herunter und öffnete die Tür. Die Stimmen waren nun lauter. Auf dem Sofa, das inmitten des Raumes stand und das seitlich zum Kamin aufgestellt wurde, saßen zwei Personen. Harry kamen sie merkwürdig bekannt vor, er konnte sie aber nicht zuordnen. Die beiden sahen ihn an und baten ihn, sich zu setzen. Harry nahm ihnen gegenüber in einem gemütlichen grünen Ledersessel Platz. Das Sofa war ebenfalls in derselben Farbe gehalten. Hinter dem Sofa war ein Fenster, das die Obstbäume zeigte, die Harry bei seinem Gang durch den Garten gesehen hatte.

Langsam dämmerte ihm, wen er vor sich sah. Und mit einem Mal hatte er Gewissheit, denn die Frau sprach ihn an: \enquote{Hallo Harry, schön, dass du hier bist.}

\enquote{Mum, Dad}, war alles, was er hervorbrachte. Und als er sich endlich gefasst hatte, sagte er: \enquote{Ihr seid Tod, ihr könnt nicht hier sein.}

\enquote{Natürlich sind wir Tod, Harry}, sagte sein Vater. \enquote{Du siehst uns nur, weil du träumst und deine Magie es dir ermöglicht. Du hast es also geschafft.}

Mit großen Augen sah er die beiden an. \enquote{Nein\abs ich\abs bin nur\abs Er hat\abs bin nur mit\abs Amulett\abs eingeschlafen\abs wenn ich schlecht schlafen könnte, oder aufgeregt bin. Es würde mich beruhigen.}

\enquote{Dann bist du nur zufällig hier?}, fragte ihn sein Vater.

\enquote{Ja}, antwortete Harry. \enquote{Aber schön, dass ihr da seid.}

\enquote{Nun,} antwortete seine Mutter, \enquote{das hängt vom Standpunkt ab.} Und als sie Harrys fragendes Gesicht sah, fügte sie hinzu: \enquote{Es könnte genauso gutheißen: Schön, dass du da bist.}

Und Harrys Vater sagte: \enquote{Ob wir in deinen Träumen sind, oder du bei uns bist, kann man nicht genau sagen.}

\enquote{Harry, hör mir zu. Wir haben nicht viel Zeit. Es braucht lange, bis die Verbindung aufgebaut ist, und es kostet uns viel Kraft. Du kannst nicht jeden Abend mit uns Kontakt aufnehmen. Es gibt etwas, was wir dir sagen wollen. Wir wissen nur nicht, wie du es auffassen wirst}, sagte seine Mutter.

\enquote{Na ja, ich werde euch keine Vorwürfe machen}, antwortete Harry.

Seine Mutter hielt sich eine Hand auf ihren Bauch und nahm seines Vaters Hand in ihre andere.
\end{traum}



\begin{kommentar}
Während den Weihnachtsferien, an einem Dienstag, ist Elber im Malfoy Manor und spielt gerade mit Lucius Schach, als Bellatrix ihre Nichte zu foltern beginnt. Elber rennt sofort in den Salon und entwaffnet Bellatrix. Dann bestraft er sie, indem er sie mit blau-violetten Blitzen belegt. Jene Blitze, die der Imperator in Krieg der Sterne (Star Wars) auf Luke Skywalker legte. Eine nette kleine Anspielung, wie ich finde.
\end{kommentar}

\begin{kommentar}
Nachdem er Bellatrix in einen Sessel gesetzt hat, berührt er ihre Stirn und meint, dass sie eine halbe Stunde für sich hätte. Ein weiterer kleiner Hinweis darauf, dass Bellatrix eine gespaltene Persönlichkeit hat.
\end{kommentar}

\chapter{In der Kammer}

\begin{traum}
\enquote{Harry, wenn Voldemort uns damals nicht getötet hätte, \gst dann hättest du jetzt eine kleine Schwester}, sagte Harrys Mutter.

Harry musste das erst einmal verarbeiten. Er stand auf und setzte sich zwischen seine Eltern. Dann legte er seine Hand auf ihre. Sein Vater legte eine Hand auf seine Schulter. Ihre Berührungen fühlten sich so real an. Davon hatte Harry lange geträumt. Endlich seine Eltern in seinen Armen zu halten. Harry rutschte näher an seine Mutter heran und legte einen Kopf auf ihre Schulter. Seinen Vater zog er mit einer Hand zu sich, um ihm zu signalisieren, dass er näher kommen möge.

\enquote{Welchen Namen wolltet ihr meiner\abs Schwester geben?}, fragte Harry.

\enquote{Als wir noch lebten}, sagte sein Vater hinter ihm, \enquote{hatten wir noch keine Zeit dazu, aber jetzt haben wir uns auf Jamie geeinigt.}

Harry lächelte seine Mutter an und gab ihr einen Kuss auf die Wange. Eine Hand war noch immer auf ihrer, welche auf ihrem Bauch lag.

Dann begann sich Schwere in seinen Gliedern bemerkbar zu machen. Die Umgebung wurde langsam undeutlicher. \enquote{Bis bald, Harry}, sagten beide Eltern. Dann verschwand die Umgebung und Harry schwebte.
\end{traum}

Er öffnete seine Augen und wunderte sich, warum die Decke in seinem Bett plötzlich aus Stein bestand. Dann realisierte er, dass er keine Bettpfosten sah und auch das Bett fühlte sich anders an.

\enquote{Madame Pomfrey. Er ist wach}, hörte er eine Stimme sagen.

\gedanke{Hermine, das war Hermine. Ich bin im Krankenflügel.} Er hob leicht seinen Kopf und sah in Hermine und Rons Augen.

Madame Pomfrey trat an ihn heran und meinte: \enquote{Endlich, Mister Potter. Wir hatten uns schon Sorgen gemacht. Sie sind gar nicht mehr aufgewacht.}

Harry sah sie verwundert an. \enquote{Ich habe doch nur normal geschlafen. Wie spät ist es?}

\enquote{Fast Mittag}, sagte Ron.

Harry schaute Ron erstaunt an. \enquote{Dann hatte ich doch etwas länger geträumt.} Er richtete sich auf, hob die Hand vor seinen Bauch und öffnete sie. Darin lag noch immer sein Basilisken-Amulett. Er lächelte, als er an seinen schönen Traum dachte. Oder war es doch mehr? Hatte er wirklich Kontakt zu seinen Eltern hergestellt? Oder hatte das Amulett nur seine Träume beeinflusst und ihm die Illusion seiner Eltern gegeben? Er hatte nur eine Chance das zu erfahren. Er musste jemanden finden, der seine Eltern gekannt hatte und die von seiner ungeborenen Schwester wusste. Aber das war unmöglich. Er musste also später noch einmal diesen Traum erleben und sie etwas fragen, was er unmöglich über einen andere Wege erfahren konnte, um es sich bestätigen zu lassen. Denn andersherum würde es nicht funktionieren.

\trenn

Er saß alleine auf einer der vielen Treppen in Hogwarts, um über seine Vision nachzudenken. Er wusste nicht, wie er es herausfinden sollte, ob das, was er in seiner Vision gesehen hatte, wahr war, oder doch nur ein Traum. Ganz in Gedanken, seinen Blick durch die Mauer hindurch in die Unendlichkeit gerichtet, begann die Luft vor ihm zu flirren und Salazar erschien ihm wieder.

\enquote{Harry, hör mir bitte zu. Ich habe mich\abs na ja\abs etwas in deinem Kopf umgesehen und festgestellt, dass dir Okklumentik-Übung fehlt. Du hast hier einen guten Lehrer darin, nur seid ihr nicht gerade gut aufeinander zu sprechen. Gehe deshalb in meine Kammer, hole dort vom toten Basilisken Trankzutaten und besorge eines meiner Bücher. Sie sind auf Parsel geschrieben. Du wirst sie lesen können. Gib die Sachen deinem Lehrer. Ich denke, er wird dir dankbar sein und den Unterricht wieder aufnehmen, wenn du ihn darum bittest. Die Tränke sind sehr alt und wertvoll. Schreib ein paar davon ab \gst du wirst wissen, welche \gst und bring sie ihm mit den Zutaten.}

\enquote{Du bist in meinem Kopf?}, fragte Harry erstaunt.

\enquote{Du bist ein Teil von mir, Harry.}

\enquote{Ja, aber\abs}

\enquote{Kein aber, Harry. Du bekommst von mir ebenso viel zurück. Meine magische Stärke, meine Duellierfähigkeiten und mein Wissen, das ich über die Jahre angesammelt habe. Es liegt noch verborgen in dir. Du wirst darauf Zugriff haben, wenn du es brauchst. Danach kannst du es dauerhaft und wirst dich immer daran erinnern.}

Die Luft begann wieder zu flirren und Salazar verschwand.

Harry saß noch einige Minuten auf der Treppe und dachte nach, bis er schließlich auf sein Zimmer ging und zu packen begann.

Er holte seinen Rucksack heraus und packte Pergamente ein, sowie eine Feder und ein kleines Tintenfass. Er zauberte sich Phiolen und kleine Tücher zum Einwickeln herbei, kleine Holzkästchen für größere Gegenstände, diverse Werkzeuge zum Zerkleinern, verkleinerte alles und zog sich danach festes Schuhwerk an.

Salazar hatte ihm nicht gesagt, dass er Hilfe mitnehmen sollte, also ging Harry alleine.

Im Stockwerk mit der maulenden Myrte angekommen, betrat er das Mädchenklo.

\enquote{Myrte?}, fragte er in den Raum hinein, doch es kam keine Antwort.

Vorsichtig schaute er sich um, bis er schließlich feststellte, dass sie nicht da war. Er ging zum Waschbecken, suchte das spezielle mit der Schlange auf dem Wasserhahn und sprach in Parsel: \parsel{Öffne dich.}

Das Waschbecken gab den Zugang frei und Harry rutschte die Röhre hinab in die Dunkelheit.

Unten angekommen fiel er wie letztes Mal in zahlreiche Knochen und Knochenreste.

\gedanke{Mahlzeiten eines Basilisken}, dachte er amüsiert. Doch nach Entspannung, oder gar Frohsinn, war ihm nicht. Er hörte ein dumpfes \geraeusch{Klonk}. Der Zugang muss sich wieder verschlossen haben. \gedanke{Darüber mache ich mir später Gedanken}, dachte sich Harry und stieg über das Geröll und den Schutthaufen, den damals Lockhart durch seinen fehlgeschlagenen Zauber ausgelöst hatte. Kurz danach sah er schon die riesige Schlangenhaut.

Er holte ein Messer aus seinem Rucksack und schnitt ein großes Stück heraus, vergrößerte ein eingepacktes Tuch, wickelte die Schlangenhaut darin ein und verkleinerte es. Danach schob er es in seinen Rucksack. Nachdenklich sah er den Rest der Haut an und entschloss sich, einen Verkleinerungs-Zauber an der ganzen Haut zu probieren. Nachdem der Versuch geklappt hatte, steckte er diese Haut ebenfalls ein.

\gedanke{Ich denke, ein persönlicher Vorrat an Trank"-zu"-ta"-ten wäre vielleicht nicht schlecht.}

Er schritt weiter mit leuchtendem Zauberstab den gewundenen Gang entlang, bis er abermals vor der kreisrunden Tür mit den vielen Schlangen stand.

Wie schon in seinem zweiten Jahr befahl er der Tür erneut sich zu öffnen und trat danach in die Kammer. Am Ende konnte er noch immer den Körper der Schlange sehen. Er schien kaum verwest zu sein. Vor dem riesigen Tier angekommen, sammelte er alles ein, von dem er dachte, er brauche es. Er nahm sich Schuppen, Zähne, Blut, Speichel und Gift. Und wieder zweigte er sich einen Teil für seinen persönlichen Vorrat ab.

Dann versuchte er den Basilisken zu verkleinern und steckte ihn ebenfalls in seinen Rucksack. Langsam wurde er schwer, daher stelle er ihn an den Rand des Raumes und sah zu Salazars Statue.

Er dachte eine Weile nach, bis ihm die Idee kam. \gedanke{Das muss Salazars Wissen sein}, dachte er sich. Er rief ihn. \enquote{Salazar?}

Kurz darauf schwebte der alte Zauberer vor seiner eigenen Statue.

\enquote{Ja, Harry?}

\enquote{Wollen wir ein bisschen spazieren gehen?}

\enquote{Gerne, aber ich kann doch nicht lange\abs}

Harry unterbrach ihn. \enquote{Ich verfüge über dein Wissen und eine Statue von dir. Sagt dir das was?}

Salazar runzelte erst die Stirn, dann kam ihm die Erkenntnis. Er nickte, drehte sich kurz um, um sich seine Staute zu beschauen, und sah dann wieder zu Harry. Er stellte sich so wie seine Statue hin und wartete.

Harry zog seinen Zauberstab und murmelte einen Zauber, woraufhin es Salazars Geist nach hinten in seine Staute zog. Kurz darauf fing sie an zu blinken und bewegte sich. Es dauerte noch eine knappe Minute, bis der weiße Marmor Farbe bekam und es so schien, als ob Salazar Slytherin leibhaftig vor ihm stand. Beide wusste, dass es nicht von langer Dauer war, aber die gemeinsame Zeit wollten beide nutzen.

\enquote{Sag deinem Elfen Bescheid, Harry, dass er deine Trankzutaten für dich einlagern soll und den Rest für dich auf deinem Zimmer sicher verstauen soll. Du wirst es dort wieder finden.}

Harry nickte und rief nach Kreacher.

Dieser erschien und verneigte sich. \enquote{Sir Harry hat Kreacher gerufen}, krächzte der Elf.

Kreacher hatte sich in den letzten Wochen schwer verändert. Seit Harry ihm ein Erbstück von Meister Regulus und zu Weihnachten schließlich die Köpfe seiner Vorfahren geschenkt hatte, war der Elf glücklich und ihm gegenüber loyal geworden.

Harry erklärte ihm, was er von ihm verlangte, und Kreacher verschwand mit einem Kopfnicken samt Rucksack.

Dann winkte Salazar Harry heran und umarmte ihn erst einmal. Dicke und glückliche Tränen flossen an ihm herab. Nachdem er sie fort gewischt hatte, Harry traute sich nicht zu fragen, wieso, ging er voran und Harry folgte ihm durch einen der Gänge. Salazar zeigte auf einen Steinvorsprung und Harry fühlte, wie das Wissen in ihn strömte. Er öffnete die verborgene Tür mit einem sanften Vorbeifahren seiner Hand und trat nach Salazar hindurch. Dahinter war ein kleines Arbeitszimmer mit einem kleinen Bücherregal.

\enquote{Es gibt noch einen anderen Zugang, von meinen Privaträumen aus, aber den zeige ich dir ein anderes Mal.}

Harry nickte, da er überwältigt war, das Arbeitszimmer eines seiner Ahnen zu sehen. Salazar zeigte auf eines der Bücher und Harry fühlte wieder, wie ihm das Wissen seines Ahnen etwas zuflüsterte. Er legte einen Schutzzauber auf das gute Dutzend Bücher, um sie vor dem Zerfall zu schützen. Alle Bücher waren in Parsel geschrieben.

Harry wurde bewusst, dass niemand wusste, dass man Parsel auch schreiben konnte, aber die wenigen, die Parsel sprechen konnten, schienen miteinander verbunden zu sein, denn Salazar hatte herausgefunden, nachdem er Hogwarts verlassen hatte, dass alle, die Parsel sprechen konnten, dieselben Symbole verwendeten und sie auch lesen konnten.

Parsel hatte Ähnlichkeiten mit dem Arabischen, was die Symbolik anbelangte, wich aber doch davon ab. Kleine Schlangen schienen die Symbole zu sein.

Harry las: \parsel{Gesammelte Rezepte und Tränke mit und über Schlangen von Salazar Slytherin}

Sein Geist übersetzte und er verstand. \accentuate{Gesammelte Rezepte und Tränke mit und über Schlangen von Salazar Slytherin}

Harry hatte eine ungefähre Ahnung, was sich in den Büchern befand, und ignorierte die gefährlichen Bücher, die sich mit den dunklen Künsten beschäftigten. Er nahm das Buch mit den Tränken und schlug es auf. Zwar hatte Kreacher Harrys Rucksack mit den Pergamenten, der Feder und dem Tintenfässchen mitgenommen, aber auf dem Tisch lag genug davon. Nur die Tinte musste reaktiviert werden. Harry wollte schon eine der Federn eintauchen, entschied sich aber dann um und duplizierte das Tintenglas, legte das Original zu den Büchern und benutzte die Tinte aus dem duplizierten Glas, um einige Tränke abzuschreiben.

\enquote{Warum hast du das gemacht, Harry?}, fragte ihn sein Ahne.

\enquote{Alte Tinte ist kostbar. Diese hier ist über tausend Jahre alt. Wenn sie weg ist, gibt es nichts mehr davon. So kann ich die Rezepte abschreiben und meine ganzen schulischen Sachen mit deiner alten Tinte schreiben.}

Salazar bekam große Augen.

\enquote{Tust du mir einen Gefallen, wenn wir hier fertig sind?}, fragte er.

\enquote{Welchen?}

\enquote{Verkleinere meine Statue und nimm sie mit. Stelle sie bei dir auf deinen Nachttisch, oder tue sie in deinen Koffer. Nur lass mich hier nicht alleine zurück}, sagte er mit gewisser Sehnsucht in der Stimme.

Harry nickte und schrieb weiter. Kreacher kam mit einem Teller voll belegter Brote und zwei selbst auffüllenden Gläsern mit Kürbissaft. Beide aßen und tranken und Salazar beteiligte sich an Harrys Kopieraktion. Dann legte er das Buch zurück und beide verließen den Raum.

Salazar begleitete ihn noch bis zu der Röhre, die er heruntergekommen war, dann konnte er sich nicht mehr bewegen. Die Statue nahm die alte Form und Farbe an, Salazars Geist wurde herausgepresst und verblasste. Harry hörte ihn noch erschöpft in seinem Geist: \stimme{Danke, aber jetzt muss ich mich erst einmal einen Tag ausruhen. Wir können frühestens übermorgen wieder reden.}

Harry verkleinerte die Marmorstatue und schob sie ein. Er hatte kein schlechtes Gewissen, da sie als seinem Erben ja eh ihm gehörte. Außerdem hatte er Salazars Erlaubnis. Dann betrachtete er die Röhre und dachte nach, als er hinter sich ein Flügelschlagen hörte.

Er drehte sich um und da war Fawkes. Harry lächelte ihn an und hob ihm seine Hand hin. Der Vogel flog auf seine Schulter, wo ihn Harry streicheln konnte. Gerne ließ sich das Tier das gefallen und wippte nach einiger Zeit mit dem Kopf. Harry verstand, Fawkes hob ab und Harry hielt sich an seinen beiden Füßen fest. Dann flog der Phönix mit ihm die Röhre hinauf. Harry zischte auf gut Glück im passenden Moment Parsel und kam durch den offenen Zugang im Mädchenklo an.

Als er wieder auf seinen eigenen Füßen stand, sah Harry Myrte, wie sie schüchtern um eine Ecke sah.

\enquote{Hi Myrte. Schön dich zu sehen. Wie geht es dir?}

\enquote{Du kommst mich besuchen?}, fragte sie.

\enquote{Eigentlich wollte ich dich vorher mit in die Kammer nehmen, weil ich was brauchte, aber du warst nicht da.}

\enquote{Oh, ich war beschäftigt. Sitzung mit den anderen Geistern. Aber schön, dass du da bist.} Sie flog auf ihn zu und drückte ihm einen kalten Kuss auf die Wange. Danach wurde sie rot. Wenn man es für rot halten konnte, wenn man von einem Geist spricht.

Harry fuhr ihr über die Wange, was ihr einen Schauer über den Rücken laufen ließ. Er musste noch eine Menge an Magie in sich haben, nachdem er gerade eben erst aus der Kammer kam und mit Fawkes herauf geflogen war. Er lächelte ihr zu und verabschiedete sich von ihr. Myrte schien glücklich über die kurze Freundschaftsbezeugung zu sein.

Dann lief er zurück in sein Zimmer, die Pergamente und die Trankzutaten waren in einem kleinen neuen Koffer durch Kreacher bereitgelegt, und stellte Salazars Statue auf seinem Nachtschrank ab. Er versah sie mit einem Sicherungszauber, damit sie kein anderer mitnehmen konnte. Jeder konnte sie zwar berühren und hochheben zum Anschauen, aber das Zimmer konnte keiner mit ihr verlassen. Selbst ein Umstellen war nicht möglich. Ein Fremder wurde gezwungen, die Statue wieder auf Harrys Nachtschrank zu stellen.

\trenn
\onelineback % Anderenfalls werden 2 Leerzeilen gesetzt

\begin{traum}
Er lief einen dunklen Gang entlang. Immer wieder musste er abbiegen, bis er in einen runden Raum gelangte. Und dort stand er, sein Angstgegner. Instinktiv zog er seinen Zauberstab, um sich verteidigen zu können. Beide waren alleine, niemand konnte ihnen helfen. \gedanke{Endlich}, dachte er. Dieses Mal wird es vorbei sein. Es gibt kein Entkommen mehr. Zauber um Zauber schleuderte er ihm entgegen, doch alle wurden abgewehrt. Dann wollte er es beenden und griff zum äußersten. Doch selbst dieser Zauber wurde abgewehrt und der grüne Blitz flog abermals auf ihn zurück.

Schweißgebadet und schreiend saß er im Bett. Nach wenigen Sekunden wurde seine Tür aufgerissen und Bellatrix stand im Rahmen. \enquote{Alles in Ordnung, mein Lord?}, fragte sie besorgt.

Er sah sie nur an.

Gerade als sie gehen wollte, hielt er sie auf. \enquote{Bleib hier, Bellatrix. Komm her.}

Ihr Herz machte einen Hüpfer, als sie die Tür hinter sich schloss. Voldemort rückte ein wenig zu Seite und sagte: \enquote{Komm her, Bellatrix. Neben mich, auf die Decke.}

Sie kam wie eine Katze auf das Bett zu und legte sich hin.

Er beugte sich über sie und besah sich ihr Gesicht. Irgendwie musste er sich beruhigen und seinen Ausbruch wieder unter Kontrolle bringen. Mit seinen langen Fingern fuhr er ihre Gesichtskonturen nach.

Bellatrix bekam durch diese Behandlung eine Gänsehaut und schloss ihre Augen.

Vollkommen Gefühlslos beruhigte er damit seine Gedanken. Dann legte er sich hin und schlief ein.

Bellatrix lag glücklich an seiner Seite und begann sich nach kurzem Wachen an ihn zu kuscheln.
\end{traum}

Harry erwachte und dachte über das nach, was er eben geträumt hatte. Er war Harry und duellierte sich mit Voldemort. Nein, er war Voldemort und duellierte sich mit Harry. Er erwachte schweißgebadet in Malfoy-Manor. Nein, er war Harry und erwachte normal mitten in der Nacht. Nein, er war Voldemort, der Bellatrix sanft streichelte und seine Gedanken beruhigte. Er war Harry, der alleine in seinem Bett lag und noch immer die Berührungen von Bellatrix’ Haut spürte. Mit eigenartigem Gefühl schlief er wieder ein. Er musste sich erst einmal klar werden, wer er war. Solche Träume waren einfach zu verwirrend.

Noch am nächsten Morgen dachte Harry darüber nach, was er vergangene Nacht erlebt hatte. Beim Frühstück erzählte er Ron und Hermine darüber.

\enquote{Harry, du musst deine Okklumentik-Stunden wieder aufnehmen.}

Vorsichtig legte er einen Finger auf seine Lippen und sprach leise: \enquote{Mache ich schon, momentan geht es etwas schleppend.}

\enquote{Du machst was?}, fragte Ron.

\enquote{Ich arbeite an Okklumentik-Stunden mit Snape. Ich hoffe, ich habe demnächst Glück. Es geht langsam bergauf.}

Hermine verengte ihre Augen, um Harry intensiv zu durchleuchten, kam aber zu dem Entschluss, dass er doch recht hatte.

Vor der vorletzten Unterrichtsstunde griff sich Harry aus seinem Raum den Rucksack und nahm ihn mit zu Zaubertränke. Heute musst er, wie seine Mitschüler auch, alleine einen Trank brauen. Professor Snape nannte ihnen die geforderte Seite und auch Harry schlug sein Buch auf.

Dieser Trank kam ihm bekannt vor. Es wurden Teile einer Schlange verwendet. Er blätterte kurz durch seine Pergamente welche er aus Salazars Büchern abgeschrieben hatte, und erkannte, dass er in Salazars Rezept nur das Gift eines Basilisken, anstelle des Giftes einer normalen Schlange verwenden musste. Da er ja Schlangengift dabei hatte, war das kein Problem und obwohl ihm Snape immer wieder über die Schulter sah und ihn versuchte zu Triezen, ließ er sich nicht irritieren. Harry tropfte die geforderte Menge des Basiliskengiftes in einem ruhigen Moment in den Trank und braute weiter. Er las immer mal wieder das Rezept durch und erkannte die Unterschiede zu Salazars Aufzeichnungen. Da er schlechte Erfahrungen mit seinen Schulbüchern gemacht hatte, hielt er sich bei diesem Rezept an das von Salazar. Es musste anders umgerührt werden und auch die Reihenfolge von zwei Zutaten war anders.

Doch Harry hatte ein gutes Gefühl und auch die Farbe seines Trankes nahm den gewünschten Farbton an. Er konnte Snape hinter sich förmlich sehen, wie er seine Stirne runzelte, als er mal wieder in seinen Kessel sah. Doch dieses Mal sagte er nichts und ließ Harry weiter brauen.

Am Ende der Stunde gab er seinen Trank ab und packte seine Sachen zusammen.

\enquote{Mister Potter}, schnarrte Snape, \enquote{50 Punkte Abzug für Gryffindor wegen eines misslungenen Zaubertrankes.} In Harry begann Wut hochzukochen, aber er hielt sich unter Kontrolle, schließlich hatte er ein Ziel. Er schluckte seinen Ärger herunter und trödelte, da er mit seinem Lehrer noch sprechen wollte. Er schnappte seine Tasche und folgte Snape in sein Büro.

\enquote{Was wollen Sie hier, Potter? Soll ich Ihnen nochmals Punkte abziehen?}

\enquote{Nein, Sir. Ich habe etwas für Sie. Ich dachte, das könnte Sie interessieren.}

Harry öffnete seinen Rucksack und nahm eine kleine Holzkiste heraus. Er stellte sie auf den Boden vor Snapes Schreibtisch und vergrößerte sie. Dann entnahm er eine Phiole und stellte sie auf den Tisch. Skeptisch besah sich Snape den Inhalt. Danach nahm Harry ein Tuch heraus und schlug es auf. Auch dieses Exemplar besah sich Snape. Als Nächstes holte Harry seine Kopie seiner Aufzeichnungen heraus und legte sie Snape auf seinen Tisch. Dieser warf einen kurzen Blick auf das oberste Rezept und wand sich dann der Phiole wieder zu, als sein Blick wieder zurück zur Rezeptur flog.

\enquote{Das haben Sie also gebraut heute}, sagte er. \enquote{Nun gut. Fünfzig Punkte für Gryffindor wegen eines hervorragenden Trankes. Die Farbe stimmte und auch die Konsistenz war in Ordnung.}

Harry traute seinen Ohren nicht. Fünfzig Punkte von Snape. Entsetzt und skeptisch sah er seinen Professor an.

\enquote{Was wollen Sie dafür?}, fragte Professor Snape und holte so Harry wieder in die Realität zurück.

\enquote{Dafür? Direkt nicht\abs Ich möchte nur die Okklumentik-Stunden wieder aufnehmen; und zwar richtig.}

\enquote{Und wenn ich ablehne? Nehmen Sie die Sachen wieder mit?}

Harry verkleinerte die Phiole und die Schlangenhaut samt Tuch und verstaute sie wieder in der Holzbox. Er legte die Pergamente oben auf und schloss die Box. Dann stellte er sie auf Professor Snapes Tisch.

\enquote{Nein. Ich überlasse sie Ihnen. Vielleicht sind sie für den Unterricht interessant. Es ist genug für alle Klassen und genug für Ihren Vorrat. \gst Ich möchte die Stunden unabhängig von dem, was ich ihnen gerade gegeben habe, erhalten.}

\enquote{Sie haben mir also gerade etwas geschenkt und erwarten keine Gegenleistung dafür?}

\enquote{Exakt.}

\enquote{Warum also sollte ich die Mühe auf mich nehmen und Ihnen Okklumentik-Stunden geben?}

\enquote{Lassen Sie mich folgendermaßen antworten: Wenn Sie ablehnen, werde ich mir das benötigte Wissen aus Büchern und mit meinen Freunden erarbeiten müssen. Es wird mir schwerer fallen und es wird mich eine Menge Arbeit und Anstrengung kosten. Aber wenn Sie nach einem Grund fragen\abs} Er stützte sich auf dem Schreibtisch auf und schaute seinem Professor in die Augen, \enquote{sehen Sie in meine Augen und beantworten Sie sich die Frage selber. Dann tarnen Sie die Unterrichtsstunden mit Nachsitzen.}

Harry nahm seine Tasche auf, wandte seinen Blick ab und verabschiedete sich mit den Worten: \enquote{Ich muss zur nächsten Stunde.}

Nachdenklich sah Snape eine Weile seinem Schüler nach und durch die offene Tür in sein Klassenzimmer. Mit einem Schlenker seines Zauberstabes schloss er seine Tür und verriegelte sie. Danach holte er ein Glas aus seinem Schreibtisch und goss sich einen Finger breit Feuerwhisky ein. Das Glas nahm er und setzte sich in einen schweren bequemen Sessel vor dem Kamin in einem Nebenzimmer. Nach fünf Minuten nahm er den ersten Schluck aus seinem Glas und schaute es nachdenklich an. Es schien grün zu schimmern. Genauso wie die Flammen einen Grünton haben. Lilys Grün.

Sichtlich erschöpft schlief er ein. In dieser Nacht träumte Severus Snape von Lily. Sie stimmte ihm zu, Harry wieder Okklumentik-Unterricht zu geben. Das Glas fiel ihm aus der Hand, erreichte aber nicht den Boden. Ein Hauself, der spät in der Nacht den Raum säuberte, bemerkte es rechtzeitig und ließ das Glas auf den Schreibtisch schweben. Er hinterließ unter dem Glas einen Zettel mit seinem Namen, denn Severus hasste es, wenn die Dinge nicht dort waren, wo er sie verlassen hatte. So wusste er, dass das Glas von einem Hauselfen gerettet wurde und nicht jemand im Zimmer war, der es ihm aus der Hand genommen hatte.

\enquote{Potter, diesen Zaubertrank üben Sie bei mir so lange, bis Sie ihn können}, ranzte Professor Snape durch das ganze Klassenzimmer. \enquote{Montagabend fangen Sie damit an. Das gibt Ihnen die Zeit, übers Wochenende noch einmal das Rezept durchzusehen und sich zu überlegen, was Sie falsch gemacht haben.} Damit war die Stunde beendet und Harry wusste, dass sich Snape dazu herabgelassen hatte, ihm die gewünschten Stunden zu geben. Denn sein Trank war heute perfekt. Er hatte als einziger die richtige Farbe, nachdem er sich an Salazars Rezept gehalten hatte. Es war ein sanfter Grünton. Da aber alle anderen einen Blaustich hatten; sie arbeiteten mit dem Schulbuch-Rezept; hatte Snape die perfekte Ausrede für das Nachsitzen.

Hermine schüttelte nur den Kopf und Draco konnte sich ein Grinsen nicht verkneifen. Harry aber nahm es stoisch hin. Die Punkte, die er wieder offiziell verloren hatte, würden ihm wahrscheinlich eh nicht offiziell abgezogen werden, da sein Trank die gleiche Punktezahl bekommen würde. Also konnte sich Snape den Abzug sparen und er zog nur wieder eine Show ab.

Er würde zwar nie mit ihm gut Freund werden, aber es schien für Harry so, als respektierten sie einander. Harry begann zumindest ihn zu respektieren.

Abends begann er endlich in dem Buch über Dementoren zu lesen. Er schlug das Buch auf und entdeckte unter einer Zeichnung eines Dementoren den Namen der Autorin. \gedanke{Adriana de Mimsy-Porpington. Ob sie mit Sir Nicolas verwandt ist? Den muss ich mal fragen.}

Harry schlug die ersten Seiten um und begann das erste Kapitel zu lesen:

\begin{buch}
\block{Vom Inferi zum Dementor}

Ursprünglich von Inferi abstammend, sind Dementoren eine der gefürchtetsten Kreaturen der magischen Welt. Bei Nachforschungen bin ich auf diese interessante Tatsache gestoßen. Da ich von Natur aus Neugierig bin, wollte ich diese Entwicklung natürlich selbst sehen. Ich besorgte mir also einen Inferi. (Nein, ich habe ihn nicht selber erzeugt.) Dieser Inferi hatte seine Aufgabe schon erledigt und war somit antriebslos. Ich nahm ihn also mit nach Hause und beobachtete ihn. Immer wieder musste man ihn feucht halten, damit er nicht austrocknete. Reden konnte er nicht, denn sonst hätte er mir sicherlich gesagt, warum er eines Tages verschwunden war. Doch kaum waren ein paar Wochen vorbei, kam er mit einer weiblichen Inferi an. Sechs Monate später kam ein kleiner Inferi auf die Welt.

Ich war wieder beruhigt. Doch eines Nachts hatte ich richtig Angst. Ich erwachte mitten in der Nacht und sah in zwei rote Augen. Das wenige Licht, das ins Zimmer schien, lies es mir kalt den Rücken herunterlaufen. Auf meiner Bettkante am Fußende saß eine Kreatur, schwarz und mit leuchtenden roten Augen, Finger und Zehen spitz zulaufend und weiße Linien über den Körper laufend. Eine Hand griff nach mir.

Als ich den ersten Schreck verdaut hatte, griff die Gestalt nach einer kleinen Wolke, die sie aus meinem Kopf heraus zog. Dann nahm sie diese in ihren Mund auf. Sie sah mich noch eine Weile an und verließ dann auf vier Pfoten mein Zimmer. Neugierig folgte ich ihr und stellte nach einer Weile fest, dass es der weibliche Inferi gewesen war; über Nacht hatte sie sich verändert. Sie hatte sich von meinem Albtraum genährt. In den folgenden Jahren erlebte ich kaum noch Albträume.

Doch wieder änderte sich etwas über Nacht. Plötzlich begannen meine Inferi; oder sollte ich sie schon seit der letzten Transformation anders nennen; über dem Boden zu schweben. Das Tageslicht begannen sie zu scheuen und sie bedeckten sich immer mehr. Zuerst nur mit Hüten oder T-Shirts, später mit dünnen Leinentüchern. Dann begannen sie meine guten Erinnerungen zu verdauen, zuerst nur zaghaft, dann aber immer aggressiver. Ich konnte das nicht mehr lange machen. Glücklicherweise hatte ich den Patronus-Zauber gut drauf, sodass ich meine Studienobjekte noch lange Zeit halten konnte.

Aber nur solange, bis ich entdeckte, dass sich in einem Teil meines Gartens, wo sie sich immer aufhielten, neue Dementoren (Ja, ich nenne sie jetzt so.) wuchsen. Wie Geister entstiegen sie dem Feld, langsam aber beständig über mehrere Stunden. Dann verlangten sie sofort nach Nahrung. Jetzt wusste ich, was ich wissen wollte. Dementoren entwickeln sich aus Inferi, wenn sie nicht nach Erfüllung ihrer Aufgabe beseitigt werden. Die neuen Nachkommen waren leider nicht mehr so pflegeleicht, wie die Eltern. Da aber scheinbar keine familiäre Bindung bestand, wurden sie weggeschickt. Meine beiden behielt ich, bis ich sie eines Tages schweren Herzens abgeben musste. Ich schickte sie nach Askaban. Da hatten sie es besser als hier. Doch bevor ich sie wegschickte, gab ich ihnen noch zu verstehen, sie mögen ihre Tücher lüften. Ich wollte sie noch einmal ohne sehen. Sie unterschieden sich kaum von dem Wesen, das zum ersten mal auf meiner Bettkante saß. Nur hatte es keine leuchtend roten Augen mehr, sondern schorfige Augenhöhlen. Die Nase war platt und hatte nur zwei Löcher. Das Loch im Mund war größer geworden und konnte nicht mehr geschlossen werden.

Dann schickte ich sie endgültig los. Das Männchen, mein ursprünglicher Inferi, legte eine seiner Hände gegen meine Stirn und übermittelte mir Bilder. Dann schwebten sie davon.\\
Erst nach einigen Wochen erkannte ich das außergewöhnliche Geschenk, das sie mir gaben. Sie gaben mir die Gabe, sich gegen sie immun zu machen. Außerdem teilten sie mir mit, wie man sie zerstören konnte. Es gab nur zwei Möglichkeiten. Ein konzentrierter Lichtzauber mit einem Zauberstab direkt in ihre Mundöffnung, oder durch gelenkte Patroni, die nicht nur als Nahrung dienten, sondern sie aktiv angriffen. Man musste es seinem Patronus aber befehlen und es musste ein gestaltlicher sein.
\end{buch}

Harry legte sein Buch weg, legte sich hin und schlug die Decke über sich. Er dachte nach. Über das was er eben erfahren hatte. Diverse Gedanken schlugen unkontrolliert in seinem Bewusstsein auf. \gedanke{Hatte Professor Elber Inferi bei sich, die jetzt zahme Dementoren sind?} Dann sah er eine glatte Wasseroberfläche mit einer Menge toter Körper darin. \gedanke{Inferi.} Dann nickte er weg und begann zu träumen. In einem Traum sah er die Kreatur auf seiner Bettkante sitzen. Die roten Augen sahen ihn an und nahmen die schlechten Gedanken von ihm.

\trenn

Es war wieder Samstag und Harry übte mit Ron und Hermine auf dem Besen den freien Fall. Sein dritter Versuch war bereits perfekt, doch er machte noch ein paar Sprünge. Er verstand sich mittlerweile blind mit seinem Besen. Er wusste, was er ihm zumuten konnte, und der Besen wusste, wie weit er gehen konnte, ohne Harry zu schaden. Ein letztes Mal für heute stieg er von seinem Besen und fiel im freien Fall. Der Besen raste ihm hinterher und fing ihn vor dem Aufprall ab. Bei seinem nächsten Spiel würde er sich bei passender Gelegenheit hinunterstürzen und die anderen schocken. So hoffte er bei einem Rennen seinen Gegner zu schlagen. Durch Überraschung.

Abends verschwand Luna mit Harry nach dem Abendessen in den Gemeinschaftsraum der Paare. In allen Sesseln und Stühlen saßen Pärchen, die sich küssten und streichelten. Einige bemerkten Lunas und Harrys Anwesenheit gar nicht. Andere wiederum grüßten sie und machten sofort weiter. Es war hier vollkommen normal und keiner musste sich schämen. Einige waren mit ihren Händen unter den Roben ihres Partners oder ihrer Partnerin. Luna und Harry entschlossen sich, sich in ihr Zimmer zurückzuziehen. Er betrachtete die Truhe, die schon immer in der Ecke des Zimmers stand. Harry fiel auf, dass er sich niemals Zeit genommen hatte sie sich anzuschauen. Er öffnete den Deckel und fand zu seiner Überraschung ein Schachbrett. Nach ein paar Runden in denen er 2:4 verloren hatte, räumte er es beiseite.

Als er das Spielbrett in die Truhe legte, hörte er nur von hinten: \enquote{Bleib so stehen Harry und drehe dich nicht um. Warte kurz, ich habe eine Überraschung für dich.}

Harry stand da, das Spielbrett noch immer in den Händen. Er ließ es los und schloss den Deckel der Truhe. Dann richtete er sich auf und stand mit dem Gesicht zur Wand. Er musste warten.

Nach einer Weile meinte Luna. \enquote{Du kannst dich jetzt umdrehen.}

Harry drehte sich um und wusste nicht, was er sagen soll. Da stand sie nun im fahlen Mondschein. Sie hatte nichts an. Sie war vollkommen nackt. Und außer ihrem Haupthaar fand er beim herunterschauen an ihr kein weiteres. Harry ließ vor Staunen seinen Mund offen stehen. So hatte er sie noch nie gesehen und das schimmernde Licht, dass sich auf ihrer zarten hellen Haut brach, ließ sie noch viel schöner erscheinen als die ganzen Male zuvor. Harry lief das Wasser in seinem Mund zusammen und er spürte, wie sein Herz anfing schneller zu schlagen. Er spürte plötzlich ein Verlangen und versuchte, seine Gedanken abzulenken, sodass Luna nicht mitbekommen würde wie er anfing sie zu begehren. Mit jedem langsamen Schritt, den sie ging, nur einen Fuß vor den anderen setzend, hörte Harry sein wallendes Blut in seinem Körper zirkulieren. Es war so, als ob er seine Umgebung nur noch schemenhaft wahrnehmen konnte. Er war nur auf Luna fixiert. Er konnte einfach seinen Blick nicht mehr von ihrem lösen. Langsam ließ er seinen Blick an ihr heruntergleiten. Unten angekommen fing er wieder an, seinen Blick nach oben zu wenden. Er schaute ihr tief in die Augen und wusste, er sollte nicht diesen begehrenden Blick haben.

\begin{abAchtzehn}
Er wusste, er sollte seine Gedanken ändern. Wusste, sie würde seine Gedanken in dieser Sekunde lesen können. Luna blieb nur wenige Zentimeter vor ihm stehen und ging mit ihrem Oberkörper nach vorne. Automatisch zog es Harry ihr entgegen. Ihre Lippen trafen auf seine und beide versanken in einem langen und wundervollen Kuss. Sie ging noch einen Schritt nach vorne und ihre Brüste berührten seinen Oberkörper. Harry durchfuhr ein Schauer wie er ihn noch nie erlebt hatte. So intensiv, so schön. Als er begann sich daran zu gewöhnen, fing er wieder Lunas Gedanken auf und verstand, dass sie ihn genauso wollte. Er fing an seinen Mund leicht zu öffnen und mit ihrer Zunge zu spielen. Sie begann sein Shirt zu öffnen und ließ es einfach auf den Boden fallen. Jetzt kam sein nackter Oberkörper voll zu Geltung. Seine Hände wanderten ihre Arme entlang hoch und er vergrub sie in ihren Haaren. Diese schimmerten in diesem Licht als wären sie aus purem Gold. Die Kerzen im Raum fingen an, langsam, kaum merklich, heller zu werden, um eine Stimmung zu schaffen, die jedem sagen würden, das sei der perfekte Ort, die perfekte Zeit. Langsam und für Harry fast unbemerkt näherte sie sich mit ihren Händen seiner Hose und begann sie zu öffnen. Sie fiel herunter und Luna ging einen Schritt zurück, damit Harry heraussteigen konnte. Er brach den Kuss und zog sie sanft zum Bett.

Sie hatten Zeit, unendlich viel Zeit, fand er. Plötzlich kam ihm der Gedanke an Verhütung.

\enquote{Mach dir keine Sorgen, Harry}, hörte er Luna sagen. \enquote{Ich war vor einiger Zeit bei Madame Pomfrey. Sie hat mir etwas gegeben. Wirkt ein paar Monate.}

Harry war erleichtert. Er fuhr mit seinen Händen ihren Konturen nach und er spürte ein leichtes Zittern. Jetzt begann er ihren Hals zu küssen. Mit seinen Händen umkreiste er ihren Rücken. Seine Hände wanderten nach vorne und er küsste sie weiter. Sie erwiderte seinen Kuss und begann ihren Mund leicht zu öffnen. Seine Hände glitten über ihre zarte Haut nach vorne, über ihre Brüste. Ihre Zunge umspielte seine Zähne, bis sie seine Zunge traf. Ihre Hände glitten weiter und begannen Harry sein letztes Stückchen Stoff auszuziehen. Als er den Kuss löste, streifte Luna ihm seine Unterhose ab und Harry saß nun auch vollständig nackt auf dem Bett. Harry legte sich mit dem Rücken mitten auf das Bett und Luna saß auf seinem Bauch. Er zog sie sanft an sich und küsste ihre Brüste. Ein wohliges Schnurren entfloh Luna. Er fing an, ihre Brüste mit seiner Zunge zu umspielen. Luna entfloh ein leichtes Keuchen und ihre Gedanken und Gefühle begannen sich mit denen Harrys zu vermischen. Sie küsste ihn und begann mit ihrer Zunge seinen Hals entlangzufahren. Sie umspielte seinen Adamsapfel und Harry begann sich aufzurichten. Sie umklammerte seinen Rücken mit ihren Händen und seine Hüfte mit ihren Beinen. Harry sah Luna in die Augen, schloss die seinen und begann sie erneut zu küssen. Langsam drückte sich Luna an ihm hoch und rieb an seinem Bauch entlang. Ein wohliger Schauer durchfuhr ihn. Dann ließ sie langsam ab und er glitt ihn sie hinein. Für ihn quälend langsam ging sie nach unten bis es nicht mehr ging. Sie hatten sich vereint. Mit geschlossenen Augen saßen sie einige Zeit nur da, sich umarmend und küssend. Die Blockaden waren gefallen und Harry spürte, als sie begann sich auf und ab zu bewegen, wie er in ihren Körper glitt. Er hatte nicht mehr das Gefühl in ihr zu stecken, sondern in sich selbst. Er spürte, wie er in Lunas Körper war; so als wäre es sein eigener. Luna empfand ebenso. Das war eine Erfahrung, die er wohl mit niemand anderem jemals machen würde, dachte Harry und wünschte sich, der Moment würde nie vergehen.

Er glitt wieder in seinen Körper und beherrschte sich mit ganzer Kraft. Gemeinsam kamen sie dem Höhepunkt näher und ließen sich den Namen des anderen rufend auf das Bett fallen. Voll von Schweiß bedeckt, küssten sich Luna und Harry und schliefen glücklich ein.

\end{abAchtzehn}

\begin{safedivide}
\fskdivider
\end{safedivide}

Währenddessen ging es bei Philip weiter.

Madame Pomfrey hatte die letzten Tage viele Stunden damit verbracht, Bücher zu lesen. Sie frischte ihre Kenntnisse in Muggelmedizin auf, die sie nach ihrer Ausbildung als Medi-Hexe erlangt hatte. Sie hatte als angelernte Hilfskraft in einem Krankenhaus viel über Muggelmedizin gelernt. Dies kam ihr jetzt zugute. Philip saß auf einem Holzstuhl mit Lehne, der mit einem Kissen auf der Sitzfläche und einer dicken Wolldecke an der Lehne belegt war. Er saß bequem. Vor ihm saß Madame Pomfrey und erklärte ihm, wie der Vorgang ablief. Professor Elber stand daneben und assistierte.

\enquote{Also, Mister Allman}, begann sie. \enquote{Die Prozedur geht folgendermaßen: Während des Ablaufes werden Sie nichts sehen, denn ich muss leider Ihr anderes Auge transparent machen, damit ich die zweite Augenhöhle nachbilden kann. Die zweite Augenhöhle wird dann jucken. Dieses Jucken wird noch ein paar Stunden anhalten. Es ist wichtig, dass Sie den Reiz ignorieren. Sie dürfen nicht kratzen. Sie bekommen ein kühlendes Tuch in ihre Augenhöhle, um den Reiz zu lindern. Darüber lege ich Ihnen eine Augenklappe, die Sie bitte den restlichen Tag tragen werden. Kommen Sie morgen gleich vor dem Frühstück zu mir. Dann werde ich ihnen die Augenklappe kurz entfernen und das Kühlpack erneuern. Nach dem Frühstück geht es dann zu einem Arzt. Ich werde Sie begleiten.}

\enquote{Kommt Professor Elber mit?}

Dieser schüttelte den Kopf.

\enquote{Sind Sie bereit?}, fragte Madame Pomfrey. Philip nickte und sie begann.  Das kleine kühlende Kügelchen und die Augenklappe lagen bereit. Sie begann, indem sie einen Transparenzzauber auf das intakte Auge legte. Dann besah sie sich die intakte Augenhöhle lange und genau. Jeden Schritt teilte sie Philip mit, damit er beruhigt war, denn er sah nichts mehr. Auf der anderen Seite bildete sie mit ihrem Zauberstab die Augenhöhle nach, so wie sie sein sollte. Nur der hintere Teil der Höhle war nicht so tief, da die Verbindung zum Sehnerv wieder hergestellt werden musste, wenn das Auge fertig war. Zuletzt legte sie das kühlende Kügelchen aus Stoff in die neue Augenhöhle. Nachdem sie den Transparenzzauber entfernt hatte, legte sie ihm die Augenklappe an und Philip sah wieder etwas. Morgen stand der Termin bei dem Squib-Arzt an. Dieser würde eine Gewebeprobe nehmen, um das Auge zu züchten.

\trenn

\enquote{Professor Snape?}, rief Harry durch das Klassenzimmer.

\enquote{Büro. Erste Tür links}, kam die Antwort.

Harry betrat das Büro und suchte die erste Tür auf der linken Seite. Sie war ganz hinten. Er betrat einen Raum mit allerlei Trankzutaten und Kesseln. Unter einigen brannte ein Feuer, andere waren mit einer Flüssigkeit gefüllt. An den Wänden standen Regale mit unzähligen, fein sortierten Trankzutaten. Flüssige und feste Zutaten standen in Gläsern oder Phiolen da. Alle waren sauber beschriftet in Snapes üblicher Handschrift. Harry stieg ein herber Duft in die Nase. Ein Geruch der gerade von einem der Kessel ausging.

\enquote{Was brauen Sie?}, fragte Harry seinen Lehrer.

\enquote{Einen Ihrer Tränke. Woher haben Sie überhaupt die Rezepte? Sie sind scheinbar sehr alt. \gst Und danke, für die Zutaten.}

Das hatte Harry jetzt nicht erwartet. Ein Lob von Snape. Zwar ein kurzes und knappes, aber dennoch ein Lob. Er dachte kurz nach und antwortete knapp: \enquote{Sie wissen, was es ist?}

Snape nickte. \enquote{Große Schlange.}

\enquote{Basilisk}, antwortete Harry, leicht amüsiert über Snapes Antwort. Denn ein Basilisk unterschied sich doch sehr von einer Schlange.

\enquote{Große Schlange, wie ich sagte. Aber woher haben Sie die Sachen?}

\enquote{Kammer des Schreckens} antwortete Harry knapp. \enquote{Ich war noch einmal dort.}

\enquote{Wie kamen Sie auf diese Idee?}

Harry schwieg. Dann sagte er: \enquote{Finden Sie es während unserer Stunden heraus. Das macht die Sache interessanter. Für Sie, weil Sie es wissen wollen, und für mich, weil ich es nicht unbedingt sagen möchte.}

Snape nickte und meinte dann: \enquote{Ich brauche noch fünf Minuten. Der Trank ist gleich fertig. Ich denke, dass Sie heute und morgen wegen des misslungenen Tranks \gst dafür übrigens die abgezogenen Punkte retour und noch zehn darauf \gst Nachsitzen werden. Beim nächsten Mal behalte ich Sie die ganze Stunde im Auge, was Sie wieder zweimal Nachsitzen kosten wird. Dann sehen wir weiter. Sie sollten ihre Braukünste dann etwas verbessert haben. Ich nehme dann den Trank der lebenden Toten. Wir werden ihr Rezept brauen und nicht das aus dem Schulbuch. Ich erwarte eine gute Leistung. Sie werden keine Punkte im Unterricht dafür bekommen, da Sie meine Unterlagen gesehen haben und sich darauf vorbereiten konnten. Das macht wieder einmal Nachsitzen. Dann haben wir schon fünf Termine, in denen ich Ihnen hoffentlich etwas beibringen konnte.}

Harry nickte und rührte den Kesselinhalt für seinen Professor dreimal um.

\enquote{Gut bemerkt}, meinte Professor Snape.

Harry nickte nur. Dann stellte er seine Tasche ab und wartete, bis Snape den Inhalt umgefüllt hatte. Dann leerte er den Kessel, löschte das Feuer und wischte sich seine Hände ab. Harry machte sich bereit für den Angriff. Snape zog seinen Zauberstab und sprach: \zauber{Legilimens!}

Harry konzentrierte sich und versuchte seinen Geist zu entleeren. Die erste halbe Minute schaffte er es ganz gut und konzentrierte sich auf einen runden Raum mit vielen Türen. Denselben Raum, den er im Ministerium gesehen hatte. Doch dann zog es ihn zu einer der Türen, sie öffnete sich und er schritt hindurch. Er stand in der Kammer des Schreckens. Der Basilisk war nicht mehr dort. Harry hatte keine Zeit mehr, sich darüber zu wundern, denn die Verbindung brach ab.

Er hatte leichte Kopfschmerzen und setzte sich auf einen Stuhl. Snape nahm ebenfalls Platz und sah ihn an.

\enquote{Wie sind Sie so weit gekommen?}, fragte Snape.

\enquote{Trotz allem, was letztes Jahr passiert ist, habe ich bedingt weiter gemacht. Zwar nicht oft, aber wenn, dann habe ich vor dem Einschlafen immer gedacht, dass ich in einem leeren schwarzen Raum in einem Bett liege und träume. Mein Traum war, dass ich in einem leeren schwarzen Raum liege, schlafe und träume. Mein Traum war\abs und so weiter.}

\enquote{Interessante Technik. Davon habe ich noch nie gehört. Es scheint aber, wenn Sie sonst nichts unternommen haben, dass Sie dann damit erfolgreich waren.}

Harry nickte. \enquote{Ich habe nach dem Ende der ersten \accentuate{Runde} noch etwas gelesen. Über Geistentleerung. Leider hat mir das nicht viel weitergeholfen, bin dann aber eher durch Zufall auf eine Beruhigungs-Technik gestoßen, falls man aufgewühlt sein sollte. Es war ein Medizinbuch. Ich habe mir nichts dabei gedacht, aber, als ich Probleme beim Einschlafen hatte, es doch ausprobiert. Am nächsten Tag merkte ich, dass ich besser geschlafen hatte. Ich hatte die ganzen Ferien über Zeit, diese Art für mich zu perfektionieren. Hier in der Schule ist sie wieder etwas in den Hintergrund gerückt, da ich hier keine Schlafprobleme habe. Ich habe erst vor zwei Wochen wieder damit angefangen. Zusätzlich habe ich mir überlegt, wie ich bei Binns Unterricht nicht einschlafe. Doch das hat nicht so gut geklappt, aber das Resultat um etwa zehn Minuten verzögert.}

Snape nickte. \enquote{Bereit?}, und fing wieder an. \zauber{Legilimens!}

Wieder schaffte es Harry, sich zu konzentrieren. Wieder stand er in dem runden Raum und er drehte sich langsam um seine eigene Achse. So, als ob er die richtige Tür suchen würde. Nach knapp einer Minute wurde er wieder durch einen Drang zu einer der Türen gezogen. Wider schritt er durch die sich öffnende Tür und stand wieder in der Kammer des Schreckens. Dieses Mal vor Salazar Slytherins Statue. Harry dachte an ein blaues Auto, um seine Gedanken abzulenken. Doch das Einzige was passierte war, dass sich Salazars Marmorstatue blau färbte. Er spürte abermals den Drang sich zu bewegen, blieb aber standhaft. Doch nach einigen Minuten ging er doch auf die Statue zu, die ihn begrüßte und sich ebenfalls zu bewegen schien und auf ihn zulief. Dann brach die Verbindung wieder ab.

\enquote{Enorm, für den ersten Versuch. Sie sind noch sehr schwach und der Dunkle Lord würde Ihre Barriere schneller durchbrechen. Ehrlich gesagt ich auch, wenn ich möchte. Aber sei’s drum. Ich möchte, dass Sie sich eine andere Technik zu Gemüt führen. Mal sehen, ob die bei Ihnen anschlägt, oder ob die normalen Wege bei Ihnen nicht zum Erfolg führen und Sie sich ihren eigenen Weg suchen müssen. Ich möchte, dass Sie vor dem Schlafengehen und auch sonst bei jeder Gelegenheit ein Lied in Gedanken singen. Das müssen Sie immer auf Abruf haben, damit der Angreifer nur das hört. Sie müssen sich darauf konzentrieren nur an dieses Lied denken zu können. Versuchen Sie es gleich heute Abend. Ich möchte sehen, ob es morgen einen Effekt hat. \gst Wer war übrigens die Staute? Sie wirkte irgendwie fehl am Platz. Und was war das für ein Raum, den Sie mir zeigten, den ich gesehen habe?} Die letzten Fragen kamen ziemlich unsicher von Professor Snape. Er wusste wohl nicht, ob er reale Bilder sah, oder von Harry inszenierte, war aber doch eher der Meinung, dass sie real waren.

\enquote{Wollen Sie es wissen, oder wollen Sie es, wie von mir vorgeschlagen, herausfinden?}, fragte Harry.

\enquote{Ich werde darüber nachdenken. Wir haben morgen ja noch einmal einen Termin. Sie sind entlassen.}

Harry nickte. \enquote{Professor? Ich habe unter Aufsicht noch einmal den Trank von heute gebraut?}, fragte er nach.

Snape nickte und Harry verließ mit seiner Tasche den Tränkeraum, durchquerte Snapes Büro und ging durch den Gemeinschaftsraum in sein Zimmer, um sich schlafen zu legen.

Er dachte nach, welches Lied er sich suchen konnte. Musste es ein leichtes sein, damit es eingängig sein würde, oder musste es ein schweres sein, damit es verwirrend sein würde? Diverse Titel gingen ihm durch den Kopf.

Dann meldete sich Salazar: \stimme{Nimm ein leichtes. Das ist eingängig und so schaffst du es auch, es im Kanon mit dir selbst zu singen.}

Harry drängte sich ein sehr altes ihm unbekanntes Lied in seine Gedanken. \gedanke{Es muss von Salazar sein}, dachte er noch, sang es in seinen Gedanken und schlief ein.

Am selben Tag als Harry bei Snape war, ging Philip kurz vor dem Frühstück zu Madame Pomfrey, um seinen Verband prüfen zu lassen. Sie nahm den kleinen Wattebausch heraus und ersetzte ihn durch einen neuen. Dann schickte sie ihn zum Frühstück und bat ihn, danach gleich wiederzukommen.

Nachdem er wieder da war, reisten sie per Kamin in den Tropfenden Kessel. Madame Pomfrey hatte eine Ausnahmegenehmigung von Dumbledore erhalten, der ihr den Kamin für diese beiden Reisen extra frei schaltete. Im Tropfenden Kessel angekommen, begrüßten sie kurz Tom und verschwanden dann durch die Tür ins Muggellondon. Sie nahmen die U-Bahn zur entsprechenden Klinik und betraten nach kurzem Fußmarsch den Eingangsbereich. In Muggelkleidung, welche sie sich vor ihrer Abreise angezogen hatten, fielen beide nicht auf. So konnten sie sich unbemerkt bewegen. Philip, für den alles neu war, beherrschte sich und sah sich nur staunend um.

Dann saßen sie in einem Behandlungszimmer und warteten auf den Arzt. Madame Pomfrey erklärte Philip die verschiedensten Muggelerfindungen, soweit sie sich daran erinnern konnte. Dann ging die Tür auf und der Arzt kam herein.

\enquote{Doktor Carrow?}, fragte Madame Pomfrey nach.

Dieser nickte und setzte sich auf seinen Untersuchungsstuhl. \enquote{Poppy Pomfrey und Philip Allman, richtig?}, fragte er nach. Beide nickten. \enquote{Gut. Das Ganze geht recht schnell, wenn Sie Magie einsetzen. Dann können Sie das Auge schon in knappen zwei Stunden mitnehmen. Ich werde die Schwester rufen und dann machen wir eine Biopsie an Ihrem intakten Auge. Das heißt: Ich werde ihnen ein Betäubungsmittel in das Auge tropfen. Dann wird eine winzige Gewebeprobe entnommen und inkubiert. Dann werde ich Ihnen das Gegenmittel tropfen. Der ganze Vorgang dauert etwa fünf bis zehn Minuten. Danach wird das Auge mithilfe von Magie in einem Brutschrank herangezüchtet. \gst Die Schwester weiß übrigens nichts davon. Die weiß nur von einer Vorsorgeuntersuchung wegen des grauen Stars. Das ist eine Erkrankung der Augen, bei der man zunehmend einen Schleier vor den Augen bekommt.}

Philip nickte, sobald er etwas verstanden hatte. Dann rief der Doktor nach der Schwester, indem er auf einen Knopf an einer der Apparaturen drückte. \enquote{Schwester Margret, kommen Sie bitte in Behandlungszimmer vier. Augenbiopsie.} Nach knapp einer Minute kam die Schwester mit einer Nierenschale, zwei Kunststofffläschchen, einem Skalpell, sowie einer Lupe mit Kopfband und einem kleinen Glasgefäß mit Korkverschluss.

Philip musste sich auf die Liege hinter ihm legen und der Arzt tropfte einen Tropfen auf das Auge. Nach etwas mehr als zwanzig Sekunden meinte er: \enquote{Jetzt nicht mehr bewegen und versuchen Sie, das Blinzeln zu unterdrücken.} Er setzte die Lupe auf und nahm das Messer. Dann schabte er vorsichtig eine hauchdünne Schicht ab und legte diese in das Glasfläschchen, welches die Schwester sofort verschloss. Der Doktor wartete noch ein paar Sekunden, dann tropfte er das Gegenmittel auf das Auge. Philip konnte wieder blinzeln und ihm wurde ein kleines Papiertaschentuch gereicht, um seine Tränenflüssigkeit abzuwischen.

\enquote{Kommen Sie in zwei Stunden wieder}, sagte der Arzt, als er der Schwester das Glasfläschchen abnahm. \enquote{Dann machen wir die andere Untersuchung, weswegen Sie hier sind.} Dann ging er. Die Schwester folgte ihm.

\enquote{Welche andere Untersuchung?}, fragte Philip nach.

\enquote{Ich denke, er hat das nur zur Schwester gesagt, damit sie sich nicht wundert, wenn wir wieder kommen. Sie kann dann diesen Termin einplanen.}

Philip nickte und sie verließen das Krankenhaus. Madame Pomfrey zog eine Karte hervor und besah sich diese. Dann führte sie Philip in ein kleines Café, in dem es auch Eis gab. Der Becher war leider viel zu schnell geleert, deshalb schlenderten die beiden die Einkaufsstraße zuerst in eine Richtung und danach in der anderen Richtung.

Pünktlich nach zwei Stunden kamen sie wieder an der Anmeldung an und wurden abermals in den vierten Stock geschickt. Die Schwester erkannte die beiden wieder; ein Junge mit Augenklappe fiel halt auf; und bat sie gleich wieder in ein Behandlungszimmer. Nach knappen zwei Minuten kam der Arzt wieder.

\enquote{Danke, Schwester. Für diese Untersuchung brauche ich Sie nicht.}

\enquote{Dann werde ich weiter Ordnung im Archiv schaffen. Falls Sie mich brauchen sollten.} Dann verschwand sie.

Der Doktor schloss die Tür und stellte das bedeckte Auge im Glasfläschchen auf der Nierenschale auf dem Tisch ab. \enquote{Soll ich Sie alleine lassen?}, fragte er Madame Pomfrey, nachdem er das Tuch entfernt hatte und das Auge durch das Fläschchen zu sehen war.

\enquote{Nein. Sie können uns aber nachher zu einem Optiker schicken. Dann können wir gleich die Augen untersuchen lassen.} Doktor Carrow nickte und wartete. Madame Pomfrey nahm das Glasfläschchen und entkorkte es. \enquote{Sie kennen die Prozedur ja schon, Mister Allman.} Dieser nickte und sie begann. Wieder erklärte sie alles, was sie tat, da Philips Auge wieder einmal transparent war und er nichts mehr sah. Jetzt passte sie die Augenhöhle exakt dem an, was sie sah, und zusätzlich noch der Eigenart des neuen Auges. Dann setzte sie es ein und sprach einen normalen Anwachszauber für Pflanzen.

\enquote{Das juckt}, beklagte sich der junge Zauberer. \enquote{Fürchterlich.} Doch er widerstand dem Drang zu kratzen. Nach mehr als einer Minute hörte es auf.

Madame Pomfrey legte die Augenklappe wieder auf das Auge und löste den Zauber auf dem anderen. \enquote{Sehen Sie mich wie bisher?}, fragte sie. Philip nickte. \enquote{Und auf der anderen Seite? Irgend eine Veränderung?}

Er nickte. \enquote{Kleine, leicht rötlich schimmernde Punkte.} Madame Pomfrey nahm die Augenklappe ab. Philip blinzelte etwas und nahm nun deutlicher seine Umgebung wahr. Ihm wurde leicht schummerig.

\enquote{Halten Sie das andere Auge zu}, sagte Doktor Carrow. Philip tat dies und es wurde besser. \enquote{Sie müssen sich daran erst noch gewöhnen. Nach so einer langen Zeit mit nur einem Auge passt sich das Gehirn an. Nehmen Sie am Anfang die Augenklappe nur beim Essen oder Abends ab. Nicht während des Unterrichts. Wenn es dann besser wird, dann lassen Sie sie länger weg, bis Sie ganz auf sie verzichten können. Sie werden es selbstständig merken. \gst Decken Sie jetzt besser wieder ihr neues Auge zu, bis Sie im Fahrstuhl sind. Mein Kollege weiß beschied, zweiter Stock.} Dann stand er auf und ging zur Tür. Als Philip die Augenklappe wieder aufhatte, öffnete der Doktor die Tür und sagte noch: \enquote{Viel Glück}, bevor er verschwand.

\enquote{Danke}, riefen ihm beide hinterher.

Philip sah seine Krankenschwester an. \enquote{Was erwartet mich jetzt noch?}

\enquote{Nur eine einfache Augenuntersuchung. Es wird deine Sehstärke überprüft.}

Dann gingen beide hinaus und in Richtung Fahrstuhl.




\begin{kommentar}
Gleich am Anfang des Kapitels sagt Harrys Mutter zu ihm, dass er eine Schwester hätte haben können, wenn sie nicht von Voldemort umgebracht worden wären. Die Idee dazu habe ich aus einer anderen Fan-Geschichte, in der Harry durch einen Unfall Jahre in die Vergangenheit zurückgeworfen wurde. Dort lebte seine Mutter und er hatte eine Schwester. Zudem hat es JKR selbst einmal gesagt.
\end{kommentar}

\begin{kommentar}
Nachdem Harry aus der Kammer ein paar Zutaten vom Basilisken geholt hat, gibt es einen kurzen Schwenk zu Voldemort, der schweißgebadet aufwacht und sich danach Bellatrix ins Bett holt. Ein kleiner Seitenhieb auf eine meiner anderen Geschichten, 'Ihr größter Wunsch'.
\end{kommentar}

\begin{kommentar}
Poppy Pomfrey begleitet Phillip Allman zu einem Muggelarzt. Doktor Carrow. Eine nette Anspielung auf die Carrow-Zwillinge aus dem siebten Band.
\end{kommentar}

\chapter{Doktoren, Planeten und Seelenteile}

Im Inneren der Kabine angekommen meinte er: \enquote{Schon toll, was diese Muggel so alles erfunden haben. Immer wieder faszinierend, wenn man so davon hört, oder etwas darüber liest.}

Madame Pomfrey nickte nur. Philip nahm die Augenklappe ab und trat, von seiner Krankenschwester gestützt, heraus. Allerdings hatte es den Anschein, dass er seine Großmutter stützen würde. Sie meldeten sich an der Anmeldung zur Augenuntersuchung an und wurden gebeten, sich kurz im Wartezimmer zu setzen. Kaum saßen sie, wurden sie auch schon aufgerufen.

\enquote{Philip? Komm bitte mit deiner Großmutter mit. Zimmer drei.}

Philip stutzte kurz, reagiert dann aber schnell, da Madame Pomfrey nicht reagierte. \enquote{Komm schon, Granny. Wir sind dran.} Er zog an ihrer Jacke und sie reagierte.

Erst sah sie ihn entgeistert an. Dann schaltete es. \gedanke{Er improvisiert.} Sie lächelte ihn an und stand auf, um hinter ihm das Wartezimmer zu verlassen und danach das Zimmer drei aufzusuchen.

Auf dem Gang kam ihnen bereits der Herr entgegen und lief direkt hinter ihnen in das Zimmer. Er schloss die Tür und zeigte Philip seinen Platz. Dieser setzte sich und konzentrierte sich auf das Bild nur eines Auges, um dem Schwindelgefühl zu entgehen.

\enquote{Warst du schon einmal bei einem Optiker?} Philip schüttelte den Kopf. \enquote{Gut, dann erkläre ich dir, was dich heute erwartet. Ich werde nur deine Sehschärfe überprüfen.
Dann lege mal deinen Kopf auf diese Kunststoffwanne und drücke dein Kinn gegen die Begrenzung hier.}

Der Herr zeigte auf eine Apparatur, die zwei Sehlöcher hatte. Philip tat, wie ihm aufgetragen wurde, und legte sein Kinn auf die Stelle. Dann stellte der Optiker die Höhe ein, damit der junge Mann hindurchsehen konnte. Das linke Auge deckte der Optiker ab und schaltete einen Projektor ein, der Buchstabenreihen verschiedener Größe untereinander an die Wand warf.

\enquote{Lies bitte die dritte Zeile}, sagte der Optiker.

\enquote{A, F, H, E, G}, sagte Philip.

\enquote{Gut}, sagte er und verdeckte das andere Auge. Dann änderte er die Buchstaben. \enquote{Und nochmals die dritte Reihe.}

\enquote{B, G, R, O, P}, sagte Philip.

\enquote{Dann schauen wir jetzt mal, ob eine leichte Korrektur notwendig wäre. Ich werde jetzt nacheinander verschiedene Linsen vor das Auge setzen. Du sagst mir dann, ob es besser oder schlechter ist, Ok?}

\enquote{Ja.}

Der Optiker legte nacheinander Linsen vor die Augen und Philip sagte seine Meinung. \enquote{Schlechter, schlechter, schlechter, schlechter.}

\enquote{Wunderbar, deine Augen sind vollkommen in Ordnung.} Er klappte die Apparatur wieder weg und sah Philip an. \enquote{Keine Probleme. Null Dioptrien. Du brauchst keine Brille. Du und deine Großmutter könnt gehen.}

Philip stand auf und bedankte sich bei dem freundlichen Herrn. Dann gingen beide zurück zum Aufzug, und Philip legte seine Augenklappe wieder an. Sie verließen das Krankenhaus und kehrten zum Tropfenden Kessel zurück. Dort aßen sie erst einmal etwas und kehrten dann nach Hogwarts zurück.

\trenn

Es war wieder einmal Donnerstagmorgen, als Professor Elber gut gelaunt das Klassenzimmer betrat. \enquote{Es wird Zeit, dass wir anfangen zu üben. Dafür brauchen wir aber einen Raum. Ihr kennt euch hier aus und könnt mir doch sicherlich einen nennen.}

Stille herrschte im Raum. Harry summte in Gedanken das Lied vor sich hin und versuchte sich gerade an einem zweistimmigen Kanon.

\enquote{Wie wäre es mit der Großen Halle?}, fragte Lavender Brown.

\enquote{Dann bekommen wir Ärger mit den anderen, die wir kurz nach dem Frühstück herauswerfen müssen}, sprach Hermine. \enquote{Aber vor allem mit den Hauselfen. Die werden dann überfordert.}

Harry überlegte. Er musste einen Raum finden, der \gst plötzlich fiel ihm etwas ein. Aber würde es auch bei den anderen Anklang finden? \enquote{Wie wäre es?}, und Harry verließ plötzlich der Mut.

\enquote{Ja Harry, wie wäre was?}, fragte Professor Elber.

Harry druckste herum und meinte schließlich: \enquote{Wie wäre es mit der Kammer des Schreckens?} Die ganze Klasse erstarrte. \enquote{Schließlich wird sie nicht mehr gebraucht}, fügte Harry hinzu.

\enquote{Hm}, Professor Elber dachte nach. \enquote{Können Sie mir die Kammer mal auf die Tafel zeichnen?}, fragte er.

Harry stand auf und ging zur Tafel. Er malte eine grobe Skizze der Kammer auf und setzte sich danach wieder.

Professor Elber schaute sich die Zeichnung an, drehte sich um und meinte: \enquote{\accentuate{Kammer des Schreckens} \gst klingt irgendwie verboten und gefährlich. Wie lang ist die Kammer eigentlich? Und wie groß ist der Durchmesser des Hauptkreises?} Harry nannte Professor Elber die Daten, der sie auf die Tafel schreib, und erzählte ihm die Geschichte mit der Kammer und dem Basilisken, ließ aber Ginny außen vor, da noch immer keiner wusste, dass sie die Kammer geöffnet hatte.

\enquote{Hm. Das scheint ein interessanter Ort zu sein. Wo liegt die Kammer eigentlich?}

\enquote{Unter dem Schloss}, sagte Harry.

\enquote{Scheint mir ein guter Platz dafür zu sein}, meinte Professor Elber. \enquote{Und wie kommen wir dahin?}

Harry hatte diesen Punkt nicht bedacht. Er war sich nicht sicher, was er sagen sollte. Professor Elber lehnte sich wieder an sein Pult und schaute Harry an. Harry konnte schon Malfoys Blicke und gehässigen Sticheleien spüren, wenn er der ganzen Klasse sagte, wo der Zugang lag.

\enquote{Na ja}, begann Harry, \enquote{er ist im dritten Stock \gst im Mädchenklo.} Keine Reaktion kam aus der Klasse. Es war mucksmäuschen still. Harry wurde zunehmend unwohl.

\enquote{Woher}, begann Professor Elber.

Doch Hermine unterbrach ihn und meinte: \enquote{von mir.}

Professor Elber schaute sie an und zeigte ein leichtes Schmunzeln.

\enquote{Nun gut. Harry, Sie und Hermine kommen am Samstag nach dem Frühstück hierher und dann gehen wir gemeinsam dorthin und sondieren die Lage.} Er schaute wieder zur ganzen Klasse und fuhr fort. \enquote{Ich werde diese Kurse außerhalb der normalen Schulzeit halten. Alle Fünft-, Sechst- und Siebtklässler die dazu Lust haben können sich ab nächsten Montag in eine Liste vor der Großen Halle eintragen.} Er löste sich von seinem Pult und ging Richtung Tafel. Er schrieb die Worte \accentuate{Abwehr und Angriff} an die Tafel. \enquote{Was wisst ihr bereits aus dem Buch?} Er drehte sich wieder um und schaute durch die Reihen.

Hermine streckte wie immer die Hand in die Höhe, doch Professor Elber winkte ab. \enquote{Ich weiß Hermine, dass Sie das schon gelesen haben. Sie können ihre Hand herunternehmen.} Hermine senkte ihre Hand und lief leicht rosa an. \enquote{Sonst noch jemand?} Plötzlich streckte Draco Malfoy die Hand in die Höhe. \enquote{Ja Draco?}

\enquote{Krausets kann man sehr oft, aber nicht immer, an ihrem krausen Haupthaar erkennen. Ihnen kann man nicht ansehen, ob sie einen in die Irre führen wollen, oder einem beistehen. Oftmals ist es sogar so, dass sie plötzlich ihre Meinung ändern und ins genaue Gegenteil umschlagen. Viele von ihnen haben auch eine gespaltene Persönlichkeit, bei denen ein Teil einem Helfen, der andere aber einem Schaden will.}

\enquote{Sehr gut! Und wie kann man sich gegen sie zur Wehr setzen?}

\enquote{Mit \spruch{Anamorph, anatus, kressare}. Es bringt sie dazu, ehrlich zu bleiben. Auch wenn sie dadurch einem nicht helfen, falls sie gerade einen in die Irre führen wollen. Sie bleiben dann bei dieser Einstellung.}

\enquote{Wunderbar. Fünfzehn Punkte für Slytherin.}

Harry drehte sich nur um und schaute Draco leicht verärgert an. Dieser grinste Harry nur an.

\trenn

Nach dem Essen gingen Harry und Ron ihre Hausaufgabensachen holen, machten sich auf den Weg zu einem kleinen Innenhof innerhalb Hogwarts und setzten sich unter einen Baum. Ron nahm den Baum auf der anderen Seite des Weges, sodass sich beide anschauen konnten. Es war angenehm, obwohl es nicht besonders warm war. Die Sonne schien und erwärmte den Boden. Obwohl es Winter war, war es im Innenhof nicht besonders kalt. Irgendein Zauber verhinderte, dass sich das Klima mit dem außerhalb anglich. Professor Sprout hatte einige Pflanzen im Hof, die ständig gleiche Klimabedingungen benötigen. Harry hatte seine Hausaufgaben aufgeschlagen, als Luna sich ihm näherte und sich neben ihn setzte. Mittlerweile wusste die ganze Schule, dass die beiden zusammen waren. Seine Korrekturen für \fach{Geschichte von Hogwarts} hatte Harry bereits fertiggestellt und rechts neben sich gelegt. Harrys Blick wanderte über den Hof, als er Professor Dumbledore und Professor McGonagall nebeneinander herlaufen sah. Sie kamen an Harry vorbei und er fing ein paar Gesprächsfetzen auf.

\enquote{Ach Albus, ich glaube, ich hole mir bei Poppy ein Abführmittel. Ich kann schon seit Tagen nicht mehr}, sagte Professor McGonagall.

Doch Professor Dumbledore entgegnete ihr: \enquote{Nicht nötig, Minerva. Ich hab was Besseres}, und grinste.

Harry blieb der Atem stocken. Er schaute zu Ron und fing leicht zu grinsen an. Ron erging es genauso. Professor Dumbledore blieb stehen und griff in seine Tasche. Er holte zwei durchsichtige Plastiktütchen heraus und zeigt sie Professor McGonagall.

\enquote{Hier, Minerva. Die grauen sind bestens gegen Verstopfung geeignet. Oder vielleicht die schwarzen, die haben mich irgendwie mehr amüsiert}, sprach Professor Dumbledore. Als Professor McGonagall zu dem Tütchen mit den schwarzen griff, meinte Dumbledore nur: \enquote{Äh, ich glaube, Sie nehmen besser eines von den grauen. Die haben weniger pepp.} Sie hob eine Augenbraue und griff dann nach einem der grauen. \enquote{Und vergessen Sie nicht, sie erst kurz vor der Tür einzunehmen. Die wirken sehr schnell.}

Mit hochgezogenen Augenbrauen und einem ausdruckslosen Gesichtsausdruck ging Professor McGonagall, das Toffee fest verschlossen in der Hand, durch den Innenhof und verschwand in einem der Seitengänge. Professor Dumbledore drehte sich um und lächelte Harry an.

\enquote{Darf ich mich zu dir setzen?}, fragte Professor Dumbledore höflich. Harry nickte und Dumbledore nahm neben ihm auf dem Boden Platz und lehnte sich mit dem Rücken an den Baum. Er sah sich ein bisschen um und fragte Harry, nachdem er seinen Hausaufgabentitel gelesen hatte: \enquote{Darf ich mir die mal durchschauen?}

Harry nickte und Professor Dumbledore nahm sich die Rollen Pergament, die auf dem Boden lagen. Harry kümmerte sich inzwischen um seine anderen Hausaufgaben, als Professor Dumbledore ihn fragte: \enquote{Korrekturen für das Buch \buchtitel{Geschichte von Hogwarts}? Ich wusste gar nicht, dass das Professor Binns unterrichtet.}

\enquote{Das hatten wir auch nicht von Professor Binns auf}, antwortete Harry, \enquote{das war Professor Elber, er hatte Professor Binns vertreten. Und als er nicht wusste, was Professor Binns gerade unterrichtet, hat er uns gefragt was wir wissen wollen.}

\enquote{Und dann solltet ihr Fehler in der Geschichte Hogwarts finden?}, fragte Dumbledore.

\enquote{Nicht direkt}, antwortete Harry. \enquote{Er erzählte uns vielmehr etwas darüber, als wir ihn fragten, wie die Zauberei entstanden ist.}

Dumbledore drehte sich um und sah Harry in die Augen. \enquote{Ich glaube, ich sollte mich mal mit Professor Elber unterhalten. Mir scheint, er weiß mehr als er uns sagt. Aber was ist der Zweck dieser Hausaufgabe?}

Harry antwortete ihm. \enquote{Er hat uns aufgegeben, einen Korrekturzusatz zu schreiben, nachdem er uns einige Fehler im Buch aufgezeigt hatte, und die Besten gehen an den Autor für die nächste Ausgabe. Wir müssen aufgrund dieser Fehler die Einträge und Absätze im Buch korrigieren und dann abgeben.}

Dumbledore legte die Pergamentrollen wieder auf den Boden und drehte sich wieder zurück. \enquote{Ja, ich glaube, ich sollte mich mal mit ihm unterhalten.} Professor Dumbledore legte nun seine Hände auf seinen Bauch und entschloss sich, ein Nickerchen zu machen.

Etwas später hatte auch Harry seine Hausaufgaben beendet. Er räumte sein Tintenfass, seine Rollen an Pergament, die fertigen Hausaufgaben und seine Schreibunterlage in seine Tasche und verschloss sie. Luna legte ihren Kopf auf Harrys Schulter, lehnte sich an Harry und schloss die Augen. Harry fuhr ihr etwas durchs Haar, legte dann seinen Kopf gegen ihren und spürte ihr Haar an seiner Backe. Er schloss die Augen und schlief ein.

Harry fühlte sich eigenartig. Die Welt um ihn drehte sich und verlor sich in einem grauen Schleier. Nach einiger Zeit sah er wie sich seine Umgebung immer mehr herauskristallisierte und entdeckte Luna.

Sie drehte sich um und begrüßte Harry. \enquote{Hallo Harry.}

\enquote{Luna, was machst du hier?}

\enquote{Das hier ist mein Haus. Und da steht meine Mutter.}

Harry sah sich um und stand in der Küche. Die Decke und die Wände waren rußig und auf dem Küchentisch in der Mitte waren tiefe Brandflecken zu sehen.  Der Raum war kreisrund. Aus dem Tisch züngelten kleine Flammen. Harry vermutete, dass es magische Flammen sein mussten. Auf einem Stövchen darüber stand ein kleiner Kessel. Am Ende des Tisches war ein Kinderhocker mit einem kleinen Mädchen. Harry nahm an, dass es sich um Lunas Schwester handeln musste, da sie die gleiche Haarfarbe wie sie hatte. Um den Kinderhocker war eine Art Schildzauber gelegt worden. Wohl um zu verhindern, dass sich die kleine verletzte. Jetzt kam eine gut aussehende Frau um die Ecke und trug einige Kräuter auf dem Arm und legte sie auf den Tisch. Sie trug einen langen dunkelblauen Zauberumhang, der über ihre rot schimmernde Robe hing. Über dem Umhang lag ihr langes, schneeweißes Haar. Harry dachte an Luna. Sie hatte die Haare ihrer Mutter geerbt. Er trat etwas näher an Luna heran und nahm ihre Hand. Lunas Mutter sah auf, schwang ihren Zauberstab und holte damit ein paar weitere Kräuter aus einem Regal mit vielen Schubfächern heran. Ein Fach flog heran und blieb in der Luft schweben. Sie nahm wenige Zweige heraus und das Fach machte sich auf den Rückflug. Harry konnte ihr Gesicht sehen. \gedanke{Sie hat die Augen ihrer Mutter}, dachte Harry. Plötzlich brodelte und blubberte es. Harry fühlte wie Luna sich weiter an ihn schmiegte und seine Hand fester drückte. Die Funken im Kessel wurden immer stärker und Lunas Mutter griff wieder zu ihrem Zauberstab, mit einem leichten Ausdruck von Panik im Gesicht. Plötzlich explodierte der Kessel und warf Lunas Mutter zurück. Luna warf sich ihm nun ganz um den Hals und er spürte ihre Tränen auf seiner Haut. Die Welt fing wieder an sich zu drehen und Harry nahm Lunas Gesicht in seine Hände und schaute sie an.

\enquote{War das der Tag, an dem sie gestorben ist?}, fragte er.

\enquote{Ja}, sagte Luna mit Tränen im Gesicht.

\enquote{Dann träume ich gar nicht?}, fragte Harry.

\enquote{Nein}, sagte Luna, die sich die Tränen aus dem Gesicht wischte. \enquote{Früher hatte ich öfters diesen Traum. Ich sah, wie sie wieder eines ihrer Experimente machte. Aber seit wir zum ersten Mal nebeneinander gelegen und unsere Nacht verbracht hatten, waren sie seltener geworden. Und so habe ich das ganze auch noch nicht gesehen.}

Harry hielt sie wieder fest in seinen Armen, als die Umgebung um sie herum wieder klarer zu werden begann. Er schluckte, als er sah, wo sie sich befanden.

Luna löste sich von ihm und entdeckte ein Schlafzimmer mit einer hübschen jungen Frau in der Mitte, sie hatte lange, rötliche Haare und, wie sie feststellte, Harrys Augen. \enquote{Deine Mutter?}, fragte Luna.

\enquote{Ja}, entgegnete ihr Harry,  als ihm eine Träne die Wange herunterlief.

Harrys Mutter hielt ihren kleinen Sohn fest in ihren Armen, als sich Geräusche von draußen in Harrys Ohren wiederfanden. Harrys Mutter erschrak und lief hinter das Bett, um ihn abzusetzen. Von der Tür her kamen spratzelnde Geräusche. Es hörte sich so an, als ob sich zwei Zauberer duellierten. Es gab einen Aufschrei und durch den Türschlitz drang ein grüner Lichtschein. Jetzt war es stiller. Harry hörte nur die Fußtritte auf knarrendem Holz. Er musste schlucken. Er hörte eine Stimme und die Tür sprang auf. Eine Gestalt trat in das Zimmer. Sie war mit einem schwarzen Umhang verdeckt und hatte eine Kapuze auf. Harry konnte sein Gesicht nicht sehen, war sich aber sicher, er wusste, wer es war.

Seine Mutter schrie: \enquote{Nicht Harry, nicht Harry, bitte nicht Harry!}

\enquote{Geh zur Seite, du dummes Mädchen\abs geh weg jetzt\abs}

\enquote{Nicht Harry, bitte nicht, nimm mich, töte mich an seiner Stelle \gst}

Betäubender, wirbelnder, weißer Nebel füllte Harrys Kopf\abs was tat er da? Warum flog er? Er musste ihr helfen\abs sie würde sterben\abs sie wurde umgebracht\abs

Er fiel, fiel durch den eisigen Nebel.

\enquote{Nicht Harry! Bitte\abs verschone ihn\abs verschone ihn\abs}

Eine schrille Stimme lachte, die Frau schrie, und Harry schwanden die Sinne.

Er hörte nur noch: \enquote{Avada Kedavra} und ein grüner Blitz schoss aus der Spitze des Zauberstabes und auf seine Mutter zu. Sekunden später lag sie Tod auf dem Boden und Harry fiel vor lauter Trauer und innerlichem Schmerz auf die Knie und fing an zu schreien. Tränen quollen aus seinen Augen heraus. Luna nahm ihn in ihre Arme und streichelte seinen Kopf. Harry beruhigte sich wieder und sah, wie die vermummte Gestalt um das Bett herumlief und auf Klein-Harry zuging.

Harry stand auf, die Tränen noch immer in seinem Gesicht, um sich das ganze anzuschauen. Er wollte verstehen, was damals passiert war, wie er als einziger überleben konnte. Er sah dem Zauberer zu, von dem er sicher war, dass es sich um Voldemort handelte. Es war die Nacht, in der seine Eltern ums Leben gekommen waren. Die Nacht als Voldemort seine Macht verlor und für vierzehn Jahre verschwand. Der Zauberer hob wieder seinen Zauberstab und sprach wieder: \enquote{Avada Kedavra}. Jetzt sah Harry wie sich alles in Zeitlupe abspielte. Die Spitze von Voldemorts Zauberstab leuchtete auf und ein grüner Lichtstrahl schoss heraus. Er bewegte sich langsam auf Harry zu. Gerade als er wenige Zentimeter von ihm entfernt war, traf er auf eine Art Schild, der den größten Teil zurückwarf und Voldemort im Gesicht traf. Ein geringer Teil jedoch ging hindurch. Genau an der Stelle, wo Harry jetzt seine Narbe hatte. Er hörte, wie Voldemort aufschrie und sich die Hände vor das Gesicht hielt. Er begann sich aufzulösen bis nur noch eine kleine Wolke übrig blieb, die sich schnell durch die Tür aus dem Zimmer heraus bewegte und dabei ein leises Schreien und Weinen hinterließ. Ein winziger Teil bewegte sich unbemerkt auf Harry zu und verschwand in dessen Narbe.

Die Welt drehte sich wieder und Harry wachte auf. Er stieß einen Schrei aus. Schweiß lief von seinem Gesicht. Luna nahm ihn in ihre Arme und auch Hermine eilte zu ihm. Ron blickte von seinen Hausaufgaben hoch und Dumbledore erschrak.

\enquote{Was, wie, wo}, entfuhr es Dumbledore, der anscheinend gerade träumte, als Harry ihn aus seinem Schlaf riss. Er drehte sich um und sah Harry. \enquote{Was ist los, Harry? Hattest du wieder einen Traum?}, fragte ihn Dumbledore.

\enquote{Ja. Wir haben \gst ich habe \gst hatte einen Alptraum.} Harry schnaufte. \enquote{Ich war dabei, Professor, dabei, und sah wie Voldemort mich umbringen wollte. Als kleines Kind. Es war so, als wäre ich in einer Erinnerung gewesen. Wie vor zwei Jahren, als ich in ihrem Denkarium die Verurteilung von Barty Crouch Jr. sah.}

Dumbledore sah ihn an. \enquote{Und? Viel dir etwas auf?}

Harry überlegte kurz und sagte dann: \enquote{Ja. Ich sah, wie der Fluch auf mich zukam. Doch wenige Zentimeter bevor er aufschlug, wurde er durch einen seltsamen Schild abgehalten und zurückgeworfen. Nur ein bisschen ging durch.} Harry zeigte auf seine Narbe. \enquote{Voldemort löste sich auf und nur eine kleine Wolke blieb zurück. Sie floh durch die Tür hinaus. Dann bin ich aufgewacht.}

Dumbledore sah Harry an und rieb sich sein Kinn.

\enquote{Was hat das zu bedeuten, Professor?}, fragte Harry.

\enquote{Ich weiß es nicht, Harry. Es ist uns allen immer noch nicht genau klar, wie dich deine Mutter schützen konnte. Warum genau der Fluch dich nur gezeichnet, aber nicht getötet hatte. Die Liebe deiner Mutter hat dich wahrscheinlich geschützt. Ihr Opfer, um dich zu retten. Das ist alte Magie, dessen genaue Wirkung nicht mehr bekannt ist. Zu viel ging verloren.} Er schnaufte und sprach weiter. \enquote{Ich weiß, das ist jetzt nicht einfach, aber könntest du versuchen, falls du noch einmal diesen Traum hast, herauszufinden, wer Voldemorts Zauberstab mitgenommen hat, oder wohin er verschwand?}

\enquote{Aber Professor}, sagte Harry, \enquote{Ich dachte, man kann nur die unmittelbare Umgebung sehen.}

\enquote{Harry, so viele Sachen zwischen dir und Voldemort liegen für uns noch im Nebel. Wer weiß, ob die Verbindung zu Voldemort dir nicht erlaubt, ihm zu folgen und zu sehen was er sieht.}

Harry nickte, immer noch schwer schnaufend.

\enquote{Was ist denn hier los?}, fragte Professor McGonagall, die sich ihnen näherte.

Harry sah auf und bemerkte erst jetzt, dass wohl alle, die noch im Innenhof waren, um ihn herum standen.

\enquote{Harry hatte einen Alptraum, Minerva}, sagte Dumbledore. \enquote{Aber sagen Sie mal, haben Sie ihre Geschäfte nun erledigt?}

Professor McGonagall schnürte ihre Lippen zusammen und hatte einen undefinierten Gesichtsausdruck. \enquote{Hmpf. Nie wieder esse ich eines von Ihren verrückten Toffees.}

\enquote{Oh, das sind nicht meine verrückten Toffees. Die stammen von den Weas\-ley-Zwillingen.}

\enquote{Was?}

\enquote{Ja, sie boten mir eines an, als ich im Vorbeigehen meine \gst ja, Problemchen vor mich hin murmelte. Aber glauben Sie mir, Minerva. Wenn Sie das schon schlimm fanden, dann sollten Sie keines von den schwarzen nehmen. Die haben nämlich einen lustigen Nebeneffekt.} Dumbledore grinste, als Professor McGonagall mit einem weiter undefinierten Gesichtsausdruck den Innenhof verließ. Dumbledore drehte sich wieder zu Harry und zwinkerte ihm zu. Er griff in seine Hosentasche und schaute auf seiner Uhr nach.

\enquote{Wird langsam Zeit fürs Abendessen}, meinte Dumbledore und stand auf.

\gedanke{Was für ein eigenartiger Traum}, dachte Harry. \gedanke{Oder war es doch mehr?} Harry sah zu Luna, die in Richtung Ginny sah, die vor ihm stand. Er blickte zu ihr und bemerkte einen eigenartigen Gesichtsausdruck, der ihm auf eine seltsame Art und Weise zu sagen schien, dass da mehr war als nur die Sorge um einen guten Freund und Schulkameraden.

Er hörte Luna wieder in seinen Gedanken. \gedanke{Ich versuche mal herauszufinden, wie sie fühlt. Lass mich nur machen.} Luna stand auf und meinte zu Ron, Ginny und Hermine: \enquote{Kommt, lasst uns essen gehen.}

Ron packte seine Sachen zusammen und stand auf. Zu fünft gingen sie Richtung Große Halle. Ron und Hermine vor ihnen und Harry lief neben Luna und Ginny. Sie berührten sich nicht. Als Ginny in die Große Halle hinein lief, ging Luna ihr hinterher. Harry entschied sich, sich ihnen gegenüber hinzusetzen, und lief daher auf die andere Seite der Tafel. Nachdem das Essen auf den langen Tafeln der Häuser erschienen war, begannen alle zu essen. Harry sah auf und entdeckte, dass einige der Ravenclaws die Köpfe zusammensteckten und zu seinem Tisch herüberschauten. Einige drehten sich kurz um, nur um danach die Köpfe wieder zueinander zu stecken. \gedanke{Vermutlich unterhalten die sich wegen Luna.} Denn Luna, eine Ravenclaw, saß am Tisch der Gryffindors. Harry sah zum Tisch der Lehrer und bemerkte wie Dumbledore zu ihnen heruntersah. Er bemerkte Luna und sah dann zurück zu Harry. Er zwinkerte ihm zu und Harry begann zu schmunzeln. Er war sich nicht sicher, was Dumbledore wusste oder vermutete, aber er wusste, dass er ihn nicht fragen würde.

Ron und Harry waren bald mit dem Essen fertig und Ron zog sich in die Bibliothek zurück. Ginny und Luna unterhielten sich scheinbar gut und so ließ ihnen Ron einen kurzen Zettel zurück, auf dem stand, wohin er gegangen war. Harry wollte mit, wurde ab er durch den fast kopflosen Nick abgelenkt.

\enquote{Sir Nicolas, kann ich Sie mal was fragen?}

Der Geist drehte sich um. \enquote{Oh, Harry. Gerne doch, um was geht es?}

\enquote{Ich frage mich, ob sie eine Adriana de Mimsy Porpington kennen?}

\enquote{So hieß meine Schwester, aber warum fragen Sie?}

\enquote{Ich habe ein Buch von einer Adriana de Mimsy Porpington gelesen. Und da Sie, Sir Nicolas de Mimsy Porpington, diesen nicht sehr häufigen Namen ebenfalls tragen, habe ich mich gefragt, ob es jemand aus Ihrer Verwandtschaft ist.}

\enquote{Ja, das kann durchaus sein. Meine Schwester hat viele Bücher über Tiere geschrieben. Welches war es denn?}

\enquote{Aufzucht und Pflege von Dementoren}, antwortete Harry.

Nick staunte. \enquote{Nein, so ein Buch kenne ich nicht. Ich habe mich eigentlich immer mit ihr über ihre Bücher unterhalten. Aber ein derartiges Buch kenne ich nicht. \gst Die letzten Jahre haben wir uns selten gesehen und als ich sie mal besucht hatte, hatte sie zwar ein eigenartiges Haustier, aber als Dementor würde ich es nicht bezeichnen.}

Harry zog die Augenbrauen zusammen. Er setzte seine Tasche ab, holte Pergament, eine Unterlage und einen Stift hervor und begann zu zeichnen. Als er fertig war, zeigte er es Sir Nicolas. \enquote{Sah das Tier so aus, Sir Nicolas?}

Nick sah sich das Bild lange an und versuchte sich zu erinnern. \enquote{Ja, ich glaube, es hatte eine große Ähnlichkeit mit dem Tier, das sie damals besaß. Aber fragen Sie sie doch persönlich. Irgendwo im Schloss hängt ein Bild von ihr. Lassen Sie mich überlegen. \gst Ah ja, in der Nähe des Direktoren-Büros.}

\enquote{Das ist Ihre Schwester?}, fragte er. Er hatte sie schon ein paar mal gesehen. Sie hing genau gegenüber eines Torbogens, der die Aufzüge in Hogwarts markierte. \enquote{Sie haben mir sehr geholfen, Sir de Mimsy Porpington}, bedankte sich Harry artig und ging mit der Gewissheit, Nick in Zukunft jede Frage stellen zu können. Denn so wollte er am liebsten genannt werden.

\enquote{Sie können mich in Zukunft Sir Nicolas nennen}, rief ihm der Geist hinterher.

Harry drehte sich noch einmal um und winkte zur Bestätigung, bevor er nun in die Bibliothek ging.

\trenn

Harry sah, wie Madame Pomfrey neben einem künstlichen Körper kniete, den Hintern auf ihren Füßen. Heute würden sie Reanimationstechniken lernen.

\enquote{Kommen Sie ruhig näher und sehen Sie genau zu. Wir nehmen heute Reanimationstechniken durch. Sowohl nach Muggelart, als auch auf magische Art und Weise. \gst Sprechen Sie die Person erst einmal an. Sollte sie bewusstlos sein, fühlen Sie den Puls der Person. Entweder an den Handgelenken}, sie zeigte ihnen die genaue Stelle, \enquote{oder an den Fußgelenken}, abermals zeigte sie ihren Schülern die genaue Stelle, \enquote{oder am Hals. Entweder links oder rechts, aber niemals an beiden Stellen gleichzeitig.}

Sie nahm den Kopf der Puppe und überstreckte ihn. Danach tat sie genau, was sie den Schülern beschrieb. \enquote{Überstrecken Sie den Kopf des Bewusstlosen und öffnen Sie den Mund, um ihn nach Erbrochenem oder anderen Gegenständen zu untersuchen, die er eventuell in seinem Mund hat. Sollte etwas in seinem Mundraum sein, so drücken Sie Daumen und Zeigefinger in die Backen um zu verhindern, dass er seinen Kiefer zuklappt, während Sie in seinem Mundraum einen oder mehrere Finger haben. Wenn Sie sichergestellt haben, dass der Mundraum frei ist, schließen sie den Mund und Pusten Sie ihm über die Nase Luft ein.}

\enquote{Dies machen Sie zehnmal. Danach gehen Sie her und suchen einen Punkt zwei Finger breit unter dem letzten Rippenbogen und führen eine Herz-Druckmassage durch. Aber nur, wenn der Patient keinen Puls hat. Haben Sie keine Angst, wenn Sie ihm dabei ein paar Rippen brechen. Die heilen wieder. Sie wiederholen diese Technik ständig; also Atemspende und Herz-Druckmassage, solange bis ein Muggelarzt kommt, sollten sie sich in der Muggelwelt bewegen und einige Leute um sie herum stehen. Verteilen Sie Aufgaben an die Leute. Sprechen Sie sie direkt an und sagen Sie ihnen, was sie tun sollen: Einen Arzt rufen, Schatten spenden, und so weiter.}

Die Klasse führte die Anweisungen in kleinen Gruppen nun aus. Jeder musste einmal das ganze Programm durchziehen. Dann fuhr Madame Pomfrey fort.

\enquote{Und jetzt machen wir das Ganze auf magische Art und Weise. Schauen Sie mir wieder zu. Wir prüfen auf dieselbe Art, ob der Patient bei Bewusstsein ist.} Sie schüttelte ihn leicht an seinen Schultern und zwickte ihn danach in den Arm. \enquote{Dann fühlen Sie seinen Puls und schauen sich den Mundbereich an. Jetzt machen Sie die Herz-Druckmassage mit dem Zauberstab.}

\enquote{Richten Sie Ihren Zauberstab auf das Herz des Bewusstlosen und sagen Sie deutlich: \spruch{cor-pressus aliptes}. Beachten Sie aber, dass je nach Konzentration die Stärke des Druckes auf das Herz stärker oder schwächer ist. Sie brauchen etwas Übung, bis Sie die richtige Stärke finden. Zusätzlich müssen Sie einen zweiten Zauber ausführen. Sie müssen sich auf den Ersten zusätzlich Konzentrieren, dass er nicht abbricht. Im Verlaufe des weiteren Kurses werden wir mehr als nur einen Zauber gleichzeitig ausführen. Der zweite Zauber ist der für die automatische Luftzufuhr. Sagen Sie wieder deutlich: \spruch{Aer pulmo mutare}}

Nachdem sie wiederum alles vorgeführt hatte, durften die anderen ran.

\trenn

Am nächsten Samstag, kurz nachdem Professor Elber sein Frühstück beendet hatte, verließ er die Große Halle.

Harry und Hermine bemerkten nichts davon, als plötzlich Ron fragte: \enquote{Wo ist denn Professor Elber abgeblieben?}

Harry erschrak und schaute sofort zum Lehrertisch, genauso Hermine. Ohne zu Ende zu essen, stand er auf und lief kauend nach draußen.

Hermine konnte ihm nur noch ein \enquote{Warte} hinterherrufen, doch knapp außerhalb der Großen Halle blieb er plötzlich stehen. Hermine, die aufholte, konnte gerade noch bremsen, bevor sie auf ihn aufschlug. \enquote{Was ist, warum bleibst du plötzlich stehen?}, fragte sie.

\enquote{Professor Elber! Warten Sie schon lange?}

\enquote{Nicht der Rede Wert}, antwortete Professor Elber. \enquote{Bereit?}

\enquote{Ja}, antworteten Harry und Hermine fast gleichzeitig.

Professor Elber machte eine Geste, um Harry und Hermine den Vortritt zu lassen. Sie gingen die Treppen hoch, die sich ständig veränderten, und gelangten irgendwann in den dritten Stock. Ein paar zehn Meter vor dem Eingang zum Mädchenklo fing Hermine an zu Professor Elber zu sprechen.

\enquote{Äh. Professor? Da gibt es etwas, was Sie wissen sollten.}

\enquote{Ja?}

\enquote{Auf dem Mädchenklo gibt es einen Geist. Die Maulende Myrte.}

\enquote{Und das heißt jetzt genau was?}, fragte Professor Elber.

\enquote{Sie ist ein wenig schwierig}, antwortete Hermine.

An der Tür angekommen, öffnete Hermine sie und trat zuerst ein. Dann folgte ihr Harry und zuletzt Professor Elber.

\enquote{Uhhhh, Harry. Schön, dich mal wieder zu sehen} kam es Harry von Myrte entgegen.

Professor Elber zeigte sich wenig interessiert an Myrte und ging um die Waschbeckenanordnung herum. Myrte schwebte von oben auf Harry herab und kam kurz vor ihm in der Luft schwebend zum Stillstand.

\enquote{Hast du mich vermisst?}, fragte sie.

\enquote{Myrte}, sagte Hermine leicht verärgert. \enquote{Wir wollen in die Kammer.}

\enquote{Uhhh, kann ich mit? Sag ja, Harry}, bettelte Myrte.

\enquote{Das ist eine gute Idee}, kam es von Professor Elber. Harry und Hermine drehten sich um und sahen Professor Elber fragend und verdutzt an. \enquote{Ich finde, es ist von Vorteil}, sagte Professor Elber und kam auf Myrte zu. Diese sah ihn nur verwirrt an. \enquote{Einen Geist zum Üben zu haben ist geradezu ideal. Ein Ziel, das nicht so leicht zu treffen ist.}

\enquote{Uaaaah}, entfuhr es Myrte. \enquote{Kommt, lass uns auf Myrte zielen und sie versuchen zu treffen. Was für ein Spaß.}

\enquote{Geister können sich prima wehren und haben eine größere Resistenz gegenüber Zaubern und Flüchen, wenn sie es richtig machen}, antwortete Professor Elber. \enquote{Sie sind schnell und können einen Zauberer ganz schön ins Schwitzen bringen. Und außerdem haben sie sicherlich auch jede Menge Spaß daran! Und etwas Abwechslung.} Professor Elber sah Myrte interessiert an und fuhr dann fort. \enquote{Ich kann Ihnen ein paar Tricks zeigen, wie Sie es den Sterblichen so richtig zeigen können. Die werden Mühe haben, Ihren Attacken und Angriffen auszuweichen.}

Myrtes Augen fingen an ein Leuchten zu zeigen, obwohl sie durchlässig war. \enquote{Das hört sich interessant an.} Sie drehte sich zu Harry. \enquote{Ich glaube, wir werden eine Menge Spaß haben.}

\enquote{So. Wie kommen wir jetzt in die Kammer?}, fragte Professor Elber. Harry bewegte sich auf die Waschbecken zu und ging auf den Wasserhahn mit den eingegossenen Schlagen an der Seite zu. Dann fing er an Parsel zu sprechen und der Eingang zur Kammer öffnete sich.

\enquote{Jetzt müssen wir springen}, sagte Harry und hüpfte hinunter. Hermine folgte ihm etwas unsicher kurz darauf.

Professor Elber sah Myrte nur leicht verwundert an und fragte sie: \enquote{Wollen Sie mitkommen?}

\enquote{Nein, nein}, kam es ihm entgegen. \enquote{Ich warte noch ein wenig und muss es erst den anderen Geistern erzählen.}

Professor Elber nickte und hüpfte den beiden hinterher. Unten angekommen führte Harry Hermine und Professor Elber den Gang entlang, an den heruntergefallenen Steinen und an dem leeren Platz, an dem die Haut des Basilisken lag, vorbei, zur großen Türe mit den Schlangen. Wieder sprach er etwas auf Parsel, worauf sich die Tür zu öffnen begann.

\enquote{Bleibt die Tür denn offen, oder geht die immer wieder zu?}, fragte Professor Elber.

\enquote{Ich weiß es nicht}, antwortete Harry und ging in die Kammer hinein. \gedanke{Beim ersten Mal habe ich nicht aufgepasst und vor ein paar Tagen habe ich sie geschlossen}, dachte Harry.

Hermine und Professor Elber folgten ihm. Sie gingen den langen Gang entlang und kamen in einem großen Raum an, worauf Professor Elber sich umschaute. Er strich mit seinen Händen die Steinmauer entlang und sah sich sehr genau um. Als er alle Gänge einmal abgelaufen war, ging er wieder auf Harry und Hermine zu. Dann sprach er: \enquote{Nett hier. Ein prima Übungsraum.} Er zog seinen Zauberstab und richtete ihn auf Harry. \enquote{Entwaffnen Sie mich mal}, sagte er zu Harry.

\enquote{Was?}, fragte Harry nach. \enquote{Na entwaffnen. Sie nehmen ihren Zauberstab, richten ihn auf mich und sagen dann: \spruch{Expelliarmus}.} Harry stand da, holte seinen Zauberstab heraus und sah mit einem leicht mulmigen Gefühl Hermine an. \enquote{Na los. Oder haben sie Angst mich anzugreifen?}

Harry drehte sich um und ohne ein Wort zu sagen, sprach er: \zauber{Expelliarmus}, worauf sein eigener Zauberstab aus seiner Hand Richtung Professor Elber flog, welcher ihn geistesgegenwärtig fing.

\enquote{Ah, ja}, kam es von Professor Elber. \enquote{Hab ich vermutet.}

Er gab Harry seinen Zauberstab wieder zurück und ging an der Wand der großen runden Halle entlang. Er suchte eine kleine Spalte oder Ritze im Mauerwerk und steckte, nachdem er sie gefunden hatte, seinen Zauberstab mit der Rückseite voraus hinein. Dann stellte er sich an seine Seite und fuhr dann mit seiner Hand an seinem Schaft entlang und ging ein paar Schritte zurück, worauf eine Explosion aus der Wand kam und ein großes Loch in sie hinein riss. Der Zauberstab flog herunter und blieb auf den Schutthaufen am Boden liegen.

\enquote{Wie mir scheint, kann man hier noch keine Zauberstäbe verwenden.}

\enquote{Ja aber}, fragte Harry, \enquote{wie sollen wir dann hier\abs} doch Harry kam nicht weiter, denn Professor Elber hatte bereits seine Hand über seinen Zauberstab ausgestreckt und rief: \enquote{Auf.} Der Zauberstab erhob sich und war Sekundenbruchteile später in seiner Hand, welche er schloss.

\enquote{Wie haben Sie das denn gemacht? Ich dachte hier kann man keine Zauber verwenden?}, fragte Hermine, denn Harry stand immer noch da, als wolle er seinen Satz beenden.

\enquote{Ich habe lediglich gesagt, dass hier noch keine Zauberstäbe verwendet werden können. Ich habe nichts von zauberstabfreier Magie gesagt.}

\enquote{Zauberstabfreie Magie?}, entfuhr es aus Harry. \enquote{Was ist das denn?}

\enquote{Das, was sie soeben gesehen haben. Manche sagen auch stabfreie Magie dazu.} Professor Elber öffnete seine Hand mit der Handinnenfläche nach oben und ließ seinen Zauberstab in ein paar Zentimeter Höhe schweben. Plötzlich begann er sich leicht zu drehen. \enquote{Das ist zauberstabfreie Magie. Wenn sie wollen, können wir das auch an den Samstagen lernen, an denen wir hier üben.}

Beide nickten und stimmten zu. \enquote{Aber wo üben wir dann mit Zauberstab?}, fragte Hermine.

\enquote{Auch hier}, antwortete Professor Elber. \enquote{Aber zuvor müssen wir schauen, dass wir diese Schutz\-zau\-ber loswerden. Ihr könnt euch ja mal was überlegen. Ich habe zwar schon eine Idee, wie wir sie weg\-be\-kom\-men, aber alleine schaffe ich das nicht. Und außerdem, drei Gehirne denken mehr als eines. Vielleicht fällt euch etwas ein, was mir entgangen ist.}

Harry wunderte sich und fragte sich, was Professor Elber sonst noch so drauf habe. \enquote{Jetzt räumen wir erst einmal diesen Schutt hier weg und dann schaue ich mal in der Bibliothek vorbei. Ihr könnt jetzt gehen, wenn ihr wollt. Ich habe vorerst genug gesehen. Und wenn das Problem mit dem stabfreien Zauber hier gelöst ist, können wir anfangen zu trainieren.}

Professor Elber sah auf den Schutthaufen hinunter und bewegte seine Hände über ihm. Er zog eine kleine Schleife und zog seine Hände dann nach oben, worauf der Haufen Schutt anfing sich ebenfalls nach oben zu bewegen. Mit einer eleganten Bewegung beförderte er die Steine und die kleinen Brösel in das Loch an der Wand. Es schien so, als ob die Steine und der Staub seinen Bewegungen folgten und die exakt selbe Stelle in der Mauer einnahmen, die sie vorher innehatten. Als Professor Elber fertig war, sah man keine Spur mehr von einem Ausbruch in der Wand.

Er lief voraus und Hermine und Harry folgten ihm. Sie verließen die Kammer.

Gegen Abend ging Harry durchs Schloss. Salazar hatte ihm gesagt, wo seine Privatgemächer wären. Bald war Sperrstunde, aber Harry hatte noch genug Zeit. Notfalls würde er in Salazars altem Bett schlafen. Je näher er den Kerkern und somit seinem Ziel kam, desto bekannter kam ihm der Weg vor. Er war im selben Gang, wie in seinem zweiten Jahr mit Ron, als sie sich in Crabbe und Goyle verwandelt hatten, um Draco auszuspionieren. Er ging gerade am Zugang zum Gemeinschaftsraum der Slytherin vorbei und blieb nach einigen Metern vor einem Bild von Salazar Slytherin stehen.

Er schien zu schlafen, doch machte er seine Augen auf, sobald Harry vor ihm stehen blieb. Misslaunig sah er ihn an.

\enquote{Wer sind Sie denn? Sie sind der Erste, der sich für mich interessiert.}

\enquote{Ich bin's, Harry}, sagte Harry gut gelaunt.

\enquote{Wie kommen Sie dazu, mich derart vertraut anzureden? Wir kennen uns doch gar nicht.}

\enquote{Aber wir haben uns doch schon öfter unterhalten.}

\enquote{Wir? Unterhalten? Sie haben wohl einen Schlickschlupf zum Frühstück gegessen!?}

Jemand trat im Schatten an Harry und das Bild heran, hielt sich jedoch im Schatten und beobachtete nur.

\gedanke{Schlickschlupf?}, fragte sich Harry. \enquote{Schlickschlupf? Was ist denn ein Schlickschlupf?}

\enquote{Sie kennen keine Schlickschlupfe? Das sind kleine fliegende Wesen. Unsichtbar und sie schwirren einem durch den Kopf, um einen ganz wuschig zu machen. Nicht im erotischen Sinne. Sie sorgen für Irrsinn und übertriebene Heiterkeit, sowie zu viel an Humor. Je nach Charakter des Befallenen.}

\enquote{Nein, ich habe keine Schickschlupfe.}

Dann stutzte Harry. Er senkte leicht seinen Kopf, danach seinen Blick und dachte nach. Dann sah er Salazar Slytherin wieder an. Dieser blickte ihn fragend an.

\enquote{Wie kann es sein, dass ich spüre, wenn mich jemand beobachtet}, fragte er Slytherin.

\enquote{Sie haben die notwendige Magie in sich}, antwortete ihm das Bild.

\enquote{Kann das jeder?}

\enquote{Ja. Jedes Objekt, jedes Lebewesen ist miteinander durch die Magie verbunden. Wenn jemand einem näher kommt, dann gibt es Interferenzen. Diese Interferenzen kann man spüren, wenn man seinen Zugang zur ihr gefunden hat.}

\enquote{Apropos Zugang, ich bin hier, weil ich\abs}

\enquote{Längst im Gemeinschaftsraum sein sollten, Mister Potter. Sie haben noch fünf Minuten Zeit um ihn zu erreichen, dann gibt es zehn Punkte Abzug}, sagte Severus Snape, der aus dem Dunklen an Harry herantrat.

\enquote{Professor Snape?}, sagte Harry erstaunt. Er wusste, dass er es nicht schaffen würde. Er grübelte nach. Ihm war so, als ob die Schulregeln in diesem Punkt nicht sehr genau waren. \enquote{Wie meinen sie das, Zeit ihn zu erreichen?}

\enquote{Zum Gemeinschaftsraum der Gryffindors geht es da lang} und er zeigte in die Richtung aus dem Kerkern heraus.

\enquote{Fünf Minuten? Das schaffe ich nicht. Dann kann ich ja noch etwas mit Sal\aabs Mr. Slytherin plaudern. Punktabzug bekomme ich ja wohl in jedem Fall.}

\enquote{Treiben Sie es nicht zu bunt, Mister Potter. Sonst bekommen Sie noch mehr Punkte abgezogen.}

Harry hatte eine Idee, wie er sich die fünf Minuten raus schinden konnte. Er wusste, dass neben ihm ein Gemeinschaftsraum war. Und er hatte die Möglichkeit da reinzukommen. Kurz vorher würde er die Tür öffnen und sich knapp hinter den Eingang stellen. Dann dürfte ihm Snape keine Punkte mehr abziehen.

Er drehte sich zu Mr. Slytherin und fragte ihn: \enquote{Mr. Slytherin, in dem Rezept über die \accentuate{verbesserten Sinne}, wie kamen Sie auf die Idee, eine Basilisken-Haut zu nehmen, anstelle der einer Schlange, zumal in allen Rezepten vor der falschen Schlangenhaut gewarnt wird?}

Damit hatte er das Interesse der beiden Männer gewonnen.

\enquote{Daher haben Sie also die Rezepte}, meinte Professor Snape.

\enquote{Wie kommen Sie an meine Rezepte?}, fragte Slytherin skeptisch nach.

Harry antwortete nur: \enquote{Ich habe meine Quellen.} Dann drehte er sich um und ging vor den Eingang zum Gemeinschaftsraum der Slytherins. Er hatte noch wenige Sekunden Zeit. Er flüsterte das Passwort: \parsel{Öffne dich.} Die Tür öffnete sich und Harry trat über die Schwelle. Dann drehte er sich um und sah auf seine Uhr.

Dann passierte es \geraeusch{Dong. Dong. Dong. Dong. Dong. Dong. Dong. Dong. Dong. Dong.} Die Uhr schlug zehnmal. Alle Schüler mussten in den Gemeinschaftsräumen sein.

\enquote{Was tun Sie da, Potter?}, fragte Snape gereizt.

\enquote{Ich bin in einem Gemeinschaftsraum und das rechtzeitig. Laut unserer Schulordnung muss ich nur in einem Gemeinschaftsraum sein, wenn es zehn Uhr schlägt. Und das bin ich.}

Snape kniff die Augen zusammen. \enquote{Kommen Sie raus, ich bringe Sie in Ihren Gemeinschaftsraum.}

Harry hob einen Fuß und wollte ihn schon aufsetzen. Doch er zog ihn zurück und sagte: \enquote{Wenn ich von Ihnen aufgefordert werde, dann bekomme ich keine\abs} Er trat heraus und die Tür schloss sich. \enquote{\aabs Strafe.}

Snape verkniff sich einen bissigen Kommentar und ging neben Harry her, um ihn zum Gryffindorturm zu bringen.

\enquote{Was mich interessiert, woher haben Sie die Rezepte? Dass sie von Slytherin sind, weiß ich jetzt. Aber woher haben Sie sie?}

\enquote{Die Rezepte habe ich aus einem Buch von Slytherin. Ich fand es sehr interessant.}

\enquote{Das würde ich gerne mal sehen.}

\enquote{Das würde Ihnen nichts nützen. Die Rezepte können Sie eh nicht lesen. Sie sind alle in Parsel geschrieben.} Sie liefen weiter nebeneinander her und erreichten die letzten Stufen zum Zugang zum Gryffindorturm. \enquote{Ich habe Sie extra abgeschrieben. Sonst hätte es keinen Sinn gehabt. Parsel ist nicht leicht lesbar. Ich habe es mal Hermine gezeigt. Nach einer Woche meinte sie, jemand hätte mich damit hereingelegt. Wissen Sie, ich habe ihr erzählt, dass ich es von einem Händler hätte. Es würde zu einem wertvollen Gegenstand führen. Sie fand es nicht heraus. Also habe ich mich schamhaft und reuig gezeigt. Ich will nicht, dass mich jeder schon wieder für etwas hält, was ich nicht bin. Sonst wäre es wieder wie im zweiten Jahr, als jeder geglaubt hat, ich sei Slytherins Erbe.} Im Stillen gab er sich damit recht. Er war Slytherins Erbe. \enquote{Und als mich jeder schief angesehen hatte, als ich die Schlange von Justin abhielt\abs}

\enquote{Das haben Sie also gemacht.} Snape blieb stehen.

Sie brauchten noch ein paar Meter, um zum Bild der fetten Dame zu kommen. So störten Sie sie nicht.

\enquote{Ja. Ich habe der Schlange gesagt, dass sie auf mich hören soll. Sie soll Justin nicht angreifen. Sie soll den Jungen nicht angreifen. Sie gehöre mir und soll auf mich hören. Und als die Schlange von ihm abließ, haben Sie sie vernichtet. Mann, war ich damals sauer. Nicht nur auf Sie. Sondern auf all die anderen, die mich isolierten, als hätte ich eine ansteckende Krankheit. Ich wollte das Gefühl nicht noch einmal erleben. Allerdings hatte ich im vierten Jahr mit demselben Problem zu Kämpfen. Wieder glaubte mir keiner.}

\enquote{Der Fall mit dem Feuerkelch?}

\enquote{Ja.}

Dann schwiegen sie eine Weile.

\enquote{Und das Zweite; wie konnten Sie die Tür zum Gemeinschaftsraum der Slytherin öffnen?}

Harry lächelte. \enquote{\inner{Öffne dich.} Mehr habe ich dem Zugang nicht gesagt. Allerdings auf Parsel. Es ist Sal\aabs Slytherins Haus. Logisch, dass er einen Zugang für sich geschaffen hat, der nicht an ein ständig wechselndes Passwort gebunden ist. Ich versuchte einfach mein Glück, da ich es sonst nicht geschafft hätte, in einem Gemeinschaftsraum zu sein., denn der der Ravenclaws ist noch etwas weiter weg und den der Hufflepuff kenne ich noch nicht einmal.} \gedanke{Aber das wird demnächst noch anstehen, wenn ich durch den speziellen Zugang über die Röhren zur Kammer dorthin gehe und auf der Karte nachsehe. \gst Die Karte. Ich habe nie auf der Karte nachgesehen, wo die Gemeinschaftsräume liegen.}

Snape nickte und trat einen Schritt zurück. \enquote{Gehen Sie nun schlafen. Morgen haben Sie die nächsten Stunden bei mir.}

Harry nickte und teilte der fetten Dame das Passwort mit. Leicht schläfrig schwang sie beiseite und ließ ihn ein.

Innen wurde er bereits erwartet und gleich von Ron gelöchert. \enquote{Harry, wo kommst du denn her? Was, wenn dich ein Lehrer erwischt hätte. Snape zum Beispiel.}

\enquote{Och, wir hatten eine nette Unterhaltung auf dem Weg hierher.}

\enquote{Was? Snape hat dich erwischt? Wie viel Punkte hat er dir abgezogen?}, mischte sich Hermine ein.

\enquote{Keine. Ich habe ihn dieses Mal\abs sagen wir\abs überrascht, so hat er mir nichts abgezogen.}

\enquote{Nichts abgezogen?}, fragte Hermine nach?

\enquote{Nein, er wollte, da ich nicht rechtzeitig in einem Gemeinschaftsraum sein würde.} Er setzte sich neben Ginny und berührte ihre Hand, als er sie neben sich legte und erzählte weiter. Er bemerkte es nicht einmal. \enquote{Ich war in den Kerkern. Dort, wo Ron und ich im zweiten Jahr waren. Beim Gemeinschaftsraum der Slytherins. Ich habe ein Bild von Salazar Slytherin gefunden. Nicht sehr gesprächig. Dort hat mich Snape getroffen.}

\enquote{Erwischt hat er dich}, warf Ron ein.

\enquote{Er hat gemeint, er müsse mir Punkte abziehen, wenn ich nicht rechtzeitig im Gemeinschaftsraum bin. Ich habe ihn dann hingehalten und bin rechtzeitig im Gemeinschaftsraum der Slytherins gestanden. Dann hat die Turmuhr zehn Uhr geschlagen und ich war laut Schulregeln nicht außerhalb der Gemeinschaftsräume. Es steht nirgends, dass man in seinem eigenen sein muss.} Harry grinste.

Ron und Hermine sahen ihn mit großen Augen an.

\enquote{Wie hast du es geschafft, dort hineinzukommen?}, fragte Hermine.

\parsel{Öffne dich.}

\enquote{Wie bitte? Sprich Deutsch}, sagte Ron.

\enquote{\inner{Öffne dich.} Ich habe der Tür einfach gesagt, dass sie sich öffnen soll.}

\enquote{Wie bist du darauf gekommen?}, fragte Ginny.

Harry drehte sich zu ihr. Er bemerkte seine Hand an ihrer, nahm sie aber nicht weg. Sie lag weiter neben ihrer. \enquote{Ich habe spekuliert. Slytherin war ein Parselmund. Es erschien mir logisch, dass er sich einen Zugang schaffte, der nicht an das aktuell gültige Passwort gebunden ist.} Dann gähnte er.

Er legte seinen Kopf gegen die Lehne des Sofas und dann seinen kleinen Finger über den Ginnys. Gedanken versunken spielte er mit seinem kleinen Finger an ihrem. Er strich an der Innenseite ihres Fingers auf und ab. Dabei schaute er an die Stelle, die vor wenigen Wochen noch den Zugang zum Gästebereich darstellte.

Ginny schien es nichts auszumachen, dass Harry mit ihrem Finger spielte. Obwohl er nicht wusste, wie sie reagieren würde, oder ob sie einen Freund hatte, versuchte er sein Glück. Den andern blieb diese Spielerei verborgen.

\trenn

\enquote{Heute beschäftigen wir uns mit Helene}, sprach Firenze, als seine Hufe auf dem bewaldeten Klassenzimmerboden kaum Geräusche hinterließen.

Parvati war ganz begeistert, den stattlichen Zentauren zu sehen, fragte sich aber, warum er eine Figur der griechischen Geschichte in Wahrsagen dran brachte. \gedanke{Vielleicht war sie jemand, die wahrsagte}, dachte sie sich. Beeindruckt sah sie zu ihm auf.

Heute saß die Klasse auf einer Waldlichtung; die Sterne über ihnen. Firenze zog mit ausgestrecktem Arm einen Halbkreis vor sich und die umstehenden Bäume begannen zu schrumpfen. Dann griff er in den Himmel und zog mit leicht geschlossener Faust seinen Arm herunter. Die Punkte am Himmel veränderten sich und kamen näher. Am Horizont konnte man eine rote Kugel erkennen, die gerade unterging. Dann kam eine näher und wurde zunehmend größer. Die Kugel hatte viele schmale Ringe, die in einer Ebene um ihren Zentralplaneten herum kreisten. Dann stand der Planet groß am Himmel und einer seiner Monde war deutlich sichtbar.

\enquote{Helene}, fuhr Firenze fort, \enquote{ist einer der vielen Saturnmonde. Neben Epimetheus, Pandora, Ymir, Suttungr, Skathi, Tarqeq und vielen anderen, insgesamt 62 Monde, von denen die Muggel gerade einmal 18 entdeckt haben, werden wir nur die eben genannten dieses Jahr durchnehmen. Wir werden uns ansehen und anschaulich machen, wie die verschiedenen Monde unseres Planetensystems, nicht nur die des Saturn, sondern auch anderer Planeten unsere wahrsagerischen Fähigkeiten beeinflussen können.}

Er lief durch die Gruppe.

% Ausschnitt aus Wikipedia: http://de.wikipedia.org/wiki/Helene_(Mond)  (18.08.2012)
\enquote{Helene umkreist Saturn in einem mittleren Abstand von ziemlich genau 377.420~km in 65 Stunden und 41 Minuten. Die Bahn weist eine Exzentrizität von 0,0071 auf und ist 0,21° gegenüber der Äquatorebene des Saturn geneigt. \gst Sie ist einer von zwei kleinen Monden auf der Bahn des großen Monds Dione. Helene läuft Dione in einem Winkelabstand von 60° im führenden Lagrangepunkt L4 voraus. Im folgenden Lagrangepunkt L5 läuft der Mond Polydeuces Dione im Winkelabstand von 60° hinterher. \gst Bevor sie ihren offiziellen Namen erhielt, wurde Helene von den Muggeln üblicherweise als \accentuate{Dione B} bezeichnet.}
%Bevor sie ihren offiziellen Namen erhielt, wurde Helene üblicherweise als „Dione B“ bezeichnet.}

Die Klasse starrte Firenze an, als sei der ein Astronom, der von seinem Lieblingsthema \accentuate{Saturn} sprach und nicht über Wahrsagen. Für einen kurzen Moment schienen sie vergessen zu haben, dass sie einem Zentauren zu Hufen saßen. Er knickte mit den Vorderhufen ein und saß kurz darauf ganz auf dem Boden auf, die Klasse im Halbkreis vor sich.

\enquote{Um uns die Genauigkeit unserer Vorhersagen ins Bewusstsein zu rufen, müssen wir uns die Bahnen der einzelnen Planeten, Monde und Kometen in unserem Sonnensystem bewusst sein, da wir sie in unsere Berechnungen einfließen lassen müssen. \gst Früher war das eine zeitraubende Angelegenheit, bis die Muggel die Bahnen der Planeten anfingen zu berechnen und Tabellen erstellt hatten, die für sehr lange Zeiträume die Positionen aufzeigten. Seit dieser Zeit ist es einfacher, die Konstellationen abzulesen. Da die Muggel aber noch nicht alle Monde unserer Planeten entdeckt haben, mussten wir die fehlenden Tabellen und Werte selbst berechnen. Wir wurden aber mit entsprechenden Formeln beliefert und errechneten uns die Tabellen selber. Den Rest haben wir von eingeweihten Muggeln und Squibs erhalten.}

\enquote{Professor?}, fragte Parvati nachdenklich nach.

\enquote{Ja, Miss Patil}, sagte Firenze und drehte sich zu ihr.

\enquote{Was ist der Lagrangepunkt?}
%http://de.wikipedia.org/wiki/Lagrangepunkt (18.08.2012)
%Die Lagrange-Punkte oder Librations-Punkte sind die nach Joseph-Louis Lagrange benannten Gleichgewichtspunkte des eingeschränkten Dreikörperproblems der Himmelsmechanik. An diesen Punkten im Weltraum heben die Gravitationskräfte zweier benachbarter Himmelskörper und die Zentrifugalkraft der Bewegung einander auf, sodass jeder der drei Körper kräftefrei ist und bezüglich der anderen beiden Körper immer denselben Ort einnimmt.

\enquote{Am Lagrangepunkt im Weltraum heben sich die Gravitationskräfte zweier benachbarter Himmelskörper und die Zentrifugalkraft der Bewegung gegenseitig auf.}

Ratlose Gesichter beherrschten die Szene.

\enquote{Stellen Sie sich unsere Sonne und die Erde vor. Die Erde kreist um die Sonne und beide ziehen einander an. Legen Sie jetzt einen Körper zwischen die beiden, dann wird er mit der Zeit zu einem der beiden Körper hingezogen. Es gibt aber einen bestimmten Punkt zwischen den beiden, an dem die Anziehungskraft der beiden Körper gleich groß wirkt. Also wird ein Körper, der dort liegt, von beiden gleichermaßen angezogen und bleibt auf diesem Abstand praktisch stehen, in stetig gleicher Entfernung.}

Langsam sickerte die Erkenntnis von den Ohren in die Gehirne seiner Schüler. Er wartete geduldig ab, bis es jeder begriffen hatte. Firenze konnte sehr gut an ihren Reaktionen lesen, ob dies der Fall war.

Wieder einmal bewegte er seine Hand und vor seinen Schülern erschienen Blätter mit Tabellen, die zu aktuellen Daten und Tageszeiten Werte enthielten, die die Wahrscheinlichkeit angaben, mit denen eine korrekte Vorhersage zutraf.

Harrys Augen wuselten über das Papier und er sah keinen Wert über 90\%.

\gedanke{Wie Firenze sagte, Wahrsagen ist keine exakte magische Wissenschaft}, dachte sich Harry.

Für den Rest der Stunde mussten sie sich hinlegen und ihrem Geist freien Lauf lassen. Dies lief Harrys Ok"-klu"-men"-tik-Übungen entgegen. Firenze musste dies bewusst sein, da er ihn kaum dran nahm und ausfragte. Nur wenn es um konkrete Benotung ging. Harry war es recht und so konnte er vor sich hin dösen und seine Gedanken sortieren. Doch heute hatte er eine zweifelhafte Vision, über die er sich die nächsten Tage den Kopf zerbrechen würde.

Als ihn jemand anstupste, sah er nur zwei Hufe über sich und Firenze der ihn anblickte. \enquote{Die Stunde ist zu Ende, Mister Potter.}

Harry rappelte sich auf und sagte: \enquote{Danke, Professor Firenze. Passen Sie die nächsten Tage besonders auf sich auf.} Dann etwas leiser und nur zu ihm gewandt. \enquote{Ich hatte einen Traum \gst eine Vision \gst jemand will Ihnen Schaden. Ich konnte ihn \gst sie, nicht richtig sehen. Aber anders als bei einem Traum habe ich dabei ein sehr ungutes Gefühl gehabt. Es war doch eine Vision.} Den letzten Satz sagte er mehr zu sich selbst, um sich zu beruhigen. Dann wandte er sich ab.

\enquote{Professor?}, fragte Lavender, die nun auf ihn zukam.

\enquote{Ja}, antwortete Firenze.

\enquote{Wie haben Sie das mit den Bäumen und dem Planeten gemacht?}

\enquote{So}, antwortete er und fuhr erneut mit seiner Hand vor sich her. Die Bäume wuchsen wieder.

\enquote{Das meinte ich nicht}, antwortete sie leicht unsicher. \enquote{Ich meine, mir ist nicht bekannt, dass Zentauren große magische Fähigkeiten haben.}

\enquote{Ah, belesen die junge Dame ist}, neckte Firenze sie. \enquote{Der Raum wurde auf meine Bedürfnisse zugeschnitten. Er reagiert auf meine Gedanken und Gesten}, antwortete er. \enquote{Er hat alles und kann alles, was ich brauche.}

Sie nickte verstehend und verabschiedete sich, während Harry um die Ecke bog, als er durch die Tür das Klassenzimmer verließ.

Auch an diesem Abend sortierte Harry seine Gedanken und dachte nur an das Lied von Salazar, während des Essens und zwischen den Stunden, sowie kurz vor dem Schlafen gehen. Morgen musste er wieder an seinem Trank \accentuate{üben}, was Okklumentik-Stunden bei Snape hieß, da die Tränke während des Unterrichts besser wurden, wenn er sich nicht ablenken ließ.

\begin{traum}
Er schlief ein und Träume durchzogen seinen Geist. Er stand auf einem Felsen vor einer Steilküste. Das Wasser brandete an die Klippe und wiegte sich im Takt des Windes. Harry sah noch Richtung Meer und drehte sich dann um. Er sah nun gegen die Klippen und entdeckte eine Höhle. Das Wasser war plötzlich ganz ruhig. Harry schien über das Wasser auf die Höhle zu zu schweben, obwohl er seine Füße bewegte. Nachdem er in der Höhle war, ging es langsamer voran. Der Vorwärtsdrang verschwand und Harry hatte das Gefühl, normal laufen zu müssen. Nach knappen zwanzig Metern erreichte er das trockene Ufer. Sein Blick ging durch die Höhle und blieb an einer Wand hängen. Er trat auf die Wand zu und sie verschwand. Dann ging er nach Innen. Er sah einen See, dessen Oberfläche spiegelglatt war und sich nicht zu bewegen schien. Es war dunkel in der Höhle. Trotzdem konnte Harry alles erkennen. Er sah eine kleine Insel mitten im See. Sein Blick schien die Insel heranzuholen. Er sah ein kleines Podest. Als er sich umsah, stand er bereits auf der Insel. Hinter sich fand er ein Boot, mit dem er auf die Insel gekommen war. Das war zumindest seine Meinung. Er ging an die Säule aus Bergkristall heran und sah hinein. Dort war eine Vertiefung mit einer Flüssigkeit. Unter der Flüssigkeit im Stein war ein Medaillon. Er griff hinein und zog es heraus. Dann versuchte er es zu öffnen. Er fand einen Zettel darin. Die Worte konnte er nicht genau lesen, aber er erkannte, dass sie eine Fälschung sein musste.

\enquote{Eines meiner Verstecke}, hörte er plötzlich eine Stimme neben sich.

Er kannte die Stimme und es überraschte ihn nicht im geringste. Es sah neben sich und die Gestalt sah auf den Kristall. Dann sah er Harry an.

\enquote{Warum?}

\enquote{Um nicht zu sterben}, antwortete er.

Harry nickte. \enquote{Gibt es noch weitere?}

Voldemort nickte.

\enquote{Wie viele?}, fragte Harry.
\end{traum}

Er drehte sich um und schlief weiter. Jetzt träumte er von Ginny.




\begin{kommentar}
Später macht Harry in einem Innenhof unter einem Baum seine Hausaufgaben. Er hat einen Alptraum und erwacht. Kurz darauf kommt McGonagall und beschwert sich bei Dumbledore über seine Toffees. Es sind Abführpralinen. Die anderen sind welche, bei denen man lachen muss, während sich der Darm entleert. Kann sein, dass es nicht so genau rüberkommt in der Geschichte.
\end{kommentar}

\begin{kommentar}
Einmal sagt Firenze: »Ja, belesen die junge Dame ist.« Eine kleine Hommage an Yoda aus Krieg der Sterne.
\end{kommentar}

\chapter{Besuch vom Ministerium}


\enquote{Nachdem Sie nun alle ihre Patroni beherrschen, ist es an der Zeit, denke ich, dass Sie lernen wollen, wie man mit ihnen Nachrichten versenden kann}, sagte Professor Elber, als die gesamte Klasse wieder vor ihm versammelt war. Alle Schüler nickten. Selbst die Slytherin waren in seinem Unterricht voll dabei. Vermutlich, weil es auch dunkle Künste gab. \enquote{Das ist ganz einfach. Zuerst die Theorie, dann die Praxis. Der erste Schritt ist, dass Sie sich Ihren Patronus erzeugen müssen. Das ist bei allen mittlerweile so. Sie beherrschen ihn. Der nächste Schritt ist etwas schwerer, aber dank Madame Pomfreys Bemühungen, Ihnen die erste Hilfe beizubringen, sind Sie auch da schon gut gerüstet. Sie müssen sich nämlich, nachdem Sie Ihren Patronus vor sich haben, mehrere Dinge vorstellen. Zum einen, die Zielperson, oder eine Gruppe von Personen, wenn diese an einem Ort sind. Zum anderen, die Nachricht. Locken Sie Ihren Patronus zu sich. Stellen Sie sich die Personen und die Nachricht vor und teilen sie dies Ihrem Patronus mit. Das geht auch mündlich, wenn Sie das Bild intensiv in Ihren Gedanken haben.} Der Professor lief im Zimmer umher. \enquote{Wir teilen die Gruppe auf. Nach Häusern getrennt, werden Sie eine Nachricht an Ihr Gegenüber schicken, das auf der anderen Seite der Mauer steht. Die Gryffindors gehen aus dem Klassenzimmer, über den Flur und danach in das Zimmer dahinter.} Professor Elber streckte seine Hand aus und zeigte auf den Raum dahinter. \enquote{Die Slytherin ziehen ihren Namen aus einer kleinen Vase und senden Ihnen dann die Nachricht. Diese müssen Sie sich anhören und dann zurücksenden. Natürlich mit Ihrem eigenen Patronus. Die Gryffindors werden nicht wissen, von wem der Patronus kommt, bis Sie ihn sehen und gegebenenfalls erkennen.}

Damit schickte er die Gryffindors zur Tür hinaus, über den Flur und in das andere Klassenzimmer. Harry wartete mit den anderen eine Weile, bis endlich die Patroni erschienen. Da die Schüler gleichmäßig im Raum verteilt waren, konnte es zu keinen Missverständnissen kommen, weil die Patroni eventuell die falsche Person ansprachen.

Harry wartete, bis sein Patronus ihn erreichte und seine Nachricht überbrachte. \enquote{Na Potter, klappt das auch mit dem Patronus?}, gab der kleine Vogel von sich, als er auf Harrys Knie landete, da er auf einem Tisch saß.

Harry nickte, was der Vogel aber nicht registrierte, da er sich bereits aufzulösen begann. Dann schalt er sich selber über sein Verhalten. Der Patronus war nur der Überbringer. Er konnte die Antwort nicht mitnehmen. \gedanke{Warum eigentlich nicht?}, dachte er sich. Dann erschuf er seinen Hirsch. Sachlich trug er ihm auf, was er zu übermitteln habe. Er sagte ihm exakt dasselbe, was er von dem kleinen Vogel hörte. Dann lief sein Patronus durch die Mauer.

Ein paar Minuten später, nachdem alle Patroni zurückgeschickt wurden und das Ganze noch einmal durchgemacht wurde, kam wieder ein Schwarm von Insekten herein und verkündete, dass die Schüler zurückkehren sollten.

Als alle wieder auf ihren Plätzen saßen, fragte Professor Elber: \enquote{Wie lange hält denn so ein Patronus?}

\enquote{Bei guter Pflege sicherlich ein paar Wochen}, antwortete Dean, was die halbe Klasse zum Lachen brachte. Selbst der Professor musste dabei lachen.

\enquote{Damit liegen Sie gar nicht mal so falsch}, antwortete er. \enquote{Die Lebensdauer hängt von der Menge an magischer Energie ab, die Sie ihm zuteilen. Und der Menge an Konzentration. Sie haben damit die Möglichkeit, eine Nachricht zu hinterlassen, falls Sie einen Ort verlassen und wissen, dass jemand kommt.}

\enquote{Aber wenn man sich nicht auf den Patronus konzentriert, dann vergeht er doch}, warf Hermine ein.

Professor Elber sah sie eine Weile an. \enquote{Das hieße dann, dass Ihr Patronus, wenn Sie ihm eine Nachricht übergeben und losgeschickt haben, nach hundert Metern zerfließt.}

Das war nachlässig von Hermine. Daran hatte sie nicht gedacht.

\enquote{Ihr Patronus hält, solange Sie eine Arbeit für ihn haben.} Dann läutete es. \enquote{Die Stunde ist beendet.}

\trenn

Stumm saß Harry seinem Ahnen gegenüber und betrachtete ihn. Er sah in das schmale Gesicht, die eingefallenen Wangen und den Bart, der Dumbledore Konkurrenz machen konnte. Nur sein Haupthaar hatte oben eine Glatze. Harry hatte es einmal gesehen, als Salazar seinen Hut abnahm und sein Haar über den Kopf strich. Es war früh Morgens und Harry kam gerade vom Laufen und Dumbledore und duschte sich danach. Er saß in seinem Sessel und sah Salazar nur an. Ihm war einfach nur nach Gesellschaft, ohne großartig zu reden.

\enquote{Warum reagiert dein Bild anders? Seid ihr nicht ein und dieselbe Person?}

\enquote{Nein. Es ist mein Ich, beim Zustand des Malens.}

Als er Stimmen und Schritte hörte, verblasste Salazar und verschwand. Das hatte ihm gutgetan. Einfach nur dazusitzen und zu schweigen, ihn nur anzusehen. Harry nahm sich vor, Dumbledore mal über die magischen Bilder auszufragen, besonders, wenn die Person auf den Bildern noch lebten.

\trenn

Harry war gerade auf dem Weg in den Innenhof, um sich mit seinen Mitschülern in das weiche Gras zu setzen, oder auch um ein paar Zauberspiele zu spielen. Eines davon war Zaubertennis, welches mit einem normalen Tennisball gespielt wurde. Allerdings gab es keine Schläger, sondern der Ball wurde durch gezielten Einsatz von Schilden des Protego-Zaubers abgelenkt. Professor Flitwick zeigte ihnen dieses Spiel, bei dem er ziemlich gut war. Vor allem hatte es den Vorteil, dass die Größe keine Rolle spielte. Professor Flitwick trat gerade gegen Hannah Abbott an, als ein Schrei die entspannte Atmosphäre durchbrach.

\enquote{Peeves!}, drang ein Schrei an die vielen Ohren im Hof. \enquote{Was habe ich dir gesagt, über dreimal denselben Streich in den ich hinein tappe?} Peeves gackerte. \enquote{Jetzt bist du fällig.}

Das war die Stimme von Professor Elber. Dann hörte man schnelle Schritte und kurz darauf Peeves, wie er quer über den Hof flog. Ihm hinterher ein wütender Professor. Ein kurzer Lichtblitz kam von links und dann erstarrte Peeves mit einem Grinsen mitten in der Luft. Er konnte nur noch seine Augen bewegen. Professor Elber hielt in seiner Bewegung inne, als er die Schüler und Lehrer im Hof sah.

\enquote{'Tschuldigung}, sagte er. Es braucht etwas, bis er sich gefasst hatte. Dann sagte er: \enquote{Wenn jemand Lust auf eine kleine Vorführung bezüglich Geister möchte, dann wäre jetzt ein passender Zeitpunkt, in den Rosenhof zu kommen.} Er schwang seinen Zauberstab und Peeves wurde wie ein Stück Eisen von einem Magneten hinter dem immer noch wütend blickenden Professor und seinem Zauberstab hergezogen.

Als er an eine Mauer kam, knallte Peeves dagegen, wurde ein paar Zentimeter zurückgeschleudert und wie an einem Gummiband durch den steinernen Bogen gezogen.

\enquote{Au!}, schrie Peeves. Dieser Aufprall hatte wohl sein Sprachzentrum ein wenig gelockert.

Harrys Interesse war nun geweckt. Er packte sein Buch in seinen Rucksack und ging den beiden hinterher. Scheinbar hatten die anderen dieselbe Idee, denn der Innenhof war kurz darauf wie ausgestorben. Als sie im Rosengarten ankamen, war Mister Filch gerade dabei, die Rosen zuzuschneiden. Er hatte eine Schere in der Hand und schnitt einige Zweige weg. Eine magische Schere schnitt auf der anderen Seite desselben Heckengewächses.

Erstaunt drehte sich der Hausmeister um, als er die Gruppe den Hof betreten sah. Peeves schwebte immer noch fluchend in der Luft dem Zauberstab hinterher.

\enquote{Hat Peeves was angestellt?}, fragte Mister Filch genüsslich.

\enquote{Ja}, bekam er als knappe Antwort. \enquote{Dieses Mal hat er es übertrieben. Er hat seine Chance verspielt, sich zu bessern. Das hatte damals nur eine Woche gehalten. Jetzt hat er den Bogen überspannt. Er wird dafür büßen.}

\enquote{Werden meine Rosen Schaden nehmen?}, fragte der Hausmeister ängstlich.

\enquote{Falls es so sein sollte, werde ich den Schaden reparieren und der Garten wird so sein wie vorher.} Er drehte sich kurz um sich selbst und besah sich den Garten. \enquote{Ich muss mich korrigieren. Der Garten wird komplett beschnitten sein, wenn ich hier fertig bin.}

Harrys Blick wanderte umher. Er hatte gar nicht bemerkt, wie sich Dumbledore neben ihn stellte und interessiert das Schauspiel beobachtete.

\enquote{Sie müssen ein paar Sachen wissen, bevor ich beginne}, begann Professor Elber. \enquote{Geister sind bereits tot. Also kann man an ihnen die unverzeihlichen Flüche anwenden, ohne bestraft zu werden. Außerdem spüren Geister keinen Schmerz. Trotzdem krümmt sich ihr Körper, wenn sie \gst \spruch{Crucio} \gst dem Folterfluch ausgesetzt sind.} Peeves Körper begann zu zucken und sich wie unter dem Crucius-Fluch hin und her zu bewegen, als Professor Elber den Fluch sprach. Und langsam begannen sich Laute aus Peeves heraus zu bilden. \enquote{Jetzt fängt Peeves an zu simulieren.} Er brach den Fluch und die Zuckungen hörten auf.

\enquote{Dies war also der Erste der drei Flüche. Kommen wir nun zum zweiten. Dem Befehlsfluch. Leider funktioniert dieser bei Geistern nicht, sodass ich zu Demonstrationszwecken einen anderen Fluch einsetzte, der dem Bereich der dunklen Künste zuzuordnen ist. \spruch{Anima tua, anima mea!}}, sprach er, auf Peeves erneut zeigend.

Peeves' Blick wurde glasig und Professor Elber löste die Bewegungsfessel. \enquote{Du nimmst dir jetzt diese Schere und beschneidest die Rosensträucher hier. Mister Filch wird dir sagen, was und wie.} Peeves nickte. \zauber{Anima tua!}

Peeves setzte sich in Bewegung, nahm die Schere, sie schwebte in seiner kleinen Geisterhand, flog auf Mister Filch zu und erwartete seine Befehle. Dieser war ganz begeistert und erklärte Peeves, was er zu tun hatte. Dann legte er los und schnitt in Rekordzeit die Rosen. Als er fertig war und die Schere fallen ließ, fiel der Fluch von ihm ab und Peeves schaute erstaunt um sich.

\enquote{Du}, schrie er boshaft zu Professor Elber. \enquote{Das wirst du mir büßen.} Er flog auf ihn zu.

\zauber{Avada Kedavra.}

Peeves wurde zunehmend langsamer und schien in der Luft wie versteinert. Er sah nun aus wie eine durchsichtige Steinstatue. Professor Elber schwang seinen Zauberstab und ein kleiner rotierender Wirbel erschien unter ihm. Die steinerne Hülle, die Peeves umgab, floss von ihm ab. Dann wurde auch Peeves, durch viele Flüche die er aussprach begleitet, in den Wirbel gezogen, der in sich kollabierte und mit einem leisen \geraeusch{Plopp} verschwand.

\enquote{Was hast du gemacht?}, fragte Professor McGonagall.

\enquote{Peeves entsorgt}, sagte er todernst, als er seinen Zauberstab einsteckte. \enquote{Er hat es eindeutig übertrieben.} Und dann mit einem gehässigen Lachen. \enquote{Er hat nur eine Chance, wiederzukommen. Er muss schwören, keine Streiche mehr zu spielen.}

\enquote{Das dürfte er, ohne mit der Wimper zu zucken tun.}

\enquote{Ja}, lachte er weiter, \enquote{aber wenn er das tut und einen Streich vorbereitet, wird er magisch dazu gezwungen, selber in die Falle zu tappen, wenn einer Gefahr läuft, Opfer des Streiches zu werden.}

Mister Filch lauschte den Ausführungen nur mit halbem Ohr. Er besah sich die Rosenschnitte und meinte dann anerkennend: \enquote{Zumindest hat Peeves noch etwas Gutes getan, bevor er gegangen wurde. Hähä.}

Da Harry nun etwas freihatte, wollte er es noch einmal versuchen, zu Salazars Bild zu gelangen. Also machte er sich auf den Weg zu den Kerkern. Ab und an traf er einen Slytherin, der ihn komisch ansah, aber noch war er in einem Bereich, der kein Aufsehen erregen sollte. Doch dann wurde es kritisch. Er schaffte es dennoch, indem er sich hinter Rüstungen drückte und in schattigen Bereichen aufhielt. Dann stand er wieder dicht vor dem Bild Salazar Slytherins.

\enquote{So, sind Sie wieder da?}, kam von dem Bild. \enquote{Da haben Sie mich aber schön hereingelegt. Ein Gryffindor, der sich für einen aus meinem Haus ausgibt}, spottete er Harry an.

\enquote{Ich habe nie behauptet, dass ich ein Slytherin bin, wenn hier einer das reininterpretiert hatte, dann warst das du.} In Harry baute sich eine gewisse Sicherheit auf, was das Bild anbelangte.

\enquote{Wie kommen Sie dazu, mich zu duzen, Sie unverschämter Schüler. Ich werde Snape rufen, damit er Sie von hier entfernt\abs}

Doch er verstummte, als Harry das Amulett hinter seiner Schulrobe hervorholte.

\enquote{Woher haben Sie das?}, fragte der Salazar im Bild ganz aufgeregt.

\enquote{Die Frage, die du mir stellen solltest, ist, was ich sehe, wenn ich es in der Hand halte. Aber das bereden wir besser im Inneren.}

\enquote{Ha, versuch es ruhig. Das Amulett zu haben, bedeutet nicht, eingelassen zu werden.}

Harry kam dem Bild näher und berührte mit dem Amulett den Bilderrahmen. Salazar verstummte und das Bild schwang auf. Ob Salazar generell stumm wurde, wenn das Bild aufschwang, oder vor Schreck, konnte Harry nicht sagen.

Er trat hinter die Schwelle in den dunklen Raum. Das Bild klappte hinter ihm zu und Harry stand im Dunklen. Er wartete, bis sich die Alarmanlage, wie ihm Salazar mitgeteilt hatte, bei neuen Personen beim ersten eintreten zeigte. Er stand still und horchte in die Dunkelheit hinein.

\parsel{Ein Eindringling}, hörte er.

\parsel{Ich bin kein Eindringling. Ich darf hier sein. Als Erbe und Nachfahre Slytherins bin ich hier, um meinen Besitz zu übernehmen.}

\parsel{Ein Nachfahre unsseress Herrn isst hier. Bereitet alless vor.} Und nach einer Weile sagte die Stimme: \parsel{Bereit. Jeder, der lichtempfindlich isst, hat ssich verkrümelt.}

Harry hatte ein komisches Gefühl und hielt den Zauberstab direkt vor sein Gesicht um ihn dann zu entzünden. Eine Schlange, die direkt vor seinem Gesicht mit geöffnetem Maul da stand, zuckte zurück.

\parsel{Hast du etwa vorgehabt, mich anzugreifen?}, fragte Harry.

\parsel{Nicht mehr}, antwortete die Schlange. \parsel{Das Amulett isst eindeutig. Und dass ess unss bei Licht bessehen ssagt, dassss ssie hier ssein dürfen, isst eindeutig. Wilkommen Meisster.} Die Schlange verzog sich.

\parsel{Danke}, antwortete Harry. Er trat einige Schritte durch eine Art Tunnel und sah in den dunklen Raum. Mit acht Bewegungen seines Zauberstabes öffnete er die Vorhänge der vier Fenster.

Er sah sich in dem kleinen Wohnzimmer um. Links war an der Wand eine verschlossene Tür. Daneben ein Kamin. \gedanke{Schlafzimmer oder Bad würde sich gut dahinter machen. Wegen des Kamins}, sinnierte Harry. Gerade aus waren vier Fenster, dessen Vorhänge vor kurzem noch geschlossen waren. Er sah hinaus und meinte, dass er auf einem Turm stand, doch eigentlich war er unter der Erde. \gedanke{Muss ein Zauber sein}, dachte er. Er drehte sich nach rechts und sah ein kleines Bücherregal mit Büchern verschiedenen Alters. Er trat um den Tisch und die Sessel herum näher heran und sah grob über die Auswahl an Büchern. \gedanke{Die müssen schon zu Salazars Zeiten alt gewesen sein. Die Zauberer müssen schon lange vor den Muggeln die Bücher erfunden haben. \gst Wann haben die nochmal die Bücher\abs? Nein, das war der Buchdruck}, überlegte Harry.

Er drehte sich herum und sah sich nun die Möbel an. Alte, aber stilvolle Möbel waren da. Drei Sessel und eine Doppelcouch waren um einen passenden Tisch herum gestellt. Der Tisch hatte die Höhe, um bequem Getränke abzustellen und die Füße hinauf zu legen. Für das Mittagessen war er allerdings zu niedrig.

\enquote{Willst du mir nicht den jungen Mann vorstellen, der da in unseren Räumen herumläuft?}, hörte er plötzlich.

Er drehte sich in die Richtung, aus der die Stimme kam, und sah ein Bild über dem Eingang hängen. Darauf waren zwei Personen abgebildet. Eine Frau um die sechzig, schätzte Harry, und Salazar in demselben Alter.

\enquote{Ich kenne den Jungen doch gar nicht. Er behauptet, mein Nachkomme zu sein. Und jetzt hat er sich auch noch gewaltsam Zugang verschafft.}

\enquote{Salazar Slytherin, der junge Mann ist hier hereingekommen und du hast mir immer wieder gesagt, dass hier keiner reinkommt, der es nicht verdient hat, oder hier sein darf. Wie kannst du so etwas behaupten?} Dann gab sie ihm eine Klaps auf den Hinterkopf.

Harry musste grinsen, als Salazar seine Frau böse ansah. Er setzte sich auf einen Sessel und schaute den beiden zu.

\enquote{Der Bengel hat sich als Slytherin ausgegeben, um mich in ein Gespräch zu locken.}

Seine Frau warf einen seitlichen Blick auf Harry, der nur schmunzelnd zusah.

\enquote{Also Mister\abs}

\enquote{Potter, Ma'am. Harry Potter.}

\enquote{Ach, nennen Sie mich Agatha, ich bin da nicht so kompliziert wie mein Mann.}

\enquote{Glauben Sie mir Ma'am, das lässt nach. Er wird sich noch ändern.}

\enquote{Aber sagen Sie mal Harry, wie sind Sie eigentlich an das Amulett gekommen?}

\enquote{Das war ein Geschenk. Ich habe erst später erfahren, was es mit dem Amulett auf sich hat, und dass ich, nachdem ich was sehe, wenn ich das Amulett festhalte, mitgeteilt bekommen habe, dass das nur ein Nachfahre Salazar Slytherins kann.} Agatha bekam große Augen. \enquote{Wissen Sie Agatha, zum damaligen Zeitpunkt habe ich von Slytherin nicht viel gehalten. Er war derjenige, der die Idee der Reinblütigkeit in die Welt gesetzt hat. Schwarzmagier kamen fast ausschließlich aus diesem Haus. Wissen Sie, ich bin unter Muggeln aufgewachsen.}

\enquote{Was?}, kreischte Salazar. \enquote{Das ist ja ungeheuerlich.}

\enquote{Klappe, Salazar. Mit dieser Idee hast du genug Ärger gestiftet.}

\enquote{Was weißt du schon? Diese Idee wird unsere Art schützen.}

\enquote{Irrtum, Salazar}, mischte sich Harry ein, sodass beide zu ihm sahen. \enquote{Diese genetische Inzucht schafft mehr Probleme, als sie löst. Krankheiten durch Jahrhunderte lange Inzucht und Degeneration der Magie sind die Folgen. Wir haben immer noch damit zu kämpfen, dass du diese Idee hattest. Sie mag vielleicht anfangs dazu geführt haben, dass eine Verbesserung zu sehen war, aber es hat sich später rausgestellt, dass das ein Fehler war. Seitdem stehst du bei den anderen drei Häusern auf schwerem Stand. Alle Schüler und Schülerinnen deines Hauses sind mehr oder weniger isoliert. Die anderen arbeiten freundschaftlich zusammen. Deine Idee hat sich so rapide durchgesetzt, dass wir die Folgen noch heute spüren. Du hast es zeitlebens nicht mehr geschafft, deinen Fehler zu korrigieren, obwohl du es versucht hast.}

\enquote{Woher willst du das wissen?}, fragte er ohne Anflug von Sarkasmus und Widerwillen.

Harry stutzte. \enquote{Hast du mir etwa was vorgemacht, um deinen Ruf zu schützen? Zuzutrauen wär’s dir. Slytherins sind ja bekanntlich verschlagen.}

Salazar zog seinen Kopf ein und murmelte: \enquote{Ja, vielleicht. Ich wusste ja nicht, dass du es weißt. Ich bin hier im Schloss schon ein paar mal angeeckt deswegen.} Und dann wieder normal. \enquote{Woher weißt du von meinem Sinneswandel? Das steht doch nirgendwo.}

\enquote{Ich bin auf ein älteres Ich von dir getroffen und dann habe ich noch von jemandem etwas über dich erfahren. Er scheint dich zu kennen. Ich bin mir nur nicht im Klaren darüber, woher er soviel über dich weiß.}

Salazar und seine Agatha sahen sich an.

\enquote{Wie auch immer, willkommen in unserem\abs äh deinem Heim.}

\enquote{Danke, Agatha. Wo geht es eigentlich hin, wenn ich durch die Türen gehe?} Jetzt bemerkte er auch den Aufgang, der neben der Tür beim Bücherregal, aber an der Wand mit dem Bild, war. Eine Wendeltreppe führte ein Stockwerk höher.

\enquote{Die Tür neben dem Kamin führt ins Bad und zur Toilette. Eine weitere Tür führt auf die Insel im großen See vor dem Schloss. Es ist eine magische Tür und sie lässt sich nur von hier öffnen. Wenn man sie verschließt, kann man von draußen nicht mehr rein. Die Tür neben dem Bücherregal führt zu den anderen Gemeinschaftsräumen und\abs zu den anderen Räumen der Schulgründer.}

\enquote{Was ist mit der Kammer?}, fragte Harry direkt an Salazar gewandt.

Der wurde sofort bleich. \enquote{Welche Kammer?}, stotterte er.

Harry zog eine Augenbraue hoch. \enquote{Die Kammer unter dem Schloss mit einem Basilisken darin, der seit vier Jahren tot ist und der bereits eine Schülerin auf dem Gewissen hat. Sie haust seit fünfzig Jahren auf einem Mädchenklo im dritten Stock hier im Schloss.} Salazar verlor jegliche Farbe. \enquote{Weißt du, ein mächtiger schwarzer Zauberer hat eine gute Freundin von mir dazu gebracht, die Kammer zu öffnen.}

\enquote{Wie überzeugt?}

\enquote{Mit einem Tagebuch. Es war ein Horkrux.}

Salazar bekam große Augen. \enquote{Erzähl weiter.}

\enquote{Wir, das heißt zwei meiner Freunde und ich, haben herausgefunden, wo der Eingang ist. Dort ist auch der Geist des Mädchens. Ich öffnete den Eingang und rutschte mit meinem Freund herunter, da meine andere Freundin versteinert wurde. Sie hatte mit einem Spiegel um die Ecke gesehen und so den Basilisken, wie sie herausgefunden hatte, gesehen. Unten angekommen bin ich dann weiter gegangen und habe die Kammer geöffnet.}

\enquote{Du kannst Parsel?}

\enquote{Sicher. Auf jeden Fall habe ich es geschafft, mithilfe eines Phönix’, der dem Basilisken die Augen ausgekratzt hatte, ihn mit Gryffindors Schwert zu töten. Dann habe ich den Horkrux mit einem Basiliskenzahn zerstört, den das Tier leider in meinem Arm versenkt hatte, während unseres Kampfes. Zum Glück haben mich die Tränen des Phönix geheilt. Es hat nie einer, außer meinen drei Freunden und den Lehrern, erfahren, was wirklich unten in der Kammer passiert ist.}

\enquote{Das war aber noch nicht alles?}, fragte Agatha Slytherin ehrlich interessiert nach.

\enquote{Nein und dabei soll es auch bleiben. Das ist das Wichtigste. Alles andere sind Details, die nicht so wichtig sind. Euch habe ich das als einzigen Außenstehenden erzählt.} Beide nickten. \enquote{Wohin geht denn der Aufgang? \gst Halt, führt der Gang jetzt auch zur Kammer?}

Agatha sah ihren Mann fragend an.

Dieser nickte. \enquote{Bei der Abzweigung Hufflepuff und Ravenclaw einfach gerade aus. Drücke dein Amulett gegen die Wand. Das ist eine Abkürzung. Der andere Weg ist farbig hinterlegt.}

\enquote{Die Wendeltreppe führt zu unserem\abs alten Schlafzimmer und zu weiteren Toiletten, sowie Gästezimmern, obwohl diese selten benutzt wurden. Nach unserer Rückkehr nach Hogwarts, ohne dass mein Mann wieder lehrte, hatten wir kaum noch Gäste.}

\enquote{Ihr kamt wieder zurück?}

\enquote{Ja, es wurde nicht sehr begeistert aufgenommen, aber in späteren Jahren haben wir hier wieder ein Zuhause gefunden. Kurz danach ist dieses Bild entstanden. Wir haben praktisch mitbekommen, wie wir gestorben sind. Die Elfen hatten den Auftrag, uns in der Familiengruft beizusetzen. Seitdem war es hier dunkel und wir haben geschlafen. Wir wissen nur noch, dass etwa tausend Jahre vergangen sind, bevor Sie aufgetaucht sind.}

Harry sah auf seine Uhr. \enquote{Es ist schon spät, ich sollte\abs}

\enquote{Hier schlafen. Rufen Sie einen Hauselfen, der wird Ihnen das Zimmer herrichten.}

\enquote{Ich denke, das werde ich selber schaffen}, sagte Harry.

\enquote{Wie?}, meinte Salazar. \enquote{Selber?}

\enquote{Na ja}, gab Harry zu, \enquote{ich habe ein eigenwilliges Verhältnis zu Hauselfen.} Und er fügte hinzu: \enquote{Aber das ist eine andere Geschichte.} Er stand auf und ging nach oben. Er öffnete die erste Tür und sah in das Bad. Dann schloss er die Tür und widmete sich der zweiten und sah jetzt in das Schlafzimmer der Slytherins. Es sah aus wie eine luxuriöse Version der Schlafzimmer, die er einmal gesehen hatte, nur konnte er sich beim besten Willen nicht mehr erinnern, wo. Er rief nach Kreacher und trug ihm auf, ihm seinen Schlafanzug zu bringen. Nachdem er wieder da war, sagte er noch: \enquote{Ich werde diese Nacht hier schlafen. Sag bitte Hermine und Ron, dass ich einen sicheren Schlafplatz heute habe und sie sich keine Sorgen machen müssen. Er ist aber in keinem der Gemeinschaftsräume.}

Kreacher nickte, verneigte sich und verschwand. Dann richtete sich Harry her und ging ins Bett. Er sah an die Decke und dachte nach. Über sich, über Salazar, dessen Frau, über Luna, Ginny und sich. Er dachte über alle nach, die ihm etwas bedeuteten. Langsam dämmerte er ins Reich der Träume.

Schon lange hatte er nicht mehr so gut geschlafen. Er hatte wieder an Erkenntnissen gewonnen. Er wusste nicht, was er davon seinen Freunden erzählen wollte. Er wusste nicht, ob sie es verstehen würden. Mit diesen Gedanken erwachte er und fand unten Kreacher mit einem Frühstück vor. Er versuchte, Kreacher zu überreden, mit ihm zu essen, doch er schaffte es nicht. Dann verschwand Kreacher mit den Resten in die Küche.

Jetzt meldete sich Agatha wieder: \enquote{Ich glaube, ich verstehe Sie Harry. Was war Ihr erster Kontakt mit einem Hauselfen?}

\enquote{Ein Elf, der mich warnte. Er diente einem dunklen Magier. Zumindest bin ich davon überzeugt. Er hat ihn nicht sehr gut behandelt.}

\enquote{Der Elf ist gestorben?}

\enquote{Nein, er wurde befreit.}

\enquote{Befreit?}

\enquote{Ja, ich habe seinen Meister\abs hereingelegt.}

Agatha schmunzelte. \enquote{Die Geschichte müssen Sie mir mal erzählen, Harry.}

Harry nickte und verließ die Räumlichkeiten durch die Gänge und tauchte im Gemeinschaftsraum der Gryffindors auf. Keiner entdeckte ihn, als er neben der Tür durch die offene Wand kam. Er holte seine Schulsachen und machte sich auf den Weg.

Jetzt hatte Harry wieder Unterricht im Fach \VgddK. Professor Elber war bereits im Klassenzimmer und lehnte sich an sein Pult. Die Klasse träufelte so nach und nach ein und jeder nahm seinen Platz ein. \enquote{Heute}, so fing er an, \enquote{werden wir uns mal ein wenig mit der zauberstabfreien Magie befassen. Und demnächst, nachdem ihr bei Professor Flitwick schon Zaubern ohne Worte durchgenommen habt, das Ganze auch ungesagt. Wir werden nicht viel schaffen, aber kleine Aufrufzauber oder ein einfaches Alohomora dürfte drin sein.}

Ein leises Raunen durchfuhr die Klasse. \enquote{Dies ermöglicht es Ihnen, aus einer misslichen Situation einen kleinen Vorteil zu erlangen und eventuell zu verschwinden}, fuhr Professor Elber seinen Vortrag fort. \enquote{Fangen wir mal mit was Einfachem an.} Er zog seinen Zauberstab und legte ihn auf seinen Tisch, setzte sich dahinter und sagte dann: \enquote{Nehmen Sie bitte alle Ihren Zauberstab heraus und legen Sie ihn vor sich auf den Tisch.}

Die ganze Klasse tat wie geheißen und alle nahmen ihre Zauberstäbe heraus und legten sie vor sich auf den Tisch. Professor Elber fasste sich mit drei Fingern an sein Kinn und schaute auf seinen vor ihm liegenden Zauberstab. Er strich an seinem Kinn entlang und sah danach hoch. \enquote{Erinnern Sie sich noch an ihre erste Stunde Flugunterricht bei Madame Hooch?}, fragte er.

Die ganze Klasse bejahte.

\enquote{Wie genau spielte sich das ab?} Er zeigte mit einem Finger auf einen seiner Schüler, worauf dieser anfing zu erzählen.

\enquote{Wir haben uns links neben unseren Besen aufgestellt, dann unsere rechte Hand über ihn ausgestreckt und danach \accentuate{Auf} gerufen.}

\enquote{Nun, das hier funktioniert ganz ähnlich}, fuhr Professor Elber fort. Er streckte seine Hand über seinen Zauberstab aus und rief: \enquote{Auf}. Sein Zauberstab hob vom Tisch ab und war kurz darauf in seiner Handinnenfläche, welche sich reflexartig schloss. Er drehte seine Hand und öffnete sie danach, nahm seinen Zauberstab aus seiner Hand und legte ihn wieder zurück auf den Tisch. Danach faltete er seine Hände und legte diese auf dem Tisch ab. \enquote{Habt ihr alle gut aufgepasst? Oder soll ich es nochmal vormachen?} Die Klasse stimmte einstimmig für nochmal vormachen. Professor Elber hob abermals seine Hand über seinem Zauberstab und rief: \enquote{Auf}. Der Zauberstab schnellte in seine Handfläche und seine Hand schloss sich. Er drehte abermals seine Hand und legte danach seinen Zauberstab auf den Tisch. Er stand auf und fuhr fort. \enquote{Wenn Sie mit Ihrem Besen gut umgehen können, brauchen Sie hier nur etwas mehr Konzentration und Ausdauer, denn die Besen sind dafür extra vorbereitet worden. Ihre Zauberstäbe hingegen nicht. Bitte, versuchen Sie es alle.}

Er lief durch die Reihen, während alle versuchten, ihren Zauberstab in ihre Hand zu rufen.

\enquote{Sobald es Ihnen einmal gelungen ist, legen Sie ihn bitte zurück und wiederholen die Übung.}

Die Stunde verflog rasch. Jeder schaffte es nach dem Ende der Stunde, seinen Zauberstab aufzurufen. Selbst vom Boden mit der Hand ganz nach unten gestreckt klappte es. Dann läutete es und die Stunde war beendet.

\enquote{Wo warst du letzte Nacht?}, fragte Ron Harry.

\enquote{In einem Bett\abs alleine}, fügte er hinzu. \enquote{Ich habe so gut geschlafen wie schon lange nicht mehr. Das Bett war sehr angenehm.}

\enquote{Wo hast du geschlafen? In \accentuate{unserem} Gemeinschaftsraum?}, fragte Hermine.

\enquote{Besser}, antwortete Harry.

Ron und Hermine sahen ihn angespannt und neugierig an. Er fiel mit ihnen etwas zurück.

\enquote{Ich war in}, doch plötzlich fiel ihm wieder ein, dass er noch nichts erzählen wollte, \enquote{einem Bett im Raum der Wünsche. Ich hatte eine Idee und wollte ein Zimmer. Eines, wie es von einem der Gründer bewohnt hätte sein können. Es sah Klasse aus.}

\enquote{Du hast was?}, fragte Ron aufgeregt. \enquote{Und, wie ist Gryffindors Zimmer so?}

\enquote{Als ob der Raum wüsste, wie Gryffindor so eingerichtet war. Das Zimmer entsprang nur Harrys Fantasie. Und überhaupt, wie kommst du auf Gryffindor? Es könnte auch Hufflepuff oder Ravenclaw sein.}

\enquote{Du vergisst Slytherin, Hermine}, fügte Harry hinzu.

\enquote{Damit macht man keine Witze, Harry}, beschwerte sich Ron.

\enquote{Wenn du meinst}, sagte Harry und lief schneller, um zur nächsten Stunde zu gelangen.

Ron und Hermine sahen sich nur verständnislos an.

\enquote{Willst du mir damit sagen}, fragte Hermine, als sie wieder mit Harry gleich auf war, \enquote{dass du ein Zimmer von Slytherin hast erscheinen lassen? Wie bist du denn da drauf gekommen?}

\enquote{Ich musste gerade an ihn denken. Ich habe beim Laufen das Amulett gespürt. Dann habe ich an ihn gedacht. Und so hat mir der Raum der Wünsche sein Zimmer erschaffen. Zumindest glaube ich das.}

\enquote{Ja aber\abs}

\enquote{Nichts aber. Ich bin einer der Nachkommen Slytherins. Also sein Erbe. Ich werde mich deswegen nicht verstecken. Außerdem habe ich interessante Dinge über ihn herausgefunden.} Und so erzählte Harry ihnen, was er wusste und mitteilen durfte. Bei Ron und Hermine waren die Geheimnisse sicher.

\trenn

Er schlich wieder einmal in die Küche, um sich mit Kreacher zu unterhalten. Je länger er mit Kreacher Zeit verbrachte und sich mit ihm unterhielt, desto besser verstanden sie sich. Auf einmal standen drei kleine Elfen vor ihm und schauten ihn kurz an. Danach verschwanden sie wieder. Harry wunderte sich darüber, dachte sich aber nichts Großes dabei.

\enquote{Und Kreacher, wie geht es dir?}, fragte ihn Harry.

\enquote{Kreacher geht es gut, Sir.}

\enquote{Wie lange hast du den Blacks gedient?}, fragte Harry den Elfen.

\enquote{Seit Kreachers Geburt.}

\enquote{Und wann war das?}

\enquote{Kreacher weiß es nicht genau, aber er erinnert sich an ein Weihnachtsfest im Jahre 1607. Er dürfte damals gerade ein Jahr alt gewesen sein.}
% 1996 minus 390 Jahre; Elfen können bis zu 400 Jahre alt werden

%1606
Harry rechnete nach. \gedanke{Das sind 390 Jahre.} \enquote{Und wie alt können Elfen werden?}

\enquote{Kreachers Art kann bis zu 400 Jahre alt werden.}

\enquote{Das heißt, du wirst bald sterben?}

\enquote{Ja Meis\gst Sir Harry}, antwortete der Elf.

Harry fragte weiter. \enquote{Was soll mit deinem Körper passieren, wenn du nicht mehr bist?}, fragte Harry nach.

Der Elf starrte ihn an. Harry sah, wie sein Kopf arbeitete. \enquote{Kreachers Vorfahren und Kollegen wurden alle geköpft und deren Köpfe wurden in der Großen Halle des Hauses aufgehängt.}

\enquote{Das hast du mir bereits erzählt, Kreacher. Aber, was soll mit \accentuate{dir} passieren, wenn du gestorben bist?}, fragte Harry erneut.

Kreacher dachte nach. Er dachte lange nach. Harry bekam in der Zwischenzeit von einem der drei Hauselfen, die ihn vorhin angesehen hatten, ein Glas Kürbissaft.

\enquote{Wenn Sir Harry damit einverstanden ist, dann würde Kreacher gerne\abs verbrannt werden.}

Harry nickte und verstand.

\enquote{Noch etwas anderes, Kreacher. Du kennst dich doch in meinem Haus\abs dem Haus der Blacks aus.}

Kreacher nickte.

\enquote{Hast du da ein Medaillon gesehen? Es ist etwa daumenlang und einen halben Handteller groß. Es ist rechteckig und hat eine Schlange auf dem Deckel, die mit einem S verworren ist.}

Kreacher fing an zu schluchzen und schnappte sich eine Pfanne, die ihm Harry wegnahm.

\enquote{Du wirst dich nicht bestrafen, Kreacher.} Der Elf nickte. \enquote{Setzt dich zu mir und erzähl.}

Also fing Kreacher an zu erzählen, wie er Regulus Black, seinen liebsten Meister nach Harry, einen Gefallen erfüllen sollte. Er wurde mit Voldemort losgeschickt auf eine Mission. Er war in einer Höhle und fuhr auf einem kleinen Boot über einen See zu einer Insel. Dort musste er eine Flüssigkeit trinken. Sie brannte furchtbar und er bekam Durst. Er musste alles trinken. Dann legte Voldemort ein Medaillon hinein und verschwand. Kreacher erzählte, wie er ans Wasser ging, um zu trinken. Inferi zogen ihn unter Wasser und versuchten ihn zu ertränken. In diesem Moment konnte sich Harry wieder an die Inferi im Wasser im See erinnern. Dann erzählte Kreacher, wie er heimkehrte. Harry verstand nicht. Erst als Kreacher deutlich wurde, verstand er, dass Kreacher einfach aus der Höhle apparierte. \gedanke{Voldemort hat einen Fehler begangen}, folgerte er. \gedanke{Einen sehr großen sogar.} Dann erzählte Kreacher weiter. Wie er von Regulus ausgefragt wurde und er ihn mit in die Höhle nehmen musste. Er musste Regulus versprechen, niemandem aus der Familie hiervon zu erzählen. Dann nahm er Kreacher das Versprechen ab, das Medaillon zu zerstören. Er Trank die Flüssigkeit, legte eine Fälschung hinein, ging voller Durst ans Wasser und wurde nach unten gezogen.

Kreacher weinte während er von Regulus erzählte und Harry musste ihn beruhigen. Er musste dieses Medaillon haben.

\enquote{Kreacher, kannst du mir das Medaillon\abs?} Er unterbrach sich. \enquote{Kannst du mich zum Grimmauldplatz mitnehmen?}

Kreacher nickte, nahm Harrys Hand und verschwand mit ihm. Er tauchte mit ihm im Salon auf, in dem auch die Vitrinen standen. Kreacher zeigte auf die Vitrine, in der das Medaillon lag. Harry kam dem Holzkasten mit den Glasscheiben näher und öffnete ihn. Dann nahm er das Medaillon heraus und hielt es an der Kette. Er schloss die Tür wieder und setzte sich auf den Boden. Kreacher saß ihm gegenüber.

\enquote{Von ihm geht eine Menge dunkler Magie aus}, erklärte Kreacher. \enquote{Kreacher konnte es nicht zerstören. Er hat jeden Zauber versucht, den er kann. Er hat versucht, es mit Messern und Gabeln zu zerstören und auch ins Feuer hat er es geworfen, doch nichts half. Kreacher hat sich oft dafür bestraft.}

\enquote{Kreacher, das wirst du nicht tun. Du wirst dich nicht mehr deswegen bestrafen.}

Kreacher nickte erleichtert. Die Strafen hatten seinem Körper schwer zugesetzt.

\enquote{Das ist keine normale schwarze Magie. Das ist mehr. Es lebt, es pulsiert}, sagte Harry, das Medaillon betrachtend. \enquote{Es ist etwas, was verhindert, dass Vold\abs naja du kennst ihn, Kreacher \gst dass er stirbt.}

Der Elf nickte. \enquote{Kreacher kennt einen solchen Zauber, den der Herr meint. Es gibt in der Bibliothek ein Buch darüber.}

\enquote{Bevor wir gehen holst du es mir bitte und tust es in meinen Koffer in Hogwarts. Ich habe Vorkehrungen getroffen, dass niemand ran kann. Er macht für dich eine Ausnahme. Es ist der Kleine.}

Kreacher nickte erneut. \enquote{Darf Kreacher etwas vorschlagen?} Harry nickte. \enquote{Gebt Kreacher das Medaillon in die Hand. Dann nehmt ihr ein Stofftuch und nehmt es damit entgegen und wickelt es ein. Es versucht nämlich diejenige Person zu beeinflussen, die es als Letztes in der Hand hatte. Auf Elfen hat es keine Auswirkung. Und da ich der Letzte bin, der es in Händen halten wird, kommt keiner von uns zu Schaden.}

Harry nickte und gab Kreacher das Medaillon. Dann zog er aus seiner Tasche ein Taschentuch und nahm das Medaillon mit dem Taschentuch von Kreacher entgegen. Er wickelte es in das Taschentuch ein und gab es Kreacher.

\enquote{Lege das zu dem Buch in meinen Koffer. Wird Zeit, dass wir gehen.}

Kreacher nickte, stand mit Harry auf und nahm ihn bei der Hand. Dann apparierte er in die Bibliothek des Hauses und ließ ein Buch zu sich heranschweben. Als er es in seiner Hand hatte, war er mit Harry auch schon verschwunden. Sie tauchten an derselben Stelle in Hogwarts wieder auf, von der sie verschwunden waren.

\enquote{Warum hier?}, fragte Harry.

\enquote{Elfen können zwar durch die Anti-Apparitionszauber der menschlichen Zauberer hindurch, müssen aber an dieselbe Stelle wieder zurück, bevor sie innerhalb weiterapparieren}, erklärte eine der Elfen aus der Küche, die gerade an beiden vorbeilief und nicht im Geringsten darüber erstaunt war, dass beide plötzlich auftauchten.

Harry nickte. \enquote{Danke dir.} Dann wandte er sich Kreacher zu. \enquote{Ich werde mich um das Medaillon kümmern. Darf ich es auspacken und untersuchen, oder hat das auch Auswirkungen auf mich?}

\enquote{Nein Sir. Nur ein körperlicher Kontakt wirkt sich aus. Es ergreift von einem Besitz.}

\enquote{Du kannst die Sachen aufräumen, Kreacher.}

Der Elf verschwand.

Nachdenklich verließ er die Küche. \gedanke{Wie bei Ginny in ihrem ersten Jahr. Das Tagebuch hat Besitz von ihr ergriffen. Dumbledore weiß vermutlich, was es ist. Wenn ich nicht herausfinde, wie man es zerstört, dann gehe ich zu ihm.}

Auf dem Weg zurück traf er auf Professor Elber, der an einem der vielen Leuchtkörper in Hogwarts scheinbar seinen Frust ausließ. Gerade lief er an einer Lampe vorbei, als er abrupt anhielt und die Lampe anstarrte. Dann zog er seinen Zauberstab und sprengte sie von der Wand weg. Verächtlich schnaufend trat er näher an die Wand und setzte seinen Zauberstab scheinbar in der Luft an und fuhr von unten nach oben die Kontur der Lampe nach. Sein Zauberstab näherte sich der Wand. Die Lampe baute sich, seinem Stab folgend, auf, verband sich am nächsten Punkt mit der Wand und bildete sich weiterhin aus dünner Luft zu einem massiven Messing-Gebilde aus. Als sich der Schirm ausgebildet hatte, stoppte die Bewegung des Stabes und die Lampe begann zu leuchten. Mit einem achtlos ausgeführten Schlenker verschwand auch die zerstörte Lampe und das zerbrochene Glas vom Boden vollständig.

\enquote{Darf ich Sie was fragen, Professor?}, fragte Harry, als er sich ihm näherte. Professor Elber hob eine Augenbraue. \gedanke{Ach ja: \enquote{Sie können mich alles fragen, was Sie wollen. Ich werde Ihre Fragen immer vertraulich behandeln. Sie bekommen nur nicht auf jede Frage eine Antwort}}, dachte Harry und begann: \enquote{Warum haben Sie gerade die Lampe zerstört?}

Jetzt lächelte sein Professor. \enquote{Wohin sind Sie unterwegs?}, fragte er.

\enquote{Zum Essen}, antwortete Harry.

\enquote{Gut, gehen wir ein Stück.} Damit setzte er sich in Bewegung und Harry lief neben ihm her. Nach einigen Schritten sagte er: \enquote{Ist Ihnen noch nie aufgefallen, dass einige Lampen in Hogwarts gelegentlicher Auffrischung bedürfen, andere hingegen nicht?}

Jetzt begann Harrys Hirn zu arbeiten. Doch als sie in der Großen Halle ankamen, konnte er ihn nichts mehr fragen, da sich ihre Wege trennten.

\trenn

Er saß wieder alleine im Gemeinschaftsraum. Es war Sonntagnachmittag und alle waren draußen. Doch Harry hatte keine Lust. Er hatte Ron, Hermine und den anderen gesagt, dass er nicht mitgehen würde. Diese Nacht war zu anstrengend für ihn gewesen. Er hatte wieder schlecht geschlafen und saß auf einem Sofa in der Ecke und dachte an die schrecklichen Träume und Visionen der letzten Nacht zurück.

\begin{rueckblick}
All dieses nur, weil er wieder einmal unter seinem Tarnumhang im Schloss herumgeschlichen war und die Punkte auf der Karte auszumachen versuchte. Zumeist waren es simple kleine Räume mit jeder Menge Plunder. \gedanke{Mister Filch hätte seine Freude daran}, dachte er noch bei sich. Einmal hatte er eine Passage entdeckt, die ihn vier Etagen auf einmal überbrücken ließ. Und ein anderes Mal ein geheimer Zugang zu einem der Vertrauensschüler-Bäder. Doch wenn er baden wollte, ging er in Salazars Räumen baden. Ron und Hermine hatten ihm dabei geholfen, die Punkt zu finden. Doch noch immer waren nicht alle Punkte gelöst.
\end{rueckblick}

Er sah, wie Voldemort jemanden ermordete, wie er versuchte, Informationen aus jemanden herauszuquetschen. Er war schrecklich müde und schlief ein. Als er erwachte, saß Luna ihm gegenüber. \enquote{Luna? Wie bist du hier hereingekommen.} Doch sie verschwand wieder. Zurück blieb nichts als leere Luft.

\gedanke{Na toll. Jetzt fantasiere ich auch noch}, dachte sich Harry. \gedanke{Ich sehe schon Gespenster.} Doch kaum hatte er das gedacht, schwebte schon Salazar Slytherin vor ihm im Raum. Immer noch müde schloss er die Augen und sagte nur matt. \enquote{Hallo Salazar. Bist du auch eine Einbildung?}

Dann durchfuhr ihn ein eiskalter Schauer. Harry schrak hoch und Salazar war verschwunden. \enquote{Hinter dir}, hörte er plötzlich. Er drehte sich um und sah in das Gesicht seines Ahnen. \enquote{Ich dachte, an so einem schönen Tag würdest du nach draußen gehen.}

\enquote{Ich bin müde, Salazar.} Und Harry drehte sich wieder um und ließ sich zurück auf das Sofa fallen. Ihn durchfuhr abermals ein eiskalter Schauer.

\enquote{Jetzt besser?}, fragte ihn sein Ahne mit schelmischem Grinsen im Gesicht.

\enquote{Etwas}, gab er matt zurück.

Dann begann er sich wieder daran zu erinnern.

\begin{rueckblick}
\enquote{Warum}, fragte Harry, \enquote{bist du ein Geist? Warum habe ich dich bisher nie gesehen?}

Salazar senkte seinen Kopf. \enquote{Als ich gestorben bin, entschied ich mich fürs Weitergehen. Aber vor kurzem hatte ich eine interessante Erscheinung.  Jemand sagte mir, dass du Probleme hast, dass du eine Aufgabe hast, eine Last. Und ich kann dir dabei helfen. Ich wurde gebraucht. Ich bin zurück, ja, aber nicht für ewig. Es braucht einige Zeit, bis ich länger in dieser Welt verweilen kann. Und nein, Harry. Ich kann dir keine Fragen über das Leben danach beantworten}, sagte er, als er Harrys nachdenkliches Gesicht sah. \enquote{Ich kann dir nichts über deine Mutter oder deinen Vater erzählen. Wenn ich in dieser Welt bin, weiß ich nichts mehr von der anderen. Und wenn mir doch etwas in den Sinn kommt, dann kann ich es nicht erzählen. Sobald ich anfangen möchte, vergesse ich es.}

\enquote{Das heißt, du kannst mir nichts über meine Eltern erzählen!}, folgerte er laut.

Salazar nickte.
\end{rueckblick}

Harry saß mit eingeknicktem Kopf da. Er zog seine Pantoffeln aus und zog die Beine zu sich heran auf das Sofa. Dann sah er Salazar wieder an.

\enquote{Schlaf etwas, mein Junge. Wenn du wieder aufgewacht bist, dann wird es dir besser gehen und du wirst dich mit deinen Freunden draußen Treffen}, sagte sein Urahn.

\enquote{Salazar?}

\enquote{Hmm!}

\enquote{Ich habe bisher mit keinem über dich gesprochen. Nur, dass ich dein Nachfahre bin.}

\enquote{Das ist gut so. Noch ist es dafür nicht an der Zeit.}

Salazar schwebte auf Harry zu und legte eine Hand auf seine Schulter.

\enquote{Ich wollte nicht, dass du\abs dass ich wieder von anderen\abs}

Doch dieses Mal war der Kontakt von Salazar herrlich warm. Die Wärme breitete sich in Harrys ganzem Körper aus und er spürte, wie er langsam müde und seine Glieder schwer wurden. Salazar verschwand und kurz bevor er nichts mehr wahrnahm, sah er undeutlich einen durchsichtigen, glühenden Umschlag in der Luft schweben.

Er dachte an gestern Abend, wie er die Unterhaltung über die Lampen hatte. \gedanke{Neue Lampen}, dachte er. \gedanke{Das ist es. Es ist ein Zauber. Und die neu hinzugekommenen Lampen wegen der Helligkeit sind mit einem anderen Zauber gemacht worden. Deshalb muss man sie ab und an auffrischen. Die Magie ist verbraucht.} Dann dämmerte er wieder weg und nahm nichts mehr wahr.

Erfrischt und gut ausgeruht wachte Harry nach einer halben Stunde wieder auf. Der Umschlag schwebte noch immer vor dem Sofa. Harry fragte sich, was er damit wohl anfangen könnte. Er versuchte ihn zu greifen, aber als er seine Hand danach ausstreckte wurde ihm klar, dass das so wohl nicht funktionieren würde und seine Hand griff ins Leere. Seine Nervensignale erreichten die Hand nicht mehr rechtzeitig. Er zog seinen Zauberstab und versuchte den Umschlag im Raum zu bewegen. Dann öffnete er den Umschlag auf magische Weise und ein Brief kam zum Vorschein. Nachdem er die wenigen Worte dort gelesen hatte, verschwand der Umschlag samt Brief.

Für Harry ging das viel zu schnell, aber der Spruch hatte sich schon in sein Unterbewusstsein gegraben. \gedanke{Mir wird er rechtzeitig wieder einfallen}, dachte er. Er ging in sein Zimmer, um sich ordentliche Schuhe anzuziehen. Dann kam ihm der Spruch wieder in den Sinn. \spruch{Ego Basiliskum per horam quinque iunctio tibi.}
% Ego Basiliskum   per   horam    quinque     iunctio        tibi
% Ich   Basilisk   für Stunden      fünf      Verbunden     mit dir  (du  tu, dir  tibi, es (ist)  est )

Er rannte förmlich, voll gesogen mit frischer Energie durchs Schloss und wäre fast mit Dumbledore zusammengestoßen.

\enquote{Entschuldigung, Albus} und als er merkte, was der hinter sich her schweben ließ, \enquote{gehen wir Schlitten fahren?} Harry biss sich auf die Zunge. Er wollte es nicht so klingen lassen und Dumbledore auffordern mit ihm Schlitten zu fahren.

Doch Dumbledore lachte nur und meinte: \enquote{Genau deswegen wollte ich zum Gryffindor-Turm. Du fehlst noch draußen. Gehen wir?}

Harry lachte seinen Schulleiter an und sagte nur: \enquote{Gerne! Aber nur, wenn ich ihn rauf Ziehen darf} und zeigte auf den Schlitten. Dumbledore lachte und ging neben Harry den Rest durchs Schloss und dann hinaus in die kalte, schneebedeckte Landschaft um Hogwarts. \gedanke{Das würde McGonagall wohl nie machen}, dachte er bei sich.

Draußen angekommen, sank der Schlitten in den Schnee und Dumbledore setzte sich hinter Harry auf den Schlitten, da der Weg zum See ständig abfallend war. Harry lachte die ganze Fahrt über. Bald hatte er das Gefühl, als ob er seine Gesichtsmuskeln danach nicht mehr bewegen konnte. Und auch Dumbledore schien hinter ihm sichtlich Spaß zu haben. Gerade fuhren sie an Hermine, Neville, Ron, Dean und einigen Ravenclaws und Hufflepuffs sowie zwei Slytherin vorbei die sich eine Schneeballschlacht lieferten. Ungläubig schaute die Gruppe sie an, als die beiden an ihnen vorbeifuhren. Unten angekommen, drehte der Schlitten um und fuhr noch schneller nach oben, als sie nach unten gefahren waren. Harry sah mit einem erstaunten Gesichtsausdruck auf die Gruppe, die sich eben noch mit Schneebällen beworfen hatten, als sie zusammenstanden und geschäftig miteinander diskutierten.

Harry hätte es sich denken können, als sie ein weiteres Mal auf ihrem Weg nach unten vorbeifuhren. Denn schon flogen Unmengen an Schneebällen auf sie zu, was dazu führte, dass der Schlitten kenterte und Harry mit Dumbledore im Schnee landete. Dumbledore zauberte noch eine kleine Schneewehe vor ihnen herbei, damit sie vor weiteren Angriffssalven geschützt waren, und begann sofort Schneebälle zu formen. Harry schaute ihn erstaunt an und begann gleich danach ebenfalls Bälle zu formen. Dann schossen sie zurück.

Es dauerte eine Weile bis Professor McGonagall angerannt kam und sie aufforderte, die Schlacht zu beenden.

\enquote{Bitte, bitte, hören Sie auf. Das ist nicht der richtige Ort für eine Schneeballschlacht.} Sie kam genau zwischen den Fronten zum Stehen. Dann traf sie ein Schneeball. \enquote{Mister Finnigan, ich muss doch sehr bitten}, sagte sie verärgert und streifte den Schnee ab.

\enquote{Entschuldigung Professor, aber Sie stehen im Weg}, antwortete er.

\enquote{Genau} und ein weiterer Schneeball traf sie und noch einer. Erschrocken drehte sie sich um und sah Dumbledore einen weiteren Schneeball aufnehmen und sich für einen Wurf vorbereiten. \enquote{Sie stehen einfach im Weg. Wenn Sie nicht getroffen werden wollen, dann gehen Sie aus der Schusslinie, Minerva.} Immer noch erschrocken schaute sie Dumbledore an, ungläubig, dass er sie gerade mit einem Schneeball beworfen hatte.

Doch das ließ Minerva McGonagall nicht auf sich sitzen. Sie zog ihren Zauberstab und formte einige Bälle. Dann steckte sie ihn weg und warf sie Richtung Harry und Dumbledore. Die beiden mussten sich ducken, denn Professor McGonagall konnte sehr gut und genau werfen und treffen. Als der Schauer vorüber war, war sie auch schon hinter dem Schutz, den Ron und Neville aufgebaut hatten, verschwunden und in Deckung gegangen. Nach einer halben Stunde mussten Harry und Dumbledore aufgeben, da sie eindeutig in der Unterzahl waren und ihnen langsam die Arme weh taten. Dumbledore nahm seinen Schlitten, setzte Professor McGonagall und sich darauf und fuhr mit ihr hinunter zum See. Die anderen folgten ihnen zu Fuß. Dort standen sie noch eine Weile schweigend da, bevor Professor Dumbledore mit Professor McGonagall zurück zum Schloss liefen. Harry hatte noch schwach in Erinnerung, dass sie ermahnt wurden kein Wort darüber zu erzählen, aber das war wohl eh nur eine halbherzige Drohung, da sich solche Sachen erfahrungsgemäß sehr schnell im Schloss verbreiteten. Und außerdem würden es sich die Schüler nicht nehmen lassen, das weiterzuerzählen.

\trenn

Professor Elber stand nach den Weihnachtsferien wieder vor seinem Pult und sagte: \enquote{Steht nun bitte alle auf und kommt hierher zu mir nach vorne.} Alle standen auf und kamen auf die Empore, auf der Professor Elber stand. \enquote{Und jetzt werfen Sie bitte ihre Zauberstäbe an die Wand da hinten, sodass sie auf dem Boden knapp davor zum Liegen kommen.} Viele Zauberstäbe flogen durch die Luft und auch Professor Elber warf seinen an die Wand. Es waren noch einige in der Luft, als sich plötzlich die Tür öffnete und Professor Dumbledore mit einem gut gekleideten Mann hereinkam.

\enquote{So\abs} fing Professor Elber gerade an, als die beiden hereinkamen.

\enquote{Ich hoffe doch, wir stören nicht}, meinte Professor Dumbledore, als er mit seinem Gast durch die Türe hereinkam und ihn ein Zauberstab knapp verfehlte.

\enquote{Nicht im Geringsten}, erwiderte Professor Elber. \enquote{Was können wir für euch tun?}

\enquote{Dieser Herr hier ist vom Zaubereiministerium und möchte einen Blick in unseren Unterricht werfen. Wir würden gerne ein bisschen zusehen.}

\enquote{Wollen \block{sie} meinen Unterricht diese Stunde führen?}, fragte Professor Elber freundlich. \enquote{Oder wollen sie vielleicht mitmachen? Oder schauen sie nur zu?}

\enquote{Oh nein, nein, ich beschränke mich auf’s zusehen}, meinte der Herr von Ministerium.

\enquote{Und sie, Albus?}, fragte Professor Elber.

\enquote{Ach, gerne. Ich mache gerne mit. Wird sicher lustig.}

Professor Dumbledore betrachtete die Zauberstäbe, die dort am Boden lagen und meinte dann nur lapidar. \enquote{Da muss ich wohl meinen dazulegen.} Ohne eine Antwort abzuwarten, griff er in seine Tasche und legte seinen Zauberstab dazu.

Dann ging er mit dem Mann vom Ministerium ebenfalls auf die Empore und stellte sich an die Seite.

\enquote{So}, fuhr Professor Elber fort. \enquote{Wir machen genau da weiter, wo wir gerade eben aufgehört haben. Nur mit dem Unterschied, dass die Distanz nun größer ist und wir statt dem \accentuate{Auf} nun ein \accentuate{Her} verwenden. Mit entsprechender Übung schaffen Sie das dann auch ohne direkte Worte. Ich fang’ mal wieder an.} Er streckte seine Hand in die ungefähre Richtung, in der der Zauberstab lag und rief dann: \enquote{Her.} Sein Zauberstab erhob sich und fand kurze Zeit später den Weg in seine Hand, die er schloss. \enquote{Bitte, jetzt sind Sie an der Reihe.}

Mehrere \accentuate{Her}’s erklangen durch den Raum und eine Menge Zauberstäbe erhoben sich, um kurz danach in der Hand ihres Besitzers zu landen, bei manchem erst nach dem zweiten oder dritten Versuch. Alle Schüler hatten ihren Zauberstab in der Hand, nur Professor Dumbledore stand noch ohne da.

\enquote{Beeindruckend}, meinte der Mann vom Ministerium. \enquote{Sehr beeindruckend. Unterrichten Sie schon lange hier?}, fragte er Professor Elber.

\enquote{Nein, das hier ist mein erstes Jahr; und vermutlich auch mein letztes. Ich bin nur hier, weil es zurzeit wohl keinen anderen gibt, der dieses Fach unterrichten möchte.}

\enquote{Ah ja}, kam es ihm vom Mann vom Ministerium entgegen. \enquote{Ich denke, ich habe hier genug gesehen. Wir können weiter gehen}, meinte der Mann zu Professor Dumbledore. Dieser ließ ihm den Vortritt und begleitete ihn Richtung Tür.

Jetzt bemerkte Professor Elber ein leises Kichern und Tuscheln in der Klasse. \enquote{Was ist denn so lustig?}, fragte er. Einer antwortete \enquote{Nur ein Zauberstab ist noch übrig und alle Schüler haben ihren. Und Sie haben Ihren auch noch in der Hand.}

Professor Elber betrachtete seinen Zauberstab, lächelte und sah dann zu Dumbledore, der seinen Blick auffing. Professor Elber hob seinen Zauberstab und zeigte auf ihn. Professor Dumbledore drehte sich um und da lag er noch, sein Zauberstab. Er lächelte zurück und zwinkerte Harry zu. Danach streckte er seine Hand aus und ohne ein Wort zu sagen, fand der Zauberstab den Weg in seine Hand. Professor Dumbledore verließ den Raum und schloss hinter sich die Tür.

Das leise Lachen war mittlerweile verstummt.

\enquote{Übt weiter}, meinte Professor Elber und setzte sich auf seinen Stuhl, um einen weiteren Eintrag ins Klassentagebuch zu machen.


\chapter{Auf und ab}


Er war auf dem Weg zum Astronomieturm und leider viel zu spät, um pünktlich zum Unterricht zu erscheinen. Der Weg hoch zum Turm war lang und anstrengend. Fast wäre er von hinten in Professor Elber und Professor Dumbledore gerannt. \enquote{Entschuldigung} sagte Harry, als er sie sah.

\enquote{Wohin so schnell?}, fragte Dumbledore ihn.

\enquote{Zum Astronomieturm}, antwortete Harry.

\enquote{Wir auch}, entgegnete ihm Dumbledore. \enquote{Lust ein Stück mit uns zu laufen? Wäre außerdem eine gute Entschuldigung fürs zu spät kommen!} Sein Schulleiter blitze ihn an.

\gedanke{Warum nicht}, dachte sich Harry und ging neben seinen beiden Professoren her.

\enquote{Du hast dich von deiner Freundin getrennt, habe ich gehört?}

Harry fragte sich, woher Dumbledore das schon wieder wusste. \enquote{Ja, es läuft gerade nicht so gut zwischen uns. Etwas Abstand tut uns ganz gut.}

Professor Elber bog plötzlich ab. Dumbledore und Harry sahen ihn nur staunend an.

\enquote{Zum Astronomieturm geht es da lang, Frederick}, sagte ihm Dumbledore.

Professor Elber drehte sich um und entgegnete dann: \enquote{Ich laufe doch nicht die ganzen Treppen, nur um Aurora diese Dokumente zu geben.}

Erst jetzt bemerkte Harry die große Mappe, welche Professor Elber in der Hand hatte. Professor Elber drehte sich wieder um und ging weiter. Dumbledore und Harry folgten ihm. Kurz bevor sie unter einem der vielen Torbogen, welche über das ganze Schloss verteilt waren, ankamen, blieb Professor Elber stehen und drehte sich nach rechts.

Harry kannte diese Torbogen genau. Für ihn erfüllten sie immer nur stilistische Zwecke. Wenn man unter so einem Torbogen stand und sich nach rechts drehte, waren in etwa 1,60 m Höhe beidseitig Steine, die aus der normalen Mauer herausragten und die gleiche Höhe zum Torbogen hatten. Auf der linken Seite war ein schmaler Stein und auf der rechten Seite ein breiter Stein. Drehte man sich im Torbogen um, dann war der schmale Stein natürlich auf der anderen Seite.

Professor Elber zählt vom breiten Stein fünf weiter und drückte diesen dann in die Wand. Er gab nach und nach ca. 5 Sekunden öffnete sich die Wand. Er schritt in den quadratischen Raum von ca. 4 Quadratmetern Grundfläche, drehte sich um und sah wieder nach draußen. Harry und Dumbledore waren überrascht, aber folgten brav und drehten sich ebenfalls um. Im Inneren waren nur glatte Steinwände zu sehen.

Nachdem sich Harry aber umgedreht hatte, bemerkte er auf etwa Augenhöhe einen großen glasierten Stein, der ein merkwürdiges Symbol zeigte. Auf Handhöhe war ein Mosaik abgebildet. Es bestand aus vielen kleinen bunten Steinen mit ebenso vielen unterschiedlichen Symbolen darauf. Es stellte keinerlei Figur oder Form dar. Professor Elber suchte kurz und drückte dann auf das Mosaik, welches Harry verdächtig an einen Turm mit nebenstehendem Fernrohr erinnerte. Der Stein versank ein Stück und blieb gedrückt in der Wand. Die Mauer schloss sich auf die gleiche Art und Weise wie sie sich schon zuvor geöffnet hatte. Harry erinnerte sich an seinen ersten Besuch mit Hagrid in der Winkelgasse zurück.

Die Wand hatte sich nun geschlossen und die drei standen schweigend im nun schwach erleuchteten Raum. Harry konnte keine Lichtquelle ausmachen, aber man konnte alles klar und deutlich erkennen. Der Boden fing leicht an zu vibrieren und das große Symbol auf Augenhöhe an der Wand begann seine Form, seine Farbe und sein Symbol beständig zu verändern.

Als es so aussah, wie eine vergrößerte Form des Symbols welches Professor Elber gedrückt hatte, hörte das Vibrieren des Bodens auf. Die Wand öffnete sich wieder und Professor Elber stieg aus und bog sofort nach links ab. Immer noch völlig erstaunt schauten sich Dumbledore und Harry an. Harry hatte diesen Zustand schneller überwunden und verließ ebenfalls den Raum, gefolgt von Dumbledore.

Nach wenigen Metern bog Professor Elber abermals nach links und jetzt eröffnete sich der runde Astronomieturm. Sie waren kurz vor dem Klassenzimmer angelangt und sahen nach unten. Von dort konnten sie die Stimmen der Schüler erkennen, die sich auf den Weg nach oben machten. Professor Elber öffnete auf der linken Seite die Tür und trat ein. Harry und Dumbledore folgten ihm. Harry warf seine Schultasche auf seinen Platz und schaute Professor Elber zu.

\enquote{Aurora \gst Aurora.} Nichts geschah. Dann sagte er nach einer kleinen Pause etwas, was Harry die Sprache verschlug. \enquote{Aurora Schätzchen.}

Es dauerte keine 5 Sekunden, da stand auch schon Professor Sinistra vor ihm und meinte mit leicht erröteten Wangen. \enquote{Frederick, Sie alter Charmeur.} Er überreichte ihr die Mappe, worauf sie sich bedankte und wieder verschwand. Professor Dumbledore betrachtete das ganze mit kindlichem Interesse. Professor Elber drehte sich zu Harry und meinte nur. \enquote{Wenn Sie nicht reagieren, dann muss man dem Ganzen etwas Nachdruck verleihen.} Dann verließ er den Raum und ging. Professor Dumbledore stellte sich an eines der Fenster des Turmes und sah verträumt nach draußen. Harry verließ das Klassenzimmer wieder und schaute die Wendeltreppe entlang nach unten, lehnte sich auf das Geländer und sah seinen Klassenkameraden zu, die sich noch immer die Treppe hoch quälten. Er musste schmunzeln. Noch hatte er keine Ahnung, dass er keine Minute zu früh von den Hogwarts-Aufzügen erfahren hatte. Zumindest wusste er jetzt von einem.

Die nächste Zeit würde er kein Wort darüber gegenüber seinen Klassenkameraden verlieren. Er wollte zuerst die Ziele erkunden.

Ron kam oben an und sah erstaunt in Harrys Gesicht. \enquote{Du warst doch hinter mir. Wie bist du\abs}, fragte er.

\enquote{Ich habe dich auf halber Strecke überholt Ron. Nicht mitbekommen?}, grinste Harry ihn an.

Ron sah ihn nur erstaunt an. Harry ging in das Klassenzimmer und setzt sich. Professor Dumbledore verließ den Raum, um den Unterricht nicht zu stören. Sie mussten wieder die Bahnen der Planeten berechnen. Dieses Mal die des Uranus. Alles verlief ruhig und die Schüler zeichneten ihre Bahnen und sahen immer mal wieder durch ihre Fernrohre. Plötzlich fiel Parvati von ihrem Stuhl und krümmte sich. Schaum quoll aus ihrem Mund. Sie zuckte, als hätte sie einen Anfall. \enquote{Professor}, rief Lavender. \enquote{Kommen Sie schnell.} Professor Sinistra kam angerannt und sah Parvati voller entsetzen an. Lavender sah nun durch Parvatis Fernrohr und fing kurz darauf ebenfalls an, dieselben Symptome zu zeigen. Auch sie fiel zusammen und begann mit Schaum vor dem Mund zu zucken.

\enquote{Jemand möge Madame Pomfrey holen}, schrie Professor Sinistra durch das Zimmer. \enquote{Sofort!}

\enquote{Ich gehe und hole sie}, rief Harry. Er wusste, dass er sie schneller als alle anderen erreichen würde.

Er stürmte nach draußen und schloss die Tür hinter sich. Dann bog er zweimal nach rechts ab und drückte den richtigen Stein neben dem Torbogen. Die Wand ging wieder auf und Harry trat ein. Er suchte nach einem bestimmten Symbol. Schließlich fand er ein rotes Kreuz neben einem roten Halbmond auf weißem Untergrund. Sofort drückte Harry darauf und die Wand schloss sich. Wenige Augenblicke danach öffnete sich die Wand und er verließ den Raum.
% Das Rote Kreuz auf weißen Grund ist in der westlichen Welt das Zeichn für Erste Hilfe. In der muslimischen Welt ist dies der Rote Halbmond

Harry blickte sich kurz um und rannte nach rechts, auf die großen Flügeltüren des Krankensaales zu. Hastig öffnete er sie und lief zu Madame Pomfreys Büro. Er riss förmlich die Tür auf.

In ihrem Büro saß Madame Pomfrey und ihr gegenüber Professor Dumbledore.

\enquote{Potter, etwas mehr Disziplin bitte}, maßregelte ihn Madame Pomfrey.

\enquote{Keine Zeit, Madame Pomfrey}, entgegnete er. \enquote{Wir haben einen Notfall im Astronomieturm. Lavender Brown und Parvati Patil liegen mit epileptischen Zuckungen und Schaum vor dem Mund auf dem Boden. Eventuell eine Vergiftung. Es fing an, nachdem sie durch das Fernrohr gesehen hatten. Erst Parvati, danach Lavender, als sie durch Parvatis Fernrohr schaute.}

Madame Pomfrey schnappte sich ihre Notfall-Tasche und verließ, ohne sich von Dumbledore zu verabschieden, die Krankenstation.

Sie wollte schon den üblichen Weg einschlagen, als ihr Harry geschwind hinterherlief und sie am Handgelenk packte. Er zog sie Richtung Torbogen, von dem er gerade eben gekommen war.

\enquote{Was tun Sie da Mister Potter}, reagierte Madame Pomfrey genervt. \enquote{Zum Astronomieturm geht es da lang.}

Harry ignorierte sie und griff nur umso fester zu. Er drückte den richtigen Stein und wartete kurz, während er ihr sagte: \enquote{Abkürzung.} Die Wand öffnete sich und Harry zog Madame Pomfrey hinter sich her. Harry drückte den Knopf für den Astronomieturm und die Wand begann sich zu schließen. Nach kurzer Fahrt waren die beiden auch schon oben abgekommen. Harry hatte sie noch immer an ihrem Handgelenk gefasst und zog sie nun um die Ecke zum Klassenzimmer. Als er die Tür geöffnet hatte, ließ er sie los und ihr den Vortritt.

Nachdem Madame Pomfrey beiden einen Bezoar gegeben hatte, beruhigten sich die beiden. In der Zwischenzeit hatte Professor Sinistra das entsprechende Fernrohr durch einen Schildzauber abgeschirmt. Die Unterrichtsstunde war zu Ende und die Klasse wurde zu Bett geschickt.

Am nächsten Morgen saß Harry zwischen Lavender und Parvati auf der Krankenstation. In einer Hand hielt er Parvatis, in der anderen Lavenders Hand.

Lavender wachte als erste auf und drückte Harrys Hand, da er abwesend durch den Raum blickte. \enquote{Oh Lavender}, sagte Harry. \enquote{Alles in Ordnung?}

\enquote{Ja}, meinte sie. \enquote{Du warst wahnsinnig schnell, hat man mir erzählt, als ich kurz aufgewacht bin. Danke Harry.}

Harry lächelte sie nur an. Dann spürte er einen Druck an seiner anderen Hand und drehte sich zu Parvati hin.

\enquote{Hallo Harry}, sagte sie sanft. \enquote{Danke für die Rettung.} Sie blickte kurz zu Lavender und gab Harry durch Handzeichen zu verstehen, sie möge ihm doch Bitte näher kommen.

Harry folgte der Aufforderung. \gedanke{Sie will mir wohl etwas sagen, das Lavender nicht hören darf}, dachte er.

Doch Parvati zog ihn zu sich und gab ihm einen Kuss auf seine Wange nahe seinem Mundwinkel. Harry war erstaunt. \enquote{Ein kleines Dankeschön}, flötete sie und lächelte ihn an.

\enquote{Harry}, kam es ihm von hinten entgegen.

\gedanke{Lavender, natürlich. Das lässt sie nicht auf sich sitzen}, dachte Harry. Er drehte sich um und sah, wie ihm Lavender die Arme entgegenstreckte. \gedanke{Das könnte interessant werden}, dachte Harry. Er ging zur ihr hin. Als sie ihn in ihre Hände bekam, zog sie ihn zu sich und gab ihm einen langen und feuchten Kuss auf den Mund. Dann ließ sie von ihm ab und funkelte wild zu Parvati hinüber.

Harry blickte zwischen beiden hin und her. Sie tauschten böse Blicke. Dann hatte Harry einen Einfall. Er ging wieder einen Schritt auf Parvati zu und meinte dann: \enquote{Willst du nachziehen, Parvati?}

\enquote{Was?}, gab sie erstaunt zurück, wobei sie ihn anschaute.

\enquote{Na ja}, antwortete Harry, \enquote{du hast mich \accentuate{fast} geküsst, Lavender ging einen Schritt weiter und \accentuate{hat} mich geküsst. \gst Möchtest du nachziehen, oder einen Schritt weiter gehen?} Nun schauten ihn beide Mädchen an. \enquote{Ich meine, so wie ihr euch angeschaut habt, könnte man doch meinen, ihr habt eine Wette am Laufen. Wer traut sich bei Harry Potter am meisten.} Damit, so hoffte er, hatte er entweder eine Lawine losgetreten, was ihm gefallen würde; andererseits könnte es aber den gegenteiligen Effekt haben. Das wäre Harry dann auch recht. Auf jeden Fall hatte er seinen Spaß.

Parvati griff nach ihrem Kissen und warf es Harry entgegen. Er fing es auf und begann zu lachen. Dann mussten Parvati und Lavender ebenfalls loslachen.

Harry gab Parvati ihr Kissen zurück. Sie legte es hinter ihren Kopf und setzte sich dann auf. Sie drehte sich und ließ ihre Beine vom Bett herunterhängen. Harry setzte sich neben sie auf die Bettkante. Lavender verließ ihr Bett und setzte sich neben Harry.

Stumm saßen sie nun zu dritt nebeneinander. Harry hielt wieder ihre Hände. Nach einigen Minuten sagte Harry dann: \enquote{Ich gehe dann mal wieder. Ich habe noch Unterricht.} Er drehte sich zu Lavender und gab ihr einen Kuss auf ihre Wange. Sie wurde leicht rot. Er drehte sich zu Parvati und wollte sie ebenfalls auf die Wange küssen. Doch sie drehte sich kurz vorher zu ihm und so traf er ihren Mund. Ihre Lippen waren zarter, stellte Harry mit Genugtuung fest.

Dann lehnte sich Parvati nach vorne und meinte zu Lavender: \enquote{Jetzt sind wir quitt.}

Harry stand auf und meinte nur: \enquote{Ich gehe dann mal, bevor das hier noch ausartet.}

Jetzt mussten beide Mädchen wieder lachen. Madame Pomfrey kam gerade aus ihrem Büro, als Harry auf dem Weg zur Tür war. Dann verließ er die Krankenstation und machte sich auf den Weg zur nächsten Stunde. Auf dem Weg dorthin kam er an einer Tür vorbei, die leicht geöffnet war. Professor Elber saß hinter einem Tisch und hielt ein Amulett in der Hand. Er konnte seinen Professor von der Seite aus sehen. Er blieb interessiert stehen. Hermine trat von hinten an ihn heran, um zu sehen, was Harry so interessant fand.

Dann sagte Professor Elber: \enquote{Ich bereue zutiefst, gib dich frei.} Es kam eine kleine blaue Kugel heraus, die in seinen Professor eindrang und verschwand. Dann brach er bewusstlos auf dem Tisch zusammen. Harry und Hermine rannten zur Tür, doch sie verschloss sich vor ihnen.

\enquote{Hermine, du versuchst die Tür zu öffnen, ich hole Madame Pomfrey}, sagte er zu Hermine. Er rannte um die nächste Ecke zu einem Torbogen und drückte den Stein. Die Wand öffnete sich und Harry merkte sich das angezeigte Symbol. Nachdem er den Stein für den Krankenflügel gedrückt hatte, suchte er das passende Symbol, um zurückzukehren. Doch die Strecke war nicht sehr lang.

Mit Madame Pomfrey angekommen, führte er sie in das Zimmer. Hermine hatte die Tür in der Zwischenzeit geöffnet und Madame Pomfrey untersuchte ihn. \enquote{Brauchen Sie uns noch?}, fragte Hermine, \enquote{wir haben Unterricht.}

\enquote{Nein}, entgegnete ihnen Madame Pomfrey. \enquote{Sie können gehen.}

\trenn

Auf dem Weg zur Großen Halle kam von hinten Katharina Chapel aus Slytherin an ihn heran. \enquote{Harry? Darf ich dich was fragen?}

Harry war erstaunt, dass eine Slytherin normal mit ihm sprechen konnte. Sie war attraktiv, hatte mittelblonde schulterlange Haare und war nur ein paar Zentimeter kleiner als er. \enquote{Ja. \gst Katharina, richtig?}, fragte er.

\enquote{Genau. \gst Ich\abs äh\abs wollte\abs}.

Harry ging der Valentinstag durch den Kopf. Wollte sie ihn vielleicht fragen, ob er mit ihr\abs? Aber das erschien ihm doch etwas zu abwegig.

\enquote{Ich\abs} sie atmete noch einmal durch. \enquote{Ich wollte dich fragen, ob du mit mir am Valentinstag nach Hogsmeade gehen willst!}

\gedanke{Also doch. Aber wie sage ich es ihr.} \enquote{Hör mir bitte zu Katharina.}

\enquote{Du willst nicht?} Ihre Augen wurden leicht feucht.

Harry griff sofort ein Taschentuch und wischte ihr leicht über die Augen damit. \enquote{Das habe ich nicht gesagt. Ich möchte nur nicht, dass es zu einem Missverständnis kommt.} Jetzt sah sie ihn erstaunt an. \enquote{Ich gehe gerne mit dir dorthin. Aber mehr als diesen Ausflug kann ich dir nicht bieten. \gst} Harry sah sich kurz um. \enquote{Sag das bitte niemandem, aber Luna und ich haben uns wieder versöhnt. Nur wollen wir noch etwas Abstand haben und es langsamer angehen. Ich bin also offiziell nicht gebunden.} Dann lächelte er sie an und fuhr über ihre Wange. \enquote{Ehrlich gesagt freue ich mich darauf. Das wird eine Menge Aufsehen erregen.}

\enquote{Das reicht mir schon}, antwortete sie. Harry gab ihr einen Kuss auf die Wange und wollte gerade gehen, als sie ihn am Handgelenk festhielt. \enquote{Du weißt schon, dass wir unser \accentuate{Date} mit einem echten Kuss beenden müssen?}

Harry zog eine Augenbraue hoch, begann dann aber zu lächeln. \enquote{Das bekommen wir dann hin, wenn es so weit ist. Wir haben ja am vierzehnten genügend Zeit zur Planung}, grinste Harry leicht. Katharina lächelte zurück und ging dann fröhlich weiter. Harry war der Meinung, ihr Grinsen zu sehen. Und das, obwohl sie ihm den Rücken zeigte.

\stimme{Braver Junge}, hörte er in seinem Kopf.

\gedanke{Luna, ich\abs}

\stimme{Geh ruhig hin mit ihr. Ich mag das Valentinsgehabe eh nicht. Und wenn es dir Spaß macht\abs}

Mitte der nächsten Woche, kurz nach dem Essen, als Hermine mit Ron und Harry im Schlepptau aus der Großen Halle kam, schrie plötzlich eine Schülerin aus Slytherin: \enquote{Schnell, holt jemand sofort Madame Pomfrey, Professor Elber ist schwer verletzt.} Ein wildes und hektisches Durcheinander herrschte Sekunden später im Vorraum und der Eingangshalle. Hermine sprang sofort zu Professor Elber und ging vor ihm auf die Knie. Kaum hatte ihn Madame Pomfrey von seinem Ohnmachtsanfall kuriert, lag er schon wieder verletzt im Schloss.

\enquote{Hermine}, keuchte Professor Elber, \enquote{Bibliothek\abs zwischen den Regalen\abs N-P\abs Posessium\gst Avada Kedavra.} Dann brach er zusammen. Hermine stand auf und rannte sofort zur Bibliothek, wo sie Madame Pomfrey begegnete, die auf dem Weg in die Eingangshalle war. Harry beobachtete, wie Madame Pomfrey, nachdem sie angekommen war, ihren Zauberstab zog und ein paar Worte murmelte. Funken sprühten aus ihrem Zauberstab und fingen an, den reglosen Körper Professor Elbers zu bedecken. Sie färbten sich grünlich. Madame Pomfrey zuckte zusammen und fing an zu zittern.

\enquote{Bringen Sie ihn bitte in die Krankenstation.} Leicht zitternd ging sie voraus, während einige Schüler, darunter auch Harry, Professor Elber in den Krankenflügel schleppten. Oben angekommen, legten sie ihn auf ein Krankenbett und traten zurück. Langsam bewegte sich Professor Elber und brachte nur ein \enquote{Severus, Vertretung, mein Buch, Büro} heraus, um gleich danach wieder regungslos liegenzubleiben.

\enquote{Was ist mit ihm?}, fragte Harry.

\enquote{Wenn ich das wüsste}, antwortete Madame Pomfrey.

\enquote{Ich hab es, Professor}, schrie Hermine, als sie in die Krankenstation stürmte. Abrupt blieb sie stehen, als sie bemerkte, dass er bewusstlos da lag. \enquote{Was ist mit ihm?}, fragte Hermine.

\enquote{Ich weiß es nicht}, antwortete Madame Pomfrey abermals.

Hermine betrachtete das Buch und schlug es auf. \enquote{Er hatte irgendwas von \inner{Avada Kedavra} gesagt.} Sie schlug im Inhaltsverzeichnis nach und bleib bei einer Zeile mit dem Eintrag \enquote{Wenn man von Avada Kedavra gestreift wird und überlebt hat\abs} stehen. Hastig schlug sie die Seite auf und gab ein leises Quieken von sich.
Sie überflog die Seite und gab das Buch dann an Madame Pomfrey.

\enquote{Hier, Madame Pomfrey, ich habe noch nie so einen komplizierten Trank gesehen.}

Madame Pomfrey nahm das Buch an sich und las aufmerksam die ganze Seite durch. \enquote{Oh weh. Das schaffe ich nicht alleine. Potter. Würden Sie bitte Professor Snape holen, ich brauche seine Hilfe.}

Harry verließ die Krankenstation und machte sich auf den Weg zum Kerker.

\enquote{Professor Snape\abs}, sagte Harry, als er dort angekommen war.

\enquote{Jetzt nicht Potter, ich habe zu tun.}

\enquote{\aabs Madame Pomfrey braucht Ihre Hilfe. Es geht um Professor Elber.}

\enquote{Was hat er denn angestellt?}

\enquote{Er liegt bewusstlos auf der Krankenstation und Madame Pomfrey braucht dringend ihre Hilfe bei einem Heiltrank.}

Misstrauisch beäugte Professor Snape Harry. Schließlich stand er auf und ging um seinen Schreibtisch herum. \enquote{Sie gehen voraus, Potter}, sprach er. Er verschloss sorgsam sein Büro und folgte Harry in den Krankenflügel.

Oben angekommen standen schon Professor Dumbledore und Professor McGonagall am Krankenbett.

\enquote{Worum geht es, Poppy?}, fragte Professor Snape. \enquote{Wie kann ich Ihnen helfen?} Madame Pomfrey reichte ihm das Buch und Professor Snape las sich aufmerksam die Seite durch. \enquote{Das dürfte eine ganze Weile dauern. Ich schätze so etwa eine Woche, bis wir alle Zutaten beisammen haben. Dann nochmal etwa eineinhalb Wochen bis der Trank fertig ist. Und was steht hier? Drei Wochen Genesungszeit mit 2/3 der Zeit totale Amnesie?}

\enquote{Kriegen Sie das hin?}, fragte Professor Dumbledore.

\enquote{Ich denke gemeinsam werden wir das schon schaffen}, sprach Professor Snape und dreht sich zu Madame Pomfrey. \enquote{Wo haben sie eigentlich das Buch her?}

\enquote{Das hat mir Miss Granger gegeben.}

Professor Snape drehte sich zu Hermine und schaute sie mit hochgezogener Augenbraue an.

\enquote{Bevor er zusammengebrochen ist, hat er mir gesagt, was ich aus der Bibliothek holen sollte.}

Und Harry fügte hinzu: \enquote{Er hat danach nur noch ihren Namen, Vertretung und etwas von einem Buch in seinem Büro gesagt.}

Snapes Gesicht war wie immer unergründlich. Genauso gut hätte Harry gegen eine Wand sprechen können.

Plötzlich fiel Harry der Samstagskurs ein. \enquote{Was ist mit unserem Samstagskurs?} Harry biss sich auf die Zunge, denn das wollte er eigentlich nicht sagen, denn seit einigen Wochen übten sie jeden Samstag in der Kammer ihre Zaubersprüche.

\enquote{Samstagskurse?}, fragte Snape mit einer Gleichgültigkeit, die nur von ihm kommen konnte.

Auch Professor Dumbledore und Professor McGonagall drehten sich um.

\enquote{Welche Samstagskurse?}, fragte Professor Dumbledore nach.

\enquote{Ich dachte, das hätte Professor Elber mit Ihnen ausgemacht?}, sagte Hermine. \enquote{Wir üben seit einiger Zeit Verteidigung gegen dunkle Künste in der Kammer. Während der normalen Unterrichtsstunden holen wir den restlichen Stoff nach, der uns seit einigen Jahren fehlt. Wir sind fast ganz durch mit aufholen, meinte Professor Elber.}

\enquote{In welcher Kammer?}, fragte Professor Dumbledore nach. Doch dann kam ihm die Erleuchtung. \enquote{Ihr meint doch nicht etwa die Kammer des Schreckens?}

\enquote{Doch}, antwortete Harry. \enquote{Wir haben sogar zusammen die ganzen Flüche und Schutzzauber entfernt und berichtigt.}

\enquote{Ich wusste schon, warum ich ihn als Lehrer für dieses Jahr wollte. Er beherrscht sein Handwerk}, meinte Professor Dumbledore.

\enquote{Severus, übernehmen Sie seine Stunden?}, fragte Professor Dumbledore.

\enquote{Kann ich machen, Schulleiter, aber was ist mit den Samstagsstunden?}

\enquote{Die werde dann ich übernehmen, falls Sie keine Zeit haben}, antwortete Dumbledore.

\enquote{Und, wie war's?}, fragte Ron, als er Harry auf einem der Gänge traf.

\enquote{Snape vertritt Elber in VgddK}, antwortete Harry. \enquote{Und bei dir?}

\enquote{Ich habe mein Valentinsdate klargemacht.}

\enquote{Hermine?}

\enquote{Ja. \gst Und du? Wen hast du gefragt?}

\enquote{Niemand.}

\enquote{Also gehst du nicht nach Hogsmeade?}

\enquote{Doch}, antwortete Harry und grinste Ron an.

\enquote{Hä. Aber wenn du niemanden gefragt hast, wie\abs? Oh.} Harrys grinsen wurde breiter. \enquote{\accentuate{Dich} hat jemand gefragt?}, fragte er dann. \enquote{Wer? Luna?}

Harrys Grinsen wurde nur noch breiter. \enquote{Das siehst du dann, wenn es so weit ist.}

Ron bettelte noch eine Weile, doch Harry ließ sich nicht erweichen. Natürlich würde es die Gerüchteküche anheizen, wenn Harry nicht mit seiner Freundin nach Hogsmeade ginge, sondern mit einer Slytherin. Doch ihm war es egal. Er erregte sowieso aufsehen. Egal, was er tat.

\trenn

Die Doppelstunde bei Professor McGonagall verlief angenehm entspannt, da die Tipps von Professor Elber, welche er vor seinem komatösen Zustand verbreitete, es allen leichter machten. Professor McGonagall war erstaunt, dass die Fortschritte der Klasse nun größer waren als sonst. Dieses Mal sollten sie Katzen verschwinden und an anderer Stelle wieder auftauchen lassen. Bereits nach der Hälfte der Stunde waren drei Viertel so weit und nach eineinhalb Stunden konnten alle ihre Katzen verschwinden lassen. Professor McGonagall war beeindruckt. Alle übten noch eine Weile weiter, bis die Glocke das Ende der Stunde einläutete. Harry packte, wie alle, seine Sachen zusammen und machte sich mit Ron und Hermine auf zur nächsten Stunde \fach{Verteidigung gegen die dunklen Künste}. Was würde sie alle die nächsten Wochen bei Snape wohl erwarten? Mit leicht flauem Magen ging er in das Klassenzimmer und wartete auf Professor Snape. Mit dem Läuten der Glocke kam auch schon Professor Snape herein und schritt mit schnellen Schritten durch die Reihen.

\enquote{Es mag sich vielleicht noch nicht bei allen herumgesprochen haben, aber Professor Elber ist für die nächsten Wochen unpässlich und er hat mich gebeten, seine Stunden zu übernehmen. Machen Sie weiter mit dem, was sie letztes Mal gemacht haben, ich werde in Kürze bei Ihnen sein.}

Er lief Richtung Büro und öffnete die Tür, nur um darin zu verschwinden und die Tür hinter sich zu schließen. Die Klasse übte währenddessen ihren Zauberstab über ihrer Hand schweben zu lassen, während er leicht rotierte.

Nach einiger Zeit kam Professor Snape wieder in die Klasse zurück und baute sich vorne auf.

\enquote{Die nächsten acht Wochen wird Ihr Unterrichtsstoff etwas anders sein. Ich sehe Sie haben bereits alle relevanten Dinge durchgenommen. Ich habe Professor Elbers Absichten für die nächste Zeit studiert und werde sie fortführen. Die samstäglichen Übungskurse wird Professor Dumbledore leiten.}

Ein Murmeln und flüstern ging durch die Reihen. Vereinzelt fielen Zauberstäbe herunter und prallten auf die Tische und den Boden.

\enquote{Würden Sie Ihre Zauberstäbe wieder aufheben? Wir brauchen sie für Ihre nächste Übung}, fuhr Professor Snape sie an.

Es herrschte ein Stimmen Wirrwarr im Raum und viele \accentuate{Auf}'s klangen durch das Zimmer. Nachdem alle wieder ihre Zauberstäbe in ihren Händen hielten (keiner von ihnen hatte sich gebückt), fuhr Professor Snape fort.

\enquote{Bevor ich es vergesse. Morgen wird Ihre Unterrichtsstunde auf dem Schlossvorplatz stattfinden.} Abermals ging ein Murmeln und Staunen durch die Klasse, doch Professor Snape ließ sich dadurch nicht beeindrucken und aus der Ruhe bringen. \enquote{Halten Sie Ihre Zauberstäbe bereit und machen Sie mir die folgende Bewegung nach.} Professor Snape vollführte eine knappe und kurze Handbewegung.

\enquote{Wichtig ist, dass Sie dabei an das zu schützende Objekt denken müssen. Also meistens sich selbst. Wir nehmen heute einen Illusionierungszauber durch}, sprach Professor Snape, schwang seinen Zauberstab und war kaum noch zu sehen. Er lief durch den Raum und man konnte schemenhafte Umrisse erkennen. \enquote{Dieser Zauber verbirgt nicht bewegliche Teile komplett. Deshalb müssen Sie ruhig stehen, wenn Sie sich mit diesem Zauber tarnen wollen. Üben Sie alle mal. Tun Sie sich in Gruppen von zwei Personen zusammen und prüfen dann das Ergebnis Ihres Gegenübers. Der Spruch lautet: \spruch{Desuillusio.}}

Die Schüler drehten sich so, dass sie ihren Partner ansehen konnten, und sprachen den Zauber auf sich selbst. Harry war fast sofort verschwunden. Es dauert nur etwas länger, bis er durchsichtig war, aber sein Schutz war dem Professor Snapes ebenbürtig. Es machte eine Menge Spaß, sich so zu tarnen. Leider durfte man sich nicht bewegen, sonst wurde man entdeckt. Man musste statuenhaft dasitzen. Harry kam eine Idee und er belegte sich mit einem Zauber, den er gedanklich sprach und der seinen Körper vor versehentlichen Bewegungen schützte. Seine Kleidung wurde über seinem Bauch fester, sodass er sich nicht durch Atembewegungen verraten konnte. Das einzige, was man noch sehen konnte, waren zwei kleine Verzerrungen, in Augenhöhe, wenn er blinzelte. Ron teilte ihm das mit, worauf hin Harry daran dachte, seine Augendeckel transparent zu machen. So konnte er weiterhin blinzeln, aber nicht mehr gesehen werden. Er war nun komplett unsichtbar. Ron stach ihm in die Backe, als er ihn suchte.

\enquote{Au}, kam aus dem Nichts und Harry wurde wieder sichtbar. \enquote{Nicht so grob, Ron}, meckerte er.

Kurz vor dem Ende der Stunde behielt Professor Snape, Harry und Ron bei sich. \enquote{Sie kommen morgen früh zu mir ins Büro. Sie werden mir helfen, die benötigten Gegenstände auf den Vorplatz zu schaffen.} Harry und Ron nickten nur.

Abends im Gemeinschaftsraum fragte Hermine Harry: \enquote{Meinst du, er ändert den Lehrplan? Oder hält er sich an den Professor Elbers?}

\enquote{Zumindest hat er das mal gesagt}, meinte Ron, der gerade von oben herunterkam.

Harry stimmte ihm zu. \enquote{Ich meine auch, dass er ihn in großen Teilen übernimmt. Und scheinbar verläuft der Unterricht bei ihm besser, als Zaubertränke.} Hermine und Ron nickten nur. \enquote{Aber was mich am meisten interessieren wird, sind die Samstagskurse bei Professor Dumbledore.}

Hermine nickte nur wieder.

\enquote{Meinst du, er weiß, wo der Eingang ist, oder sollen wir ihn hinführen?}, fragte Ron.

\enquote{Fragen wir ihn doch Morgen beim Frühstück}, meinte Hermine.

\trenn

Als Hermine am nächsten Tag mit dem Frühstück fertig war, stand sie auf und ging vor zum Lehrertisch. Harry und Ron folgten ihr. \enquote{Professor Dumbledore?}

\enquote{Ja, Hermine.}

\enquote{Sollen wir Sie am Samstag nach dem Frühstück abholen und zur Kammer führen?}

Einige um sitzende Lehrer bekamen große Augen und konnten nicht mehr schlucken.

\enquote{Gerne, ich habe nämlich keine Ahnung, wie ich da hinkomme.}

\enquote{Dann bis Samstag}, sagte Harry und verabschiedete sich. Er schnappte nur noch ein: \enquote{Welche Kammer?}, von Professor Sprout auf, bevor er außer Hörweite kam.

Harry, Ron und Hermine grinsten.

Am nächsten Tag, in Professor Snapes Büro angekommen, stand dieser auch schon auf und gab Ron und Harry zu verstehen, ihm zu folgen. Sie gingen weiter hinunter als sonst und schienen einer fast endlos langen Treppe zu folgen. Unten angekommen führte ein Gang einige Meter geradeaus. Doch plötzlich war er zu Ende.

\gedanke{Sackgasse, na toll. Snape hat sich verirrt}, dachte Harry.

Professor Snape griff in seine Tasche und zog einen Zettel heraus. Er murmelte einige Worte und zeigte mit seinem Zauberstab auf die Wand vor ihm. Dann fuhr er wie auf einem Muster die Ziegelsteinfugen ab. Die Wand teilte sich und gab eine Tür frei. Professor Snape öffnete sie und trat ein. Harry und Ron folgten ihm. Er drehte sich um und meinte dann: \enquote{Sie beide sind mir für das Arbeitsgerät verantwortlich. Sie bringen es, falls wir es brauchen, zum Unterricht und räumen es hinterher wieder auf.} Er drehte sich wieder um und suchte eine Kiste. Als er sie fand, zeigte er darauf und deutete den beiden an, sie mitzunehmen. Er verschwand und ließ Harry und Ron alleine.

Harry näherte sich und las die Aufschrift: \accentuate{Konzentrations- und Vertrauensscheiben.}

\enquote{Was das wohl heißen mag?}, fragte Ron.

\enquote{Keine Ahnung}, entgegnete Harry. Er zog seinen Zauberstab und rief: \enquote{Locomotor Kiste}. Sie fing an, leicht zu schweben. Harry griff sich eine Seite der Kiste und Ron die andere. Ohne Mühe trugen sie die Kiste hinaus und schlossen die Tür.

Als sie oben ankamen, warteten schon alle Schüler und Professor Snape fing mit dem Unterricht an.

\enquote{Heute werden Sie sich mit diesen Teilen auseinandersetzen.} Er zeigte auf die Kiste und Ron öffnete sie.

\enquote{Sie stellen sich nebeneinander auf und jeder nimmt sich eine Scheibe aus der Kiste. Es sind gleich viele schwarze wie weiße.}

Jeder der Schüler trat vor die Kiste und nahm sich eine der Scheiben heraus. Professor Snape machte weiter.

\enquote{Wenn Sie sich nachher auf Ihre Scheiben stellen, werden diese etwas vom Boden abheben und zu schweben beginnen. Sie werden automatisch Ihren Partner finden. Er wird also keine bewusste Auswahl geben. Treten Sie nun alle auf Ihre Vertrauensscheiben.}

Alle Schüler traten auf ihre Übungsscheiben, worauf diese zu schweben begannen. Nacheinander machten sich die Scheiben auf und suchten ihren Partner unter der Menge. Auch Harrys Scheibe fing an sich zu bewegen und machte vor Dracos Scheibe halt. Beide bekamen nur große Augen und machten einen ziemlich unglücklichen Gesichtsausdruck.

\enquote{Wenn Sie inzwischen alle Ihre Partner gefunden haben, dann fangen wir jetzt an.} Er zog wieder seinen Zettel aus der Tasche, nahm seinen Zauberstab und sagte: \enquote{Macrone. Astate. Lewioss.} Die Scheiben färbten sich grün und begannen etwas höher zu schweben. \enquote{Wenn ich den Startschuss gebe, fangen Sie an sich zu konzentrieren, um die Scheibe ihres Partners auf etwa gleicher Höhe wie Ihre eigene zu halten. Sie dürfen maximal um einen dreiviertel Meter abweichen, sonst verlieren beide ihren Kontakt und stürzen zu Boden. Halten Sie sich bereit. \gst Amigosa!}

Einige Scheiben zuckten gefährlich und es schien so, als ob einige die gefährliche Grenze des Abstandes erreichten. Draco fing an zu Grinsen, aber Harry kümmerte das wenig.

\enquote{Du stürzt genauso ab wie ich. Falls du also etwas vorhast, lass es lieber bleiben.}

Dracos Augen verengten sich, aber er verstand. Notgedrungen mussten beide diese Stunde gemeinsam überstehen und konzentrierten sich auf die Scheibe des anderen, um den Abstand nicht unnötig groß zu halten. Von Zeit zu Zeit ging ein Impuls durch die Scheiben und sie bewegten sich auseinander. Mal in der Ebene voneinander weg; Mal in der Höhe.

\enquote{Passen Sie auf, wenn sich die Farbe Ihrer Scheibe verändert heißt das, dass Sie den Abstand immer kleiner halten müssen. Es gibt noch die Farben gelb (50 cm), blau (10 cm) und rot (2 cm). Konzentrieren Sie sich und sprechen Sie sich ab. Hier ist Teamfähigkeit gefordert. Wir werden das die nächsten paar Male durchführen. Sie werden dann zusätzlich ihre Augen verbunden bekommen. Ach und noch eines. Ihre Partner werden wohl die gleichen bleiben.} Professor Snape schaute noch einmal auf seinen Zettel.

Harry war gar nicht wohl. Am liebsten hätte er sich jetzt in der Krankenstation gemeldet. Aber da musste er wohl durch. Noch ein paar Stunden mit Malfoy.

Der Rest der Stunde verlief nicht gerade angenehm, da Malfoy ihn immer wieder versuchte aus seiner Konzentration zu werfen, doch als die Farbe seiner Scheibe sich plötzlich ins Gelbe zu verändern anfing, riss er sich zusammen und verringerte schnell den Abstand von Harrys Scheibe zu seiner.

Diese Scheiben hatten etwa einen Meter im Durchmesser und eine Dicke von ca. drei Zentimetern. Es schien so, als wären sie aus einer Art flüssigem Metall gemacht, denn sie schimmerten und gaben leicht nach, wenn man sie bestieg. Es war so, als würde man auf Wasser laufen. Jeder Tritt gab einen kleinen Ring von sich, der mit dem des anderen Fußes Interferenzen bildete.

Die Glocke läutete und alle machten sich auf zur nächsten Stunde, nachdem sie ihre Scheiben auf den Boden schweben gelassen hatten. Harry und Ron verstauten noch den Koffer mit den zurückgelegten Scheiben und trafen zur nächsten Stunde 5 Minuten verspätet ein.

Die Kräuterkunde-Stunde bei Professor Sprout verlief dieses Mal nicht ganz so gut, denn Harry konnte sich irgendwie nicht mehr Konzentrieren. Es schien so, als ob er seine ganze Konzentrationskraft während der Stunde bei Snape aufgebraucht hatte. Dauernd schnitt er bei seinem Setzling daneben, worauf dieser sich lautstark beschwerte. \enquote{Tut mir leid}, beschwichtigte Harry ihn immer wieder, doch das half auch nicht weiter. Am Ende der Stunde war sein Setzling total verhunzt und Madame Sprout musste ihn bis nächstes Mal wieder aufpäppeln. Harry hatte erst mal Hunger und setzt sich nach dem Essen auf seinen Platz bei Professor Binn. Er wartete, bis er durch die Tafel schwebte und nahm eine Schlafposition ein. Er fühlte sich so, als könnte er neue Kraft schöpfen. Doch die Stunde war schnell wieder vorbei. Ron weckte ihn und so ging er hinter ihm und Hermine her, um zu Hagrid zu gehen. Heute hatte er wenig zu tun, da sich sein Pflegetier bester Gesundheit freute.
Ron unterhielt sich etwas mit Hagrid und Hermine wohnte dem Gespräch bei.

\enquote{Rosalie. Nein}, rief Harry, als sich das Tier Malfoy näherte. Er hatte das Gefühl, sie würde ihn schwer verletzen.

Draco Malfoy drehte sich um und erschrak, als er eine warme feuchte Zunge auf seiner Backe spürte. Draco fasste sich schnell wieder und sagte dann: \enquote{Rosalie, magst du mich?} Zuerst nahm er sie nur leicht schemenhaft wahr, doch schließlich wurde ihre Gestalt immer deutlicher. \enquote{Ein nettes Haustier hast du da. Nimmst du es mit nach Hause, damit es deiner Familie das letzte Hemd wegfrisst?} Scheinbar verstand Rosalie, dass es eine Beleidigung für Harry war, denn sie biss ihn leicht ins Ohr, welches anfing zu bluten. \enquote{Au}, rief Malfoy. Hagrid kam mit wenigen Schritten herüber, sah sich das Ohr kurz an und machte einen Verband daran. Draco konnte gar nicht so schnell reagieren.

\trenn

Nervös ging Harry am Freitagabend zu Bett. Morgen war es so weit. Gleich nach dem Frühstück würden sie wieder in der Kammer des Schreckens ihre VgddK-Übungen haben. Und dieses Mal würde Professor Dumbledore sie führen und anweisen.

Die Meisten der anderen waren mit dem Frühstück schon fertig, als Hermine zu Harry sagte: \enquote{Iss auf, dann nehmen wir Professor Dumbledore mit.}

\enquote{Ja, ist gut}, antwortete Harry und schluckte seinen letzten Bissen Toast herunter. Er trank seinen Becher aus und zu zweit gingen sie vor zum Lehrertisch. Professor Dumbledore bemerkte sie und stand ebenfalls auf. Ron war noch kurz zur Bibliothek gelaufen, um ein Buch für seine Hermine zu holen, da sie es noch unbedingt haben wollte, und war vorausgelaufen. Dumbledore umrundete den Tisch und kam ihnen entgegen.

\enquote{Bereit?}, fragte Hermine.

\enquote{Um ehrlich zu sein, bin ich etwas nervös}, antwortete Dumbledore. \enquote{Aber sagt es nicht den anderen.}

Sie führten Professor Dumbledore in den dritten Stock wo sich das Mädchenklo mit der maulenden Myrte befindet. Hermine öffnete die Tür und ging voran, dicht gefolgt von Professor Dumbledore.

\enquote{Uhh, Harry}, kam es ihm wieder entgegen. \enquote{Heute werde ich dir so richtig einheizen}, meinte Myrte. \enquote{Frederick hat mir neulich ein paar tolle Tricks und Kniffe gezeigt. Ich werde euch ALLE dran kriegen.} Plötzlich fiel ihr Blick auf Professor Dumbledore. \enquote{Ach, was wollen Sie denn hier?}

\enquote{Er wird uns heute begleiten, Myrte. Professor Elber ist auf der Krankenstation. Er kann die nächsten Male nicht.}

\enquote{Uaaah}, entfuhr es aus Myrte. \enquote{Das kann er mir doch nicht antun. Ich sollte heute doch meinen großen Auftritt haben.}

\enquote{Ich denke, es läuft alles so, wie es sich Professor Elber gedacht hat}, meinte Professor Dumbledore. \enquote{Und wenn das beinhaltet, dass Sie sich heute mit den Schülern beschäftigen, dann wird das wohl so sein. Außerdem habe ich keinerlei Informationen, was ihr hier bisher gemacht habt. Ich kann euch momentan lediglich zusehen und etwas anleiten, beziehungsweise euch helfen}, fügte er hinzu.

\enquote{Dann kommen Sie mal, Professor}, sagte Hermine, als sie vor dem Zugang stand, sich umdrehte und verschwand.

Ron ging ebenfalls in Richtung Eingang und folgte Hermine, da er gerade zur Tür hereinkam.

\enquote{Ich bleibe dicht hinter dir, Harry}, meinte Myrte. Harry schmunzelte etwas verkrampft und schwang sich in die Rohre. Unten angekommen, kam lange Zeit nichts. Als sich die vier schon zu wundern begannen, hörten sie plötzlich etwas.

\enquote{Huuiiii. Klasse.} Unten angekommen meinte Professor Dumbledore nur: \enquote{Hat irrsinnig Spaß gemacht. Hier sollten wir öfter mal herkommen.}

\enquote{Wir sind hier jede Woche}, meinte Ron.

Dumbledore folgte Harry und Ron durch die Gänge in die Kammer. Hermine war bereits vorausgelaufen, um die Tür zu öffnen. Sie musste die entsprechenden Parselworte lernen, um dies zu tun.

Drinnen angekommen meinte Professor Dumbledore nur: \enquote{Beeindruckend.} Er folgte Harry und Ron an den Kopf der Kammer, sah sich weiter um und sagte dann. \enquote{Tja, dann machen Sie mal. Ich schaue Ihnen zu.}

Er zauberte sich einen bequemen Sessel her und setzt sich darauf, seinen Zauberstab in einer Hand festhaltend. Die Schüler schauten sich nur verwundert an, als Hermine die Initiative ergriff und zu Harry meinte: \enquote{Was machen wir jetzt?}

\enquote{Was schaust du mich so an?}, fragte Harry.

\enquote{Du hast uns doch letztes Jahr unterrichtet, also mach was draus. Du hast von uns allen am meisten Erfahrung von uns.}

\enquote{Äh ja}, sagte Harry.

Myrte schwebte von hinten an Harry heran, legte ihren Kopf auf seiner Schulter ab und machte ein glückliches Gesicht. Die anderen Schüler schmunzelten.

\enquote{Ihr wisst doch sicher alle, was Professor Elber das letzte Mal gesagt hatte. Wir sollen heute den Stoff vom letzten mal wiederholen. Nächstes Mal fangen wir dann mit den Sachen an, die wir am Mittwoch im neuen Kapitel lesen. Stellt euch alle zu kleinen Gruppen zusammen.}

Alle Schüler stellten sich auf und machten sich bereit.

Harry zog seinen Zauberstab und fing an bunte wunderbar leuchtende Blitze auf die Gruppen zu schleudern. Die kleinen Gruppen, deren Mitglieder mit den Rücken zueinander gewandt waren, bildeten einen Schild um sie herum und lenkten so die Blitze ab. Als die Blitze die ersten Gruppen trafen und von den Schilden abprallten, hörte Harry auf und gesellte sich zu seiner Gruppe. Dieses Mal hatte er Neville und Dean Thomas in seiner Gruppe.

Die Blitze prallten von den Schilden der Gruppen und von den Wänden der Kammer ab. Sie sausten kreuz und quer durch die Kammer und kamen praktisch von allen Seiten. Ab und zu beobachtete Harry Professor Dumbledore, der begeistert in seinem Sessel saß und immer mal wieder einen bunten, schwächer werdenden Blitz zurückschleuderte, der auf ihn zukam. Langsam klangen die Blitze ab und verschwanden mit der Zeit. Die Gruppen lösten sich wieder auf und alle bildeten automatisch einen Kreis.

\enquote{Malfoy?}, fragte Harry. \enquote{Fängst du an?}

Malfoy antwortete nicht, sondern fing nur an einen weißen, hellen Blitz auf Harry abzufeuern. Dieser nahm ihn mit seinem Zauberstab auf, lenkte ihn um und schickte ihn weiter an Millicent Bulstrode. Diese wiederum schickte ihn weiter an Parvati Patil, welche ihn auf Neville losließ. Als der Blitz anfing schwächer zu werden, nahm ihn der nächste an und lenkte ihn in ein kleines Loch in der Decke. Kurz darauf erzeugte er einen neuen und das Spiel ging von vorne los. Nach einigen Runden fragte Harry Professor Dumbledore: \enquote{Wollen Sie nicht doch auch mal mitmachen?}

\enquote{Ach \gst warum nicht}, antwortete Dumbledore, stand auf und meinte: \enquote{Ich habe noch einen kleinen Tipp für euch.} Er trat auf die Gruppe zu und sagte zu Harry: \enquote{Schleudere nochmal so einen Blitz auf mich zu und ihr anderen, passt auf und haltet eure Zauberstäbe bereit.}

Harry begab sich in Position und schleuderte einen Blitz auf Professor Dumbledore. Dieser wollte ihn aufnehmen, doch es schob ihn zurück und schmiss ihn fast um.

Alle lachten, besonders Malfoy.

\enquote{Upps, tut mir leid Professor}, meinte Harry.

\enquote{Schon in Ordnung}, entgegnete ihm Dumbledore, \enquote{ich war nur nicht darauf vorbereitet. Was war das denn für ein Blitz?}, fragte Professor Dumbledore.

\enquote{Ein Pulsierender}, meinte Harry lapidar. \enquote{Hat uns Professor Elber beigebracht. Er meinte, wenn wir den abwehren können, sind wir gut bedient. Damit schaffen wir 90 Prozent aller unserer Gegner.}

\enquote{Da hat er wohl leicht untertrieben. Das ist hohe Magie. Also nochmal.}

\enquote{Wie meinen Sie das}, fragte Draco Malfoy nach.

\enquote{Nun ja, diese Art von Blitze können nur von sehr mächtigen Magiern erzeugt werden. Anscheinend ist Harry\abs}, doch plötzlich hörte er auf und begann seine Stirn zu runzeln. \enquote{Habt ihr diese Blitze vorher auch schon gehabt, als ihr im Kreis gestanden seid?}

\enquote{Ja}, antwortete Hermine. \enquote{Jeder von uns musste schon einmal einen erzeugen. Zuerst fiel es uns schwer, aber mittlerweile hat es jeder von uns drauf.}

Professor Dumbledore sagte nichts. \enquote{Äh ja. Ich war vorhin bei einer kleinen Demonstration. Also Harry, noch einmal bitte.}

Harry begab sich in Position und schleuderte wieder einen Blitz auf Professor Dumbledore. Der nahm ihn auf, teilte ihn und schleuderte ihn auf zwei verschiedene Schüler, die ihn sofort neutralisierten.

\enquote{Ich dachte mir, dass ich euch das heute noch zeige und nächsten Samstag könnt ihr das dann wiederholen und euren neuen Stoff aus dem Buch dazu machen. Kann ich mir von jemand das mal ausleihen?}, fragte Professor Dumbledore.

\enquote{Ja, von mir}, sagte Parvati. \enquote{Ich lerne sowieso immer mit Lavender zusammen. Da können wir eines eine Weile entbehren.}

\enquote{Und wann komm' ich dran?}, fragte Myrte.

\enquote{Jetzt!}, antwortete Harry.

Darauf schien Myrte gewartet zu haben. Sie stieg hoch und wollte bereits anfangen, als sie mitten in ihrer Bewegung stehen blieb und aus ihrer Hand kleine Kügelchen auf die Schüler losließ. Plötzlich stürzte sie herab und fing an durch die sterblichen hindurchzufliegen. Das war alles andere als angenehm. Es war schon so nicht angenehm, wenn einen ein Geist durchquerte, aber zu erleben, wie eine ekstatische Myrte durch einen glitt, toppte alles. Es war so, als ob man fror und gleichzeitig einer großen Hitze ausgesetzt war. Nach einer viel zu anstrengenden Stunde war Harry zu müde zum Abendessen.

Bevor er ins Bett ging, bemerkte Harry eine Eule an einem der Fenster. Ron war schneller und öffnete es. Die Eule flog zielstrebig zu Harry und ließ den Brief vor Harry fallen. Danach drehte sie gleich wieder ab und flog durch das Fenster hinaus in den dunklen Nachthimmel. Ron schloss das Fenster und ging wortlos ins Bett. Harry öffnete den Brief, nachdem er sichergestellt hatte, dass ihm keiner zusehen konnte.

\begin{brief}
Lieber Harry,

ich hoffe, du bist mir nicht böse, aber ich habe vor wenigen Minuten mit jemand geredet, der die Menge, die Morgen sicher am Eingang wartet und sehen will, mit wem du dich triffst, schocken wird. Sie ist eine gute Freundin und wird auf jeden Fall dicht halten. Warte bitte pünktlich zum vereinbarten Zeitpunkt am Haupteingang zum Schloss. Pansy wird sich kurz neben dich stellen, bevor sie dann zu ihrem Date geht und ich dann komme.
\signumspace
Liebe Grüße und Bussi

Katharina
\end{brief}


Harry ließ beinahe den Brief fallen. \gedanke{Pansy? Sie hat es Pansy erzählt?} Harry ließ sich auf sein Bett sinken. \enquote{Ausgerechnet sie.} Er schloss den Brief in seinen Koffer ein und legte sich unruhig ins Bett.

\begin{traum}
Harry lief gedankenverloren durch das leere Schloss. Es störte ihn nicht, dass er alleine war. Er setzte sich in die Große Halle und begann zu Essen, als sich ihm gegenüber Salazar Slytherin setzte und mitaß. Plötzlich war die Große Halle voll von frühstückenden Schülern und selbst der Lehrertisch war wieder besetzt. Salazar sah ihn lange an und meinte dann zu Pansy Parkinson, welche neben ihm saß: \enquote{Und, was meinst du? Hat er Angst?}

\enquote{Ja, ganz eindeutig. Er hat Angst.}

\enquote{Wovon redet ihr überhaupt}, wollte Harry wissen.

\enquote{Von dir, Harry.}

\enquote{Lass mich in Ruhe, Parkinson.}

\enquote{Na na na, wie redest du denn mit deiner weitschichtigen Verwandtschaft.}

Harry verschluckte sich an seinem Bissen und Pansy musste ihm auf den Rücken klopfen, damit der Speisebrocken hinunter in seinen Magen rutschte.

\enquote{Danke \gst Pansy.}

\enquote{Keine Ursache, Harry. Für dich mache ich das doch gerne.}

Jetzt erst sah er zu Pansy hinüber und war nicht im Entferntesten darüber erstaunt, dass sie eine Gryffindor-Uniform trug. Da in der Halle niemand außer ihnen dreien war, besah er sie eine Weile. \gedanke{Ich mag sie}, dachte sich Harry. \enquote{Was steht heute auf dem Stundenplan?}

\enquote{Ihre beide habt jetzt Tränke, Wahrsagen, Verwandlung, Zauberkunst und heute Abend Astronomie.}

Harry nickte und frühstückte vorsichtig weiter.
\end{traum}

Danach schlief Harry entspannter weiter und wurde durch das Zurückziehen seiner Vorhänge durch Kreacher geweckt.

\enquote{Guten Morgen, Sir Harry}, kam es vom alten Elf.

\enquote{Guten Morgen, Kreacher}, antwortete Harry. \enquote{Was ist los?}

\enquote{Sie haben heute einen wichtigen Termin, Sir.}

\enquote{Woher\abs?}

\enquote{Kreacher wäre ein schlechter Elf, wenn er die Termine seines Herrn nicht kennen würde. Kreacher hat sich erlaubt, Ihre Sachen selbst vorzubereiten, die Sie heute Nachmittag tragen werden. Kreacher hat außerdem noch ein spezielles Parfum hergerichtet.} Er zeigte auf Harrys Nachttisch. \enquote{Es wirkt nur bei Ihnen, Sir.} Kreacher verbeugte sich und sagte dann abschließend, bevor er verschwand: \enquote{Kreacher wird nun seinen Dienst in der Küche weiterführen.}

\trenn

Obgleich es draußen schneite, strahlte die Krankenstation in Hogwarts eine Wärme aus, die den Kranken dort helfen sollte, schneller zu genesen. Neben dem seit einigen Wochen in einem komatösen Zustand liegenden Professor Elber, lagen dort noch ein paar andere Schüler von Hogwarts. Ein paar Zauber gingen schief und so waren einigen Eselsohren, Hirschgeweihe oder Flügel gewachsen.

Madame Pomfrey hatte alle Hände voll zu tun. Die halbe Krankenstation war mal wieder voll von Schülern. Jedes Mal, wenn man ihr einen brachte, machte sie ein vorwurfsvolles Gesicht, stellte ein paar Fragen und begann dann mit der Arbeit. Sie wollte nicht wissen, wie das genau geschah, sondern nur, was das verursacht hatte. Sie stellte nie zu viele Fragen. Fragen die manchen Schüler in eine peinliche Lage gebracht hätte. In dieser Zeit geschah es auch, dass Professor Elber sich auf den Rücken drehte und plötzlich ein Keuchen von sich gab.

Viele Schüler und auch Madame Pomfrey schauten nun in seine Richtung. Er begann die Augen zu öffnen. Sie leuchteten grün. Es schien so, als ob er nicht bei Bewusstsein wäre. Sein Mund war ebenfalls leicht geöffnet und ein kleiner Lichtstrahl, der immer stärker wurde, verließ seinen Mund.

\enquote{Schnell}, rief Madame Pomfrey zu einem der Schüler, \enquote{holen Sie Dumbledore.}

Die angesprochene verließ die Krankenstation und kam schon wenige Sekunden danach zurück. Offenbar hatte sich Dumbledore entschlossen, die Krankenstation und ihre Insassen zu besuchen. Sein Blick fiel jetzt ebenfalls auf Professor Elber.

Der Lichtstrahl wurde jetzt nicht mehr stärker, dafür schlängelte sich ein zweiter um den ersten herum und verformte sich zu einer über Professor Elber frei schwebenden durchscheinenden Ebene. Langsam bildeten sich Personen und die sie umgebende Landschaft heraus. Es war Professor Elber, der Lord Voldemort gegenüber stand. Hinter Voldemort standen einige seiner Anhänger. Es gab einen heftigen Streit.

\enquote{Na gut}, meinte Voldemort, \enquote{wenn du nicht willst. Ich habe noch andere Mittel und Wege das zu bekommen, was ich will.} Er zückte seinen Zauberstab und sprach: \enquote{Imperius}.

Professor Elber schaffte es gerade noch, ebenfalls seinen Zauberstab zu ziehen und den Imperius-Fluch umzulenken. Er traf nun einen von Voldemorts Anhängern, der nun bewegungslos wie eine Marionette herumstand.

Voldemorts Augen begannen zu funkeln und sein Blick wurde zorniger und seine Stimme wurde ärgerlicher. \enquote{Avada Kedavra}, schrie Voldemort.

Die umstehenden Schüler erschraken und auch Professor Dumbledore zuckte ein wenig. Doch etwas Unerwartetes geschah. Der grüne Blitz aus Voldemorts Zauberstab wurde von dem Zauberstab Professor Elbers aufgefangen, umgeleitet und in Richtung Himmel geschickt.

Den umstehenden Personen fiel das Kinn herunter. Einzelne \accentuate{Boah!}-Rufe hallten durch die Krankenstation.

Etwas geschwächt durch den umgeleiteten Tötungsfluch sagte Professor Elber: \enquote{Der nächste kommt zu dir zurück.}

Das aber beeindruckte Voldemort sichtlich gar nicht. Er sprach abermals: \enquote{Avada Kedavra} und schleuderte einen weiteren Blitz auf Professor Elber. Dieser nahm auch diesen auf, schleuderte ihn aber Richtung Voldemort, welcher gerade noch ausweichen konnte und zur Seite hechtete. Am Boden liegend ließ Voldemort erneut einen Tötungsfluch auf Professor Elber los. Dieser streifte ihn leicht und lies Professor Elber zu Boden sinken. \enquote{Tja, wenn man sich mit mir anlegt, verliert man}, sprach Voldemort.

Professor Elber lag regungslos am Boden.

\enquote{Ist er tot?}, fragte ein Schüler.

Professor Dumbledore antwortete ihm. \enquote{Wenn er tot wäre, wäre er jetzt nicht hier, oder?}

\enquote{Stimmt}, gab er als Antwort zurück.

Voldemort lief auf Professor Elber zu und wollte noch einen Zauber nachsetzen, als Professor Elber begann, sich aufzulösen und zu verschwinden. Voldemort fluchte umher und war außer sich.

Die Szene löste sich langsam auf und begann ihre Form und Gestalt zu ändern. Man sah jetzt Hogwarts im Hintergrund. Professor Elber lag am Bahnsteig; immer noch regungslos. Langsam rührte er sich und stand auf. Mühsam schleppte er sich den Pfad hinauf nach Hogwarts, wo er schließlich gänzlich erschöpft auf den Stufen zum Eingangstor zusammenbrach. Die Projektion löste sich auf, Professor Elber schloss seine Augen und dreht sich abermals.

Stille herrschte im Raum und keiner wagte es etwas zu sagen, bis Madame Pomfrey plötzlich in die Hände klatschte und meinte: \enquote{So, Kinder, es wird Zeit, wieder in die Betten zu steigen, wir haben noch viel zu tun. Oder wollt ihr eure Hörner, Ohren oder Flügel etwa behalten?} Die Schüler begaben sich wieder in ihre Betten und Madame Pomfrey machte sich an ihre Heilung.

Professor Dumbledore murmelte nur \enquote{Ich hoffe, er wird wieder}, während die Schüler mit den Eselsohren nur laut zustimmten.

Währenddessen trieb es Harry aus einem inneren Antrieb heraus wieder in die Kammer. Er streifte umher und untersuchte die Gänge, um etwaige Geheimtüren zu erkennen. Hinter einer dieser als abgesetzte Steinbogen getarnten Tür fand er eine kleine Kammer mit einer Handvoll Büchern und einem kleinen Holzstuhl. Wahlweise nahm er eines der Bücher heraus und las. Es handelte von Basilisken. Keines der Bücher war sehr dick, sodass er alle auf einmal las. Sie handelten von Aufzucht, Pflege und Schutz vor Basilisken.

\fluestern{Interessant}, murmelte Harry. \fluestern{Man kann einen Basilisken auf einen Prägen, wenn man einen speziellen Trank braut und ihm eines der eigenen Haare zufügt. Dann muss man das bebrütete Ei hineinlegen. Fortan ist man vor den Blicken des Basilisken geschützt.}

Nachdenklich strich er über die letzte Seite, auf der der Trank abgebildet war. Er schloss das Buch und verließ die kleine Kammer, dessen Tür sich sofort schloss, nachdem er hinausgetreten war.

Bevor er die Kammer verließ, sah er noch einmal auf das Loch, aus dem der Basilisk damals auf ihn zukam. Ein innerer Drang überkam ihn und so kroch er durch die Röhre. In einer Kammer, die ein Nest aus Stroh enthielt, fand er ein Ei. Vorsichtig trat er mit gezücktem Zauberstab näher. Skeptisch betrachtete er das Ei.

\gedanke{Ob es wohl befruchtet ist?}, fragte sich Harry.

\stimme{Ja}, erklang es in seinen Gedanken. \stimme{Das ist ein Basilisken-Ei.}

Panisch trat Harry einige Schritte zurück.

\stimme{Keine Panik, Harry. Du hast einen Schutztrank gesehen. Dieser prägt die kleine Schlange auf dich. Du kannst ihr dann zwar noch nicht in die Augen sehen, aber wenn sie geschlüpft ist, kannst du ihr befehlen, die Augen zu schließen. Dann belegst du die Augen mit einem Zauber, damit der gefährliche Blick nicht mehr tötet. Es kribbelt bei dir nur noch und schmerzt in den Augen, sollte der Basilisk dich danach ansehen. Andere versteinert er noch. Erst wenn du ihm einen Trank gibst, verliert sich die Gefährlichkeit des tödlichen Blickes.}

Harry dachte nach. Er beherrschte mittlerweile einen Kopierzauber, der auf Entfernungen reagierte. So kopierte er den entsprechenden Trank, den er heute bei einer seiner Strafstunden brauen würde. Denn trotz seiner Okklumentikstunden, braute er ab und an einen Trank, damit sich seine Braukünste besserten. Er las das Rezept noch einmal durch und machte sich dann auf den Rückweg, um vor seinem Treffen mit Katharina noch ein paar Hausaufgaben zu erledigen. Er hatte noch zwei Stunden Zeit. Da ihm Kreacher nachher beim Anziehen und Zurechtmachen helfen würde, konnte er sich die Zeit nehmen.

\trenn

Als Harry gerade seine Hausaufgaben im Gemeinschaftsraum zusammen mit Ron machte, kam Lavender herein und setzte sich in die Nähe der beiden zu Parvati.

\enquote{Hast du schon gehört, was Professor Elber im Krankenflügel passiert ist?}

Harry drehte sich herum, um die Unterhaltung besser mit anhören zu können und auch Ron blickte plötzlich auf.

\enquote{Also}, fuhr sie fort, \enquote{ich lag gerade da um meine\abs nun ja ich war gerade in Behandlung}, fuhr sie fort, als sie Harry erblickte, \enquote{als Professor Elber seine Augen auf unnatürliche Art und Weise öffnete und ein grüner Lichtschein sich aus seinem Mund bahnte. Sie zeigte Professor Elber, der Du-weißt-schon-wem gegenüber stand.} Ein leichtes Zittern lag immer noch in ihrer Stimme. \enquote{Sie schienen sich zu streiten, als V-Voldemort den Tötungsfluch auf ihn warf.}

Plötzlich hatte sie die ganze Aufmerksamkeit des gesamten Gemeinschaftsraumes.

\gedanke{Den Tötungsfluch überlebt}, dachte Harry. \gedanke{Außer mir ist es Professor Elber nun auch gelungen.}

\enquote{Jedenfalls}, fuhr Lavender fort, \enquote{er hat ihn einfach umgelenkt und in den Himmel geschleudert. Dann hat es Du-weißt-schon-wer noch einmal versucht, aber Professor Elber warf ihn einfach zurück. Du-weißt-schon-wer konnte sich noch zur Seite werfen und warf einen weiteren auf Professor Elber.} Parvatis Augen weiteten sich und sie hielt ihre Hand vor den Mund. \enquote{Voldemort warf noch einmal einen Fluch auf ihn. Dieser streifte ihn allerdings und er brach zusammen. Er konnte gerade noch nach Hogsmeade apparieren und sich dann ins Schloss zurückschleppen.} Parvatis Blick fiel auf Harry, und Lavender drehte sich zu ihm herum.

\enquote{Woher weißt du das?}, fragte Harry sie.

\enquote{Ich habe es gesehen}, antwortete Lavender. \enquote{Es war wie eine Luftprojektion. Sie schwebte in der Luft und zeigte das Duell. Harry, ich habe noch nie von jemandem gehört, der den Tötungsfluch überlegt hat.} Und plötzlich fiel es ihr wieder ein. \enquote{Tut mir leid, Harry, ich meinte natürlich, von keinem weiteren.}

\enquote{Ist schon in Ordnung, Lavender}, antwortete Harry. \enquote{Aber wenn er den Tötungsfluch abgewehrt hatte, dann können wir das vielleicht auch lernen.}

\enquote{Daran habe ich noch gar nicht gedacht}, meinte Lavender. \enquote{Aber wieso hat Dumbledore so etwas nicht gemacht? Oder dir so was beigebracht?}

Harry konnte nur vermuten, antwortete aber: \enquote{Vielleicht kennt Dumbledore diese Art von Zauber nicht, oder Professor Elber wendete dafür schwarze Magie an. Schließlich unterrichtet er dieses Fach. Vielleicht sind seine Kenntnisse in schwarzer Magie derart umfangreich, dass er ihn abwehren konnte.} Plötzlich fiel ihm etwas ein. \enquote{Wir haben doch unten in der Kammer diese pulsierenden Blitze mit unseren Schilden abgewehrt. Vielleicht läuft das in genau diese Richtung und Professor Elber wollte uns nur noch nicht sagen, dass wir damit sogar den Tötungsfluch abwehren können.}

Schweigen herrschte im Gemeinschaftsraum.

\enquote{Wäre möglich}, meinte Parvati. \enquote{Aber ich will es nicht ausprobieren.}

\enquote{Ich werde mich jetzt umziehen und dann auf mein Valentinsdate warten}, sagte Lavender, stand auf und ging in ihr Zimmer.

\enquote{Gute Idee Lavender, das mache ich auch}, meinte Harry.

Einige im Raum mutmaßten bereits, dass \accentuate{die beiden} ein Date hatten.

Am Nachmittag stand Harry mit Kreachers vorbereitetem Parfum und frisch gewaschenem valentinstauglichem Dress im Haupteingang zum Schloss und wartete auf seine Überraschung. Wie von Katharina vorausgesehen, warteten viele Pärchen im Vorhof und starrten gespannt auf Harry. Nach etwa zwei Minuten kam, wie zuvor verabredet, Pansy und stellte sich kurz neben Harry.

Den anderen fiel ihre Kinnlade herunter. Sie raunte ihm in sein Ohr. \enquote{Sei froh, dass Katharina so eine gute Freundin ist.} Dann lief sie an ihm vorbei und traf sich mit Blaise Zabini und ging mit ihm den Schotterweg hinunter nach Hogsmeade.

Dieser Schock hat gesessen. Harry vernahm Gemurmel, doch Katharina kam gleich darauf und das Gemurmel verstummte wieder. Harry lächelte sie an.
Dann zog er einen kleinen Haarreif aus seinem Umhang und setzte ihn ihr auf. \enquote{Eine kleine Leihgabe, damit du auch entsprechend aussiehst}, sagte er und bot ihr seinen Arm wie bei einem Ball an. Diesen ergriff sie, indem sie ihren Arm ein hing und mit Harry loslief. Doch schon nach wenigen Metern, als sie an ihren Mitschülern vorbei, doch immer noch gut sichtbar waren, ließ Harry seinen Arm sinken, löste sich von Katharina und griff um ihren Arm herum. Dann verschränkte er seine Finger in ihren und lief den restlichen Weg bis nach Hogsmeade mit ihr Händchen haltend.

Harrys Lippen hingen immer wieder an Katharinas Ohr und immer wieder lachte sie. Mal, weil er eine Idee hatte, mal, weil seine Haare kitzelten, oder auch seine Lippen ihr Ohr sanft umspielten.

Luna war der Typ von Freundin, die die Bedürfnisse ihres jungen Freundes voll verstand. Ihr Verhältnis zueinander war durch ihre Art der Verbundenheit wohl etwas lockerer. Denn diese Art war einmalig. Auch sie hatte ein Date; mit Neville. Neville hatte ein Faible für Luna und genoss den Nachmittag mit ihr. Aber erst, nachdem ihm Luna und vor allem Harry mehrmals versichert hatten, dass es in Ordnung sei. Neville hatte sie mal erwischt, wie sie sich küssten. So weihten sie ihn, als einzigen Außenstehenden, ein; noch bevor sie es offiziell machten. Das, was Harry mit Luna verband, war mehr als nur Freundschaft. Es war eine Art Liebe, die man mit nichts vergleichen konnte. Doch Harry hatte das Gefühl, nicht bis zum Jahresende mit Luna zusammen zu sein. Trotzdem genoss er jeden Moment, den er mit ihr erleben durfte. Er verstand sie immer besser. Und sie ihn auch.

Aber heute war er mit Katharina da. Als er seinen Gedanken nachhing, überlegte er, wie er seine Mitschüler foppen konnte. In Hogsmeade angekommen, gingen beide erst einmal in den Honigtopf, um ihren Vorrat an Süßem aufzufüllen. Dann ging es die Einkaufsstraße entlang. Harry trug einen mit silbernen Runen bestickten nachtblauen Seidenumhang über seinem Hemd und seiner Stoffhose. Auf eine Krawatte oder Fliege verzichtete er. Katharina trug ein dunkelblaues Kleid, das mit seinem Umhang wunderbar harmonierte. Als sie ein Bekleidungsgeschäft betraten, um sich umzusehen, kam ein Verkäufer auf sie zu um sie zu bedienen und zu beraten. Als er Harrys Umhang sah, veränderte sich seine Miene.

Der Verkäufer sah das Paar ehrfürchtig an. \enquote{Ein ausgezeichnetes Stück, das Sie da tragen. Wissen Sie um die Symbolik der Runen?}, fragte der Verkäufer.

\enquote{Mein Elf hat sie mir erklärt}, antwortete Harry. \enquote{Diese hier}, er zeigte auf eine der Runen, \enquote{hält Neider ab. Diese hier}, er zeigte auf eine andere, \enquote{Stärkt Bindungen.}

\enquote{Wissen Sie auch um die tiefere Bedeutung, aus den alten Zeiten?}, fragte der Verkäufer nach.

Harry und Katharina verneinten.

\enquote{Sie stärkt die \accentuate{familiäre} Bindung \gst Früher wurden solche Umhänge zur Verlobung oder zur Hochzeit getragen, damit die Familien stärker aneinander gebunden wurden. Das hier ist ein sehr alter Umhang. Sie werden seit mehreren hundert Jahren nicht mehr hergestellt. Ich wusste gar nicht, dass die Familie Potter solche Sachen hat.}

\enquote{Der Umhang stammt nicht aus der Familie Potter}, erklärte Harry, was dem Verkäufer ein Stirnrunzeln einbrachte, \enquote{sondern aus der Familie der Black.}

Dem Verkäufer entglitten die Gesichtszüge. Sein Verhalten wurde, wenn es denn überhaupt noch möglich war, noch unterwürfiger. \enquote{Bitte verzeiht mir, Mister Potter.} Dann drehte er um und ging.

\enquote{Hatte der gerade Angst vor mir?}, fragte Harry Katharina.

\enquote{Sah ganz so aus}, antwortete sie. \enquote{Lass uns zu Madame Padifoot’s gehen.}

Harry nickte und gab Katharina einen Kuss auf die Wange. Diese errötete leicht und schmunzelte. Mit ihr hatte er ein Date, das Maßstäbe setzte. Die Kluft zwischen Gryffindor und Slytherin war heute etwas schmäler geworden.

Im netten Café angekommen fand sich schnell ein nettes Plätzchen, da ein Großteil der Paare den beiden gefolgt war, weil es so ungewöhnlich war. Nachdem beide eine Tasse Tee und ein Stück der Valentinstorte hatten, aßen und tranken beide, bevor Harry Katharinas Hände in seine nahm und mit ihren Fingern spielte und immer wieder küsste. Beide saßen an einem Tisch am Rand, aber nicht in einer Ecke, sodass beide eine gute Sicht auf die anderen Tische hatten. Sie funkelten sich an, als sie merkten, wie einige sich mit der Gabel und einem Stück Kuchen in die Wange oder das Kinn stachen. Sie mussten sich konzentrieren und die Augen des Gegenübers fixieren, damit sie nicht zu lachen anfingen. So sah es für Außenstehende aus, dass sie sich wirklich anschmachteten. Nachdem sie im Café fertig waren, schlenderten sie den Rest des Dorfes ab und machten sich danach auf den Rückweg zum Schloss.

Auf dem Rückweg tuschelten beide miteinander, wie sie ihre Show zu Ende führen wollten. Doch keinem der beiden fiel etwas ein, also entschlossen sie sich spontan zu sein.

Am Schloss angekommen legte Harry seine Hände um Katharina Taille, woraufhin sie ihre Hände um seinen Nacken schlug. Beide kamen sich näher und dann folgte der erste zarte Kuss, gefolgt von einem stürmischen. Dann gab es ein Poltern. Als sich beide wieder lösten und zum Schlossausgang sahen, lag eine Mitschülerin ohnmächtig auf dem Boden; vermutlich aus Schock, weil sich die beiden geküsst hatten. Katharina löste sich von Harry und lief auf ihre Mitschülerin zu, um sich um sie zu kümmern. Harry hingegen lief mit einem Lächeln auf den Lippen in das Schloss, um sich in den restlichen Stunden, die vor dem Abendessen lagen, noch für die morgigen Stunden vorzubereiten.

\trenn

Diesen Samstag ging es nach dem Mittagessen wieder in die Kammer. Heute würde ihnen Dumbledore etwas Neues beibringen. Die Schüler und ihr Schulleiter rutschten wie immer die Röhre hinunter, dann ging es durch das mittlerweile abgestützte Gewölbe in die Kammer. Die Tür stand bereits offen und Harry wartete mit ein paar Schülern auf den Rest, da sich die Tür immer wieder schloss; man konnte sie nicht dauerhaft aufhalten. Dumbledore kam mit ein paar Schülern herein, so langsam füllte sich der runde Raum.

Nachdem alle anwesend waren, begann Dumbledore und sagte: \enquote{Ich dachte mir, dass wir heute mal gelenkte Zauber durchnehmen. Nachdem ihr in diesen Kursen erweiterte Magie lernt, finde ich ist das passend. \gst Normalerweise}, sagte er und lief mit gezogenem Zauberstab im Raum umher, \enquote{wirken Zauber auf direktem Wege.} Er schwang seinen Zauberstab und ein leuchtend violetter Strahl verließ die Spitze seines Stabes und schlug auf der felsigen Wand ein. \enquote{Dies ist der Weg, den eure Zauber, wenn sie eure Stäbe verlassen, nehmen. Heute werden wir gelenkte Zauber durchnehmen, das heißt\abs} Er schwang seinen Stab und der Zauber vollzog eine Kurve, bis er hinter Dumbledore in den Felsen einschlug. \enquote{Ihr könnt den Zielort des Zaubers, nachdem er euren Stab verlassen hat, immer noch bestimmen. Das werden wir heute üben.} Dann stellte sich Dumbledore vor alle Schüler und begann erneut einen Zauber. Dieses Mal kam der Strahl viel langsamer aus seinem Stab und wieder beschrieb der Strahl eine Kurve. Dumbledore bewegte seinen Stab und es schien, dass der Strahl der Stabspitze folgte. Als alle aus dem Staunen heraus waren, hielt Dumbledore seinen Stab auf einer Position und lenkte den Strahl mit seinen Augen und seinem Kopf. \enquote{Es ist egal, ob sie ihren Stab nehmen, ihre Augen, oder sich ihr Ziel nur gedanklich vorstellen. \gst Wir fangen mit dem ersten an, da die anderen Sachen komplizierter sind.}

Harrys Magen wurde flau. Er musste aufpassen, dass er nicht zu viel von seinen Fähigkeiten preisgab. Er hielt sich zurück und wartete, wie sich die andern Schüler machten. Er hoffte, dass es bei ihm nicht durch Zufall auf Anhieb klappte. Er hing seinen Gedanken nach und bekam nicht ganz mit, was Dumbledore sagte. Er bekam nur mit, dass man sich darauf konzentrieren musste.

Nachdem Dumbledore fertig war, bat er seine Schüler, sich nebeneinander aufzustellen und mit dem Rücken zueinander zu stehen, damit sie sich nicht gegenseitig verletzten konnten. Dumbledore zählte von drei runter und die Schüler begannen ihre Zauber auszuführen. Viele \spruch{Ignitio solare}-Rufe klangen durch den Raum, mit denen die Schüler die leuchtend violette, aber sonst nichts wirkenden, Strahlen aus ihren Stäben lösten, um sie auf das gewünschte Ziel zu lenken. Bei ein paar wenigen konnte man eine kleine Korrektur um ein paar Zentimeter erkennen, obwohl es viel mehr sein sollte.

Harry sprach nun seinen Spruch und der Zauber kam genau dort an, wo er es beabsichtigte, nämlich ohne Abweichung genau dorthin, wo er gezielt hatte, als er den Zauber sprach und nicht dorthin, wo er danach mit seinem Stab zeigte. Während er den Zauber sprach, war er der Meinung, dass er die verstreichende Zeit langsamer empfand und so genügend Zeit hatte, den Zauber geradeaus zu lenken, obwohl er seinen Stab in einer Kurve beschrieb und der Strahl seiner Stabspitze zu folgen schien.

\enquote{Gut, gut}, sprach Dumbledore. \enquote{Das üben wir noch mal.} Erneut führte er den Zauber vor, damit die Schüler sich alles noch einmal einprägen konnten.

Draco Malfoy lächelte zu Harry herüber, als er seinen Zauber erfolgreich um mehrere Dezimeter abgelenkt hatte. Harry beherrschte sich jedoch und verkniff sich ein Duell mit ihm. Bei seinem nächsten Versuch lenkte er ihn auch weit ab, blieb aber deutlich unter dem von Draco. Innerlich freute er sich, da er es verstanden hatte und jederzeit seinen Zauber umlenkten konnte, wie er wollte. Zumindest war er der Meinung, dass es so sei.




\begin{kommentar}
Elber zeigt Harry und Dumbledore die Aufzüge. Aber wenn man bedenkt, dass er das Schloss erbaut hat, sollte man annehmen, dass er darüber Bescheid weiß.
\end{kommentar}

\begin{kommentar}
Nachdem Harry die Aufzüge entdeckt hat, muss er diese sofort benutzen, um Lavender und Parvati zu helfen. Im Aufzug drückt er das Symbol des roten Kreuzes und des roten Halbmondes. Die Symbole für medizinische Hilfe in der christlichen und muslimisch/arabischen Welt.
\end{kommentar}

\begin{kommentar}
Auf dem Rückweg von der Krankenstation sieht er Elber, wie er ein Amulett öffnet und eine blaue Kugel in seinen Körper eindringt. Er hat seinen Horkrux vernichtet und möchte seine Seele wiederherstellen.
\end{kommentar}

\begin{kommentar}
Kurz darauf kommt Elber schwer verletzt auf die Krankenstation. Dort zeigt sich einmal eine seiner wenigen Schwächen. Ab und an ist er sorglos oder zu unkonzentriert. Oder etwas zu arrogant und überheblich, was seine Fähigkeiten betrifft. Er überschätzt sich manchmal.
\end{kommentar}

\begin{kommentar}
Der erste Brief von Harrys Date enthält mehrere Anspielungen oder Vorgriffe auf die Geschichte. Pansy scheint ja eine ziemlich gute Freundin von Katharina zu sein. Später in der Geschichte fliegen ja alle Mädchen auf den jungen Zauberer. Hier kann man bei Pansy schon die ersten Anzeichen sehen.
\end{kommentar}

\begin{kommentar}
Der Name Katharina ist eine Anspielung auf Kathryn Janeway aus Star Trek Voyager. Dort nennt eine Figur den Captain Katharina.
\end{kommentar}

\begin{kommentar}
Pansy meint in einer Art Traum zu Harry, dass beide miteinander verwand seien. Erst im nächsten Teil erfährt Pansy, dass sie mit Gryffindor verwandt ist. Hier habe ich einen kleinen Hinweis darauf versteckt, obwohl man nicht darauf kommen kann. Erst, wenn man den zweiten Teil gelesen hat. Denn die Uniform kann im Traum auch nur so dagewesen sein.
\end{kommentar}

\begin{kommentar}
Ziemlich am Ende des Kapitels erfährt Dumbledore, dass Elfen Zauberstäbe haben dürfen und dass ein Gesetz von 1640 ungültig ist, da hierzu das vollständige Gamot hätte tagen müssen. Wenn man bedenkt, dass Elber schon ziemlich alt ist, kann man durchaus darauf kommen, dass er damals nicht dabei war, aber Mitglied des Gamots ist. Aber damals sahen das die anderen wohl etwas anders.
\end{kommentar}

\chapter{Erwachen}


\enquote{Brauen Sie Ihren Trank}, fuhr ihn Snape an, \enquote{ich habe noch zu tun.}

Harry nickte und stellte seinen Rucksack ab, holte einen Kessel und füllte ihn mit Wasser. Während das Feuer das Wasser erwärmte, packte er seine Zutaten aus und legte die Pergamentrolle mit den Zutaten und der Brauanweisung daneben. Ruhig und gewissenhaft warf er seine Zutaten nacheinander in den Kessel und rührte an den entsprechenden Stellen um.

Eine Portion füllte er in eine Glasflasche die er verkorkte. Den Rest füllte er in einen großen Becher, den er heute noch in die Kammer tragen würde. Anschließend räumte er sorgfältig auf und leerte seinen Kessel mit einem \spruch{Evanesco}.

Am Büro seines Lehrers angekommen fragte er: \enquote{Dauert es noch lange, Professor?}

\enquote{Es dauert so lange es eben dauert.}

\enquote{Länger als eine halbe Stunde? Dann könnte ich noch diesen Trank hier\abs}

\enquote{Was haben Sie damit denn vor?}

\enquote{Das werden Sie nachher erfahren. Wir üben doch heute bewusstes Lenken von Gedanken.} \gedanke{Ich werde es darin unterbringen}, fügte er in Gedanken hinzu.

Snape nickte nur und gab ein undefiniertes Grunzen zurück. Harry machte sich auf den Weg zur Kammer. Dort angekommen legte er das Ei vorsichtig in den Becher. Nun musste er bis morgen warten und dann das Ei wieder in das Nest legen, noch ein kleiner Meldezauber kombiniert mit einem Verzögerungszauber und er konnte der schlüpfenden Kreatur zusehen und sie auf sich prägen, dachte er.

Zurück bei Snape ging es auch schon los. Wieder einmal drang er in Harrys Geist ein. Und wieder sah er den Gang, der zu dem runden Raum mit den Türen führte. Eine dieser Türen öffnete sich und er stand in einer mit Stroh gefüllten Kammer. Ein Ei lag in einer nestartigen Ansammlung von Stroh und anderen Materialien. Snape ging auf das Ei zu, nahm aus seiner Tasche ein Glas und eine kleine Flasche mit einer Flüssigkeit. Dann nahm er das Ei, legte es in das Glas und schüttete die Flüssigkeit darüber. Dann drehte er sich um und fand sich im selben runden Raum wie am Anfang wieder. Und wieder einmal stand diese steinerne Staute von Salazar Slytherin vor ihm. Ein paar kurze Bildfetzen durchzogen Snape. Er sah sich in einem Sessel im Gemeinschaftsraum der Gryffindors sitzen und einen Geist anschauen, der Ähnlichkeiten mit der Statue hatte. Dann lag er plötzlich in einem unbekannten Bett und schaute an die Decke. Er stand vor einer kristallenen Säule mit einer Vertiefung und schaute durch eine klare Flüssigkeit, die in der Mulde lag, auf ein unscharfes Amulett. Dann stand er dicht hinter einem Mädchen mit blonden Haaren und umfasste ihre Schultern. Dann brach die Verbindung ab.

\enquote{Ab wann}, schnaufte Snape, \enquote{habe ich gesehen, was sie mir nicht zeigen wollten?}

\enquote{Ab dem Sessel im Gemeinschaftsraum}, antwortete Harry.

\enquote{So langsam geben die Bilder einen Sinn. Mittlerweile weiß ich, dass es sich bei der Statue und bei dem Geist um Salazar Slytherin handelt. Ich wusste gar nicht, dass Sie eine derartige Vorstellungskraft haben und Sie Slytherin plastisch sehen können.}

\enquote{Das war nicht vorgestellt}, sagte Harry.

\enquote{Wie darf ich das verstehen?}

\enquote{Was ich Ihnen jetzt sage, darf diesen Raum nicht verlassen. Und es darf Ihren Mund im Beisein anderer nicht verlassen.}

Snape nickte. \enquote{Einverstanden.}

\enquote{Den Geist, den Sie gesehen haben}, sagte Harry, der sich jetzt in einen Sessel am Kamin gesetzt hatte; wobei Snape ihm gegenüber Platz nahm; \enquote{war wirklich Salazar Slytherin. Ich weiß nicht, wie es geht, aber er ist wirklich da. Ich kann mich mit ihm unterhalten.}

\enquote{Sie sind sich sicher, dass Sie sich nicht den Kopf angestoßen haben?}, fragte Snape.

\enquote{Ganz sicher}, erklang es hinter Snape.

Überrascht drehte er sich um und sah dem Geist ins Gesicht. \enquote{Wie?}

\enquote{Ich bin wirklich da. Ich weiß nicht wie, aber ich weiß warum. Harry braucht alle Hilfe, die er bekommen kann. Er kann Ihnen noch viel mehr erzählen. Ich denke, wenn er Ihnen soweit vertraut, dass er Ihnen zeigt, was es mit dem Ei auf sich hat, dann sollten Sie ihm vertrauen. Beim nächsten Mal dürften Sie mehr erfahren.} Dann verschwand er wieder.

Snape sah Harry nachdenklich an. \enquote{Dann bis zum nächsten Termin}, sagte er.

Harry stand auf und ging.

Mitten in der Nacht erwachte er durch seinen Alarmzauber. Er schlich sich aus seinem Zimmer und kontrollierte die Karte des Rumtreibers im Gemeinschaftsraum, als sich Salazar zeigte und meldete. \enquote{Neben dem Ausgang, Harry.} Harry erschrak, als er Salazars Stimme hörte. \enquote{Neben dem Ausgang, Harry}, sagte er erneut. Harry löschte die Karte und packte sie ein. Dann ging er zu Salazar und stellte sich hinter ihn. Sein Urahn zeigte auf einen Stein den Harry mit seinem Zauberstab berühren sollte. Harry tat wie ihm geheißen und berührte den Stein. \enquote{Schnell, Harry, den darunter auch und dann einfach durchlaufen.}

Harry folgte brav und durchschritt die scheinbar massive Wand. Harry stand im Dunkeln. Sofort ließ er seinen Zauberstab aufleuchten und sah sich um. Er stand in einem schmalen Gang vor einer Steinwand.

Als er sich wieder umgedreht hatte, sah er nach ein paar Metern eine Treppe, die beständig nach unten führte. Harry folgte der Treppe die gewendelt war, dann wieder ein Stück gerade ging, oder leicht schräg nach unten, oder über Treppen führte. Er kam einem Abzweig näher. Auf seiner Strecke zeigte plötzlich ein roter Pfeil in seine Richtung.  Auf dem Boden eines Abzweiges war ein blauer Pfeil, der weg zeigte.

\enquote{Der Weg zu Ravenclaw?}, fragte Harry.

\enquote{Gut erkannt}, sagte Salazar, der neben ihm kurz erschien, da es so länger hielt und er sich Harry immer wieder kurz zeigen konnte.

Auf seinem Weg weiter nach unten traf er auch die Abzweigungen von Hufflepuff und Slytherin. Am Ende angekommen zeigte ihm Salazar wieder einen Stein, damit er neben der Röhre raus kommen würde. Er sah sich kurz im Schein seines Zauberstabes um und schüttelte ihn ein paar mal, sodass kleine Lichtkugeln um ihn herum schwebten. Wieder in der Kammer kroch er in die Röhre, aus der der Basilisk kam und wartete, bis die Eierschale zu knacken anfing. Dann schloss er seine Augen und wartete.

Salazar sagte ihm in seinem Geiste: \stimme{Ich seh’s mir an und gebe dir dann meine Erinnerungen daran. Da man mich nicht sieht, passiert mir nichts. Außerdem hatte ich auch mal einen Basilisken. Sie wurde aber leider böse und du hast die Basiliskendame getötet. Ich wünschte, ich hätte noch die Gelegenheit gehabt, sie abzuholen und mit ihr zu sterben.}

Nach wenigen Minuten war es so weit und der kleine Basilisk war geschlüpft. Harry zuckte zusammen, als er die Berührung der kleinen Zunge spürte.

\parsel{Schließe deine Augen.}

\parsel{Warum, Meisster?}

\parsel{Tu es. Ich erkläre es dir gleich. Lass sie zu, bis ich es dir sage.}

\parsel{Ok, Meisster.}

\stimme{Er hat seine Augen geschlossen}, vernahm Harry in seinem Geist.

Harry öffnete seine Augen und richtete seinen Zauberstab auf die Augen des kleinen Basiliskenmännchens. Nacheinander sprach er über jedem der beiden Augen einen Zauberspruch. Dann wagte er den letzten Schritt.

\parsel{Du kannst deine Augen jetzt öffnen.}

Der kleine Basilisk öffnete seine Augen und sah Harry an. Dann schloss er sie schnell wieder. \parsel{Dass war gefährlich, Meisster. Ich hätte euch töten können.}

\parsel{Nein, ich habe Vorkehrungen getroffen.} Der Basilisk öffnete wieder seine Augen und sah Harry erstaunt an. \parsel{Ich habe einen Zauber auf deine Augen gelegt, damit mich dein Blick nicht tötet. Ich werde dir noch einen Trank geben, damit auch andere durch dich nicht getötet werden.}

Mit einem seligen Ausdruck schlängelte sich das Männchen in Harrys Schoß, der sich in der Zwischenzeit auf das Stroh gesetzt hatte. Vorsichtig strich er über den kleinen Basilisken.

\parsel{Wie isst euer Name, Meisster?}, fragte der Basilisk.

\parsel{Ich heiße Harry}, antwortete Harry.

\parsel{Und ich? Wie isst mein Name?}

\parsel{Tja, \gst wie wäre es mit Marcel?}

\parsel{Marssel? \gst Klingt gut. Marssel! \gst Ich bin müde, Meisster Harry.}

\parsel{Harry reicht, du brauchst mich nicht Meister zu nennen.} Dann nahm er die kleine Schlange vorsichtig hoch und legte sie auf dem Stroh ab. Er sprach noch einen Wärmezauber und verabschiedete sich. Marcel schloss seine Augen, rollte sich zusammen und schlief.

Harry stand auf, verließ die Geburtskammer und machte sich auf den Weg zurück.

\enquote{Warte, Harry. Hole noch eines der Bücher. Du brauchst Informationen über die Aufzucht und die Pflege von Basilisken.}

Harry nickte, holte das Buch und schleppte sich mit schmerzenden Augen in den Weg nach oben und in sein Zimmer zurück.

\trenn

Gedanklich ging er noch einmal die Okklumentik-Stunde mit Snape durch, die er, kurz nachdem er das Ei in den Trank gelegt hatte, absolviert hatte. Sein Geist schweifte dabei ab.

\begin{rueckblick}
Sein Lehrer drang wie üblich in seinen Geist ein, den er zunehmend besser kontrollieren konnte. Dieses Mal zeigte er ihm bewusst die Bilder, die er ihm zeigen wollte. Ein Ei auf dem Stroh, dann wieder diese Statue aus Marmor. Dann das Ei, das er in den Becher mit der gebrauten Flüssigkeit legte. Dann, wie er ihn braute; voll konzentriert und auf den Punkt gebracht. Dann begann er Snape zurückzudrängen, doch noch gab dieser nicht auf. Kurz dachte er über seine Mutter und schickte für einen Augenblick ein Bild seiner Mutter, das er in seinem Album gesehen hatte, Snape zu. Doch er entschied sich um und zeigte ihm stattdessen etwas, was er sich gedanklich zusammen reimte. Dann kamen auf Snape Bildfetzen zu, die Harry nicht kontrollieren konnte.

\enquote{Nehmen Sie keine Rücksicht darauf, wenn sich ein Feind in ihrem Geist aufhält.}

\enquote{Aber wir üben doch nur.}

\enquote{Wir gehen reale Bedingungen durch. Wenn Sie einen Feind so schlagen können, dann tun Sie das auch.}

Er ließ sich diesen Gedanken noch einmal durch den Kopf gehen, was zur Folge hatte, dass Snape sich selbst sah, wie er es Harry mitteilte. Harry zog amüsiert einen Mundwinkel hoch, als er Snapes Gesicht sah.

Plötzlich durchflutete ihn eine Erinnerung und obwohl er wusste, dass er sie so nicht gesehen haben konnte, weil er auf sich selbst blickte und noch zu klein war, wusste er, dass sie real war.

Er stand sich selbst gegenüber. Seinen weißen Zauberstab in der Hand auf sein kleineres selbst gerichtet und den Tötungsfluch sprechend, der an ihm abprallte und ihn selber traf. Er hörte einen unmenschlichen Schrei und Schmerzen durchfuhren ihn. Schmerzen, wie er sie noch nie erlebt hatte. Schlimmer als der Cruciatus-Fluch. Doch so schnell, wie der Schmerz kam, ging er auch wieder. Die Verbindung zu Snape brach ab und Harry sackte zusammen. Schwer atmend lag er bei vollem Bewusstsein auf dem Boden und atmete stoßartig. Langsam schwenkte er seinen Kopf zu Snape hinüber und sah ihn in einem Sessel sitzen und sich die Hand auf die Brust legte.

\enquote{Alles in Ordnung, Potter?}, fragte er.

\enquote{Geht so, Professor, und bei Ihnen?}, antwortete Harry fragend.

\enquote{Was war das?}, wollte er wissen.

\enquote{Voldemort, wie er versuchte mich zu töten. Denke ich. Ich\abs Ich habe einmal davon geträumt, aber ohne\abs Schmerzen.}

Dann fielen Harrys Augen endgültig zu.

Um zwei Uhr morgens wachte er auf und fand Snape schlafend im Sessel sitzend vor. Harry stand auf und suchte auf Snapes Schreibtisch ein leeres Pergament. Dann schrieb er darauf:

\begin{brief}
Bin ins Bett gegangen, gute Nacht Professor.
\signumspace
Harry Potter
\end{brief}

Dann schlich er sich vorsichtig zurück und schlief in seinem Bett ein.
\end{rueckblick}

Er überlegte, ob er Snape und seinen Freunden zumindest einen Teil seines Wissens zeigen sollte. Er könnte ihnen immerhin die privaten Räume Slytherins zeigen. Müde und mit geschlossenen Augen dachte er in den Raum hinein.

\gedanke{Salazar?}

\stimme{Ja, Harry.}

\gedanke{Ich habe mich gefragt, ob ich ein paar Leuten deine privaten Räume hier im Schloss zeigen darf. Professor Snape weiß immerhin von deiner Existenz. Und da ich nachweislich von dir abstamme, könnte mich das Amulett immerhin in seine Räume geführt haben. Ich werde weiterhin zu dir schweigen, aber deine Räume würde ich schon gerne\abs}

\stimme{Von mir aus gerne, Harry. Aber wie willst du es dem Bild am Eingang beibringen?}

\gedanke{Ich dachte, dass du\abs immerhin seid ihr zwei ja ein und dieselbe Person. Du könntest dich doch mit seinem Bild verbinden und ihn so überzeugen.}

\stimme{Ja, das stimmt. Wir sind ein und dieselbe Person. Aber ich weiß nicht, ob das, was du vorschlägst, auch klappt. Es käme auf einen Versuch an. Dann wäre mein anderes Ich nicht so komisch auf dich zu sprechen.}

\gedanke{Danke.}

\stimme{Wofür?}

\gedanke{Dafür, dass du es versuchst.}

\stimme{Für Versuche brauchst du dich nicht zu bedanken. Zumindest nicht bei mir. Danke mir nur für gelungene Taten.}

\gedanke{Verstanden Sal. \gst Verzeihung. Salazar.}

\stimme{Sal ist schon in Ordnung, wenn wir alleine sind.}

\gedanke{Aber wir sind doch immer alleine.}

\stimme{Und wenn dein Professor dabei ist? Oder wenn noch jemand dazukommt? \gst Später?}

\gedanke{Du bist doch derjenige, der sagt, ich darf nichts erzählen.}

\stimme{Noch nicht, Harry. Später.}

\gedanke{Wann ist später?}

\stimme{Falsche Frage, Harry.}

Dann kehrte Ruhe ein.

\trenn

Nach Wochen im Koma zeigte am Dienstag Professor Elber erstmals wieder eine Reaktion. Er öffnete die Augen und stöhnte. \enquote{Mein Kopf. \gst Was ist passiert?}

Die Tür zu Madame Pomfreys Büro ging auf und sie kam herein. \enquote{Ah, Professor Sie sind schon auf?} und schaute ihn dabei an.

\enquote{Wen meinen Sie?}

\enquote{Na Sie, Frederick. Haben Sie irgendwelche Beschwerden?}

\enquote{Nein, aber \gst wo bin ich?}

\enquote{Sie sind im Krankenflügel in Hogwarts.}

\enquote{Ah ja}, antwortete Professor Elber. Er erhob sich leicht und stützte sich auf seinen Ellenbogen und seinen Unterarmen auf. \enquote{Wenn Sie mir jetzt noch sagen können, wie ich hier hergekommen bin und was eigentlich passiert ist\abs}

In diesem Moment ging die Tür auf und Dumbledore kam herein.

\enquote{Und Frederick, alles wieder in Ordnung? \gst Vor allem, wie geht es dir?}

\enquote{Mir geht es gut, aber wer sind Sie? Ich verstehe, dass sich meine Frau um mich sorgt}, er zeigte auf Madame Pomfrey, \enquote{aber jemand den ich nicht kenne und mich besuchen kommt?}

Professor Dumbledore hob seine Augenbrauen und meinte dann zu Madame Pomfrey. \enquote{Kann ich Sie mal kurz sprechen} und dann zu Professor Elber gewandt: \enquote{Sie bekommen sie gleich wieder.}

Professor Elber hob seinen Oberkörper weiter an und saß nun im Bett, während Professor Dumbledore und Madame Pomfrey in ihr Büro verschwanden. Etwas benommen schaute er auf die andere Seite des Zimmers und durch das Fenster hindurch.

Nach einigen Minuten kam Professor Dumbledore mit Madame Pomfrey wieder aus ihrem Büro heraus. Professor Elber entdeckte sie und meinte nur: \enquote{Hallo Krankenschwester, wann gibt es was zu essen?}

\enquote{Sie wollen von ihrer Frau was zu essen?}, fragte Professor Dumbledore.

Professor Elber stutzt und meinte: \enquote{Ich bin nicht verheiratet, aber ich habe Hunger und fragte gerade die Krankenschwester, wann es was zu essen gibt.}

\enquote{Aber vor fünf Minuten haben Sie noch behauptet sie wäre ihre Frau.}

\enquote{Wer?}

\enquote{Sie.}

\enquote{Ich?}

\enquote{Ja.}

\enquote{Ne.}

\enquote{Doch.}

\enquote{Ohh.}

\enquote{Erinnern Sie sich nicht mehr daran?}

\enquote{Nein}, antwortete Professor Elber.

Madame Pomfrey brachte ihm eine Suppe und Professor Dumbledore verließ den Krankenflügel auf dem Weg zur Großen Halle, da es bereits Zeit für das Abendessen war. Er würde seinen Schülern diese großartige Nachricht noch heute mitteilen.

Währenddessen war Harry in der Kammer des Schreckens und gab Marcel seinen neuen Trank, damit keiner durch den Blick des kleinen Basilisken versteinert oder getötet werden würde. Das Brennen in seinen Augen verschwand und die Schlange konnte mit Harry die Kammer verlassen. Er nahm sie auf dem Arm mit nach oben.

Doch vor dem Feuer im Gemeinschaftsraum oder in seinem Zimmer fühlte sich Marcel nicht besonders wohl.

\parsel{Bitte, bring mich zurück. Hier fühle ich mich nicht so wohl.}

\gedanke{Scheinbar hat der Trank sein Marcels lispeln behoben}, dachte Harry.
\parsel{Warum? Möchtest du lieber alleine sein?}

\parsel{Ja. Ich ziehe die Einsamkeit oder die Gesellschaft anderer meiner Art vor.}

\parsel{Und Schlangen?}, fragte Harry nach.

\parsel{Die auch, wieso?}

\parsel{Dann habe ich eine passende Umgebung für dich.}

Harry nahm Marcel wieder auf den Arm und verdrückte sich durch die Gänge des Schlosses Richtung Salazars private Räume. Dort setzt er ihn ab und erklärte ihm, wie er in die Kammer kommen könnte. Der kleine Basilisk schien glücklich zu sein.

\parsel{Tust du mir noch einen Gefallen, Marcel?}, fragte Harry den Basilisken.

\parsel{Wenn ich kann.}

\parsel{Gib mir etwas deines Giftes.}

\parsel{Wie?}

Harry zauberte ein kleines Reagenzglas hervor und überzog es mit einer Gummimembran. Dann hielt er es seinem Freund hin. Marcel zögerte kurz, biss dann aber doch zu und wenige Tropfen seines Giftes rannen im Inneren des Reagenzglases entlang zu Boden.

\parsel{Danke Marcel.}

\parsel{Gerne geschehen, Harry.}

Harry verabschiedete sich und machte sich auf den Rückweg. In seinem Zimmer holte er noch eine Ampulle Basiliskengift der alten Schlange und nahm vorsichtig ein paar Tropfen von Marcels Gift, um es in seinen privaten Vorrat zu lagern. Er hielt es trotzdem für viel und nahm sich vor Kreacher zu sagen, dass er einen großen Teil davon im Grimmauldplatz in seinem Labor lagern sollte. Durch ein paar Bücher und die wenigen Besuche dort, hatte er einen guten Überblick über die Räumlichkeiten dort.

Nach getaner Arbeit, machte er sich auf den Weg zu Madame Pomfrey, um ihr etwas von dem Gift zu geben. Er vergewisserte sich vorher, dass sie nicht nachfragen würde, woher er das habe, was er ihr nun geben würde. Schließlich sagte sie zu und Harry überreichte ihr die kleine Phiole des Basiliskengiftes und das Reagenzglas mit den wenigen Tropfen.

Er vermutete, dass sie eine Ahnung hatte, woher das Gift in der Phiole sein könnte, dass sie aber keine Ahnung hatte, woher das Gift des jungen Basilisken war. Und so sollte es auch sein.

\trenn

Schweißgebadet wachte Harry auf. Er schrak aus seinem Traum hoch und schwitzte überall an seinem Körper. Er starrte auf einen imaginären Punkt weit draußen vor dem Schloss. Seine Hand fuhr auf seine Brust um zu fühlen, ob er sein Amulett noch bei sich trug. Er hatte es seit es ihm Ginny geschenkt hatte nicht abgelegt. Doch er fühlte sein Amulett nicht. Panisch suchte er danach. Es lag auf seinem Nachttisch. Er nahm es auf und legte es sich um. Wie auf ein unsichtbares Kommando wurde er ruhiger und ausgeglichener. Dann begann er sich wieder zu erinnern, was letzte Nacht passiert war.

Schlaftrunken hatte er, bevor er ins Bett stieg, statt seiner Brille sein Amulett abgenommen, hatte es auf seinen Nachttisch gelegt und sich danach hingelegt. Dann schlummerte er ein.

\begin{traum}
Voldemort war seit vier Jahren Tod und es stand ein Frühstücks-Treffen bei den Weasleys an. Draco Malfoy war auch dabei, da er sich mittlerweile von seinem Vater abgewandt hatte und sich in letzter Zeit besser mit Harry und seinen Freunden verstand. \gst

Harry und Draco saßen spät am Abend noch draußen und unterhielten sich über alle möglichen Dinge. Ihre Hände berührten sich und beide zogen den anderen mit den Augen aus. \gst

Draco lag mit Harry im Bett und Harry liebkoste Dracos freie Brust mit seiner Zunge. Draco keuchte unter Harrys warmer, feuchter Zunge. Sie begehrten sich so sehr. Sie spielten miteinander. Sie konnten sich kaum noch halten. Dann war es so weit. Harry zog Dracos Boxershorts herunter und glitt immer weiter mit seinem Mund an ihm herunter. Draco griff mit seinen Händen in die Matratze und stöhnte auf, als ihm Harry \gst
\end{traum}

So langsam kamen in Harry die Erinnerungen an seinen Traum wieder hoch. Er schaute auf seine Uhr und merkte, dass es sich um sieben Uhr morgens nicht mehr lohnte, sich hinzulegen. Er duschte erst einmal, zog sich danach an und ging nach unten.

Tamara saß in einem der Sessel und las. Sie war immer eine der Ersten, die morgens wach war. \enquote{Guten Morgen, Harry}, trällerte sie ihm entgegen. Mittlerweile war sie für ihn wie eine kleine Schwester.

\enquote{Guten Morgen, Tamara.}

\enquote{Schlecht geträumt?}, fragte sie ihn.

\enquote{Ja. Von Draco.} Doch das wollte Harry nicht sagen.

Interessiert horchte sie auf, klappte ihr Buch zu, nachdem sie ihr Lesezeichen hineingelegt hatte, und ließ das Buch mit ihrem Zauberstab an die Decke schweben. Dort sah es keiner und so nahm es ihr auch keiner weg. Harry hatte ihr gezeigt, wie sie es machen müsste. Und auch, dass ihr keiner dabei zusah.

Sie hob eine Augenbraue hoch und bohrte so lange nach, bis Harry ihr ziemlich grob und oberflächlich die Zusammenfassung sagte. Tamara bekam große Augen.

\enquote{Hast du meinen Bruder schon einmal nackt gesehen?}, fragte sie.

\enquote{Nein}, gab Harry empört zurück.

\enquote{Du hast ihn ziemlich genau beschrieben}, sagte sie. \enquote{Gehen wir frühstücken. Es sollte schon etwas da sein.}

Sie nahm Harry bei der Hand und zog ihn hinter sich her. Wenige Meter, nachdem sie das Porträt hinter sich gelassen hatten, lies sie von ihm ab und lief nun neben ihm her. Weiter unten trafen sie auf Draco. Seine Schwester begrüßte ihn und sah ihm in die Augen. Sie ließen sich etwas zurückfallen, um sich ungestört zu unterhalten. Plötzlich hörte er laufende Schritte hinter sich. Zuerst nur ein paar Füße, danach zwei Paar.

\enquote{Harry}, hörte er Tamaras Stimme. Doch weiter kam nichts. Harry drehte sich um und merkte, wie ihr Malfoy den Mund zu hielt. Doch Tamara konnte sich wehren, sie stieg ihm auf den Fuß und versuchte ihn zu beißen. Draco ließ von seiner Schwester ab, worauf sie gleich sagte: \enquote{Er hatte denselben Traum wie du, Harry.}

Er konnte sich jetzt nicht mehr bewegen. Steif stand er da. Schrittweise bewegte er seine Augen von Tamara hinauf zu Draco und sah ihm direkt in die Augen. Mechanisch drehte er sich um und lief die restlichen Meter in die Große Halle. Er setzte sich auf die erstbeste Bank und starrte in die Ferne. Er saß auf der Bank der Ravenclaws. Draco und Tamara kamen um die Ecke und setzen sich neben ihn. Harry war in der Mitte. Nach einer Minute, die Harry brauchte um sich zu beruhigen, sah er zu Draco. Dieser sah ebenfalls an die Wand. Dann drehte sich Harry um und nahm sich Croissant. Er biss hinein und nahm etwas Orangensaft dazu, um es hinunterzuspülen.

\enquote{Madame Pomfrey!}, sagte Harry matt und stand auf. Draco nahm ebenfalls ein Croissant, trank schnell etwas und trabte Harry hinterher.

In der Krankenstation angekommen war Madame Pomfrey gerade dabei aufzuräumen. \enquote{Mister Malfoy, Mister Potter. Was kann ich für Sie tun?}

\enquote{Nun ja Madame Pomfrey}, druckste Harry herum. \enquote{Haben sie etwas gegen Alpträume?}

Draco nickte nur stumm und betreten.

Madame Pomfrey hob eine Augenbraue und schaute die beiden an. \enquote{Welcher Art waren Ihre Träume?}, fragte sie. Die beiden schauten die Krankenhexe erstaunt an. \enquote{Für die Behandlung}, fügte sie rasch hinzu, als sie die beiden betrachtete.

\enquote{Unangenehme erotische}, sagte Draco.

Harry sah ihn fassungslos an.

\enquote{Und bei Ihnen, Mister Potter?}

Mechanisch und ohne zu überlegen, sagte er ebenfalls: \enquote{Unangenehme erotische.} Dabei sah er Draco unentwegt an. Wieder sah er zu Madame Pomfrey, die ihn jetzt mit beiden hochgezogenen Augenbrauen ansah und immer wieder zwischen den beiden hin und herwechselte.

\enquote{Oh}, sagte sie plötzlich. \enquote{Sie hatten denselben Traum?}, fragte sie. Keiner der beiden sagte ein Wort.

\enquote{Ich bin gleich wieder da. Setzen Sie sich.} Dann drehte sie sich um und ging in ihr Büro.

Harry und Draco nahmen sich jeder einen Stuhl und warteten. Nachdem sie ihren Trunk erhalten hatten, gingen sie in die Große Halle, wo Tamara schon am Slytherin-Tisch saß und auf ihren Bruder wartete.

Nach dem Frühstück hatte Harry zwar noch Hausaufgaben zu erledigen, aber ihm war im Moment überhaupt nicht danach diese auch wirklich zu erledigen. Ziellos streifte er durch das Schloss. An Klassenzimmern und anderen Räumen vorbeilenkten ihn seine Füße ohne bestimmtes Ziel. Erst als er stehen blieb, ohne dass er sich dessen bewusst geworden war, hob er seinen Kopf. Er stand genau unter einem dieser steinernen Bogen, die es in Hogwarts massenweise gab.

Sein Kopf wurde langsam klarer und als ihm unbekannte Stimmen so langsam leiser wurden, drehte er sich und drückte einen bestimmten Stein in der Wand. Nach ein paar Sekunden öffnete sich dieselbe und Harry trat wie immer in den schmalen kleinen Raum ein.

Er drehte sich um und wartete, doch nichts geschah. Also begann er die Zeichen abzusuchen. Er hatte vor einer Woche genügend Zeit gehabt, all diese Symbole auf ein Pergament zu übertragen, welches er fast immer bei sich trug und auf das Ron ihn einmal angesprochen hatte. Er hatte ihm nur erzählt, dass er bald mehr darüber erfahren würde. Es wäre eine Überraschung.

Mit seinem Finger fuhr er über die Symbole und drückte erschrocken auf eines, nachdem er Stimmen hörte die näher gekommen waren. Harry hatte den Eindruck, sie waren in seinem Gang zu hören. Die Wand schloss sich und der Boden vibrierte leicht. Harry schaute noch auf das Symbol. Es sah so aus, als ob es Gitterstäbe darstellen würde. \gedanke{Snape. Die Kerker. Sein Büro?}, fuhr es Harry durch den Kopf. Es kam in ihm leichte Panik auf, doch als sich die Türen öffneten, war keine Spur von einem Kerker, oder von Snape zu sehen.

Ein laues Lüftchen umspielte sein Gesicht. Er ging nach draußen und ohne es zu bemerken, schloss sich hinter ihm die Wand. Harry stand nun im Freien. Er stand auf einer steinernen halbrunden Plattform. Etwa einen Meter lang und anderthalb Meter breit, in der Schlossmauer eingelassen. Den Halbkreis umgab ein schmiedeeisernes Geländer, welches an manchen Stellen mit Moos bewachsen war und auch schon leicht zu rosten anfing.

Er genoss den Augenblick, genoss den Ausblick. Dann blickte er zu Boden. Er trat an das Geländer und sah nach unten. Dort war nichts. Es ging über hundert Meter in die Tiefe. Man konnte den Boden kaum sehen. Nur erahnen. Der nackte Fels schloss nahtlos an die Schlossmauer an und Harry hatte den Eindruck, dass eines in das andere überging. Nach einigen Sekunden trat er einen Schritt zurück und nahm so langsam über seine Ohren Vogelgesänge und Gezwitscher wahr. Es war ganz leise, sodass er es kaum hörte.

Er sah nach links. Dort war nur Mauerwerk zu sehen, ebenso rechts. Dort konnte er jedoch Fenster und in der Ferne den Rand eines Turmes erahnen.

Wo war er bloß? In welchem Teil des Schlosses trieb er sich bloß herum? Er würde mit der Karte des Rumtreibers wieder kommen und dann schauen, wo im Schloss er sich befand. Er hielt sich am Geländer fest und sah nach oben. Dort konnte er einige Türme erkennen. Aber von außen war es schwer zu sagen, um welche Türme es sich handelte.

Harry rutschte das Herz in die Hose, als er wieder geradeaus blickte. Die Wand, aus der er herausgetreten war, war wieder verschlossen und Harry hatte absolut keine Ahnung, wie er zurückkommen sollte. Er tastete die Wand ab, doch er blieb erfolglos. Also begann er seine anderen Optionen in Betracht zu ziehen. Er drehte sich wieder um und sah über das Geländer. Er könnte ein langes Seil heraufbeschwören und es am Geländer festmachen. Danach könnte er sich einfach abseilen. Doch unter ihm war nichts, worauf er am Ende seiner Reise stehen konnte. Keine Plattform, die um das Schloss herum führte und auf der er ins Innere des Schlosses zurückkehren konnte. Er könnte natürlich auch seinen Besen aufrufen.

Die Vogelgesänge wurden lauter. Nachdenklich sah Harry nach oben zu den Vögeln. Langsam beruhigte er sich wieder, da er, zwar nur unbewusst aber doch deutlich genug, Gesänge eines Phönix' wahrnahm. Oben am Himmel zog gerade ein Vogel seine Kreise und es schien, als würde er immer weiter sinken und auf Harry zufliegen. Mit wenigen Flügelschlägen kam er an und setzte sich auf das Geländer. Harry presste sich an die Wand. Der rote Vogel war doch ziemlich groß und unheimlich. Irgendwie kam der Vogel Harry bekannt vor. Dieser legte seinen Kopf leicht schräg, so als ob er versuchen würde Harry einzuschätzen. Harry trat nun langsam auf den Vogel zu und sah ihm direkt in die Augen. Dann fiel ihm ein Stein vom Herzen. Es war Fawkes. Professor Dumbledores Phönix. Er ging auf ihn zu und strich ihm durch sein Gefieder. Der Phönix stimmte ein Lied an und Harry durchfuhr eine innere Ruhe und Behaglichkeit.

\enquote{Schön, dich mal wieder zu sehen, Fawkes.} Der Vogel sah ihn an und nickte. Harry hatte den Eindruck, der Vogel verstünde ihn. Er strich ihm noch einige Minuten durch sein Gefieder, was der Vogel durch seine schönsten Gesänge erwiderte. Harry durchfuhr ein wohliger Schauer. \enquote{Kannst du mir helfen, Fawkes? Ich komme hier nicht weg.}

Der Vogel sah ihn an und gab einen kleinen Laut von sich. Harry war erleichtert, würde ihn der Vogel doch zurück ins Schloss fliegen. Doch das tat er nicht. Er sah an Harry vorbei und nickte mit seinem Kopf in Richtung Mauer. Harry drehte sich verunsichert um.

Die Mauer sah nicht anders aus, als vorher. Jeder Stein hatte dieselbe Farbe. Er schaute wieder fragend zu Fawkes. Dieser nickte weiterhin mit seinem Kopf zur Mauer. Also ging Harry dort hin und zeigte mit seinem Finger auf einen beliebigen Stein. Der Kopf des Phönix wippte zu Seite, um Harry zu verdeutlichen, er möge seinen Finger auf einen anderen Stein bewegen. Als der Phönix schließlich nickte, wusste Harry, dass er den richtigen Stein hatte. Er drückte ihn leicht, und der Stein gab nach. Er zählte von der linken Seite des Geländers zwei nach oben und drei nach rechts und notierte sich diese Position auf seinem Pergament. Dann drückte er den Stein und die Wand teilte sich. Harry trat ein und drehte sich um. Fawkes kam auf ihn zu geflogen und Harry konnte gerade noch seinen Arm hochheben, damit der Phönix darauf landen und sich setzen konnte. Er lief den Arm entlang hoch zu seiner Schulter und schmiegte sich danach an Harrys Kopf.

Harry wollte schon einen Knopf drücken, als ihm Fawkes zart in sein Ohr biss. \enquote{Au Fawkes. Lass das}, sagte Harry und sah ihn verärgert an. Doch Fawkes schüttelte den Kopf und nickte zweimal nach links und einmal nach unten. Harry stutzte. Dann sah er wieder auf die kleinen Knöpfe. Er folgte mit seinem Blick den Anweisungen Fawkes und sah einen Knopf, der wie ein Fernrohr aussah. Doch da gab es noch einen, der ähnlich war und daneben ein Turm abgebildet war. Harry legte seinen Finger auf den Knopf und sah Fawkes an. \enquote{Diesen hier?} Der Phönix nickte und Harry drückte ihn (den Knopf). Die Fläche oberhalb der Knöpfe leuchtete gelblich auf. Harry wusste nicht, was das zu bedeuten hatte. Nach einigen Sekunden hörte das Leuchten auf und der Knopf sprang wieder heraus. Fawkes gab einen enttäuschenden Laut von sich. Harry sah den Vogel noch einmal an und drückte danach erneut den Knopf. Wieder leuchtete die Fläche auf, aber dieses mal legte Harry seine Hand darauf. Das Leuchten veränderte sich und begann zu pulsieren. Danach leuchtete es in einem dauerhaften Rotton und der Knopf sprang erneut heraus. Resigniert wollte Harry einen anderen Knopf drücken, als sich Fawkes mit seinen Krallen in Harry Schulter fester hielt und ihn zusammenzucken ließ.

Sauer sah er den Vogel an, der nun den Kopf schüttelte und einen kleinen Laut von sich gab. \gedanke{Also gut}, dachte Harry, \gedanke{dann eben noch einmal.} Harry drückte abermals den Knopf, welchen Fawkes ihm gezeigt hatte und legte danach seine Hand auf die leuchtende Fläche. Dieses Mal jedoch begann Fawkes seine Gesänge und die Fläche leuchtete Grün auf.

Eine unheimliche Stimme erklang. \stimme{Berechtigung erteilt.} Danach verstummte sie wieder.

Harry hörte sie klar und deutlich, hatte aber das untrügliche Gefühl, nur er würde sie hören. Aber das konnte er nicht nachprüfen, da außer Fawkes niemand sonst hier war.

Die Fläche leuchtete nun grün. Harry spürte ein leichtes, aber kurzes Kribbeln. Der Boden begann leicht zu vibrieren und Harry nahm seine Hand herunter. Danach verschwand das grüne Leuchten.

Kurz darauf öffnete sich die Wand vor ihm und Harry trat vorsichtig heraus. Fawkes hatte inzwischen seine Krallen wieder etwas eingezogen und hatte sich wenige Zentimeter von Harrys Schulter entfernt, um sich daneben festzuhalten. Harry sah ein Fernrohr, dessen Ende in eine Halbkugel zu führen schien. Sie bildete eine Art Sessel, erkannte er, als er sich um sie herum bewegte. Er befand sich im obersten Stockwerk eines Turmes. Ihm kam der Ort unbekannt vor. Er war verlassen. Harry hörte keine Stimmen. Es lag auch kein Staub herum. \gedanke{Entweder wird der Ort noch benutzt, oder er wurde verzaubert, dass sich kein Staub darauf absetzen konnte.}

Außer dem Fernrohr und kahlen Wänden, an denen ein einzelnes Poster über die verschiedenen Mondphasen hing, war der Raum leer. Er entdeckte eine Wendeltreppe, welche er hinunter in das nächste Stockwerk ging. Dort standen einige Reihen voller Bücher sämtlicher Stilrichtungen. Harry konnte auch einiges an Muggelliteratur erkennen. Auch hier waren die Wände kahl und der Raum sonst leer. Ein Stuhl stand vor einem Schreibtisch und einem Fenster, sodass das Licht direkt auf die Bücher fiel, die da lagen und somit leicht gelesen werden konnten. Harry konnte jetzt erkennen, dass der Raum in der nächst tieferen Etage breiter wurde, aber er konnte sich nicht über das Geländer lehnen, um runter zu sehen.

Also trat Harry die Wendeltreppe eine weitere Etage hinunter. Er hielt sich am Geländer fest, da Fawkes doch auf seiner Schulter drückte. So langsam dämmerte ihm, wo er war. Als er den Boden erreicht hatte, sah er wieder die ihm bekannten storchenbeinigen Tische mit den filigranen silbernen Gerätschaften und der schweren Eichentür. Sein Blick erhob sich und er sah die ebenso bekannten Bilder der schlafenden Direktoren Hogwarts. \enquote{Ich bin in Dumbledores Büro}, kam es aus Harry vor Staunen raus. Er drehte sich so, dass Fawkes auf seine Stange laufen konnte und drehte sich wieder um. Sein Blick wanderte durch den leeren Raum. \gedanke{Ich muss hier raus}, dachte Harry. Plötzlich spürte er etwas Warmes auf seiner Schulter. Er schaute sie an und entdeckte, das Fawkes wenige Tränen auf seiner Schulter niederließ. Es brannte leicht, doch Harry wich nicht zurück. Als Fawkes wieder seinen Kopf erhob, drehte sich Harry zu ihm und sah ihn dankbar an. Er fuhr mit seiner Hand über seine Schulter. Sie war scheinbar vollständig geheilt. Unter seiner Kleidung konnte er allerdings nicht mehr erkennen. Mit beiden Händen strich er Fawkes jetzt über sein Gefieder. Der Phönix stimmte wieder seinen Gesang an. Dieses Mal einen anderen. Er steckte beide Flügel weit von sich und Harry fuhr ihm knapp unter den Flügeln über seinen Bauch. Er spürte ein tiefes inneres Gefühl der Ruhe und Ausgeglichenheit. Dann strich er mit Daumen und Zeigefinger auf der Ober- und Unterseite an Fawkes Flügeln entlang. Der Phönix schaute ihn verträumt an und Harry konnte nicht anders, als nur zu lächeln und ihn ebenso verträumt anzusehen.

Die beiden waren so miteinander beschäftigt, dass sie nicht mitbekamen, dass hinter ihnen die Tür aufging und Professor Dumbledore mit Professor McGonagall in den Raum kam. Sie schienen sich zu unterhalten, als sie plötzlich mitten im Satz verstummten, da sie Harry sahen, wie er Fawkes über sein Gefieder strich.

\enquote{Harry?}

Plötzlich wurde er aus seinen Gedanken gerissen. Fawkes klappte seine Flügel schuldbewusst zusammen und Harry wirbelte herum. \enquote{Professor, Professor!} war alles, was er herausbekam. \enquote{Fawkes hat\abs ich war\abs bin\abs} Harry atmete einmal tief durch, während er interessiert den Boden betrachtete. Dann sah er Dumbledore direkt in die Augen und begann zu erzählen. \enquote{Ich bin gerade durchs Schloss gelaufen, als ich nicht mehr wusste, wo ich war. Dann bin ich Fawkes begegnet. Ich habe ihn gestreichelt und wollte ihn zurückbringen. Er hat mich dann hereingebracht.} Danach sah er wieder zu Boden. \enquote{Ich werde jetzt gehen und nachher meine Strafarbeiten bei Professor McGonagall abholen.} Er drehte sich nochmals um und sagte. \enquote{Danke, Fawkes.} Dann durchquerte er Dumbledores Büro.

Er hielt den Türgriff in der Hand, als Dumbledore sagte: \enquote{Harry.} Dieser drehte sich um.

Dumbledore und McGonagall hatten sich bereits gesetzt und Professor McGonagall sagte ihm wortlos, nur durch ihre Gestik, dass er sich neben sie setzen möge. Dumbledore saß wie immer in seinem Stuhl hinter dem Schreibtisch und hatte seine Fingerspitzen gegeneinander gelegt. Harry trottete mit einem Kloß im Hals auf die beiden zu und setzte sich.

\enquote{Wie sind Sie hier hereingekommen, Mister Potter?}, fragte ihn Professor McGonagall.

Dann sah Harry kurz zu Dumbledore und danach wieder zu Professor McGonagall. \enquote{Da ich schon mal hier bin, können Sie mir auch gleich sagen, was ich für Strafarbeiten bekomme.}

Professor McGonagall hob eine Augenbraue hoch.

\enquote{Hast du was angefasst?}, fragte ihn Dumbledore. \enquote{Nein\abs} Harrys Kopf wirbelte herum. \enquote{Doch\abs den Türgriff und\abs das Geländer.}

Dumbledore sah in an. \enquote{Was noch?}, fragte er. Harry sah zu Fawkes. \enquote{Außer Fawkes, der wohl nicht zählt\abs bei diesem Besuch nichts.}

Dumbledore lächelte Harry an. \enquote{Nun, Minerva. Harry ist in Ihrem Haus.} Er sah zu McGonagall.

Diese ließ ihren Blick nicht von Harry ab. Schweigend sah sie Harry an. Am liebsten würde er jetzt im Boden versinken.

\fluestern{Wenn sie doch nur schreien würde, oder zumindest irgendetwas sagen würde.}

\enquote{Oh, ich sage etwas, Mister Potter.}

Jetzt war Harry wieder bei vollem Bewusstsein. \enquote{Habe\abs ich\abs etwa\abs etwa laut geredet Professor?}

\enquote{Ja Mister Potter, zwar nicht sehr laut, aber ich habe es doch gehört.}

Harry schluckte wieder, zwang sich aber Professor McGonagall anzusehen. Er konnte für den Bruchteil einer Sekunde ein Lächeln erkennen. Jetzt riskierte er es. \enquote{Das habe ich gesehen}, sagte Harry frech.

\enquote{Was?} fragte Professor McGonagall.

\enquote{Dass Sie mich angelächelt haben. Was ist jetzt mit meiner Strafarbeit?}

\enquote{Gibt es keine und nun gehen Sie Mister Potter}, sagte sie.

\enquote{Danke, Professor.} Harry sprang auf und gab ihr einen Kuss auf ihre Wange. Dann rannte er zur Tür, öffnete sie und sah noch einmal kurz zu Dumbledore. Dann schloss er die Tür, lauschte aber an selbiger.

\enquote{Mir scheint, dass Sie einen Verehrer gefunden haben, Minerva}, sagte Dumbledore.

\enquote{Seien Sie ruhig, Albus}, schnauzte sie Dumbledore an.

Harry musste grinsen. Er verließ seinen Horchposten und ging kurz in seinem Zimmer vorbei, um seine Schulsachen zu holen. Dann setzte er sich in die Bibliothek zu seinen Freunden, um die Hausaufgaben zu erledigen.

Kurz nach dem Abendessen ging er in einen selten benutzen Flügel des Schlosses. Er sah zu einem Fenster hinaus auf den See, wo der Krake gerade seine Bahnen zog.

\enquote{Kreacher?}, rief Harry mehr fragend als bestimmend in die kühle Luft hinaus.

Der Elf erschien und verbeugte sich. \enquote{Was kann Kreacher für seinen Herrn tun?}

\enquote{Erzähl mir etwas über den Umhang, den du mir zu meinem Valentins-Ausflug gegeben hast.}

\enquote{Sir Harry will sicherlich etwas über diese Runen wissen.}

\enquote{Ja Kreacher. Und warum die Familie Black solch eine Angst hervorruft. Als ich in einem Laden in Hogsmeade stand}, er sah Kreacher nun an, \enquote{hatte der Verkäufer scheinbar Angst vor mir.}

\enquote{Da hat der Verkäufer richtig gehandelt, falls er euch täuschen wollte. Diese Kleidungsstücke sind sehr selten und kostbar. Es geht eine Magie von ihnen aus, die den Träger spüren lässt, wie der gegenüber eingestellt ist. Die Familie Black hat ihn in früheren Zeiten immer zu Verhandlungen angezogen. Kreacher dachte, es sei ein passendes Kleidungsstück für ein erstes Date. Viele Frauen wollen sich einem nur des Geldes und des Ruhmes wegen nähern.}

\enquote{Was passiert, wenn einem der Gegenüber wohlgesonnen ist und keine bösen Gedanken oder Absichten hat?}

\enquote{Dann wirkt die Magie nicht. Man spürt nichts. Es ist so, als wenn man den Umhang nicht tragen würde.}

\enquote{Dann hat Katharina keine Hintergedanken}, meinte Harry mehr zu sich selbst und sah wieder zum Fenster hinaus.

\enquote{Das hat Kreacher auch gespürt}, sagte der Elf und zog ruckartig den Kopf zwischen seine Schultern, bevor er ihn wieder herausnahm.

Harry bekam davon nichts mit. Er reagierte kurz darauf, indem er wieder zu Kreacher sah. \enquote{Du hast sie überwacht?}

\enquote{Kreacher wäre ein schlechter Elf, wenn ich die gröbsten und groben Sachen, die man leicht herausfinden kann und die meinen Herrn vor Schaden bewahren können, nicht machen würde. Kreacher hat nicht in den privaten Sachen von Miss Chapel geschnüffelt, oder andere Verletzungen der Privatsphäre vorgenommen. Er hat sich freiwillig gemeldet, um ihr Zimmer ein paar Mal zu säubern. Und auch sonst hat er ein paar Gespräche mitbekommen. Kreacher schweigt aber über das Erfahrene, sofern es nicht seinem Herrn schadet.}

Harry nickte. \enquote{Das hast du gut gemacht, Kreacher. Über die kleine Einmischung hättest du mich aber vorher informieren können.}

Kreacher nickte und verstand.

\enquote{Was tun wir heute noch?}, fragte Harry, als er wieder zum Fenster hinaussah. Er meinte damit nicht einmal seinen Elfen, sondern mehr sich selbst.

\enquote{Kreacher würde gerne zum Grimmauldplatz zurückkehren, um dort seinen privaten Besitz zu holen.}

Harry sah seinen Elfen wieder an. \enquote{Verrätst du mir, was es ist?}

\enquote{Kreacher würde es vorziehen, es für sich zu behalten.}

\enquote{Essensreste, oder vielleicht alte Lumpen?}

Kreacher schüttelte seinen Kopf.

\enquote{Von mir aus. Wenn es dir gehört. Aber warum fragst du mich?}

\enquote{Sir Harry hat das Aufenthalts-Bestimmungsrecht für Kreacher. Wenn Sir Harry sagt, dass Kreacher in Hogwarts bleiben soll, dann darf Kreacher nicht von sich aus woanders hin. Es sei denn, sein Herr schwebt in Gefahr und diese kann dadurch abgewendet werden.}

Harry verstand. \enquote{Du darfst jederzeit in den Grimmauldplatz, um etwas zu holen, abzulegen, oder etwas zu erledigen, falls es mir oder dem Haus nicht schadet. Auch keinen Freunden von mir.}

\enquote{Das würde Kreacher niemals tun, Sir. Kreacher dankt euch.} Dann verschwand der Elf.

\trenn

Bald war wieder Frühlingsanfang, die Blumen begannen schon seit einiger Zeit ihre Blüten zu zeigen und Harry saß wieder im Unterricht. Professor Snape hielt gerade einen Vortrag über Kneuzers, als die Tür zum Klassenzimmer aufging und unter großem Staunen und Gemurmel Professor Elber das Zimmer betrat.

\enquote{Lassen Sie sich nicht stören Severus. Ich will mir nur einen kurzen Überblick über die vergangenen Wochen verschaffen und komme erst gegen morgen Nachmittag zum Unterrichten. Poppy hat mich zwar entlassen, mir aber noch einen Tag Ruhe verordnet. Also bringe ich mich erst einmal auf den neuesten Stand.}

Leichtfüßig schritt er durch die Klasse, obwohl er einen leicht müden und erschöpften Eindruck machte. Er ging an Professor Snape vorbei und öffnete die Schublade des Pults, nahm das kleine Buch heraus und ging die Treppen zu seinem Büro hoch, nachdem er die Schublade wieder geschlossen hatte. Er verschwand darin und ließ Professor Snape die Klasse weiter unterrichten. Professor Snape unterrichtete weiter, als ob nichts geschehen wäre. Nach einigen Minuten kam Professor Elber wieder aus seinem Büro, legt das kleine Buch zurück in das Pult, bedankte sich bei Professor Snape und verschwand wieder.

Professor Snape wandte sich wieder der Klasse zu und führte seinen Unterricht über die Kneuzers fort. \enquote{Diese Kreaturen haben die Eigenschaft, Halluzinationen und Verwirrung hervorzurufen. Je näher man einem Kneuzer kommt, desto realer werden diese Irritationen.} Die Schulglocke läutete und Professor Snape fügte hinzu. \enquote{Bis morgen Früh eine Zeichnung über Kneuzers mit exakter Beschriftung der einzelnen Körperteile. Sie dürfen gehen.} Alle stöhnten und verließen das Klassenzimmer. Draußen wartete Professor Elber, der jetzt ein paar Notizen in der Hand hielt. Professor Snape kam nun aus dem Klassenzimmer und Professor Elber fing ihn ab.

\enquote{Severus? Auf ein Wort, oder zwei. Ich habe mir den Stoff, den Sie unterrichteten, angeschaut und muss sagen, ich bin beeindruckt. Die Änderungen, die Sie gemacht haben, haben dem Ganzen noch die nötige Würze gegeben.}

\enquote{Ich freue mich, dass es Ihnen gefällt.}

\enquote{Ja, durchaus. Aber wie sieht es mit den Samstagskursen aus? Haben sie die auch gehalten?}

\enquote{Die hat Professor Dumbledore übernommen.}

\enquote{Ach}, und er grinste.

\trenn

\enquote{Ist es jedes Jahr hier so stressig für die Lehrer?}, fragte Elber Dumbledore.

\enquote{Nein, nein}, kam von Dumbledore.

\enquote{Aber wieso sind dann dieses Jahr so viele schwarz-magische Vorgänge zu finden? Wir sind doch auf dem Weg zur Krankenstation, also muss wieder etwas vorgefallen sein \gst Albus, das ist mir nicht geheuer. Ich bin mittlerweile mehr mit anderen Sachen, als mit dem Lehrauftrag beschäftigt. Ich bin dabei, Harry zu unterrichten, er lernt übrigens sehr gut, habe aber sonst Sachen zu tun, für die ich nicht hier bin.}

\enquote{Aber, dass Sie Poppy dazu gebracht haben mit Einschränkungen schwarze Magie zu akzeptieren und es sogar bei mir mit Abstrichen geschafft haben \gst das rechne ich Ihnen hoch an.}

Elber lächelte ihn nur an und meinte: \enquote{Wie ich bereits meinen Schülern sagte: \inner{Es gibt keine schwarze Magie. Es gibt nur die Intention desjenigen, der einen Zauber ausspricht und die Ansichten der Auswirkungen. Magie hat keine Farbe.}}

\enquote{Wie ich hörte, haben Sie Harry viel beigebracht.}

\enquote{Ja.}

\enquote{Er ist richtig gut geworden und weiß schon viel.}

\enquote{Das sind lediglich Grundlagen.}

\enquote{Grundlagen?} Dumbledore blieb stehen. \enquote{Grundlagen?}

Auch Elber bleib stehen und sagte: \enquote{Ja. Wir fangen gerade an, uns in den fortgeschrittenen Bereich zu arbeiten.}

\enquote{Was verstehen Sie unter \accentuate{fortgeschrittenen Bereich}?}

Elber drehte sich wieder um und sah Richtung Krankenflügel. \enquote{Wie lange ist der Weg noch?}

\enquote{Etwa dreihundert Meter.}

Elber schnippte mit den Fingern und beide standen vor dem Krankenflügel. \enquote{Das meine ich mit \accentuate{fortgeschrittenem Bereich}}, antwortete er.

\enquote{Sie beherrschen die Magie der Elfen?}

\enquote{Warum auch nicht. Wir können viel voneinander lernen. Ich habe meinen beispielsweise Tricks beigebracht, die nur von Zauberern beherrscht werden. Auch können sie mit Zauberstäben umgehen. Es entlastet sie bei schwierigen Aufgaben.}

\enquote{Aber, Elfen dürfen doch gar keine Zauberstäbe\abs}

\enquote{Wer sagt das?}

\enquote{Das ist Gesetz.}

\enquote{Von wann?}

\enquote{1640, oder so.}

\enquote{Dann ist es nicht einmal gültig. Es gibt ein Gesetz von 932, das ausdrücklich jedem magischen Wesen den Gebrauch von Hilfsmitteln, magisch oder nicht, gestattet, wenn es eine Erleichterung darstellt und nicht zum Schaden der Gesellschaft ist. Also ist das neue Gesetz unwirksam. Es wurde meines Wissens nicht aufgehoben. Zudem müsste man für solch eine Aktion das gesamte Gamot-Kremium mit allen Mitgliedern einberufen, was seit 1602 nicht mehr stattgefunden hat.}

\enquote{Woher wissen Sie das?}

\enquote{Ich habe\abs darüber gelesen. In der Bibliothek meiner Familie gibt es viele Bücher über Geschichte, Gesetze und allerlei anderes.} Er drehte sich um und öffnete die Tür.

Drinnen saßen Madame Pomfrey und ihre Patientin. An der Schuluniform konnte man das Haus Slytherin erkennen. Ihre Haare verbarg sie unter einem bunten Handtuch. Als sie sich umdrehte um die beiden anzusehen, sah ihre linke Gesichtshälfte wie versteinert aus. Die andere Gesichtshälfte sah nicht glücklich aus.

\enquote{Alraunensaft?}, war Dumbledores erste Frage.

\enquote{Hilft nicht.}

\enquote{Was ist mit dem Handtuch?}

\enquote{Das hat sie zu einem Turban gebunden, um die anderen nicht zu versteinern. Auf ihrem Kopf haben sich lauter Schlangen gebildet.}

\enquote{Welche anderen?}, fragte Elber.

\enquote{Die anderen hinter dem Vorhang sind ganz versteinert. Miss Chapel hier versicherte mir glaubhaft, dass es was mit ihren Haaren zu tun hat. Sie wurde vermutlich Opfer eines Streiches. Ihre Haare sind laut ihrer Aussage zu Schlangen geworden.}

\enquote{Medusa?}, fragte Elber.

\enquote{Ja}, antwortete Katharina Chapel.

\enquote{Vermutlich ein schlechter Scherz eines Mitschülers}, meinte Madame Pomfrey.

\chapter{Prüfungen und Erkenntnisse}


Professor Elber sah sie nachdenklich an und setzte sich auf das Bett gegenüber, um das Mädchen anzuschauen.

\enquote{Wie haben Sie es bemerkt? Ich meine das mit Ihren Haaren und wieso ist eine Hälfte Ihres Gesichtes versteinert?}

\enquote{Bemerkt habe ich es, als ich aufgestanden bin und mich Marlene angesehen hatte und sofort versteinerte. Dasselbe ist mit Sandra passiert, als sie ins Zimmer kam. Pansy war glücklicherweise noch draußen. Ich wollte schon zu meinem Handspiegel greifen, entschied mich dann aber, die Augen zuzumachen und nur kurz eines zu öffnen. Das reichte schon, um eine Gesichtshälfte zu versteinern. Ich habe meine Haare sofort eingewickelt und bin zu Madame Pomfrey gekommen und habe ihr von den anderen erzählt. Jetzt liegen sie hier.}

\enquote{Können Sie mir die Schlangen auf Ihrem Kopf beschreiben?}, fragte er weiter.

\enquote{Ich habe sie nur kurz gesehen. Sie waren grau \gst grün. Es waren viele kleine Schlangen. Genauer weiß ich es nicht.}

\enquote{Darf ich in diesem Fall Ihre Erinnerungen daran sehen?}

\enquote{Falls es Ihnen hilft.} Sie nahm ihren Zauberstab und wollte ihn an ihre Schläfe setzen.

\enquote{Nein, das meinte ich nicht. Ich werde sie mir direkt in Ihrem Kopf ansehen.}

Leicht verunsichert meinte sie: \enquote{OK.}

\enquote{Keine Angst. Denken Sie nur an das Bild von heute Morgen. Ab dem Zeitpunkt, wo sie aufgestanden sind. Und falls sie sich schämen, ich werde auf diesem Wege die unwichtigen Sachen vergessen. In einem Denkarium würde ich sie nicht so leicht vergessen.}

Katharina nickte und entspannte sich. Sie stellte sich den Zeitpunkt des Aufstehens vor und spürte eine Präsenz in ihrem Kopf. Dann durchlebt sie die Szene ein zweites Mal. Zuerst entdeckte sie Marlene, die gerade in ihrem Bett aufrecht saß und sie anstarrte und dann versteinerte. Dann kam Sandra herein und versteinerte ebenfalls, nachdem sie Marlene sah und dann aufgeregt Katharina fragen wollte, was denn passiert sei. Dann griff Katharina nach einem Handspiegel und es wurde kurz dunkel. Schließlich öffnete sie ein einzelnes Auge und erblickte ihr Spiegelbild. Die Präsenz in ihrem Kopf schien das Bild anzuhalten und nun konnte Katharina sich selber für eine längere Zeit sehen. Dann wandelte sich das Spiegelbild und ihr Professor erschien im Spiegel.

\enquote{Ich glaube immer weniger, dass es sich um einen Scherz eines oder mehrerer Ihrer Mitschüler handelt.}

Das Spiegelbild wandelte sich wieder und die Realität kehrte zurück. Sie war wieder in der Krankenstation und saß Professor Elber gegenüber, der sie immer noch nachdenklich ansah.

Ihre stumme Frage konnte er von ihren Augen ablesen: \gedanke{Was meinen Sie mit: \inner{Ich glaube immer weniger an einen Streich ihrer Mitschüler}?}

\enquote{Wie viel wissen Sie über ihre Familie? Genauer über Ihre Ahnentafel}, wollte Professor Elber wissen.

\enquote{Weshalb, Professor?}

\enquote{Ich habe einen Verdacht. Es ist nämlich so, dass bei weiblichen Nachfahren der Medusa unter bestimmten Umständen dieser Fluch wieder aufflammt.}

Erschrocken sah sie ihn an. Madame Pomfrey hatte mittlerweile einen Arm um sie gelegt, um sie zu trösten. Dumbledore saß auf der anderen Seite und sah sie großväterlich an.

\enquote{Kommen Sie nachher zu mir. Dann trinken wir zusammen eine Tasse Tee. Dann können Sie mir etwas über ihren Kopfschmuck erzählen}, sagte Dumbledore.

Katharina nickte und schniefte.

\enquote{Ich werde mal in der Bibliothek lesen und schauen, was man dagegen tun kann. Ich nehme an, Sie wollen nicht immer mit Hut oder Handtuch auf dem Kopf herumlaufen.} Katharina schüttelte den Kopf. \enquote{Ein kleiner Lichtblick bleibt Ihnen auf jeden Fall. Wenn Ihre Gesichtshälfte geheilt ist und Sie sich noch ein paar mal im Spiegel mit einem Auge betrachten, gewöhnen Sie sich daran. Dann können Sie, wenn Sie alleine sind, ihren Kopfbewuchs betrachten.} Dann stand er auf und verließ den Krankenflügel Richtung Bibliothek.

\enquote{Ich werde Ihre Eltern benachrichtigen}, sagte Madame Pomfrey und drehte sich um, um einen Brief zu schreiben.

\enquote{Bitte nicht}, jammerte Katharina

\enquote{Aber Ihre Eltern haben ein Recht darauf}, antwortete Madame Pomfrey, nachdem sie sich umgedreht hatte. Danach setzte sie ihren Weg zu ihrem Büro fort.

Katharina fing an zu weinen, was Dumbledore veranlasste, sich neben sie zu setzen.

\enquote{Sie werden mich bestimmt verstoßen. Sie sind sehr altmodisch und haben mich entsprechend erzogen. Außerdem halten sie viel vom reinen Blut und haben einen Hass auf alles, was kein Zauberer oder Hexe ist. Wenn sie erfahren, was mit mir passiert ist, brauche ich nicht mehr nach Hause zu kommen.}

\enquote{Sie meinen, es ist so schlimm?}, fragte Dumbledore vorsichtig nach.

Katharina nickte und fuhr fort. \enquote{Ich habe bei meiner Auswahlzeremonie immer wieder innerlich gefleht, nach Slytherin zu kommen, da ich anderenfalls nichts hätte, wo ich sonst in den Sommerferien hingehen könnte.}

Die Tür ging wieder auf und Professor Elber durchquerte das Zimmer. \enquote{’Tschuldigung, hab’ was vergessen \gst Was ist mit Ihnen los?}, fragte er.

\enquote{Sie hat Angst, verstoßen zu werden.}

Professor Elber nickte und lief weiter.

Dumbledore blickte wieder zu Katharina und bemerkte nicht, dass sich durch das Kopfschütteln und Weinen ihr Handtuch lockerte. Als er aus einem Augenwinkel eine Bewegung sah und reflexartig dort hinsah, erblickte er eine Schlange und erstarrte zu Stein.

Katharina wurde bleich, als sie das sah und wickelte ihr Handtuch sofort wieder fester um ihren Kopf und versuchte die Schlangen unter ihrem Handtuch zu verstecken. In ihrem Nachtschränkchen suchte sie nach einer Sicherheitsnadel und sicherte damit ihr Handtuch um ihren Kopf. Danach versucht sie ihren Schulleiter anzusprechen. Doch diese steinerne Statue gab kein einziges Wort von sich. Außerdem fühlte er sich kalt wie Stein an.

\enquote{Das darf doch nicht wahr sein}, jammerte sie erneut.

Die Tür ging auf und Madame Pomfrey kam mit Professor Elber heraus. Als sie die steinerne Statue sahen, blieben sie stehen.

\enquote{Ich war das nicht\abs Ich meine, es war ein Unfall\abs mein Handtuch \gst} und wieder weinte Katharina. Die beiden Erwachsenen sahen sich nur kurz an, nickten einander zu und setzten sich links und rechts neben Katharina, nachdem sie die Statue von Professor Dumbledore auf ein Nachbarbett schweben gelassen hatten, und nahmen sie in den Arm.

Instinktiv drehte sich Katharina zu Madame Pomfrey und heulte hemmungslos auf die Schürze ihrer Krankenschwester. Doch nach wenigen Minuten hatte sie sich beruhigt und schlief einfach ein. Madame Pomfrey bettete sie vorsichtig und verabschiedete sich von beiden. Auch Professor Elber verließ die Krankenstation, um seinen ursprünglichen Weg zur Bibliothek fortzuführen.

Etwas später war Harry mit Elber alleine. Sie saßen auf einer Bank vor dem schwarzen See. Nur durch einen Kiesweg vom Ufer getrennt. Vor beiden lag ein runder, kugelförmiger Stein.

\enquote{Bereit?}, fragte Elber Harry.

\enquote{Ja.}

Professor Elber hob seine Hand über die Steinkugel und hatte kurz darauf die Kugel in der Hand. \enquote{Und jetzt Sie.}

Harry hob ebenfalls seine Hand über seine Kugel und sie schwebte in seine Hand. Allerdings etwas langsamer.

Professor Elber drehte seine Hand um, öffnete sie und zog sie nach unten weg. Die Kugel blieb in der Luft stehen. Harry machte es ihm nach. Er versuchte es zumindest, da die Kugel der Hand folgte und von dieser herunterrollte und zu Boden fiel.

\enquote{Nochmal}, sagte sein Lehrer. \enquote{Bis es klappt.}

Harry wunderte sich, was das Ganze sollte. Doch er hatte bisher immer erst später begriffen, warum er gerade diese Übungen machen musste. So auch bei dieser. Er würde schon noch erfahren, was es damit auf sich hatte. Das Schöne daran war, dass Harry einfach mal nachdenken konnte. Nicht wie im Unterricht. So fing er nach dem fünften fallen lassen der Kugel an nachzudenken, was er anders machen könnte. Er hielt die Kugel wieder in seiner Hand. Die letzten Male kam sie schneller in seine Hand. \gedanke{Würde bremsen helfen?}, dachte er sich. \gedanke{Wenn ich die Kugel einfach in der Luft bremse.} Er ließ seine Hand auf seinem Schoß, konzentrierte sich und ließ die Kugel wieder hochschnellen. Leider schoss sie in die Höhe, bis Harry sie nicht mehr sah. Professor Elber schob ihn etwas zur Seite und rutschte ebenfalls an den Rand der Bank. Harry kannte so etwas schon. Er wurde positioniert. Kurz darauf schlug die Kugel durch das mittlere Brett der Sitzfläche.

Harry wunderte sich nicht mehr darüber, dass sein Lehrer immer wusste, wo die Kugel oder andere Dinge ankamen.

\enquote{Woher wussten Sie, wo die Kugel runterkommt?}

\enquote{Wenn Sie die Kugel beherrschen, dann kennen Sie das Geheimnis}, sagte er schlicht. \enquote{Versuchen Sie es noch einmal.}

Harry nahm die Kugel wieder in die Hand. Er musste sie nicht mehr anheben. Er überlegte. Lange.

\stimme{Denke einfach, Harry.}

\enquote{Sal!}

\enquote{Was meinen Sie?}, fragte ihn sein Professor.

\enquote{Ach, ich habe nur laut gedacht.}

\enquote{Ach so.}

\gedanke{Und was soll ich denken?}

\stimme{Dass der Zauber funktionieren soll.}

\gedanke{Wie?}

\stimme{Denke einfach an den Zauber. Oder besser: Denke einfach an das, was passieren soll.} Gedankenversunken nickte Harry.

Er sah die Kugel erneut an und senkte danach seine Hand. Die Kugel gab etwas nach, blieb aber mit einer ständigen leichten Auf- und Abwärtsbewegung in der Luft schweben.

\gedanke{Einfach an das denken, was passieren soll}, ging ihm durch den Kopf.

\enquote{Gut, dann können Sie das jetzt üben, wenn Sie alleine sind. Sie wissen doch noch, was wir besprochen hatten?}

\enquote{Ja. Kein Wort zu niemandem. Selbst zu Ron und Hermine nicht.}

\enquote{Exakt.}

Harrys Blick war immer noch auf die Kugel gerichtet. Er wollte gerade fragen, wie er denn jetzt an das Geheimnis kommen konnte. Doch bevor er seine Frage stellen konnte, hatte er einen Verdacht. Er konzentrierte sich und ließ die Kugel nach oben schnellen. Im Geiste verfolgte er die Kugel und wusste nun, wo sie auftreffen würde. Er suchte die Stelle auf dem Kiesweg und wartete, bis die Kugel aufkam. Er hatte sich nicht geirrt. Leicht lächelnd sah er auf die Kugel und ließ sie wieder in der Luft schweben.

\enquote{Ich glaube, ich habe es verstanden.}

Professor Elber lies seine Kugel in die Luft schnellen und fragte dann: \enquote{Wo kommt sie auf?}

Harry dachte kurz nach, stand schnell auf und die Kugel schlug hinter ihm durch das hintere Brett der Bank.

Mit einem einfachen Zauber reparierte Elber die Bretter wieder und machte sich wortlos auf den Rückweg.

Harry wusste, dass die Stunde jetzt zu Ende war. Er grinste in sich hinein, dass er es wieder einmal geschafft hatte. Nur musste er im Unterricht aufpassen, dass er nicht zu viel zeigte. Er hatte teilweise mehr Zauber drauf als seine Mitschüler. Sein Privatunterricht hatte sich bezahlt gemacht. Aber er wusste immer noch nicht, worauf die Übungen abzielten. Trotzdem war er froh, nach einigen Terminen endlich den richtigen Dreh raus zu haben.

Einige Tage später, Madame Pomfrey verarztete gerade die Hand von Elber, fragte sie: \enquote{Was hast du wieder angestellt?}

\enquote{Ich war unvorsichtig und tat in der Bibliothek einen falschen Griff und \gst}

\enquote{Hast nicht aufgepasst}, vervollständigte Madame Pomfrey den Satz.

Die Tür ging auf und zwei elegant und leicht adelig-arrogant wirkende Personen betraten den Raum. \enquote{Wo ist sie?}, fragte der Mann in einem Ton, aus dem man anwidern heraushören konnte.

Hinter ihnen lief Professor Sprout und drückte sich in eine Ecke, um nicht aufzufallen.

Madame Pomfrey stand auf und begrüßte die beiden Personen. \enquote{Ich nehme an, Sie sind die Eltern von Miss Chapel.}

Die Gesichtszüge der beiden versteinerte sich. \enquote{Wir möchten sie sehen, um uns von ihrem Zustand zu überzeugen}, antwortete die Frau scharf.

\enquote{Sind Sie ihre Eltern? Denn sonst darf ich Sie nicht durchlassen.}

Die beiden sahen sich kurz an. Dann nickte die Frau Madame Pomfrey zu. \enquote{Wir sind ihre Eltern}, antwortete sie und schluckte.

Skeptisch führte sie Madame Pomfrey zum Bett mit dem Vorhang und zog ihn vorsichtig zurück. Dahinter lag Katharina mit ihrem Handtuch um den Kopf gewunden auf dem Bett und hatte ihre Augen zu.

Sie öffnete sie und sah ihre Eltern an. \enquote{Mama, Papa}, begrüßte sie die Beiden, doch diese sahen sie nur streng an.

\enquote{Stimmt es, was man uns schriftlich mitteilte?}

Katharina nickte nur. Sofort zog ihr Vater seinen Zauberstab und richtete ihn auf das Handtuch. Nach einem gemurmelten Zauber und einem kurzen intensiven rosa Schimmer um das Handtuch herum,  zog ihre Mutter ein Päckchen aus ihrem Umhang heraus, warf es auf das Bett und sagte: \enquote{Du bist nicht mehr unsere Tochter. Wage es ja nicht, nach Hause zu kommen. Zu deinen Großeltern brauchst du ebenfalls nicht zu kommen.}

Und ihr Vater fügte hinzu: \enquote{Du bist kein Mitglied unserer Familie mehr. Wir möchten kein solches\abs in unserer Familie haben.}

Dann drehten sich beide um und verließen ohne ein Wort den Krankenflügel, um danach das Schlossgelände zu verlassen.

Katharina liefen wieder die Tränen übers Gesicht. Madame Pomfrey war zu geschockt, um etwas zu unternehmen, so kam Professor Sprout aus ihrer Ecke und nahm sich Katharinas an.

\enquote{Weinen Sie nicht, Miss Chapel. Wissen Sie, ich habe keine Kinder und wollte schon immer welche. Wenn Sie wollen, dann nehme ich Sie bei mir auf. Ich habe Sie die letzten Jahre kennengelernt und möchte Sie gerne adoptieren.}

Katharinas Tränen versiegten. \enquote{Wirklich?}, fragte sie ganz entgeistert.

\enquote{Ja Katharina, wenn du möchtest, gerne, dann bin ich aber ab sofort Pomona für dich, denn ich nehme an, dass du mich nicht Mum nennen wirst.}

Katharina wollte gerade ja sagen, da unterbrach sie Madame Pomfrey: \enquote{Sie müssen wissen, dass wenn Sie jetzt ja sagen, die magische Bindung sofort ausgeführt werden wird.}

Katharina überlegte nicht eine knappe Minute und sagte dann schließlich: \enquote{Ja, gerne \gst Pomona} und wurde leicht rot.

Diese nahm sie in den Arm und beide begannen leicht zu leuchten.

Als das Leuchten nachließ, erhob sich Professor Sprout und meinte: \enquote{Ich muss jetzt in den Unterricht, aber nachher komme ich wieder vorbei und hole dich ab. Dann unterhalten wir uns bei mir. Da ich jetzt für dich verantwortlich bin, werden wir die Ferien miteinander verbringen. Also werden wir uns erst einmal richtig kennenlernen.}

Katharina nickte und sah ihrer neuen Adoptivmutter gedankenversunken nach. Dann wanderte ihr Blick Richtung Professor Elber und sah ihn an. Es schien, dass ihr Blick ins Leere lief und sie ihn gar nicht richtig wahr nahm. Dann hörte sie eine Stimme in ihrem Inneren. \stimme{Ich habe mich schlau gemacht. Ich kann dir nicht helfen.} Katharinas Mut sank wieder. Und sie dachte nach, wollte ihren Entschluss, sich adoptieren zu lassen, rückgängig machen. Sie wollte Professor \gst Pomona nicht damit belasten, nie ihre Haare sehen zu können. \stimme{Aber jemand anderes kann dir helfen.} Ihre Augen schienen jetzt die ihres Lehrers zu finden. \stimme{Du musst nach Griechenland reisen, in den Tempel der Medusa. Der dortige Orden kann dir helfen. Pomona kann dich bis zum Eingang des Tempels begleiten. Ab da musst du alleine durch, aber du wirst im Inneren des Tempels von einem Führer begleitet. Du wirst dort, wie die Mönche, ins Innerste des Tempels vordringen, um Hilfe zu erlangen.}

Jetzt fixierten Katharinas Augen die von Professor Elber und starrten ihn an. Dann hörte er in seinem Geist: \stimme{Ich hätte Professor \gst Pomona nicht mit dieser Aufgabe belasten sollen.}

\stimme{Unsinn. Sie macht das gerne. Denke daran, dass sie dich trotz deines Zustandes akzeptiert hat und immer noch aufnehmen will.}

\stimme{Da hast du vermutlich recht.} Sie merkte nicht, dass sie ihren Lehrer duzte. Als es ihr auffiel, wurde sie rot und versuchte eine Entschuldigung zu stammeln.

\stimme{Mach dir nichts draus, Katharina. Wir unterhalten uns gedanklich. Das ist die wohl intimste Art einer Unterhaltung zwischen zwei Menschen. Wenn wir uns hier also duzen, ist das mehr als nur ok. Ich persönlich sehe es als Voraussetzung. \gst Du hast die Art der Kommunikation aber sehr schnell gelernt. Oder hast du vorher schon geübt?}

\stimme{Ich hatte vorher keine Ahnung davon.} Sie verlor ihre Scheu. \stimme{Frederick.}

Er lächelte sie leicht an, und legte sich danach hin und ließ die Salbe unter seinem Verband, um sie wirken zu lassen, da er mit seinem Arm eh nichts tun konnte. Katharina legte sich ebenfalls hin und ließ ihre Gedanken schweifen.

\stimme{Denk etwas leiser, Katharina}, hörte sie in ihrem Geist und meinte ein leises Schmunzeln zu erkennen.

\enquote{Was haben Sie gehört?}, fragte sie nach.

\enquote{Nichts, ich wollte Sie nur Testen, damit ich weiß, wie lange Sie das schon machen. Außerdem war es ein netter Scherz, oder?}

Katharina musste lächeln und schloss ihre Augen. Auf merkwürdige Weise fühlte sie sich mit ihren beiden Lehrern und der Krankenschwester verbunden.

\enquote{Wir werden Sie überwachen, damit wir einen Hinweis erhalten, falls die Mönche und Schwestern Ihnen nicht helfen können. Sie bekommen einen Zauber auferlegt der es ihnen ermöglicht, wieder zurückzukehren. Einen internen Portschlüssel quasi.}

\trenn

Leicht schwankend tauchten sie aus dem Nichts mit einem Dreher auf und Pomona ließ den unansehnlichen Nachttopf fallen, um ihr eigenes Gleichgewicht zu finden. Katharina ging sofort in die Knie, um den Schwung abzufangen und drehte sich einmal um ihre eigene Achse. Pomona hatte nicht so viel Glück und musste sich mit einem Baum bremsen, auf den sie zu steuerte.

Ein paar Meter weiter weg wartete ein in ein weißes Gewand gehüllter Mann und empfing sie freundlich. \enquote{Wenn sie mir bitte folgen würden.} Er drehte sich um und ging in Richtung eines Waldes. Katharina und Pomona folgten ihm. Schweigend liefen sie mehrere Minuten hinter ihm her, bis es an einen Platz im Wald ging, an dem eine Luke im Boden etwas verdeckte. Der Mann öffnete die Falltür und stieg hinab. Die beiden Frauen folgten ihm. Unten angekommen ging es einen Gang entlang. Nichts als festgetretener Dreck war zu sehen; am Boden, an den Wänden und dem bogenförmigen Dach des Tunnels. Sie kamen durch eine Tür in eine Kammer mit vielen Stühlen und Bänken. \enquote{Die Begleitung wartet hier bitte, der Prüfling durchquert bitte diese Tür. Ihr Wegbegleiter erwartet Sie dort drin \gst viel Erfolg.} Dann ging der Mann wieder den Gang entlang und verließ die beiden.

\enquote{Hör zu, wenn du willst, dann belege ich dich mit ein paar Schutzzaubern. Sie überwachen deine Vitalfunktionen und wenn es kritisch wird, dann kann ich dich damit zurückholen.}

\enquote{Damit würde ich mich wirklich besser fühlen, nach alldem, was ich bisher über solche Art von Ritualen gelesen habe. Ich werde vermutlich Prüfungen bestehen müssen, die arg an die Substanz gehen. Körperlich, wie auch seelisch. Es wird mir sicher helfen, jetzt da ich meine Familie ver\aabs Ich meine, jetzt, da ich eine neue Familie gefunden habe.}

Pomona nickt und zog ihren Zauberstab. Sie überzog Katharina mit mehreren Schutzzaubern, belegte sie mit dem Rückholzauber und lief danach zu den Stühlen.

\enquote{Ich gehe dann mal. Eine Prüfung wartet auf mich}, sagte Katharina ganz entschlossen.

\enquote{Na dann}, sagte Pomona fröhlich, \enquote{du weißt, wie du zurückkommst?}, zog sich ein Buch heraus und setzte sich.

Katharina wurde zunehmend nervös und öffnete vorsichtig die Tür, um zu sehen, was dahinter wäre.

\trenn

Zeitgleich in Hogwarts lief Professor Elber sichtlich angespannt im Lehrerzimmer auf und ab.

\enquote{Nun setzt dich endlich, Frederick}, mahnte in Professor McGonagall.

\enquote{Ich kann nicht, ich hätte bei ihr bleiben sollen, hätte sie begleiten sollen}, antwortete er.

\enquote{Aber Pomona ist bei ihr. Sie ist jetzt für sie da und kümmert sich um sie.}

\enquote{Du verstehst das nicht, Minerva.} Er blieb stehen. \enquote{Meine Linie ist für ihren Zustand verantwortlich.}

\enquote{Wie meinst du das?}

\enquote{Meine Linie warf den Fluch auf die erste Medusa. Ich fühle mich für sie verantwortlich.}

Harry hingegen traf sich mit Professor Snape und Ron, Hermine und Ginny. Ihnen wollte er zeigen, was er über Salazar Slytherin herausgefunden hatte. Nachdem er sie hatte versprechen lassen, nichts zu sagen, nahm er sie mit Richtung Slytherin Gemeinschaftsraum. Vor dem Porträt von Slytherin angekommen, begrüßt dieser ihn jetzt etwas freundlicher.

\gedanke{Hat Sal also mit ihm gesprochen.}

\enquote{Hallo Harry, wen bringst du denn da mit?}

\enquote{Freunde\abs mehr oder weniger.}

Das Bild grinste ihn an. \enquote{Wollt ihr rein?}

Harry nickte, worauf das Bild von Slytherin zur Seite schwang.

\trenn

Katharina trat in eine große Höhle mit schweren steinernen Säulen, welche die Decke hielten. Symbole, die sie nicht kannte, waren in den Stein gemeißelt. Sie trat um eine der Säulen herum und entdeckte zwei verhüllte Gestalten von hinten, die wie Priester oder Gelehrte aussahen. Sie wollte sich ihnen gerade nähern, als sie von hinten Geräusche hörte. Sie drehte sich um und lief einige Schritte auf den Mann zu, der sich ihr zu nähern schien. Aber dieser beachtete sie nicht und lief an ihr vorbei in einen der Gänge, die von der großen Höhle abführten. Also drehte sie sich wieder herum und versuchte die beiden Gestalten anzusprechen, welche sie zuerst gesehen hatte, als sie die Höhle betreten hatte. Doch diese gingen bereits in einen anderen Gang.

Da hörte sie eine Stimme. Nein, es war eher ein Schnaufen und Keuchen. Sie näherte sich den Geräuschen und entdeckte eine junge Frau. Sie dürfte gerade erst die Volljährigkeit hinter sich gebracht haben. Diese versuchte an einem Seil einen Eimer voll Wasser aus einem Brunnen zu holen. Katharina vermutete, dass es sich um einen Eimer handeln musste, denn sie konnte nur das Seil sehen, welches über eine Umlenkrolle in das Innere des Brunnens führte. Der gemauerte Brunnen zeigte verschiedene Wesen. Alle gleich groß und alle auf derselben Ebene abgebildet. Auf der linken Seite konnte sie noch das Hinterteil eines Pferdes erkennen. \gedanke{Vermutlich ein Zentaur}, dachte sie. Daneben war ein Zauberer, der einen Muggel anblickte. Katharina vermutete dies, da eine der beiden Personen einen Zauberstab in der Hand hielt, die andere nicht. Daneben standen auf kleinen, unterschiedlich hohen Hockern ein Elf und ein Kobold. Ganz rechts auf dem runden Brunnen erkannte sie nur noch einen Flügel.

Sie wollte lieber nicht wissen, welchem Wesen dieser Flügel gehörte.

Sie kam näher und ging der jungen Frau zur Hand. Dann zog sie mit ihr am selben Seil, doch es bewegte sich nichts.

Schließlich gab sie auf uns zog ihren Zauberstab, richtete ihn in den Brunnen und sprach einen \spruch{Accio}. Was auch immer am Ende befestigt war, kam nun hoch.

Eine schwere Statue, die einen asiatischen Mönch zeigte, brach aus dem Brunnen hervor und zerstörte den hölzernen Oberbau, der das Seil und die Umlenkrolle hielt. Diese reparierte sie mit einem weiteren Zauber.

\enquote{Das ist aber ein schöner Zauberstab, den Sie da haben}, sagte die junge Frau, nahm ihn in ihre Hand und untersuchte ihn. \enquote{Weißdorn und Einhornhaar}, erkannte sie richtig.

Katharina nickte.

\enquote{Gut, wenn man einen hat}, sagte sie und steckte ihn ein.

\enquote{He, das ist meiner, geben Sie ihn zurück.}

\enquote{Nein, das werde ich nicht}, sagte die junge Frau und schaute sie abwartend an.

Katharina war verblüfft. Doch dann überkam sie die Erkenntnis. \enquote{Sie sind mein Führer.}

\enquote{Führer, Begleitperson, Hilfe auf dem Pfad. Wählen Sie eine Bezeichnung.}

\enquote{Warum haben Sie sich nicht schon früher zu erkennen gegeben?}

\enquote{Wollen wir?}, sagte sie, Katharina ignorierend.

Katharina nickte und die junge Frau ging voraus. Sie gingen einen schmalen Gang entlang, der in einer kleinen Kammer endete. Dort warteten bereits ein paar Frauen, die sich sofort Katharina näherten, als sie eintrat und begannen, sie zu entkleiden.

Katharina schreckte zurück und wollte den Raum schon verlassen, doch ihre Führerin beruhigte sie. \enquote{Es ist alles in Ordnung, Katharina.} Also gab diese nach, ging zurück und ließ sich vorbereiten.

Mit den Händen über den Brüsten und den Kopf nach hinten zu ihrer Führerin gerichtet, fragte sie: \enquote{Würden Sie irgendeinen Schwur brechen, wenn Sie mir sagen, was das Ritual beinhaltet?}

Währenddessen fingen die Helferinnen an, sie mit Schwämmen zu waschen und mit Symbolen zu bemalen.

\enquote{Warum glauben Sie, dass ich das weiß?}

\enquote{Haben Sie es nicht auch schon durchlaufen?}

\enquote{Keine Angst. Ich werde Ihnen helfen, Ihren Weg zu finden. \gst Sagen Sie mir, sind sie vollkommen bereit, diese Reise zu unternehmen?}

\enquote{Ja.}

\enquote{Sind Sie willens, all das zu durchleben, was Ihre Vorgänger seit Jahrhunderten durchlebt haben, um Hilfe zu finden?}

\enquote{Ja.}

Die Helferinnen begannen Katharina in neue Kleidung zu hüllen, die weit und bequem war.

\enquote{Damit Sie im Notfall abbrechen können, Ihre Schutz- und Verfolgungszauber, die Ihre Adoptivmutter auf Sie gelegt hat wirken können und Sie nicht sterben können, falls es schwierig werden sollte?}

Katharina schaute sie entgeistert und leicht verängstigt an.

\enquote{Wir haben hier die Möglichkeit, jeden Zauber zu entdecken.}

\enquote{Ich hoffe, das ist kein Problem?}

\enquote{Es spielt nicht die geringste Rolle. \gst Sie halten viel von der Magie, nicht wahr?}

\enquote{Sie leistet mir gute Dienste.}

\enquote{Das glaube ich gerne. \gst Kommen Sie mit.}

Sie liefen einen weiteren Gang entlang und die Führerin ließ Katharina den Vortritt, um eine weitere Kammer zu betreten. Kaum war sie über die Schwelle getreten, schloss sich die Schiebetür hinter ihr.

Im Inneren saßen drei Personen in der gleichen Kleidung. Sie hatte lediglich eine andere Farbe. Eine Frau und zwei Männer. Ein kurzer Blick genügte, um zu erkennen, dass der letzte der Männer, der neben der Frau auf der Bank saß, etwas mürrisch war.

Erschrocken von dem Geräusch der sich schließenden Tür, drehte sich Katharina um. Doch sie hatte nicht viel Zeit darüber nachzudenken, denn einer der Männer unterbrach ihren Gedankengang.

\enquote{Wer sind Sie?}, fragte er barsch.

\enquote{Ich bin Katharina Chapel}, sagte sie, als sie sich wieder umgedreht hatte. \enquote{Ist hier der Beginn des Rituals?} Sie sah sich weiter um und entdeckte eine weitere Tür.

\enquote{Oh, das Ritual}, sagte der andere Mann. Er hatte eine tiefe angenehme Stimme. \enquote{Ja.}

\enquote{Wir warten}, sagte die Frau. \enquote{Kommen Sie, gesellen Sie sich zu uns.}

Katharina ließ ihren Blick von der Tür ab und wandte sich den sitzenden Personen zu. \enquote{Dürfte ich erfahren, worauf Sie warten?}, fragte Katharina nach.

\enquote{Wir warten nur}, antwortete einer der Männer.

Katharina verarbeitete die Informationen und überlegte kurz. \enquote{Und wie lange warten Sie schon?}

\enquote{Wie lange sitzen wir schon hier?}, fragte der Mann seinen Nebensitzer.

\enquote{Was fragen Sie mich}, antwortete er aufbrausend. \enquote{Ich habe kein Zeitgefühl.}

\enquote{Es ist schon eine Weile}, sagte der Mann zu Katharina gewandt. \enquote{So viel weiß ich.}

\enquote{Oh wir warten hier, solange wir zurückzudenken vermögen}, antwortete die Frau.

Das war ein Schock für Katharina. \enquote{Soll das bedeuten, dass ich mein ganzes Leben lang warten muss, bis ich das Ritual durchlaufen kann?}

\enquote{Wir haben lediglich gesagt, dass wir hier warten}, merkte der ruhige Mann an.

\enquote{Ich möchte nur begreifen, wie es funktioniert. Die Mönche, die ich durch die Pforte gehen sah, waren jung. Sie können nicht lange gewartet haben, bis sie die Rituale durchliefen.}

\enquote{Damit hat sie recht}, erwiderte die Frau.

\enquote{Sie ist ein kluges Köpfchen.}

\enquote{Sie hält sich jedenfalls dafür}, sagte der andere Mann bissig.

\enquote{Das ist ein Test}, folgerte Katharina. \enquote{Ich soll meine Entschlossenheit beweisen.}

\enquote{Ein Test?}, fragte die Frau. \enquote{Sie glaubt, wir seien ein Test? Wovon spricht sie überhaupt?}

\enquote{Offenbar mag sie Tests. Ich nehme an, Tests sind wichtig für sie.}

\enquote{Hat schon mal jemand versucht diese Tür zu öffnen?}, fragte Katharina nach einer kleinen Pause.

\enquote{Wie oft sollen wir Ihnen noch sagen, dass wir hier lediglich warten}, raunzte einer der Männer.

\enquote{Meine Liebe, wieso setzen Sie sich nicht einfach zu uns und entspannen sich? Sie sind viel zu verkrampft.}

\enquote{Ich möchte nicht warten, ich habe zu Hause Freunde und Menschen, die mir etwas bedeuten. Ich kann nicht warten. Ich muss das Ritual durchlaufen.}

\enquote{Ich frage mich, ob sie immer vor Ungeduld platzt.}

\enquote{Oh, sie hat nur einen starken Willen. Sie möchte, dass es endlich vorwärtsgeht.}

Katharina wandte sich der Tür zu und klopfte, da sie sie nicht aufbrachte. Als die Tür aufging, staunte sie, da ihre junge Führerin sie hinter der Tür erwartete und sie begrüßte. \enquote{Ja?}

\enquote{Ich möchte nicht respektlos erscheinen, aber wenn ich hier nicht etwas tun muss, dann würde ich gerne mit dem Ritual weitermachen.}

\enquote{Wie Sie wünschen}, entgegnete die junge Frau und deutete mit ihrer Hand Katharina an, dass sie den Gang vorausgehen soll.

Sie gingen in einen Raum mit einer Empore, die rings herum von Stufen umsäumt war. In der Mitte war eine quaderförmige Vertiefung. Etwa so groß wie ein Mensch.

Katharina lief die Stufen hinauf und umlief die Empore. \enquote{Ich weiß nicht wie ich beginnen soll.}

\enquote{Sie glauben also, dass es nur darum geht zu tun, was man Ihnen sagt?}

\enquote{Nein, ich bin überzeugt davon, dass es meine Magie stärkt und den Fluch von mir nimmt. Ich habe viele Rituale studiert und viele haben Gemeinsamkeiten, aber dieses hier könnte komplett anders sein.}

\enquote{Das bezweifle ich nicht. Aber ist Ihnen bewusst, dass das hier alles bedeutungslos ist? \gst Wichtig ist einzig und alleine, dass Sie die Verbindung zur Magie finden.}

\enquote{Ich werde alles tun, um das zu erreichen. Aber ich bin nicht nur gekommen, um mich um meiner selbst willen von diesem Fluch zu befreien. Ich habe lieb gewonnene Menschen, die mir etwas bedeuten, versteinert. Ich bin für sie verantwortlich.}

\enquote{Ein ehrenwerter Grund. Ich hoffe, die Magie schenkt ihnen Gehör.}

Katharina konnte diesen Satz nicht nachvollziehen. \gedanke{Meine Magie gehört zu mir. Ich benutze sie. Wieso sollte sie mir nicht gehorchen?}

\enquote{Also, lassen Sie uns beginnen.} % 00:21:30

Die Führerin stieg die kleine Empore hoch und stellte sich gegenüber Katharina auf. \enquote{Also}, sprach sie. \enquote{Die erste Herausforderung \gst Stellen Sie sich so hin.} Sie stellte ihre Füße hüftbreit auseinander, schob den rechten Fuß nach vorne und ging leicht in die Knie. Katharina machte es ihr nach. Dann hob sie einen Stein auf, legte ihn in Katharinas Hand und zog die gefalteten Hände etwas zu sich. Der Stein ruhte nun in ihren übereinander liegenden Händen auf Brusthöhe.

\enquote{Nun \gst was sehen Sie.}

\enquote{Ich sehe \gst einen Stein}, antwortete Katharina.

\enquote{Und was weiter?}

Also sah Katharina den Stein intensiv an, atmete einmal kräftig durch und konzentrierte sich.

Unterdessen saß Pomona im Wartebereich und las an ihrem Buch weiter. Sie hatte inzwischen die Alarmzauber nach Hogwarts dupliziert, damit man auch dort zumindest etwas mitbekommen würde. Sie war gerade an einer spannenden Stelle, als einer der Alarmzauber anschlug und Sekunden später ein Pergament vor ihr erschien. Sie legte ihr Buch beiseite und las den Text.

Auf Hogwarts war Professor Elber aufgeregt, als ein Pergament vor Madame Pomfrey erschien. \enquote{Und, was ist mit ihr? Ist sie in Gefahr?}

\enquote{Nun mal langsam, Frederick. Lassen Sie mich erst mal lesen.} Gemütlich nahm sich Madame Pomfrey das Pergament vor. \enquote{Alles in Ordnung. Nur ein Anstieg der Milchsäure-Konzentration in den Streckmuskeln.}

\enquote{Also strengt sie etwas an!}, folgerte er. \enquote{Gefährliche Werte?}, fragte er nach.

\enquote{Nein, keinesfalls. Aber es könnte dem Fluch entgegenwirken, den sie in sich trägt.}

\enquote{Also nichts, was man als Ansatz für eine Heilung hernehmen könnte?}

\enquote{Nein.}

Zeitgleich hielt Katharina immer noch den Stein in ihren Händen. Schweiß rann von ihrer Stirn und die körperliche Erschöpfung stand ihr ins Gesicht geschrieben.

\enquote{Was sehen Sie denn jetzt?}, wurde sie gefragt.

\enquote{Ich sehe\abs immer noch\abs einen Stein.} Dann wurde sie ohnmächtig und viel auf eine weiche Unterlage.

Nachdem sie wieder wach wurde und einen neutral schmeckenden, aufbauenden Trank bekommen hatte, wurde eine weiße unbemalte Leinwand hereingebracht. Katharina musste sich setzen und anfangen zu malen. Am Boden standen Töpfchen mit Farben drin. Katharina griff hinein und bedeckte ihre Finger mit der blauen Farbe.

\enquote{Ich nehme an, Sie werden mir nicht verraten, was ich malen soll.}

\enquote{Das wäre zu einfach, nicht wahr? \gst Sie haben die Wahl. Malen Sie, wonach ihnen ist.}

\enquote{Eigentlich konnte ich nie malen. Meine Schwester war die Künstlerin in der Familie.}

\enquote{Und Sie die Wissenschaftlerin.}

\enquote{Ja das stimmt. Während andere Kinder spielten, löste ich Rätsel. Mathematische, sowie logische. Außerdem befasste ich mich mit unserer Familien-Geschichte, oder mit Zaubertränken.}

\enquote{Mathematik. Ich kann ihre Freude daran nachvollziehen. Ein Problem zu lösen, eine Antwort zu erhalten. Einen Trank zu brauen. Die Antwort ist falsch oder richtig. Ein Trank gut gebraut, oder Abfall. Es ist sehr absolut.}

\enquote{Ich fand es immer sehr befriedigend}, sagte Katharina und sah ihre Führerin an.

\enquote{Da bin ich mir sicher.}

Nach einer kleinen Pause, in der das Bild trocknete, wurde sie in eine Kammer geführt. Sie sollte ganz nach oben klettern und etwas herunterholen.

Katharina machte sich auf den Weg nach oben. Sie erklomm die erste kleine Ebene in drei Metern Höhe.

Langsam setzte sie einen Schritt über den anderen und kam stetig aber sicher voran. Doch je höher sie kletterte, desto unsicherer wurde sie. Immer schwerer kam sie voran und rutschte immer wieder mit Hand oder Fuß ab. Dann gab noch ein Felsvorsprung nach und rutschte ab, sodass sie nur noch an ihren Händen in der Wand hing und mit den Füßen sicheren Halt suchte. Doch sie schaffte es und setzte ihre Reise nach oben fort.

Doch oben war nichts. Nichts, außer einem schmalen Gang, der zu einer Treppe führte, welche Wendel-treppenartig nach unten führte. Unten angekommen musste sie sich gegen eine Mauer lehnen, die wie eine Geheimtür aussah. Sie war wieder in der Kammer und wurde von ihrer Führerin mit dem Stein in der Hand empfangen.

Katharina war ziemlich erschöpft, aber nahm es klaglos hin. Sie wollte sich schon Erleichterung verschaffen, indem sie sich kurz unter ihrem Turban kratzen wollte, aber sie erinnerte sich gerade noch rechtzeitig daran, dass unter ihrem Kopfschutz gefährliche Medusen waren und ließ davon ab. Sie wischte sich lediglich mit dem Ärmel den Schweiß von ihrer Stirn und stellte sich erneut hin, um den Stein zu halten.

Der Stein schimmerte im Licht und Katharinas Augen wurden immer schläfriger und schläfriger. Ihre Führerin betrachtete sie wartend und geduldig und sah zwischen Katharina und dem Stein immer wieder hin und her. Sie stand einfach nur da und wartete.

\enquote{Was haben Sie gesehen?}, fragte sie plötzlich.

\enquote{Ich weiß es nicht genau.}

\enquote{Beschreiben Sie es.}

\enquote{Das kann ich nicht.}

Währenddessen las Madame Pomfrey unermüdlich die immer neu erscheinenden Pergamente und gab Professor Elber gerade eine kleine Zusammenfassung ab, als Professor Sinistra den Raum betrat und sich anschloss.

\enquote{Was haben Sie?}, fragte Professor Elber Madame Pomfrey und war ganz zittrig. \enquote{Etwas nicht in Ordnung?}

\enquote{Es gab einen Anstieg bei ihrer Atmung und bei den neuralen Peptiden.}
% und den Adenosin-Triposphaden.}
%24:40

\enquote{Ist das schlimm?}, fragte er. Es nahm ihn sichtlich mit.

\enquote{Es weist auf schwere körperliche Anstrengung hin. Aber die neuralen Peptide sind sehr interessant. Die Sättigung ist auf einem höchst anormalen Niveau.}

\enquote{Ist das gut, oder schlecht?}

\enquote{Das weiß ich noch nicht, aber es könnte zu ihrer Heilung beitragen, oder sie zumindest erleichtern. \gst Es könnte den Fluch beeinflussen.}

\enquote{Also hilft es ihr?}

\enquote{Das weiß ich nicht und, hängen Sie nicht so an mir}, sagte sie erregt, da Professor Elber schon über ihrer Schulter hing und mitlas. \enquote{Gehen Sie. Ich melde mich, wenn es etwas zu berichten gibt.}

\enquote{Aber\abs}

\enquote{Nichts aber, raus hier.}

Folgsam trollte er sich und verließ die Krankenstation. \enquote{Ich sollte diese Medusen auf dem Kopf haben und nicht sie. Es waren meine\abs}

\enquote{\extase{Frederick}}, schrie Madame Pomfrey. \enquote{So etwas will ich nicht wieder von dir hören. Du hast daran keine Schuld. Denk nicht mal so etwas.}

Er nickte nur noch und verschwand um eine Ecke.

Katharina, die immer noch ihre Prüfungen meistern musste, bekam eine Tasse aufbauende Flüssigkeit. Dankbar nahm sie sie entgegen.

\enquote{Ich bin erschöpft}, brachte sie matt hervor. \enquote{Oh, danke sehr.}
%25:36

\enquote{Ihre Alarmzauber müssten inzwischen eine ganze Menge interessanter Informationen zu ihrer Adoptivmutter und nach Hogwarts geschickt haben.} Sie ging zu einer Wand und holte einen schmalen hohen Weidenkorb, der mit einem Tuch bedeckt war. Diesen stellte sie vor Katharina ab.

Bedrohliche Zisch- und leichte Rasselgeräusche kamen aus dem Inneren des Korbes.

\enquote{Was ist das?}, fragte Katharina, als sie ihren Becher leer getrunken hatte.

\enquote{Das ist ein Nesset. Sie sind in der Lage, Ihnen bei ihrer Suche zu helfen.}

\enquote{Dann bin ich bereit meinen Fluch zu verlieren?}

\enquote{Glauben Sie, dass Sie ihn dadurch verlieren?}

Sie legte ihren Becher beiseite. \enquote{Ja, das tue ich.}

\enquote{Sie haben recht \gst Kommen Sie, greifen Sie da hinein.}

Langsam kam sie dem Korb näher. Die rasselnden Geräusche wurden etwas lauter und auch das Zischen wurde etwas aggressiver, dachte sie sich. Ihre Hand näherte sich dem Tuch über dem Korb. Wieder zischte etwas kurz auf und sie zog ihre Hand etwas zurück.

\enquote{Wir hören sofort auf, wenn Sie wollen.}

\enquote{Nein, ich gebe nicht auf.} Mutig und mit geschlossenen Augen griff sie hinein. Der Nesset biss zu und Katharina zog ihre Hand schreiend heraus.

Sie rollte ihren Ärmel hoch und besah sich schwer atmend und mit Schmerz verzerrtem Gesicht ihre Bisswunde.

\enquote{Haben Sie keine Angst}, beruhigte sie ihre Führerin.

\enquote{Es verbrennt mich}, zitterte Katharina. \enquote{Oh, mein Brustkorb \gst Er wird immer enger.} Sie stöhnte und schrie, bis sie bewusstlos zusammenbrach.

Rückblenden-artig durchlief sie im Schnelltempo die Zeremonie. Das Entkleiden und bemalen mit rituellen Symbolen. Das Halten des Steins. Das Bemalen der Leinwand, und das Klettern an der Wand.

\enquote{Katharina, \gst Katharina}, hörte sie leise und weit entfernt. Doch mit der Zeit zunehmend klarer.

Als sie die Augen aufschlug, lag sie in der Vertiefung, auf dem Podest im ersten Raum.

\enquote{Ich sterbe}, brachte sie schwach hervor.

\enquote{Letzten Endes stirbt ein jeder.}

Dann schloss sich die Grube und es wurde dunkel.
%27:28

\trenn

Im Inneren von Salazars Räumen standen mittlerweile Harry, Ron, Hermine, Ginny und Professor Snape.

\enquote{Das hier sind Slytherins Räume. Ich habe sie vor ein paar Tagen entdeckt.} Harry setzte sich auf einen Sessel und wartete.

Seine Gäste wussten die ersten Sekunden nicht, was sie sagen sollten. Harry wartete geduldig ab.

\enquote{Das sind\abs die privaten Räume von Slytherin}, stotterte Ron.

\enquote{Das sagte ich bereits}, meinte Harry.

\enquote{Harry, diese Bücher hier\abs}, sagte Hermine ganz begeistert.

\enquote{Nur die linke Seite ist für dich interessant. Und da nur das oberste Drittel.}

\enquote{Wieso?}, fragte sie nach.

\enquote{Die anderen wirst du nicht lesen können.}

\enquote{Dann lerne ich die Sprache. Ich finde es schon heraus.}

\enquote{Du hast es bei meinem Schriftstück auch nicht geschafft. Also mach dir keine Hoffnungen.}

Ginny und Snape sahen sich während dessen still um und betrachteten den Raum.

\enquote{Da hinten ist ein Bad}, zeigte Harry auf die Tür. \enquote{Die Stufen führen in das Schlafzimmer, Toiletten und Gästezimmer.} Harry zeigte auf die Treppe.

\enquote{Und die andere Tür?}, fragte Professor Snape nach.

\enquote{Zu verschiedenen Räumlichkeiten. Unter anderem die Kammer, Gemeinschaftsräume und vermutlich auch die Räumlichkeiten der anderen Gründer.} Dabei sah er zum Bild hoch und sah Salazar leicht mit dem Kopf nicken.

\enquote{Wie sind Sie denn hier hereingekommen?}, fragte Snape.

\enquote{Durch das Bild}, antwortete Harry.

\enquote{Sie wissen, was ich meine}, gab er zurück und setzte sich in einen Stuhl neben Harry.

Die anderen drei machten es sich auf dem Sofa gemütlich. Kreacher erschien und brachte kleine Sandwichs und Kürbissaft, sowie Wasser. Dann verschwand er wieder.

Harry griff zu und biss ab. \enquote{Mein Amulett}, sagte er und zog es unter seinem Umhang hervor. \enquote{Irgendwie hat es mich geleitet.} Er wollte nicht sagen, dass ihm Salazar persönlich gesagt hat, wie er in seine Räumlichkeiten kommt.

\enquote{Waren Sie schon einmal hier?}

\enquote{Ja, eigentlich zweimal. Beim ersten Mal haben sie mich erwischt und ich musste wieder zurück. Etwas später habe ich es dann geschafft. Es wurde dann spät und ich habe hier geschlafen. Das Bett ist übrigens sehr bequem.}

\enquote{Das würde ich auch gerne mal probieren}, meinte Ginny und fing plötzlich an zu husten.

Harry lächelte sie leicht an. Ihm wurde warm ums Herz. Er konnte sich durchaus vorstellen, mit ihr hier zu schlafen. Eine Nacht zu verbringen, korrigierte er sich. \enquote{Ich kann dich gut verstehen, da ich ja bereits hier eine Nacht verbracht habe.}

\enquote{Sie haben hier geschlafen? Wann?}

\enquote{Vor ein paar Tagen.}

\parsel{Wer isst dass?}, hörte er plötzlich.

\parsel{Einer meiner Professoren und meine Freunde}, antwortete Harry der kleinen Schlange, die vor ihm ihren Kopf aus dem Bücherregal streckte.

Die drei auf dem Sofa erschraken und drehten sich leicht bleich im Gesicht um.

\enquote{Darf ich euch Marcel vorstellen? Ich kenne ihn ein paar Wochen.} Dabei schielte er zu Snape.

Dieser zog die Stirn kraus und fragte sich, wo er die kleine Schlange schon einmal gesehen hatte.

\enquote{Wollt ihr mal das Schlafzimmer sehen? Geht einfach hoch und schaut es euch an.}

Seine drei Freunde nickten und begaben sich nach oben.

Kaum waren sie weg, regte sich Snape. \enquote{Kenne ich die Schlange nicht irgendwo her?}

\enquote{Ja, nur ist das keine Schlange. Ich habe ihn mit einem Zauber belegt und ihm einen Trank gegeben.}

\enquote{Den, den sie einmal brauten, als ich noch etwas Zeit brauchte?}, folgerte Snape.

\enquote{Ja.}

\enquote{Dann ist das keine Schlange?}

\enquote{Richtig. Es ist ein Basilisk.}

\enquote{Warum haben Sie ihre Freunde weggeschickt?}

\enquote{Sie sollten es nicht erfahren. Es würde wohl keiner von ihnen verstehen.}

\enquote{Aber mir muten Sie das zu?}

\enquote{Sie haben ihn ja schon gesehen. Zwar nur in meinem Geist, aber dennoch. Sie haben ihn gesehen.}

\enquote{Ist das nicht gefährlich?}

\enquote{Nein}, kam vom Bild. \enquote{Es gibt viele Zauber, die mit Schlangen durchgeführt werden können. Einige davon, ausschließlich von mir stammende, behandeln Basilisken. Wie vielleicht bekannt sein dürfte, hatte ich einen. Ich habe viel mit einem gearbeitet und diese Zauber und Tränke an ihnen ausprobiert. Leider hatte ich dann diese Phase, wo ich einen ohne Behandlung im Schloss einquartierte. Bedauerlicherweise habe ich später vergessen, dass ich nur den Zauber für mich anwandte. Mir ist es nicht mehr aufgefallen, dass sie für andere gefährlich werden konnte}, sagte Salazar traurig.

\enquote{Wollen Sie damit sagen, dass es Ihnen leidtut, dass Sie den Basilisken hier im Schloss ließen?}

Diesen Satz bekamen auch Harrys Freunde mit, da sie von oben wieder herunterkamen.

\enquote{Ja, ich habe meine Meinung bezüglich des reinen Blutes geändert. Ich habe Jahre lang versucht, den angerichteten Schaden wieder zu beheben, stieß aber nur auf taube Ohren bei denen, die den Floh schon hatten. Ich habe versucht, den Schaden rückgängig zu machen. Erfolglos.}

\enquote{Wie? Sie sind gar nicht der, für den wir Sie halten?}, fragte Ron nach.

\enquote{Hast du gerade nicht zugehört?}, fragte Hermine.

\enquote{Doch, aber das klingt nur so unglaublich.}

\enquote{Gibt es irgendwo Beweise?}, fragte Snape nach.

\enquote{Nein}, antwortete Slytherin. \enquote{Ihr habt nur mein Wort. Und mein Tagebuch.}

\enquote{Das könnte auch gefälscht sein}, warf Hermine ein.

\enquote{Auf wessen Seite stehst du eigentlich}, fragte sie Ron bissig.

\enquote{Auf der Seite der Wahrheit. Wenn wir schon bei der Suche nach Beweisen sind, dann sollten wir auch stichhaltige haben. Tagebücher könnten gefälscht sein. Wir brauchen Beweise, die nicht aus Slytherins Besitz sind.}

\enquote{Hermine hat recht}, gab Ginny zu bedenken. \enquote{Wenn wir damit an die Öffentlichkeit wollen\abs}

\enquote{Nein}, warf Slytherin ein. \enquote{Bitte nicht. Ich habe schon vor langer Zeit damit abgeschlossen.}

\enquote{Aber ihr Haus hat einen besonders schlechten Ruf durch die Tatsache, dass sie, der Gründer dieses Hauses, ein Muggelhasser und Reinblutfanatiker waren und durch die Tatsache, dass besonders viele Schwarzmagier aus ihrem Haus kamen.}

\enquote{Dafür kann ich nichts.}

\enquote{Das hat aber dazu geführt, dass die Schüler Ihres Hauses ausgegrenzt wurden und immer noch werden.}

\enquote{Was meinen Sie?}

\enquote{Ich meine, dass Ihre Schüler von anderen gemieden werden. Ihr Haus wird praktisch ausgegrenzt. Das könnte Sie dazu veranlassen, sich der dunklen Seite zuzuwenden.}

\enquote{Dunkle Seite?}

\enquote{Verzeihung. Böses tun}, gab Hermine kleinlaut zurück.

\enquote{Ich habe Hunger}, sagte Ron plötzlich.

\enquote{Hunger? Du hast hier gerade etwas erfahren, was unser Weltbild über Slytherin über den Haufen geworfen hat und du denkst an Essen?} Hermine wollte etwas nach ihm werfen.

\enquote{Lass ihn einfach. Dann soll er zum Essen gehen. Wir können ja weiter reden}, meinte Harry und wandte sich dem Bücherregal zu. \enquote{Ist das Tagebuch hier? Und vor allem \gst darf ich es lesen?}

Salazar sah ihn prüfend an. Dann sagte er schließlich: \enquote{Ja und ja. Aber nicht jetzt. Komm später wieder. Und, lies es nur alleine.}

Harry nickte. Da Ron immer noch da saß, rief er nach Kreacher und ließ sich das Essen für fünf Personen bringen. Minuten später waren mehrere Elfen da und beluden den Tisch. Die fünf ließen es sich schmecken und gingen nach dem Abendessen zurück in ihre jeweiligen Räume.

Auf dem Weg zu Hagrid am nächsten Tag, fragten sie sich, was sie wohl heute dran nehmen würden. Dieses mal hatten sie keine Ahnung, denn Hagrid hielt dicht.

\enquote{Heute nehm’mer Skolks durch}, sagte der Halbriese. \enquote{Wer weiß was darüber?}

Harry, der nicht gerade dafür bekannt war, so etwas zu wissen, hob seine Hand.

\enquote{Du?}, fragte Hagrid ungläubig. \enquote{Dann lass mal hören.}

\enquote{Es ist nicht viel, aber Skolks haben zweierlei Umweltschutz. Im Sommer ein Gefieder und im Winter ein Fell. Sie ernähren sich von Blut und Schweiß, können aber auch, als einige der wenigen Spezies, Dementoren töten. In Bayern kennt man sie unter dem Namen Wolpertinger. Das ist auch ihre ursprüngliche Heimat. Dort haben sich noch ein paar kleine Kolonien gehalten. Sie werden von Hexen und Zauberern abgeschirmt und dort behütet. Bis heute sind sie im bayerischen und dem angrenzenden Österreich bekannt. Meist werden sie ausländischen Muggeln als ausgestopfte Attrappen verkauft. Skolks sind Mischwesen. Sie haben das Geweih eines Hirsches \gst aber etwas kleiner \gst Füße einer Ente, den Schnabel einer Katze und den Schwanz eines Bibers. So wird er zumindest überwiegend beschrieben. In der magischen Welt sind diese Wesen fast unbekannt.}

\enquote{Das nenns’de  nichts? Das hätt’ ich nich’ erwartet. Das gibt zehn Punkte für Gryffindor. Woher weiß’ denn das?}

\enquote{Ich bin über den Begriff gestolpert und neugierig geworden. Also habe ich in der Bibliothek nachgeschlagen.}

Hagrid nickte und lief zu einem kleinen Gatter. Als die Schüler darüber sehen konnten, sahen sie zum ersten Mal in ihrem Leben einen Skolk, oder wie ihn die Muggel auch nennen Wolpertinger.

\enquote{Eure Aufgabe is’ einfach. Stellt euch rein, lasst ein’ an euch lecken und zeichnet ihn. Wie, is’ mir egal. Aber beeilt euch, wir ham’ se nur heute. Morgen sin’ se weg.}

Also betrat die Klasse das umzäunte Gelände, setzte sich und holte ihre Zeichensachen hervor. Die Tiere näherten sich den Schülern und begannen ihnen den Schweiß von der Haut zu lecken. Einige bissen sogar kurz zu und leckten dann das Blut ab. Die Schüler begannen die seltsamen Geschöpfe zu zeichnen. Beim nächsten Male würde doch nur wieder stumpfe Theorie dran kommen. So genoss ein jeder das Tier zu streicheln, sobald er fertig war.




\begin{kommentar}
Katharina Chapel, die Schülerin mit den Schlangen auf dem Kopf (und die nach Kathryn Janeway benannt wurde), muss nach Griechenland und dort eine Prüfung ablegen. Diese wurde aus einer Folge von Star Trek Voyager übernommen, wo der Captain exakt dieselben Prüfungen machen musste. (Das Ritual 3x07)
\end{kommentar}

\begin{kommentar}
Katharina sagt während des Rituals folgenden Satz: »Meine Magie gehört zu mir. Ich benutze sie. Wieso sollte sie mir nicht gehorchen?«
Dieser unscheinbare Satz spannt einen Bogen bis zum Ende des nächsten Teils, wo Harry in die Mondbibliothek kommt.
\end{kommentar}

\chapter{Tage}


Harry stand an der letzten Stelle, an dem sich ein Punkt auf der Karte zeigte. Es war eine simple Wand. Doch etwas war an ihr anders. Das spürte er. Er bearbeitete die Stelle mit verschiedenen Zaubern und endlich gab die Wand nach und bildete eine Tür aus. Mit einem recht simplen \spruch{Alohomora} öffnete er die Tür und trat ein. Er stand in einem kleinen runden Raum mit Fackeln zu beiden Seiten, die begannen aufzuleuchten, als er den Raum betrat. Gegenüber von ihm war ein Torbogen und Treppen schienen nach unten zu führen. Sonst konnte er nichts feststellen. Er trat ein und ging die Treppen hinunter. Die Tür hinter ihm schloss sich automatisch. Am Ende der Treppen kam ein kleiner Gang von etwa vier Metern. Dann hörte er auf. Alles war in Stein gehalten. Harry untersuchte alles ganz genau. Jeden Zauber, den er kannte, verwendete er. Doch nichts half. Es schien ein Überbleibsel zu sein. Etwas, was man verschlossen hatte, als man das Schloss umbaute, vermutete er. Er ging wieder nach oben und versuchte durch die Tür zu kommen. Doch auch hier half nichts.

Nach einer halben Stunde gab er auf und setzte sich auf die Stufen. Er nahm den Kopf zwischen die Hände und resignierte.

Neben ihm tauchte Salazar auf. \enquote{Tut mir leid, dass du in dieser Situation bist. Wenn ich das früher bemerkt hätte, dann hätte ich dich gewarnt. Hier kommst du nicht durch eigene Kraft heraus.}

\enquote{Warum?}

\enquote{Das ist eine kleine Gemeinheit, für allzu neugierige Schüler. Der Direktor der Schule wird frühestens in vierundzwanzig Stunden etwas davon erfahren. Dann hat er eine Idee, wo du sein könntest. Nach weiteren vierundzwanzig Stunden wird es ihm bewusster, falls er nicht darauf reagiert und nach insgesamt zweiundsiebzig Stunden wird er magisch gezwungen, dich zu holen.}

\enquote{Das heißt, ich komme vor morgen hier nicht raus?}

\enquote{Tut mir leid, Harry. Aber wir können ja immer noch reden.} Dann verschwand sein Abbild.

\trenn

Etwas früher im Krankenflügel.

\enquote{Vor drei Tagen hat sie zum letzten Mal geschlafen. Wie lange können wir ihr das denn noch zumuten?}, fragte Professor Elber Madame Pomfrey.

\enquote{Ich verstehe deine Sorge, Frederick, aber ihre Lebenszeichen sind nach wie vor stabil. Sie scheint nicht in unmittelbarer Gefahr zu sein.}

\enquote{Sie kam mit einem unbekannten Gift in Kontakt, das vielleicht für uns nicht nachvollziehbare Wirkungen hat. Es könnte umschlagen und sie in Minuten töten.}

\enquote{Willst du mir etwa in meine Arbeit hereinreden? Willst du sie etwa machen?}

\enquote{Könnte sie etwas daran hindern, zurückgeholt zu werden?}, fragte Professor Sinistra.

\enquote{Nein, sie ist in gutem Zustand und hat die volle motorische Kontrolle.}

\enquote{Ist es sicher, dass diese Strapazen ihr helfen?}

\enquote{Ja. Ihre gesamte Biochemie erfährt eine Reihe von ungewöhnlichen Interaktionen. Es wurden Veränderungen in ihr ausgelöst, die auf ihre Magie wirken. Als sich das Gift in ihrem Körper löste, hat das einige interessante Veränderungen herbeigeführt.}

\enquote{Dann kann sie sich nicht zurückholen, wenn sie unter Drogen steht.}

\enquote{Dieses Gift ist vielleicht der Schlüssel zur Bekämpfung der Medusen.}

\enquote{Na schön}, gab er nach. \enquote{Ich werde aber bis zu ihrer Rückkehr nicht von deiner Seite weichen.}

\trenn

Katharinas Kopf brummte und langsam bewegte sie ihn.

Dann durchfuhr sie ein greller Lichtblitz, der durch ihre geschlossenen Augen jeden Nerv in ihrem Körper zu reizen schien. Kurz, aber heftig. Dann stand sie an einem Strand. Hinter ihr war eine Klippe zu sehen und vor ihr das blaue Meer, das durch den Wind eine Gischt bildete. Mit leichten Kopfschmerzen rieb sie sich übers Gesicht. Dann erst realisierte sie, wo sie war.

Sie hörte Geräusche und drehte sich in die Richtung, aus der sie kamen. \enquote{Was ist das?}, fragte sie ihre Führerin, die einige Meter von ihr entfernt und leicht hinter ihr stand. \enquote{Eine Halluzination?}

\enquote{Ich bin nur hier, um als Stimme zu dienen. Als Übersetzer für Ihre Vorfahren, Ihren Leidensgenossen, Ihrer ursprünglichen Fluchmutter, der Medusa.}

\enquote{Ich verstehe}, sagte Katharina begeistert und wandte sich wieder dem Meer zu. \enquote{Wenn hier meine Vorfahren sind, kann ich sie dann sehen?}

\enquote{Sie meinen, Sie wollen einen Beweis, dass wir existieren?}

\enquote{Ja, das wäre hilfreich.}

\enquote{Es ist bedeutungslos.}

\enquote{Ich will nicht respektlos sein. Ich habe einen jeden Abschnitt des Rituals durchlaufen, der mir aufgetragen wurde.}

\enquote{Alles, was Sie durchgemacht haben, ist bedeutungslos, das wurde Ihnen doch gesagt.}

\enquote{Ich weiß \gst Es ist nur mein Wunsch, es endlich abzuschließen, um Erlösung zu finden.}

\enquote{Tun Sie das, tragen Sie ihre Bitte vor. Niemand wird Sie davon abhalten.}

\enquote{Bitte, helfen Sie mir und denen, die ich versteinert habe. Sagen Sie mir, was ich tun muss.}

\enquote{Ihre Bitte ist inkonsequent, Sie haben alles in sich, was Sie brauchen und was für Ihre Rettung vonnöten ist.}

Katharina wurde mit Glück durchflutet. Sie begann zu lächeln. Wieder wurde ihre Umgebung hell und wieder wurden ihre Nerven stimuliert. Dann wurde ihr kurz schwarz vor Augen und sie sah, wie sich die Grube, in der sie lag, öffnete.

\enquote{Schön, dass Sie wieder da sind}, sagte Katharinas Führerin und reichte ihr die Hand.

\enquote{Wie lange?}, fragte Katharina

\enquote{Spielt das eine Rolle?}

\enquote{Ich würde es gerne wissen.}

\enquote{Neununddreißig Stunden \gst Sie müssen sich etwas schonen, ihr Körper ist schwach.}

\enquote{Ich schätze, die Duellübungen und mein Lauftraining sowie meine Fitnesseinheiten haben mich darauf nicht vorbereitet.}

\enquote{War es die Sache wert?}

\enquote{Ich denke schon. Es wurde mir gesagt, dass ich hätte, was ich brauche, um meine Freunde zu retten und mir mit meinem Fluch zu helfen.}

\enquote{Das muss die Wahrheit sein, Ihre Vorfahren würden Sie nicht belügen, nicht unter diesen Umständen.} Sie ging an den Rand der Kammer und holte Katharinas Kleidung, fein säuberlich zusammengelegt, und überreichte ihr das Bündel. \enquote{Wann immer Sie bereit sind.}

Katharina nahm ihre Kleidung an, betrachtete sie und sah dann ihre Führerin aus müden Augen heraus an. \enquote{Danke}, war das Einzige, was sie hervorbrachte.

Ihre Führerin verbeugte sich leicht und verließ den Raum, nachdem sie ihr ihren Zauberstab zurückgegeben hatte. Katharina zog sich mit einem glückseligen Lächeln an und ging, durch ihre Führerin begleitet, zurück zum Wartebereich, wo sie von Pomona freudig begrüßt wurde.


Zurück auf Hogwarts wurde sie von Madame Pomfrey untersucht. \enquote{Ihnen fehlt eine Nacht Tiefschlaf und eine anständige Mahlzeit. Ansonsten sind Sie in guter Verfassung.}

Professor Elber, aber vor allem Pomona, freuten sich.

\enquote{Das Ritual war sicher sehr anstrengend, aber wie ich vermutete ist das Gift der Schlüssel, um ihren Mitschülern zu helfen. Ich werde ihnen etwas Blut abnehmen und dann mit dem Alraunenextrakt mischen. Ihre Schlangen sollten sich die nächsten Tage zurückbilden}, schloss Madame Pomfrey.
%33:57
Sie nahm ihren Zauberstab und zog magisch etwas Blut in ein kleines Reagenzglas ab. Dann ging sie in ihr Büro und holte ein bereitstehendes Gefäß mit dem Alraunenextrakt und goss das Blut, nachdem sie einen Zauber darauf angewendet hatte, in den Becher. Wieder aus ihrem Büro zurück, ging sie hinter einen der Vorhänge, die um ein Bett gezogen waren, und war außer Sichtweite.

Professor Elber und Professor Sinistra verließen die Krankenstation und Pomona setzte sich zu Katharina. \enquote{So meine liebe, du hast es überstanden. Ich bin so stolz auf dich. Dann kannst du jetzt den blöden Turban abnehmen.}

\enquote{Aber mein Anblick\abs}

\enquote{Stört mich nicht im Geringsten}, unterbrach sie ihre Adoptivmutter. \enquote{Ich mag dich so, wie du bist, Katharina. Oder glaubst du, ich adoptiere nur jemanden mit makellosen Haaren?}

Katharina musste schmunzeln. Gedankenverloren schüttelte sie ihren Kopf. Dann begann sie ihren Turban abzuwickeln und ihre Schlangen zu offenbaren.

Zuerst passierte nichts, die Schlangen auf ihrem Kopf waren ruhig und bewegten sich nicht. Doch dann erwachten sie und Pomona versteinerte.

Ein Schrei entwich aus Katharinas Kehle, der von der Krankenschwester nur deshalb nicht gehört wurde, weil sie zeitgleich ebenfalls einen ausstieß. Nach viel zu langen Schocksekunden wickelte sich Katharina wieder ihren Turban um ihren Kopf und sah ihre Adoptivmutter mit Entsetzen im Gesicht an.

Die Türen zur Krankenstation wurden aufgestoßen und Professor Elber und Professor Sinistra kamen mit geschlossenen Augen herein.

\enquote{Können wir unsere Augen öffnen?}, fragten sie.

Katharina nickte, stutzte kurz und bejahte dann die Frage.

Mit vor Schock geweiteten Augen starrten die beiden sie und die steinerne Statue von Pomona Sprout an.

\enquote{Es hat nicht geklappt}, schluchzte Katharina.

\enquote{Ich brauche hier Hilfe}, kam ein ziemlich lauter Schrei hinter einem er Vorhänge hervor.

\enquote{Was ist, Poppy?}, fragte Professor Elber.

\enquote{Die sterben mir hier weg. Die Versteinerung ist nur teilweise aufgehoben.}

\enquote{Komm da raus, Poppy}, sagte Professor Elber und lief auf den Vorhang zu.

\stimme{Katharina, du kommst her und nimmst kurz deinen Turban ab, wenn wir hier wieder raus sind}, hörte sie in ihrem Geist.

Professor Elber zog den Vorhang beiseite und zog die sich sträubende Poppy mit Gewalt fort.

Katharina verschwand hinter dem Vorhang und kam nach wenigen Sekunden wieder hervor.

Professor Elber ließ Poppy los, worauf diese sofort auf den Vorhang zu stürmte und hinter ihm verschwand. Doch sie kam schnell wieder zurück.

\enquote{Versteinert}, sagte sie, worauf Katharina wieder zu weinen anfing.

\enquote{Es hat nicht geklappt}, schluchzte sie.

\enquote{Tut mir leid Miss Chapel, aber es scheint, dass alles, was sie durchlebt haben, bedeutungslos war} , sagte Madame Pomfrey.

Katharina schaute sie mit verweinten Augen an. Tränen liefen ihre Wangen hinunter.

\enquote{Gehen Sie auf ihr Zimmer und beruhigen Sie sich erst einmal.}

\enquote{Beruhigen? Ich habe gerade meine Adoptivmutter verloren, kaum, dass ich sie habe.}

\enquote{Sie haben sie nicht verloren. Sie ist momentan nur außer Reichweite. Das wird wieder. Gehen Sie jetzt, oder ich bringe Sie persönlich in Ihr Zimmer und decke sie zu.}

Das zeigte Wirkung, denn die Vorstellung, dass ihre Krankenschwester sie durch den Gemeinschaftsraum hindurch in ihr Zimmer bringen würde, war Ansporn genug. In ihrem Zimmer legte sie sich auf ihr Bett und weinte die restlichen Stunden bis zum Abendessen.

Obwohl sie, bevor sie sich hingelegt hatte, einen Apfelschnitzel gegessen hatte, der ihren größten Hunger vertrieben hatte, war ihr Hunger doch sehr groß. Sie hatte immerhin drei Tage lang nichts gegessen. Deshalb lud sie sich in der Großen Halle auf ihren Teller, was sie dachte zu essen, denn, auch wenn das Essen in der Tischmitte verschwand, wenn die Essenszeit zu Ende war, blieb das auf dem Teller liegen, sofern man noch aß.

\enquote{Und, wie ist es gelaufen?}, fragte sie Elisabeth, eine ihrer Schulkameradin.

Katharina sah sie eindringlich an. \enquote{Nichts ist in Ordnung. Meine Schulkameraden sind immer noch versteinert, Dumbledore auch. Und meine Adoptivmutter seit dem Nachmittag auch. Also kannst du dir denken, dass es mir nicht gut geht.}

\enquote{Kann ich dir helfen?}

\enquote{Ich weiß momentan nicht wo mir der Kopf steht.}

\enquote{Verständlich}, nickte ihre Mitschülerin, legte eine Hand auf ihrer Schulter ab und sagte: \enquote{Ich schreibe weiterhin für dich mit.}

\enquote{Danke}, sagte Katharina.

\trenn

Harry hörte ein Knarzen und stand auf. Er lief auf die offene Tür zu und trat hinaus.

\enquote{Dort haben Sie sich verkrochen, Mister Potter.}

\enquote{Tut mir leid, Professor McGonagall. Ich war einfach zu neugierig.}

\enquote{Ja, das waren Sie. Melden Sie sich bei Hagrid. Er wird Ihnen hoffentlich etwas davon nehmen können.}

\enquote{Ich hoffe doch, etwas Unangenehmes.}

\enquote{Wie bitte?}, fragte Professor McGonagall nach.

\enquote{Na ja, sonst verfehlt es doch seine Wirkung}, meinte Harry und machte sich auf den Weg zu Hagrid.

Nach dem Unterricht brauchte Katharina erst einmal eine Pause. Sie nahm sich nur ein paar Sandwichs aus der Großen Halle mit und ging dann Richtung Verbotener Wald. Mit diesem Misserfolg musste sie erst einmal zurechtkommen. Sie begegnete am Waldrand Harry, der ebenfalls seinen Gedanken hinterher hing. Dadurch etwas aufgemuntert, besserte sich ihre Laune etwas.

\enquote{Hallo Katharina. Bekomme ich eine ehrliche Antwort, wenn ich dich fragte, wie es dir geht?} Sie nickte nur. \enquote{Und, wie geht es dir?}

\enquote{Schlecht. Die Prüfung in Griechenland habe ich vermasselt. Ich habe das Ziel nicht erreicht. Jetzt ist auch noch Pomona\abs Professor Sprout versteinert.} Tränen begannen ihre Wangen hinunterzulaufen. \enquote{Jetzt habe ich auch noch meine Adoptivmutter verloren.}

Harry beruhigte sie, indem er auf sie zuging, ihre Tränen mit einem Taschentuch wegwischte und sie einfach nur in den Arm nahm. Langsam beruhigte sich ihr Zittern, das Harry sofort spürte, als er sie an seine Brust drückte. Nach einigen Minuten sah er aus seinem Augenwinkel, wie sich etwas bewegte. Er drehte seinen Kopf leicht und sah in das Gesicht einer kleinen Schlange, die sich sofort wieder zurückzog, als sie Harry erblickte.

Es dauerte eine Weile, in der er verkrampft dastand, bis sich die Erkenntnis durch sein Gehirn in seinen Verstand bewegt hat. Er hatte eine der Medusen-Schlangen auf Katharinas Kopf gesehen. Dann erkannte er, dass sich Katharina einen Turban um den Kopf gebunden hatte. Einen, wie sie sich Sikhs umbinden, oder wie Professor Quirrel ihn um hatte. Und mit einem Schlag wurde ihm klar, dass die Schlangen keine Auswirkung auf ihn hatten. Ihn irritierte nur, dass Katharina anscheinend Professor Sprout versteinerte, es aber auf ihn keine Auswirkung hatte. Seine Gedanken kreisten um sich, seine Magie, das was er gelernt hatte und, um Marcel \gst und um Salazars Amulett.

Er war sich unsicher, ob er Katharina davon erzählen sollte.

Sie löste sich von ihm, sah ihn dankbar an und gab ihm einen Kuss auf die Wange.

\enquote{Kommt’s mal mit}, hörten beide plötzlich aus dem Inneren des Waldes. Etwa zehn Schritte von ihnen entfernt, sahen beide Hagrid wie er innerhalb des Waldes stand und beide heranwinkte. Zuerst etwas unsicher, dann aber zunehmend sicherer ging Katharina zu Hagrid. Harry hatte keine Scheu und ging ohne Zögern hinein. Mit Hagrid konnte ihnen nichts passieren. Zumindest gab es keine Probleme was die Schulregeln betraf.

Ansonsten wusste keiner so genau, was sich im Inneren des Waldes befand. Stumm lief Hagrid voraus und kam nach wenigen Minuten an einen steinernen Torbogen.

\enquote{Da musste reingehen, Katharina}, sprach Hagrid.

\enquote{Was muss ich da tun?}

\enquote{Du gehst durch den Torbogen, den Pfad lang und bleibt ne Stunde innem Kreis. Dann kommste wieder raus.}

\enquote{Warum?}, fragte sie.

\enquote{Frag nich so viel}, sagte Hagrid und schob sie Richtung Torbogen, was einen deutlichen Schub auf Katharinas Rücken ausübte.

Kaum war sie durch den Torbogen getreten worden, wurde ihr merklich wärmer. Sie wurde ruhiger und akzeptierte, dass sie den Pfad entlang zu einem Kreis gehen sollte, um dort die nächste Stunde zu verbringen. Sie war aber durchaus in der Lage, sich dagegen zu wehren, denn es war mehr eine Erkenntnis, als ein Zwang, der sie dazu brachte. Dann schaute sie sich noch einmal während des Laufens kurz um und lief dann den Pfad entlang. Als sie in den Kreis trat, der eher wie eine Art leichte Vertiefung aus Moos aussah, inmitten einer erdigen Waldfläche, schaute sie sich erneut um.

Einige abgeschnittene Baumstümpfe dienten als Sitzgelegenheit und in der Mitte lag eine Art Hügel, der moosbewachsen war. Katharinas Gefühle übermannten sie und sie setzte sich auf einen der Baumstümpfe. Dann fing sie wieder einmal hemmungslos an zu weinen. Nach ca. einer viertel Stunde wurde sie durch eigenartige Geräusche in ihren Gedankengängen unterbrochen. Katharina kannte das Geräusch nicht, da sie nicht bei Muggeln aufgewachsen war. Harry hingegen würde es als schnelles vorspulen eines Kassettenrecorders während des Abspielens erkennen. Doch Katharina konnte damit nichts anfangen. Sie nahm ihre Hände von ihrem Gesicht und lauschte den Geräuschen. Als ihr Blick auf den Hügel fiel, konnte sie kleine weiße Erhebungen sehen, die sich bewegten.

Im ersten Moment dachte sie an Pilze, \gst aber Pilze bewegten sich nicht. Es mussten Wesen sein, deren Köpfe \gst oder Hüte \gst wie Pilze aussahen.

Sie wollte schon \accentuate{Hallo} sagen, entschloss sich dann aber zu einem: \enquote{Kommt ruhig raus und sagt mir, wer ihr seid. Ich bin Katharina Chapel, Schülerin auf Hogwarts.}

Langsam und zögerlich kam der Erste der beiden Wesen ein Stückchen höher, um sie anzusehen. Dann traute sich auch der Zweite, nachdem der Erste eine winkende Geste gemacht hatte. Der Anblick der beiden kleinen Wesen, die Katharina bis zum Knie gingen, verschlug ihr die Sprache. Ihr Kopf ging oben in einen pilzartigen weißen Hut über, die Hose, welche auch Teil ihres Körpers war, war ebenfalls weiß. Der Oberkörper sowie der Kopf waren hingegen blau, die Schnauze sah wie die einer Katze aus, aber die Schnurrbart-Haare fehlten.

Irgendwie erinnerten sie diese Wesen an eine Zeichnung, die sie einmal gesehen hatte. Eine ihrer Mitschülerin, eine muggelstämmige, hatte so eine Zeichnung eine Weile über ihrem Bett hängen. Stark stilisiert, aber dennoch gut erkennbar. Nur an den Namen erinnerte sie sich nicht mehr.

\enquote{Welcher Spezies gehört ihr an?}, fragte sie. \enquote{Ich bin ein Mensch.}

\enquote{Das stimmt nicht. Du bist eine der Medusen. Wir ernähren uns von den bösartigen Kreaturen, bevor sie zur Gefahr werden. Deshalb sind wir hier.}

\enquote{Ihr wollt mich essen?}, fragte sie panikartig.

\enquote{Nein, wir sind aus einem anderen Grund hier.}

Die beiden Wesen kamen näher. Einer der beiden hatte einen Wurfspieß, an dem eine Schnecke und eine kleine Rispe Beeren hing. Außerdem hatte er noch ein Bündel Seile geschultert.

\enquote{Wir sind hier, um auf dich aufzupassen und nur im Notfall uns von dir zu ernähren, falls du dich nicht unter Kontrolle bekommst.}

\enquote{Das habe ich schon versucht. Ich war in Griechenland und habe mich einem Ritual unterzogen. Aber es hat nicht geklappt. Meine Adoptivmutter wurde durch mich versteinert, nachdem ich meinen Kopfschutz abgenommen hatte.}

\enquote{Es sieht aus}, antwortete der mit der Schnecke, \enquote{als ob alles, was du durchgemacht hast, bedeutungslos war.}

Katharina erschrak. Genau das hatte ihre geistige Führerin in Griechenland auch gesagt. Sie meinte, was sie sagte.

\enquote{Was ist los, Katharina?}, fragte der andere kleine Kerl.

\enquote{Sie meinte, was sie sagte \gst meine Führerin \gst und Madame Pomfrey. Als ich dort war und meine Prüfung absolvierte \gst Ich muss da noch einmal hin.}

\enquote{Deine Stunde ist aber erst in fünf Minuten um.}

Katharina war wie in einem Rausch und musste sich Zwanghaft beruhigen. Dann fragte sie: \enquote{Wie heißt ihr eigentlich?}

\enquote{Ich bin Nimu und das ist mein Kollege Minu. Wir gehören dem Volk der Medusoner an. Wir nähren uns von Medusen, die gefährlich für andere Spezies sind. Aber bei dir haben wir keine große Sorge. Wir werden wieder gehen und uns anderen Medusen widmen.}

\enquote{Wann geht ihr? Ihr bleibt hoffentlich noch ein paar Tage, ich möchte noch mit euch reden, eventuell um Hilfe bitten, falls ich es nicht schaffen sollte\abs}

\enquote{Deine Zeit ist abgelaufen}, sagte der Jäger der beiden.

Katharina erschrak, was der andere deutlich sehen konnte.

\enquote{Die Stunde ist um, meint mein Kollege. Wenn du rechtzeitig wieder da bist, triffst du uns noch.}

Katharina nickte und verließ den Kreis, rannte den Pfad entlang und durch den Torbogen. Sie zog Harry mit sich, der gerade aufstand, als er sie sah.

\enquote{Entschuldigen Sie, Professor Hagrid, ich erklär’s ihnen später. Harry, komm mit.}

Harry wurde von ihr mitgezogen, murrte aber nicht. Sie nahm ihn direkt mit ins Schloss, zog ihn in die Kerker hinunter, durch den Gemeinschaftsraum, in dem es Sekunden nach ihrem Eintreten totenstill wurde, und hinauf in ihr Zimmer. Die Gespräche, die im Gemeinschaftsraum stattfanden, konnte Harry nur erahnen, er hatte aber keine Zeit zum Nachdenken.

In ihrem Zimmer angekommen, zeigte sie auf ihr Bett und gab Harry das Zeichen sich zu setzen. Währenddessen kramte sie in ihrem Koffer nach einem Gegenstand.

Er ließ seinen Blick durch das Zimmer schweifen. Er konnte nun einige Gemeinsamkeiten mit dem Gemeinschaftsraum der Paare feststellen. Die Betten hatten ebenso wie die der Gryffindors Vorhänge und sahen genauso aus. Allerdings waren die Farben der Vorhänge grün und nicht rot. Er ließ seinen Blick weiter schweifen und entdeckte Pansy, die ein Buch las. Sie schien nicht realisiert zu haben, dass jemand im Zimmer war, der dort nicht hingehörte. Harry überlegte lange, ob er sich einen Scherz erlauben sollte. Aber bevor er an die Ausführung denken konnte, war Katharina schon zurück mit einem Bündel Stoff, in das etwas eingewickelt war.

\enquote{Das ist ein Portschlüssel}, sagte sie, als sie etwas auswickelte. \enquote{Ich habe ihn behalten \gst er funktioniert noch. Er wird uns nach Griechenland bringen. Du kannst mir vermutlich als einziger helfen.}

\enquote{Wieso ich?}, fragte Harry.

\enquote{Du kannst aus irgendeinem Grund meinen Medusen widerstehen.}

\enquote{Wie kommst du darauf?}

\enquote{Ich habe es gespürt, als wir uns vor dem Wald umarmt haben.}

\enquote{Ich habe dich gehalten}, warf Harry ein.

\enquote{Wie auch immer. Auf jeden Fall kannst du mir helfen. Sie scheinen auf dich nicht die Wirkung zu haben, die sie auf andere haben.}

Harry nickte und wurde von Katharina an der Hand gehalten, bevor sie den Portschlüssel aktivierte, indem sie ihn anfasste. Beide wurden mit einem Ruck hinfort gezogen und fanden sich Sekunden später in Griechenland wieder.

Zielstrebig ging sie voraus und Harry ihr hinterher. Im Warteraum setze sich Harry hin und Katharina ging wieder hinein. Harry hing seinen Gedanken hinterher und ihm kam die Idee, dass die Medusen auf ihrem Kopf kleine Basilisken sein könnten. Durch seine Arbeit mit dem kleinen Marcel war er schon etwas abgehärtet. Harry überlegte, ob noch andere Theorien infrage kamen.

Währenddessen traf Katharina wieder auf ihre Prüferin und sagte: \enquote{Sie meinten, was Sie zu mir sagten. Alles, was ich durchgemacht habe, war bedeutungslos.}

Sie nickte. \enquote{Ja.}

\enquote{Ich tat alles, was Sie von mir verlangten. Und sie ließen mich glauben, den anderen und mir helfen zu können. Warum haben Sie mich dann durch dieses Ritual geführt?}

\enquote{Ich habe Sie nirgendwo hingeführt Katharina. Sie haben mich dorthin mitgenommen, wohin Sie gehen wollten. Das war Ihr Ritual. Sie haben sich diese Herausforderungen selbst gestellt.}

\enquote{Das ist richtig. Ich bin hier mit gewissen Erwartungen erschienen. Das heißt, Sie haben das alles nur gemacht, um meine Erwartungen zu erfüllen?}

\enquote{Alles andere hätte Sie nicht zufriedengestellt.}

Das musste Katharina akzeptieren. Sie hatte damals irgendwie auf ein Ritual gedrängt, obwohl ihr gesagt wurde, dass es bedeutungslos war.

\enquote{Ich bin nicht bereit aufzugeben. Falls es noch möglich ist den Anderen zu helfen, will ich es versuchen.}

\enquote{Sind Sie um ihre Magie zu suchen wieder gekommen?}

\enquote{Ich weiß nicht mehr, was ich suche.}

\enquote{Dann glaube ich, dass Sie bereit sind zu beginnen.}

Nachdem sie wieder gereinigt und umgezogen war, begann ihre Reise erneut und Katharina befand sich wieder in der Kammer mit den drei wartenden.

% ~35:40
\enquote{Seht doch mal, wer wieder hergekommen ist}, sagte der mürrische Mann.

Katharina sah nicht gerade begeistert aus.

\enquote{Also}, fuhr er fort. \enquote{Ihr kleines Abenteuer ist nicht ganz so ausgefallen, wie Sie es sich erhofft hatten. Sie haben sich völlig umsonst in Schwierigkeiten gebracht.}

\enquote{Keine Gewissensbisse}, fuhr der ruhige Mann fort. \enquote{Sie würden es nicht für möglich halten, was sich manche Leute auf der Suche nach der eigenen Magie alles angetan haben.}

\enquote{Ein echtes Ritual existiert nicht, oder?}

\enquote{Echt ist ein so relativer Begriff. Die meistern Herausforderungen des Lebens werden von uns heutzutage selbst geschaffen.}

\enquote{Und Sie sind besonders hart zu sich selbst}, bemerkte die Frau, \enquote{nicht wahr?}

\enquote{Ich war immer bestrebt, erfolgreich zu sein}, gab Katharina zur Antwort.

\enquote{Starrköpfig, würde ich sagen. Sie haben es nie in Betracht gezogen, gemeinsam mit uns zu warten, nicht wahr?}

\enquote{Ich bin nun hier und bitte Sie um Hilfe \gst Ich möchte verstehen, welchen Sinn es hat \gst in diesem Raum zu warten.}

\enquote{Aber reicht es nicht}, antwortete die Frau, \enquote{gesellschaftliche Beziehungen zu pflegen? Wir sind unterhaltsame Leute.}

\enquote{Das also wird von mir erwartet. Ich soll mit den Geistern meiner Vorfahren sprechen.}

\enquote{Oh, zuerst waren wir ein Test, und jetzt sind wir die Geister ihrer Vorfahren.}

\enquote{Sind Sie es denn?}

\enquote{Das würde Ihnen gefallen}, murrte der Mann, \enquote{nicht wahr? Wenn wir alle greifbar wären, wenn man uns anfassen könnte. Wenn Sie mit Ihrer Magie auf uns einwirken könnten.}

\enquote{Wenn man alles erklären kann, was bleibt dann noch übrig, das einen selbst ausmacht?}

\enquote{Ich weiß, dass ich einzigartig bin. So wie jedes Individuum. Es fällt mir aber schwer, die Magie als einen eigenständigen Organismus zu sehen, dessen vertrauen ich gewinnen soll.}

\enquote{Soviel zu Ihrer toleranten Einstellung gegenüber anderen.}

\enquote{Jetzt mal langsam. Ich habe mich immerhin erfolgreich gegen den Einfluss meiner reinblütigen und Rassen-wahnsinnigen Eltern und Großeltern wehren können. Ich bin vielleicht toleranter als so manch anderer Reinblüter.}

\enquote{Halten Sie sich für etwas Besseres?}

\enquote{Das kommt auf den Standpunkt des Vergleichsobjektes an}, konterte Katharina.

\enquote{Ah, wir werden wissenschaftlich. Sagen wir, gegenüber einem Muggel, einem Kobold und einem Hauselfen.}

\enquote{In welchem Bereich?}

\enquote{Bereich Magie.}

\enquote{Überlegen, überlegen, unterlegen}, antwortete sie.

\enquote{Schön, wenn Sie sich für etwas Besseres halten, dann gehen Sie wieder zurück an Ihre Schule und helfen Sie den anderen.}

\enquote{Aber es hat ja augenscheinlich nicht funktioniert, oder? Ihren Mitschülern, oder Ihrem Direktor geht es nicht besser. Im Gegenteil. Ihre Adoptivmutter ist jetzt auch versteinert.}

\enquote{Nein, leider nicht.}

\enquote{Warum nicht?}

\enquote{Unsere Krankenschwester versteht es ebenfalls nicht. Sie war sich so sicher.}

\enquote{Also ein unerklärlicher Vorgang. Eine mysteriöse Nicht-Wieder"-ge"-ne"-sung.}

\enquote{Sie\abs Wir\abs haben den Grund dafür noch nicht gefunden.}

\enquote{Aber Sie werden ihn natürlich finden. Wenn Sie nur genügend Zeit haben. Das glauben sie doch als reinblütige Magierin, nicht wahr?}, fragte der gutmütige Mann.

\enquote{Bitte ehrlich sein}, murrte der andere.

\enquote{Ja, das habe ich immer geglaubt. Mit Magie schafft man alles.}

\enquote{Selbst dann, wenn sie versagt}, sagte der ruhige Mann, \enquote{sind Sie immer noch voller Glauben in sie. Das kommt einem Glaubensbekenntnis gleich.}

\enquote{Bedingungsloses Vertrauen}, sagte die Frau, \enquote{das klingt vielversprechend.}

\enquote{Wenn sie sagen, dass Magie Ihnen nicht helfen kann, was hilft Ihnen dann?}

\enquote{Das wird Ihnen nicht gefallen}, gab der mürrische Mann zurück.

\enquote{Ich werde tun, was notwendig ist.}

\enquote{Alles?}

\enquote{Alles!}

\enquote{Töten Sie sie. Sie sind bereits so gut wie tot. Warum wollen sie es nicht zu Ende bringen. Geben Sie ihnen noch eine Ladung ihrer Medusen.}

\enquote{Oh, das würde wirken}, antwortete die Frau.

\enquote{Wie wird es wirken?}

\enquote{Da ist es schon wieder}, murrte er. \enquote{Sie fangen schon wieder an. Immer müssen Sie eine rationale Erklärung verlangen. Aber es gibt gar keine. Ihre Krankenschwester und Ihre Bücher sind eindeutig. Ihre Schlangen auf dem Kopf sind tödlich. Erst versteinern sie, dann bringen sie einen tatsächlich um.}

Katharina erschrak. Sie hatte ihre Mitschülerin erneut angesehen. Aber diese war nicht mehr versteinert. Zumindest nicht ganz. Sie hoffte von ganzem Herzen, dass sie es überlebt.

\enquote{Falls Sie diesen Fakten glauben}, antwortete die Frau.

\enquote{Lassen Sie all das von sich fallen}, sagte der ruhige Mann, \enquote{Sehen Sie ihre versteinerten Mitschüler, Ihren Direktor und Ihre Adoptivmutter in der Reihenfolge der Versteinerung an und vertrauen sie darauf, dass sie wieder gesunden.}

\enquote{Das Ritual war vollkommen bedeutungslos}, sagte Katharina, den Umstand erst jetzt komplett verinnerlicht. \enquote{Und ich habe nichts getan, um mich darauf vorzubereiten, kein Trank, kein Zauber. Wie kann ich es dann schaffen?}

\enquote{Wenn Sie glauben, dass es funktioniert, dann sind Sie bereit. Das ist alles, worum es hierbei geht.}

\enquote{Aber wenn Sie es mit dem geringsten Zweifel angehen, mit Zögern und Zaudern, dann werden die Versteinerten sterben}, murrte der andere. \enquote{\gst Also, was werden Sie nun tun, Katharina.}

\enquote{Sie wissen, ich werde nicht zusehen, wie meine Freunde, mein Direktor und meine Mutter\abs in diesem versteinerten Zustand vor sich hin vegetieren, wenn ich sie irgendwie retten kann. Ich möchte glauben, dass es mir möglich ist \gst Ich werde es versuchen.}

Katharina verließ die Kammer, zog sich um und traf auf Harry.

\enquote{Und, wie sieht es aus?}, fragte er.

\enquote{Ich muss glauben}, sagte sie. Auf Harrys fragenden Blick fügte sie hinzu: \enquote{Nur so kann die Magie in mir wirken, sich entfalten. Nur so, kann ich sie überzeugen, dass sie in einer Art und Weise wirken soll, sie zu retten.}

Harry verstand. Er sagte: \enquote{Weißt du noch, was uns Professor Elber erzählt hat?} Katharina sah ihn fragend an. \enquote{\inner{Die Magie ist mein Verbündeter, und ein mächtiger Verbündeter ist sie. Ihre Energie umgibt uns, verbindet uns mit allem. Erleuchtete Wesen sind wir, nicht diese rohe Materie. Sie müssen sie fühlen, die Magie die sie umgibt\abs}}

Und Katharina vervollständigte: \enquote{\inner{\aabs, hier, zwischen Ihnen, mir, dem Baum, den Felsen dort, allgegenwärtig ja, selbst zwischen dem See und dem Stein auf seinem Grund.} \gst Ich verstehe. Er hat uns schon damals den Schlüssel zu dem gegeben, was ich jetzt tun muss. Was ich gerade anfange zu verstehen. Was in meinen Verstand, in meinen Körper\abs in meine Magie sickert.}

Harry verstand nur die Hälfte, nickte aber.

% Aus Kapitel 13 Ortswechsel
%\enquote{Die Magie ist mein Verbündeter, und ein mächtiger Verbündeter ist sie. Ihre Energie umgibt uns, verbindet uns mit allem. Erleuchtete Wesen sind wir, nicht diese rohe Materie. Sie müssen sie fühlen die Magie die sie umgibt, hier, zwischen ihnen, mir, dem Baum, den Felsen dort, allgegenwärtig ja, selbst zwischen dem See und dem Stein auf seinem Grund.}

\trenn

Am Samstag darauf wartete bereits Professor Elber am Ende der Kammer, als die ganze Gruppe mit Professor Dumbledore eintrat.

\enquote{Ah Albus, schön, dass Sie hier sind. Wie haben sich meine Schüler denn so geschlagen?}

\enquote{Mehr als beeindruckend. Ich hatte beim ersten Mal keine Ahnung wie weit, die schon sind. Dann wollte ich Ihnen etwas zeigen und Harry hier hätte mich fast umgehauen.} Professor Elber zog beide Augenbrauen hoch und fing an zu lachen. \enquote{Ich hatte keine Ahnung, das Sie schon so viel können.}

\enquote{Das ist nur eine Sache der inneren Einstellung und der Fantasie}, meinte Professor Elber und lief Richtung einer Wand. \enquote{Vieles in der Magie hängt von der Fantasie und Vorstellungskraft des Zauberers oder der Hexe ab. Das wird im allgemeinen suggestive Magie genannt. Was steht heute an, Albus? Haben Sie sich was Bestimmtes vorgenommen?}, fragte Professor Elber nun lässig an die Wand gelehnt.

\enquote{Ich dachte}, sagte Professor Dumbledore, \enquote{dass wir einfach mit den Duellen weitermachen, die wir bereits seit zwei Wochen durchführen.}

Professor Elber schaute plötzlich interessiert und meinte nur. \enquote{Dann lasst mal sehen was\abs}

Doch Professor Dumbledore unterbracht ihn und meinte nur: \enquote{Ich weiß, Sie konnten sich nicht vorbereiten, aber wie wäre es mit einem kleinen Show-Duell?}

Professor Elber zog wieder beide Augenbrauen hoch und meinte nur: \enquote{Wir beide?}

\enquote{Ja}, antwortete Professor Dumbledore.

\enquote{Hm, und wer ist der Böse?}

\enquote{Ich dachte, das übernehmen Sie.}

\enquote{Recht gerne}, entgegnete Professor Elber. \enquote{Dann roste ich wenigstens nicht ein. Was darf ich alles nicht verwenden?}, fragte er nach.

\enquote{Wie darf ich das verstehen}, fragte Professor Dumbledore.

\enquote{Na ja, muss ich mich im gesetzlichen Rahmen bewegen?}

\enquote{Ja sicher}, sagte Professor Dumbledore.

\enquote{Schade}, sagte Professor Elber und grinste.

Harry fühlte sich in der Magengegend plötzlich etwas mulmig. \gedanke{Schade? \gst Welche Zauber kennt denn Professor Elber noch, wenn er sich eingeschränkt fühlt? \gst Oder spielt er nur mit Dumbledore, um ihm ein flaues Gefühl zu bescheren?}

Professor Elber ging in die Mitte der Kammer und zog seinen Zauberstab. Danach ging er in Position und wartete auf Professor Dumbledore.

Die Schüler stellten sich wie auf Zuruf in einem Kreis am Rande der Kammer auf. Professor Elber schlenkerte seinen Zauberstab und zog ein schützendes Kraftfeld um die Zuschauer. Dann warf er etwas, das wie ein Gedanke aussah an die Innenseite, die sich derzeit in leichtem Blau zeigte, nun rot färbte und dann verschwand.

Dann, ohne Vorwarnung, schlug Professor Elber zu und schleuderte einen Blitz auf Professor Dumbledore, welcher ihn mit einem Schildzauber abwehrte. Professor Elber fing unmerklich an die Augenbraue zu heben und leicht zu grinsen. Abermals schleuderte er einen Blitz auf Professor Dumbledore zu, der wieder versuchte ihn mit einem Schildzauber zu blocken. Doch dieses mal schlug der Blitz durch den Schild und der Zauberstab flog Professor Dumbledore aus der Hand und streifte ihn leicht am Handgelenk. Professor Elber stoppte sofort den Zauber und wartete, bis Professor Dumbledore seinen Zauberstab holte, ohne sich zu bewegen. Einige Oh's und Ah's durchzogen die Halle und widerhallten an den Wänden.

\enquote{Nicht aufhören, in einem echten Kampf habe ich auch keine Gelegenheit mir meinen Zauberstab zu holen}, ermahnte ihn Dumbledore.

\enquote{Ok}, gab Professor Elber lapidar zur Antwort und schleuderte sofort einen weiteren Blitz auf Dumbledore zu, der ihn dieses mal ablenkte und gegen das Kraftfeld lenkte, welches ihn sofort neutralisierte. Dumbledore ging nun zum Gegenangriff über und lenkte einen dicken blauen Strahl, der aus seinem Zauberstab schoss, auf Elber. Dieser reagierte sofort und schleuderte einen gelben zurück, sodass sich beide Zauber in der Mitte trafen.

Beide Kontrahenten liefen jetzt langsam sich umkreisend umeinander herum. Die Zauberstäbe immer noch verbunden.

Jetzt kamen Harry wieder die Erinnerungen auf, wie er und Lord Voldemort sich auf dem Friedhof duellierten.

\begin{rueckblick}
Er kauerte sich auf dem Friedhof hinter einem Grabstein, als Voldemort ihn einen Feigling nannte. Nein, das konnte er nicht auf sich sitzen lassen. Also stand er auf, um sich ihm zu stellen. Er musste nur schneller sein als er. Ihn entwaffnen und dann irgendwie fliehen. Also stand er auf und stellte sich ihm. Seinem Feind. Seiner Nemesis. Er erinnerte sich daran, wie Voldemort ihn alleine töten wollte und niemand eingreifen sollte. Dann sprach Voldemort den Tötungsfluch und Harry den Entwaffnungszauber. Die beiden Zauberstäbe wurden miteinander verbunden und ein Schild aus Licht baute sich um ihnen herum auf. Gesänge des Phönix’ erklangen und nach und nach erschienen Cedric, der alte Mann und seine Eltern in der Mitte, wo sich beide Zauber trafen. Seine Eltern sagten ihm, was er zu tun hatte und Cedric bat ihn, seinen toten Körper mitzunehmen. Als er den Zauber brach, sprang er zu Cedric, holte den Portschlüssel herbei und verschwand mit ihm.
\end{rueckblick}

Elber und Dumbledore liefen noch immer langsam im Kreis umeinander herum. Die Zauberstäbe immer noch verbunden. Dann, mit einer kurzen Bewegung seines Handgelenkes, zog Elber einen schmalen Strahl ab und lenkte ihn auf Dumbledore. Der Strahl traf ihn am Arm und ließ ihn zusammenzucken. Professor Elber versuchte es ein paar Mal hintereinander, doch die weiteren Male wurde er von Dumbledore aufgefangen oder mit der Hand abgelenkt. Das erzeugte Erstaunen in der umstehenden Gruppe. Elber brach den Zauber, indem er Dumbledores Strahl gegen das sie umgebende Kraftfeld lenkte. Dumbledore reagierte sofort und entwaffnete Professor Elber. Dieser konnte seinen Zauberstab nicht mehr halten und er flog in hohem Bogen in Dumbledores Hand. Professor Elber griff nach seinem Zauberstab, doch Dumbledore wehrte den Sog mit einem Schildzauber ab.

Elber stand nun ohne Zauberstab da. Harry ließ für einen Moment seinen Blick schweifen, um die Stimmung in der Kammer aufzunehmen. Er sah in bestürzte Gesichter und hörte Getuschel. \enquote{Jetzt ist er fällig. Er hat keine Waffe mehr.}

Elber hielt jetzt eine Handfläche nach oben und eine kleine Flamme erschien schwebend über seiner Hand. Sie wurde leicht größer und rötlicher. Dann schob er sie mit seiner anderen Hand weg, worauf sie herunterfiel und zu einer Schlange wurde, die nur aus Feuer zu bestehen schien. Die Schlange kam auf Dumbledore zu und wurde beständig größer. Dumbledore traf ein paar Schritte zurück, wohl um sich Zeit zu verschaffen. Er schleuderte ein paar Zauber entgegen, doch es half nichts. Die Schlange kam weiterhin auf ihn zu. Dann entsprang seinem Zauberstab ein dünner Wasserstrahl, mit dem er der Schlange den Kopf abtrennte.

Der Kopf fiel ab und löste sich auf. Der Rest der Schlange zuckte kurz zusammen, aber dann wuchsen an der Stelle, wo Dumbledore den Kopf abgeschlagen hatte zwei Köpfe heraus. Die Schlange kam immer näher.

Elber schob seine Hand nach vorne und kurz darauf wurde Dumbledore nach hinten gedrückt und fiel zu Boden. Die Schlange war nur noch wenige Zentimeter von ihm entfernt. Dann hüllte Dumbledore die Schlange vollkommen mit Wasser ein. Die Schlange erstarrte kurz, bewegte sich dann im Inneren der mittlerweile zu einer Kugel geformten Wassermasse und versuchte zu entkommen. Aber sie lebte immer noch. Dann trennte Dumbledore einen Teil des Wassers ab und schleuderte ihn auf Elber zu. Dieser war darüber so überrascht, dass er die Wassermasse gar nicht mehr abwehren konnte und so patschnass dastand. Die Schlange löste sich augenblicklich auf und Dumbledore ließ die Wasserkugel herunterfallen.

Elber griff nach seinem Zauberstab und er entwich Dumbledores Griff. Nun hatten beide Kontrahenten wieder ihre Zauberstäbe. \enquote{Machen wir weiter, oder darf ich mir erst etwas Trockenes anziehen?}, fragte Elber.

Dumbledore schnaufte \enquote{Beenden wir’s.} Dann schwang er seinen Zauberstab und das Kraftfeld löste sich auf.

Auf dem Rückweg konnte Harry Teile der Unterhaltung seines Direktors und seines Lehrers mit anhören. \enquote{Warum sind Sie zusammengebrochen, Frederick?}

\enquote{Das wissen Sie}, antwortete Professor Elber. \enquote{Voldemort hat mir eine Falle gestellt. Er hat jemanden benutzt, dem ich dahingehend vertraue. Der Person mache ich keinen Vorwurf. Nur Voldemort. Außerdem war ich geschwächt.}

\trenn

Harry stand in einem der Innenhöfe Hogwarts und wartete auf den Beginn der Stunde.

Katharina kam kurz vorbei und sagte ihm, dass es geklappt hat. \enquote{Alle sind wieder gesund. Madame Pomfrey wundert sich noch immer darüber, dass es geklappt hat.} Dann winkte sie ihm zu und verschwand zur nächsten Stunde.

Harry hatte es bereits vermutet, da Professor Dumbledore bereits wieder normal war.

Professor Elber kam mit einer Menge an Besen, die ihm schwebend folgten, um die Ecke. Sie schwebten hinter ihm und folgten den Bewegungen seines Zauberstabes. Er dirigierte sie an eine Wand und stellte sie dort schräg ab. Mit dem Schweif nach unten. \enquote{Bitte stellen Sie sich gegenüber eines Besens auf}, orderte Professor Elber seine Schüler an. Harry fragte sich wie seine anderen Mitschüler, was das mit dem Unterricht in \VgddK zu tun hat.

\enquote{Sie werden sich jetzt sicher fragen, weshalb ich Ihnen in \VgddK Besen hinstelle.} Fast alle Schüler nickten und Professor Elber grinste kaum merkbar. \enquote{Lassen Sie mich es so ausdrücken, Sie werden es am Ende der Stunde erkennen. Holen Sie nun alle ihre Besen zu sich.} Einige Schüler begannen bereits auf ihre Besen zuzulaufen. \enquote{Halt}, schrie, Professor Elber \enquote{ich sagte nichts davon, dass Sie auf Ihre Besen zulaufen sollen. Zurück auf Ihre Ausgangspositionen.} Die Schüler, welche bereits vorgelaufen waren, traten schüchtern zurück und gingen wieder auf ihre Plätze. \enquote{Nutzen Sie die Magie in Ihnen, um die Besen zu sich zu holen. Denken Sie, dass Ihnen der Besen entgegenkommt und strecken sie Ihre Hand aus, um ihn zu empfangen.}

Die Schüler fingen an zu üben und ihre Besen schwebten langsam aber zielsicher zu ihnen hin. Nach einigen Minuten kam Professor McGonagall auf Professor Elber zu und zog ihn etwas zur Seite. Harry konnte ihre Unterhaltung kaum hören.

\enquote{Was?}, meinte Professor Elber. \enquote{Ich kann doch nicht \gst}

Professor McGonagall unterbrach ihn. \enquote{Doch, Frederick.} Sie zog ihn noch ein Stück weiter weg. \enquote{Alle anderen sind Krank. Und die Leute im Ministerium \gst}

Doch Harry konnte nichts mehr hören, sie sprachen nun sehr leise und waren zudem in sicherer Entfernung. Er konnte aus Professor Elbers Gesicht nur leichten Missmut erkennen. Er wandte sich wieder seinem Besen zu und ließ ihn, nachdem er ihn in seiner Hand gehalten hatte, wieder zurück schweben. Er war so auf seine Aufgabe konzentriert, dass er gar nicht merkte, dass sich Professor Elber hinter ihn gestellt hatte und ihm interessiert zusah.

\enquote{Zu faul zum Laufen, Harry?}, fragte er ihn.

Harry erschrak. Sein Besen zitterte gefährlich in der Luft. Nur mit etwas Mühe konnte er ihn still halten. Er drehte sich um und meinte schuldbewusst, \enquote{Ja.}

\enquote{Na na na, kein Grund Schuld zu zeigen. Sie haben nur mitgedacht und mir den nächsten Schritt schon voraus genommen.} Jetzt sagte er zur Klasse gewandt: \enquote{Wenn Sie das Herholen Ihres Besens beherrschen, dann schicken Sie ihn wieder zurück an die Wand.}

Die meisten Schüler taten dies dann auch, ohne besonders zu reagieren.

\trenn

Aus einer Tür hörte Harry wie sich jemanden unterhielt. Vorsichtig kam er näher.

\enquote{Ahh! Es brennt, der Dunkle Lord ruft die Todesser.}

\enquote{Shh Draco. Komm und setz dich.}

Er hörte Schritte. Er kannte die Stimmen genau.

Dann hörte er einen Singsang, den er nicht deuten konnte. \zauber{Sema nibina, sema duljana. Sema nibina, sema duljana. Sema nibina, sema duljana.}

Harry versuchte durch ein Schlüsselloch etwas zu erhaschen, aber er fand keines.

\enquote{Besser, Draco?}, fragte die Stimme weiter.

\enquote{Es kribbelt nur noch leicht.}

\enquote{Gut \gst Warum hast du es dir überhaupt geben lassen?}

\enquote{Mein Vater. Ich hatte keine Wahl. Er hätte mich sonst umgebracht.}

\enquote{Möchtest du es denn?}

\enquote{Nein, aber ich bekomme es wohl nicht los.}

\enquote{Sicher?}

\enquote{Es wurde magisch eingebrannt, Frederick. Man kann es nicht entfernen.}

\enquote{Aber du möchtest es loshaben?}

\enquote{Sicher.}

\enquote{Du möchtest also im Sommer mit kurzer Kleidung herumlaufen, ohne dass du ständig das Mal schminken, oder sonst überdecken musst.}

\enquote{Ja.}

\enquote{Dann sag es. Schau auf das Mal und sage ihm, dass du es loshaben möchtest. \gst Dann werde ich versuchen, dir zu helfen. \gst Aber ich muss dir auch sagen, es brennt, wenn man es entfernt. Zwar nicht so stark, wie zu dem Zeitpunkt an dem man es bekam, aber es schmerzt doch. Es versucht sich zu wehren. Das ist mächtige schwarze Magie. Es strahlt eine Signatur aus, die man spüren kann.}

Dann herrschte eine Weile Stille. Harry stand neben der Tür und konnte die Unterhaltung klar und deutlich, aber dennoch durch die Tür gedämpft wahrnehmen. \gedanke{Warum? Die Türen sind doch aus dickem Holz, die Steinwände schlucken den Schall? Warum?}

\enquote{Dumbledore. Kann er es auch spüren?}

\enquote{Ich denke schon. Immer, wenn du in seiner Nähe bist.}

Draco dachte nach. \gedanke{Er hat nie etwas gesagt. Hat mich nicht darauf angesprochen. Warum nur?}, dachte er. \enquote{Wieso hat er mir deswegen keine Vorhaltungen gemacht?}, fragte er leise in den Raum hinein.

\enquote{Weil er immer an dich geglaubt hat und es immer noch tut. Er sieht deinen guten Kern. \gst Wollen wir beginnen?}

Dann herrschte eine Weile Stille. Harry stellte sich vor, wie Draco durchatmete und dann nickte.

\zauberextase{Erdrom Srom!}, hörte er plötzlich und erschrak, denn er hörte Draco schreien. Doch dieser beruhigte sich schließlich recht schnell. Harry verkrampfte, als ob er die Schmerzen selber spüren würde. Dann hörte er wieder diesen Sing sang. \zauber{Sema nibina, sema duljana. Sema nibina, sema duljana. Sema nibina, sema duljana.}

Er umfasste instinktiv sein Amulett und nahm nun verschwommen eine Szene wahr. Es war wie in einem Traum. Sein Lehrer und Draco saßen auf einer Couch und Elber hielt die Spitze seines Zauberstabes auf Dracos Arm. Das dunkle Mal verblasste.

Professor Elber hob seinen Kopf und es hatte den Anschein, dass er Harry direkt in die Augen sah. Danach schaute er wieder auf Dracos Arm. \zauber{Sema nibina, sema duljana.}

Harry ließ sein Amulett los und rannte lautlos um die nächste Ecke, um dann durch das Schloss zu rennen. Seine Schritte hallten an den Wänden wieder. Er wollte nur noch weg.

Etwas später unterhielt er sich gerade mit Luna auf dem Weg in die Große Halle, als ihm plötzlich schlecht wurde und sich sein Sichtfeld zunehmend schmälerte. Er fühlte sich so, als ob er jeden Moment zusammenbrechen würde. Dann wurde es schwarz um ihn und er sank auf dem Boden zusammen. Harry bekam es nicht mit, aber Luna brach ebenfalls zusammen und beide wurden auf die Krankenstation gebracht.

Als Harry das Bewusstsein wieder erlangte, standen Professor Dumbledore und Professor McGonagall um das Krankenbett neben ihm herum und an seinem Krankenbett standen Professor Flitwick und Madame Pomfrey.

\enquote{Und, Harry, wie geht es dir?}, fragte Professor Dumbledore die Person neben ihm.

\enquote{Mir geht es gut}, antwortete Harry. \enquote{Aber ich bin hier.}

Professor Dumbledore und Professor McGonagall drehten sich um und sahen ihn an. \enquote{Ah Miss Lovegood}, antwortete Professor McGonagall, \enquote{schön Sie wieder wach zu sehen.}

Harry Augen weiteten sich. \enquote{Aber ich bin Harry, Harry Potter. Erkennen sie mich nicht mehr?}

Plötzlich hörte er aus dem Nebenbett eine Stimme, eine Stimme, die ihm bekannt vorkam, die er aber nicht zuordnen konnte. \enquote{Das habe ich denen auch schon gesagt, aber mir glaubt natürlich nie einer was.}

Harry stutze, hob seinen Oberkörper und blickte nun der Person neben ihm ins Gesicht. Seine Kinnlade fiel herunter, als er in das fremde Gesicht sah. \enquote{Luna? Bist du das?}, fragte er mit einer Stimme die, wie ihm plötzlich auffiel, anders klang als sonst. Harry sah sich ins Gesicht. Er sah seinen Körper, aber er war nicht er selbst. Er blickte sich in der Krankenstation um und bemerkte einen kleinen Spiegel auf dem Krankentisch neben ihm. Er nahm ihn und schaute sich an. Sein Spiegelbild zeigte Luna.

\enquote{Ich habe denen schon gesagt, dass ich nicht Harry bin, aber sie sagten mir, ich sei verwirrt}, sagte Luna (in Harrys Körper).

Harry wusste nicht, wie ihm geschah, irgendwie hatte er den Körper mit Luna getauscht.

\enquote{Professor Dumbledore, anscheinend haben Luna und ich unsere Körper getauscht.}

Madame Pomfrey zuckte zusammen.

\enquote{Wie ist das möglich?}, fragte Luna.

Dumbledore antwortete, \enquote{Tja, Luna, ich habe keine Ahnung.} Dann zu Madame Pomfrey gewandt. \enquote{Können Sie sich einen Reim darauf machen Poppy?}

\enquote{Nein}, antwortete sie knapp. \enquote{Wie ist das passiert?}

\enquote{Ich weiß es nicht}, sagten Harry und Luna.

\enquote{Tja, bis wir herausgefunden haben, was mit Ihnen los ist, werden Sie sich daran wohl gewöhnen müssen}, sagte Madame Pomfrey.

Harry musste ein Lachen unterdrücken, da er bereits mehrmals in Luna war.

\enquote{Oh, daran haben wir uns schon gewöhnt}, meinte Luna. Harry drehte reflexartig seinen Kopf zu ihr (in seinem Körper) und auch die anderen Professoren und Madame Pomfrey ebenso. \enquote{Äh, ich meinte, daran werden wir uns schon gewöhnen}, schob Luna schnell hinterher, ebenfalls ein Lachen unterdrückend.

Madame Pomfrey hob eine Augenbraue. \enquote{Sie beide bleiben heute auf jeden Fall hier, da Sie eine leichte Gehirnerschütterung haben. Morgen früh können Sie dann gehen.}




\begin{kommentar}
Katharina wird von Hagrid in den Verbotenen Wald geschickt. Dort trifft sie auf Medusoner die aussehen wie - Schlümpfe.
\end{kommentar}

\begin{kommentar}
Nachdem Katharina zum zweiten Mal ihre Prüfung in Griechenland abgelegt hat, erinnert sie Harry daran, dass die Magie ihr Verbündeter ist. Und ein mächtiger Verbündeter ist sie.
Dieselben Worte hatte auch Yoda in Krieg der Sterne zu Luke gesagt. Eine weitere nette Anspielung.
\end{kommentar}

\begin{kommentar}
Etwas später sind Draco und Frederick Elber in einem kleinen Raum und Draco will das Dunkle Mal loswerden. Frederick benutzt einen Zauber. Schaut mal nach welchen und lest diesen dann rückwärts.
\end{kommentar}

\chapter{Seelenwanderung}


Die Tür zur Krankenstation ging auf und Professor Elber trat herein. Mit einer Hand hielt er sein rechtes Handgelenk und meinte zu Madame Pomfrey \enquote{Poppy, könnten Sie mir helfen? Mein Handgelenk ist, glaube ich, gebrochen.} Sie führte ihn an das Bett nebenan und untersuchte seine Hand. Dann verschwand sie wortlos in ihr Büro, um etwas zu holen. Professor Elber drehte sich zu Harry und Luna und sagte zu beiden: \enquote{Hallo Harry}, zu Harrys Körper schauend, \enquote{Hallo Luna}, zu Lunas Körper schauend, \enquote{auch der Gnade unserer Krankenschwester ausgesetzt?}

Beide mussten schmunzeln. \enquote{Ja Professor, aber ich bin nicht Harry}, sagte Luna (immer noch in Harrys Körper steckend).

\enquote{Genau}, meinte Harry (in Lunas Körper steckend), \enquote{und ich bin nicht Luna.}

Professor Elber hob ungläubig eine Augenbraue. Schweigend blickte er zwischen Luna und Harry hin und her. Mit seiner gesunden Hand griff er sich an sein Kinn und rieb es leicht, so als ob er überlegen würde, und sah beide immer noch abwechselnd an.

Die Tür zu Madame Pomfreys Büro ging auf und sie kam mit einer Dose Salbe und ein paar Verbänden herein. Sie öffnete den Schraubverschluss und salbte vorsichtig Professor Elbers Handgelenk ein, um es danach zu verbinden. Anschließend sagte sie \enquote{Bleiben Sie noch hier, solange die Salbe wirkt und halten Sie Ihr Handgelenk ruhig. Ich komme dann wieder zum Verband entfernen. Dann können Sie gehen.} Professor Elber nickte, sein Blick immer noch auf Luna und Harry gerichtet.

\enquote{Aber was soll jetzt aus den beiden werden?}, fragte Professor McGonagall.

\enquote{Was soll mit den beiden schon sein; die werden ganz normal ihren Unterricht weitermachen. Alle anderen Möglichkeiten sind zu kompliziert und führen zu mehr Problemen, als wir lösen oder handhaben könnten}, sagte Professor Elber. \enquote{Vor allem erzeugt es zu viele Gerüchte. Ein Pärchen allein auf der Krankenstation in der Nacht. Eine mysteriöse Krankheit. Die Schüler werden sonst was denken. So wissen zwar alle Bescheid, können aber damit umgehen und es entstehen weniger Gerüchte.}

Professor McGonagall drehte sich zu ihm um und meinte: \enquote{Wie kannst du so was sagen\abs}

\enquote{Ich stimme ihm zu}, antwortete Professor Dumbledore. \enquote{wenn ich mir die Alternativen durch den Kopf gehen lasse. Es wäre dann so, dass die beiden Extrastunden bekommen würden, oder ihren Stoff und die Hausaufgaben hierher geliefert bekommen.}

\enquote{Ja, aber}, entgegnete Professor McGonagall.

\enquote{Nein, nein. Es ist besser, wenn die beiden ihren Unterricht so normal wie möglich durchziehen.}

\enquote{Aber, wie stellen sie sich das vor? Mister Potter in Miss Lovegoods Körper im Jungenschlafsaal der Gryffindors}, hielt ihm Madame Pomfrey entgegen.

\enquote{Ja, was soll damit schon sein. Wir reden immer noch von Harry \gst und nichts für ungut Luna, auch von Ihnen \gst er wird wohl kaum die Lust auf ein romantisches Abenteuer mit einem seiner Zimmergenossen haben, nur weil er jetzt in einem Frauenkörper steckt.}

Betretene Stille herrschte nun in der Krankenstation, die nur durch ein leises Kichern von Professor Flitwick durchdrungen wurde, der immer noch an Harrys Bett saß.

Plötzlich stand Luna auf und ging Richtung Toilette. Sie nahm den Griff in die Hand und drückte ihn herunter, als sie Harry ansah und meinte: \enquote{Hilfst du mir mal?}

Er antwortete ihr: \enquote{Du wirst doch wohl noch allein aufs\abs}, stockte dann kurz und sprang aus dem Bett heraus. \enquote{In Ordnung, ich komme} und ging auf Luna zu. Kurz vor der Toilettentür angekommen drehte er sich noch einmal kurz um und sah Professor Dumbledore, wie er mit weit aufgerissenen Augen und der Hand vor dem Mund ein lautes Lachen unterdrückte. Harry musste schmunzeln.

Nachdem Luna ihr Bedürfnis verrichtet hatte und beide wieder aus der Toilette zurückgekehrt waren, meinte Professor Dumbledore: \enquote{Ich werde am besten eure Zimmerkameraden zur Mittagszeit hier herholen, damit diese schon vorab Bescheid wissen und dann zum Abendessen es wohl oder übel bekannt geben müssen.} Er wandte sich zu Madame Pomfrey. \enquote{Dürfen die zwei zum Abendessen heruntergehen, bevor sie die Nacht hier verbringen?}

\enquote{Von mir aus}, antwortete sie.

\enquote{Also dann}, meinte Professor Dumbledore, \enquote{ich werde dann nach dem Mittagessen hier herkommen und eure Zimmerkameraden mitbringen. Gehen wir?}, fragte er die anderen Professoren und marschierte schon Richtung Tür, um kurz darauf die Krankenstation zu verlassen.

Die anderen Professoren schauten die beiden Patienten nur an und verließen kurz darauf auch die Krankenstation.

\enquote{Also}, meinte Professor Elber plötzlich, als die drei allein in der Krankenstation waren, \enquote{was genau ist geschehen? Und erzählt mir keine Märchen, oder die Geschichte für die anderen Professoren, wenn ihr wollt, dass ich euch helfe, ohne dass die anderen Lehrer oder Schüler davon Wind bekommen.} Harry und Luna schauten sich nur an. Professor Elber machte weiter. \enquote{Habt ihr vorher schon etwas Ähnliches erlebt? Und vor allem wann?}

Harry und Luna schauten sich immer noch an.

Dann begann Luna plötzlich zu sprechen. \enquote{Schon öfters. Wir können schon seit Monaten die Gedanken des anderen lesen und wenn wir uns darauf konzentrieren durch die Augen des anderen sehen und mit den Ohren des anderen hören und manchmal auch die Kontrolle über den anderen übernehmen. Aber am besten war es beim Sex.} Harry errötete sofort und wollte schon protestieren, hielt sich aber aus irgendeinem Grund zurück. \enquote{Dann war es so, als ob wir für kurze Zeit die Körper gewechselt haben.}

Professor Elber nickte und meinte: \enquote{Das würde einen Teil erklären. Aber wo habt ihr miteinander geschlafen? Das ist die viel interessantere Frage.}

\enquote{Professor}, sagte Harry jetzt, \enquote{ich glaube, Sie wissen sehr genau, wo. Sie haben uns dort schon einmal getroffen und etwas von Sardak und Selvine gesagt.}

\enquote{Ah ja. Westflügel dritter Stock?}, fragte Professor Elber.

\enquote{Genau}, antwortete Luna.

\enquote{Wie viele wissen noch davon, außer euch Zweien?}

\enquote{Niemand}, log Harry.

Doch Luna warf ein: \enquote{Eigentlich alle Pärchen in Hogwarts von denen wir wissen.}

Harry schaute vollkommen geschockt Luna an (eigentlich seinen Körper). \enquote{Luna, das sollte doch keiner wissen, besonders kein Lehrer.} Harry zuckte zusammen und biss sich auf die Zunge. Das wollte er vor Professor Elber nicht sagen.

\enquote{Ah ja, dann bin ich wohl der erste und einzige Lehrer, der davon je etwas erfahren wird.}

\enquote{Ja}, antwortete Harry kleinlaut.

\enquote{Das macht nichts, ihr könnt mit den anderen Pärchen reden und mich, wenn ihr wollt, mal einladen. Ich werde einen Teufel tun und euch verraten. \gst Ich glaube, ich habe euch gegenüber schon einmal etwas aus Versehen ausgeplaudert; über Selvine und Sardak. Ich war es, der ihnen damals die nötigen Sprüche und Zauber lehrte. Aber dass daraus so etwas wurde, wusste ich nicht. Es scheint, als ob da etwas schiefgelaufen ist, und das muss korrigiert werden. \gst Haben die anderen Pärchen auch solche Symptome gezeigt?}

\gedanke{Symptome}, dachte Harry, \gedanke{das hört sich nach einer Krankheit an.}

\enquote{Nicht dass uns welche bekannt wären}, antworte Luna frei heraus und dann zu Harry gewandt: \enquote{Erinnerst du dich an den Eintrag über die Empfindungen der Pärchen, ob es dem Partner gut geht?} Harry nickte. \enquote{Es kommt nur sehr selten vor}, sagte Luna zu Professor Elber gewandt, \enquote{dass zwischen den Pärchen mehr ist, stand in dem Buch, dass die beiden hinterlassen haben.}

Professor Elber hob eine Augenbraue. \enquote{Davon haben die mir nie etwas erzählt.}

\enquote{Sie haben die beiden gekannt?}, fragte Harry.

\enquote{Ja}, antwortete Professor Elber. \enquote{Wir hatten uns in den Ferien öfter getroffen und uns sehr gut verstanden.}

Die Tür zur Krankenstation ging wieder auf und Madame Pomfrey kam mit einem kleinen Schubwagen voller Flaschen, Tiegeln und kleinen dampfenden Kesseln zurück. Vor dem Krankenbett von Professor Elber blieb sie stehen, schaute auf ihre Uhr, die sie immer um den Hals hängen hatte, ging zu Professor Elber und nahm ihm den Verband ab. \enquote{Sie können gehen, Frederick.}

\enquote{Danke}, antwortete er und stand auf. Dann wandte er sich Harry und Luna zu und meinte: \enquote{Ich sehe euch später. Ich muss mich jetzt erst einmal einlesen.}

\enquote{Wie geht es Draco?}, fragte Harry plötzlich.

\enquote{Seelisch etwas angespannt, aber körperlich ganz gut}, antwortete ihm sein Professor. Dann ging er weiter, verließ die Krankenstation und lies einen nachdenklichen Harry, sowie Luna mit Madame Pomfrey zurück.

Diese nahm wieder ihren Wagen, fuhr damit in ihr Büro und schloss die Tür.

\enquote{Musste das sein, dass du so freizügig gegenüber Professor Elber warst?}, zischte Harry.

\enquote{Weißt du, Harry}, antwortete Luna, \enquote{er ist vielleicht der einzige, der uns hierbei helfen kann. Hast du Dumbledores ratloses Gesicht gesehen? Und wenn er keinen Rat weiß und auch Madame Pomfrey mehr als überfordert ist \gst Ich möchte nicht nach St. Mungo's.}

\enquote{Aber dann gleich Professor Elber davon zu erzählen, wie wir Sex hatten?}

\enquote{Erinnerst du dich}, machte Luna weiter, \enquote{als wir ihn im dritten Stock getroffen haben? Er wusste von Selvine und Sardak und zudem habe ich so ein Gefühl bei ihm, dass wir ihm trauen können.}

Harry sagte nichts, sondern schaute stattdessen an sich herunter.

\enquote{Gefällt dir mein Körper, Harry?}, fragte Luna.

\enquote{Ja \gst äh nein, \gst äh \gst ja}, stammelte Harry.

Luna lachte nur. \enquote{Weißt du Harry, von unserer gegenwärtigen Lage abgesehen, ich glaube nicht, dass unsere Beziehung weiterhin einen Sinn hat.}

Harry schaute erstaunt auf, nickte aber stumm. Er atmete tief ein und fügte dann hinzu. \enquote{Da hast du vermutlich recht.} Es schmerzte zwar, aber irgendwie hatte Luna recht. Sie liebten sich wirklich nicht mehr. Die Trennung vor dem Valentinstag hatte es gezeigt.

Plötzlich ging die Tür auf und Ron und Hermine stürmten herein. \enquote{Alles in Ordnung mit dir, Harry?}, fragte Hermine, ihre Hände bereits um Harrys Hals geschlungen.

\enquote{Ja}, antwortete Harry und Hermine drehte sich erstaunt um.

\enquote{Hallo Luna, ich habe dich gar nicht bemerkt}, meinte sie.

\enquote{Ich bin nicht Luna, ich bin Harry.}

Hermine schaute erstaunt und drehte sich dann zum vermeintlichen Harry um.

\enquote{Ja, das stimmt, ich bin Luna}, antwortete Luna in Harrys Körper. \enquote{Wir haben aus irgendeinem Grund unsere Körper getauscht.}

Hermines Augen weiteten sich und auch Ron war sprachlos. Die Tür zur Krankenstation ging abermals auf und Professor Dumbledore kam mit den anderen drei Jungs aus Harrys Schlafraum und weiteren vier Ravenclaw-Schülerinnen herein. \enquote{Ah ich seh' schon, Hermine und Ron wissen bereits Bescheid.}

\enquote{Nein,} warf Harry ein, \enquote{sie sind gerade eben erst hereingekommen und wissen nur, dass ich \gst äh.}

Dumbledore winkte ab, drehte sich herum und sagte dann: \enquote{Setzt euch erst einmal irgendwo hin. Wir müssen eine private Unterhaltung führen.} Er lief durch die Gruppe hindurch und schloss die Türen zur Krankenstation. Professor Dumbledore kehrte zurück und setzte sich auf das Bett an Harrys (Lunas Körper) fußende, die Füße leicht baumelnd und begann zu erzählen. \enquote{Bevor hier einige denken, ich sei verrückt; ihr bekommt die Gelegenheit all dies nachzuprüfen. \gst Zunächst einmal, hier seht ihr Harry,} er deutete auf Lunas Körper \enquote{und Luna}, er deutete auf Harrys Körper. \enquote{Und nein, das ist kein Missverständnis. Die beiden haben wirklich ihre Körper getauscht}, erklärte ihnen Dumbledore.

Betretenes Schweigen herrschte auf der Krankenstation. \enquote{Was soll das bedeuten?}, fragte Elisabeth, eine Zimmerkollegin von Luna.

\enquote{Dass bis auf Weiteres die beiden ihren gewohnten Tagesablauf weiterführen, bis wir eine Lösung haben. Ich werde beim Abendessen die Problematik der ganzen Schule mitteilen, aber ihr als Zimmerkollegen solltet es schon vorher wissen, da ihr mit den beiden ja im selben Zimmer schlaft. Es wird also für euch genauso eine Umstellung werden. Zumindest optisch.}

\enquote{Heißt das}, unterbrach ihn Seamus, \enquote{dass Harry} (er zeigte auf Luna), \enquote{ganz normal bei uns schlafen wird?}

\enquote{Ich denke ja}, antwortete Professor Dumbledore.

\enquote{Ja aber, er ist ein Mädchen,} entgegnete Neville.

\enquote{Nur weil er in einem weiblichen Körper steckt, heißt das noch lange nicht, dass er auch wie ein Mädchen denkt. Es handelt sich hier immer noch um Harry und Luna}, antwortete Professor Dumbledore.

Man hörte einige Schüler schlucken und mit großen Augen die beiden anschauen. \enquote{Mir ist klar, das wird sicherlich für alle hier peinlich, besonders für die beiden hier, aber es gibt keine andere Lösung, ohne noch größere Probleme.}

\enquote{Meinen Sie nicht, dass es komisch für ein Mädchen wäre\abs}, fing Seamus an, aber Harry unterbrach ihn.

\enquote{Nur weil ich im Körper eines Mädchens stecke, heißt das noch lange nicht, dass ich mich plötzlich zu Jungs hingezogen fühle und\abs}

Dumbledore lachte und auch die anderen konnten ihr Lachen nun nicht mehr unterdrücken.

Nach einer Weile meinte Dumbledore: \enquote{Ich gehe dann mal wieder, ihr könnt euch noch etwas unterhalten, bevor ihr dann zum Unterricht müsst. Harry und Luna kommen erst zum Abendessen.} Er drehte sich um und sagte zu den beiden: \enquote{Ich hole euch dann ab.}

Harry und Luna nickten.

\enquote{Tja Harry}, meinte Seamus, der Harrys Körper ansah bevor er seinen Irrtum bemerkte und schnell zu Luna wechselte, \enquote{wie fühlst du dich?}

\enquote{Komisch}, antwortete Harry.

\enquote{Aber ich denke, diese seltsame Erfahrung hat vielleicht auch etwas Gutes}, machte Luna weiter.

\enquote{Und was?}, wollte Dean wissen.

\enquote{Weiß ich noch nicht, aber alles hat sicherlich einen Zweck. Zumindest ist es eine gute Erfahrung.}

\enquote{Ob die Erfahrung so gut ist, stellt sich noch raus}, meinte Harry.

Die Tür zur Krankenstation ging wieder auf und Professor Vector steckte ihren Kopf durch den offenen Spalt der Tür. \enquote{Miss Lovegood, alles klar. Sie können mit Harrys Körper nun zu den Mädchenschlafsälen. Es ist alles mit Professor Flitwick abgeklärt.} Sie zog ihren Kopf wieder zurück und schloss die Tür.

\enquote{Was war das denn?}, fragte eine von Lunas Zimmergenossinnen, die Harry nicht kannte.

Luna antwortete ihr. \enquote{Du weißt doch, Sandra, dass normalerweise Jungs nicht zu den Mädchenschlafsälen dürfen. Vermutlich hat Professor Flitwick das für Harrys Körper aufgehoben.}

\enquote{Ah}, sagte sie.

\enquote{Und was passiert jetzt?}, fragte Elisabeth mit leichtem Glitzern im Auge.

\enquote{Was soll schon sein?}, antwortete Luna. \enquote{Ich werde ganz normal weitermachen. Die Klassen besuchen, unsere abendlichen Gespräche führen.}

Jetzt herrschte betretenes Schweigen im Raum. Dann ertönt die erste Glocke zum Unterrichtsbeginn. Kurz darauf waren die beiden auf der Krankenstation allein.

\enquote{Harry, da gibt es noch etwas, was du wissen musst}, sagte Luna nach einigen Minuten. Er drehte sich zu ihr um, verließ sein Bett und lief zu ihrem. Dann setzte er sich neben sie. \enquote{Ich habe bald meinen Monatszyklus}, sagte Luna und blickte Harry an.

Er bekam große Augen. \gedanke{Was will mir Luna sagen?}, dachte er.

\enquote{Meine Monatsblutung}, fügte Luna hinzu. \enquote{Jedes Mädchen ab einem gewissen Alter bekommt sie.} In Harry kam leichte Panik auf. \enquote{Und außerdem mag ich keine Binden. Ich benutze Tampons.} Harry wurde schwindelig. Er ließ sich auf das Bett zurückfallen. \enquote{Also hör mal. Das ist nichts Ekliges. Ich mache das an fünf Tagen im Monat durch. Eigentlich unterliegt das einem 28-Tage Zyklus. Ähnlich dem des Mondes. Wenn du weißt, wie der Mondzyklus ist, dann wirst du keine Probleme mit meinem \gst deinem Zyklus haben.}
% Die Dauer der Menstruation liegt in der Regel zwischen vier und sechs Tagen; die Stärke ist individuell verschieden. Durchschnittlich verliert die Frau während der Menstruation zwischen etwa 60 und 80 ml Blut. Das Blutungsmaximum liegt am zweiten Tag.
%Der Follikelsprung (auch Eisprung oder Ovulation), findet unter dem Einfluss von LH bei einem 28 Tage dauernden Zyklus ungefähr in der Mitte desselben statt.
%Während einer regelrechten Blutung von normaler Dauer (drei bis fünf Tage) und Stärke (Eumenorrhoe genannt), gehen ungefähr 30 bis 60 Milliliter Blut verloren (Werte zwischen 10 und 80 ml werden als normal angesehen, das Blutungsmaximum liegt meist am zweiten Tag).

Harry entschloss sich die Augen zu schließen und Lunas Ausführungen zu lauschen. \enquote{Fahr fort, Luna.}

\enquote{Ich finde Binden nicht so ansprechend. Ich sehe sie als Fremdkörper in meinem Slip an. Deshalb nehme ich während der fünf Tage Tampons. Sie hinterlassen zudem ein angenehmes Gefühl beim Einführen. Du musst dich daran gewöhnen, sie zu wechseln. Die ersten beiden Tage zweimal, mittags und abends. An den anderen beiden Tagen reicht es abends. Ich mache das immer nach dem Waschen. Erschrecke nicht, wenn du Blut darauf siehst. Am zweiten Tag kommt am meisten Blut zum Vorschein. Wir können das nachher mal üben. Oder du nimmst dir eine von meinen Zimmergenossinnen zu Hilfe. Die werden dir sicher gerne helfen.}

Harry öffnete seinen Augen wieder und sah Luna ins Gesicht. Er beruhigte sich wieder und fing leicht an zu lächeln. \enquote{Wäre ich als Mädchen geboren worden, dann wäre das alles ganz normal für mich}, sagte er.

Luna nickte und schloss ihre Augen.

Harry ließ die letzten Tage und Wochen an seinem inneren Auge vorbeigleiten. Dann dachte er an Ginny. Jedes Mal, wenn er sie sah, machte sich in seinem Inneren etwas bemerkbar und er hatte so etwas wie ein schlechtes Gewissen. Er konnte es sich nicht erklären. Dann erinnerte er sich an seinen Kuss mit ihr, in Arabellas Haus, und wie er zu ihr sagte, sie sei wie eine Schwester für ihn. \fluestern{Nein, du bist viel mehr für mich}, flüsterte er leise in den Raum hinein.

\fluestern{Ich hoffe doch, du meinst mich}, hörte er jemand sagen und spürte einen Kuss auf seiner Wange.

Erschrocken öffnete er die Augen und sah Ginny. Doch als er richtig realisierte, dass sie es war, verblasste sie und war einen Augenaufschlag später verschwunden. Harrys Magengegend wurde wärmer und er spürte Schmetterlinge in seinem Bauch, als er an sie dachte. Dieses Gefühl hatte er bei Luna nicht, obwohl er sich zu ihr hingezogen fühlte. Doch diese Anziehung war mit dem Körpertausch irgendwie verflogen. Und vermutlich würde sie auch nicht mehr wiederkommen.

\trenn

Als die Tür zur Krankenstation aufging und Dumbledore hereinkam, saßen Harry und Luna auf Lunas Bett und unterhielten sich. Beide schauten auf, als der Schulleiter die Krankenstation betrat, auf sie zulief und sich neben Luna auf das Krankenbett setzte. Nach einem Moment des Schweigens fragte Dumbledore die beiden: \enquote{Bereit?}

\enquote{Ja}, antworteten beide fast gleichzeitig.

\enquote{Gut. Es läuft folgendermaßen ab: Wir gehen jetzt gemeinsam in die Große Halle und ihr lauft neben mir her bis nach vorne zum Lehrertisch. Dort werde ich dann die Situation den anderen erklären. Alles klar?}

\enquote{Ja}, antworteten beide abermals.

Nachdem Dumbledore aufgestanden war, folgten ihm Harry und Luna. Schweigend gingen sie die Treppen hinunter auf dem Weg zur Großen Halle, neben Professor Dumbledore herlaufend.

\enquote{Gibt es keine andere Möglichkeit, als es allen zu sagen?}, fragte Harry nun doch.

\enquote{Ich fürchte nein, Harry. Du weißt ja, wie sich Nachrichten, und vor allem Geheimnisse, im Schloss verteilen. Ich möchte Spekulationen so weit als möglich vorbeugen.}

\enquote{Verstehe}, fügte Harry hinzu.

Die Türen der Großen Halle waren bereits geschlossen, als die drei unten ankamen. Dumbledore öffnete sie und erlangte somit die Aufmerksamkeit aller. Gemeinsam mit Luna und Harry lief er durch die Tischreihen Richtung Lehrertisch. Vorne angekommen drehte er sich Richtung Saal und wartete, bis Harry und Luna ihre Plätze neben ihm eingenommen hatten. Sie trugen Umhänge ohne spezielle Hauszuordnung und ihre Krawatten waren schwarz und mit dem Hogwarts-Wappen bestückt.

\enquote{Diese beiden hier}, fing Dumbledore an und augenblicklich verstummt der ganze Saal. Es war so leise, dass man eine Stecknadel hätte fallen hören können, \enquote{haben aus einem mir nicht bekannten Grund\abs}, leises Kichern war vonseiten der Slytherins zu hören und insbesondere Malfoy schaute, wie Harry auffiel, auf Luna.

Harry wunderte sich, warum er ausgerechnet Luna anschaute, aber dann erinnerte er sich, dass er ja in Lunas Körper steckte. \gedanke{Lunas Körper}, dachte Harry, \gedanke{ich stecke tatsächlich in einem Mädchenkörper.}

Dumbledore fuhr fort, \enquote{ihre Körper getauscht.} Das leise Kichern der Slytherins verstummte und Ratlosigkeit machte sich auf fast allen Gesichtern in der Großen Halle breit. \enquote{Es wird sich für die beiden nichts ändern, außer dass Miss Lovegood hier}, Dumbledore zeigte auf Harrys Körper, \enquote{und Mister Potter hier}, Dumbledore zeigte auf Lunas Körper, \enquote{im Körper des anderen Stecken. Sie werden wie gewohnt in ihren jeweiligen Häusern leben, arbeiten und sich ihren sozialen Kontakten widmen.}

Vereinzelte Schüler begannen zu lachen und zu kichern und vereinzelte Huhu-Rufe drangen durch die Große Halle.

\gedanke{Soziale Kontakte}, dachte Harry. \gedanke{Ich hatte mit Luna genug soziale Kontakte. Aber was, wenn ich mich in ein Mädchen verliebe? In Lunas Körper kann ich mich doch kaum an ein Mädchen ran machen. Vor allem nicht an Ginny. Außer sie steht auf Frauen.} Harry verwarf seine Gedanken wieder und lauschte weiter Dumbledores Erklärungen.

\enquote{Beide wurden heute Morgen bewusstlos aufgefunden und in den Krankenflügel gebracht.} Dumbledore legte seine Hände auf die Rücken der beiden und schloss: \enquote{Ihr könnt euch setzen und euer Abendessen genießen, falls ihr in eurem Zustand Hunger habt.}

Harry und Luna trotteten zu ihren Plätzen an ihren jeweiligen Haustischen und nahmen neben ihren Kollegen Platz.

In dieser Nacht konnte Harry nicht schlafen, obwohl er müde in seinem Bett lag. Er ließ seine Gedanken gleiten und schaute ohne besonderes Ziel durch sein Mondlicht-durchflutetes Zimmer. Seit einiger Zeit schlief er wieder mit offenen Bettvorhängen, da es draußen wieder etwas wärmer wurde. \gedanke{Ich glaube, ich muss meine morgendliche Jogging-Runde abkürzen, ich glaube nicht, dass Lunas Körper an das gewöhnt ist \gst Lunas Körper.} Er schaute sich nun vorsichtig im Zimmer um und hörte Neville und Seamus leicht schnarchen. Dean lag wie immer auf dem Bauch und auch Ron atmete ruhig und gleichmäßig. Langsam zog er seine Bettdecke zurück und blickte an sich herunter. Er hob seine Hände an und glitt unter sein Nachthemd, das er als Mädchen jetzt anhatte.

\begin{rueckblick}
Er erinnerte sich, wie Dean, Seamus, Ron und Neville ihn leicht sabbernd und mit offenen Augen beobachteten, als er sich zur Nachtruhe umziehen wollte. Erst als er nur noch in Unterwäsche da stand und ihren Gesichtsausdruck bemerkte, zog er sich hinter eine spanische Wand zurück und dort auch um.

\enquote{Ach menno}, hörte er Dean sagen.

\enquote{Genau}, pflichtete ihm Seamus bei.

\enquote{Sie wird es von uns nie erfahren.}

\enquote{Aber von mir}, antwortete Harry hinter der Wand hervor.
\end{rueckblick}

Er schmunzelte jetzt und seine Hände kamen seinen Brüsten immer näher. Er umspielte sie mit seinen Fingerkuppen und wand sich in kreisenden Bewegungen um sie herum, seinen Brustwarzen nähernd. Die Berührung seiner eigenen Hände ließ ihn leise aufstöhnen. Er versuchte so ruhig wie nur möglich zu sein und brach sofort ab, als er auch nur die geringste Änderung seiner Zimmergenossen wahrnahm. Er erkundete noch eine Weile seinen neuen Körper und schlief dann gegen ein Uhr morgens ein.

\begin{traum}
Harry stand auf einer Wiese und spürte eine laue Sommerbrise. Er sah sich staunend um. Rings herum waren lauter Bäume. Er stand also auf einer Lichtung mitten im Wald, vermutete er. Dann erspähte er eine Gestalt, die langsam auf ihn zukam. Es war Luna. Er ging auf sie zu und nahm sie, als er sie erreichte, in seine Arme. Gemeinsam setzten sie sich ins Gras und unterhielten sich.
\end{traum}

Harry wachte erfrischt wieder auf und begann sich zum Joggen anzuziehen. Unten angekommen, traf er auf Ginny.

\enquote{Guten Morgen, Lu\aabs Harry}, sagte sie.

\enquote{Guten Morgen, Ginny}, sagte Harry.

Beide schlüpften durch das Porträt-Loch, hinunter zur Großen Halle und öffneten das große Tor, um draußen mit ihren Aufwärmübungen zu beginnen. Ihr Weg führte hinunter zum Quidditch-Feld und ein Stück um den großen See herum. Kurz vor ihrem üblichen Wendepunkte musste sich Harry setzen.

\enquote{Ginny, warte, mein Körper kann nicht mehr, ich brauche eine Pause.}

Ginny drehte sich um und lächelte. \enquote{Ich glaube, wir sollten Luna überreden, in Zukunft mit uns zu joggen.}

Harry schmunzelte. \enquote{Luna \gst weißt du, Ginny} und er zog sie neben sich auf den Boden, \enquote{Luna und ich haben uns auf der Krankenstation gestern getrennt. Wir sind nicht mehr länger ein Paar.}

Harry konnte in Ginnys Augen ein Leuchten sehen. \enquote{Weißt du, Ginny}, sagte Harry, als er ihr direkt in die Augen sah, \enquote{wenn ich nicht in Lunas Körper stecken würde, dann würde ich dich jetzt küssen.}

\enquote{Ist das alles?}, fragte sie mit einem schelmischen Grinsen im Gesicht. \enquote{Mich würde das nicht abhalten.}

\enquote{Ginny, hör auf, ich werde sonst noch rot.} Harry atmete tief durch, stand dann auf und meinte nur: \enquote{Joggen wir zurück.}

Auf dem Rückweg zum Schloss blickte er immer wieder zu Ginny und jedes Mal fuhr es ihm durch den Kopf: \gedanke{Mich würde das nicht abhalten.}

Dann saßen sie wieder beim Frühstück in der Großen Halle, als ihm ein Gedanke durch seinen Kopf fuhr. \gedanke{Ich habe mich in letzter Zeit gar nicht mit Luna unterhalten. Wir haben keinen gedanklichen Kontakt miteinander. Und auch das Hineinversetzen in den anderen funktioniert nicht. Warum kann ich Lunas Gedanken nicht mehr wahrnehmen?} Er drehte sich zu Luna um und bemerkte, wie sie sich \gst allein an ihrem Platz \gst mit Professor Flitwick unterhielt, der sich neben sie gesetzt hatte. Er stand gerade auf um zum Lehrertisch zu gehen, als Harry Lunas Blick auffing. Sie sagte ihm mit ihren stummen Mundbewegungen nur: \enquote{Später}.

Nachdem Luna ihm gezeigt hatte, was er während seiner Regel zu beachten habe, führte er sich einen Tampon ein und wechselte ihn über die nächsten fünf Tage präzise aus. Es war eine Sache, als Mann einen weiblichen Körper zu berühren. Aber jetzt in einem zu stecken und sich dort zu berühren, wo er sonst nur Mädchen berühren würde, war etwas vollkommen Neues für ihn. Aber daran musste er sich wohl gewöhnen.

Zu anfangs kostete es ihn etwas Überwindung, sich vor Luna auszuziehen, aber sie hatten sich schon mehrmals nackt gesehen und das fiel Harry wieder ein. Zudem befand er sich nicht einmal in seinem eigenen Körper, sondern in dem von Luna.

Es waren drei Tage nach dem mysteriösen Körperwechsel vergangen, als er an Kreacher dachte. \gedanke{Wie er wohl reagieren wird, wenn er mich so sieht? Ob er überhaupt kommt, wenn ich ihn rufe?}

Harry kam diesen Morgen, wie auch die Tage zuvor, in die Große Halle zum Frühstücken. Doch dieses Mal zog es seinen Blick nicht nur Richtung Lehrertisch, wie sonst immer, sondern direkt in die Ecke, in der Professor Snape und Professor Elber saßen. Professor Elber war in den letzten Tagen ständig mit dem Gesicht in einem Buch anzusehen, aber heute hatte er ein leuchtendes Buch, mit einem smaragdgrünen Umschlag, der von einem eigenartigen ebenso smaragdgrünen Leuchten umgeben war. Das Buch schwebte leicht schräg in der Luft. Professor Elber aß gemütlich an einem Müsli. Zumindest dachte Harry es sei ein Müsli, da er aus einer für Müsli üblichen Schale löffelte.

Professor Elber ließ seinen Blick durch die Halle schweifen und entdeckte Harry, er ließ ihn weiter gleiten und blieb auch bei Luna hängen. Immer noch sein Frühstück kauend schaute er wieder zurück auf sein Buch und ließ sich nur durch das Löffeln des Frühstücks abbringen. Harry schaute immer wieder zum Lehrertisch, als plötzlich Professor Elber verschwunden war. Erschrocken drehte er sich um, als er eine Hand auf seiner Schulter spürte. \enquote{Kommen Sie bitte nach dem Frühstück in mein Büro? Luna weiß schon Bescheid. Ich möchte mich mit Ihnen über Ihr Problem unterhalten.} Harry nickte kurz und kaute an seinem Frühstück weiter. Professor Elber schaute weiter durch die Halle und ging zu einem der Hausgeister hin. Harry konnte seine Unterhaltung mit der grauen Lady von Ravenclaw nicht hören. Er drehte sich wieder um und frühstückte weiter.

\enquote{Was wollte Professor Elber von dir?}, fragte ihn Ron.

\enquote{Ich bin mir nicht sicher, aber ich denke, es hat etwas mit meinem Zustand zu tun}, antwortete Harry.

Nachdem Harry zu Ende gefrühstückt hatte, stand er auf und ging zu Professor Elbers Büro. Unterwegs traf er Luna und ging den Rest gemeinsam mit ihr. Er klopfte an die offene Tür, um sich bemerkbar zu machen. Professor Elber winkte die beiden herein und deutete ihnen an, sich zu setzen. Die graue Dame, Ravenclaws Hausgeist, schwebte bereits im Raum und Professor Elber zauberte mit seinem Zauberstab einen Geisterstuhl herbei. Harry und Luna schauten beeindruckt zu. \enquote{Nur ein wenig schwarze Magie}, sagte Professor Elber mit einem Augenzwinkern. \enquote{Aber erzählt es nicht weiter.}

Harry fiel wieder das smaragdgrün leuchtende Buch ins Auge. \enquote{Was ist das für ein Buch, Professor?}, fragte Harry.

\enquote{Das}, so antwortete Professor Elber, \enquote{ist das grüne Index-Buch der Magie. In ihm ist praktisch jeder Zauberspruch aufgeführt mit einer kurzen Beschreibung. Ich habe es mir geholt, nachdem mir nichts mehr eingefallen ist. Ich habe es per Fernleihe aus einer Bibliothek.}

\enquote{So etwas habe ich noch nie gesehen}, antwortete Luna.

\enquote{Das glaube ich gerne, es enthält, wie ich schon sagte, \accentuate{sämtliche} Sprüche und Hexereien. Auch die, die man allgemein als schwarz-magisch bezeichnet.}

\enquote{Oh, das erklärt es natürlich}, meinte Luna.

Professor Elber wandte sich nun zur grauen Lady und fing an, in einer kurzen Zusammenfassung, sie über den Zustand von Harry und Luna zu unterrichten. \enquote{Weshalb ich dich jetzt brauche, Helena, ist, du sollst mir etwas feststellen. Etwas, wozu Geister in der Lage sind, ohne sich einem Zauber zu unterziehen, oder einen Zaubertrank zu sich zu nehmen. Es ist nichts Gefährliches; vielleicht ungewohnt.} Professor Elber ließ seinen Blick über die drei Gestalten in seinem Büro gleiten.

\enquote{Haben Sie schon eine Lösung}, fragte jetzt Harry, dem die Situation zusehends unangenehm wurde.

Professor Elber schaute Harry jetzt an und meinte: \enquote{Ich habe eine Idee. Mir geht es erst einmal darum, festzustellen, ob eure Seelen noch in euren Körpern sind.}

Harry und Luna machten große Augen. \enquote{Wie genau darf ich das verstehen?}, fragte die graue Lady.

\enquote{Nun, wie ich eben schon gesagt habe, scheinen die beiden ihre Körper, bzw. ihre Geister getauscht zu haben. Was ich jetzt feststellen möchte ist, ob eure Seelen}, Professor Elber schaute jetzt zu Harry und Luna, \enquote{intakt sind und sich in eurem eigenen Körper befinden. In diesem Falle werden scheinbar nur die externen Sinnesreize dem anderen Körper übermittelt.}

\enquote{Und was habe ich damit zu tun?}, fragte die graue Lady.

Professor Elber schaute sie wieder an und meinte: \enquote{Geister sind in der Lage, geöffnete Seelen der Menschen, auch wenn sie nur leicht geöffnet sind, zu erkennen und den Verlauf, bzw. die Verbindung der Seele zum Körper zu erkennen.}

\enquote{Aber ich kann nichts erkennen}, warf die graue Lady ein.

\enquote{Ich sprach auch von geöffneten Seelen}, machte Professor Elber weiter.

\enquote{Und wie öffnet man eine Seele?}, fragte Luna.

Professor Elbers Blick schwenkte nun zu Luna. \enquote{Das wird mit einem komplizierten Trank gemacht. Ich hoffe, Severus braut ihn. Wir begeben uns damit in einen sehr schwarzen Bereich der Magie.}

Harry und Luna schluckten.

\enquote{Mehr kann ich euch jetzt noch nicht sagen, ich muss erst in der Bibliothek ein Buch holen, um dort weiterzulesen}, sagte Professor Elber. \enquote{Ich hoffe schwer, ich irre mich.}

Luna und Harry schauten sich an und meinten dann unisono: \enquote{Dürfen wir mitkommen?} Professor Elber hob beide Augenbrauen. \enquote{Bitte!}, bettelten die beiden.

\enquote{Das würde mich auch interessieren}, meinte die graue Lady.

Professor Elber senkte seine Augenbrauen wieder und meinte dann: \enquote{Ok. Von mir aus. Nichts leichter als das, kommt mit.} Er stand auf und verließ mit den dreien sein Büro, durchquerte das Klassenzimmer für \accentuate{Verteidigung gegen die dunklen Künste} und machte sich auf den Weg zur Bibliothek.

Viele Schüler blieben mit einem Staunen stehen, als sie das merkwürdige Quartett zur Bibliothek laufen, bzw. schweben sahen. Einer lief sogar vor lauter Staunen gegen eine Steinsäule und rieb sich danach im Weiterlaufen die Nase.

In der Bibliothek angekommen sahen sie Dumbledore, der sich gerade mit Madame Pince unterhielt und nur kurz grüßte. Professor Elber verschwand mit seinem Trio im Anhang um ein paar Ecken und blieb dann vor einer Holzpaneel mit der Aufschrift \accentuate{Tränke} auf einem Holztäfelchen und darunter auf einem weiteren Täfelchen \accentuate{R bis T} stehen. Professor Elber stand vor einer Zwischenwand in der Bibliothek. Links davon ging ein Gang zum Fenster und auch von rechts ging ein Gang zum Fenster. Die vier standen in einem der vielen Hauptgänge der verwinkelten Bibliothek von Hogwarts. Professor Elber drückte auf das kleine Holztäfelchen auf dem \accentuate{Tränke} stand. Die Schrift auf der kleinen Tafel darunter begann zu leuchten. Professor Elber drückte auf die beiden hervorstehenden Buchstaben R und T und es erschien auf der Tafel nun ein \accentuate{S}.

Jetzt drückte er auf den großen, runden hervorstehenden Knopf und die Zwischenwand begann sich zu teilen und gab eine weitere Reihe frei. Harry erinnerte sich wieder an seine Besuche in London am Grimmauldplatz, als sich das Haus seines Paten aus dem Nichts den Weg bahnte und zu wachsen schien. Professor Elber nahm Luna bei der Hand und sagte zu Harry: \enquote{Halten sie Luna bei der Hand. Und du}, jetzt zu Helena gewandt, \enquote{hältst deine Hand bitte in irgendjemanden und verliere uns nicht, sonst kommst du nicht rein. Es wird nur der, der die Passage geöffnet hat eingelassen; es sei denn, wir sind miteinander verbunden.} Er machte jetzt einen vorsichtigen Schritt und zog Luna mit sich. Die graue Lady fuhr mit ihrer Hand durch Harrys linke Schulter, wobei ihn ein leichter Schauer durchfuhr. Im Gang angekommen, ließ Professor Elber Luna wieder los und auch Harry ließ Lunas Hand los, worauf die graue Lady ihre Hand aus Harrys Schulter nahm. Harry zuckte abermals.

Professor Elber lief langsam durch den Gang, seinen Blick auf die Buchrücken gerichtet. Endlich entdeckte er ein Buch mit dem Titel: \buchtitel{Die Seele des Menschen und ihre Besonderheiten innerhalb der Magie}. Er zog das Buch heraus, setzte sich auf einen Stuhl, und legte es auf einen kleinen Tisch an der Außenwand. Er schlug es auf und Harry und Luna schauten ihm über die Schulter. Die graue Lady schwebte über ihnen. Professor Elber blätterte durch das Buch, bis er die passende Seite gefunden hatte.

\enquote{Ah, hier haben wir es. \enquote{Sichtbarmachen der menschlichen Seele}. Au weh, das ist ein komplizierter Trank. Er braucht mindestens zwei Wochen Vorbereitungs- und acht Stunden Brauzeit. Das wird nicht einfach.} Professor Elber las weiter und blätterte eine Seite um.

\enquote{Ah ja, Besonderheiten. \enquote{Eine menschliche Seele kann man nur in geöffnetem Zustand sehen. Der Grad der Öffnung ist dabei nicht entscheidend. Die Seele schließt sich von selbst wieder. Die Rekonvaleszenz\footnote{Genesung} hängt vom Grad der Öffnung ab. Menschliche Zauberer können eine Seele nur sehen, wenn sie vorher einen \accentuate{Sourlicitus} Trank eingenommen haben. Von der Stärke des Trankes hängt die Dauer der Fähigkeit ab, wie lange man eine Seele sehen kann. Der stärkste bekannte Trank ermöglichte eine Zeitdauer von einem Monat.}}

\enquote{Ist ja interessant.}

\enquote{Geister können aufgrund ihres besonderen Aufenthaltsortes zwischen dem Leben und dem Danach eine geöffnete Seele ohne besondere Tränke sehen. \gst Das ist der Punkt wozu ich dich brauche}, sagte Professor Elber zu Helena. \enquote{Du sollst mir dann sagen, was du siehst.} Professor Elber schaute wieder in sein Buch und las weiter. \enquote{Au weh, ich habe mir schon irgendwas in der Art gedacht. Hier steht: \enquote{Sollte ein Zauberer oder ein Geist die Position der Seele in einem Körper feststellen können, so kann er damit wunderbare oder schreckliche Sachen anstellen. Als Beispiel sei anzumerken, dass zu den schrecklichsten Dingen eine vervielfachte Ausführung des Crutiatus-Fluchs gehört.}}

Harry musste schlucken.

Professor Elber schaute wieder hoch und meinte dann: \enquote{Darum sprechen wir hier von schwarzer Magie.} Er schaute noch einmal in sein Buch und dann abermals hoch und abwechselnd Harry und Luna ins Gesicht. \enquote{Mit einer geöffneten Seele kann ein Geist oder ein dunkler Zauberer schreckliche Dinge anstellen. Die Qualen, die man einer Seele zufügen kann, übertreffen die Wirkung des Cruciatus-Fluchs bei weitem. Ich glaube nicht, dass dem Autor hier die Ausmaße klar sind. Wenn man Zugriff auf die Seele eines Menschen hat, hat man Kontrolle über ihn.} Er wandte sich wieder zur grauen Lady. \enquote{Ich hoffe, du weißt, welche Verantwortung dir und den anderen Hausgeistern hier für kurze Zeit auferlegt wird. Sie sollten es vermeiden, oder besser gesagt sie müssen es unter allen Umständen vermeiden, die beiden zu berühren. Das könnte den beiden mehr als schaden.}

Er schaute an Harry vorbei und meinte nur lapidar: \enquote{Wir werden gesucht.}

Harry, Luna und die graue Lady drehten sich um. Direkt vor dem Eingang stand Professor Dumbledore und schaute sich ratlos um. Professor Elber stand auf und drängte sich durch Harry und Luna, schritt mit einem Fuß hinaus und lehnte sich vor, tippte Dumbledore kurz am Arm an, worauf dieser sich umdrehte, schnappte sich dann den anderen und zog ihn in den Gang rein.

Professor Dumbledore war erstaunt und entzückt zugleich. \enquote{Ich hätte nicht gedacht, was sich so alles in Hogwarts versteckt. Da bin ich schon so lange hier und lerne immer noch dazu. Was macht ihr hier?}

\enquote{Wir versuchen dem Problem von Harry und Luna nachzugehen}, sagte Professor Elber. \enquote{Mir ist zwar so einigermaßen klar, woran es liegen könnte, dass sie in dieser Situation sind, aber ich weiß noch absolut nicht, was man dagegen tun kann. Ich bin noch am Ergründen. \gst Zumindest habe ich eine Befürchtung.}

Professor Elber drehte sich wieder zu Harry und Luna um und wollte gerade weitermachen, doch Luna unterbrach ihn.

\enquote{Professor? Was ist, wenn unsere Seelen wirklich vertauscht sind?}

Das erregte Professor Dumbledores Aufmerksamkeit und er lauschte interessiert.

\enquote{Tja, dann könnten wir eine Seelentrennung durchführen. Da sehe ich kein Problem. Das traue ich mir durchaus zu. Was aber viel komplizierter ist, ist die anschließende Seelenreintegration. Das ist dermaßen schwarze Magie, noch schwärzer als die Seelenöffnung. Ich weiß nicht, ob ich dazu die nötige Kraft habe. Ein Fehler und ihr seit gefangen in einer Zwischenwelt zwischen Leben und Tod.}

\enquote{So etwas, wie Voldemort die letzten Jahre war?}, fragte Harry.

\enquote{So ähnlich, oder schlimmer}, gab Professor Elber zur Antwort.

\enquote{Was wollen Sie machen?}, fragte Dumbledore Elber.

\enquote{Ich möchte bei beiden erst mal feststellen, ob die beiden Seelen noch in ihren Körpern sitzen und nur die externen Reize an den anderen Körper geleitet werden. Das wäre das einfachste. Deswegen habe ich den beiden die gefährliche Prozedur der zeitlich begrenzten Seelenöffnung vorgeschlagen, damit die graue Lady hier sehen kann, ob beide noch in Ordnung sind.}

Professor Dumbledore machte ein Gesicht, mit dem er sagen wollte, er sei damit überhaupt nicht einverstanden. \enquote{Gibt es keine Alternativen?}

\enquote{Ich lasse mich gerne davon abbringen, wenn Ihnen was Besseres einfällt Albus. So ganz begeistert bin ich davon auch nicht. Ich suche noch nach Alternativen.}

Dumbledore machte ein betretenes Gesicht und wandte sich schließlich den Bücherregalen zu. \enquote{Was sind das alles für Bücher hier? Lauter dunkle Künste?}

\enquote{Oh nein}, gab Professor Elber zur Antwort, \enquote{das ist ein breites Spektrum an verschiedenen Bereichen. Wir sind hier im Bereich Tränke, in den anderen Abteilungen gibt es zu jedem Fachgebiet eigene Gänge. Die Palette hier umfasst, wenn man so sagen will, das gesamte magische Spektrum.}

Professor Elber wandte sich wieder zu Harry und Luna. \enquote{Ihr solltet es euch durch den Kopf gehen lassen, ob ihr diesen Trank zu euch nehmen wollt. Eigentlich müsste ihn nur einer einnehmen, aber ich bin mir nicht sicher, was die graue Lady (und die anderen Geister) schließlich sehen werden. Ich habe so was noch nie gemacht: Eine Seelenuntersuchung bei zwei Personen, die auf diese Weise miteinander verbunden sind. Ich werde das Rezept mit Severus durchsprechen. Falls ihr euch entschließen solltet, sagt ihm bitte Bescheid, damit er den Trank vorbereiten und brauen kann, er wird sich dann bei mir melden.} Er lief zurück zum Tisch, schlug das Buch zu und marschierte Richtung Ausgang. Kurz vorher drehte er sich noch einmal um und meinte dann: \enquote{Geht einfach raus, wenn ihr fertig seid. Von außen sieht alles ganz normal aus. Die merken nicht einmal, dass wir hier drin sind.} Er drehte sich wieder Richtung Gang und verschwand kurz darauf um die Ecke. Man konnte ihm ansehen, dass er ein flaues Gefühl im Magen hatte.

\trenn

Harry lag nackt im Bett und ließ es sich gut gehen. So langsam sickerte in seinen Verstand der Gedanke, dass er träumte.

\enquote{Ist das schön}, meinte Harry mit geschlossenen Augen.

\enquote{Hmm?}, hörte er plötzlich.

Dann verließ Harry die Entspannung. Er merkte, dass die Stimme nicht Ginny gehörte. Diese Stimme hörte sich überhaupt nicht weiblich an. Diese Stimme hörte sich aber vertraut an. Er öffnete die Augen und sah, nachdem er seinen Kopf zur Seite gedreht hatte, in zwei graue Augen, die von einem schmalen Gesicht und blonden, fast schon weißen, glatten Haaren umrandet waren.

\enquote{Draco}, keuchte er. Doch er empfand keine Scham. Auch empfand er keinen Zorn, oder Wut.

\enquote{Harry}, keuchte auch Draco. Doch auch er zog sich nicht zurück.

Harry schwang seine Beine aus dem Bett und Draco setzte sich neben ihn. Es herrschte lange Zeit eine Stille.

\enquote{Tamara hat mir erzählt\abs}, begann Harry und pausierte kurz.

\enquote{Was hat sie dir erzählt?}, fragte Draco nach und sah zu Harry.

\enquote{Sie hat mir erzählt, dass du\abs na ja\abs neidisch auf mich bist\abs warst}, verbesserte er sich.

Draco sah ihn erstaunt an. \enquote{Du meinst unser erstes Treffen in Hogwarts?} Harry nickte. \enquote{Erzähl weiter}, forderte Draco.

\enquote{Tamara erzählte mir, dass ich Freunde habe und dein Angebot ausschlug. Das verursachte nicht nur den Hass, sondern auch den Neid auf mich.}

\enquote{Ich habe auch Freunde}, begehrte Draco auf.

\enquote{Wohl eher Bodyguards. Mal ehrlich Draco. Mit Crabbe und Goyle kann man sich nicht richtig unterhalten und andere Personen habe ich in deiner Nähe nie gesehen.} Draco schwieg. \enquote{Weißt du, ich weiß nicht, ob es dir Tamara erzählt hat, aber vor meinem elften Geburtstag wusste ich nicht, dass ich ein Zauberer bin.} Draco schüttelte den Kopf und sah Harry ungläubig an. Dieser senkte den Kopf, nahm die Hände zwischen seine Beine und erzählte. \enquote{Ich wuchs bei Onkel und Tante und meinem Cousin Dudley auf. Ich wurde mehr oder weniger nur geduldet. Ich erwarte deswegen kein Mitleid von dir. Es ist halt so. Dann traf ich im Zug auf Ron. Er erzählte mir ein wenig von Hogwarts; dass es vier Häuser gibt und die meisten Schwarzmagier aus Slytherin kommen. Und dann, zum denkbar ungünstigsten Zeitpunkt, stehst du vor mir, mit der Gewissheit in der Stimme, dass du nach Slytherin kommst, und bietest mir deine Freundschaft an. Du kannst dir natürlich vorstellen, dass ich nicht gerade begeistert von dir war. Dann, als ich auf dem Stuhl saß,\aabs} Draco grinste, da er diesen Moment klar vor sich sah, \enquote{\aabs sah der Hut in meinen Kopf, sah Mut und den Drang mich beweisen zu wollen.} Dann blickte er Draco in die Augen. \enquote{Der Hut wollte mich nach Slytherin schicken.} Draco riss die Augen auf. \enquote{Doch ich wollte nicht. Mir war es egal, wo ich hinkommen würde. Mir war das Haus egal. Ich wollte nur nicht nach Slytherin. \gst Also schickte mich der Hut nach Gryffindor.}

Draco sah ihn lange an. Dann sagte er: \enquote{Das war es also. Und ich dachte: Der berühmte Potter wolle sich nicht mit mir abgeben.} Dann schwieg er wieder.

\enquote{Sag mal. Fühlst du irgendwas?}, fragte Harry.

\enquote{Was meinst du?}

\enquote{Hass, Zorn, Wut, Neid.}

\enquote{Auf dich? Nein.}

\enquote{Du weißt, dass wir träumen?}

\enquote{Ja.}

\enquote{Scheint so, als ob wir uns wenigstens in unseren Träumen unterhalten können.}

Draco schluckte. \enquote{Mum ist gestern aus dem Manor ausgezogen. Sie ist bei F\aabs bei Tamaras Paten.}

Vereinzelte Erinnerungen drangen in Harrys Gedanken. Er erinnerte sich an seinen Traum, nein seine Vision, korrigierte er sich gedanklich. Er sah, wie Draco und Tamara mit Professor Elber mitgingen und Narcissa das Angebot gemacht wurde, nachzukommen.

\enquote{Sie ist bei Professor Elber}, präzisierte Harry.

\enquote{Woher?}, fragte Draco erstaunt nach.

\enquote{Es war eine Art Vision. Ich war da, als Tamara von Bellatrix\abs} Harry konnte nicht weiter reden. \enquote{Kurz danach jedenfalls. Wie krank ist deine Tante eigentlich? \gst ’Tschuldige, eine elfjährige mit dem Cruciatus-Fluch zu foltern.}

Draco schwieg; legte seine Hände mit den Handinnenseiten nach oben auf seinen zusammengepressten Beinen ab.

Harrys Blick fiel auf die leere Stelle an Dracos linkem Unterarm. \enquote{Ist es nur in unserem\abs ist es nur in der gedanklichen Welt weg, oder auch in der realen?}

Dracos Blick folgte dem von Harry und er sah ebenfalls auf seinen linken Unterarm. \enquote{Es ist auch real weg.}

\enquote{Tamaras Pate?}, fragte Harry nach.

\enquote{Gibt es auch etwas, was du nicht weißt?}

\enquote{Hab’ es durch Zufall erfahren als ich am Raum vorbeilief, in dem ihr wart.}

Draco schüttelte nur den Kopf und fragte dann beiläufig: \enquote{Wie war es, Granger zu küssen? War es angenehm?}

\enquote{Woher?}

\enquote{Hab’ es gehört, als sich jemand aus deinem Haus darüber unterhalten hat}, grinste er. Dann wechselte er das Thema. \enquote{Wieso haben wir wieder diesen Traum? Hast du dein Amulett wieder nicht an? \gst Nein, dann wäre es ein Alptraum.}

Harry dachte kurz nach. \enquote{Ich habe vor dem Einschlafen mein Amulett umfasst. Ich bin mir nicht sicher, ob ich auch an dich gedacht habe. \gst Und ja, es war angenehm, Hermine zu küssen.}

\enquote{Schmeichelhaft.} Draco verzog sein Gesicht. \enquote{Beherrsche dich das nächste Mal, wenn wir wieder einen gemeinsamen Traum haben werden. \gst Heißt das, du kannst jederzeit so einen Traum herbeirufen?} Harry zog die Schultern nach oben und ließ sie wieder fallen. \enquote{Und was, wenn ich mit dir zu reden habe?}

\enquote{Gib mir ein Zeichen. Sag etwas, was du sonst nicht zu mir sagen würdest.}

\enquote{Und was?}

\enquote{Kannst du französisch?}

\enquote{Ja.}

\enquote{Merde moutons.}

\enquote{Willst du mich beleidigen?}

\enquote{Nein Draco, das sollst du zu mir sagen.}

\enquote{Und Granger?}

\enquote{Hermine wird natürlich nachschlagen und es als Schimpfwort erkennen, also völlig gefahrlos.} Harry grinste.

Draco nickte und stand auf. Dann drehte er sich zu Harry und fragte ihn: \enquote{Und wie kommen wir jetzt hier raus?}

Harry sah gerade aus und Draco genau auf den Schritt. Dann hob er seinen Kopf und sagte: \enquote{Denk einfach, dass du aufwachst.}

Draco nickte und schloss seine Augen. \enquote{Mum ist gestern aus dem Manor ausgezogen.} Zwischen seinen Augen bildeten sich runzeln. Dann verblasste er und verschwand.

Harry spürte, wie auch er weggezogen wurde. Sein Geist leerte sich und er schlief, ohne seine weiteren Träume bewusst zu erleben, ein.




\begin{kommentar}
Harry und Luna haben ihre Körper getauscht und liegen nun auf der Krankenstation. Plötzlich steht Luna auf und möchte auf die Toilette. Sie bittet Harry um Hilfe, da sie weder das Pinkeln im Stehen kennt, noch genau weiß, wie man den Penis abschüttelt. Die Idee kam mir, nachdem ich  mich gefragt hatte, was man denn so im Körper des anderen alles machen kann oder auch muss.
\end{kommentar}

\begin{kommentar}
Kurz bevor Elber dann die Krankenstation verlässt, fragt ihn Harry über Draco aus (dem Elber ja das Dunkle Mal entfernt hat). Elber gibt ihm eine Antwort und geht dann. So, als wäre es ganz normal, dass Harry dabei gewesen war.
\end{kommentar}

\begin{kommentar}
Was mir jetzt erst beim Schreiben der Kommentare bewusst wird ist, dass Harry ja nun eine Ravenclaw-Uniform anziehen muss und Luna die von Gryffindor.
\end{kommentar}

\begin{kommentar}
Als Harry und Luna mit der grauen Dame bei Elber im Büro sitzen und Elber in der Bibliothek nachschauen will, wollen sie unbedingt mitkommen. Frederick sagt daraufhin: »Nichts leichter als das, kommt mit.« Ein Satz, der älteren Lesern bekannt vorkommen dürfte, wenn er sich an Pickeldi und Frederick noch erinnert. Dort fragte das kleine Schweinchen immer nach etwas und das großte sagte dann genau diesen Satz.
\end{kommentar}

\begin{kommentar}
Am Ende des Kapitels unterhalten sich Harry und Draco in einem Traum ganz normal. Es ist einer der Träume, in denen sie normal reden können. Der erste war allerdings nicht sehr angenehm, da sich beide gegenseitig liebkosten. Als ich nach Alpträumen suchte, die einen ersten Schritt der Kommunikation darstellen sollten, kam mir die Idee, dass im Traum es anders sein musste. Also ließ ich die beiden Gefühle füreinander haben. Diese Idee kam mir, nachdem ich eine einzige Slash-Geschichte, einen One-Shot, gelesen hatte.
\end{kommentar}

\chapter{Drachenstein}


Harry dachte über das gerade eben gelesene Kapitel im Buch über Dementoren nach. Sie bedürfen keiner Pflege. Sie gedeihen da, wo wenig glückliche Gedanken sind. Also in Askaban. Allerdings konnte man sie in jungen Jahren unter vielen Mühen auf einen prägen. Sie folgten einem dann. Doch das interessanteste war, dass sich Dementoren auch von schlechten Erinnerungen nähren konnten. Sie waren zwar nicht so effektiv wie die guten, aber sie sättigten doch.

Er erzählte es gerade Ron und Hermine, als er unterbrochen wurde. Ein Fauchen und ein Zittern durchfuhr das Schloss. Alle, die noch beim Essen saßen, sprangen auf und wollten bereits nachsehen, oder fluchtartig in ihre Gemeinschaftsräume rennen.

\enquote{Ruhe}, rief Dumbledore, stand auf und trat durch die Menge hindurch.

Die Schüler und Lehrer folgten ihm. Einige waren bereits an der Eingangstüre angelangt und sahen den Quell der Störung. Viele Schüler gingen ängstlich zurück oder drückten sich an die Wand. So auch Harry, der gerade noch etwas frische Luft schnappen wollte, als das Tier landete. Es war ein grüner Drache. Smaragd-Grün. Dann wartete er, bis sich die Schüler beruhigt hatten und nicht mehr so ängstlich waren.

Dumbledore kam durch die Tür und blieb stehen. Skeptisch sah er zu dem Drachen, der da stand. Es war ein kleiner Drachen. Etwa drei Meter maß er mit Schwanz und sein Stockmaß betrug gerade mal einen Meter fünfzig. Wenn er seinen Kopf ganz streckte, dann hatte er zwei Meter zwanzig. Langsam besah sich der Drache die Reihen der Schüler. Unsicher und vorsichtig zog Dumbledore seinen Zauberstab, was der Drache sofort registrierte und leicht zu fauchen begann. Er fixierte Dumbledore und wartete ab. Harry sah sich ebenfalls um. Einer stand mit einer Seelenruhe da, als würde ihn das gar nicht interessieren, und aß mit einem Löffel aus einer Müslischale. Auch das registrierte der Drache. Harry sah noch einmal auf den Drachen und entdeckte, dass sein Hals silbern schimmerte.

Ein Bild formte sich in seinem Kopf und ohne es bewusst zu wollen, sagte er leise: \enquote{Tabaluga} und sah dabei den Drachen an.

Dieser blickte zurück. \stimme{Du kennst also meinen Namen. Sehr gut. Dann kannst du auch meine Botschaft vermitteln.}

\enquote{Du hast eine Botschaft?}, fragte Harry.

Dumbledore hörte ihn und wendete seinen Blick nun Harry zu, den Drachen aber immer noch aus dem Augenwinkel heraus beobachtend.

\stimme{Denke mit mir. Sprich nicht. Hör zu und erzähle es den anderen.} Harry nickte. \stimme{Ich komme aus Rumänien. Charlie Weasley schickt mich. Ich soll hier Hilfe ersuchen. Es geht eine Seuche um, die uns Drachen befällt. In unserem Reservat sind alle erkrankt außer mir. Keiner weiß warum. Deshalb hat man mich los geschickt.}

Harry nickte und erzählte Dumbledore, was ihm der Drache erzählt hat. Dumbledore ließ seinen Zauberstab sinken, hielt ihn aber immer noch in der Hand.

\enquote{Eine Seuche unter den Drachen?}, sagte Dumbledore und dachte nach.

Währenddessen fühlte sich Harry merkwürdig zu dem Drachen hingezogen. Er verließ das Schloss und ging in den Außenbereich und auf den Drachen zu. Er blieb vor ihm stehen. Nach einigen Schritten blieb auch Draco Malfoy neben ihm stehen, der dem Ruf des Drachen ebenfalls zu folgen schien.

\stimme{Ich spüre eine eigenartige Verbindung zwischen euch. Zuerst Hass und Neid, Misstrauen und Ablehnung. Doch dies scheint sich gewandelt zu haben. Ich spüre den Keim einer Freundschaft, der nur langsam wächst. Ihr feindet euch nicht mehr an. Das ist gut.}

\enquote{Du kannst so etwas spüren?}, fragte Draco.

\stimme{Ja, wir sind Wesen der Magie, empfänglich für ihre Schwingungen. Aber auch für dich gilt, denke mit mir, nicht, rede mit mir.}

Harry lächelte kurz. Hatte ihn der Drache erst selbst kurz zuvor ermahnt.

\enquote{Mir ist keine derartige Seuche bekannt}, sagte Dumbledore. Der sah den Drachen nun wieder an und bemerkte erst jetzt, dass er Harry und Draco neben ihm stehen sah. \enquote{Was macht ihr denn neben dem Drachen?}, fragte er nach.

\enquote{Wir unterhalten uns mit ihm.}

\enquote{Aber ihr habt doch kein Wort mit ihm gewechselt.}

\enquote{Es gibt auch noch andere Arten der Kommunikation}, sagte Draco Malfoy.

Harry nickte nur.

Dann sah der Drache Elber an. \stimme{Du kennst die Ursache?}

Dieser nickte nur. \stimme{Ich bin mir aber sicher, dass dies nicht der Auslöser ist. Es müsste schon jemand in das Heiligtum eingebrochen sein, die Fallen und Hindernisse überwunden haben und das Teil befreit haben. Dann muss er oder sie es unbemerkt nach Rumänien geschafft und unter euch ausgesetzt haben. Das hätte mächtige Spuren hinterlassen.}

Der Drache sah ihn skeptisch an. Elber legte seinen Löffel in seine Müslischale, hob sie in die Luft und zog seinen Arm zurück, als ob er sie nur auf einer hohen Kante abgestellt hatte. Die Schale blieb dort stehen. Dann kam er auf die drei zu.

\gedanke{Ich werde das aber trotzdem prüfen. Wenn du mich begleiten möchtest?}

\stimme{Sehr gerne sogar.}

\enquote{Ich bin weg, Albus. Ich begleite den Drachen auf eine kleine Expedition.}

Dumbledore, der die letzten Sätze des Drachen und seines Lehrerkollegen mitbekommen hatte, sowie die ganze Schule, nickte nur und sah nachdenklich drein.

\enquote{Passt auf euch auf, denn aufhalten werde ich dich wohl nicht können?}

\enquote{Nein}, gab dieser zurück.

\stimme{Apparieren wir zusammen?}, fragte er den Drachen. Dieser nickte und ließ sich berühren. Dann waren sie auch schon verschwunden.

\enquote{Und nun?}, fragte Harry.

\enquote{Ich gehe ins Bett}, sagte Draco und verabschiedete sich Wortlos.

Harry tat es ihm gleich und ging ebenfalls ins Schlossinnere. Die Menschenmenge löste sich auf und zurück, blieb ein einsamer Dumbledore, der seinen Zauberstab einsteckte und in die Dämmerung sah.

\trenn

Der Drache tauchte mit seinem menschlichen Begleiter auf und beide orientierten sich erst einmal. Sie waren an einer Steilküste auf einem Vorsprung von gerade einmal vier Metern aufgetaucht. Hinter ihnen das Meer, welches zwanzig Meter unter ihnen war, und vor ihnen eine aufgebrochene große Steintüre, die an manchen Stellen immer noch so aussah, als seien nur deren Konturen in den Stein gehauen worden.

Beide sahen in den Gang dahinter und Elber zog seinen Zauberstab. Langsam ging er Schritt um Schritt auf den Eingang zu. Wachsam bei jedem Schritt. Der Drache dicht neben ihm. Plötzlich trat der Drache vor ihn und schirmte ihn gegen einen ankommenden Zauber ab.

Erschrocken und auch dankbar sah er zu seinem Begleiter. \enquote{Danke.} Er schloss kurz die Augen und schüttelte seinen Kopf. \gedanke{Danke.}

\stimme{Gerne geschehen}, sagte der Drache. \stimme{Ich nehme dich auf den Rücken, damit dir nichts passiert.}

Elber nickte und stieg vorsichtig auf den Rücken des Drachens. Dieser wollte gerade losgehen, als er in der Anfangsbewegung innehielt und ihn ansprach.

\stimme{Du musst noch viel lernen, obwohl ich eine Menge Magie in dir spüre. Eine ungewöhnliche Menge an Magie.} Der Drache konzentrierte sich und sagte dann: \stimme{Ich spüre nicht nur Gutes in dir. Auch das Böse hat Besitz von dir ergriffen.}

\gedanke{Das ist wahr. Ich war nicht immer so ausgeglichen und um das Gleichgewicht bemüht. Es gab Zeiten in meinem Leben, da war ich ganz anders. Da habe ich schreckliche Dinge getan.}

Der Drache nickte zum Zeichen, dass er verstanden hatte, und schritt voran in den dunklen Gang. Elber warf ein paar Lichtkugeln aus seinem Zauberstab den Gang entlang, um ihn zu erleuchten. Nach einigen Abzweigungen, durch die er den Drachen führte, und mehreren hundert Metern an Gängen, ließ er ihn anhalten.

\gedanke{Stopp. Ich spüre etwas.}

\stimme{Was? Ich habe dieses Gefühl nicht.}

\gedanke{Etwas Seltsames wartet auf uns in diesem Gang.}

\stimme{Ich spüre immer noch nichts}, sagte der Drache, nachdem er sich konzentriert den Gang angesehen hatte.

\gedanke{Ihr könnt doch über Fähigkeiten anderer verfügen.}

\stimme{Ja.}

\gedanke{Dann benutze meine mit}, sagte Elber und öffnete seinen Geist speziell für den Drachen, damit dieser seine Fähigkeiten erweitern konnte.

Den Drachen schüttelte es kurz, als er mit seinem \accentuate{Reiter} verbunden war.

\stimme{Jetzt verstehe ich, was du meintest. Und die Gefahr spüre ich auch. Dank deiner zusätzlichen Fähigkeiten, können wir gefahrlos durchgehen. Du musst nur Körperkontakt mit mir halten. Ein einfaches Aufsitzen mit Kleidung dazwischen funktioniert nicht.}

Elber klammerte sich an den Schuppen des Drachen fest, der jetzt durch den Gang schritt. Die aufkommende Kälte machte ihm nichts aus. Wie ein Leichentuch versuchte etwas die beiden Wesen zu umhüllen, doch es konnte keinen Angriffspunkt erkennen.

Schließlich hatten sie den Gang hinter sich gelassen und sie kamen in eine zentrale Kammer. Dort schien alles normal zu sein. Auf einem Podest in der Mitte stand ein kleines Holzkästchen. Es war verschlossen, stellte Elber fest, nachdem er abgestiegen war und versucht hatte den Deckel zu heben. Gedanklich war er noch immer mit dem Drachen verbunden.

\gedanke{Weißt du}, sagte er, als er verschiedene Zauber auf das Kästchen warf um es zu öffnen. \gedanke{Ich wünschte, ich könnte mein Wissen und meine Erfahrung weitergeben. Ich bin dabei einen jungen Zauberer zu unterweisen. Er versteht so langsam die Zusammenhänge der Magie. Viel Zeit bleibt mir aber nicht mehr.}

\stimme{Dann gib ihm dein Wissen einfach weiter.}

\gedanke{Das sagte ich doch bereits, dass ich ihn unterrichte.}

\stimme{Das meinte ich nicht}, sagte der Drache. Er nahm seine Pranke hoch und hielt eine Zehe an die Stirn seines Begleiters.

Dieser zuckte kurz zusammen und sah danach entgeistert Tabaluga an. \gedanke{Ich hatte keine Ahnung, dass das überhaupt funktioniert. Und dazu noch so einfach.}

\stimme{Du musst noch viel lernen}, sagte der Drache und legte seine Zehe wieder an die Stirn.

Nach etwa zehn Minuten war er fertig. Er nahm seine Pranke zurück und sah ihn an.

\gedanke{Danke}, sagte Elber, den Drachen ehrfürchtig anblickend.

\stimme{Auch ich habe zu danken. Ich habe durch die Verbindung zu dir viel erfahren. Wir beide haben dadurch profitiert. Ich weiß jetzt auch, warum ich verschont blieb.}

Elber nickte. \gedanke{Basiliskengift?}, fragte er nach.

\stimme{Richtig. Meine Mutter war mit mir schwanger, erzählte man mir. Sie hatte sich mit einem Basilisken gestritten, der sie dann biss. Nach meiner Geburt starb sie. Wenige Wochen danach. Ich hatte sie kaum kennengelernt. Aber sie hatte mir eine Menge Wissen vermittelt.}

\gedanke{Deswegen auch deine ungewöhnliche Farbe.}

Der Drache schaute ihn an. \stimme{Daran habe ich noch gar nicht gedacht.}

Elber öffnete das Kästchen und beide sahen hinein. Es war leer.

\stimme{Und nun?}

\gedanke{Ins Reservat. Ich würde mich dort gerne umsehen.}

\stimme{Das dauert aber, bis wir dort angekommen sind.}

\gedanke{Wieso? Apparieren geht schneller als fliegen.}

\stimme{Aber auf die\abs} Weiter kam der Drache nicht mehr. \stimme{\aabs Entfernung geht das doch\abs} Sie tauchten mitten im Zelt mit den Arbeitern auf, die gerade zu Abend aßen. \stimme{\aabs gar nicht}, beendete Tabaluga seinen Satz.

Sofort wirbelten die Männer und die Frau herum und richteten ihre Zauberstäbe auf die beiden vermeidlichen Eindringlinge. Als sie Tabaluga erkannten, nahmen sie ihre Stäbe herunter, steckten sie ein und aßen weiter. Charlie Weasley kam auf ihn zu und wurde freudig begrüßt.

\enquote{Hallo kleiner}, sagte Charlie. \enquote{Hast du meine Nachricht überbracht?}

\stimme{Ja, ich habe jemanden mitgebracht, der uns helfen wird.}

Charlie schüttelte die Hand von Elber und stellte sich vor. \enquote{Charlie Weasley. Alle sagen aber nur Charlie.}

\enquote{Dann nennen Sie mich Frederick.}

\enquote{Und wie noch?}

\enquote{Elber.}

\enquote{Dann sind sie der anfangs mutmaßliche Todesser?}

\enquote{Das hat Ihnen Ihr Bruder geschrieben, stimmt’s?}

\enquote{Ups. Ja}, gab er kleinlaut zu.

\enquote{Sie sagten ja, anfangs und mutmaßlich, dann passt das schon. Es scheint, dass er seine Meinung geändert hat.}

Charlie nickte nur. \enquote{Was machen wir jetzt?}

\enquote{Jetzt möchte ich als Erstes einen ihrer Drachen sehen, damit ich weiß, ob mein Verdacht stimmt.}

\stimme{Hat es was mit diesem Holzkästchen zu tun?}

\enquote{Ich hoffe nicht, aber ich befürchte es. Das, was darin war, ist verschwunden.} Charlie Weasley macht ein besorgtes Gesicht. \enquote{Keine Angst, wenn es das ist, dann ist eine Heilung nicht nur möglich, sondern auch einfach und relativ unkompliziert.}

\enquote{Aber was befürchten sie dann?}

\enquote{Dass andere Drachen-Kolonien auch betroffen sind. Wir müssen das, was entwendet wurde, wieder bekommen.}

\enquote{Vernichten wir es dann?}

\enquote{Das geht leider nicht. Zumindest habe ich keine Ahnung, wie ich es anstellen soll.}

\enquote{Dann werden sich andere darum kümmern.}

\enquote{Die werden das Teil nicht mal analysieren können. Es ist mit einer speziellen Art von Magie ausgestattet. \gst Reden wir draußen weiter.}

Sie verließen das geräumige Zelt und traten in die warme und dunkle Sommerluft hinaus und gingen einen ausgetretenen Pfad entlang.

\enquote{Das klingt alles so, als ob sie das Teil, was auch immer es ist, sehr genau kennen.}

\enquote{Ich habe mich damit beschäftigt. Es studiert.}

\enquote{Was ist es?}

\enquote{Es ist etwas um Drachen umzubringen. Es verströmt eine Art von Magie, die Drachen krank macht. Ihre Zellen werden verändert. Da es durch die Magie selbst geschützt ist, kann man es fast nicht zerstören.}

Charlie dachte nach. \enquote{Sie sagten fast.}

\enquote{Es kann unter Umständen gelingen, es zu zerstören.}

\enquote{Wie?}

\enquote{Man wendet etwas gegen das Objekt, das ebenfalls Zugriff auf die Magie hat. Ich meine damit keinen Zauber, oder Fluch. Ich spreche von der reinen Magie.}

\enquote{Das verstehe ich nicht. Erst sagen Sie: \enquote{Man kann es nicht zerstören.} Dann sagen Sie: \enquote{Unter Umständen.} Und jetzt plötzlich: \enquote{Man muss nur.} Was nun?}

\enquote{Ich weiß, dass und wie man es zerstören kann. Aber ich traue mich nicht, diese Methode anzuwenden. Sie ist gefährlich.}

\enquote{Dann werde ich es machen.}

Der Drache begleitet die beiden und lief zwischen beiden hin und her. Er beobachtete sie und lauschte ihren Geistern, damit er verstand, was sie zu sagen hatten, denn die Sprache der Menschen beherrschte er nicht.

\enquote{Sie werden sterben. Ich werde vielleicht für eine Weile außer Gefecht gesetzt. Oder mithilfe von Tabaluga sogar unbeschadet davon kommen.}

Charlie schaute ihn erstaunt an.

\stimme{Wir sind in der Mitte des Reservates}, informierte Tabaluga sie.

\gedanke{Dann werden wir uns hier niederlassen und die Gedanken schweifen lassen.}

\enquote{Sie können mit Tabaluga kommunizieren?}

\enquote{Ja.}

\enquote{Ich dachte nicht, dass das jemand kann, der nicht in direktem Kontakt mit Drachen steht.}

\enquote{Lassen Sie uns später darüber reden. \gst Schlafen Sie gerne draußen?}

\enquote{Ja.}

\enquote{Dann werden wir uns an\abs} Er sah Tabaluga an. \enquote{\aabs dich ankuscheln.}

\stimme{Warum?}

\gedanke{Drachen haben eine Menge an Magie in sich. Sie können sie zwar nicht aktiv nutzen, aber deshalb können sie sich sehr gut dagegen wehren. Kaum ein Zauber kommt durch sie durch. Wenn wir mit Körperkontakt entspannen, dann wird sich diese Magie mischen und wir können die Verursacher aufspüren, sofern sie noch hier sind. Sie verstehen eine Menge von Magie.}

Tabaluga nickte und legte sich auf die Seite. Frederick und Charlie legten sich mit ihren Köpfen auf seinen Bauch. Dann schlossen sie die Augen und ließen ihren Geist treiben. Jeder hatte das Gefühl über dem eigenen Körper zu schweben. Die Distanz vergrößerte sich. Mehrere hundert Meter über dem Reservat drehten sich die Geister um und überblickten das Reservat. Langsam bildeten sich Magie-Inseln auf dem Boden. Eine stach daraus deutlich hervor. Von ihr gingen in zyklischen Abständen Wellen aus, die den Luftraum zu überwachen schienen. Vorsichtig und immer nur dann, wenn keine Welle kam, bewegte sich die geistige Energie auf die große Insel zu.

Als sie in Reichweite waren, sahen und hörten sie zwei Todesser.

\enquote{Wir warten noch zwei Tage, dann gehen wir ins nächste Reservat.}

\enquote{Meinst du, das ist sicher? Wir sind schon einige Tage hier. Ich bin mir nicht sicher, ob der Zauber hält.}

\enquote{Den Zauber hat mir der Dunkle Lord persönlich beigebracht und getestet. Er wird halten. Keiner kann uns entdecken.}

\enquote{Na schön. Dann schalten wir das Gerät aber aus.}

\enquote{Von mir aus. Die Drachen schaffen es eh nicht mehr.}

Dann fanden sie sich wieder in ihren eigenen Körpern.

Charlie schlug die Augen auf und meinte: \enquote{Dann mal los, wenn wir sie noch erwischen wollen.}

\stimme{Du hast doch gehört, dass sie noch zwei Tage hier bleiben wollen. Also keine Hektik.}

Doch Charlie gab sich nicht ganz geschlagen. \enquote{Und die Drachen. Dann sollten wir denen helfen.}

\stimme{Wenn du dich wieder beruhigt hast, dann kannst du uns helfen.}

\gedanke{Wobei?}

\stimme{Den Drachen zu helfen?}

\gedanke{Wie?}

\stimme{Leg dich hin, Charlie. Du wirst es merken.}

Charlie war verwirrt. Doch dann legte er sich hin und dieser schwebende Zustand stellte sich wieder ein. Zusammen mit Tabaluga und Frederick reiste er gedanklich zu jedem Drachen und half ihm mithilfe seiner Magie. Zumindest hatte er das Gefühl. Er glaubte, dass er mit jedem Drachen eine Verbindung einging, eine Verbindung die einer Gedankenverschmelzung gleich kam. Doch nach und nach verloren sich die erlangten Erkenntnisse und Gedanken, die die Verbindung mit sich brachte. Die Essenz der Drachen blieb zurück.

Infolgedessen, wurde Charlie ruhiger. Sein Verhältnis zu den Drachen wurde besser, doch das würde er erst in ein paar Wochen langsam feststellen.

Der Vorgang, die Drachen zu heilen, dauerte die ganze Nacht. Doch als sie bei Morgendämmerung fertig waren, hatte keiner der drei Ermüdungserscheinungen.

Die anderen Drachenbändiger kamen und tuschelten, als sie die drei liegen sahen.

\enquote{Wir sind wach}, antwortete Charlie. \enquote{Ich weiß, wo die Todesser sind.}

Frederick setzt sich auf und auch Tabaluga sah ihnen interessiert zu.

Nach einer Weile fragte Charlie: \enquote{Wollen Sie sich nicht daran beteiligen, Frederick?}

\enquote{Nein. Ich warte ab, höre zu, komme mit und kümmere mich ausschließlich um das Teil. Es darf keinen Schaden nehmen. Es muss richtig zerstört werden. Wenn es angeschlagen ist, kann dies unvorhergesehene Auswirkungen haben.}

Charlie nickte und beredete mit seinen Kollegen die weitere Vorgehensweise.

\trenn

\enquote{Ich dachte immer, von Hogwarts aus kann man nicht apparieren}, sagte Hermine vorwurfsvoll.

\enquote{Kann man auch nicht. Aber Drachen sind mächtige Geschöpfe.}

\enquote{Aber Drachen können doch nicht zaubern.}

\enquote{Das nicht, aber Drachen können sich durch die Magie schützen. Warum wohl, kann man Drachen fast nichts anhaben.}

\enquote{Wie?}

\enquote{Drachen haben, wie auch immer, Zugriff auf die Magie. Sie können sich damit hauptsächlich schützen.}

Jetzt zeigten mehr Interesse an der Unterhaltung zwischen Hermine und Dumbledore. Draco und Harry hielten an und lauschten.

Als Dumbledore dies merkte, winkte er allen zu und ging Richtung Große Halle. \enquote{Machen wir eine Unterrichtsstunde daraus.}

In der Großen Halle angekommen, lies er mit seinem Zauberstab die Bänke verschwinden und viele weiche Kissen und Decken in der Halle erscheinen. Er selber setzte sich an den Kopf des Arrangements auf ein weiches Kissen und begann zu erzählen, nachdem die Schüler ihre Plätze eingenommen hatten.

\enquote{Wie sicher alle wissen, haben Drachen einen guten Schutz gegen viele Zauber. Es wird ihnen nachgesagt, dass ihre Haut die Zauber abwehrt. In gewisser Art und Weise ist das richtig, denn in ihrer Haut ist besonders viel Magie versammelt, die sie vor Angriffen zu schützen versucht. Das ist bei den Drachen genetisch veranlagt und lässt sich nicht ändern. Leider hat das den Nachteil, dass die Drachen dadurch schwierig werden, kurzum, sie werden aggressiver. Selber können sie nicht aktiv über ihre Magie verfügen. Sie haben aber die Möglichkeit, sofern es sich um zahme Drachen handelt, was selten vorkommt, dass sie ihre Magie mit einem Zauberer verbinden. Dann bildet sich eine Einheit. Beide profitieren davon, während der Verbindung. Diese muss aber eine körperliche Verbindung sein.}

Jetzt kamen die ersten Fragen.

\enquote{Aber ich dachte, dass Drachen von sich aus böse sind.}

\enquote{Nein, aber missmutig und skeptisch gegenüber anderen Arten.}

\enquote{Wie kann ich das verstehen?}

\enquote{Wenn die Drachen nicht diese Fähigkeit, beziehungsweise Magie, in ihrer Haut hätten, dann wären sie friedlicher. Zwar immer noch aggressiv und misstrauisch, aber etwas freundlicher.}

\enquote{Sie meinen also, dass die Magie die Drachen beschützt?}

\enquote{In gewisser Weise ja.}

\enquote{Was werden die zwei wohl machen?}, fragte ein anderer Schüler.

\enquote{Wie meinen Sie das?}

\enquote{Es sah so aus, als ob sich die beiden unterhielten, bevor sie verschwanden.}

\enquote{Ja, das hat mich auch verwundert}, sagte Dumbledore nachdenklich.

\enquote{Ich finde ihn ja manchmal etwas eigenartig}, sagte ein weiterer Schüler.

\enquote{Inwiefern?}

\enquote{Er redet über Zaubersprüche und Flüche, egal welcher Couleur, so, als würde er sich mit jemandem über das Wetter unterhalten. Er macht keinen Unterschied, ob es sich um schwarze oder weiße Magie handelt.}

\enquote{Hast du nicht aufgepasst? Es gibt k\aabs}, konterte ein anderer Schüler.

\enquote{Ich weiß, ich weiß, das war nur ein Beispiel. Drück es besser aus. Auf jeden Fall spricht er über diese Zauber als seinen sie ganz normal. Er unterscheidet nicht. Man könnte wirklich meinen, er arbeitet für\abs die andere Seite.}

\enquote{Ja}, gab Dumbledore zu. \enquote{Das könnte man glauben. Jetzt aber nicht mehr, oder? Ist noch irgendjemand der Meinung, dass es sich um einen Todesser handelt? Oder um einen Unterstützer von Voldemort?}

Es war totenstill in der Großen Halle. Keiner sagte mehr etwas. Stumm schüttelten sie ihre Köpfe.

\enquote{Dann meine ich, dass es an der Zeit wird, ins Bett zu gehen.}

Die Versammlung löste sich auf und ging in ihre Zimmer. Auch Harry legte sich nach der abendlichen Toilette in sein Bett. Während des gesamten Vortrages kam ihm immer wieder der Drache in den Sinn.

\gedanke{Er hatte die Farben Slytherins. Grün und Silber. Aber Drachen sind nicht Silber. Basiliskengift schimmert silbern. Dementorenblut ist, glaube ich silbern. Quecksilber ist \gst Quatsch. Quecksilber. \gst Am wahrscheinlichsten ist Basiliskengift.} Er schloss seine Augen und dachte weiter. \enquote{Das wäre schön, jetzt zu fliegen, wie ein Drache, oder ein Vogel.}

Harry versank in einen leichten Dämmerschlaf.

\begin{traum}
Er stand auf einer Klippe, doch er war nicht er selber. Trotzdem erkannte er seinen Körper. Dann nahm er ein paar Schritte Anlauf und stürzte von der Klippe. Seine Flügel breiteten sich sofort aus und er flog. Er war ein dunkler Rabe. Er flog über Wälder und Äcker. Über Wiesen und Hügel. Dann sah er Futter. Er bewegte sich auf ein Säugetier zu und spie Feuer. Er war ein Drache. Die Verwandlung bekam er nicht bewusst mit. Er wusste nur, dass er ein Drache war und unter sich einen Wildhund entdeckt hatte, der keine Chance hatte; genüsslich verspeiste er ihn.
\end{traum}

Harry bekam nicht mit, wie sich während seines Traumes seine Haut zu verwandeln begann. Sie bildeten Schuppen aus und verschwanden wieder, nachdem der Traum beendet war.

Am nächsten Morgen wunderte er sich darüber, dass es ihm nicht aufgefallen war, dass er sich plötzlich von einem Raben in einen Drachen verwandelt hatte. Beim Raben war die Frage der Farbe noch einfach. Raben sind schwarz. Aber Drachen können viele Farben haben. Da er sich nicht sehen konnte, hatte er keine Ahnung.

\enquote{Ron, Hermine. Guten Morgen, Ginny. Hört mal her. Ich hatte heute Nacht einen komischen Traum.} Er erzählte den dreien auf dem Weg zum Frühstück von seinem Traum.

\enquote{Meinst du, das hat was zu bedeuten?}

\enquote{Klar, aber was?}

\enquote{Du sehnst dich nach Freiheit, schätze ich}, sagte Hermine.

\enquote{Vielleicht willst du auch durchs Feuer gehen}, meinte Ginny.

\enquote{Sehr witzig}, gab er zurück.

\enquote{Schau doch in der Bibliothek nach einem Buch über Traumdeutung}, schlug Ron vor.

\enquote{Weißt du, wie viel man dafür beachten muss? Ich habe mal eine Sendung darüber ansehen müssen. Das ist mir zu aufwändig. Beizeiten werde ich es schon erfahren.}
%Zitat: Der Mächtige wird erst mächtig, wenn er seine Macht gebraucht. (Ernst R. Hauschka, deutscher Aphoristiker, geb. 1926) (Oktober 2007)




\begin{kommentar}
Auf dem Hof vor dem Großen Tor taucht plötzlich ein Drache auf. Sein Name ist Tabaluga. Dieser Name setzte sich schon aus den Buchstaben zusammen, die Harry bei seiner Übung hinter Tarnungen zu sehen, entdecken musste und denen auf dem Boden davor.
\end{kommentar}

\chapter{Drachenhaufen}


Charlie schlich sich mit seinen Kollegen an die Grenzen des Todesser-Lagers heran. Im Schutze von Büschen und Bäumen warteten sie, bis Tabaluga und Frederick ihren ausgemachten Überraschungsangriff starteten. Sie wollten direkt in das Lager apparieren und somit den Schild zum Einsturz bringen. Dann kamen Charlie und seine Kollegen dran. Für die Todesser im Inneren des schützenden Feldes sah das so aus, sollten sie zusehen, dass sie sich an einen Drachen heranschlichen. Denn ein Drache hatte sich in die Nähe gelegt und war somit ein prima Alibi.

Zeitgenau apparierte Frederick mit Tabalugas Hilfe in das Innere der schützenden Magie-Kuppel, die mit einem starken Desillusionierungszauber, Verzerrungszaubern und Aufspür-Verhinderungszaubern ausgestattet war. Die beiden Todesser konnten gar nicht so schnell reagieren, wie der Drache ihnen Feuer entgegen spie, um seinen Begleiter zu schützen und die Todesser abzulenken. Frederick lies die Zauber brechen und die Wildhüter griffen von hinten an. Nach knappen zwanzig Sekunden war das Gefecht auch schon zu Ende. Gegen die Übermacht hatten sie keine Chance.

Als sie gefesselt am Boden saßen, ihre Zauberstäbe wurden ihnen abgenommen und ein Anti-Apparierfeld errichtet, fragte einer der Todesser \gst besser gesagt er schrie es fast heraus: \enquote{Wie habt ihr es durch die Abschirmung geschafft? Sie war perfekt.}

Charlie wollte schon antworten, doch Frederick war schneller. \enquote{Wissen Sie, es ist eine Sache einen Zauber zu wirken, wenn man ihn kennt. Aber eine ganz andere ist es, die Magie zu beherrschen und die Zauber unwirksam zu machen, weil man ihnen die Grundlage dessen entzieht, was sie antreibt.}

\enquote{Schöne Worte ohne Inhalt}, maulte er zurück und er zerrte an seinen Fesseln.

Frederick schnippt einmal mit seinen Fingern und der Mund des Todessers verschwand. Seine Stimmbänder brachten keinen Ton mehr heraus.

\enquote{Es ist leicht, sich zu beschweren, wenn man es nicht besser weiß, wenn einem die Fantasie fehlt.}

Dann machte er sich daran mit Tabaluga zum Zelt der Wildhüter zurückzugehen. Zuvor jedoch nahm er den schwarzen matt-schimmernden Brocken mit, legte ihn in das hölzerne Kästchen und nickte Tabaluga erleichtert zu.

\enquote{Was wird jetzt mit dem stimmlosen Todesser?}, fragte Randolf, einer der Wildhüter.

\enquote{Was soll mit ihm sein?}

\enquote{Wird er seine Stimme für immer los sein?}

\enquote{Nein, die Zauber von Askaban lösen die Blockade. Falls es für seine Verhandlung notwendig sein sollte, wird er dort auch sprechen können. Halt, wir sind ja in Rumänien. \gst Die im hiesigen Gefängnis wirkenden Zauber sollten das auch hinbekommen.}

Zurück im Zelt holte er unter den wachsamen Augen von Tabaluga den Stein wieder heraus. Er passte in eine Hand und sah unförmig aus.

\enquote{Ich brauche deine Hilfe}, sagte er. \enquote{Du musst mich berühren, mit einer Pranke auf nacktem Oberkörper. Je mehr Hautkontakt, desto besser.} Während dessen machte er sich oben herum frei. \enquote{Mit der andern Pranke, verletzt du meine Hand mit einer Kralle. Das Blut muss über den Stein fließen, während ein Zauber gesprochen wird. Das wird den Drachen hier helfen. Als Nächstes muss dein Blut über den Stein fließen, während du einen Zauber wirkst. Keine Angst, durch den Hautkontakt wird es dir möglich sein. Damit werden wir feststellen, ob noch andere Drachen betroffen sind. Sollte das sein, dann müssen wir einen stärkeren Zauber wirken. Anderenfalls werden wir den Stein zerstören.}

\enquote{Wird der Stein auf jeden Fall zerstört?}, fragte Tabaluga nach.

\enquote{Ja, auch wenn wir vorher noch einer anderen Kolonie helfen müssen, falls es notwendig wird.}

Tabaluga nickte und legte eine seiner Pranken auf den Rücken von Frederick. Die Innenseite seiner Pranke war ganz leicht pelzig und warm. Sofort spürten beide die Verbindung zueinander, die durch die Magie ausgelöst wurde, welche beide verband. Dann folgte der komplizierte Zauber. Zuerst wurde die Herde geheilt, dann eine Prüfung ausgesandt, um die restlichen Drachen zu finden, die noch betroffen sein sollten.

Ein einzelnes Echo kam zurück. Es kam von einem sehr schwachen Drachen. Dann erfolgte zwischen beiden eine Kommunikation, die keiner Worte bedarf. Sie verschwanden und tauchen in einer kalten Höhle auf; dort, wo der Drache lag, schon halb tot und nicht mehr in der Lage, ein Feuer zu speien, obwohl er es versuchte, um die Eindringliche zu verscheuchen.

Davon unbeeindruckt führten die beiden den Zauber durch und der Drache wurde von dem Fluch befreit. Erneut schickten sie eine Welle los. Dieses Mal kam kein Echo zurück. Deshalb zerstörten sie den Stein.

Vorsichtig ging Tabaluga auf den alten Drachen zu.

\enquote{Lasst mich allein. Ich sterbe doch schon. Lasst mir meine Ruhe und quält mich nicht länger.}

\enquote{Anscheinend ist der Drache nicht nur blind, sondern hat ein Defizit im Geruchssinn.}

\enquote{Wer seit ihr?}, fragte er jetzt.

\enquote{Tabaluga und Frederick}, antwortete er.

\enquote{Zauberergesocks}, gab er abwertend zurück.

\enquote{Ich verbitte mir diese Bezeichnung}, gab Tabaluga beleidigend zurück. Tabaluga sprach plötzlich. Irgendwie hatte er die Fähigkeit zu sprechen gewonnen, doch der Drache war sich der Tatsache noch nicht bewusst.

\enquote{Oh, ein Drache. Behandelt er dich gut?}, antwortete der alte Drache. Auch er sprach.

\enquote{Bisher ja, aber wir kennen uns erst seit gestern Mittag.}

\enquote{Und dann zwingt er dich schon dazu, ihn hier herzuführen, damit er mich quälen kann?}

\enquote{Nein.}

\enquote{Warte. Er muss schon länger von jemand gequält worden sein. Seine Haut hat mehrere Risse und seine Gesundheit ist sehr angeschlagen. Kaum Geruchssinn. Kein Augenlicht, Probleme mit den Lungen und den Nieren. Seine Milz arbeitet kaum noch. Seine Flugmuskeln sind sehr verkümmert.}

\enquote{Waren das die beiden Todesser? Haben Sie hier ihr erstes Opfer gefunden, um den Stein zu testen?}, fragte Tabaluga nach.

\enquote{Ja}, antwortete der Drache. \enquote{Zwei tote Männer. Zauberer. In Schwarz und mit Maske. Sie hatten den Drachenstein dabei.}

\enquote{Wie sieht er aus?}, fragte Tabaluga.

\enquote{Sah}, meinte Frederick.

\enquote{Etwa so groß, dass er in eine menschliche Hand passt. Schwarz und unförmig.}

Tabaluga sah Frederick an.

\enquote{Ich sagte doch: \inner{Sah.}}

\enquote{Das war der Drachenstein?}, fragte Tabaluga nach.

\enquote{Ja.}

\enquote{Der Schrecken aller Drachen. Ein Teufelszeug}, gab der alte Drache matt zurück.

Frederick nickte nur. \enquote{Leider}, gab er mit gesenktem Kopf zurück.

\enquote{Gibt es da mehr?} Erneutes nicken. \enquote{Was?}

\enquote{Das erzähle ich dir später. Ich hänge an meinem Leben. Jetzt werden wir erst einmal\abs Wie ist dein Name?}

\enquote{Tabaluga, aber das weißt du doch.}

\enquote{Ich meinte den anderen Drachen.}

\enquote{Nepomuk}, gab dieser zurück.

\enquote{Dann werden wir Nepomuk erst einmal helfen.}

\enquote{Wie?}

\enquote{Wir nehmen dich in ein Drachenreservat mit. Dort kümmern sich nette Zauberer, Tabaluga kann dir mehr davon erzählen, um dich\abs} Er berührte Tabaluga kurz, worauf hin dieser wusste, was und wie er es zu tun hatte. \enquote{Während die Wildhüter sich um deine körperliche Verfassung kümmern, Nahrung, Trinken und den Muskelaufbau, werde ich versuchen, dir das Augenlicht zurückzugeben.}

\enquote{Das geht?}, fragte Nepomuk mit ängstlichen und ungläubigem Unterton zurück.

\enquote{Ich versuche es. Ich weiß um die Wirkungen des Drachensteins. Daher bin ich guter Dinge.}

\enquote{Und mein Feuer?}

\enquote{Das weiß ich nicht. Es ist möglich, dass ich es nicht schaffe, dass es von allein wieder kommt, oder, dass es keine Möglichkeit gibt, es wiederzuerlangen.}

Nepomuk schnaufte einmal durch. \enquote{Ich habe keine Möglichkeit, eure Aussagen zu prüfen.}

\enquote{Wenn ich fies wäre, dann würde ich sagen, dass du eh keine Chance hast. Wenn ich es wollte, dann hättest du keine Chance gegen mich. Nicht in diesem Zustand, in dem du bist.}

\enquote{Gib mir ein paar Minuten zum Nachdenken.}

\enquote{In Ordnung}, gab Frederick zurück und wartete.

Dann sagte der Drache schließlich zu und zu dritt verschwanden sie aus der Höhle und tauchten vor dem Zelt auf. Frederick zog sich sein Hemd und seinen Umhang wieder an, als die Wildhüter zu ihrem Zelt zurückkamen.

\enquote{Was ist mit ihm denn los?}, fragten sie.

Nepomuk wurde aufgeregt und richtete sich auf. Er ging einige Schritte zurück.

\enquote{Ganz ruhig. Das gilt für alle. Bleibt stehen}, sagte Tabaluga.

\enquote{Du kannst sprechen?}, fragte Charlie.

\enquote{Ja. Seit ein paar Stunden.}

Langsam beruhigte sich Nepomuk.

\enquote{Er heißt Nepomuk und wurde vermutlich von den beiden Todessern misshandelt und gequält. Denen, die ihr geschnappt hattet. Geht vorsichtig und ruhig mit Nepomuk um. Er ist sehr schwach. Er braucht Nahrung und was zum Trinken. Seine Flugmuskeln sind verkümmert. Er hat kein Augenlicht und seine Geruchszellen funktionieren auch kaum.}

Charlie und seine Kollegen nickten und gingen behutsam mit dem Drachen um. Zitternd und ängstlich, aber dennoch geduldig, lies er alles mit sich machen. Immer bereit aufzuspringen und in Panik zu verschwinden. Währenddessen kümmerte sich Frederick um seine anderen Verletzungen. Sorgfältig begann er Augen und Nase zu untersuchen.

Zeitgleich knüpfte Tabaluga zu Nepomuk ein Band. Er erzählte ihm in Wort und Bild direkt in den Geist des Drachen von seinen Erlebnissen mit anderen Drachen und den Menschen. Nepomuk beruhigte sich zunehmend. Dann fing er an auf einem Auge hell und dunkel zu erkennen. Dann folgte das andere.

\enquote{Für jetzt ist es genug}, sagte Frederick. \enquote{Das muss erst einmal eine Stunde ruhen. Dann können wir weiter machen.} Dann kam die Nase dran.

\enquote{Ihr Menschen riecht noch immer sehr streng}, sagte Nepomuk. Frederick musste über diese Aussage lachen. \enquote{Was lachst du? Ich habe dich eben beleidigt.}

\enquote{Hast du nicht. Ich weiß, dass wir für euch streng und eigenartig riechen. Außerdem hast du mir bestätigt, dass mein Zauber gewirkt hat und dir einen Teil deiner Riechnerven wieder gegeben hat. Das war ein schweres Stück Arbeit.} Er schnaufte schwer. \enquote{Ich brauche einen Trank für Nepomuk. Seine Milz ist stark angegriffen. Leber und Nieren erholen sich von allein wieder.}

Einer der Wildhüter nickte und verschwand. Nepomuk konnte die Bewegung sehen. \enquote{Da läuft was}, sagte er.

\enquote{Einer meiner Kollegen}, antwortete Arthur, einer der Pfleger.

Nepomuk senkte seinen Kopf wieder und legte ihn auf das weiche Gras. Kurz darauf spürte er Arme auf seiner Schnauze.

Er hörte Tabaluga in seinem Geiste sagen: \stimme{Das muss ihn schwer erschöpft haben. Er ist auf deiner Nase eingeschlafen und liegt auf seinen Händen.}

Nepomuk brachte nur noch ein Lächeln zustande und döste danach etwas. Als der Trank für seine Milz ankam, öffnete er schläfrig sein Maul etwas, lies die Flüssigkeit hineinlaufen, schloss ihn wieder und schluckte den Trank. Dann schlief er weiter.

Als er wieder erwachte und seine Augen öffnete, sah er bereits Umrisse und konnte Konturen von Gestalten erkennen. Er konnte Mensch und Drache unterscheiden.

\enquote{Meine Augen sind besser geworden}, erklärte er.

\enquote{Es sind auch zwei Stunden vergangen}, sagte Tabaluga.

\enquote{Dann hast du eine weitere Behandlung vorgenommen?}, fragt er. In der Hoffnung, Frederick sei in der Nähe, da er die Konturen nicht wirklich auseinander halten konnte. Zumindest sahen alle Menschen für ihn gleich aus.

\enquote{Ja}, antwortete dieser und begann mit der nächsten.

Der nächste Trank wurde ihm gereicht und auch diesen schluckte er, ohne sich zu beschweren. Seine Augen begannen etwas zu tränen und leicht zu kribbeln. Er schloss sie.

\enquote{Es kribbelt leicht, nicht wahr?}, fragte Frederick nach. Nepomuk nickte mit seinem Kopf. \enquote{Das ist ein sehr gutes Zeichen.}

\enquote{Du bist auf mir eingeschlafen}, gab er amüsiert zurück.

\enquote{Du bist auch sehr weich}, antwortete Frederick.

Wieder glitt ein Lächeln über das Gesicht des Drachen. Nach einer weiteren Stunde, es war bereits Mittag, hatte er sein Augenlicht zurückerhalten. Glücklich fuhr er Frederick mit seiner Zunge über das Gesicht und wunderte sich, warum dieser nicht nass war. \gedanke{Automatischer Säuberungszauber}, dachte er. \gedanke{Er wird nass, aber das trocknet sehr schnell wieder.}

\enquote{Deine inneren Organe brauchen noch ein paar Tage. Deine Augen schonst du bitte noch die nächste Zeit. Einer der Wildhüter wird sie noch kontrollieren.}

\enquote{Und meine Nase?}

\enquote{Ich bin noch geschwächt. Ich glaube, heute nicht mehr. Eventuell auch morgen nicht.}

\enquote{Und wenn ich dir helfe?}, fragte Tabaluga nach.

Frederick sah ihn an. \enquote{Das kostet eine Menge Kraft. Du wirst dich die nächsten Tage dann nicht gegen Zauber zur Wehr setzen können.}

Tabaluga dachte nach und sah beide abwechselnd an. \enquote{Ich glaube, dann schaffe ich das nicht.}

\enquote{Das denke ich auch. Zumal ich nicht allein zurück apparieren kann. Ich brauche jemand, der mich begleitet. Ich muss unterrichten. Ich habe nicht mehr die Kraft, allein zu apparieren.}

\enquote{Du wirkst aber noch immer fit.}

\enquote{Es ist auch keine körperliche Schwäche. Es ist eine Schwäche der Magie.}

\enquote{Darf ich dann die paar Tage bei dir bleiben?}

\enquote{Nicht im Schloss. Aber bei unserem Wildhüter.}

\stimme{Wildhüter? Hogwarts? Arbeitet Hagrid immer noch dort?}, hörten beide eine Stimme von hinten.

Ein Drachenweibchen näherte sich ihnen. Sie war größer als Tabaluga, aber immer noch ein Teenager. Sie war rot und für einen Drachen hübsch. Zumindest aus Menschensicht.

\enquote{Ja}, antwortete Frederick schlicht.

\stimme{Darf ich mitkommen? Ich würde ihn gerne mal sehen.}

\gedanke{Das geht leider nicht. Ihr steht noch unter Quarantäne. Aber ich kann dir versprechen, dass du ihn bald sehen wirst. Ich schätze mal in spätestens zwei Jahren.}

Norberta, das Drachenweibchen, begann zu lächeln und ihre Augen leuchteten. \stimme{Schade, dass es jetzt nicht geht. Aber wir Drachen sind sehr geduldig.}

\gedanke{Geduldig? Ich habe da anderes gehört. Beim trimagischen Turnier waren deine Freunde anderer Meinung. Sie verteidigten ihre Eier sehr schnell. Und das, obwohl keine Brutzeit war.}

\stimme{Du scheinst viel über uns zu wissen.} Frederick nickte. \stimme{Aber es stimmt. Es wurden die wilderen von uns geschickt. Und sie wurden angewiesen, nicht zu geduldig zu sein}, sagte sie und grinste.

Frederick begann zu gähnen.

\enquote{Gehen wir?}, fragte Tabaluga.

\enquote{Gehen? Die Zeit haben wir nicht. Aber wir können apparieren.}

Tabaluga und Frederick berührten sich und apparierten. Leider nur ein paar Meter. Frederick fiel vor Müdigkeit fast um. Norberta kam auf beide zu und nahm Frederick vorsichtig hoch, um ihn auf Tabaluga abzusetzen. Doch auch der zweite Versuch klappte nicht. Frederick konnte sich kaum noch auf Tabaluga halten. Deswegen fixierte man ihn.

\stimme{Gibt es dort andere Drachen?}, fragte Norberta?

\gedanke{Nein.}

\stimme{Dann werde ich euch absetzen. Drachen können ihre Magie durch Zauberer nutzen.}

\gedanke{Das stimmt. Mir fehlt die Kraft. Leitet mich}, sagte Frederick matt und schlief schon wieder fast ein.

Norberta schüttelte ihn etwas, damit er wieder ein paar Sekunden wach blieb und sie verschwanden. Keine Sekunde später tauchte sie wieder auf. \stimme{Ich habe die beiden abgesetzt. Sie werden in zwei Tagen wieder kommen.}

\stimme{Dann kann ich endlich wieder richtig riechen.}

\stimme{Komm, ich zeige dir deinen Schlafplatz. Und später dann deine Herde, falls sie dich akzeptieren.}

Nepomuk nickte und folgte Norberta, dann begab er sich auf einem Strohbett zur Ruhe. Nepomuk war einer der wenigen Drachen, die umgänglich waren. Neben Tabaluga und Norberta und ein paar wenigen anderen, die friedlich waren, war der Rest eher wild. Zwar geduldig, aber dennoch wild.

Die drei tauchten außerhalb des Schulgeländes auf und Norberta verschwand kurz darauf gleich wieder.

Tabaluga blieb mit Frederick zurück und lief Richtung Schloss. Dieses erkannte die beiden und öffnete automatisch die Tore. Gelbe Leuchtpunkte erschienen auf dem Boden und zeigten dem jungen Drachen, wohin er musste. Vor dem Wandteppich wartete er, da dieser gerade nach oben wegrollte. Dann trat er mit seiner Fracht auf dem Rücken in den Bereich hinein. Die Punkte waren nun grün. Vor Fredericks Tür schaute der Drache auf das sich aus schimmernder Magie bildende Gesicht. Es stutzte, schaute den Drachen an und schien ihn zu durchleuchten. Dann nickte es und verschwand. Die Tür öffnete sich mit einem \geraeusch{Klick}.  Dann trat der Drache ein und ging auf die leuchtende Tür zu. Diese öffnete sich ebenfalls. Jetzt wurde Frederick von einer unsichtbaren Macht angehoben, bis auf die Unterhose entkleidet und schwebte in sein Zimmer. Dort hob sich die Decke an, er wurde abgelassen und die Decke legte sich über ihn. Tabaluga wusste nicht genau, was er jetzt tun sollte.

\stimme{Lege dich zu ihm. Leiste ihm in seinem Zimmer Gesellschaft}, hörte der Drache in seinem Geiste.

Tabaluga nickte, betrat den Raum und legte sich auf den weichen Teppich neben dem Bett, nachdem er ihn etwas von ihm weg gezogen hatte. Er legte sich so, dass er Frederick beobachten konnte, und schloss seine Augen um zu schlafen.

Morgens um vier Uhr erwachte Frederick. Er konnte im Dämmerlicht des frühen Morgens zwar wenig, aber dennoch genug sehen, um sich orientieren zu können. Doch er schien noch verwirrt. Er stand auf, zog seine Pantoffeln an und setze sich auf den Stuhl in seinem Zimmer. Die Hände zwischen den Beinen und umherblickend, sah er aus, als wäre es ihm unangenehm, in einem ihm fremden Zimmer zu sitzen. Leicht verstört sah er sich um und entdeckte den Drachen, der noch friedlich auf den Teppich schlief. Frederick legte seinen Kopf schief und schien nachzudenken, woher er ihn kennen könnte. Er sah ihm eine ganze Weile zu, bis um fünf Uhr Tabaluga erwachte.

Das erste, was der Drache sah, war ein leeres Bett. Er richtete seinen Kopf auf und blickte durch den Raum. Dann sah er Frederick. \enquote{Guten Morgen}, sagte der Drache.

\enquote{Guten Morgen, Tabaluga. Was ist gestern noch alles passiert? Ich weiß noch, dass Nepomuks Augen geheilt wurden, aber danach ist bei mir Feierabend.}

\enquote{Feierabend? Es ist Morgens und fünf Uhr zwei.}

Frederick schmunzelte. \enquote{Das ist eine Redewendung der Menschen. Sie bringen damit zum Ausdruck, dass danach nichts mehr kommt. Das heißt, ich kann mich an nichts mehr, was danach passierte, erinnern. Andererseits ist es ein Ausdruck dafür, dass man mit der täglichen Arbeit aufhört und die Freizeit beginnt.}

Tabaluga nickte. Wieder hatte er etwas gelernt.

\enquote{Frühstück?}, fragte Frederick ihn.

\enquote{Gerne.}

\enquote{Was essen Drachen denn so? Rind? Hase? Schwein? Gemüse?}

\enquote{Bevorzugt Fleisch. Aber auch Gemüse und Obst. \gst Wegen der Vitamine}, fügte er noch hinzu.

\enquote{Malcomin? Kommst du?}

Ein Elf erschien im Raum und staunte, als er den Drachen sah.

\enquote{Wir hätten gerne Frühstück. Für einen Menschen und einen Drachen. Ausgewogen bitte.} Der Elf nickte mit dem Kopf und verschwand. \enquote{Dann lass uns mal ins Wohn- und Speisezimmer gehen.} Frederick ging voraus und der Drache folgte ihm.

Als beide saßen, tauchte wieder der Elf auf und belegte den Tisch mit verschiedensten Früchten, Broten und Fleischstücken. Dann setzte er sich dazu und begann sich auf seinen Teller Früchte zu laden. Frederick tat ebenfalls Früchte auf seinen Teller. Ebenso eine Semmel, sowie Butter und Marmelade darauf, nachdem er ihn aufgeschnitten hatte. Tabaluga nahm sich einen Brocken Fleisch und schluckte die Brocken, die er abbiss ganz hinunter. Danach kamen noch einige Früchte dazu, sowie ein Apfel als Abschluss.

\enquote{Was machst du heute?}, fragte ihn Tabaluga.

\enquote{Den Vormittag habe ich immer frei, zumindest montags. Am Nachmittag habe ich die erste und die dritte Klasse in \VgddK. Wie wäre es, wenn wir zusammen zu Hagrid gehen? Hogwarts’ Wildhüter und Lehrer im Fach \fach{Pflege magischer Geschöpfe}. Vielleicht nutzt er die Gunst und die Schüler bekommen etwas über Drachen zu hören.}

\enquote{Er will mich als Anschauungsobjekt verwenden?}, fragte Tabaluga verwundert nach. \enquote{Warum nicht. Dann erfahre ich, wie die Menschen uns sehen und kann das meiner Herde weitergeben.}

\enquote{Pass aber auf. Hagrid ist etwas eigen, was den Umgang mit anderen Spezies betrifft.}

\enquote{Inwiefern eigen?}, fragte er und schob noch einen ganzen Hähnchenflügel nach.

\enquote{Na ja. Er wollte schon immer einen Drachen haben.}

\enquote{Norberta.}

Frederick nickte. \enquote{Erzähl ihm bitte noch nichts davon. Du wirst es später verstehen. Aber nun zurück zu Hagrid. Er sieht \gst verzeih mir \gst Drachen und andere Tiere nicht als gefährlich an. Aber seine Umgebung tut das. Also alle anderen Menschen.}

\enquote{Deswegen sind sie manchmal grob.}

\enquote{Ihr seid ja auch wilde Gesellen.}

\enquote{Ich nicht}, empört sich Tabaluga schmunzelnd.

\enquote{Dich würde ich auch jederzeit streicheln.}

\enquote{Vor der Klasse?}

\enquote{Mal sehen}, scherzte Frederick. \enquote{Soll ich dich noch etwas herumführen? Wir haben ja noch Zeit.}

\enquote{Gerne.}

Die beiden machten sich auf den Weg, während der Hauself den Frühstückstisch abräumte, nachdem er selber fertig war. Zuerst ging es in den Krankenflügel. Madame Pomfrey war bereits wach und war überrascht, die beiden zu sehen. Nach einer kurzen Untersuchung des Drachens durch die Krankenschwester ging es auch schon weiter. Sie musste diese seltene Gelegenheit nutzen, um etwas Erfahrung zu sammeln. Da aber Tabaluga den ganzen Tag auf dem Gelände verweilte, könne sie später noch einmal ein paar nicht invasive Untersuchungen durchnehmen, versicherte er.
%Invasiv
%Der Begriff invasiv wird in folgenden Zusammenhängen verwendet:
%1. In der medizinischen Diagnostik oder Therapeutik werden solche Methoden als invasiv (Vase) bezeichnet, die in den Körper, ein Korpergefäß eindringen, also z. B. eine Biopsie, Abstrich der Nasenschleimhaut. Eine Sonografie- oder Röntgenuntersuchung ist dagegen nicht invasiv.
%2. Beim Krebs spricht man von einem invasiven Tumor, wenn er in das umgebende Gewebe hineinwuchert.
%3. In der Softwareentwicklung spricht man von invasiven Programmiermodellen, wenn sie Auswirkungen auf viele Komponenten haben.
%4. In der Ökologie versteht man unter invasiven Organismen Neobiota, die sich durch eine hohe Vermehrungs- und Ausbreitungsrate auszeichnen. Siehe auch Biologische Invasion.

Als Nächstes ging es ins Lehrerzimmer. Nach einigen Minuten kam Professor Dumbledore herein und blieb wie angewurzelt stehen. Eine Weile schaute er den beiden zu. Frederick und Tabaluga standen am Fenster und sahen hinaus. Tabaluga stand wie ein Hund mit den Vorderpfoten auf dem Fenstersims.

\stimme{Wer ist das hinter uns?}

\gedanke{Mein Chef. Direktor Dumbledore. Der Schulleiter.}

\stimme{Du hast ihn gespürt. Nicht wahr?}

\gedanke{Du doch auch.} \enquote{Guten Morgen Albus}, sagte Frederick.

\enquote{Guten Morgen, Frederick. Wer ist Ihr Begleiter?}

\enquote{Tabaluga. Das war der Drache, der um Hilfe ersuchte.}

\enquote{Welche Art von Hilfe?}

\enquote{Seine Herde von einer Seuche zu befreien.}

\enquote{Hat er sie bekommen?}

\enquote{Ja.}

\enquote{Welche Seuche denn?}

\stimme{Drachenstein}, sagte Tabaluga.

\enquote{Wer?}

\stimme{Ich.}

\enquote{Der Drache.}

\enquote{Er kann sprechen?}, fragte Dumbledore.

\enquote{Es ist mehr eine gedankliche Kommunikation. Das müssten Sie doch wissen. Wie war das mit den zwölf Anwendungen für Drachenblut?}

\enquote{Da habe ich mich nur mit dem Blut beschäftigt und auch teilweise auf Arbeiten anderer aufgebaut. Ich habe nie mit einem Drachen gearbeitet.}

Tabaluga ging wieder auf seine vier Pfoten und meinte: \enquote{Zeigst du mir noch mehr?}

Dumbledore war ganz erstaunt darüber. \enquote{Du kannst sprechen?}

\enquote{Das hat mich auch schon gewundert}, sagte Elber. \enquote{Vermutlich liegt es an seinem besonderen Erbe.}

\enquote{Welche Art von Erbe?}, fragte Dumbledore und Tabaluga.

\enquote{Gehen wir ein Stück, dann erzähle ich es euch.} Er trat voran und verließ das Lehrerzimmer mit Dumbledore und Tabaluga, der zwischen ihnen lief. \enquote{Tabaluga hat mir erzählt, dass seine Mutter von einem Basilisken gebissen wurde. Deswegen auch der silberne breite Streifen auf seiner Brust. Das hat wahrscheinlich dazu geführt, dass er mehr Verständnis für die Magie entwickelt hat. Zudem dürfte ihm dies auch behilflich gewesen sein, unsere Sprache zu beherrschen.}

Beide staunten über diese Theorie. Erklärte sie doch sehr gut, was passiert war, aber nicht warum.

\enquote{Was ist mit dem Drachenstein?}, fragte Dumbledore nach. \enquote{Es gibt eine Legende, dass er einst von einem dunklen Zauberer erschaffen wurde. Damit sollten die Drachen im Zaum gehalten werden. Es war eine dunkle Zeit für die Drachen. Doch plötzlich verschwand der dunkle Tyrann und mit ihm der Drachenstein.}

Dumbledore und Tabaluga sahen zu Elber, doch dieser schien sie zu ignorieren.

\enquote{Was wissen Sie darüber?}, fragte Dumbledore nach.

\enquote{Ich möchte darüber nicht sprechen.}

Die Reaktionen der beiden Begleiter waren recht unterschiedlich.

Tabaluga dachte an die Gelegenheit, etwas darüber zu erfahren, wenn er wieder bei Nepomuk war, um ihm sein Riechorgan zu kurieren.

Dumbledore schaute Elber kritisch an und dachte erst einmal nach. Dann fragte er: \enquote{Etwas aus Ihrer Familiengeschichte?}

\enquote{So ähnlich}, gab er knapp zurück. Er wollte keine weiteren Nachfragen.

Sie verließen das Schloss und liefen hinunter zu Hagrids Hütte.

Dieser begrüßte die beiden Frühaufsteher. \enquote{Guten Morgen die Professoren.} Und dann, als er den Drachen erkannte: \enquote{Verdamm’ mich. Ein echter Drache. Wo habt’s denn den her? Das ist ja ein prächtiges Exemplar. Irgendwie erinnert er mich an Norbert.}

\gedanke{Sag jetzt bloß nichts Falsches}, ermahnte Frederick Tabaluga. \enquote{Er kam vorgestern Abend hier an und suchte nach Hilfe. Da es bei einigen Exemplaren noch etwas dauert, hat er mich begleitet. Ich dachte, Sie könnten heute eine kleine Stunde über Drachen halten, wenn schon mal einer da ist?}

\enquote{Wie zahm isser denn? Wegen reiten und so. Falls einer der Schüler mal.}

\stimme{Ich lasse keinen auf meinen Rücken steigen nur aus purem Vergnügen. Aber anfassen dürfen die mich und streicheln, wenn sie wollen.}

\enquote{Zahm genug zum Streicheln, Hagrid. Aber auf ihm reiten lassen, würde ich keinen. Da ich heute Vormittag freihabe, dachte ich mir, wir können uns noch etwas zusammen setzen, bevor die Schüler kommen. Tabaluga kann uns begleiten. Schließlich wird er heute kräftig dabei sein, nehme ich an.}

\enquote{Ich gehe dann mal eine Runde spazieren}, meinte Dumbledore und verabschiedete sich von den dreien.

Hagrid bat die zwei hinter die Hütte um auf einer großen Decke, auf der bereits Fang lag, sich zu unterhalten.

Als sich zur normalen Frühstückszeit die Halle füllte, fing Dumbledore mit einer kleinen Rede an, indem er mit einem Löffel gegen seinen Kelch schlug. Die Menge verstummte.

\enquote{Ich habe eine kleine Überraschung für euch. Zumindest für einen Teil von euch. Diejenigen unter euch, die heute bei Hagrid Unterricht haben, werden eine kleine Änderung des Schulplanes haben. Hogwarts hat das seltene Vergnügen, für heute einen jungen Drachen zu haben. Einen ungefährlichen Drachen. Hagrid wird euch heute etwas über ihn erzählen. Den anderen wünsche ich trotzdem einen schönen Unterricht. Auch wenn diese ihn sicherlich nicht haben werden.} Dann setzte er sich wieder.

\enquote{Drachen? Heute?}, fragte Ron entsetzt. \enquote{Nicht bei Hagrid.}

\enquote{Wovor hast du Angst Ron?}, fragte Harry.

\enquote{Angst? Eine Scheißangst. Harry das ist ein Drache. Du weißt doch noch, was während des trimagischen Turniers passiert ist.}

Die um die beiden herumsitzenden Gryffindors nickten und schauten nicht gerade begeistert drein.

\enquote{Warum habt ihr denn Angst?}, fragte Hermine.

\enquote{Hallo? Hagrid! Drache! Klingelt da etwas?}

\enquote{Also bei mir klingelt da nichts. Ich habe den Drachen als ungefährlich in Erinnerung.}

Das Klappern in der Umgebung wurde leiser.

\enquote{Als ungefährlich in Erinnerung?}, fragte Dean nach.

\enquote{Ja}, antwortete Harry. \enquote{Äh, ich denke es zumindest. Es war doch vor zwei Tagen einer da. Ich nehme einfach an, dass\abs}

\enquote{Du nimmst einfach an?}

Harry fiel erst jetzt auf, dass er einfach von dem Drachen ausgegangen war, der um Hilfe ersuchte. \enquote{Ups. Ja, ich nahm einfach an, dass es sich um diesen Drachen handelt. Der war ja zahm.}

\enquote{Dann träum’ mal weiter. Wenn Hagrid den Drachen ausgesucht hat, dann\abs}

\enquote{Moment}, unterbrach ihn Harry. \enquote{Jetzt nimmst du aber an.}

\enquote{Wie auch immer. Wir werden es bald sehen.}

Die Unterhaltung wurde durch ein klingendes Geräusch abgebrochen.

\enquote{Gerade haben mir einige Kollegen gesagt, dass sie heute keine Zeit haben zu unterrichten, da sie}, jetzt musste er schmunzelnd, \enquote{anderweitig beschäftigt sind.} Vereinzelt war jetzt ein Kichern zu hören. \enquote{Also wird es heute nur einen Unterricht geben. Hagrid wird euch alle heute unterrichten.} Es wurde ganz still in der Halle. Man konnte eine Stecknadel fallen hören. \enquote{Er wird euch etwas über Drachen erzählen. Auf diesem Gebiet ist er spitze, wenn ihr mir den Ausdruck gestattet.}

\enquote{Und wenn wir es nicht gestatten?}, fragte ein Schüler nach.

Dumbledore pausierte kurz. \enquote{Dann ist er sehr versiert und kann euch neben einem Drachenhüter sehr viel über diese Tiere erzählen. Ich schlage vor, wir essen zu Ende und werden dann gemeinsam\abs}, dabei sah er zu seinen Lehrerkollegen, \enquote{\aabs zu Hagrid gehen.} Dann setzte er sich wieder und aß weiter.

Während er aß, sinnierte Harry darüber, ob Snape freiwillig seine Stunden heute ausfallen ließ, oder ob er einfach überstimmt wurde. Er blickte kurz zum Lehrertisch und wartete einige Sekunden. Snapes Auge zuckte kurz, als er in seine Richtung sah. Der Rest seiner Miene war wie immer unergründlich. Er versuchte seine Legilimentik an ihm und übermittelte ihm die Frage, ob er es freiwillig getan habe. Als Antwort bekam er nur einen kleinen Jungen zu sehen, der auf eine Frage hin nickte.

\enquote{Was meinst du, Harry?}, fragte ihn Hermine.

Harry drehte langsam seinen Kopf. \enquote{Wozu?}

\enquote{Ob Snape seine Stunden freiwillig hergegeben hat!}

Harry überlegte, was er sagen konnte, oder wie er sich ausdrücken sollte. \enquote{Ich nehme an, er will auch einen Drachen sehen. Vor allem aber, wie Hagrid mit ihm fertig wird. Vielleicht hofft er auch, etwas abzubekommen. Drachen haben immerhin wirkungsvolle Substanzen, um sie in Zaubertränken zu verarbeiten.} Da er immer noch mit Snape verbunden war, bekam er als Antwort ein lachendes Gesicht eines kleinen Kindes. Unwillkürlich musste er lächeln. Dann verblasste es, als er aufblickte und das Gesicht seiner Mutter sah. Dann brach die Verbindung ab. Harrys Kopf zuckte kurz, aber er zwang sich, nicht mehr hinzusehen.

Dann, nach dem Essen, standen sie auf und gingen hinaus, um Hagrid aufzusuchen. In der Gruppe der Schüler ging es den üblichen Hang hinab. Doch als die Gruppe abbog, stutzte Harry \gst bis er den Grund dafür sah. In der Luft waren leuchtende Punkte, die sich zu bewegen schienen. Sie bildeten einen Pfeil. Darunter bildeten ebenfalls leuchtende Punkte einen Schriftzug. \accentuate{Drache} stand da. Sie folgten, wie die anderen, den Hinweisen und Wegpfeilern. Bunte Leuchtwürmchen überflogen sie und bauten sich weiter vorne wieder zu einem Schild auf. Harry musste grinsen.

\enquote{Die Idee stammt bestimmt nicht von Hagrid}, sagte er, als er knapp vor dem Leuchtwürmchen abbog.

Die Würmchen bildeten kurz das Wort, \accentuate{Stimmt!}, worauf hin Harry lachen musste.

\enquote{Warum lachst du}, fragte ihn Ron.

\enquote{Ich musste gerade an etwas denken. Etwas Lustiges.}

Dann kamen sie auch schon am Quidditch-Feld an und bestiegen die Ränge, da die Leuchtwürmchen ihnen wieder den Weg wiesen. Oben auf den Rängen angekommen stellte er fest, dass in der Mitte eine Bühne aufgebaut wurde, auf der bereits Hagrid und der Drache stand. Ganz am Rand konnte er Professor Elber sehen, der auf dem Boden der Bühne auf einem Kissen saß. Auf der anderen Seite saß Professor Hooch. Die beiden Professoren schienen zu ruhen. Von hier sah es aus, als ob beide ihre Augen zu hatten und dösten, oder meditierten.

Ginny saß neben ihm. Es war angenehm.

Nachdem sich alle hingesetzt hatten, begann Hagrid. \enquote{Doch so viele gekommen, ja? Na gut. Wie bereits alle wissen, haben Drachen sehr gute Chancen, einem Zauber zu entgehen, da dieser keine Wirkung auf sie hat. Dies verdanken sie der ihnen eigenen Magie, die in ihrer Haut lagert.}

Die beiden meditierenden Professoren hoben ihre Hände. Dazwischen bildete sich eine Kugel aus Magie. Eine blaue auf der Seite Professor Elbers und eine gelbe auf Seiten Professor Hoochs. Sie rasten auf den Drachen zu und verpufften wirkungslos. Dann saßen die beiden wieder stumm da.

So ging es die ganze Stunde durch. Immer wieder erzählte Hagrid etwas über einen Drachen und manchmal führten die beiden Professoren dies plastisch vor.

Dann, gegen Ende der Stunde, welche bereits vier ganze Stunden dauerte, fragte Hagrid in die Menge: \enquote{Wer unter euch stellt sich für eine kleine Mutprobe zur Verfügung? Diese beinhaltet, sich in die Mitte dieses Kreises zu stellen}, Hagrid zeigte darauf, \enquote{und sich von Tabaluga hier mit Feuer umfließen zu lassen. Wir machen das, wenn die Mittagspause beendet ist.}

Dann ging es an das Mittagessen. Die Elfen erschienen und brachten jedem Schüler ein belegtes Sandwich und einen Becher Kürbissaft, welcher so verzaubert war, dass man daraus trinken, es aber nicht verschütten konnte.

Als Harry wieder aufsah, nachdem er von einem Elfen sein Mittagsmahl entgegengenommen hatte, waren Professor Elber und Tabaluga verschwunden.




\begin{kommentar}
Nachdem der Drachenkolonie geholfen wurde und der Drachenstein zerstört worden war, sprechen Frederick und Tabaluga mit dem Drachen Nepomuk. Dieser Name stammt aus der Kindersendung 'Hallo Spencer'. Dort heißt der Erzähler so. Zwar gibt es auch einen Drachen, aber dessen Name ist Leopold oder auch kurz Poldi.
\end{kommentar}

\begin{kommentar}
Natürlich wurde der Drachenstein von Elber erschaffen. Daher wusste er auch, wie man ihn zerstört.
\end{kommentar}

\chapter{Drachenfeuer}


Die beiden landeten in Rumänien und wurden schon freudig von Nepomuk und den anderen Drachenhütern erwartet. Frederick machte sich sofort an die Arbeit und setzte sich vor Nepomuk. Dieser legte sich nun hin und hielt ihm seine Schnauze entgegen. Beide schlossen ihre Augen und die Heilung begann. Währenddessen unterhielten sich die beiden Drachen Tabaluga und Nepomuk über Tabalugas Ausflug. Die Idee der Feuerprobe gefiel Nepomuk. Darüber musste er unbedingt mit Frederick sprechen.

Während der Heilung wurde sein Geruchssinn immer besser.

Nachdem sie fertig waren, fragte Nepomuk nach. \stimme{Habe ich das richtig verstanden? Ihr wollt eine Feuerprobe bei einem eurer Schüler vornehmen?}

Frederick sah auf und dem Drachen direkt in sein Gesicht. Dann blickte er kurz zu Tabaluga. Schließlich sagte er einfach nur: \enquote{Ja!}

\stimme{Da würde ich gerne der zweite Drachen sein.}

Frederick hob seine Augenbrauen. \gedanke{Bist du dazu in der Lage? Verstehe mich nicht falsch. Aber den Feuerzauber an einem Menschen zu vollziehen schafft nicht jeder.}

\stimme{Ja, das schaffe ich. Meine Mutter hat mir das beigebracht. Sie meinte, es sei alte Familientradition, das zu können.} Wie zum Beweis hob er seinen Kopf und schuf einen Feuerwirbel, der sich wie ein Korkenzieher in die Höhe schraubte.

Fredericks Augen wurden größer. \gedanke{Von Tabaluga wusste ich es, aber dass auch du diese Kunstfertigkeit besitzt\abs}

\enquote{Wollt ihr noch bei uns zu Mittag Essen?}, fragte Charlie.

\enquote{Wir haben leider nicht viel Zeit. Noch eine knappe viertel Stunde.}

\enquote{Das schaffen wir. Bei uns gibt es heute Bohneneintopf. Wir haben für\abs zwanzig Lebewesen gekocht.} Dabei sah er die beiden Drachen an.

\enquote{Also gut}, sagte Frederick und die beiden Drachen nickten.

Nach einem nicht sehr üppigen Mahl, das dennoch gut schmeckte und sättigte, verschwanden die beiden Drachen mit Frederick.

\enquote{Wer unter euch stellt sich für eine kleine Mutprobe zur Verfügung? Diese beinhaltet, sich in die Mitte dieses Kreises zu stellen und sich von Tabaluga hier mit Feuer umfließen zu lassen. Wir machen das, wenn die Mittagspause beendet ist.} Diese Frage stellte Hagrid vor der Mittagspause.

Draco Malfoy meldete sich, worauf Hagrid erstaunt war und nur nickte. Draco stand bereits auf der Plattform in der Mitte des Kreises und wartete.

Als spürten sie es, erschienen die beiden Drachen links und rechts von ihm. Auf der einen Seite stand Tabaluga, auf der anderen Seite Nepomuk.

\stimme{Ist er das?}, fragte Nepomuk.

\gedanke{Ja, das scheint der Freiwillige zu sein. Er hat sich wohl während unserer Abwesenheit gemeldet.}

Hagrid entfernte sich und setzte sich auf den Boden. Er war fasziniert, dass nun zwei Drachen zur Verfügung standen. Aber im Gegensatz zu den anderen war er ganz ruhig. Die Stimmung im Publikum konnte man am ehesten als gespannt bezeichnen. Draco hingegen wusste, dass er dem Paten seiner Schwester vertrauen konnte.

Zeitgleich warfen Tabaluga und Nepomuk einen Feuerschwall seitlich an Draco vorbei. Es entstand eine Feuersäule, die in den Himmel empor stieg. Beide Korkenzieher-Feuerwände vereinten sich und schlossen die sichtbare Lücke. Dann änderten sich die Farben und Strähnen in vielen Farben durchzogen das ansonsten in Gelb und Rot gehaltene Feuer. Blau, Grün und Violett färbten sich Streifen aus Feuer und wirbelten um einen Mittelpunkt herum. Es sah fantastisch aus.

Nachdem das Feuer verstummt war, referierte Hagrid noch etwas über das Feuer der Drachen. Er erzählte, dass Drachen ihre Beute niemals roh essen würden. Auch wenn das so aussähe. Wenn sie ihre Beute in den Mund nähmen, würde sie intern gebraten. Schnell und zuverlässig. Drachen seien außerdem sehr kultiviert. Doch kaum einer glaubte dies.

Nach einer weiteren Stunde war der Unterricht beendet und die Drachen konnten von den Schülern von Nahem gesehen werden. Ein paar mutige konnten sich sogar auf die Drachen setzen. Den beiden gefiel das.

\enquote{Wollen wir mal starten?}, fragte Nepomuk, der aufgrund der Nähe zu Hogwarts und den vielen Zauberern vorübergehend ihre Sprache beherrschte.

Erstaunt nickten diese. Dann hob Nepomuk ab und drehte eine Runde über das Schloss. Der Start verursachte einiges an Aufregung. Lehrer, wie Schüler riefen aufgeregt umher, da sie nicht wussten, was passierte. Nur drei Personen standen da und grinsten sich eins.

\enquote{Das war abgemacht Professor, nich?}, fragte Hagrid.

\enquote{Zwischen den Drachen und den Schülern vielleicht. Ich wusste davon nichts.}

\enquote{Ich würde jetzt auch gerne da oben sein}, meinte Draco.

\enquote{Beherrsche dich. Aber wenn du willst, frag Tabaluga. Er nimmt dich bestimmt mit.}

Draco nickte und wartete kurz. Dann trat er an den Drachen heran und fragte ihn, ob er auf ihm reiten könne. Dieser nickt nur und lies Draco aufsitzen. Dann startete auch er.

Bis in den frühen Abend dauerte es, bis alle Schüler die wollten, einmal auf einem Drachen fliegen konnten. Als letzter wollte noch Dumbledore.

Als er den Drachen verlassen hatte, erklärte Hagrid noch, dass Drachen normalerweise nicht so gutmütig seien und jeden auf ihrem Rücken reiten ließen. \enquote{Wir ham hier zwei außergewöhnliche Exemplare}, sagte er noch. Dann löste sich die Schar langsam auf und es ging zum Abendessen.

Zurück blieben die beiden Drachen und Frederick. Die hölzerne Bühne sank magisch ab und verschwand dann. Nun standen die drei auf dem unveränderten Quidditch-Feld auf dem sandigen Boden.

\enquote{Was ist jetzt mit dem Drachenstein? Was weißt du darüber?}, fragte Nepomuk. \enquote{Und keine Angst. Ich werde dir schon nichts tun}, ergänzte er.

Frederick setzte sich auf den sandigen Boden und Tabaluga legte sich ebenfalls auf den Boden. Frederick erschuf ein wärmendes Feuer, da es merklich kühler wurde. Nepomuk blieb mit seinen Vorderpfoten stehen und senkte lediglich sein Hinterteil ab.

Frederick atmete einmal schwer durch. Dann begann er. \enquote{Sagt euch der Name Mantigru etwas?} Beide Drachen nickten erschrocken. \enquote{Er stand in meinen Diensten. Bevor ich der wurde, der ich heute bin.}

\enquote{Der Schrecken aller Drachen}, sagte Nepomuk. \enquote{Ich hatte einmal das Vergnügen mit ihm. Zum Glück ist er besiegt worden.} Dann stutzte er. \enquote{Wieso kanntest du ihn?}, fragte er nach.

\enquote{Er stand in meinen Diensten. Wisst ihr, ich war nicht immer der ausgeglichene und neutrale Zauberer, also der, den ihr kennengelernt habt. Früher war ich phasenweise brutal und dunkel. Nachdem Mantigru gescheitert ist, erschuf ich den Drachenstein. Dann erkannte ich, dass es falsch war, was ich tat. Auf die genaueren Umstände möchte ich nicht eingehen. Ich entschied damals, den Drachenstein so gut es mir möglich war zu sichern und zu verstecken. Er hatte eine Menge Macht in sich. Man konnte sie auch für das Gute verwenden. Leider hat ihn vor kurzem jemand gefunden und euch damit geschadet. Zwar konnte es abgewendet werden\abs} Er brach ab.

\enquote{Du fühlst dich schuldig?}, fragte Nepomuk mit eigenartigem Belag in der Stimme nach. Eine Weile herrschte stille. Dann sagte Nepomuk: \enquote{Du musst dafür bestraft werden.}

\enquote{Glaube mir, das werde ich gerade.} Die beiden Drachen sahen ihn skeptisch an. Vorsichtig zog er seinen Zauberstab, errichtete um die Gruppe ein Bronze-schimmerndes Feld, das am Boden einen kreisrunden Rand aufwies. Dann legte er seinen Zauberstab zwischen sich und dem Feuer ab. Er atmete noch einmal durch und sagte schließlich: \enquote{Prüft es ruhig nach.} Dann schloss er seine Augen und öffnete seinen Geist, sodass die Drachen seine Gedanken, Wünsche und Erfahrungen sehen konnten. Seine Vergangenheit, sowie seine erwartete Zukunft. Er legte ihnen sein Leben offen.

Ruhig, aber tief atmete er, während die beiden Drachen vorsichtig und zurückhaltend seinen Geist sondierten. Sie drangen immer tiefer in seine Vergangenheit ein. Ihre Mienen spiegelten die ganze Palette an Gesichtsausdrücken, die die Drachen hatten, wider. Wut, Zorn, Freude, Liebe, Abneigung und Hass, Freundschaft und Zuneigung. Über eine Stunde, für die Drachen aber eine Strapaze von mehreren Tagen, dauerte die Prozedur.

Dann zogen sich die Drachen wieder zurück. Frederick verschloss wieder seinen Geist und nahm danach vorsichtig seinen Zauberstab, löste die Barriere auf und steckte ihn ein. Währenddessen sortierten die Drachen ihre Gedanken. Dann herrschte mehrere Minuten Ruhe. Jeder hing seinen Gedanken nach. Die Drachen verarbeiteten das, was sie erfahren hatten, und auch Frederick musste sich ausruhen, denn es kostete ihn eine Menge Kraft, gleich zwei Drachen in seinen Gedanken zu haben.

\enquote{Ich hab mich schon gewundert, was für ein rot-schimmerndes Licht vom Quidditch-Feld kommt}, sagte Dumbledore, der neben Hagrid am Eingang stand und zu den dreien um das Feuer sitzenden sah. \enquote{Dürfen wir uns zu euch setzen?}, fragte er.

Die beiden Drachen schauten sie erst an, dann nickte Tabaluga. Dumbledore und Hagrid kamen näher und setzten sich um das Feuer.

\enquote{Was macht ihr hier? Es ist schon ungewöhnlich, dass Menschen und Drachen so friedlich beieinander sitzen.}

\enquote{Wir erzählten uns Geschichten aus unserer Vergangenheit.}

\enquote{Ich vermisse Norbert}, sagte Hagrid.

\enquote{Norberta}, korrigierte Tabaluga.

\enquote{Wie?}

\enquote{Sie ist ein Weibchen.}

\enquote{Geht es ihr gut?}, fragte Hagrid ganz aufgeregt.

\enquote{Ja. Sie ist eine der zahmeren. Sie erinnert sich noch an dich. Erst vor wenigen Tagen hat sie von dir gesprochen. Immerhin ist sie ja auf dich geprägt.}

Jetzt war Hagrid glücklich und zog ein Taschentuch, in das er kräftig schnäuzte und dabei das Feuer kurzfristig höher flackern ließ.

Dann sahen alle wieder eine Weile in die Flammen, bis Tabaluga meinte: \enquote{Wollen wir wieder zurück?} Nepomuk nickte und erhob sich. Tabaluga tat es ihm gleich. \enquote{Bringst du uns?}

Frederick schüttelte lächelnd den Kopf, stand auf und zog seinen Zauberstab. \enquote{Gute Reise}, sagte er und schwang ihn.

Dann waren die beiden Drachen verschwunden.

\enquote{Wo sind sie hin?}, fragte Dumbledore.

\enquote{Zurück in Rumänien.}

\enquote{Wie haben Sie das gemacht?}

\enquote{Fernapparition. Eine neue Methode.}

\enquote{Ach Professor. Ich habe Ihnen hier ein Sandwich vom Abendessen mitgebracht.}

\enquote{Danke Hagrid.} Professor Elber nahm das Sandwich und wickelte es aus. Dann biss er hinein und setzte sich wieder. Die drei übrig gebliebenen Zauberer saßen noch eine Weile, bevor sie das Feuer löschten und sich auf den Weg zum Schloss, oder der Jagdhütte machten.

Gerade als Dumbledore das Schlosstor durchquerte, wurde er von Harry abgefangen.

\enquote{Auf ein Wort, Professor.}

\enquote{Gerne Harry. In meinem Büro?}

Harry nickte und so gingen sie in Dumbledores Büro und saßen kurze Zeit später auf einem gemütlichen Sofa.

\enquote{Was ich dich fragen wollte\abs wie ist das mit den magischen Bildern? Ist das nur ein Zauber? Ein aktuelles Abbild an Wissen und Charakter? Oder kann sich das Bild weiter entwickeln? Wie sieht es aus, wenn die gemalte Person noch lebt? Sind beide verbunden? Wie ist es, wenn die Person schon gestorben ist? Besteht dann eine Verbindung zu einem Leben danach? Davon abgesehen, dass sie keine Fragen dazu beantworten können.}

Dumbledore dachte eine Weile nach. \enquote{Das sind ziemlich viele Fragen, Harry. Ich weiß nicht, ob ich sie dir alle beantworten kann. Aber eines kann ich dir mit Gewissheit sagen, die Bilder der Direktoren wurden alle nach deren Tod erschaffen. Diese sind aber auch etwas Besonderes. Dieser Zauber gelingt nur der amtierenden Schulleiter-Person.} Dann grinste er und Harry verstand, dass er Frauen wie Männer damit meinte. \enquote{Ich habe mir diese Fragen auch schon mal gestellt. Ist es nur ein Abbild? Ein Abbild dessen, was wir glauben, dass diese Person ist oder ausmacht? Oder ist es wirklich ein Zauber, der die Essenz einer Person kopiert. \gst Eines kann ich dir jedoch schon sagen: Zumindest die Direktoren in Hogwarts haben eine Verbindung zu ihrem Leben danach. Sie können zwar nichts darüber erzählen, oder etwas über andere Personen erzählen, aber sie können zumindest Nachrichten und Grüße in eine Richtung weitergeben.}

\enquote{Das heißt, ich könnte über einen der Direktoren hier meine Eltern grüßen lassen.}

\enquote{Falls die Person das auch tatsächlich macht. Wir haben keine Garantie, dass sie es auch tut, oder vielleicht nur versucht. Es kann sogar sein, dass sie deine Eltern gar nicht finden, oder nicht zu ihnen durchdringen. Es könnte mehrere Ebenen der Existenz im Leben danach geben. Oder Fremde, die sich nicht miteinander unterhalten dürfen oder können. Wir wissen nichts darüber.}

\enquote{Was würde passieren, wenn ich von mir ein Bild anfertigen lassen würde? Könnte man diesen Effekt, dass beide miteinander verbunden sind, dann nicht ausprobieren?}

\enquote{Hmm.} Dumbledore dachte nach. \enquote{Was ist, wenn ein Zauber auf dem Bild verhindert, dass er die Wahrheit darüber sagen kann? Wenn er nein sagt, obwohl ihr miteinander verbunden seid?}

Das brachte Harry zum Nachdenken. Er griff in die Schale mit den Lakritz-Schnappern und kaute auf einem herum. \enquote{Und wenn das Bild zerstört wird, während man noch lebt? Geht das Wissen des Bildes dann in einem auf?}, fragte er mehr sich selbst und sah zu Dumbledore. Dieser hob nur seine Schultern. \enquote{Danke Albus. \gst Auch wenn du mir nicht wirklich weiterhelfen konntest.}

\enquote{Tut mir leid. Ich wünschte, ich hätte andere Informationen für dich.}

Dann wurde Harry wieder einmal schwarz vor Augen. Er schaffte es noch, sich auf den Boden zu legen und Albus ein Zeichen zu geben. Dann war er wieder weg. Als er wieder erwachte, saß Albus neben ihm auf dem Boden. Fawkes saß auf seiner Stange und pfiff einmal kurz, um auf sich aufmerksam zu machen. Harry lächelte ihn an und ging, nachdem er aufgestanden war, auf ihn zu. Der Phönix ließ sich von ihm streicheln. Dann flatterte er kurz mit seinen Flügeln und marschierte über Harrys Arm an seine Schulter.

\stimme{Nimm mich heute Nacht mit}, hörte Harry in seinen Gedanken.

Harry sah erstaunt zu Fawkes, der nur einmal kurz seinen Kopf senkte. Harry verstand und strich noch einmal durch sein Gefieder. Dann verließ er das Büro und ging durch das Schloss. \enquote{Nehmen wir den kurzen Weg?}, fragte er das Tier. Fawkes knabberte leicht an seinem Ohr, was Harry als Bestätigung auffasste und in den nächsten Aufzug stieg. Kurz vor dem Gryffindor-Gemeinschaftsraum stieg er aus und trat auf das Porträt zu.

\enquote{Passwort?}, fragte die fette Dame im Bild, doch das, welches ihr Harry nannte, akzeptierte sie nicht.

Es war schon spät und keiner reagierte auf Harrys Klopfzeichen. Er sah Fawkes an und grinste ihn an.

\enquote{Dann eben anders}, sagte er und lag kurze Zeit später in Salazars Privaträumen im Bett. Fawkes hielt sich an Harrys Fußende fest und schlug seine Augen zu.

Doch nach kurzem hörte Harry die bekannte Stimme, die er sonst nur Punkt zwölf mittags hörte. \stimme{Komm zu mir.} Noch immer konnte er sie nicht genau orten. Sie schien jedes Mal von woanders zu kommen. Schien zu wandern. Doch andererseits schlief er um Mitternacht meist, oder war zu beschäftigt.

Fawkes öffnete seine Augen und meinte nur: \stimme{Mondbibliothek.} Dann schloss er seine Augen wieder und schlief erneut ein.

\gedanke{Er hat es also auch gehört. Dann hat es definitiv etwas mit dieser Bibliothek zu tun.}

Beim Frühstück am Tag darauf blätterte er in einem von Salazars Büchern und fand den Zauber, der die Lampen im Schloss mit Leben erfüllte. Er übte den Zauber ein paar mal und fragte bei Salazar nach, ob er es denn auch richtig mache. Dieser war stolz auf ihn, dass einer seiner Enkel sich um das Schloss kümmern wollte und half ihm nach Kräften.

Am nächsten Tag hörte Harry eine Unterhaltung zwischen zwei Slytherin ein Jahr unter ihnen mit.

\enquote{Also jetzt weiß ich, was ich mache, wenn ich mal groß bin. Ich werde Drachenhüter.}

\enquote{Aber Drachen sind doch wild. Du hast doch Hagrid zugehört, oder?}

\enquote{Ja, habe ich. Aber ich will trotzdem Drachenhüter werden.}

\enquote{Die nächste Drachenkolonie ist aber in Rumänien. Willst du jeden Abend oder jedes Wochenende nach Hause zu deiner Frau oder Freundin apparieren? Oder zieht sie mit um?}

\enquote{Ne, Amalia mag das nicht. Das habe ich nicht bedacht.}

\enquote{Amalia?}, fragte nun sein Mitschüler. \enquote{In unserem Haus gibt es keine Amalia.}

Jetzt musste Harry grinsen. Er kannte eine Amalia. Es war die Einzige auf Hogwarts. Er erinnerte sich noch genau an die Auswahlzeremonie. Eine kleine Brünette mit ein paar Sommersprossen wurde nach Hufflepuff gesteckt. Er kannte den jungen Adrian nur vom sehen, aber die beiden anderen hatten kaum Verständnis für ihn. Im Gegensatz zu dem Jungen, der keinesfalls an die Reinheit des Blutes glaubt. Er dachte nach, ob er ihm helfen konnte.

\enquote{Welche Amalia?}, fragte einer der beiden Begleiter nach. \enquote{Es gibt hier keine Amalia!}

\enquote{Wie geht es der kleinen Beauxbaton?}, fragte Harry.

Geistesgegenwärtig sagte Adrian: \enquote{Sehr gut. Leider sehen wir uns nicht so häufig.}

Harry nickte und lief weiter.

\enquote{Woher kennst du sie denn, Potter?}, fragte der größere der beiden um Adrian stehenden.

\enquote{Persönlich nicht, aber ihre Schwester hatte über sie\abs Warum erzähle ich euch das eigentlich?} Dann lief er weiter.

Nach dem Essen fing ihn Adrian ab. \enquote{Danke Potter, Harry. Äh.}

\enquote{Schon gut. Adrian, richtig?}

Dieser nickte. \enquote{Wieso?}, fragte er nach.

\enquote{Ich glaube kaum, dass deine Mitschüler Verständnis dafür hätten, dass ein Slytherin eine Hufflepuff zur Freundin hat. Zumindest ein großer Teil.}

Adrian wurde bleich. \enquote{Woher?}

\enquote{Es gibt nur eine Amalia. Und die ist in Hufflepuff. \gst Gratuliere. Du hast einen\abs Ihr habt einen guten Geschmack.}

\enquote{Danke.} Nach einer Weile fragte er nach. \enquote{Sag mal. Du hast doch, habe ich gehört, letztes Jahr versucht ein paar Mitschüler zu unterrichten, als wir Umbridge hatten.}

\enquote{Ja}, antwortete er vorsichtig.

\enquote{Machst du das wieder? Ich meine, ich würde mich gerne verteidigen können.}

Harry sah ihn eine Weile an. Adrian wurde mittlerweile leicht mulmig. Dann fragte er: \enquote{Würdest du dich denn in dieser Gruppe überhaupt wohlfühlen? Es waren damals ja keine Slytherin dabei. Du wärst dann quasi der Einzige.}

Adrian dachte eine Weile nach. \enquote{Na ja. Ich schätze, anfangs wird es schwer werden. Es würde sicherlich dauern, bis mich die anderen akzeptieren werden.}

\enquote{Hmm. Nehmen wir mal an, dass es diese Gruppe, immer noch gibt. Dann müsste ich zuerst mit denen reden. Ich melde mich in drei Tagen. Ist das in Ordnung?}

Adrian dachte kurz nach. \enquote{In Ordnung. Bis dann. \gst Warte mal. Was ist, wenn noch mehr aus meinem Haus mitmachen möchten?}

\enquote{Du meinst, ein Slytherin würde sich von einem Gryffindor etwas sagen lassen? Dich mal ausgenommen!} Dann ging Harry und lies einen nachdenklichen Adrian stehen.

\trenn

\enquote{Du siehst so bedrückt aus}, meinte Marietta, als sie wieder im Raum der Wünsche übten.

\enquote{Ich denke nach}, antwortete Harry.

\enquote{Sagst du auch, worüber?}

\enquote{Über einen möglichen Neuzugang zu unserer Gruppe.}

\enquote{Vertrauenswürdig?}

\enquote{Ich denke schon. Er ist mir jedenfalls nicht negativ aufgefallen. Ich bin mir nur unsicher, ob er\abs}

\enquote{In die Gruppe passt?}, fragte Ron, der gerade eine Pause machte und sich neben Harry setzte.

Kurz darauf quetschte sich Ginny zwischen Marietta und Harry. Sie konnte es nicht ertragen, dass sich jemand an \accentuate{ihren} Harry heranzumachen versuchte.

\enquote{Wer ist es denn?}

\enquote{Adrian Montague.}

\enquote{Kenne ich nicht}, sagte Ginny. \enquote{Aus unserem Haus ist der nicht.}

\enquote{Aus meinem auch nicht}, meinte Marietta.

\enquote{Vielleicht Hufflepuff?}, überlegte Ron.

\enquote{Nein}, sagte Hannah Abbott, die in der Nähe stand und mit halbem Ohr mitgehört hatte.

\enquote{Moment mal}, meinte Ron. \enquote{Er ist weder in Gryffindor, noch in Ravenclaw oder Hufflepuff. Wie geht das denn?}

Marietta und Ginny mussten lachen.

\enquote{Man, Ron}, meinte Ginny. \enquote{Als ob Hogwarts nur drei Häuser hätte. Wenn er in keinem der drei Häuser ist, dann kommt er wahrscheinlich aus Slytherin.}

Marietta und Ginny lachten weiter. Dann plötzlich verstummten sie und Harry musste sich ein Grinsen verkneifen.

Ginny sah Harry an. \enquote{Entweder willst du uns alle verarschen, oder du überlegst dir wirklich, dass wir einen\abs Sag, dass das nicht wahr ist.} Jetzt verstummte die gesamte Gruppe und es war plötzlich still. \enquote{Sag mir, dass du nicht überlegst einen Slytherin bei uns aufzunehmen.}

\enquote{Doch, genau das}, antwortete Harry. \enquote{Ich bin mir nur nicht sicher, wie er aufgenommen würde. Ich halte ihn für einen guten jungen Zauberer.}

\enquote{Spinnst du jetzt völlig? Einen Slytherin?}, antwortete Ron erregt.

\enquote{Nein, ich spinne nicht}, gab Harry laut zurück. \enquote{Aber du solltest dir mal überlegen, warum es immer noch diesen Hass zwischen den Häusern gibt, wenn du einer derjenigen bist, die diesen Hass ständig anstacheln.} Dann sah er auf und in die erschrockenen Gesichter seiner Mitschüler. \enquote{Wenn ihr ihn nicht wollt und diesen Hass der Häuser weiter aufrechterhalten wollt, von mir aus. Ich werde ihn vermutlich unterrichten. Auch wenn es dann Einzelunterricht sein wird. Ich habe von dieser Hass-Sache endgültig die Schnauze voll.}

Jetzt war Harry wieder wohler. Seit zwei Tagen hatte sich dieser Frust wegen der Ablehnung eines Slytherin angestaut. Nicht zu vergessen, der monatelange Kampf gegen seine Windmühlen. Manchmal kam er sich wie Don Quijote vor.

\enquote{Wer ist es denn?}, fragte Susan Bones nach.

\enquote{Adrian Montague}, sagte Harry erneut.

Susan überlegte eine Weile. \enquote{Von mir aus}, sagte sie und begann sich wieder ihren Übungen zu widmen.

Jetzt entbrannte eine heftige Diskussion darüber, ob man Slytherin aufnehmen durfte oder nicht. Nach zehn Minuten stand Harry auf und verließ den Raum. Er trottete niedergeschlagen wegen des Streits und erleichtert darüber, dass er seinem Frust Luft gemacht hatte, durch das Schloss.

Nach einer Weile lief Professor Flitwick neben ihm her. \enquote{Probleme?}, fragte er.

Harry nickte. \enquote{Ja. Wir haben gerade einen Streit.}

\enquote{Wer ist wir?}

\enquote{Die DA. Ich würde gerne jemanden neues hinzunehmen, aber ein großer Teil der DA nicht. Vermutlich.}

\enquote{Sagen Sie mir, wer?}

\enquote{Adrian Montague.} Dann erst war Harry wieder klar bei Verstand. Er sah sich um, entdeckte aber niemanden.

\enquote{Hier unten}, gluckste der kleine Zauberer.

Harry sah nach unten und lief rot an. \enquote{Entschuldigen Sie Professor Flitwick. Das ist normalerweise nicht meine Art.} Er lief ein paar Schritte weiter zu einer Bank und setzte sich.

Professor Flitwick kam zu ihm und setzte sich neben ihn. \enquote{Wissen Sie, es gibt Zeiten, da muss man sich dem Druck der Gemeinschaft beugen.} Keiner der beiden bekam mit, dass sich Teile der DA näherten, aber außer Sichtweite blieben. \enquote{Und es gibt Zeiten, da muss man gegen den Strom schwimmen. Hogwarts ist so ein Ort. Wir haben immer wieder bewiesen, dass man manchmal mutig genug sein muss und es der Gemeinschaft beweisen muss, dass sie falsch lag. Wenn die Aktion nach hinten losgeht, hat man zwar die Häme der anderen, weiß aber, dass man falsch lag. Und die anderen wissen zumindest, dass sie definitiv richtig lagen.} Dann überlegte er eine Weile. \enquote{Werden Sie ihn trotzdem unterrichten?}

\enquote{Ich denke schon.}

\enquote{Dann werde ich Ihnen helfen!}

\enquote{Wie?}

\enquote{Ich werde Ihr Co-Lehrer. Wir können gemeinsam etwas vorführen. Eins-zu-Eins Duelle, Zwei-zu-Eins Duelle und wenn Pomona mitmacht, dann können wir auch Drei-zu-Eins Duelle machen.}

Harry gefiel die Idee. \enquote{Ich warte noch die Entscheidung meiner\abs der DA ab. Aber ich komme auf Ihr Angebot zurück. Gilt das nur im Falle des Einzelunterrichtes?}

\enquote{Wenn Sie mir anbieten, die DA mal zu besuchen, nehme ich gerne an.} Harry dachte eine Weile nach. \enquote{Kommen Sie mit? Ich gebe eine Tasse Tee aus.}

\enquote{Gerne Professor.} Die beiden standen auf und liefen den Gang entlang. \enquote{In Ihrem Büro?}

\enquote{Nein, wir gehen ins Lehrerzimmer.}

Die Gruppe der DA kam um die Ecke und sahen den beiden zu, wie sie schwatzend Richtung Lehrerzimmer im Inneren des Schlosses verschwanden.

\enquote{Meint ihr, er macht es wahr?}, wollte Zacharias Smith wissen.

\enquote{Schwer zu sagen. Er könnte es auch nur so gesagt haben, um uns umzustimmen. Oder, weil er richtig sauer war}, meinte Cho.

\enquote{Was meint ihr dazu?}, fragte Katie Bell an Ron und Hermine gewandt. \enquote{Ihr kennt ihn länger und besser als wir.}

\enquote{Ich denke, wenn er davon überzeugt ist, dass er ungefährlich ist, dann wird er es auch tun. Da ist er dickköpfig.}

\enquote{Auch, wenn wir dagegen sind?}

\enquote{Dann wird er ihn, sofern der junge Mann will, mitnehmen, oder, wie du gehört hast, ihn privat unterrichten.}

\enquote{Das ist doch bestimmt eine Falle, damit Du-weißt-schon-wer erfährt, was Harry alles kann}, meinte Zacharias. \enquote{Immerhin ist er ein Slytherin.}

\enquote{Ich glaube, das hat Harry gemeint}, warf Katie ein.

\enquote{Was hat er gemeint?}, fragte Ron nach.

\enquote{Dass wir ihn ablehnen, nur weil er aus dem falschen Haus stammt.}

\enquote{Spinnst du?}

\enquote{Genau das ist es! Du bist auch einer von denen, die ihren Kindern den Hass auf Slytherins übertragen. Und die dann auf deren Kindern. Und das geht über Generationen so weiter. Und nicht nur auf unserer Seite.}

\enquote{Bist du jetzt komplett durchgedreht?}

\enquote{Nein, aber du, Ron. Überleg mal, wie sich die Slytherins wohl fühlen.}

\enquote{Das interessiert mich nicht.}

\enquote{Genau darum geht es. Sie werden einfach in eine Ecke gedrängt, obwohl sie nichts dafür können. Der Hut verteilt sie. Und jetzt überlege mal, wie sich Harry fühlen würde, wenn ihn der Hut nach Slytherin gesteckt hätte. Denn das hat er ihm ja schließlich angeboten. Oder hast du da auch nicht aufgepasst?} Ron schwieg, wie der Rest der Gruppe. \enquote{Dann hätten wir ihn alle gehasst und für Vold\aabs Du-weißt-schon’s Liebling gehalten und ihn genau so abgelehnt. Er hätte ihn trotzdem zu bekämpfen versucht, wäre aber vermutlich schon an seinen Hauskameraden gescheitert. Harry hat recht! Wird Zeit, dass sich da was ändert! Ich habe jetzt Hunger.} Dann ging sie. Sie drehte sich noch einmal um und sagte: \enquote{Falls du weiter reden oder streiten willst, ich bin in der Großen Halle.}

Es dauert eine Weile, bis sich die Schüler gefasst hatten. Nach und nach löste sich die Gruppe auf und ging.

\trenn

Harry und Professor Flitwick kamen am Lehrerzimmer an und der kleine Koboldmischling öffnete die Tür. Im Zimmer war nur noch Professor Sinistra, die am Fenster stand und hinaussah. Als die beiden das Zimmer betraten, drehte sie sich um. \enquote{Hallo Filius.}

\enquote{Hallo Aurora.}

\enquote{Hat Mister Potter was ausgefressen?}

\enquote{Nein.}

\enquote{Nicht? Na dann}, meinte sie und drehte sich wieder um.

Flitwick bot Harry einen Sitzplatz an. Dieser setzte sich. Kurz darauf hatten beide ihren Tee und etwas Gebäck vor sich stehen.

Harry sah eine Weile stumm auf seine Tasse. Dann fing er an zu erzählen. \enquote{Ich mag nicht mehr! Dieser Hass auf das andere Haus. Ich meine, ich habe nichts gegen die Hausmeisterschaft und die Hauspunkte. Ein bisschen Ansporn sollte schon sein. Oder beim Quidditch den Pokal. Aber dass man das Haus Slytherin so ausgrenzt, wegen etwas, was so viele Jahre vergangen ist\abs Ich meine, ich verstehe mich mit Draco mittlerweile so, dass wir uns nicht mehr richtig anfeinden. Ich mag ihn zwar immer noch nicht besonders, oder würde so weit gehen und sagen, dass ich ihn respektiere, aber\abs Ich denke, ich bin ihm gegenüber einfach neutral geworden.}

\enquote{Er aber anscheinend nicht mit ihnen. Oder es ist ihm entgangen, dass\abs} Er brach ab, da ihn Harry eigenartig ansah.

\enquote{Das bleibt jetzt aber unter uns, verstehen wir uns?} Professor Flitwick musste schlucken, da Harry in dem Moment eine Autorität ausstrahlte, wie er sie sonst nur von Dumbledore kannte. Daher nickte er nur einfach. Professor Sinistra, die die beiden ansah, nickte auf Harrys Blick hin auch nur. \enquote{Wir streiten uns nur noch, wenn wir uns öffentlich treffen. Falls wir uns alleine, oder mit den entsprechenden Leuten um uns herum begegnen, dann belassen wir es dabei mit dem Kopf zu nicken und weiterzugehen. Wir nutzen nicht mehr jede Gelegenheit, um uns die Hölle heiß zu machen.}

Dann wurde es still im Zimmer. Professor Sinistra setzte sich zu ihnen und nach einer Weile kam Dumbledore herein. \enquote{Oh, Harry, hast du was ausgef\aabs}

Harrys eigenartiger, beleidigter Blick ließ ihn verstummen. Dumbledore wusste selbst nicht, was gerade mit ihm passierte. Er spürte so etwas wie Peinlichkeit. Harry hingegen war über seine Reaktion mehr als überrascht. Umso überraschter war er über die Reaktion seines Schulleiters, der sich abwandte, sich eine Tasse voll Tee aus einer Kanne goss und sich dann eine Weile an ein Fenster stellte und hinaussah, bevor er sich zu ihnen setzte. Der Direktor wusste in diesem Moment selbst nicht, warum er Harry nicht darauf hinwies, dass er auf seinem Stuhl saß und er sich doch bitte einen anderen nehmen solle.

Als dann nach einer Weile noch Professor McGonagall den Raum betrat, stutze und sich ebenfalls eine Tasse Tee holte, stand Harry auf, bedankte sich bei Professor Flitwick für den Tee, verabschiedete sich von den Lehrern und verließ das Zimmer.

\enquote{Ich bin mehr als überrascht, Albus, dass Sie Mister Potter nicht von Ihrem Stuhl verscheucht haben}, warf Professor Flitwick in den Raum.

\enquote{Ich auch Filius, ich auch}, sagte Dumbledore gedankenversunken und starrte immer noch die Tür an, durch die eben Harry gegangen war. Dann sah er Filius wieder an. \enquote{In dem Moment fand ich es einfach unpassend. Ich hatte das Gefühl, ich sitze als Schüler vor meinen Direktor.} Dann lächelte er leicht. \enquote{Es war irgendwie ein tolles Gefühl.}

\enquote{Ihnen gefällt das? Wenn man vor dem Schulleiter als Schüler sitzt?}

\enquote{Das meinte ich nicht. Es war schön, das mal wieder erfahren zu haben. So wird man als Lehrer wieder daran erinnert, welche Verantwortung man gegenüber den Schülern hat. Es war beängstigend, solange es dauerte, hat mich aber wieder daran erinnert, dass sich unsere Schüler manchmal ebenso fühlen. Wir sollten diese Gefühle nicht noch verstärken.}

\enquote{Hat Sie Potter also in die Knie gezwungen}, kam sarkastisch aus einer Ecke des Lehrerzimmers. Severus Snape, den bisher keiner beachtet hatte, kam zu ihnen und setzte sich.

\enquote{Wie Sie sich gefühlt haben, brauche ich Sie ja wohl nicht zu fragen}, kam leicht bissig von Aurora Sinistra.

Snape schwieg, doch in seinem innersten fühlte er genau so.

\enquote{Mister Potter scheint einige interessante Fähigkeiten zu entwickeln, nicht wahr Albus?}, kam jetzt von McGonagall.

\enquote{In der Tat, das tut er.}

\enquote{Anscheinend wird er gut ausgebildet.}

Sinistra verschluckte sich fast an ihrem Tee.

Flitwick sah sie daraufhin lange an. Dann meinte er: \enquote{Hat man Ihnen etwa gesagt, dass Sie nicht darüber reden sollen?}

Sie nickte nur.

\enquote{Mir aber auch}, meinte McGonagall und Dumbledore nickte nur.

\enquote{Dann hat man jedem von Ihnen und wahrscheinlich dem gesamten Lehrerkollegium dasselbe gesagt}, gab nur Snape zum Besten. \enquote{Ja, er ist halt doch ein Schlitzohr.}

\enquote{Wer?}, fragte Sinistra nach.

\enquote{Wenn Sie das noch immer nicht wissen, dann werde ich es Ihnen garantiert nicht sagen.} Dann stand Snape auf und verließ das Lehrerzimmer Richtung Kerker.

Sinistra war eine der wenigen, vielleicht auch die einzige, die Severus manchmal verstand. Daher fühlte sie sich auch nicht beleidigt. Sie dachte nach. Der Einzige, von dem sie wusste, dass er Harry unterrichtete, war ihr Kollege Frederick. Dann wusste sie, was Severus meinte.

Harry war unterdessen an seinem Ziel angekommen. Auf seinem Weg hatte er noch eine der Lampen ersetzt, die einer Auffüllung bedurften. Er setzte sich auf das grüne gemütliche Sofa. Kurz darauf rutschte er darauf herum und legte sich hin. Wieder änderte er seine Position. Dann stand er auf und lief zu einem Sessel, in den er sich setzte. Er drehte sich wieder und lag zwischen den Armlehnen quer auf dem Sessel. Er legte seinen Kopf gegen die Rücklehne und schloss die Augen.

\enquote{Er ist irgendwie aufgewühlt, Salazar}, sagte seine Frau Agatha leise zu ihm. Harry konnte sie nicht verstehen.

\enquote{Er hat sich mit seiner Gruppe gestritten.}

\enquote{Die DA, von der du mir erzählt hast?}

\enquote{Ja.}

\enquote{Weswegen?}

\enquote{Er will einen Schüler aus meinem Haus in seiner Gruppe unterrichten. Diese hat wohl was dagegen.}

\enquote{Was wird er tun?}

\enquote{Vermutlich wird er ihn trotzdem unterrichten.}

\enquote{Also so ein Dickkopf wie du?}

\enquote{Sieht so aus. Lass uns auch schlafen.}

Am nächsten Morgen erwachte Harry gerädert im Sessel und schaute verschlafen auf das Bild. Er hatte von einem rosa Regenschirm geträumt. Wieder einmal. Er rieb sich die Hände vorm Gesicht und sah hoch zu Salazar und seiner Frau Agatha. \enquote{Guten Morgen ihr zwei.}

Salazar und Agatha grüßten zurück.

\enquote{Du siehst müde aus}, sagte Agatha und sah ihn mitleidig an.

\enquote{Ja, das ist gestern nicht so gut gelaufen mit meiner Gruppe. Und außerdem hatte ich wieder diesen Traum. Ich weiß einfach nicht, wie ich ihm helfen soll.}

\enquote{Deinem Traum?}

\enquote{Nein, Hagrid. Einem meiner Freunde.}

\enquote{Was hat er denn für ein Problem?}

\enquote{Er darf nicht zaubern, weil man ihn vor fünfzig Jahren verdächtigt hat ein gefährliches Monster in der Schule frei zu setzen. Daraufhin hat man ihn hinausgeworfen. Er ist aber noch hier, als Wildhüter und nun auch als Lehrer.}

\enquote{Und wie möchtest du ihm helfen?}

\enquote{Ich möchte seine Unschuld beweisen. Das Problem ist, ich hatte einen Beweis. Das ist mir erst dieses Jahr klar geworden. Gedanken in einem Tagebuch.}

\enquote{Wie können Gedanken in einem Tagebuch stecken? Du meinst geschrieben?}

\enquote{Nein. Das Tagebuch war ein Horkrux. Ich habe ihn in meinem zweiten Jahr vernichtet. Davon habe ich euch erzählt.} Beide nickten. \enquote{Das war der einzige Beweis. Ich habe sonst keine andere Chance. Außer es gibt einen Weg in der Zeit zurückzureisen und die Gedanken aus dem Tagebuch zu extrahieren, damit man sie in einem Denkarium ansehen kann. Natürlich, ohne dass es auffällt. Besonders dem Seelenteil. Dann muss man wieder zurück in die Gegenwart. Aber außer einem Zeitumkehrer kenne ich nichts. Und die hat das Ministerium allesamt unter Verschluss.} Dann sah Harry zu den beiden hoch. \enquote{Außer ihr wisst etwas.}

Salazar setzte ein nachdenkliches Gesicht auf. Er schritt aus dem Bild und kehrte kurz danach wieder zurück, durchlief es und verschwand auf der anderen Seite, von der er nach einigen Sekunden wieder kam. Das passierte mehrere male hintereinander. Er lief praktisch im Bild umher und darüber hinaus. Dann blieb er stehen und sah Harry an. \enquote{Der Zeitumkehrer wäre nicht das Problem. Ich habe einen in meinen Beständen. Irgendwo. Das Problem, oder eher die Probleme, sind vielmehr die lange Zeit, die überbrückt werden muss. Hin und zurück, sowie die Maßnahme der Extraktion von Gedanken bei einem sich wehrenden Horkrux. Dann musst du ihn auch noch mit einem Gedächtniszauber belegen.}

Harry stand nun ebenfalls auf und lief durch den Raum. Ein Elf brachte Frühstück und verschwand wieder. Marcel kam zu ihm und wollte auf seinen Arm. Er hatte schon länger keine menschliche Wärme mehr gespürt. Harry nahm ihn auf und trug ihn herum.

\enquote{Du hast doch einen direkteren Zugriff auf diese Informationen}, sagte Salazar.

\enquote{Welche?}

\enquote{Voldemort selbst. Du hast doch eine direkte Verbindung zu ihm. Durchforste sein Gedächtnis und kopiere dir die Erinnerungen. Man wird feststellen können, dass sie echt sind. Natürlich musst du vorsichtig sein. Voldemort ist extrem gut. Mache einen Schritt nach dem anderen. Wenn du bereit bist und das Risiko eingehen möchtest, werde ich dir helfen.}

Harry war dankbar und verabschiedete sich nach dem Frühstück von beiden. Darüber musste er erst einmal nachdenken.

\trenn

\enquote{Haben Sie Zeit?}, wurde Harry von Elber gefragt, der gerade mit Ron und Hermine vom Essen kam.

\enquote{Ja}, antwortete er. Und an Ron und Hermine gewandt fragte er: \enquote{Ihr wisst euch sicherlich die Zeit alleine zu vertreiben.} Dabei lächelte er beide süffisant an. Harry folgte seinem Lehrer hinaus auf das freie Feld.

Hinter einer Bäume- und Büsche-Kombination versteckt und vom Schloss aus kaum einsehbar, stellten sich beide gegenüber auf und zogen ihre Zauberstäbe. Beide nickten einander zu und der Tanz der Kontrahenten begann. Zauber um Zauber flog umher und beleuchtete die aufkommende Abenddämmerung.

Während Harry und Professor Elber ihr Duell vollzogen, genossen Ron und Hermine die Zweisamkeit. Nur spärlich bekleidet lagen sie eng aneinander gekuschelt in ihrem Raum. Immer wieder fuhren sie mit ihren Händen am Körper des anderen auf und ab. Mit unzähligen Küssen bedeckten sie sich, bis sie irgendwann einfach so entspannt waren, dass sie einnickten.

Etwa vierzig Minuten später näherte sich eine Gruppe aus Lehrern und Schülern den eigenartigen Lichtern, die sich auf dem Gelände zeigten. Harry warf gerade einen Zauber auf seinen Lehrer, als die Gruppe in ihr Sichtfeld schwenkte. Elber knickte an seinen Knien nach hinten ab, blieb aber stehen. Der Zauber verfehlte ihn knapp über seinem Bauch. Er sah nach oben und bemerkte, dass der Zauber direkt auf die Gruppe treffen würde. Er warf einen Zauber mit solch einer Geschwindigkeit hinterher, dass dieser Harrys Zauber überholte und einen Schild erschuf, um die Zaungäste zu schützen.

Gerade als er wieder Richtung Harry sah und sich aufrichten wollte, um einen erneuten Angriff zu starten, erwischte ihn der nächste Zauber Harrys frontal, schleuderte ihn durch die Luft und warf ihn gegen das Kraftfeld. Professor Elber verlor seinen Stab und fiel bewusstlos zu Boden. Professor McGonagall wollte schon zu ihm treten, doch das Kraftfeld hinderte sie. Harry rannte auch auf ihn zu, kam aber an ihn heran, da er innerhalb des Feldes stand.

Während Professor McGonagall versuchte das Feld zu neutralisieren, untersuchte Harry seinen Lehrer, wie er es von Madame Pomfrey gelernt hatte. \gedanke{Nur Bewusstlos und Kopfschmerzen}, dachte er sich.

\enquote{Würden Sie bitte das Feld aufheben, Mister Potter?}, verlangte Professor McGonagall, die es bis dahin nicht geschafft hatte das Feld aufzuheben.

\enquote{Wie stellen Sie sich das vor, Professor?} Er nahm seinen Stab hoch. \enquote{Soll ich ihn einfach\abs}, er bewegte ihn auf das Feld zu, \enquote{\aabs auf das Feld halten\abs} Er berührte mit seiner Spitze das Feld, \enquote{\aabs und es löst sich\abs}, das Feld verschwand, \enquote{\aabs auf.} Er stutzte. \enquote{Wie ging das jetzt?}, fragte er mehr sich selbst.

Professor McGonagall besah sich ihren Kollegen und sah dann vorwurfsvoll auf Harry. \enquote{Was haben Sie sich dabei gedacht, Mister Potter.}

\enquote{Was ich mir\abs? Na hören Sie mal, wir haben hier trainiert, bis Sie uns gestört haben. Professor Elber hat das Feld nur zu Ihrem Schutz erschaffen. Leider habe ich meinen zweiten Zauber nicht schnell genug zurücknehmen können.}

\enquote{Bin ich jetzt etwa Schuld?}, echauffierte sich Professor McGonagall.

\enquote{Hilft es denn?}, fragte Harry genervt nach. Dann beschwor er eine Trage herauf, ließ seinen Lehrer darauf schweben und machte sich auf den Weg zurück zum Schloss.

Im Krankenflügel angekommen legte er ihn auf ein freies Bett.

Madame Pomfrey kam heran und untersuchte ihn. \enquote{Ihre Meinung, Mister Potter.}

\enquote{Bewusstlos und Kopfschmerzen}, antwortete er knapp.

\enquote{Wieso so gereizt?}, fragte sie auf dem Rückweg zur Apotheke, wo sie einige Tränke holte.

Als sie wieder da war, antwortete Harry: \enquote{Wir hatten ein kleines Duell. Zur Übung. Ich warf gerade einen Zauber auf ihn und setzte noch einen nach, als Zaungäste auftauchten. Professor Elber bemerkte sie rechtzeitig und warf einen Zauber der meinen überholte und unsere Besucher schützte. Leider konnte ich meinen zweiten Zauber nicht mehr aufhalten und Professor Elber war nicht schnell genug. Er erwischte ihn Frontal und schleuderte ihn gegen das Feld. Dann hat mir Professor McGonagall auch noch Vorwürfe gemacht. Besser gesagt, machen wollen. Ich bin dann etwas ungehalten gewesen und habe wohl leicht über reagiert}

\enquote{Leicht überreagiert?}, fragte Professor McGonagall, die gerade hereinkam.

\enquote{Das wäre nicht passiert, wenn Sie mich nicht gleich so angepflaumt hätten}, antwortete er, ohne zu ihr zu sehen. Dann sagte er zu Madame Pomfrey: \enquote{Sehen Sie was ich meine?}

Diese nickte nur und meinte: \enquote{Warten Sie kurz. Sie bekommen von mir etwas. Das lindert Ihren Reizzustand. Außerdem ist es beruhigend und schlaffördernd.} Sie kam mit zwei Kelchen und gab einen davon Harry, den anderen Professor McGonagall.

\enquote{Wieso gibst du mir einen?}

\enquote{Runter damit, Minerva}, befahl sie. \enquote{Du hast den genauso verdient.}

Widerstandslos tranken beide die Flüssigkeit und legten sich danach in je ein Bett. Zufrieden kümmerte sich Madame Pomfrey um ihren Patienten und ging dann. Sie ließ ihre Gäste alleine und machte das Licht aus.

\trenn

\enquote{Dürfen Sie sich denn schon wieder duellieren?}, fragte Harry, als er mit Professor Elber erneut über die Ländereien lief.

Dieser schüttelte seinen Kopf. \enquote{Nein, aber wir werden heute nichts dergleichen machen. Warten Sie es ab.}

Sie liefen Richtung Hagrid, dann an seiner Hütte vorbei und in Richtung einer kleinen Baumgruppe. Davor waren etwa zwanzig Baumstümpfe zu sehen. Hagrid wartete bereits.

\enquote{Schön, dasser da seid}, sagte er.

\enquote{Gerne, Hagrid}, dann zu Harry gewandt: \enquote{Diese Baumstümpfe müssen entfernt werden. Die Hälfte mit Zauberstab, die andere Hälfte ohne. Und jeder Stumpf muss auf andere Art und Weise entfernt werden.}

Harrys Kiefer klappte auf. \enquote{Wie?}

\enquote{Egal wie. Wichtig ist nur, immer auf andere Art.}

\enquote{Sie meinen, ich soll die Baumstümpfe entfernen?} Professor Elber nickte. \enquote{Was bringt mir das?}

\enquote{Das liegt an Ihnen, es herauszufinden. Anderenfalls sage ich es Ihnen in etwa drei Wochen.}

Harry blieb erst einmal stumm stehen und sah abwechselnd zwischen Hagrid, seinem Lehrer und den Baumstümpfen hin und her. Dann schließlich begann er. Er besah sich einen der Baumstümpfe und versuchte es mit einem \spruch{Wingardium Leviosa}. Er strengte sich an, bis er eine Reaktion des Stumpfes spürte und sah. Er spannte seine Arm- und Handmuskeln immer mehr an, sodass sie zu schmerzen begannen. Gerade als der Baumstumpf aus der Erde flog, bekam er einen Krampf in seiner Hand und ließ seinen Zauberstab fallen, um seine Hand zu halten.

\enquote{Ah!}, schrie er. Seine linke Hand umklammerte sein rechtes Handgelenk und er hoffte, so den Schmerz zu vermindern. Als er die Hand nach einigen Sekunden wieder losließ, bemerkte er eine kleine Schwellung.

Professor Elber kam auf ihn zu und fuhr mit seinem Zauberstab über sein Handgelenk in immer kürzer werdenden pendelnden Bewegungen, bis er ganz still stand und ihn nach oben wegzog.

\enquote{Danke!}, keuchte er. \enquote{Woher kennen Sie so etwas?}

\enquote{Unsere Medi-Hexe.}

\enquote{Wie?}

\enquote{Ich war bei ihr und habe ihr geschildert, was wir heute vorhaben, dann hat sie mir gezeigt, mit welchen zu erwartenden Verletzungen ich zu rechnen hätte. Die passenden Zauber hat sie mir gleich verraten. Sie sollten trotzdem am Ende der Stunde zu ihr gehen. Machen Sie weiter.}

\enquote{Gleich?}

\enquote{Es können auch ein paar Minuten Pause dazwischen sein.}

\enquote{Woll’n se was z’ trink’n?}

\enquote{Gerne Hagrid.}

\enquote{Dann komm’n se.}

Damit verschwanden beide in Hagrids Hütte und ließen Harry alleine. Als dieser die Hälfte der Stümpfe entfernt hatte, versuchte er es ohne Zauberstab. Doch er hatte keinen großen Erfolg. Also behalf er sich mit dem überwiegenden Teil der restlichen mit seinem Zauberstab. Als er fertig war, klopfte er an die Tür und wurde kurz darauf von Hagrid reingebeten. Nach einer Tasse Tee und etwas Geplauder verließen die beiden Hagrid und gingen zurück zum Schloss.

\enquote{Alles gut verlaufen?}, wurde Harry gefragt.

\enquote{Ja}, gab er knapp zurück.

\chapter{Aus grauer Vorzeit}


\kapitelvorwort{Denn tausend Jahre sind vor dir wie der Tag,\\ der gestern vergangen ist,\\ und wie eine Nachtwache.\\(Rulaman)}


Friedward ging mit seiner Frau Persope am Rande des Sees entlang. Sie war hoch-schwanger und ging entsprechend langsam.

\enquote{Wir sollten langsam zurück zum Schloss gehen, die Wehen kommen in kürzeren Abständen.}

Kurz darauf bekam sie einen Hustenanfall und ihr Mann stützte sie und klopfte ihr sachte auf den Rücken. Nachdem sie sich wieder beruhigt hatte, drehten die beiden um und gingen zum Schloss zurück. Einem kleinen Schmetterling, der sich auf einem Busch neben dem Weg niedergelassen hatte, flüsterte er etwas zu, worauf dieser Richtung Schloss davonflatterte. Im Schloss angekommen warteten bereits ein paar Hauselfen und zwei Hebammen, die schon vor ein paar Tagen eingetroffen waren, um die Geburt zu begleiten.

Für diese Zeit außergewöhnlich, begleitete ihr Mann sie in das Zimmer, in dem sie gebären sollte. Er stand an ihrem Kopfende, hielt ihre Hand und lächelte sie an. Beruhigend sprach er mit ihr und wischte immer wieder mit einem kühlen feuchten Tuch über ihre Stirn.

Während die Elfen im Hintergrund ihre Arbeit verrichteten und frisches heißes Wasser, sowie sterile Tücher brachten, kontrollierte eine Hebamme die Öffnung des Muttermundes und die andere überwachte die ganze Geburt und assistierte.

\begin{rueckblick}
\enquote{Sie werden Vierlinge bekommen}, sagte der Heiler im St. Helens. Persope war erstaunt und erzählte diese Neuigkeit ihrem Ehemann, als sie nach Hause kam. Dieser machte große Augen und starrte sie erst einmal eine Minute an.
\end{rueckblick}

\enquote{Pressen}, sagte die Hebamme und Persope presste und presste. Nach zehn Minuten war das erste Kind geboren. Ein Mädchen. Es wurde sofort einem der Elfen gegeben, der das Kind wusch und in warme Tücher bettete. Nach weiteren fünf Minuten kam der erste der zwei Jungen, der Zweite folgte eine Minute später. Es dauerte noch fünf Minuten, dann war auch das zweite Mädchen geboren. Jedes Kind wurde ebenfalls einem Elfen zum Waschen und Kleiden gegeben.

\enquote{Zwei Jungen und zwei Mädchen}, sagte die Hebamme. \enquote{Wie sollen sie heißen?}

\enquote{Die Erstgeborene soll \accentuate{Rowena} heißen}, sagte Persope.

\enquote{Die beiden Jungs \accentuate{Salazar} und \accentuate{Godric}}, sagte Friedward.

\enquote{Und die Jüngste: \accentuate{Helga}.}

Die Hebamme sprach einen einfachen Zauber über die Babys aus, die jeweils von einem der Elfen gehalten wurden und immer noch in der Reihenfolge der Geburt von der Mutter weg standen. Jetzt konnte man die kleinen nicht mehr verwechseln und die Elfen kamen einer nach dem anderen an das Bett der jungen Mutter, stiegen auf eine Empore, die dort für die Elfen stand, und gaben die Neugeborenen der Mutter. Diese nahm ihre vier Kinder nacheinander in den Arm und drückte sie.

Doch sie bekam wieder einen Hustenanfall. Die Elfen ließen die Kinder nach oben schweben und nahmen sie wieder in den Arm. Doch der Hustenanfall ließ nicht nach. Im Gegenteil, er wurde heftiger. Dann warf sie ihren Oberkörper nach hinten auf das Laken zurück und bekam keine Luft mehr. Sie rang nach Luft. Keiner konnte ihr helfen. Die Hebammen hatten weder eine Erklärung, noch einen Zauber dagegen. Ihr Ehemann, sowie die anwesenden Elfen versuchten alles. Doch auch sie waren erfolglos. Persope war gestorben.

Eine Stunde später saß er alleine in seinem Schloss. Seine Frau lag noch immer in dem Zimmer, in dem sie ihre Kinder geboren hatte. Die Hebammen, sowie die Elfen waren verschwunden. Einzig und alleine ein alter Elf, der der Familie schon viele Jahre diente und den seine Frau in die Ehe mitgebracht hatte, war noch anwesend. Er saß seinem Herrn gegenüber in einem Stuhl und hielt ein Glas mit Whisky in seiner Hand. Der Hausherr hatte seines auf einem Stuhl, der neben dem Sessel stand, abgestellt. Der Raum, in dem beide saßen, war dunkel. Die einzige Lichtquelle war ein Kamin in dem ein kleines Feuer brannte. Die vier jungen Kinder lagen in einzelnen Bettchen neben den beiden und schliefen.

\enquote{Ich kann mich nicht um sie kümmern}, sagte er und sah Gedankenversunken zu seinen Kindern. \enquote{So Leid es mir tut, aber ich werde sie in gute Hände geben müssen. Freunde\abs Gute Bekannte\abs Leute, denen ich vertrauen kann.}

\enquote{Ich könnte\abs}, doch Merowin verstummte.

Friedward sah ihn mit einem Blick an, der ihm sagte, was er nicht aussprechen wollte. \accentuate{Du bist zu alt. Du kannst deine Aufgaben schon nicht mehr richtig erledigen. Ich werde sie an vier Familien geben. Aber immer wieder nach ihnen sehen. Du kannst mitkommen. Jedes Mal, wenn ich meine Kinder besuchen werde.} Er pausierte kurz. \enquote{Ich werde meine Frau erst einmal begraben. Wo wir gerade vom Tod reden. Wie und vor allem wo möchtest du deine letzte Ruhe finden?}

\enquote{Ich gehöre zur Familie}, gab Merowin als einzige Antwort.

Friedward nickte und meinte: \enquote{Bei meiner Frau und \gst wenn es mal so weit ist \gst bei mir.}

Der Elf nickte.

\trenn

Nachdem er seine Frau beerdigt hatte, trug er Merowin auf, auf seine Kinder aufzupassen. Alle anderen Aufgaben seien unwichtig. Nur das Wohl seiner Kinder wäre jetzt noch wichtig. Das Schloss wäre unwichtig.

\enquote{Das Schloss werden wir, nachdem die Kinder versorgt sind, mit einem Haltbarkeitszauber belegen. Ich werde es wohl in nächster Zeit nicht mehr betreten. Was wirst du machen? Begleitest du mich? Dienst du deinem Volke? Gehst du in den Ruhestand?}

\enquote{Gebt mir etwas Zeit, Sir. Ich muss nachdenken.}

Friedward nickte. \enquote{Ich werde die Familien aufsuchen. Du kannst dann mitkommen, wenn ich sie morgen den Familien übergebe.}

Merowin nickte und kümmerte sich um die kleinen, während Friedward das Schloss verließ, um zu seinem ersten Ziel zu apparieren. Ein leises \geraeusch{Plopp} und er war weg.

Als er wieder auftauchte, musste er sich an einem Baum festhalten, damit er nicht umfiel. Kurz darauf erbrach er sich neben dem Baum.

\enquote{Es wird Zeit, dass da was anderes gefunden wird. Das ist ja ekelhaft.}

Er lief noch mehrere hundert Meter, bis er sein Ziel erreichte. Es war ein einfaches, aber gut-situiertes Landhaus. Er klopfte an und trat nach einer freundlichen Begrüßung ein.

\enquote{Friedward, schön dich zu sehen.}

Er stand in einer bäuerlichen Küche und umarmte gerade Molly, während die Tür aufging und ihr Mann William hereinkam. Friedward grinste ihn an, da ihn Molly immer noch im Arm hielt.

\enquote{Es scheint, dass deine Frau mich lieber hat}, scherzte er.

Molly ließ ihn los und knuffte auf seinen Oberarm. \enquote{Meinen Mann sehe ich öfter als dich, du Schlawiner.}

\enquote{Was willst du?}, fragte William.

\enquote{Ich möchte euch um einen großen Gefallen bitten.} Er lief zum Küchentisch und setzte sich. Die beiden folgten ihm. \enquote{Es geht um meinen Nachwuchs. Ich schaffe es nicht, mich um ihn zu kümmern. Ich möchte euch bitten, auf ihn aufzupassen\abs}

Dasselbe Szenario spielte sich bei den drei anderen Familien ab.

\enquote{\aabs Ich möchte euch bitten, auf ihn aufzupassen. Meine Frau ist gestorben und ich schaffe es nicht, da ich es ohne eine Frau\abs Ich möchte mich nicht mehr binden.}

Nach einem längeren Gespräch und der Zusage, dass sie sich um seinen Nachwuchs kümmern würde, verließ er sie wieder.

\enquote{Noch eine letzte Frage, wenn ihr es euch aussuchen könntet, würdet ihr euch um einen Jungen oder ein Mädchen kümmern?}

\enquote{Ein Mädchen}, antwortete Hiltrud.

Friedward grinste. Es ging genau auf. Zwei Jungen und zwei Mädchen.

Am nächsten Tag portete er achtmal, um seine Kinder schweren Herzens abzugeben. Jedes Mal nahm er seinen Elfen mit, der ihm folgte und die Übergabe überwachte.

\enquote{Und denkt dran. Zieht sie auf, als wäre es eure eigene Tochter. Sie muss aber immer wissen, dass ihre Mutter gestorben und ihr Vater auf Reisen ist. Er kann sich außerdem nicht um sie kümmern. Irgendwann wird der Zeitpunkt kommen, da wird sie erfahren, wer ihr Vater ist. Bis dahin werde ich als ein guter Bekannter ab und an vorbeikommen, damit wir einander kennenlernen. Ich werde nie ansprechen, was heute besprochen wurde.}

Dann verließ er die letzte Familie und kehrte mit seinem Elfen zurück zum Schloss.

Die nächsten Tage verbrachten beide damit, das Schloss zu konservieren. Dann hatte Merowin eine Idee, ein gemeinschaftliches Wappen der vier Familien zu erstellen und in der großen Empfangshalle anzubringen. So grübelten die zwei nach getaner Arbeit in der Halle und verwarfen immer wieder Vorschläge, bis am Ende ein Wappen gefunden wurde. Oben links ein goldener Greif auf rotem Hintergrund. Daneben eine silberne Schlange auf grünem Hintergrund. Unten rechts eine bronzene Meerjungfrau auf blauem Hintergrund und daneben einen schwarz-weißen Kraken auf gelbem Untergrund. In der Mitte der Wappen wurde ein \accentuate{H} platziert. Der Anfangsbuchstabe des Schlosses \gst \accentuate{Hogwarts}.

% Rowena, Salazar, Godric, Helga
Mit einem leisen \geraeusch{Plopp} verließen die beiden das Schloss und ihre Kinder, die bei den Familien in guten Händen waren. \accentuate{Mosley}, \accentuate{Slytherin}, \accentuate{Gryffindor} und \accentuate{Anecks}. Mit einer Träne tauchten beide wieder auf. In der Wüste Jordaniens, wo sie eine Weile bleiben wollten, bis Merowin alt genug wäre und wieder zurückreisen müsste, um seine letzte Ruhestätte zu finden. Elfen spürten meistens, wann ihre Zeit gekommen war und Merowin wusste schon seit drei Jahren, dass er noch zehn Jahre hatte, bevor er starb. Er reiste mit seinem Herrn um die halbe Welt und besuchte die abgelegensten Plätze. Diese Reise war für beide etwas. Für Friedward, um Abstand zu gewinnen und seine Frau zu vergessen; den Schmerz zu verarbeiten, und für Merowin, der seine letzten Jahre als Elf in Rente verbrachte. Er wollte nirgendwo anders sein als bei seinem Herrn. Also entschloss sich dieser, dass sie reisen würden. Merowin hatte soviel Spaß, wie zu seinen jüngsten Tagen als Elf. Und einmal im Monat, wenn Friedward für vier Tage weg war um seine Kinder zu besuchen, ruhte der Elf sich aus, falls er einmal keine Lust hatte, mitzukommen.

Als die Zeit reif war, kehrten die beiden für wenige Tage nach Hogwarts zurück, wo der Elf mit einem glücklichen Lächeln auf den Lippen im Kreise seiner Familie und seiner Verwandten starb. Friedward verließ die Ansammlung an Elfen, sobald Merowin gestorben war, da er spürte, sie würden ihn hinauswerfen. Dann verließ er Hogwarts wieder, um weitere zehn Jahre zu reisen. Er knüpfte Kontakte und erweiterte seine Künste im Bereich der Magie. Dann schaffte er es, das Apparieren für sich angenehmer zu machen.

Er spürte den Drang nach Hause zurückzukehren und tauchte wieder in Hogwarts auf. Seine Kinder hatten bereits geheiratet und lebten mit ihren Familien in ihrem eigenen kleinen Reich. So verließen Tage später vier Briefe Hogwarts. Einer an \accentuate{Rowena Ravenclaw}, einer an \accentuate{Salazar Slytherin} einer ging an \accentuate{Godric Gryffindor} und ein Brief erreichte \accentuate{Helga Hufflepuff}.

Das Wappen in der Großen Halle wurde verändert. Oben links war jetzt ein goldener Greif auf rotem Hintergrund, daneben eine silberne Schlange auf grünem Hintergrund. Unten rechts ein bronzener Adler auf blauem Hintergrund und daneben ein schwarz-weißer Dachs auf gelbem Untergrund.

Dann wartete Friedward auf die Gäste. Es dauerte ein paar Stunden, bis die Portschlüssel aktiv wurden und die Gäste mitbrachte. Jeder hatte einen kleinen silbernen Ring erhalten, den er sich anstecken musste, damit sich der Portschlüssel aktiviert. Zeitgleich trafen die Gäste ein und wurden von einem guten Freund der Familie erwartet. Das dachten sie zumindest, als sie Friedward am Tor sahen, der ihnen entgegenkam.

\enquote{Hallo meine Lieben, schön, dass ihr dem Ruf gefolgt seid.}

\enquote{Uns blieb nicht viel übrig. Der Brief war eindeutig.} Rowena stockte kurz, runzelte die Stirn und sagte dann: \enquote{Vater.}

Dann herrschte eine Weile Stille.

\enquote{Du bist wie immer die Klügste von allen. Kommt mit, ich erkläre es euch während des Essens.}

Stumm ging die Gruppe, von Friedward geführt, in das Schloss in die Große Halle zum Essen. Friedward ging um den runden Tisch herum und setzte sich der Tür gegenüber hin.

\enquote{Bitte, fangt an zu essen.}

\enquote{Und du, isst du nichts?}, fragte Salazar.

\enquote{Ich habe kurz vor eurer Ankunft etwas gegessen und esse noch etwas, wenn ihr euch beratet, bzw. das, was ich euch erzähle, sich setzt.}

Langsam und geduldig begannen seine vier Kinder zu essen, während Friedward zu erzählen begann.

\enquote{Ich fange ein paar Tage vor eurer Geburt an zu erzählen.} Da er keine Wiederworte hörte, fuhr er fort. \enquote{Eure Mutter, Persope, war mit euch Schwanger. Wir beide freuten uns auf euch. Vierlinge sind selten, aber wir hatten uns vorgenommen, das zu schaffen. Doch eure Mutter\abs} Er musste kurz unterbrechen, da sich die vier erstaunt ansahen. \enquote{Ja, ihr seid Geschwister. \gst Eure Mutter hatte sich eine schwere Lungenentzündung geholt, die nicht mehr geheilt werden konnte, da sie kurz vor der Geburt stand. Darum wollten wir uns danach kümmern. Also habe ich dafür gesorgt, dass sie immer warm angezogen war. Ihr Zustand war stabil. Dann wurdet ihr geboren. Zuerst du Rowena. Dann Salazar, Godric und schließlich Helga.} Er sah nacheinander seine vier Kinder an.

\enquote{Ihr seid Geschwister, aber auch, und das ist viel wichtiger, mächtige Hexen und Zauberer. Und deshalb habe ich euch zu mir gerufen. Einerseits, weil es Zeit ist, dass ihr mich kennenlernt. Andererseits, weil ich eine Aufgabe für euch habe. Die Situation unserer Art ist schlecht. Wir \gst das heißt, ihr \gst solltet eine Schule gründen und junge Hexen und Zauberer ausbilden.}

Alle Gabeln fielen auf den Tisch. Friedward nahm sich seine Gabel, steckte sie in geröstete Kartoffelscheiben und steckte sie danach in seinen Mund und kaute. Nachdem er seinen Bissen heruntergeschluckt hatte, fuhr er fort. Immer, wenn er eine kleine Pause machte, damit seine Kinder das gehörte verarbeiten konnten, nahm er einen Bissen vom reich gedeckten Tisch.

\enquote{Ich möchte, dass ihr euch der Ausbildung widmet. Junge Hexen und Zauberer brauchen eine fundierte Ausbildung. Und ihr hier seid dafür geeignet. Ihr seid die mächtigsten vier magisch begabten Personen dieser Zeit. Und, was noch wichtiger ist, ihr habt bereits euren Nachbarjungen und -mädchen etwas beigebracht. Dieses Schloss hier steht euch ab sofort zur Verfügung. \gst Nach dem Essen werden wir eine kleine Führung machen. Ich werde euch beim Umbau helfen, solltet ihr annehmen. Und ich werde euch die ersten Jahre als Hausmeister mit Rat und Tat zur Seite stehen. Ich werde euch aber nicht vorschreiben, wie ihr eure Aufgaben zu erledigen habt. \gst Lasst euch Zeit mit eurer Entscheidung. Ich will nichts hören, bevor wir nicht die Führung vollendet haben.}

Dann griff er endlich auch zu einem Messer und lud sich die restlichen Speisen auf seinen Teller und aß. Währenddessen saßen die vier stumm am Tisch und dachten nach.

\enquote{Dad?}

\enquote{Ja, Salazar.}

\enquote{Warum?}

\enquote{Werde genauer.}

\enquote{Warum sollen wir unterrichten?}

\enquote{Seht euch da draußen doch mal um. Die Lage ist katastrophal. Unsereins kann sich nicht richtig wehren, wenn wir von Muggeln angegriffen werden. Und vor allem, was passiert mit den ganzen magischen Kindern, die unter nicht-magischen Eltern aufwachsen. Sie müssen lernen, ihre Magie zu kontrollieren.}

\enquote{Es wird aber nicht jeder seine Kinder unterrichten lassen wollen. Viele wollen das selber machen.}

\enquote{Das geht natürlich weiterhin. Aber mit der Zeit, so hoffe ich, werden auch diese Leute merken, dass die Ausbildung hier besser ist.}

\enquote{Besser als wo?}

\enquote{Besser als bei den Eltern zu Hause. \gst Ich stelle mir das so vor. Alle magisch begabten Kinder werden in einer Liste geführt. Alle diese Kinder, die vor dem ersten September Elf werden, bekommen einen Brief zugeschickt, in dem sie eingeladen werden, die Schule zu besuchen. Das Schloss ist groß genug um die Schüler hier unterzubringen. In den Ferien werden sie nach Hause geschickt. Zumindest in den großen Ferien. Die restlichen Ferien können sie auch im Schloss verweilen.}

\enquote{Aber nicht jeder kann sich eine Ausbildung leisten}, warf Godric ein.

\enquote{Da kommen eure Familien und das Schloss mitsamt seinen Gütern ins Spiel. Die Kinder müssen nur Trankzutaten, Pergament und Tinte, Kessel und anderes Kleinmaterial kaufen, sowie eventuell, so vorhanden, die passenden Schulbücher \gst Oder normale Bücher, da es Schulbücher noch nicht gibt. \gst Der Rest, also die Unterkunft, die Verpflegung und andere Lehrmaterialien, wird von der Schule gestellt. Selbstverständlich auch die Krankenverpflegung.} Die Augen der vier wurden größer. \enquote{Wisst ihr, es soll eben nicht von der finanziellen Situation der Schüler abhängen, eine gute Ausbildung zu bekommen.}

\enquote{Aber, das viele Geld, was das kostet.}

\enquote{Das ist schon geregelt. Das Geld kommt zu hundert Prozent von mir \gst Besser gesagt, von einer Schulstiftung. In Gringotts ist bereits seit einiger Zeit ein entsprechendes Verlies mit dem Gold angelegt.}

\enquote{Mich hast du überzeugt, Dad}, sagte Helga. \enquote{Aber schauen wir uns noch die Räumlichkeiten an.}

Friedward zeigte die einzelnen Räumlichkeiten seinen Kindern und teilte ihnen mit, was für Fächer er sich vorstelle. \enquote{Am Anfang werdet es wohl nur ihr sein. Später, so hoffe ich, wird es weitere Lehrer geben.}

\enquote{Solange ich Zeit für meine Studien habe, von mir aus. Ich bin dabei}, sagte Salazar. \enquote{Wie sollen denn die Kinder untergebracht werden. Ich nehme nicht jeden}, sagte er noch.

\enquote{Wie wollt ihr denn die Auswahl treffen?}, fragte Friedward.

\enquote{Trennen wir sie doch nach Eigenschaften}, schlug Rowena vor.

\enquote{Und du nimmst dir die Besten?}, fragte Godric nach.

\enquote{Es gibt nicht die Besten. Es gibt Schüler mit stärker oder schwächer ausgeprägten Eigenschaften. \gst Ich bin dabei.}

Godric gab sich nach einem erneuten kleinen Disput geschlagen. \enquote{Da ich eh nichts anderes vorhabe, Ok.}

Friedward zeigte den Vieren noch, wo sie schlafen konnten und überließ es ihnen, das Schloss zu erkunden. Jetzt, Anfang März, dauerte es noch, bis die ersten Schüler kommen würden. Und für die Muggel-geborenen wurde vereinbart, dass der Brief persönlich überbracht werden soll, um direkt ein paar Fragen zu beantworten.

\trenn

Etwa eintausend Jahre später, Vernon Dursley stand gerade in seinem Garten und betrachtete die Ecke, in der über Nacht Pflanzen gewachsen waren.

\enquote{Petunia, kommst du mal?}

\enquote{Was ist denn, Vernon?} Petunia kam heraus und ihr Blick fiel sofort auf die Petunien, die in der Ecke gewachsen waren. \enquote{Oh Vernon, das ist so lieb von dir. Und sogar noch heute, an meinem Geburtstag.} Sie umarmte ihren Mann und gab ihm einen Kuss.

\enquote{Aber, das war ich nicht.}

\enquote{Nicht? Dann vergiss den Kuss. Aber wer war es dann?}

\enquote{Sag mal Petunia, du hast nicht etwa einen heimlichen Verehrer?}

\enquote{Nicht, dass ich wüsste. Aber die Blumen sind schön. Ich muss wieder in die Küche.} Damit verschwand sie wieder im Haus und lies einen nachdenklichen Vernon zurück. In der Küche angekommen, verrichtete sie weiterhin ihre Arbeit. Doch nach einer Weile hörte sie auf, sah noch einmal Richtung Fenster und dachte nach. \gedanke{Petunien \gst und das an meinem Geburtstag.} Sie begann leicht zu lächeln. \gedanke{Ich sollte mal wieder\abs}

Am Tag darauf erwachte Harry und war erst einmal verwirrt, als er die steinerne Decke sah. Dann nahm sein Ohr das Geklirr von Besteck und das Brechen von Zwieback wahr. Er richtete sich auf, doch er konnte die Person, die er sah, nicht mehr zuordnen, da ihm sofort schwarz vor Augen wurde und er nach hinten in das Bett zurücksackte. Wieder existierte er nur. Er roch nichts mehr, er fühlte nichts mehr, er sah nichts mehr und er hörte nichts mehr. Dann passierte etwas, was Harry noch nie erlebt hatte. Er schlug die Augen auf und lag in einem Bett. Grüne Satin-Bezüge über einer weichen Bettdecke und eine wertvolle Zimmereinrichtung. Er stand auf und sah an sich herab. Scheinbar hatte er mit seiner Kleidung geschlafen. Doch es war nicht seine Kleidung.

Er trat auf die Tür zu, öffnete sie und ging hindurch. Mehr neugierig als ängstlich ging er durch die ungewohnte und fremde Umgebung. Nach ein paar Ecken fand er eine Treppe. Er lief sie hinab. Eigenartigerweise kam ihm die Umgebung nun doch vertraut vor. Doch er konnte sie nicht genau zuordnen. Nicht so lange, bis eine Tür aufging und ein Mann herauskam. Er hatte blonde, lange Haare, welche zu einem Pferdeschwanz gebunden waren. Als er zu Harry blickte, sah er ihn ehrfürchtig, fast schon ängstlich an.

\enquote{Mylord}, nannte er ihn und ging dann durch die Große Halle in ein anderes Zimmer.

Harry blieb stehen und sah Lucius Malfoy nach. \gedanke{Er hatte Angst vor mir. Wieso hat Mister Malfoy Angst vor mir?}, fragte er sich. Er sah durch die Halle, entdeckte einen Spiegel und ging auf ihn zu. \gedanke{Und warum nennt er mich seinen Lord?} Als er in den Spiegel blickte, wusste er es. Die ersten Sekunden vergingen wie in Trance. Dann sickerte die Erkenntnis in sein Gehirn. Er war Voldemort. Jetzt könnte er Chaos verursachen. Doch was würde das Chaos für Auswirkungen haben?

Eine Tür ging auf und Bellatrix Lestrange schritt hindurch. Sofort zuckte seine Hand in der Versuchung seinen Zauberstab \gst Voldemorts Zauberstab \gst auf sie zu richten und sie zu bestrafen. Voldemorts Zauberstab. Er könnte ihn verstecken.

\enquote{Mylord}, wurde er auch von Bellatrix begrüßt.

Doch in ihrem Blick lag keine Angst. Dort fand er Hoffnung und Hingabe. Aber noch etwas entdeckte er. Ganz sachte näherte er sich ihrem Geist und drang in ihn ein. Als er Klarheit hatte, zog er sich zurück. Er fand seine Bestätigung. Sie verehrte ihn nicht nur, nein, sie liebte ihn. Das wäre doch etwas. Er könnte mit ihr\abs Und wenn er wieder zurück in seinem Körper wäre, dann würde er sie bestrafen.

\enquote{Komm mit}, sagte er, in der Hoffnung authentisch zu klingen.

Gehorsam folgte sie ihm. Er ging den Weg zurück zu seinem Zimmer. Davor angekommen drehte er sich kurz herum. \enquote{Warte kurz.} Er betrat sein Zimmer und verstaute seinen Zauberstab. Er versteckte ihn hinter seinem Nachtkästchen. Dann holte er Bellatrix herein. Kaum hatte sie den Raum betreten, stand er vor ihr und schob die Tür zu. Sie wich einen Schritt zurück und stand nun mit dem Rücken an der Tür. Harry konnte es nicht leugnen. Wenn er ihre Boshaftigkeit beiseite schob, sah sie ganz Attraktiv aus. Ihre Zähne einmal in der Farbe Weiß vorausgesetzt. Er nahm seine Hand hoch und fuhr ihr über die Lippen. Er konzentrierte sich auf einen Zahnreinigungszauber.

Bellatrix schloss ihre Augen und ein Kribbeln machte sich in ihrem Mund breit. Als sie einen Finger an ihrem Kinn spürte, der es ihr nach unten drückte, fing sie an zu lächeln und ihre Augen zu öffnen. Ihre jetzt weißen Zähne blitzten und ihr Herz begann zu klopfen.

Er löste seinen Griff von ihrem Kinn und umfasste jetzt mit beiden Händen ihre Hüfte. Langsam zog er sie zurück, worauf hin sie ihm willig folgte. Sie zeigte ihm ihr Lächeln und die neuen weißen Zähne. Er spürte, wie er begann die Kontrolle zu verlieren, doch er zwang sich, noch ein paar Sekunden durchzuhalten. Am Bett angekommen ließ er sich mit ihr zurückfallen und begann sie zu küssen. Ein paar Sekunden noch hielt er durch. Dann ließ er los.

Die Umgebung wechselte und er war wieder im Krankenflügel. Madame Pomfrey, Professor McGonagall und Professor Elber standen mit gezogenen Zauberstäben da. Ebenfalls Harry. Er ließ ihn los und griff um seinen Anhänger. Sofort stand er wieder in dem Zimmer und besah sich das Schauspiel.

Voldemort drückte Bellatrix von sich. \enquote{Was fällt dir ein}, schrie er.

Irritiert sah sie ihn an. Gerade eben war er doch noch so lieb.

Dann stutzte er und drehte seinen Kopf leicht. Er sah Harry. Dann wusste er, was Harry getan hatte. Und obwohl es ihm schwerfiel, bestrafte er Bellatrix nicht. Immerhin hatte er erfahren, dass Harry mittlerweile recht gut seine Magie beherrschte.

\enquote{Lass es gut sein, Bellatrix. Bleib einfach hier. Neben mir.} Den letzten Satz betonte er. Als Bellatrix neben ihm lag, glücklich, sah er noch einmal kurz zu Harry und lächelte ihn an. Dann schloss er selbst die Augen und entspannte. Harry zog es zurück und er dachte noch kurz nach. Voldemort dürfte die gewünschte Erinnerung an Ginny in der Kammer gar nicht haben, fiel ihm ein.

\gedanke{Warum ist das Salazar gar nicht aufgefallen?}, fragte er sich.

\stimme{Das habe ich ihn auch gefragt}, antwortete eine weibliche Stimme in seinem Kopf.

\gedanke{Agatha?}, fragte Harry nach.

\stimme{Ja, aber darüber reden wir später.}

Harry nickte innerlich. Dann war er wieder bei der Sache. Er ließ sein Amulett los und blickte in die Runde. \enquote{Habe ich mich aufgeführt?}, fragte er vorsichtig nach.

Die drei Professoren nickten nur, immer noch bereit ihn außer Gefecht zu setzen.

\enquote{Darf ich fragen, für wen Sie mich gehalten haben?}

\enquote{Für den Ungenannten}, antwortete Madame Pomfrey direkt aber zurückhaltend.

Harry nickte und hob seinen Zauberstab auf. \enquote{Das kann ich nur indirekt bestätigen. Scheinbar habe ich eine Weile mit ihm die Plätze getauscht.} Er schob seinen Stab ein, wurde aber noch immer sorgfältig beobachtet.

\enquote{Was machen wir jetzt mit ihm?}, fragte Professor McGonagall.

\enquote{Die für mich spannendere Frage ist: \inner{Was machen wir, wenn das noch mal passiert?}}, fragte Professor Elber.

\enquote{Du bist hier der Experte für diese Art von Magie. Schlag du was vor}, sagte Professor McGonagall. Elber senkte seinen Stab und schob ihn kurz darauf ein. Dann setzte er sich wieder auf sein Bett und ließ seine Beine herunterbaumeln. \enquote{Was machst du da?}, wurde er erneut gefragt.

\enquote{Ich denke nach.} Dann legte er sich auf das Bett und starrte an die Decke.

Professor McGonagall drehte sich zu Harry und sah ihn an. \enquote{Warum dürfen Sie Professor Dumbledore nicht bei seinem Vornamen anreden?}

\enquote{Wie? Ich darf sehr wohl. Allerdings nur, wenn wir alleine sind.}

Auch Professor McGonagall schob ihren Stab ein.

Nach einer Weile zog Madame Pomfrey nach und stand noch kurz da, um Harry zu beobachten. \enquote{Sie können von mir aus gehen}, sagte sie dann.

McGonagall nickte und verließ den Krankenflügel. Madame Pomfrey kümmerte sich schon um den nächsten Patienten, der bereits hereinkam und sich zur Untersuchung auf ein Bett legen musste.

Harry ging ebenfalls. Sein Lehrer würde sich schon melden. Dann fiel ihm wieder ein, dass er von der falschen Erinnerung ausgegangen war. Er suchte nicht die von Ginny in der Kammer, sondern die von Voldemort, in der er Hagrid beschuldigt hatte.

\gedanke{Aber wieso war das Agatha nicht aufgefallen?}, fragte er sich.

\stimme{Weil ich mich in deinen Gedanken verlaufen habe}, gab sie kleinlich zu.

\gedanke{Wie meinst du das?}

\stimme{Durch das Bild in unserem Raum und der Verbindung von Salazars Geist mit dem Bild, habe ich es mittlerweile geschafft auch eine Verbindung zu dir aufzubauen. Darüber war ich so erstaunt, dass ich mich einfach umgesehen habe und deine letzten Gedanken mitbekommen habe. Tut mir leid Harry, das war falsch. Das hätte ich nicht tun dürfen. Als ich es bemerkte, habe ich mich sofort zurückgezogen.}

\gedanke{Was hast du alles mitbekommen?}

\stimme{Nur die letzten paar Stunden.}

\gedanke{Gut. Dann weißt du jetzt, was ich suche.}

\stimme{Kannst du nicht einfach deine Erinnerungen nehmen?}

\gedanke{Nein, denn mir wurde nur ein Bild gezeigt. Ich brauche das Original.}

\stimme{Huch!}, sagte sie plötzlich.

\gedanke{Was ist los?}

\stimme{Ich habe gerade etwas bei dir bemerkt. Du hast einen Seelensplitter von ihm in dir.}

\gedanke{Richtig.}

\stimme{Dann brauchst du Voldemort gar nicht. Der Splitter ist so schwach, dass er sich nicht wehren kann. Es ist so, als würde er schlafen. Konzentriere dich auf die Erinnerung in seinem Geist und ziehe sie heraus.}

Harry blieb stehen.

\gedanke{Hilfst du mir?}

\stimme{Denke einfach nur an das, was du willst. Dann klappt es auch.}

Harry beschwor sich ein Glasröhrchen mit Korken herauf und suchte sich einen ruhigen Platz. Er entkorkte es und versuchte sein Glück. Er schloss seine Augen und konzentrierte sich auf den kleinen kalten Teil in sich in das er versuchte mit Legilimentik einzudringen. Doch er versagte. Er spürte nicht einmal einen Widerstand. Er spürte gar nichts. Bei jedem, bei dem er es versucht hatte und das waren eine Menge Mitschüler, hatte er Zugang zu Bildern erhalten. Er hielt sie bewusst Unscharf, da er nicht in deren Privatsphäre eindringen wollte.

Harry dachte nach. Nach einer gefühlten Ewigkeit hatte er eine Idee. \gedanke{Salazar? Agatha?}

\stimme{Ja}, sagten beide.

\gedanke{Ich habe mit Legilimentik keinen Erfolg.}

\stimme{Das haben wir bemerkt}, sagte Salazar.

\gedanke{Was habe ich sonst noch für Möglichkeiten?}

\stimme{Wenn du das nicht schaffst? Keine.}

Das gab Harry zu denken. Er holte Ginny ab und ging mit ihr spazieren.

\enquote{Harry Potter, Harry Potter!} Dobby kam auf ihn zu gerannt, als er gerade im Schloss unterwegs war.

\enquote{Dobby, schön dich zu sehen, was gibt es denn? Du bist ja ganz aufgeregt.}

\enquote{Dobby möchte Harry Potter um einen großen Gefallen bitten.}

\enquote{Gern, um was geht es denn?}

Ginny, mit der er gerade unterwegs war, blieb stehen und wartete.

\enquote{Dobby möchte gern Harry Potter als Traupaar-Führer.}

\enquote{Traupaar-Führer?}

\enquote{Ja, Harry Potter. Dobby und Winky werden bald heiraten.}

Harrys Gesichtszüge entgleisten. Er ging auf die Knie und nahm den kleinen Elfen in den Arm. \enquote{Das ist ja wunderbar, Dobby.} Dann stutze er. Er ließ Dobby los und sah ihn an. \enquote{Was ist meine Aufgabe dabei?}

\enquote{Dobby hat daran gedacht, Harry Potter.}

\enquote{Nenn mich endlich Harry, Dobby.}

\enquote{Gern, Sir Harry. Dobby hat Sir Harry Pergamente da gelassen. In Slytherins Räumen. Damit kann er sich auf die Aufgabe vorbereiten.}

\enquote{Lass mich erst einmal etwas darüber lesen. Ich weiß nichts über eure Art und eure Bräuche. \gst Wann brauchst du meine Entscheidung?}

\enquote{Ende des Schuljahres.}

Harry nickte. \enquote{Dann habe ich Zeit mir deine Informationen durchzusehen und mir meiner Aufgabe bewusst zu werden.}

\enquote{Das ist eine sehr verantwortungsvolle Aufgabe, Sir. \gst Dobby muss jetzt wieder arbeiten.} Damit verschwand er.

\enquote{Dobby will dich als Trauzeugen?}, fragte Ginny, die noch immer in der Nähe stand und daher alles mitbekam.

\enquote{Nein, er hat einen anderen Begriff verwendet. Ich weiß nicht, ob das einem Trauzeugen gleich kommt. Ich werde erst einmal darüber lesen.}

\enquote{Nimmst du mich mit?}

Harry nickte und nach dem Abendessen ging er mit Ginny zu Salazars Räumen, wo schon die Pergamente von Dobby auf dem Tisch lagen. Er nahm sie vom Tisch und begann zu lesen, während Ginny im Bücherregal ein Buch heraussuchte, es mitnahm, sich neben Harry auf das Sofa setzte und zu lesen begann. Während der nächsten Stunde, in der Harry in die Aufzeichnungen vertieft war, wurde Ginny immer müder, ihre Hände wurde schwerer und ihre Augen fielen zu. Ihr Kopf kippte leicht auf die Seite und kam auf Harrys Schulter zu liegen. Gedankenversunken nahm er sie in den Arm und las weiter, bis auch er fertig war; mit den Pergamenten und seinen Augen. Er schloss sie und legte seine Wange auf Ginnys Haar.

Unbewusst kuschelten sich beide aneinander. Das Ehepaar auf dem Bild über ihnen lächelte beide an. Danach nahm Salazar seine Frau in den Arm und schaute sie verliebt an.

Am nächsten Tag lief Harry wieder durch das Schloss. Er sollte sich heute im Pokalzimmer zu einer Unterrichtsstunde einfinden. Freudig betrat er das Zimmer und wurde bereits von Professor Elber erwartet. Dieser drückte ihm wortlos einen Eimer in die Hand. Harry stutzte. Danach nahm ihm sein Professor den Zauberstab aus dem Umhang und verließ mit den Worten: \enquote{Viel Spaß} den Raum und schloss hinter sich die Tür. Harry fragte sich, was das denn sollte. Er sah in den Eimer und entdeckte einen trockenen Lappen. Was darunter lag, sah er nicht, da er durch Rufe unterbrochen wurde.

\enquote{Putz mich}, hörte Harry.

Verwundert näherte er sich vorsichtig der Stimme, die aus einer der Vitrinen zu kommen schien.

\enquote{Putz mich}, hörte er erneut.

Er sah einen Pokal, aus dem ein Gesicht heraus schaute.

\enquote{Warum?}, fragte Harry ganz entgeistert.

\enquote{Du hast betrogen, deshalb.}

\enquote{Wann? Wo?}, fragte Harry. Er konnte sich nicht erinnern.

\enquote{Die Baumstümpfe.}

Jetzt wurde es Harry klar. Er hatte alle Baumstümpfe mit dem Zauberstab entfernt. Und nicht wie gefordert, die Hälfte ohne. Er sah in seinen Eimer und holte den ersten Putzlappen heraus. Darunter sah er weitere zwei, die er ebenfalls herausnahm und nun auf zwei Plastikfläschchen mit Schraubverschluss blickte. Auf einem stand  \accentuate{Wasser \gst Selbst auffüllend} und auf dem anderen \accentuate{Politur \gst Selbst auffüllend}.

\enquote{Putz mich}, hörte er erneut.

Also machte er sich ans Werk, tat etwas Politur auf einen der Lappen und fing an den Pokal zu reinigen, der ihn ansprach, indem er Politur auftrug und sie leicht einrieb. \enquote{Wie viele muss ich denn machen?}, fragte er, als er den ersten Pokal mit einem weiteren Lappen polierte.

\enquote{Solange, bis die Schuld abgetragen wurde}, sagte ein weiterer Pokal. \enquote{Ich bin der nächste.}

Harry kam zu ihm und bemerkte die Staubschicht auf ihm. Er sah seinen Politur-Lappen an und entschied, den Pokal erst einmal mit einem feuchten Tuch zu säubern\abs

Während seiner Strafaktion erfuhr er von jedem Pokal, den er reinigen musste, etwas über denjenigen oder diejenige, die ihn verdient hatte. Als er fertig war, schmerzten seine Arme und er konnte auf der Stelle einschlafen. Müde sank er zu Boden und wurde von dem weichen Boden aufgefangen.

\enquote{Was machen Sie hier?}, wurde er etliche Stunden später von Professor Snape unsanft geweckt.

\enquote{Polieren}, gab er matt zurück und kratze sich den Schlaf aus den Augen.

\enquote{Sieht mir aber nicht danach aus.}

\enquote{Musste die Pokale polieren\abs Habe mich bescheuert verhalten\abs Habe\abs nicht ganz sauber\abs gearbeitet.}

\enquote{Professor Elber?}, fragte Snape nach. Als ihn Harry fragend ansah, antwortete er: \enquote{Er hat was in der Richtung erwähnt.} Dann griff er in seinen Umhang und zog ein Pergament hervor. Er reichte es Harry und verschwand.

Als sich Harry das Pergament besah, fand er eine Entschuldigung für das späte Ausbleiben vor. Er konnte also zurück in dem Gemeinschaftsraum, ohne eine Strafe zu erhalten. Dann rappelte er sich auf und verdrückte sich. Und wieder überkam ihn dieser eigenartige Traum vom Anfang des Schuljahres, in dem er einen Pokal mit einem Dachs im Verlies der Lestranges sah. Dann dämmerte er in seinem Traum wieder weg.

Am nächsten Morgen wurde er von Professor Elber abgefangen. \enquote{Heute Abend bei Hagrid. Zweiter Versuch.} Dann lief er weiter und lies Harry stehen.

\enquote{Was war das denn jetzt?}, fragte Ron, der neben ihm stand.

\enquote{Ich habe beim letzten Training nicht ganz ehrlich gespielt. Darauf hin musste ich einen großen Teil der Pokale polieren. Er gibt mir wohl eine zweite Chance.}

Dann mussten sie schon zur nächsten Unterrichtsstunde.

\enquote{Harry ist dieses Jahr abweisender als sonst}, merkte Ron an.

\enquote{Er darf halt nichts über seine Extra-Stunden erzählen, genau wie du auch}, antwortete Hermine.

\enquote{Du hast ja recht. Auch wenn ich nicht verstehe, warum nicht wenigstens du mir etwas von dir erzählst. Immerhin bin ich dein Freund.}

\enquote{Du erzählst mir ja auch nichts. Und jetzt Ruhe, wir sind da.}

Der Unterricht bei Professor Flitwick verlief ruhig und nach dem Abendessen meldete sich Harry wie angewiesen bei Hagrid, der ihn in ein Waldstück führte und ihn wieder bei etwa zwanzig Baumstümpfen abstellte. Harry seufzte und setzt sich auf einen der Baumstümpfe. Die nächste viertel Stunde dachte er nach, wie er am besten vorgehen konnte. Während er seinen Gedanken nachhing, näherte sich von der Seite ein junges Testral-Männchen. Es schnupperte an seinem Haar und musste einmal durch seine Nüstern ausblasen, da Harrys Shampoo-Duft ihn in der Nase irritierte. Erst jetzt bemerkte Harry das Tier neben sich und begann es zu streicheln. Er erklärte dem jungen Männchen, was seine Aufgabe sei. Bis ihm bewusste wurde, dass das sinnlos war. Er sah wieder zu den Baumstümpfen, als er Bilder in seinem Kopf vernahm, Bilder, die ihn zeigten, wie er verschiedene Zauber auf die Stümpfe warf. Als er das Testral-Männchen ansah, nickte dieses nur einmal mit seinem Kopf und verließ ihn dann.

\gedanke{Na klar, so geht es. Immer auf andere Art und Weise. Einen Baumstumpf mit Zauberstab und Worten. Den nächsten mit Zauberstab und ohne Worte. Dann dasselbe nochmals ohne Zauberstab. Das wären dann vier Stümpfe mit demselben Zauber.} Also machte er sich ans Werk und entfernte so einen Baumstumpf nach dem anderen.

Als er fertig war, stand auch Hagrid schon wieder hinter ihm. \enquote{Gut gemacht Harry. Kannst geh’n.}

Harry war erleichtert. \enquote{Habe ich bestanden?}

\enquote{Denke schon. Hast wohl dies’mal alles richtig gemacht. Un’ nu geh mal.}

Auf dem Rückweg ging er einen Gang entlang, den er sonst selten wählte. Plötzlich blieb er stehen und sah auf eines der vielen Bilder, die alle gleich aussahen. Doch eines war leicht verändert. Es waren alles Landschaften mit Bäumen. Sein Unterbewusstsein musste den Unterschied aufgegriffen haben und mit den Markierungen auf der Karte, die er anfertigte um weitere Geheimgänge und Räume im Schoss zu erkunden, abgeglichen haben. Er sah sich das Bild genau an und wechselte zwischen mehreren hin und her. Nach mehrere Minuten entdeckte er ein fehlendes Blatt auf einem der Bilder. Mit der Hand strich er über den fehlenden Platz und hörte ein klackendes Geräusch. Der Bilderrahmen mitsamt dem Bild kam einige Zentimeter vor.

Harry schob am Rahmen das Bild beiseite und trat in den Gang dahinter. Mit gezücktem und leuchtendem Zauberstab ging er den Gang entlang. Ein klackendes Geräusch ließ ihn umsehen. Das Bild hinter ihm verschloss den Gang wieder. Harry setzte seinen Weg fort. Er versuchte sich den Plan des Schlosses vor seinem geistigen Auge vorzustellen. Laut diesem Plan dürfte er irgendwo in der Nähe der Großen Halle sein, dachte er, als er eine Tür sah. Leider hatte er die Karte des Rumtreibers nicht dabei, um auf ihr nachzusehen, wo er war.

So schob er vorsichtig die schwergängige Tür auf. Er entdeckte einen Raum, der ihm vertraut vorkam. Irgendwo hatte er ihn schon einmal gesehen. Stocksteif stand er plötzlich da, als er hinter sich eine bekannte Stimme hörte.

\enquote{Mister Potter.}

Vorsichtig drehte er sich um.

\enquote{Soso. Sie dringen also in meine Privatsphäre ein.}

\enquote{Nein, Professor. Das war keine Absicht. Ich war einfach neugierig}, stammelte er.

\enquote{Wie kommen Sie überhaupt hier herein?}

Der Gang schloss sich bereits und mit ihm das Bücherregal vor ihm, das den Gang mitsamt der Tür verschloss.

Harrys Konzentration war am Ende, so zog Professor Snape seinen Zauberstab und untersuchte Harrys Geist. Aber außer, dass er die Wahrheit gesagt hatte und diesen Gang durch Zufall entdeckt hatte, fand er nichts. \enquote{Ich hatte bisher keine Ahnung}, sagte er, als er seinen Stab wieder einsteckte, \enquote{dass sich hinter diesem Regal ein Gang befindet. Eigentlich hatte ich bisher noch nie das Bedürfnis, irgendetwas aus diesem Regal zu entnehmen.}

Harry machte das stutzig. Er schaute sich die Sachen im Regal genauer an. Könnte ein Schutzzauber darauf liegen, dass man das Regal unbeachtet ließ, wenn man nicht wusste, dass sich dahinter ein Geheimgang befand?

\enquote{Schutzzauber?}, fragte er in den Raum hinein, aber doch mehr zu sich selbst.

\enquote{Könnte sein}, antwortete Snape.

Harry ging an eines der Bücher und zog daran. Es klappte nur heraus und gab den Weg zum Gang dahinter wieder frei.

\enquote{Wie kamen sie darauf?}, fragte Snape.

\enquote{Es war das einzige, das nicht ins Gesamtbild passte}, antwortete Harry.

Snape nickte. \enquote{Wo endet der Gang?}

\enquote{Fünfter Stock. Im Gang, wo die vielen gleichen Bilder hängen. Eines der Bilder ist leicht anders. Das achte oder neunte, wenn man von Westen kommt. Bäume sind darauf abgebildet.}

Snape nickte erneut.

Harry versuchte die Stille zu überbrücken. \enquote{Wie kommen Sie mit meinen Rezepten voran?}, fragte er ablenkend.

\enquote{Sehr gut. Ich arbeite gerade an einem davon. Kommen Sie.}

Harry folgte seinem Zaubertränke-Lehrer an den Kessel mit dem Trank. Unter ihm brannte ein Feuer auf kleiner Flamme und ließ den Trank vor sich hin köcheln. Er sah auf das Pergament davor und danach in den Topf.

\enquote{Schritt vierzehn. Vor der Zugabe der Trient-Wurzel}, folgerte er.

\enquote{Gut erkannt. Es scheint, dass Ihre Kenntnisse in diesem Fach besser werden. Arbeiten Sie weiter daran. Ich muss mir langsam andere Sachen überlegen, wie ich Sie fertig machen kann.}

\enquote{Ich versaue einfach den nächsten Trank und nehme an, das wird er, oder?} Snape nickte. \enquote{Darf ich ihnen helfen?} Erneutes nicken. Harry sah wieder auf das Pergament und holte dann die entsprechenden Zutaten, um sie zu verarbeiten und dann in den Kessel zu geben. Es war ein eigenartiges Gefühl hier mit Snape zu stehen und Hand in Hand mit ihm zu arbeiten. Kein böses Wort, kein Streit, kein Hohn. Einfach nur zusammen an etwas arbeiten. Ab und an korrigiert Snape Harry, aber sonst arbeitete er sehr gewissenhaft, so gewissenhaft, dass sich sein Professor am Ende des Trankes genötigt sah, ihm dafür Punkte zu geben.

\trenn

Eine weitere, \accentuate{dunkle}, Stunde \VgddK hatte gerade begonnen, als Professor Elber anfing zu erzählen. \enquote{Wir werden uns heute mal den Imperius-Zauber vornehmen. Ich würde ihn Ihnen gern praktisch vorführen, um Ihnen die Auswirkungen zu zeigen, aber leider ist er illegal. Es gibt aber eine Reihe weiterer Zauber, die vom Ministerium nicht derart eingestuft wurden. Folglich kann ich Ihnen, sofern Sie sich zur Verfügung stellen, diesen Effekt zeigen. Zuvor aber einige andere Dinge.} Er trat durch den Raum und vor Ron. \enquote{Ich möchte Ihnen den Effekt des Imperius zunächst einmal beschreiben. Dazu werde ich Ihnen ein paar Fragen stellen. Bitte beantworten Sie diese wahrheitsgemäß. Ich verspreche Ihnen, es wird nichts Peinliches geben.} Ron nickte. \enquote{Der Imperius-Zauber hat den Zweck, einem Wesen den eigenen Willen aufzuzwingen. Dies ist aber nicht immer möglich. Es kommt zum einen auf den Charakter, oder genauer gesagt, den Willen der Person an, die unter den Zauber gestellt werden soll, auf den Wunsch der auslösenden Person und auf die Art des Befehls.} Er trat rückwärts und lehnte sich an seinen Schreibtisch. Er sah Ron wieder direkt an. \enquote{Wenn ich Ihnen einen Befehl geben würde, etwas zu kaufen, was Sie eh schon vorgehabt hatten, dann würden Sie keinen Sinn sehen, dagegen anzukämpfen. Also würde der Imperius bei Ihnen wirken. Wenn aber etwas gegen Ihre Überzeugung geht, würden Sie instinktiv versuchen, sich zu wehren und wären eventuell in der Lage, diesen zu brechen.} Ron nickte. \enquote{Ich gebe Ihnen einmal ein Beispiel, antworten Sie bitte ehrlich.} Und wieder kam ein Nicken. \enquote{Wenn ich Ihnen unter Imperius befehlen würde, Ihren Freund Harry zu ermorden, würden Sie das dann tun?}

\enquote{Nein Professor}, antwortete Ron entrüstet.

\enquote{Warum nicht?}

\enquote{Weil er mein Freund ist, ich könnte ihm nichts tun.}

\enquote{Es geht also gegen ihre Überzeugung?}

\enquote{Ja.}

\enquote{Gut}, fuhr Elber fort. \enquote{Gegenbeispiel. Wenn ich Ihnen sagen würde, dass Sie\abs} Er unterbrach sich kurz und schaute durch die Klasse. \enquote{\aabs sagen wir mal Mister Malfoy eine Ohrfeige geben sollen. Hätten Sie da etwas dagegen?}

\enquote{Nein Professor}, antwortete Ron ehrlich.

\enquote{Und wenn ich Ihnen Befehlen würde\abs}, wieder blickte er durch die Klasse, \enquote{\aabs Hermine hier zu küssen?} Er blickte zurück zu Ron und sah darüber hinweg, dass er leicht rosa anlief. \enquote{Ich meine, sie ist hübsch und für einen jungen Mann wie Sie ist es sicherlich nicht abstoßend.}

\enquote{Ich hätte nichts dagegen}, sagte Ron, der sich innerlich gestrafft hatte und daher mit fester Stimme sprach.

Nun war es Hermine, die ganz leicht rosa anlief.

Elber stieß sich vom Tisch ab und lief über das Podest. \enquote{Ich werde Ihnen nun demonstrieren, wie sich das anfühlt, wenn man unter den Imperius gesetzt wird. Es gibt einen Zauber, der dieses Gefühl hervorruft, ohne dass er weitere Auswirkungen hat. Sie werden also keinen Drang verspüren, irgendetwas zu tun.} Er zog seinen Zauberstab und fuhr in einer flüssigen Bewegung halbkreisartig über die Klasse.

Sofort fühlte sich jeder im Raum, als ob seine Gedanken zu verschwimmen begannen. Es war so, als ob einem das Denken abgenommen wurde. Jeder fühlte sich leicht und wie in einem Traum. Eine innere Stimme sagte ihnen, was sie zu tun hatten, doch bevor sie daran denken konnten, senkte Elber seinen Stab und beendete den Zauber.

\enquote{Sie haben jetzt einen Eindruck davon gewonnen, wie der Imperius sich bemerkbar macht. Dies spüren Sie allerdings nur, wenn sich jemand unerfahrenes an Ihrem Gehirn zu schaffen macht. Als Nächstes möchte ich Ihnen demonstrieren\abs}, machte er weiter und die Klasse setzte sich wieder gerade hin, da sie während des Rauschzustandes leicht auf die Sitzflächenkante gerutscht waren, \enquote{\aabs wie es sich anfühlt, wenn man unter den Imperius gesetzt wird. Wieder wird der Zauber nicht der Imperius sein, aber Sie werden einen Drang verspüren. Der Zauber, den ich verwenden werde, stellt wieder nur einen Teilaspekt dar.}

Professor Elber stieg die Treppen zu seinem Büro hoch, holte eine Kiste mit Tomaten heraus und nahm sie mit nach unten. \enquote{Jeder holt sich eine Frucht heraus}, sagte er, als er durch die Reihen lief. Als jeder eine Tomate vor sich liegen hatte, zauberte er noch Stoffservietten herbei. \enquote{Diese werden Sie danach brauchen, glauben Sie mir.} Auf den Gesichtern der Schüler begann sich gerade Staunen zu zeigen, als Elber schon anfing, den Zauber zu wirken.

In jedem der Schüler begann nun der Drang, die Tomate mit bloßen Händen zu zerreißen und essen zu wollen, stärker zu werden. Das wohlige, wolkige Gefühl blieb aber aus. Jeder Gedanke war klar im Kopf fassbar. Nach und nach stürzten sich die Schülerinnen und Schüler über die Tomate her, versuchten sie zu zerreißen und bissen ab, als sie bemerkten, dass das nicht klappte. Gierig bissen sie hinein, und sauten den Tisch vor sich ein. Nur eine Schülerin saß da und sah die Tomate mit wachsendem Interesse an. Dass auch Harry beherzt in die Tomate biss, wunderte Elber. Aber als er Harry grinsen sah, war er der Meinung, dass dieser einfach nur so zubiss, dem fremden Drang aber widerstand, oder es zumindest versuchte, da er seine Feinde so täuschen konnte.

Professor Elber brach den Zauber wieder ab und nach den ersten Schrecksekunden begannen die ersten, sich die Hände mit der Serviette zu putzen. Danach putzten sie die Tische, eventuell vorhandene Flecken und Spritzer aus der Kleidung entfernten sie wortlos mit ihren Stäben.

\enquote{Sehr gut, Sie haben nun einen Eindruck davon bekommen, wie es ist, unter fremdem Einfluss zu stehen.} Er wandte seinen Blick zu seiner Schülerin. \enquote{Sie haben sich als einzige gar nicht über die Tomate hergemacht. Wie kommt das?}, fragte er interessiert schauend.

Sie antwortete ihm: \enquote{Ich hasse Tomaten. Ich kann diese Früchte nicht ausstehen. Ihr Geschmack ist einfach eklig. Jedes Mal, wenn ich eine essen musste, konnte ich meinen Speisebrei nicht halten und musste über dem Esszimmertisch brechen}, sagte sie völlig normal.

\enquote{Gut, machen wir weiter. Dieses Mal sage ich Ihnen nicht, was\abs} Er schwang seinen Stab erneut. \enquote{\aabs dran kommt.}

Es dauerte keine zwei Sekunden, als die Schülerin aufstand und begann sich auszuziehen. Sie stand gerade in Unterwäsche da, als weitere Schüler aufstanden und begannen sich jetzt ebenfalls auszuziehen. Gerade als Astoria hinter sich griff und den Verschluss ihres BHs zu öffnen, sah Elber sie an und schnippte einmal mit den Fingern, den Blick auf sie fixiert. Astoria öffnete noch den Verschluss, als sie bemerkte, dass es nicht das war, was sie wollte. Schnell schloss sie ihn wieder. Leicht beschämt blickte sie ihren Professor an. Dieser gab ihr nur wortlos zu verstehen, sie möge sich setzen. Die stumme Frage, ob sie ihre Kleidung wieder anziehen dürfe, verneinte er mit leichtem Kopfschütteln. Jeden seiner Schüler, der so weit war, zu viel zu zeigen, fixierte er mit seinen Augen und schnippte danach einmal mit den Fingern. Als jeder Schüler und jede Schülerin nur noch in Unterwäsche da saß, brach er den Rest des Zaubers.

\enquote{Nun wissen auch Sie, wie es ist, unter diesem Einfluss zu stehen. Bei Ihnen waren die beiden Gefühle leider vertauscht. Scheinbar macht es Ihnen weniger aus, mehr Haut von Ihnen zu zeigen, als eine Tomate zu essen. \gst Das ist nichts, weswegen Sie sich zu schämen brauchen.} Das Mädchen, das bereits Schamgefühle zu spüren glaubte, war erleichtert. \enquote{Wenn es Ihnen nichts ausmacht, sich leicht bekleidet zu zeigen, heißt das nicht, dass sie sich gleich gegenüber jeder Person ausziehen würden. \gst Mir persönlich, würde es absolut nichts ausmachen, Professor McGonagall zu küssen, ich werde es aber nicht tun.} Das lockerte die mittlerweile leicht angespannt Stimmung wieder auf und die Schüler saßen jetzt entspannt auf ihren Plätzen.

Professor Elber drehte sich herum und sah kurz gehen die Wand, die sonst immer hinter ihm war. Als er wieder zur Klasse blickte, meinte er: \enquote{Ich habe noch ein letztes Experiment vor. Lassen Sie es mich Ihnen zuerst erklären, damit Sie es verstehen, was ich Ihnen damit zeigen möchte und Sie begreifen, warum ich dies von Ihnen \gst es ist keine Forderung, es ist vielmehr \gst wie soll ich mich ausdrücken \gst ich erwarte von Ihnen, dass sie Ihre Grenzen kennen \gst ich möchte Ihnen Ihre Grenze bezüglich des Imperius-Zaubers zeigen, damit Sie wissen, woran Sie arbeiten können. Im Buch zu diesem Fach wird am Ende der Stunde ein weiteres Kapitel zu sehen sein. Dies arbeiten Sie bitte für sich durch und ich erwarte am Ende des Schuljahres eine Steigerung Ihrer Grenze. Leider kann ich Ihnen dazu keine Anleitung geben. Üben Sie untereinander und finden Sie heraus, was Ihnen am besten hilft. Den entsprechenden Spruch finden Sie auch im Buch.} Er ließ der Klasse etwas Zeit, damit diese ihrem Professor folgen konnten. \enquote{Damit ist auch ein gewisses Opfer verbunden. Ich glaube aber, dass ich Sie gut genug kenne und Sie mich, dass Sie verstehen, warum ich das tun werde. Ich selber werde dabei nicht im Raum sein. Der Zauber ist so ausgelegt, dass er selbstständig abbricht, wenn Sie einen bestimmten Punkt erreicht haben. Dann wissen Sie auch gleichzeitig ihre Grenze.} Er verschwieg der Klasse allerdings ein paar wichtige Details, welche den Schülerinnen und Schülern erst später klar werden würden.

Nach einer Weile nickte die Klasse. Mit beiden Händen, die er mit den Handinnenflächen nach oben hob, stand die Klasse auf. Die Tische schwebten an die Decke und die Kleider der Schüler ordneten sich sauber hinter ihnen an. Der Unterricht würde in einer viertel Stunde beendet sein, das war jedem klar.

\enquote{Stellen Sie sich in einem Kreis auf. Abwechselnd Junge und Mädchen.} Als die Klasse kreisförmig im Raum stand, sagte er: \enquote{Drehen Sie sich um.} Jeder Schüler drehte sich und schaute nun vom Kreismittelpunkt weg. Elber ging durch den Raum zur Tür des Klassenzimmers. \enquote{Ich werde draußen warten und Ihnen die Anweisungen per Patronus übermitteln. Ich selber werde dem Rest nicht beiwohnen. Testen Sie Ihre Grenzen aus}, sagte er und verließ das Zimmer.

Wenige Sekunden später kamen viele kleine fliegende Insekten herein und verteilten sich im Raum. Vor jedem Schüler ein Insekt. Obwohl alle gleichzeitig sprachen, hörte jeder nur das Insekt vor ihm, mit der Stimme des Professors sprechen.

\enquote{Bitte entkleidet euch komplett, verdeckt eure Scham und dreht euch um. Setzt euch danach im Schneidersitz auf den weichen und warmen Boden und kämpft gegen den aufsteigenden Drang an.}

Zuerst standen die Schüler in einer Art Schockstarre da. Doch nach einer knappen halben Minute begann der erste sich zu entkleiden. Die anderen folgten ihm. Einige zuerst zögerlich, dann aber doch zunehmen schneller, da ihnen bewusst wurde, dass sie zu viel von sich denen gegenüber preisgeben würden, die schon saßen und sie daher beobachten konnten, denn zum Ausziehen der Unterwäsche musste man sich bücken, oder nacheinander beide Beine heben. Als alle, mit den Händen vor der primären Scham, auf dem Boden saßen und sich anblickten, bemerkten die Mädchen, dass es ihnen nichts ausmachte, den Jungs ihre Brüste zu zeigen.

Langsam begann der Drang, seine Hände wegzunehmen und an den Seiten des Körpers abzulegen, immer stärker zu werden. In gleichem Maße nahm der Drang, den Schneidersitz aufzulösen und die Beine auszustrecken, zu. Es dauerte knappe sieben Minuten, bis die erste Schülerin dem Drang nachgab, ihre Hände von ihrer Scham nahm und sich mit ausgestreckten Beinen und abgestützten Beinen leicht nach hinten lehnte.

%In der Geschwindigkeit, in der der Drang nachließ, kehrte die Erkenntnis ein, dass die anderen sie so nicht sahen. Die anderen sahen sie immer noch, wie sie im Schneidersitz und die Hände vor ihrer Scham dasaß. Sie konnte also nicht feststellen, wer bereits dem Drang erlegen war.
Nun wusste sie, dass sie sich anziehen und den Raum verlassen konnte.

Draußen traf sie auf ihren Lehrer. Langsam machte sich in ihr die Erkenntnis breit, dass sie sich bald nicht mehr an die genauen Details der erspähten Geschlechtsorgane der anderen erinnern würde.

\enquote{Warum wollten Sie nicht bei uns im Raum sein, Professor?}, fragte sie ihn.

\enquote{Auf den, der den Zauber ausspricht, hat er nicht dieselbe Wirkung, wie auf Sie}, antwortete er einfach.

Nach und nach kamen immer mehr Schüler aus dem Raum heraus, bis fünf Minuten vor dem Ende der Stunde der letzte Schüler den Raum verlassen hatte.

\enquote{Üben Sie kräftig}, sagte der Professor und verabschiedete sich von seinen Schülern.

\enquote{Das war eine spannende Stunde}, meinte Lavender. \enquote{Und wenn ich an Rons Gemächt denke\abs}, meinte sie, doch die Erinnerung daran wie es aussah, verblasste bereits. Nur nicht, dass sie es gesehen hatte.

Keiner der Schüler bemerkte wie der Professor mit einem dicken Grinsen im Gesicht durch das Schloss ging.

\enquote{Was haben Sie?}, fragte Professor Flitwick.

\enquote{Ich habe heute meine spezielle Stunde gehalten, über die wir geredet hatten.}

\enquote{Oh}, meinte Flitwick und grinste dann genauso breit. \enquote{Meinen Sie, die Schüler können Ihre Erinnerungen wieder zurückholen?}

\enquote{Ich habe Sie mit dem Vergessenszauber nicht daran gehindert. Wenn es eine oder einer bemerkt, dann kann sie oder er sich glücklich schätzen. \gst Wissen Sie, Filius, das ist es, was die Magie so spannend macht und mir die Stelle als Lehrer erträglich.}

Nun grinste Filius noch mehr. \enquote{Was werden die Schüler noch herausfinden?}

\enquote{Wenn sie gut sind, werden sie sich an jeden nackt erinnern können, hoffe ich mal.}

\enquote{Und Sie?}

\enquote{Ich war draußen. Ich habe schon genug damit zu kämpfen, die Magie um meine Schüler herum zu ignorieren.}

\enquote{Ach ja, das Phänomen. \gst Und bei Aurora?}

\enquote{Da konnte ich mich ein paar mal nicht konzentrieren.}

\enquote{Das heißt?}

\enquote{Nackt und in Falschfarben.}

\enquote{Und in Ihren Träumen?}

\enquote{Werden diese manchmal aufgehoben. \gst Ich brauche jedes Mal am Morgen danach eine viertel Stunde, um mit Okklumentik diese Bilder zu verdrängen. Sie würden mich sonst von meiner Arbeit abhalten.}

\enquote{Kann man das eigentlich lernen?}

Elber sah Flitwick an. \enquote{Ich würde es liebend gern abschalten können. Aber wenn Sie mich schon fragen\abs} Er berührte ihn kurz an seinem Kopf und sagte etwas auf Hebräisch. \enquote{Viel Spaß die nächsten vierundzwanzig Stunden}, meinte er noch, bevor er abbog.

Filius strich sich sauer über seinen Kopf, da es unangenehm war, wenn jemand über den eigenen Kopf strich und man das nicht erwartet hatte. Er verstand nicht, warum Elber das gemacht hatte, doch keine fünf Minuten später kam ihm die Erkenntnis, als er in das Lehrerzimmer eintrat.

Als er Elber nach Ablauf der Zeit in seinem Zimmer im Schloss besuchte, meinte er nur: \enquote{Jetzt weiß ich, warum Sie das als Fluch sehen. Die letzten drei Stunden hatte zwar ein Gewöhnungseffekt eingesetzt, aber dieser wird immer wieder aufgehoben oder abgeschwächt. Es ist einfach lästig.}




\begin{kommentar}
Vor über tausend Jahren gebar Friedwards Frau Persope Vierlinge, die sie Rowena, Salazar, Godric und Helga nannte. Damit ist schon mal alles gesagt. Und wenn man den ganzen kleinen Hinweisen folgt, wird man spätestens in den ersten Kapiteln des nächsten Teiles direkt mit der Nase darauf gestoßen, dass Friedward Frederick ist. Darum kennt er sich im Schloss auch so gut aus.
\end{kommentar}

\begin{kommentar}
Wieder zurück in der Gegenwart wachsen an Petunias Geburtstag Petunien. Jene Blumen, die Harry gepflanzt hatte. Das brachte Petunia zum Nachdenken. »Ich sollte mal wieder …« Der Satz, den sie nie vollendete, endet wie folgt: »Lilys Grab besuchen.« Warum, das wird klar, wenn sich Harry Snapes Erinnerungen ansieht, die er im nächsten Teil bekommt.
\end{kommentar}

\begin{kommentar}
Dann wacht Harry in der Krankenstation auf und fragte sich, warum er von Hexen und Zauberern mit ihren Zauberstäben bedroht wird. Aber schnell wird ihm klar, dass er mit Voldemort den Körper getauscht hat. Um zu wissen, wie es weitergeht, meint McGonagall: »Schlag du was vor.« Ein kleiner Satz, den einer der Geier aus dem Dschungelbuch gesagt hatte, als diese nicht wussten, was sie tun sollten.
\end{kommentar}

\begin{kommentar}
Nachdem Harry bei dem Entfernen der Baumstümpfe betrogen hatte, musste er die Pokale von Hogwarts polieren. Eine Arbeit, die Ron in den Büchern auch schon machen musste.
\end{kommentar}

\begin{kommentar}
Etwas später führt Elber einen Zauber aus, der den Effekt des Imperius-Zaubers nachahmen soll. Erst später wird Harry erfahren, wie man Zauber selber erstellt. Was Eber also hier macht, ist einen selber erstellten Zauber anzuwenden und diese Technik erst viel später anderen beizubringen.
\end{kommentar}

\chapter{Wechselhaftes}


\enquote{In wenigen Tagen beginnen die Osterferien. Die Schüler, die sich bereits eingetragen haben, um hier ihre Ferien zu verbringen, tragen sich noch heute in einer der beiden neuen Aushänge ein, um entweder dieses Jahr Ostereier zu suchen, oder bei den Vorbereitungen zu helfen. Professor Sinistra wird dem Vorbereitungs-Team morgen Einweisungen geben. Tippen Sie einfach eine der beiden Listen freiwillig an und Ihr Name erscheint auf der entsprechenden Liste. Und jetzt wünsche ich Ihnen einen guten Appetit und eine angenehme Nacht}, schloss Professor McGonagall ihre Rede, setzte sich und begann zu essen.

\gedanke{Ostervorbereitungen!}, ging Harry durch den Kopf. \gedanke{Was mich da wohl erwartet?}

\enquote{Auf wessen Mist das wohl wieder gewachsen ist?}, fragte Hermine.

\enquote{Na auf wessen Wohl}, antwortete Ron. \enquote{Du hast sie doch gehört.}

\enquote{Das sagt nichts aus. Sie hat nur eine Ankündigung gemacht. Es könnte sonst wer sein. Ich denke nicht, dass es Sinistras Idee ist.}

\enquote{Auf jeden Fall hat sie sich bereit erklärt, die Organisatoren einzuweisen}, antwortete Harry.

Nach dem Essen trug er sich in die Liste der Vorbereiter ein. Er hatte früher genug Gelegenheiten, Eier zu suchen und einen Teil davon selber zu essen, bevor er sie an Dudley abgeben musste, oder der sie ihm wegnahm.

Hermine und Ron hingegen waren über die Ferien zu Hause, sodass sie sich nicht in eine der Listen eintrugen.

Harry streifte an seinem schulfreien Sonntag durchs Schloss, als er durch eine geschlossene Klassenzimmertür die Stimme seines Vgddk-Lehrers hörte. Er kam gerade von Adrian. Er hatte ihm mitgeteilt, dass es nach den Ferien losgehen würde. Er hatte sich mit der DA geeinigt, dass sie ab sofort in einem leeren Klassenzimmer üben würden. \enquote{Gut so, Draco. Weiter so. Du machst das sehr gut.} Vorsichtig näherte er sich der Tür und linste durch das Schlüsselloch. Er konnte nur die hintere Hälfte eines verschlissenen Umhangs erkennen. Aber es war eindeutig Professor Elber. Aber so sehr er sich auch anstrengte, er konnte nicht mehr sehen. Er hörte nur noch ein Stöhnen und Schnaufen. Dann einen Schrei, der von Draco kam, vermutete er. Sowie einen Plumps. \enquote{Sehr schön, Draco}, sagte Professor Elber und verschwand aus Harrys Sichtfeld. \enquote{Ruhe dich aus. Bald bist du so weit, dass du es länger beherrschst.}

Er spürte auf seinem Rücken eine Hand und erschrak. Es war Dumbledore. \enquote{Albus}, stammelte Harry.

\enquote{Na}, antwortete Dumbledore, \enquote{am lauschen?} Er drückte ihn sanft zur Seite und schaute ebenfalls durch das Schlüsselloch. Beide hörten nun, wie Professor Elber etwas aufhob und danach ein knarzendes Geräusch, als ob sich jemand auf ein Sofa oder in einen Sessel setzt.

\enquote{Ruhe dich aus. Du hast die Erholung notwendig.} Dann hörte er Schritte. Professor Dumbledore richtete sich auf und griff an den Türknauf. In dem Moment öffnete jemand von der anderen Seite die Tür und stieß fast mit Dumbledore zusammen. \enquote{Da drinnen ist leider besetzt}, antwortete er, als er einen halben Schritt zurückging und die Tür hinter sich schloss. Harry fiel das eigenartige Geräusch auf, das die Tür machte, als sie ins Schloss fiel. \enquote{Sie werden sich wohl einen anderen Ort suchen müssen.} Er schritt an Dumbledore vorbei und machte sich den Gang entlang in Richtung des großen Tores.

\enquote{Frederick}, rief ihm Dumbledore hinterher. \enquote{Wer ist da drin?}

\enquote{Niemand.}

Dumbledore legte seine Hand an die Tür, zog sie kurz zurück und nahm dann seinen Zauberstab heraus. Professor Elber war inzwischen verschwunden. Professor Dumbledore schwenkte seinen Zauberstab, doch die Tür öffnete sich nicht. Er dachte nach. Harry sah an seinem Gesicht, dass er überrascht war, obwohl er es nicht zeigte. Er sah Harry an. Dann schwenkte er wieder seinen Zauberstab und die Tür gab ein leises \geraeusch{Klack} von sich. Vorsichtig betraten sie den Raum.

Doch er war leer. Keine Menschenseele war zu sehen. \enquote{Hast du nicht auch was gehört?}, fragte er Harry.

\enquote{Doch, Albus. Er nannte ihn Draco. Aber wo ist er?} Sie sahen sich intensiv in dem leeren Raum um. \enquote{Glaubst du, er ist durch einen Geheimgang verschwunden?}

\enquote{Soweit ich weiß, gibt es zu diesem Zimmer keinen Geheimgang.}

Harry kam das Ganze komisch vor. Er lief zurück zur Tür und stellte sich bei offener Tür in den Türrahmen. Dann beugte er sich so weit herunter, dass er auf Höhe des Schlüsselloches war. Er bildete mit seinen Fingern einen Ring und sah hindurch. \enquote{Das sieht mir nicht so aus, als ob es der Raum ist, den ich zuerst gesehen habe.}

Dumbledore drehte sich um und ging nun ebenfalls in Harrys Stellung. Harry musste grinsen, als er seinen Schulleiter so stehen sah. \enquote{Du hast recht, Harry. Das ist ein anderer Raum.} Dumbledore schloss die Tür. Er blieb stumm einige Sekunden stehen. Er drehte sich zu Harry um und meinte dann: \enquote{Eigenartig. Na ja, ich gehe erst mal nach Hogsmeade. Sicherheitsüberprüfungen und Vorbereitungen für eure Apparierstunden.} Er ließ Harry alleine stehen.

Dieser machte sich auf den Weg zur Großen Halle, wo die Einweisung der Eierverstecker stattfinden sollte. Unterwegs richtete er noch eine der Messinglampen an der Wand, indem er sie absprengte und eine neue an die Wand zauberte.

\enquote{Schön, dass Sie alle da sind}, sagte Professor Sinistra, als die Schüler und Lehrer, die sich in die Liste eingetragen hatten, vor ihr standen. Die Türen der Großen Halle schlossen sich und eine große aufgerollte Pergamentrolle begann zu schweben, als Professor Sinistra mit ihrem Zauberstab auf sie zeigte.

Sie zeigte die Ländereien von Hogwarts und viele eingezeichnete rote, blaue, grüne, gelbe, violette, orange und rosa Punkte, die wohl die Verstecke der Eier darstellen sollte.

\enquote{Wie kommen Sie eigentlich darauf, Ostereier zu verstecken?}, fragte eine Ravenclaw aus Harrys Jahrgangsstufe. Er war sich nicht sicher, meinte aber, dass sie Helen hieß.

\enquote{Ich persönlich kannte den Brauch nicht, bis ich davon in unserem Lehrerzimmer gehört hatte. Dieser Brauch faszinierte mich derart, dass ich mich einzulesen begann. Bei den Muggeln werden die Süßigkeiten nur gesucht und können dann verspeist werden. Ich dachte mir, dass wir das Ganze etwas abwandeln. Bei uns wird es nicht nur eine Suche, sondern auch eine Art Rätsel. Die Schüler sollen die Eier nicht nur suchen und finden, sondern auch Zauber überwinden, um an sie heranzukommen.} Professor Sinistra lief auf das Pergament zu und zeigte mit ihrem Zauberstab auf die einzelnen Bereiche. \enquote{Diese Bereiche hier sind für die einzelnen Jahrgangsstufen. Die genauen Verstecke und Positionen habe ich auf extra Pergamenten festgehalten. Eure Aufgabe ist es, heute die Eier zu verstecken. Die Schüler und Lehrer, die suchen werden, werden eine dreiviertel Stunde in der Großen Halle festgehalten, bis die Versteckaktion beendet ist. Bis hierher Fragen?}, schloss Professor Sinistra.

Es dauerte eine Weile, doch dann schüttelten die ersten ihre Köpfe.

\enquote{Gut, dann kommen wir zur Verteilung der kleinen Gruppen. Die älteren unter ihnen werden die unteren Jahrgangsstufen beim Suchen überwachen und die Verstecke entsprechend präparieren. Dies machen wir so, weil die unteren Jahrgangsstufen nicht über die entsprechenden Zauber verfügen, die notwendig wären, um die entsprechenden Schutzzauber auszuführen. Ich teile Sie also den entsprechenden Gruppen zu.}

Auf dem Boden erschienen Kreise mit den entsprechenden Farben und kleine Tische daneben mit Pergamentrollen in den passenden Farben. Professor Sinistra teilte danach die ersten Gruppen ein, die die Jahrgangsstufen eins bis vier vorbereiten mussten. Die Schüler stellten sich in die genannten Kreise und warteten ab.

\enquote{Schauen Sie bitte schon einmal auf Ihre Pergamente, damit sie die Positionen und Zauber kennenlernen. Falls etwas unklar sein sollte; ich komme nachher vorbei.}

Die Schüler nahmen sich die Pergamente und betrachteten diese, während Professor Sinistra weitere Einteilungen vornahm. Die beiden Lehrer steckte sie in die schwarze Gruppe, um für ihre Kollegen die Eier zu verstecken. Nachdem alle bis auf Harry verteilt waren, kam Professor Sinistra auf ihn zu.

\enquote{Mister Potter?}, fragte sie leise. \enquote{Trauen sie sich zu in der Lehrergruppe zu arbeiten? Ich habe das Gefühl und vor allem einiges gehört, dass ihnen die entsprechenden Zauber keine Probleme bereiten dürften.}

Harry war erstaunt, ob der großen Verantwortung und des großen Vertrauens, das ihm da entgegengebracht wurden.

\enquote{Ich kann es versuchen, Professor. Aber, wieso trauen Sie mir das zu?}

\enquote{Wissen Sie, ich habe im Lehrerzimmer viele Dinge über Sie gehört, die mich vermuten lassen, dass Sie dazu in der Lage sind. Sie könnten es durchaus schaffen, dass einige meiner Kollegen und Kolleginnen zu knabbern haben. Außerdem habe ich vor, selber teilzunehmen. Ich habe ein paar Verstecke im Auge, die ich speziell für mich angelegt habe.} Sie übergab Harry ein Pergament. \enquote{Ich habe danach mein Gedächtnis verändert, damit ich während der Suche die Verstecke nicht kenne und auch suchen muss. Ich habe für mich einen eigenen Bereich festgelegt, den Sie bitte zusätzlich für mich präparieren.}

Harry machte große Augen und nickte schließlich. Das Pergament würde er hüten wie einen Augapfel. Zusätzlich nahm er sich vor, mit einigen zusätzlichen Zaubern, um die er Salazar bitten würde, die Verstecke zu sichern. Er stellte sich zu den beiden Lehrern und wartete ab.

\enquote{Stehst du nicht falsch?}, fragte ihn Zacharias, den er aus der DA kannte.

\enquote{Ich wurde hier hingestellt, also nein}, antwortete er.

Professor Sinistra fragte in die Runde, ob noch etwas unklar sein, doch alle verneinten, nachdem auch den letzten Gruppen Gelegenheit gegeben wurde die Pläne zu studieren. Das große Pergament wurde zusammen gerollt und zur Seite gelegt. Jeder der Schüler und Lehrer warf einen Sicherungszauber darauf, damit keiner vor Ablauf der Suchaktion spicken konnte.

Dann wurden sie in die Freiheit entlassen und die suchenden Schüler und Lehrer kamen in die Großen Halle. Die Türen wurden verschlossen und die Gruppe begann, die Eier und andere Süßigkeiten zu verstecken.

Harry lief mit Professor Flitwick und Professor McGonagall in das ihnen zugewiesene Gebiet und begann die Eier und andere Süßigkeiten zu verstecken. Das Gebiet war durch Bänder und andere Stangen abgesteckt, damit es zu keinen Missverständnissen kommen konnte. Harry belegte sein Versteck mit den geforderten Zaubern. Bei einigen hatte er Mühe und fragte Professor Flitwick, ob sie auch richtig ausgeführt wurden. Nach einer kurzen Kontrolle und einer entsprechenden Bestätigung machte er sich an das nächste Versteck.

Nachdem die Eier aus Harrys Gruppe versteckt waren, ging er zu seinem speziellen Bereich und begann die Sachen für Professor Sinistra zu verstecken. Er sah auf dem Pergament nach und legte diverse Schutzzauber über die Eier. Dann kam ihm eine Idee, wie er die Eier schützen konnte. Er fragte Salazar nach dem entsprechenden Zauber und legte ein streng kontrolliertes lebendiges Feuer über die Eier. Diese färbte er so, dass sie wie Schlangeneier aussahen und das Feuer nur aus seinem Versteck kam, wenn man die Eier ergreifen wollte. Ein anderes Versteck präparierte er so, dass man der Meinung war, dass ein Dementor die Sachen bewachen würde. Es war zwar nur ein einfacher Illusionszauber, aber Harry legte noch einen Kältezauber über das entsprechende Gebiet. Da das Gebiet am See lag, legte er ein Ei, geschützt durch einen Zauber, unter Wasser und vermerkte die Stelle auf dem Pergament. Nun hatte er eine Stelle mehr, als ihm vorgegeben war. Er würde es merken, wenn Professor Sinistra schummelte, da ein Versteck übrig blieb.

Die Gruppen trafen sich wie verabredet am Sammelpunkt und Harry wurde bestimmt, die Sucher zu holen, während die anderen wieder in ihre Bereiche zurückgingen und die Suche überwachten. Harry öffnete die Tür und die Sucher wurden von Harry und einer Karte, die bereits dort schwebte und nur farbige Bereiche zeigte, entsprechend den Jahrgängen eingeteilt. Harry ging mit Professor Sinistra und den anderen Lehrern zuerst in den Lehrerbereich, bevor er mit Professor Sinistra alleine in den für sie gesonderten Bereich zum Suchen ging. Da sie auch aus ihrem Gedächtnis die Positionen der Verstecke für die Lehrer entfernt hatte, begann sie dort mit der Suche. Allerdings hörte sie nach einem erfolgreich entdeckten Versteck auf.

Die Lehrer fanden die Verstecke relativ schnell \gst verglichen mit den Schülern \gst doch hatten sie zu tun, um an ihren Preis zu kommen. Den meisten Spaß hatte Dumbledore, meinte Harry. Er freute sich wie ein kleines Kind während der Suche und auch während der Erforschung der Schutzzauber. Harry war gespannt, ob Dumbledore es gleich merken würde, da er dieses Versteck schützte und ihn zusätzlich mit einem kleinen Illusionszauber, der die Position um einige Zentimeter verschob, belegt hatte. Dumbledore fiel prompt darauf rein und griff ins Gras, als er das Körbchen mit Eiern holen wollte.

Harry verkniff sich ein Lachen und schmunzelte still in sich hinein.

\enquote{Haben Sie noch einen Zauber darübergelegt?}, fragte ihn Professor Flitwick leise.

Harry sah zu ihm und nickte. Er ging in die Hocke und flüsterte in sein Ohr: \enquote{Ich habe einen einfachen Illusionszauber darübergelegt. So sieht es aus, als ob das Körbchen ein paar Zentimeter weiter entfernt sei.}

Professor Flitwick lächelte ebenfalls in sich hinein.

Als alle Stellen im Lehrerbereich gefunden wurden, kam Professor Sinistra auf Harry zu und sie gingen zusammen in den gesonderten Bereich. Dumbledore und McGonagall folgten ihnen, warteten außerhalb und beobachteten die beiden.

Professor Sinistra machte sich an die Arbeit und besah sich ihr Gebiet. Sorgfältig suchte sie entsprechende Stellen ab, an denen sie ein Versteck vermutete. Die ersten drei fand sie recht schnell und hatte nach wenigen Minuten die Schutzzauber entfernt. Es blieben nur noch drei übrig.

An einer Stelle blieb sie stehen und schüttelte sich. Sie lief bereits weiter, als sie sich anders besann und noch einmal umdrehte. Stirnrunzelnd und mit gezogenem Zauberstab näherte sie sich einem Busch, aus dem plötzlich ein Dementor hervorkam. Vor Schreck zuckte sie einige Schritte zurück und die anderen Lehrer zogen sofort ihre Zauberstäbe. Harry errichtete einen Schild zwischen sich und den anderen Lehrern, damit sie nicht eingreifen konnten.

\enquote{Mister Potter, was soll das, sehen Sie nicht, dass ein Dementor Professor Sinistra angreift?}

Doch Harry ignorierte seine Hauslehrerin.

Professor Sinistra konzentrierte sich auf den Dementor und warf einen nebelartigen Patronus auf den Dementoren, der aber durch die Erscheinung glitt und keinen Effekt hatte. Sie stutze erneut und brauchte ein paar Sekunden, bis sie begriff, dass es kein echter Dementor war. Sie überlegte kurz und warf dann einen Zauber auf das Abbild, das daraufhin verschwand. Dann sprach sie den Gegenzauber, der die Kälte auflöste und die Temperatur wieder wärmer werden ließ.

\enquote{Klever, Mister Potter. Ein Kältezauber und eine Illusion eines Dementoren. Ich dachte tatsächlich, dass es ein echter war. Das muss ich mir merken.}

Harry entfernte den Schild, den er aufgebaut hatte, und sah entschuldigend zu seiner Hauslehrerin. Diese zog nur eine Augenbraue hoch und zeigte ein kaum erkennbares Lächeln. Harry lächelte zurück. Dann sah er wieder zu Professor Sinistra, die sich dem Busch näherte und ein Nest entdeckte. Das zweite Versteck.

\enquote{Das sieht aus wie Schlangeneier}, sagte sie, ging in die Hocke und streckte ihre Hand nach den Eiern aus. Doch sie zog sie sofort wieder zurück und fiel auf ihren Hintern, da sie so sehr zurückschreckte, dass sie das Gleichgewicht verlor.

Eine kleine brennende Schlange kroch zwischen den Ästen hervor und legte sich schützend über die Eier. Züngelnd wartete sie auf weitere Aktionen.

\enquote{Mit so etwas hätte ich nicht gerechnet, Mister Potter. Erst ein Dementor, dann noch eine brennende Schlange.}

Jetzt begann sie zu grübeln und laut zu denken. \enquote{Ich nehme einfach mal an, dass es sich um keine Illusion handelt. Das hatten wir schon mal. Aber welcher Zauber reagiert so lebendig?}

Harry sah mal wieder zu seinen Lehrern, die Reaktionen wie Überraschung, Überlegen, oder Unkenntnis zeigten. Harry fühlte sich gut. Er hatte seinen Lehrern eine Herausforderung gestellt, da jeder nachzudenken schien. Er sah wieder zu Professor Sinistra, die inzwischen auf dem Boden saß und auch nachdachte.

\enquote{Versuche es mal mit Wasser}, meinte Professor Flitwick.

\enquote{Ok}, antwortete Professor Sinistra. Sie nahm ihren Zauberstab und warf einen Wasserschwall auf die Schlange. Doch die kleine Schlange zeigte sich davon unbeeindruckt.

Harry sah wieder zu seinen Lehrern. Dumbledores Mimik zeigte etwas, wie ein Gedanke, der Form annahm und zu einer Lösung führte. Sein Blick wanderte zu Harry. Dann gingen beide Augenbrauen nach oben und ein leises Lächeln umspielte seine Mundwinkel.

\enquote{Denk an unseren \accentuate{Todesser}-Lehrer}, sagte er.

Mit unverständigem Blick sah Professor Sinistra zuerst zu Dumbledore, dann zu Harry, der lächelte. Dann zeichnete sich auch auf Professor Sinistra Gesicht eine Erkenntnis ab. Sie brach den Fluch des lebendigen Feuers und konnte endlich ihren Schatz bergen.

\enquote{Das war vielleicht eine Herausforderung. Wie kommen Sie denn dazu, Dämonenfeuer anzuwenden? Und vor allem, woher kennen Sie den Zauber?}

\enquote{Lebendiges Feuer, Professor. Und ich hatte eine Eingebung.}

\enquote{Ich hoffe doch mal, das war das letzte Versteck.}

\enquote{Nein}, antwortete Harry. \enquote{Eines ist noch übrig.}

Professor Sinistra ging noch einmal das gesamte Gebiet ab, fand aber nichts. \enquote{Fehlt wirklich noch ein Versteck?}

\enquote{Ja, Sie sind allerdings noch nicht das gesamte Gebiet abgegangen}, gab er ihr als Hilfe.

Professor Sinistra runzelte erneut ihre Stirn und sah sich um. Nach einer ganzen Weile sagte sie: \enquote{Der See.} Sie ging an den Rand des Sees und fand nach einigen Minuten suchen ein kleines Nest aus Eiern und einem Osterhasen. Überlegend und sich ihr Kinn reibend stand sie da. Harry konnte sein Lachen kaum noch zurückhalten, blieb aber trotzdem stumm. Er hielt sich seinen Bauch, da er wusste, dass keinerlei Zauber über dem letzten Nest lag. Sie musste einfach nur ihre Hände in das Wasser tauchen und das Nest an sich nehmen. Das Nest war lediglich durch einen Zauber vor magischem Aufrufen geschützt.

Nach einigen Minuten und sämtlichen misslungenen Versuchen, legte sie ihren Kopf schief und steckte ihren Zauberstab ein. \enquote{Kann es sein, dass\abs}, sagte sie und ging in die Hocke. Sie griff vorsichtig in das Wasser und holte das Nest heraus. \enquote{Ich fasse es nicht, keinerlei Schutzzauber. Einfach nur reingreifen und herausholen.} Sie lachte Harry an. \enquote{Mister Potter, das hat Spaß gemacht. Ich bin beeindruckt. Solche Zauber hätte ich Ihnen gar nicht zugetraut. Vor allem ein Nest ohne Zauber zu belegen und mit der Erwartung, dass sie geschützt sein müssen, zu spielen. Respekt!}

\enquote{Dann können wir ja jetzt wieder zurück}, schlug Harry vor.

Die anderen nickten und zusammen ging es Richtung Schloss.

\trenn

Während dessen tauchte Lucius Malfoy an anderer Stelle aus dem Nichts in der Nokturngasse auf. Kurz darauf spürte er eine Hand auf seiner Schulter. Erschrocken drehte er sich um. \enquote{Frederick, ich habe jetzt keine Zeit für dich. Der\abs}

\enquote{\aabs Dunkle Lord hat einen Auftrag für dich. Ja ich weiß.}

\enquote{Woher?}

\enquote{Unwichtig. Wie viel Zeit gibt er dir?}

\enquote{Das wird so vier Stunden dauern.}

\enquote{Gut, dann haben wir Zeit genug. Komm mit.}

\enquote{Aber ich muss meine Aufgabe erfüllen.}

\enquote{Ja ja}, antwortete Frederick und beide verschwanden.

Sie tauchten in einer kleinen Eingangshalle wieder auf. Sie war schlicht eingerichtet.

\enquote{Deine Sachen findest du in dieser Tüte. Deine Frau und dein Sohn warten im Wohnzimmer, deine Tochter ist in Hogwarts. Beim nächsten Auftrag schaue ich, dass ihr euch treffen könnt. Ich bin in der Küche und arbeite.} Dann öffnete er eine Tür und verschwand im Nebenzimmer.

Lucius stand die nächste Minute starr da und wusste nicht, was er tun sollte. Zum Glück war er alleine. So konnte niemand sehen, wie er um Fassung rang. Er sah auf die Tüte und schaute hinein, nachdem er auf sie zugegangen war. Es schien alles da zu sein. Dann legte er sie wieder zurück und betrat das Wohnzimmer.

\enquote{Lucius}, rief seine Frau erfreut. Sie stand auf und nahm ihn in den Arm.

Dann war Draco dran. Entgegen seiner sonstigen Art stand auch er auf und umarmte seinen Vater. \enquote{Schön, dich wieder einmal zu sehen, Vater.}

Dieses Mal konnte er nichts sagen. Seine Maske fiel fast zusammen und er umarmte seinen Sohn zum ersten Mal nach langen Jahren herzlich. Seine Frau gesellte sich zu ihnen. \enquote{Das ist gefährlich, was ihr hier macht. Der Dunkle Lord wird euch finden und mich bestrafen, wenn er herausfindet, dass ich bei euch war.}

\enquote{Du weißt aber nicht, wo wir sind, oder?}

\enquote{Nein.}

\enquote{Gut. Dann kann der Dunkle Lord dir nichts anhaben. Du wurdest ja mehr oder weniger gegen deinen Willen hier hergebracht.}

Die Tür ging auf und ein kleines Mädchen stürmte auf die Gruppe zu. \enquote{Vater}, rief Tamara und hing schon an Lucius’ Umhang.

\enquote{Tamara}, rief er und hob seinen kleinen Engel hoch. \enquote{Ich dachte, du bist heute nicht da.}

\enquote{Dachte ich auch, da heute im Schloss einiges los war. Wir haben\abs}

\enquote{Na, was habt ihr gemacht.} Sie zog den Kopf ein. \enquote{Sag schon.} Sie schüttelte den Kopf. \enquote{Hast du Angst vor mir?} Ein leicht verschämtes Nicken kam zurück. Lucius sah betrübt in ihr Gesicht. \enquote{Und wenn ich dir verspreche, dass ich nichts sage, oder tue?}

\enquote{Na ja}, fing sie an. \enquote{Wie haben Ostereier gesucht.}

\enquote{Was für Eier?}

\enquote{Ostereier. Das ist ein Brauch der Muggel.} Lucius verzog sein Gesicht, worauf Tamara leiser und schüchterner fortfuhr. \enquote{Da ist es Brauch, zu einer bestimmten Zeit im Jahr bunte hartgekochte Eier, oder welche aus Schokolade \gst Es gibt sogar welche in Form eines Hasen \gst zu suchen. Auf Hogwarts hat man, nachdem man eines entdeckt hatte, noch einen Schutzzauber lösen müssen.}

\enquote{Hauptsache, es hat dir Spaß gemacht}, sagte er jetzt versöhnlicher. \enquote{Wie ist es euch so ergangen?}, fragte er seinen Sohn und seine Frau.

\enquote{Zuerst du, damit du hinterher mit mehr Freude wieder gehst}, sagte seine Frau energisch.

\enquote{So viel Feuer hätte ich dir gar nicht zugetraut. Respekt}, meinte ihr Mann.

Währenddessen stand Frederick in der Küche und dachte über Lucius’ Situation nach. \gedanke{Er ist zu bemitleiden. Einerseits liebt er seine Frau und seine Kinder, andererseits hat er sich dazu entschlossen den Lehren und Predigten von Voldemort zu folgen. Aber ich denke, er bemerkt so langsam, was ihm das einbringt. Er ist ein Gefangener in seinem eigenen Haus. Er kann sich nicht von Voldemort abwenden, ohne dass er alles verlieren würde. \gst Sicher, er mag Muggel nicht besonders. \gst Ich hoffe nur, dass er eine Entscheidung trifft\abs die richtige.}

\trenn

\enquote{Woher hast du eigentlich den Zauber für das Dämonenfeuer?}, fragte Dumbledore.

\enquote{Lebendiges Feuer}, korrigiert Harry. \enquote{Wie wäre es mit einer Tasse Tee und ein paar Keksen?}, fragte er frech nach.

\enquote{Gerne. Bei mir, oder bei dir?}

Damit hatte Harry nicht gerechnet. Er überlegte kurz und sagte dann: \enquote{Bei mir. Gemeinschaftsraum.}

Dumbledore nickte und zusammen gingen sie Richtung Gryffindorturm.

\enquote{Kreacher}, rief Harry und der Elf erschien und lief dann neben den beiden her. \enquote{Wir brauchen eine Kanne Tee mit zwei Tassen und etwas Gebäck. Sorge bitte, dass sie im Gemeinschaftsraum der Gryffindors bereitstehen. Wir gehen gerade dorthin.}

Kreacher verneigte sich und verschwand.

\enquote{Du hast deinen Elfen ja gut im Griff}, sagte Dumbledore.

\enquote{Ja, das denke ich auch. Er hat sich, seit ich ihn habe, sehr verändert. Seine Hetzparolen sind verschwunden und ich habe den Eindruck, dass er jetzt glücklicher ist. Ich bin sogar davon überzeugt, dass es nicht mehr lange dauert, dann ist er mir gegenüber vollkommen loyal.}

Am Porträt angekommen, öffnete sich das Gemälde, ohne dass Harry das Passwort sagen musste.

\enquote{Sie wird nachlässig}, sagte Dumbledore.

\enquote{Nein, Professor. Ich bin so weit, ihr das Passwort per Legilimentik zu übermitteln. Nachdem ich wieder meinen Okklumentikunterricht aufgenommen habe, hat sich diese Fähigkeit auch herausgebildet. Bei Gemälden funktioniert das recht gut. Auch bei Geistern klappt das; nicht immer, aber dennoch recht gut.}

\enquote{Erstaunlich. Ich bin beeindruckt.}

Plötzlich spürte Harry ein vorsichtiges Tasten an seinem Geist. Harry blockte sofort ab und schickte Dumbledore ein Schockbild. Dadurch irritiert und geschockt, brach die Verbindung ab. Harry grinste in sich hinein.

Beide traten ein und setzten sich in zwei Sessel, die gegenüber standen. Harry schenkte beiden eine Tasse Tee ein und nahm einen Schluck.

\enquote{Die Idee für das lebendige Feuer hatte ich von Professor Elber, als dieser den Vortrag in der Großen Halle hielt. Er hat damit herumgespielt, wie mit einem gut erzogenen Haustier. Als ich die Nester versteckte, fiel mir das ein. Der Zauber und wie man ihn anwendet, ist mir plötzlich eingefallen. Ich dachte mir, das wäre ein toller Zauber um Professor Sinistra zu testen.} Vorsichtig zog er sein Amulett hervor. \enquote{Dann habe ich dem Zauber den Schutz der Eier aufgetragen. Im zweiten Versteck habe ich als Schutz einen Kältezauber mit der Dementorenillusion darüber gelegt. Das hat Professor Sinistra schön verwirrt.} Er grinste Dumbledore an, als er ihr Gesicht Revue passieren ließ.

\enquote{Ich weiß zwar jetzt, dass und warum du ihn angewendet hast, aber nicht, woher du den Zauber hast.}

Harry überlegte, was er sagen sollte und durfte. Da sie fast alleine waren und die, die im Raum saßen, beschäftigt waren, oder weit weg saßen, konnte er mit Dumbledore eine Unterhaltung unter vier Augen führen.

\stimme{Sag ihm, dass das Amulett dir Zauber übermitteln kann, wenn du sie brauchst. Dumbledore weiß, dass es meines war. Sag ihm, ich habe die notwendigen Zauber auf das Amulett gelegt.}

Innerlich nickte Harry. \enquote{Den entsprechenden Zauber hat mir das Amulett mitgeteilt. Slytherin hat es so verzaubert, dass mir bei Bedarf die notwendigen Zauber einfallen.}

\enquote{Interessant}, sagte Dumbledore und nahm einen Schluck Tee.

Wieder spürte er ein vorsichtiges Kratzen an seinem Geist. Harry war das langsam lästig, aber trotzdem eine gute Übung. Er zog an seinem Gegenüber und warf ihn in den runden Raum mit den vielen Türen, die sich sofort drehten. Mit gleichgültigem Gesichtsausdruck sah er Dumbledore in die Augen und versuchte seinerseits in den Geist seines Schulleiters einzudringen. Er erwartete keinen großen Erfolg, da Dumbledore für seine Legilimentik und Okklumentik-Fähigkeiten bekannt war.

Harry fand sich in einem Raum wieder. Dumbledore war ein Teenager und neben ihm war noch ein weiterer junger Mann und ein Mädchen. Harry wusste bereits, dass Albus einen Bruder hatte und eine Schwester, die aber in jungen Jahren verstorben war.

\gedanke{Das muss Ariana sein}, ging Harry durch den Kopf. Es kostete ihn eine Menge Kraft, für Dumbledore das Bild der Türen aufrechtzuerhalten und gleichzeitig etwas über ihn zu erfahren. Nach einigen Minuten entschied er sich, die Bilder die er empfing einfach zurückzuwerfen. Das entlastete ihn.

Dadurch wieder verwirrt und aufgeschreckt, zog sich Dumbledore aus Harrys Geist zurück. Doch er blockte Harry nicht, sodass er weiterhin zu sah. Ein weiterer junger Mann betrat den Raum. Der junge Albus begrüßte ihn und nannte ihn Gellert.

\gedanke{Grindelwald. Dumbledore kannte Grindelwald}, ging Harry durch den Kopf.

Dumbledore schien nicht zu bemerken, dass Harry immer noch in seinem Geist unterwegs war, oder er ignorierte es und wollte, dass Harry es erfuhr. Doch Harry war es egal. Er wurde Zeuge eines Streites zwischen den beiden jungen Männern, dem ein Duell folgte. Alles ging so schnell, dass Harry gar nicht mitbekam, wer den tödlichen Zauber ausführte. Aber nachdem Ariana tot am Boden gelegen hatte, endete der Streit. Gellert verließ das Haus und Harry hatte den Eindruck, dass Dumbledore ihn nur noch einmal sehen würde. Während des berühmten Duells, in dem Dumbledore ihn entwaffnete.

Harry sah noch die Trauer der beiden jungen Dumbledores, bevor die Szene verblasste und Harry sich zurückzog.

\enquote{Ich habe ihren Tod nie verwunden}, sagte Dumbledore.

Da wusste Harry, dass er ihn gewähren ließ. Oder er redete sich den Schmerz nur von der Seele, hatte aber nicht die Kraft, Harry bei genau dieser Szene hinauszuwerfen. Er würde es wohl nie erfahren, war aber für die Information dankbar.

\enquote{Sie haben sie geliebt, richtig?}, fragte er.

\enquote{Wir sind hier alleine, Harry.}

\enquote{Sie hat dir viel bedeutet. Wie war sie?}

\enquote{Sie war eine ruhige Person, die ihre Magie aber nicht kontrollieren konnte. Muggeljungen misshandelten sie in jungen Jahren, als sie einen Zauber ausführte und sie wissen wollten, wie sie das gemacht hatte. Mein Vater wurde deshalb nach Askaban geschickt, da er sich an ihnen rächte, wegen dem, was sie ihr angetan hatten.} Wieder nahm er einen Schluck Tee. \enquote{Danach ist meine Mutter mit uns dreien weggezogen. Sie starb kurz nachdem ich die Schule beendet hatte. Als ältester Sohn fühlte ich mich verpflichtet, mich um sie zu kümmern. Ich habe mich deswegen mit meinem Bruder gestritten, der mich nur als den Wunderjungen mit den hohen Zielen gesehen hatte. Er würde sich um Ariana kümmern. Ich solle mit meinem Freund Gellert um die Welt ziehen, so wie ich es vorhatte. Er sah mir an, dass mich Mutters Tod um meine Pläne brachte. Doch ich wollte bei ihr sein, mich um sie kümmern. Also beendete Aberforth die Schule. Er hätte sie sonst abgebrochen.}

\enquote{Aberforth? Der Besitzer des Eberkopfes?}

\enquote{Ja.}

\enquote{Welche Pläne, Albus?}

\enquote{Pläne für eine bessere Gesellschaft. Die Herrschaft der Zauberer über die Muggel.} Zum Glück hatte Harry gerade seine Tasse abgestellt, um sich Tee nachzuschenken, sonst wäre sie ihm aus der Hand gefallen. \enquote{Doch Arianas Tod hielt mir den Spiegel vor Augen. Wir waren nicht besser. Das änderte meine Jungendansichten dramatisch und ich wurde der, der ich heute bin. Ein Muggelliebhaber, wie man überall spottet.}

So langsam begriff Harry, dass sein Schulleiter eine dunkle Vergangenheit hatte. \gedanke{Kein Licht ohne Schatten, oder Dunkelheit}, dachte er sich. \enquote{Und der Kontakt zu deinem Bruder? Hat er sich verbessert?}

\enquote{Nicht besonders.}

\enquote{Du solltest dich darum kümmern. Zumindest aussprechen, oder den Versuch unternehmen. Die Zeit verläuft schnell.}

\enquote{Sehr schmeichelhaft, Harry}, sagte Dumbledore. \enquote{Du hast eine tolle Art, einem zu sagen: \inner{Du wirst alt.}}

\enquote{So war das nicht gemeint. Wenn man so etwas auf die lange Bank schiebt, dann ist man tot, bevor man sich\abs Du weißt, wie ich das meine.}

Plötzlich hörten beide einen kräftigen Faustschlag auf einem Tisch. \enquote{Verdammt, wieso klappt das nicht?}, meckerte ein Schüler.

\enquote{Ich glaube, ich werde gebraucht.}

\enquote{Dann ist unsere Unterhaltung wohl beendet.}

Dumbledore nickte, stellte seinen Tee ab und stand auf. Er ging zu dem Schüler, setzte sich und fragte ihn, wo denn das Problem sei.

Harry rief nach Kreacher, nahm sich noch einen Keks und ließ das Geschirr zurückbringen.

\trenn

Die vier Stunden waren beinahe vorbei und Tamara war seit einer halben Stunde wieder in Hogwarts, als die Wohnzimmertür aufging und Frederick hereinkam. \enquote{Es wird Zeit.}

Lucius nickte und stand auf. Er verabschiedete sich von seiner Frau und seinem Sohn. Ihn nahm er noch einmal in den Arm und seine Frau küsste er noch einmal stürmisch. Das hatte er die letzten Monate am meisten vermisst. Dann trat er auf den Flur und nahm seine Tüte in die Hand.

\enquote{Hier, trink das}, sagte Frederick und reichte ihm einen Becher mit einer Flüssigkeit.

\enquote{Was ist das?}

\enquote{Trink es einfach.}

Lucius leerte den Becher und stellte ihn danach ab. Dann ging eine merkwürdige Veränderung in ihm vor. Die Erlebnisse der letzten vier Stunden begannen in chronologischer Reihenfolge zu verblassen. Als er sich an die erste viertel Stunde nicht mehr erinnern konnte, kamen die ersten Minuten wieder. Im gleichen Zug verschwammen die nächsten Minuten des erlebten.

\enquote{Was war das für ein Trank?}, fragte er.

\enquote{Einer, der verhindert, dass der Dunkle Lord deine Erinnerungen an die letzten vier Stunden durchstöbern kann. Er wird das sehen, was du tun solltest.}

Er wurde wieder an der Schulter gepackt, verschwand und wurde direkt vor Borgin und Burkes abgesetzt. Nichts deutete darauf hin, dass ihn jemand mitgenommen hat. Der alte Borgin kam gerade aus seinem Laden, als er Lucius Malfoy entdeckte.

\enquote{Ah, guten Tag Mister Malfoy. Schön, dass Sie da sind. Ich habe das bestellte Teil hier. Kommen Sie.}

Irritiert darüber, folgte er dem alten Herrn in seinen Laden. Dieser griff unter die Theke und holte eine kleine Schatulle hervor. Er öffnete sie und entnahm ein Pergament, das er Lucius gab. Dieser entrollte es und erkannte die Schriftzeichen darauf. Es war genau das, was der Dunkle Lord von ihm wollte. Dann fiel ihm ein, dass er in seiner Tüte keine Schriftrolle gesehen hatte.

\enquote{Nehmen Sie sie mit. Sie ist schon bezahlt. Ihre Frau war heute schon da.}

Das machte Lucius skeptisch. \enquote{Schon bezahlt, sagen Sie? Was ist mit Ihnen los? Normalerweise würden sie doch nochmal kassieren.}

Der alte Borgin zog darauf hin an seinem Ärmel und enthüllte drei feine goldene, schon im Verblassen begriffene Linien. \enquote{Unbrechbarer Schwur}, sagte er.

Lucius begriff. Er bedankte sich und verließ den Laden. Kurz bevor er disapparierte, entdeckte er eine Person, die ihn zu beobachten schien. Vor seinem Haus tauchte er wieder auf. Er lief die Einfahrt entlang zu seinem Anwesen und die Türen öffneten sich automatisch, als er näher trat. Zielstrebig ging er in den Salon, wo die anderen, vor allem Voldemort, schon warteten.

Er wollte gerade etwas sagen, als ihn der Dunkle Lord aufhielt, indem er seine Hand hob. \enquote{Warte noch einen Moment, Lucius.}

Dieser gehorchte und nickte nur. Kurz darauf kam die Person, von er glaubte, sie gesehen zu haben, herein. Er sah, wie sie Voldemort nur zunickte und ihren Platz einnahm.

\enquote{Nun Lucius, hast du alle meine Sachen, die ich dich zu besorgen beauftragt habe?}

\enquote{Ja, Mylord}, antwortete Lucius und kam näher. Einzeln legte er die Teile auf den Tisch und Voldemort besah sie sich genau. Lucius spürte, wie sein Geist sondiert wurde. Der Dunkle Lord fand genau das, was er zu finden suchte. Lucius’ Einkäufe. Verteilt über verschiedene Orte quer durch England.

\enquote{Du kannst dich setzen, Lucius}, sagte er. Lucius tat, wie ihm befohlen wurde, und nahm seinen Platz ein. \enquote{Spuren von deiner Familie?}, fragte er ihn.

\enquote{Meine Eltern sind schon gestorben, mein Lord. Andere Verwandte habe ich nicht}, gab er selbstbewusst zurück. Er wusste, dass das eine glatte Lüge war, aber die Überzeugung, mit der er sie wieder gab, überraschte ihn doch. Zum Glück nur innerlich.

Voldemort besah sich seine Gegenstände. \enquote{Ihr könnt gehen}, sagte er und nahm die einzelnen Stücke genauer unter die Lupe. Ein dunkler Stein, eine Rolle Pergament. Eine Flüssigkeit in einer gläsernen Viole. Einen kleinen Holzstab, ein Löffel aus einem dunkel schimmernden aber dennoch glänzenden und fliesendem Material. Klauen und andere Teile von toten Tieren und ein paar Ratten. Diese waren allerdings für seine Schlange. \enquote{Essen, Nagini}, flüsterte er und warf ihr die Ratten in den Schlund.

Lucius und ein paar andere gingen hinaus. Dort traf er auch auf Snape. Zusammen verschwanden sie in Lucius’ Arbeitszimmer.

\trenn

Die Ferien waren ruhig, da kaum Schüler im Schloss geblieben war. Harry saß mit Tamara auf dem Boden im Gemeinschaftsraum. Sie spielten Monopoly. Arabella hatte Harry dieses Spiel nach Weihnachten geschickt. \accentuate{Vielleicht kannst du es brauchen}, schrieb sie damals in ihrem Brief. Nach mehreren Partien hatte Tamara Blut geleckt. Immer wieder bat sie Harry um eine Partie. Da außer Hausaufgaben, die er immer wieder machte, nichts anstand, nahm er sich die Zeit. Zudem machte es ihm Spaß. Diese Partie ging an Tamara. Harry störte das nicht, da er die letzten beiden gewonnen hatte.

\enquote{Noch eine Runde, Harry}, forderte Tamara.

\enquote{Es ist schon spät. Wir sollten ins Bett gehen.}

\enquote{Oh}, jammerte Tamara. Sie räumte das Spiel in Rekordzeit in die Schachtel und krabbelte auf Harry zu. Sie schmiegte sich mit ihrem Rücken an seine Brust und legte seine Hand  auf ihren Bauch.

Müde schloss sie ihre Augen und meinte: \enquote{Weißt du, Harry. Du bist wie ein Bruder für mich geworden. Ich habe dich sehr gerne.}

\enquote{Was ist mit Draco?}, fragte er nach, als er sich mit ihr auf die Seite fallen ließ.

\enquote{Der war schon immer mein Bruder und wird es immer sein. Aber dich habe ich neu hinzugewonnen. Willst du mein Bruder sein?}

Harry dachte nach. Er wusste nicht viel über magische Bindungen, aber er war vorsichtig.

\enquote{Weißt du Tamara, auch ich empfinde so. Aber noch möchte ich diese Bindung nicht eingehen.}

Tamara stutzte, dann begriff sie. \enquote{Keine Angst, ich will dich nicht magisch an meine Familie binden. \gst Würde ich das, wenn du ja gesagt hättest?}, fragte sie mit geschlossenen Augen und wachen Ohren nach.

Harry strich ihr über ihr Haar und gab ihr einen Kuss auf ihre Schläfe. \enquote{Keine Ahnung, meine kleine. Aber sicher ist sicher.}

Ein sanftes Lächeln zeichnete sich auf ihrem Gesicht ab. Dann dämmerte sie weg. Harry strich ihr noch eine Weile durch ihr Haar. Dann musste er sie ins Bett bringen. Er überlegte, ob er es schaffen würde, sie in ihr Bett zu bringen. Aber er konnte nicht zu den Mädchenschlafsälen gelangen. In einem Raum der Jungen könnte er sie unterbringen. Doch was, wenn sie mitten in der Nacht aufwachen würde? Also nahm er sie hoch und trug sie in seinen Raum. Er legte sie in das Bett neben sich, zog ihr ihre Robe und ihre Schuhe aus und deckte sie danach zu. Dann richtete er sich selber her und stieg in sein Bett, um zu schlafen.

Mitten in der Nacht spürte er jemand in sein Bett krabbeln. Er öffnete erst ein Auge, danach das andere. Tamara kniete an seiner Bettkante und hatte die Hände vor seinem Körper in die weiche Matratze gestemmt. Harry drehte sich auf seinen Rücken und sah sie im aufkommenden Licht der Petroleumlampen an.

\enquote{Was ist? Geht es dir nicht gut?}

\enquote{Doch, Harry.} Sie gab ihm einen Kuss auf die Wange. \enquote{Danke.}

\enquote{Wofür?}

\enquote{Dass du mich ins Bett gebracht hast. \gst Aber warum bin ich hier bei dir?}

\enquote{Ich konnte dich nicht in dein Zimmer bringen. Weißt du, Jungs können nicht zu den Mädchen. Und in einem anderen Zimmer dachte ich, hättest du Angst, wenn du mitten in der Nacht aufwachst und nicht weißt, wo du bist.}

Tamara lächelte ihn an. \enquote{Darf ich bei dir\abs?}

\enquote{Geh zurück in dein Bett und schlafe.} Tamara zog einen Flunsch. \enquote{Es sieht nicht gut aus, wenn du bei mir schläfst.}

\enquote{Aber es sieht doch keiner.}

Nun hörten beide ein Glucksen. \enquote{Ich sehe es.}

Beide sahen in die Richtung, aus der der Satz kam.

\enquote{Myrte, was machst du denn hier?}

\enquote{Ich dachte, ich schaue mal nach, ob du auch schläfst.}

\enquote{Um dann was mit mir zu machen?}

Da Tamara kalt wurde, legte sie sich hin und zog etwas an Harrys Decke.

\enquote{He, solltest du nicht in dein Bett?}

Myrte kam auf die beiden zu geschwebt. \enquote{Es scheint, als ob du nichts zu sagen hättest. Du bist in der Unterzahl.} Auch Myrte näherte sich Harry und schmiegte sich an ihn. Zumindest hatte es den Anschein.

\enquote{Raus}, sagte er energisch.

\enquote{Nein}, sagten beide gleichzeitig.

Er überlegte, wie er die beiden aus seinem Bett werfen könnte. \gedanke{Tamara kann ich kitzeln. Aber was mache ich mit Myrte? Einen Zauber?}

Während er noch so überlegte, hatten sich Tamara und Myrte bereits an ihn geschmiegt und fingen an zu schlafen. Harry merkte nicht, wie er ins Reich der Träume hinüberglitt. Als er Stunden später wieder wach wurde, dachte er wieder darüber nach, wie er seine beiden Damen aus seinem Bett werfen könnte. Für ihn waren erst wenigen Minuten vergangen. Er schlug beide Augen auf und merkte, dass die morgendliche Helle bereits in den Raum schien. Von beiden Seiten gewärmt, lag Harry in seinem Bett. Er drehte seinen Kopf nach links und entdeckte Tamara, die mit dem Bauch an seine Seite gekuschelt war. Auch rechts war es warm. Harry drehte seinen Kopf. Dort lag Myrte. Die Bettdecke auf ihr, was eigenartig aussah, da er durch ihren Körper durch Sehen konnte. Aber ihm war auf seiner rechten Seite nicht kalt. Im Gegenteil, von dort kam die gleiche Wärme, wie von links. Er spürte keinerlei Unterschied. Beide atmeten ruhig und gleichmäßig.

\enquote{Aufstehen meine hübschen}, sagte er und bewegte sich in seinem Bett, damit sie aufwachen würden. Zuerst schlug Tamara ihre Augen auf und sah ihn mit sandigen Augen an. Harry zog seine Hände hervor und wischte ihr vorsichtig mit seinem Finger den Sand aus den Augen, nachdem er seine befreit hatte. Danach streckte er seine Hände nach oben, ballte sie zu Fäusten und streckte sich. Das weckte auch Myrte auf. Von ihr bekam er einen Kuss auf die Backe und dasselbe  Lächeln, das sie schon so oft gezeigt hatte, wenn sie ihn gesehen hatte.

\enquote{Du bist ganz warm, Myrte}, sagte Harry und versuchte einen Arm auf sie zu legen. Doch wie immer glitt er durch. Ihr schien das nichts auszumachen. Sie stieg hoch und durch die Decke hindurch.

\enquote{Wie kommt es, dass die Decke auf dir lag, als ich heute Morgen aufgewacht bin?}

Myrte sah ihn fragend an und hob nur ihre Schultern. \enquote{Weiß ich nicht.}

Tamara stieg aus dem Bett heraus und ging zur Zimmertür. \enquote{In einer viertel Stunde unten im Gemeinschaftsraum? Frühstücken?}

\enquote{Gerne. Ich bin da.}

Tamara verschwand aus dem Zimmer und ging.

\enquote{Darf ich mit zum Frühstücken?}, fragte Myrte.

\enquote{Von mir aus}, antwortete Harry.

Selig lächelnd machte sie eine Luftrolle rückwärts und sah ihn abwartend an.

Harry stieg aus dem Bett und verschwand im Bad. Als er wieder kam, wartete Myrte schwebend über seinem Bett. Sie schwebte im Schneidersitz und wartete.

Harry verschwand hinter einer Wand und zog sich um. Gerade als er seine Unterwäsche anhatte, kam Myrtes Kopf durch die Wand und Harry erschrak.

\enquote{Myrte, etwas mehr An- und Abstand bitte.}

Sie gluckste wieder und zog sich zurück. \enquote{Ich warte unten auf euch}, sagte sie und verschwand durch die Wand.

Harry zog sich an und ging danach nach unten. Tamara kam gerade herunter, als er die unterste Stufe erreicht hatte. Zu dritt gingen, beziehungsweise schwebten, sie zur Großen Halle.

Einige Schüler und Lehrer staunten, als die drei in der Großen Halle ankamen. Tamara und Harry beluden sich ihre Teller und Myrte schwebte immer mal wieder auf Mundhöhe durch die Speisen. Die Erinnerung daran machte sie glücklich. Nach einem kurzen Mahl verabschiedete sich Tamara von Harry und machte sich auf den Weg zum Slytherintisch, wo sie den Rest ihres Frühstücks einnahm und sich mit einem Schulkollegen ihres Bruders unterhielt.

Den Rest der Ferien passierte es Harry noch ein paar Mal, dass er mit einer, oder beiden, Mädels die Nacht verbringen musste. Eigentlich fand er es ja ganz angenehm, die Wärme einer Person zu spüren, aber andererseits wollte er einen gewissen Schein wahren. Auch führte er während seiner Ferien ein paar Unterhaltungen mit Draco. Es waren Unterhaltungen, die sie in ihren Träumen führten und somit nicht abgehört werden konnten. Die nur über ihre Vergangenheit gingen. Wie Harry aufgewachsen war und wie Draco aufgewachsen war. Beide verstanden einander nun etwas besser und die Feindschaft zwischen ihnen wich so langsam einer neutralen Einstellung zueinander. Während dieser Kontakte erzählte er Draco, wie er bei der Auswahlzeremonie nach Gryffindor geschickt wurde.

\trenn

Weder Ron, noch Harry oder einer der anderen Schüler wussten, wen sie als Apparierlehrer hatten. Selbst keiner der Lehrer oder Hagrid ließ etwas raus. Vielleicht wussten sie es auch gar nicht. Heute sollte es sein. Nach dem Frühstück würden sie ihre erste Stunde im Apparieren haben. Hermine hatte vermutlich bereits alles gelesen, was es über das Thema zu geben schien. Denn sie redete ununterbrochen über die verschiedenen Appariertechniken und deren Unterschiede.

\enquote{Warte doch einfach ab}, gab ihr Ron einen Dämpfer, \enquote{weißt du noch, als du bei Professor Elber deine Antworten gegeben hast, die waren nur halb richtig. Würde mich nicht wundern, wenn es hier ähnlich laufen würde.} Ron hoffte, sie dadurch zur Ruhe zu bringen, was auch einige Zeit half, aber auf ihrem Weg nach Hogsmeade fing sie wieder an.

Gedankenverlorenen ging Harry Ron und Hermine hinterher. Luna war schon unten. Sie bildete eine Ausnahme, da sie mit Harry verbunden war. Zwar musste sie die Prüfung erst nächstes Jahr machen und auch Auffrischungsstunden nehmen, durfte aber schon jetzt dabei sein, da beide noch immer im Körper des anderen waren.

In der Hogsmeader Stadthalle angekommen, erwartete sie neben den anderen Schülern nur Professor McGonagall. Doch sie machte nicht den Eindruck, dass sie auf Schüler wartete. Sie machte auf Harry einen unruhigen Eindruck. Immer wieder sah sie auf ihre Uhr und lief unruhig hin und her. Gerade wollte sie zur Klasse sprechen, als aus dem Nichts hinter ihr Professor Elber auftauchte. Als sich Professor McGonagall umdrehte, erschrak sie. \enquote{Frederick, wie kannst du mich so erschrecken?}

\enquote{Nachdem du mich dazu gezwungen hast die Stunden zu unterrichten, musste ich es dir doch irgendwie zurückzahlen.}

Professor Elber machte einen leicht mürrischen Eindruck. \enquote{Hermine, ich nehme an, dass sie bereits wieder alles über das Apparieren gelesen haben?}

\enquote{Ja}, antwortete sie.

\enquote{Gut \gst welche Strecke könnten sie beim Apparieren auf einmal, also ohne abzusetzen, überwinden? Wie lange brauchen sie dafür? Und warum ist das so?} Alle schauten wie immer auf Hermine, die sicherlich die richtigen Antworten geben würde. Doch sie schien still. Nach einer Weile fragte Professor Elber in die Runde \enquote{Jemand anderes?}

Lange Zeit herrschte in der Runde stille. Gelegentliche Blicke zu Professor McGonagall ergaben auch nichts Sinnvolles. Plötzlich meldete sich Neville. \enquote{Ja Neville. Was denken Sie?}

\enquote{Ich meine, es müssten so ca. 20.000 km sein.}

Professor Elber zeigte sich erstaunt. Aber Professor McGonagall machte ein Gesicht, das sagte, das stimmt nicht. \enquote{Und wie genau kommen sie darauf?}

\enquote{Na ja, der Umfang der Erde beträgt etwa 40.000 km. Und da ich zum gegenüberliegenden Punkt nur die Hälfte brauche, sind das eben 20.000 km.}

\enquote{Ein interessanter Ansatz. \gst Wäre der direkte Weg durch den Planetenkern nicht kürzer?}

\enquote{Also ich möchte mich nicht durch über 6.500 km Erde, Fels, Gestein und Metall apparieren.}

\enquote{Da ist was dran. Noch weitere Ideen?}, fragte Professor Elber in die Runde.

Vereinzelte Zwischenrufe von \enquote{100 km}, oder \enquote{400 km} konnte man hören.

\enquote{Noch jemand anderer Meinung?}

Stille.

\enquote{Dann löse ich mal auf. Neville hat recht. Theoretisch ist es möglich, etwa 20.000 km zu apparieren. Doch bevor ich das erkläre, hätte ich noch gerne gewusst, wie lange man für diese Strecke braucht.} Wiederum war nichts als Schweigen zu hören. Plötzlich schien Hermine eine Idee zu haben, denn sie suchte in ihrer Tasche etwas. Sie nahm ein Blatt Pergament heraus und zeichnete mit ihrer Feder darauf herum. Professor Elber näherte sich ihr und schaute ihr begeistert zu, als sie auf das Papier das Bild eines Taschenrechners malte.

Er wartete ab, bis sie fertig war. Dann nahm sie ihren Zauberstab und tippte mit murmelnden Worten mehrmals auf das Pergament. Nachdem sie ihren Zauberstab wieder eingesteckt hatte, begann sie auf den aufgemalten Tastenfeldern zu tippen, woraufhin im oberen Anzeigenfeld die unten angetippten Zahlen erschienen. Nachdem sie einige Rechenoperationen durchgeführt hatte, verkündete sie: \enquote{133ms}.

\enquote{Erstaunlich}, antwortete Professor Elber. \enquote{Toller Zauber. Den müssen sie mir mal bei Gelegenheit beibringen. Kann man sicher brauchen. Die Antwort ist übrigens richtig. Doch nun zu dem, was einige von Ihnen stutzen ließ und vermuten lässt, dass nur wenige 100 Kilometer möglich sind.}  Er stellte an die gesamte Runde nun die Frage: \enquote{Wer von Ihnen ist schon einmal appariert worden? Ich meine damit Seit-an-seit Apparition.} Einige der Schüler streckten die Hand in die Luft. \enquote{War das angenehm?}, fragte er.

\enquote{Nein}, kam es allgemein zurück.

% Anmerkung: Das Vereinigte Königreich wird im Süden vom Ärmelkanal, im Osten von der Nordsee und im Norden und Westen vom Atlantik begrenzt.
\enquote{Genau. Deswegen schaffen es die wenigsten überhaupt eine größere Strecke zurückzulegen. Wenn wir uns auf den Planeten begrenzen, dann sind die 20.000~km die maximale Strecke, die aber normalerweise schwer zu schaffen ist. Mit einigermaßen guter Kondition schaffen sie auf einmal in etwa die halbe Nord-Süd-Strecke durch England. Die längste Nord-Süd-Ausdehnung ist knapp 1000~km lang, während die Ost-West-Ausdehnung knapp 500 km erreicht. Der höchste Berg ist mit 1342~m der Ben Nevis in Schottland. Das sind ca. 500~km. Wir beschränken uns die ersten Male nur darauf, durch die Halle zu apparieren, danach durch Hogsmeade durch und schließlich bei ihrer Prüfung dann so ca. 100~km.} Er lief nun auf eine Tür an der Seite der Halle zu, öffnete sie und zog einen kleinen Wagen herein, auf dem etwas stand, das Harry an eine Lottomaschine erinnerte.

Nur waren im Inneren keine Kugeln. \enquote{Jeder von euch kommt einmal dem Auswähler näher, tippt ihn mit seinem Zauberstab an und sagt dabei seinen Namen.}

Nacheinander traten alle vor und tippten den Auswähler an. Es erschien jedes Mal im Inneren eine kleine Kugel.

Jeder stand wieder an seinem Platz. Professor Elber fuhr fort. \enquote{Wenn ihr später den ersten Versuch unternehmt, werdet ihr immer zu zweit sein. Aber zuvor werden wir ein wenig Theorie machen. Was müssen wir alles beim Auswählen eines Platzes, der zum Apparieren geeignet sein soll, beachten?}

Harry wusste, dass man nicht gesehen werden durfte.

\enquote{Er muss sicher sein}, antwortete Malfoy.

\enquote{Gut Draco. Was verstehen Sie darunter?}

\enquote{Ich darf nicht gesehen werden wie ich erscheine und mein Auftauchen sollte kein Argwohn erregen. Wenn ich zum Beispiel ohne erkennbaren Grund aus einer Seitengasse herauskomme, die keinen Ausgang hat, dann kann das die Muggel misstrauisch machen.}

\enquote{Gut erkannt, Draco. Was zählt noch dazu?}

Draco überlegte. Sofort schoss Parvatis Hand in die Höhe. Professor Elber blickte kurz zu ihr und danach wieder zu Malfoy. Harry dachte, Professor Elber wollte Draco sagen, \gedanke{Denken Sie nach Draco. Noch ein Punkt.}

\enquote{Man sollte darauf achten, dass man nicht von\abs äh\abs Überwachungskameras der Muggel gefilmt wird.}

\enquote{Sehr gut, Draco}, meinte Professor Elber. \enquote{Fünf Punkte für Slytherin. \gst Sie werden sicherlich die wichtigsten Regeln gelernt haben, oder sie haben darüber gelesen. \gst Ziel. Wille. Bedacht. \gst Dies ist, was sie überall lernen werden, wenn es um das Apparieren geht. \gst Bei mir werden sie noch zwei weitere Sachen lernen. \gst Direktheit und Bequemlichkeit. \gst Wir werden zuerst die Drei-Punkt-Regel lernen, bevor ich Ihnen die beiden anderen Punkte noch beibringen werde. Ich finde, das macht die Sache angenehmer. Bilden sie bitte einen Kreis. Minerva, stellst du dich bitte mir gegenüber?}

Die Schüler nahmen Aufstellung und Professor Elber und Professor McGonagall standen sich gegenüber.




\begin{kommentar}
Frederick bringt Draco in einem leeren Zimmer etwas bei. Harry und Dumbledore lauschen dabei etwas. Was nirgends erwähnt wird, oder zumindest später nur angedeutet wird, ist, dass Frederick Draco beibringt ein Animagus zu sein. Dracos Form ist dabei ein Drache. Ich glaube es wird viel später (eventuell auch im zweiten Teil) irgendwo erwähnt.
\end{kommentar}

\chapter{Unschuldige Erinnerungen}


\enquote{Fassen Sie sich nun an den Händen. Minerva, lässt du dich führen? Wir wollen an das Ende des Dorfes. Es geht um die Unterschiede zwischen den beiden Techniken. Lasse dich leiten und ziehe einfach mit. Das macht es mir leichter.}

Minerva nickte mechanisch und die Reise begann. \enquote{Lasst gleich nach dem Ankommen los, verstanden? \gst Drei \gst Zwei \gst Eins\abs}

\geraeusch{Wupp.} Die Gruppe war verschwunden und tauchte nach Bruchteilen einer Sekunde wieder auf. Sofort lösten sich die Hände der Schüler und etwa ein Drittel begann sofort auf die Knie zu sinken und in die vor ihnen stehenden Eimer zu brechen. Nachdem Tücher herumgereicht wurden, damit sich die Schüler ihre Münder putzen konnten, fassten sie sich auf ein Zeichen wieder an den Händen und die Reise zurück begann. In der Halle angekommen mussten nur noch zwei brechen.

\enquote{Sehr gut. Damit dürften die Brechanfälle für Sie vorbei sein. Mehr als zweimal musste bisher keiner brechen. \gst Sie wurden jetzt also zweimal nach der Drei-Punkt-Regel appariert. Sie wissen also, wie man sich fühlt, wenn man von jemand mitgenommen wird, der diese Technik beherrscht, oder selber appariert und diese Technik anwendet. Wir werden jetzt nochmal hin und wieder zurück apparieren. Allerdings nach der Fünf-Punkt-Regel. \gst Bereit?}

Die Schüler und Minerva nickten und fassten sich wieder an den Händen.

\enquote{Drei \gst Zwei \gst Eins.}

\geraeusch{Wupp.} Wieder verschwand die Gruppe aus der Halle und tauchte innerhalb von Sekundenbruchteilen am Ende von Hogsmeade auf. Dann ging es wieder zurück.

\enquote{Und? Haben Sie einen Unterschied festgestellt?} Die Klasse nickte. \enquote{Was war angenehmer?}

\enquote{Die dritte und vierte Reise}, sagte Marcel, ein Ravenclaw.

\enquote{Hatten Sie das Gefühl, dabei brechen zu müssen?}

\enquote{Nein.}

\enquote{Warum nicht?}, fragte er provozierend nach.

\enquote{Ich hatte nicht das Gefühl, durch einen Schlauch gepresst zu werden. Es war vielmehr ein Verschwinden und Auftauchen an einem anderen Ort. Wie ein normaler Spaziergang.}

\enquote{Sehr schön, das meinte ich. Fünf Punkte für Ravenclaw. \gst Kommen wir nun zur Drei-Punkt-Regel. Wir werden heute schon damit anfangen, uns darauf zu konzentrieren und die ersten Versuche zu unternehmen. Allerdings einzeln und nur von einem Kreis in den anderen.}

Auf dem Boden erschienen zwei Kreise, die nur zwei Meter voneinander entfernt waren.

\enquote{Fangen wir an mit dem ersten der drei Punkte.}

Elfen erschienen und brachten Kissen mit, die sie im Raum verteilten. Professor Elber setzte sich auf eines der Kissen und machte weiter.

%Ziel. Wille. Bedacht.			%Direktheit und Bequemlichkeit
\enquote{Fangen wir mit dem Ziel an. Sie müssen sich das Ziel genau vorstellen. Je genauer sie das Ziel vor Ihrem inneren Auge haben und sich vorstellen können, desto besser ist es. Prägen Sie sich also die Halle einmal ein. Versuchen Sie sich vorzustellen, wie es aussieht, wenn Sie in diesem Kreis stehen. Je genauer Sie sich den Zielort vorstellen können, desto besser ist es. Wenn Sie zum Beispiel geografische Daten haben, also Länge und Breite auf dem Globus, dann können Sie sich auch dort hin apparieren. Egal, wie Sie ihr Ziel benennen. Je genauer Sie den Ort benennen können in Ihren Gedanken, desto besser ist es. Denken Sie während der nächsten fünf Minuten darüber nach.}

Dann herrschte Stille. Minerva nahm ihr Kissen und setzte sich neben Professor Elber. Sie tuschelten miteinander so leise, dass sie die anderen nicht störten, die konzentriert da saßen.

\enquote{Als Nächstes kommt der Wille. Sie müssen dorthin wollen. Sie müssen in sich einen Drang auf das Ziel aufbauen. Es darf nichts Wichtigeres geben, als dort hin zu wollen. Sie müssen sich vorstellen, dass Sie von hier nach da gelangen, ohne sich zu bewegen. Und der dritte Punkt. Die Bedacht. Seien Sie Präzise in Ihren Ausführungen. Arbeiten Sie Sorgfältig, denn sonst zersplintern Sie.}

Er griff in die Luft und es kam ein Bündel an Papier zum Vorschein. \enquote{Bitte sehen Sie sich diese Fotos an und reichen Sie sie herum. Bitte einzeln.} Er reichte die Fotos einem Schüler in seiner Nähe. \enquote{Wenn Sie nämlich nicht sorgfältig sind, dann werden Sie so aussehen, wie die Personen auf den Bildern. Deswegen ist es auch erforderlich, dass Sie eine Lizenz erwerben. Illegales Apparieren ist strafbar. Sie können sich denken warum, wenn Sie diese Fotos sehen.}

Dann stand er auf und belegte die Halle mit einem Zauber. \enquote{Bitte versuchen Sie es alle einmal. Stellen Sie sich Ihr Ziel vor, haben Sie den Willen, dorthin zu gelangen, und seinen Sie präzise mit Ihren Körperteilen und Ihrer Kleidung. Es kann nämlich passieren, dass Sie diese verlieren. Das ist schon mal passiert, dass jemand nackt auftauchte. Die Halle wurde mit einem Anti-Apparitionsschild belegt. Sie werden ein Kribbeln spüren, wenn Sie zu apparieren versuchen. Wenn Sie sämtliche Körperteile spüren und alles an Ihnen kribbelt, dann würden Sie komplett an ihrem Ziel erscheinen. Leider können Sie das nicht für Ihre Kleidung und Ihre Körperbehaarung feststellen. Ich habe schon von Personen gehört, die nach einer Apparition eine Glatze hatten, oder deren Schambehaarung zurückblieb.}

Einzelne Schüler schmunzelten. Dann begannen sie sich zu konzentrieren und zuckten immer wieder zusammen und rieben sich verschiedene Körperteile.

Während dessen tuschelte Elber mit McGonagall. Ab und an gestikulierte er.

\fluestern{Du musst beim Fünf-Punkte-Apparieren folgendes beachten. Du willst direkt zum Ziel und nicht über einen Umweg, oder einen Pfad, der dir vorgegeben wird. Denke dir eine Linie direkt zum Ziel. Das ist wichtig. Sonst wirst du so durch den Raum geleitet, wie es der Magie passt. Das kann unter Umständen der vielfache Weg sein. Da die Magie ja den Raum belegt, in dem wir uns befinden, wird sie versuchen, uns nur eine Passage so klein wie Möglich zur Verfügung zu stellen. Das geht dann natürlich an die Substanz. Außerdem musst du auf die Bequemlichkeit acht geben. Stell dir die Reise so bequem wie nur möglich vor. Du willst schließlich nicht gequetscht werden. Dann wird dir auch eine größere Passage zugeteilt.}

McGonagall sah ihn abwechselnd mit Staunen dann wieder mit Unverständnis, aber auch mit verstehendem Blick an.

Nachdem die erste Stunde bereits vorbei war, stand Professor Elber wieder auf und schwang erneut seinen Zauberstab.

\enquote{So, jetzt wird es ernst. Jeder hat genau einen Versuch, zu apparieren. Die Halle wurde präpariert, damit ein heraus oder herein apparieren nicht möglich ist. Sie können nur innerhalb der Halle apparieren. Professor McGonagall und ich stehen sofort bereit, falls ein Unfall passieren sollte. Wollen wir hoffen, dass so etwas nicht passiert. Es ist egal, wer anfängt. Also, die erste freiwillige Person vor.}

Lange passierte nichts, bis eine mutige Hufflepuff vortrat.

\enquote{Ich sollte jetzt von Gryffindor fünf Punkte abziehen. \gst Das werde ich jetzt auch machen.}

\enquote{Frederick, wie kannst nur\abs? Wieso?}

\enquote{Bist du mutig und so\abs dann ist Gryffindor eine gute Heimat. So heißt es doch. \gst Fünf Punkte von Gryffindor wegen Feigheit.}

\enquote{Hrmpf}, war das Einzige, was McGonagall noch an Tönen hervorbrachte.

\enquote{Also Miss Mariola. Hinein in den Kreis.}

Die Schülerin kam und nahm im Kreis Platz.

\enquote{Ziel. Wille. Bedacht}, sagte Elber.

Mariola nickte und konzentrierte sich. Kurz darauf stand sie im Zielkreis. Nur ihre Unterhose war zurückgeblieben und fiel zu Boden.

\enquote{Violette Unterwäsche. Nett}, scherzte er.

Erschrocken drehte sich Mariola um und folgte dem Blick ihres Lehrers. Peinlich berührt nahm sie ihre Unterhose auf und schob sie in ihre Tasche. Dann traute sich auch der Rest der Klasse. Nicht bei jedem lief es so gut. Einer apparierte sogar ohne Kleidung. Da die anderen aber alle dahinter standen, war das nur bedingt peinlich, da sofort ein Bademantel gereicht wurde. Einem Schüler fehlte nach dem Apparitionsvorgang sogar ein Finger, der aber sofort wieder von beiden Professoren angezaubert wurde. Sie umhüllten den Arm der Schülerin mit einem rosa Nebel und fügten den Finger wieder an. Durch den Schock, spürte sie keine Schmerzen und bevor die Nervensignale ihr Gehirn erreichten, war der Finger schon wieder dran. So spürte sie nur ein kleines Kribbeln. Nach etwa der Hälfte war Harry dran. Er apparierte erfolgreich im Ganzen und ohne Verluste. Leider kam er außerhalb des Zielkreises an. Er war zu weit appariert, was ihn auf die Entfernung bei einer Prüfung durchfallen ließ.

Insgesamt war es ein durchwachsenes Ergebnis, einige verlorene Finger, ein paar verlorene Kleidungsstücke und einen Fingernagel, sowie Augenbrauen.

\enquote{Für den ersten Versuch war das schon ganz in Ordnung. Entspannen Sie sich jetzt und beruhigen Sie sich wieder. Setzen Sie sich noch ein paar Minuten auf Ihre Kissen, damit Sie sich beim Verlassen der Halle nicht aus Versehen ver-apparieren}, scherzte Professor Elber. \enquote{Die Stunde ist beendet. Bis nächste Woche. Dann werden wir die Übung wiederholen. Die erste viertel Stunde werden Sie dann wieder auf den Kissen Ihre geistigen Fähigkeiten stärken und dann wird wieder appariert. Erst wenn Sie quer durch die Halle die Apparitionen schaffen, werden wir auf die Fünf-Punkte gehen.}

\trenn

Harry ging in Richtung der Kerker. Unten angekommen lief er durch sein Tränkeklassenzimmer und klopfte wie üblich an die Tür zu Snapes Büro. Die Tür wurde geöffnet und Snape saß wie üblich hinter seinem Schreibtisch. Wie er es in der Zwischenzeit schon gewohnt war, setzte er sich in den Sessel und konzentrierte sich. Er wusste, dass Snape bald so weit war; er hatte nur noch wenige Minuten.

\enquote{Sehr gute Hausarbeit, Potter, fünfzig Punkte dafür.}

Harry zeigte keine Reaktion. Und dann, ohne Vorwarnung, kam sein Angriff.

\enquote{Legilimens.}

Harry konzentrierte sich auf einen Gang, den er entlang lief. An dessen Ende stand eine Tür. Die Tür öffnete sich und dahinter war ein runder Raum mit vielen Türen. Er ging auf eine zu und kam wieder in einen Gang. Dieser war genauso lang wie der Erste. An dessen Ende war wieder eine Tür und wieder ein runder Raum. Dieses Mal nahm er eine andere Tür und trat wieder in einen Gang.

Dann begann sein Widerstand langsam zu bröckeln und wurde schwächer. Harry dachte an Hogwarts und dessen Gänge. Er verabschiedete sich von den schwarzen eintönigen Ministeriumsgängen und die Umgebung verwandelte sich. Dann brach sein Widerstand komplett.

Snape brach ab und gab ihm einen Trank. Harry trank ihn wie üblich und Snape setzte sich ihm gegenüber in einen weiteren Sessel.

\enquote{Nicht schlecht, Potter. Über eine Stunde. Sie bessern sich merklich. Sie sind dran.}

Dann erinnerte sich Harry an Worte, die ihm einmal gesagt wurden. \enquote{\accentuate{Wir haben viel, was Magie betrifft, vergessen. Zaubern ohne Stab war früher selbstverständlich.}}

Harry musste sich ein Grinsen verkneifen. Er konzentrierte sich kurz und spürte die Magie. Er hob seine Hand und zeigte mit seinen mittleren drei Fingern seiner rechten Hand auf Snape und murmelte leise: \enquote{Legilimens.}

Da Snape nicht darauf vorbereitet war, konnte er einige Sekunden lang in Snapes Vergangenheit eindringen.

\begin{traum}
Er sah ihn, wie er mit seiner Mutter gemeinsam in Hogwarts spazieren ging, wie sein Vater ihn niedermachte, aber auch, wie er ihm das Leben rettete, als ihn Sirius in Remus' Werwolffänge locken wollte, während er sich bei Vollmond im Gang wieder verwandelt hatte \gst im Gang unter der peitschenden Weide.
\end{traum}

Dann konnte Snape seine Gedanken verschließen und entließ Harry \gst das hieß, Harry stand auf, bedankte sich und ging freiwillig, bevor er hinausgeworfen wurde.

Er ging schlafen, doch konnte es nicht. Dann versuchte er sich auf seine Okklumentik zu konzentrieren, doch er erreichte genau das Gegenteil. Die Bilder, die er sah, kamen wieder hoch, so als seien sie die einzigen Erinnerungen, die er hatte. Langsam sickerten diese wenigen Momente, in denen er seine Mutter lachen sah, in seine Gedanken und seine Erinnerungen ein und er verstand langsam, verstand, dass seine Mutter damals mit Snape ausging; doch das war nur ein Gedanke, ein Gefühl. Sein Verstehen lag erst am Anfang.

Er wachte mitten in der Nacht auf. Ihn beunruhigte etwas. Er nahm die Karte des Rumtreibers heraus und suchte \accentuate{Severus Snape} auf ihr. Er fand ihn in der Nähe des Gryffindorturmes. Er faltete die Karte zusammen, zog seinen Bademantel und seine Schuhe an und ging hinunter, durch den Gemeinschaftsraum und verließ ihn durch das Porträt.

\enquote{Sperrstunde Potter, sie sollten schlafen}, hörte er Snape sagen.

Harry drehte sich herum und antwortete: \enquote{Ich weiß. Ich habe nach ihnen gesucht. Ich\abs} Er überlegte, wie er es am besten sagen konnte. \enquote{Das, was ich gesehen habe\abs Ich habe darüber nachgedacht\abs seit ich heute schlafen gegangen bin und meine Gedanken verschließen wollte\abs Ich konnte nicht\abs Meine Gedanken drehten sich nur um das, was ich gesehen hatte.} Er stockte kurz. \enquote{Sir, ich beginne so langsam zu verstehen. Ich habe aber eine Frage, die mich beschäftigt.} Dann sah er Snape direkt in die Augen. Emotionslos sah er ihn an. \enquote{Wie haben sie und meine Mum\abs}

Snape hob seine Hand. \enquote{Kommen Sie an ein Denkarium ran?}, fragte er ihn.

Harry war sich nicht sicher, nickte aber.

Snape griff in seinen Umhang und zog eine kleine Phiole mit einer klaren, leicht bläulichen Flüssigkeit heraus. \enquote{Hier. Das ist eine Kopie. Vielleicht wird ihnen das dann klar.} Harry nahm die Phiole vorsichtig entgegen. Er hielt sie wie einen Schatz. Dann bedankte er sich bei Snape und drehte sich um, um zurückzugehen. \enquote{Ach, bevor ich es vergesse. \gst Fünf Punkte Abzug wegen Aufenthalt außerhalb der erlaubten Zeiten}, sagte Snape. Dann rauschte er davon.

Harry grinste und betrat den Gemeinschaftsraum. Er hatte einen Weg gefunden, ohne die schlafende Dame aufzuwecken. Er berührte ihre Stirn und übermittelte ihr das Passwort mit Legilimentik. Das hatte den Effekt, dass das Porträt aufschwang, da es das Passwort erkannte und akzeptierte. Er musste die fette Dame also nicht wecken.

Er ging nach oben, zog sich den Bademantel aus, fischte die Phiole heraus und sicherte sie magisch. Dann legte er sie sicher in seinen Koffer, legte sich ins Bett und schlief beruhigt ein.

\trenn

Professor Elber traf auf Luna und Harry, die am See spazieren gingen, um sich über ihre körperlichen Bedürfnisse und sonstige Notwendigkeiten zu unterhalten. Luna musste sich jetzt immerhin rasieren und hatte absolut keine Ahnung, wie man das machte. Harry zauberte einen Spiegel hervor und zeigte Luna einen Rasurzauber. Dann fuhr er mit dem Zauberstab an seinen Barthaaren entlang. Die eine Hälfte machte er, die andere überließ er Luna, die sich bemühte es ordentlich hinzubekommen. Es sah lustig aus, wie Lunas Körper dem von Harry erklärt, wie man sich zu rasieren habe.

\enquote{Gute Neuigkeiten}, sprach Professor Elber die beiden an. \enquote{Es ist nicht notwendig, etwas mit Ihren Seelen zu machen. Ich habe noch etwas anderes gefunden. Es zeigt an, wohin die Sinneseindrücke übermittelt werden. Ein einfacher Zauber. Ungefährlich.} Harry und Luna sahen ihren Lehrer freudig an. \enquote{Bereit?}, fragte er.

\enquote{Ja}, antworteten beide.

Ihr Lehrer fuhr beiden mit dem Daumen über ihre Stirn, zog seinen Zauberstab und murmelte einen Spruch, den die beiden nicht verstanden. Dann kamen aus ihrer Stirn feine Linien, die zum jeweils anderen Körper gingen. Erleichtert atmete Professor Elber aus.

\enquote{Das war Hebräisch, richtig?}, fragte Luna.

\enquote{Ja}, antwortete Professor Elber. \enquote{Es ist jetzt leichter. Entweder das geht schnell vorbei, oder wir müssen etwas nachhelfen. Da muss ich noch weiterlesen. Ich werde Severus Bescheid sagen, dass er seinen Trank für etwas anderes verwenden kann. Das müsste noch gehen.} Dann ging er wieder.

\enquote{Wie waren eigentlich die Stunden bei meinem Vater, Harry?}, fragte ihn Luna, denn beide hatten in der Zwischenzeit kurzzeitig ihre Körper getauscht. So hatte Harry einen halben Tag bei Xenophilius Lovegood verbracht.

\enquote{Interessant}, antwortete Harry. \enquote{Ich war plötzlich in deinem Bett und sah an die Decke. Ich musste mich erst einmal orientieren. Dann habe ich mich im Zimmer umgesehen. Das Wandgemälde fand ich besonders beeindruckend. Hast du es selbst gemalt?}

\enquote{Ja.}

\enquote{Du hast mich, Ron, Hermine, Ginny und Neville an die Wand gemalt.} Luna nickte. \enquote{Und das Band. Ich hielt es erst für ein Band. Bis ich näher kam und eine feine Schrift entdeckte. \accentuate{Freunde}, stand da. Eng nebeneinander und in drei Reihen untereinander. Goldene Schrift. Hast du das auch gemacht?}

\enquote{Ja. Alles mit einem feinen Pinsel. Ich hatte Zeit und mir war danach. Es war letztes Jahr während der Ferien. Dad war ein paar Tage beschäftigt. Also habe ich mich daran gesetzt und gemalt. Er findet es schön.}

\enquote{Das hat er mir gesagt, als er in deinem Zimmer auftauchte und mich sah, wie ich das Bild betrachtete.}

\enquote{Was hast du sonst noch so erlebt?}

\enquote{Ich habe für ihn gekocht.}

\enquote{Hat er keinen Verdacht geschöpft?}

\enquote{Nein. Ich habe das Rezeptbuch hergenommen und nach einem gesucht, dessen Zutaten ich kannte. Zu Hause habe ich auch immer mal wieder gekocht. Daher hatte ich keine großen Probleme. Dein Dad hat nur einmal komisch geschaut, und gemeint: \enquote{Seit wann schmeckt dir denn das? Das hast du doch sonst immer ungern gegessen.} Ich habe ihm erzählt, dass ich mich heute anders fühle und meine Geschmacksknospen wohl etwas anderes wollen. Das hat er mir abgenommen. \gst Du hast übrigens ein sehr schönes Haus. Nur für immer würde ich mit deinem Vater dort nicht leben wollen.}

\begin{abAchtzehn}

\enquote{Ich würde jetzt gerne mit dir schlafen, Harry}, sagte sie plötzlich.

\enquote{Du weißt aber schon, dass wir nicht mehr zusammen sind, oder?}

\enquote{Ja. Stört dich das? Du bist doch noch solo, oder?}

\enquote{Ja schon, aber\abs} Harry wusste nicht mehr weiter.

\enquote{War das jetzt die Antwort auf meine erste Frage, oder auf meine zweite Frage?}

\enquote{Wie bitte?}

\enquote{Ich wollte wissen, ob das die\abs}

\enquote{Schon gut, ich habe dich verstanden. \gst Das war die Antwort auf deine zweite Frage.}

\enquote{Ah. Also stört dich die Tatsache, dass wir nicht mehr zusammen sind. Deshalb möchtest du nicht mehr mit mir schlafen.}

Harry wusste nicht genau, was er sagen sollte.

Nach einer Weile fragte Luna nach. \enquote{Was ist jetzt? Ausziehen und ins Wasser? Wir können es ja als Schwimmen tarnen.} Kaum hatte Luna ihren Satz beendet, entledigte sie sich schon ihrer Kleidung und stieg nackt ins Wasser. \enquote{Komm schon, Harry. Das Wasser ist herrlich.}

Harry wusste nicht genau, wie er reagieren sollte, folgte ihr aber brav ins kühle Nass, nachdem er sich ebenfalls ausgezogen hatte, die Kleidung zusammengelegt und gegen Diebstahl gesichert hatte.

Sie schwammen am Ufer entlang und zwischen das Schilf, wo Luna ihre Arme um Harry legte und ihn zu küssen anfing. \enquote{Weißt du, auch wenn wir nicht mehr ein Paar sind, würde ich gerne meine Bedürfnisse stillen. Mit dir.}

Harry konnte es nicht leugnen und genoss die Zärtlichkeiten mit Luna. Er legte nun ebenfalls seine Arme um sie.

\enquote{Ich muss an Ginny denken}, sagte er.

\enquote{Dann tu es, wenn du mich küsst. Ich habe nichts dagegen.} Und als sich nach kurzer Zeit etwas bei Harry regte und sie seine Erregung zwischen ihren Beinen spürte, fügte sich hinzu: \enquote{Du scheinst Ginny sehr gerne zu haben. Das kann ich bis hierher spüren.}

\enquote{Ha. Ha}, sagte er. \enquote{Du bist aber auch schon ganz schön feucht zwischen den Beinen.}

Dies veranlasste Luna herzlich zu lachen. \enquote{Schaffst du auch den nächsten Schritt, wenn du an Ginny denkst?}, fragte sie vorsichtig nach.

\enquote{Tut mir leid, Luna. Das kann ich nicht.}

\enquote{Gibst du mir dann auf andere Art und Weise, was ich brauche?}

Harry nickte und seine Finger fuhren an ihrem Körper entlang Richtung Po. Eine Hand massierte ihren Hintern, während die andere ihr Schneckchen aufsuchte und sie zwischen den Beinen stimulierte. Luna legte ihre Beine um Harrys Hüften, was es ihm erschwerte, seine Hände zwischen den Körpern zu halten. Also zog er sie dazwischen heraus, nahm Luna noch ein Stückchen höher und kam von hinten. Immer wieder wechselte sein Finger zwischen ihren Schamlippen und ihrer Klitoris hin und her. Lunas Erregung wurde immer heftiger. Immer wenn sie meinte, dass sie einen Aufschrei nicht länger unterdrücken konnte, versuchte sie ihm einen Zungenkuss aufzudrücken, um ihre Schreie zu ersticken. Sie wollte nicht zu laut sein.

Als sie ihren dritten Orgasmus innerhalb einer viertel Stunde hinter sich hatte, brach sie den Kuss und meinte: \enquote{Es ist genug, Harry. Das war der Wahnsinn. Das reicht für mindestens\abs Auf jeden Fall lange.} Sie nahm ihre Hände von seiner Hüfte und legte sie um seinen Kopf. Dann küsste sie ihn erneut. \enquote{Darf ich dir nun einen Gefallen tun?}, fragte sie.

\enquote{Später. Vor allem nicht im Wasser.}

\enquote{Oh. Ich kenne aber einen guten Zauber, um ein paar Minuten die Luft anzuhalten und wenn dich dein Sperma im Wasser stört, dann schlucke ich einfach.}

\enquote{Das ist ein verlockendes Angebot, Luna. Aber, heute nicht.}

\enquote{Ok}, sagte sie und lies von ihm ab.

Das war das Schöne an Luna. Sie schien nie richtig beleidigt zu sein. Sie fuhr ein paar mal mit hartem Griff an seiner Männlichkeit auf und ab und schwamm dann zurück. Als sie zurückkamen lagen bereits Handtücher zum Abtrocknen bereit.

Nachdem sie angezogen waren, erschien Kreacher und nahm die Handtücher wieder mit.

\enquote{Richtig fürsorglich, dein Elf}, bemerkte Luna.

\enquote{Ja, das ist er}, antwortete Harry.

\enquote{Und? Wann darf ich dir nun einen blasen?}, fragte Luna unverfroren.

\enquote{Erwartest du von mir ein Datum?}

\enquote{Wäre nett}, sagte sie, während beide wieder zurück zum Schloss gingen.

\end{abAchtzehn}

\begin{safedivide}
\fskdivider
\end{safedivide}

Harry dachte nach. Er wusste noch immer nicht, wie genau er in Voldemorts Geist eindringen sollte. Dann hatte er einen Einfall. Es klang so logisch. Er konnte deshalb keine Verbindung herstellen, da er mit sich selbst keine Verbindung herstellen konnte. Er holte sein Glasröhrchen aus seiner Tasche, das er immer mit sich trug, und entkorkte es wieder.

Luna blieb stehen und sah ihn interessiert an.

Harry lächelte sie an und dachte nach. Er versuchte sich zu erinnern. Er schloss seine Augen und holte seinen Stab hervor. Als er die Szene klar und deutlich vor sich sah, kopierte er sie, indem er seinen Stab an seine Schläfe hielt und den Gedankenfaden herauszog. Dann legte er ihn vorsichtig in das Glasröhrchen und verkorkte es. Jetzt musste er es nur noch jemanden zeigen.

\enquote{Luna? Morgen}, sagte er. Dann machte er sich weiter auf den Weg zum Schloss. \enquote{Ich habe noch was vor. Ich muss zu Dumbledore.} Dann verschwand er schnellen Schrittes. Er holte seine Erinnerung von Snape aus seinem Koffer und machte sich damit zu Dumbledores Büro. Er hoffte, dass er Zugang bekommen würde, hoffte, dass Dumbledore ihm die Gelegenheit geben würde, die Erinnerung von Snape zu sehen und durch die andere Hagrid zu entlasten.

Vor Dumbledores Büro angekommen, stand bereits Professor Sprout vor dem Eingang und versuchte ihn zu öffnen.

\enquote{Professor}, begrüßte er sie.

\enquote{Hallo, Mister Potter.}

\enquote{Versuchen Sie zu Dumbledore zu gelangen?}

Professor Sprout nickte. \enquote{Ich schaffe es nicht zu ihm zu gelangen.}

Harry stutzte und schaute zum Wasserspeier. \enquote{Hallo}, begrüßte er ihn. \enquote{Warum haben wir keinen Zugang?}

\enquote{Es gibt kein Passwort, um die Tür zu öffnen}, antwortete dieser.

\enquote{Wie? Kein Passwort}, fragte Professor Sprout. \enquote{Es gibt immer ein Passwort.}

\enquote{Es sollte geändert werden. Aber es kam kein neues nach. Also kann ich euch nicht hereinlassen.}

\enquote{Kannst du feststellen, ob es ihm gut geht?}, fragte Harry nach.

\enquote{Ja, das kann ich}, antwortete der Wasserspeier.

\enquote{Und, wie geht es ihm?}, fragte Professor Sprout nach.

\enquote{Das kann ich euch nicht beantworten}, sagte der Wasserspeier.

Mittlerweile kam auch Professor McGonagall an. Sie wollte wohl ebenfalls zum Schulleiter.

Harry hielt sich zurück, als Professor Sprout und Professor McGonagall sich über das Problem unterhielten. Seine Gedanken rasten. Kurz, nachdem Professor McGonagall einen Patronus losgeschickt hatte, kam Madame Pomfrey heran und wollte wissen, was los sei. Sie konnte nur feststellen, dass der Direktor bewusstlos am Boden seines Büros lag. Harry bekam das mit. Er erinnerte sich an das eine mal, an dem er Madame Pomfrey mitnahm, um Parvati und Lavender zu helfen. Er lächelte.

\gedanke{Na klar, die Aufzüge}, dachte er und ging auf einen Bogen zu, drückte den passenden Stein und trat ein. Er drückte den passenden Knopf und legte seine Hand auf die Schaltfläche.

Nach einer kurzen Fahrt stieg er im obersten Stockwerk des Turmes aus, in dem das Schulleiterbüro war. Er trat auf die oberste Ebene, auf der das Fernrohr stand. Er ging die Stufen schnellen Schrittes hinunter. Als er unten ankam, sagte er nur schnell \enquote{Hallo Fawkes} und sah kurz darauf Dumbledore am Boden liegen. Er fühlte seinen Puls und seinen Herzschlag. Beides war schwach, aber vorhanden. Er öffnete die hölzerne Tür und rief nach unten: \enquote{Öffnen.} Nachdem er das Geräusch der sich bewegenden Steine vernommen hatte, lief er zurück zu Dumbledore und untersuchte ihn mit seinem Zauberstab, damit er Madame Pomfrey eine kurze Analyse geben konnte. Einerseits brachte dies wieder eine mündliche Note, anderseits war es eine prima Übung. Aber am wichtigsten war die ersparte Zeit.
%Blutdruck Normalwerte
%Einteilung der Blutdruck-Werte laut WHO (Weltgesundheitsorganisation):
%						systolisch (mmHg)		diastolisch (mmHg)
%optimal				  < 120						 < 80
%normal					  < 130						 < 90
%hochnormal				130-139						85-89
%Hypertonie Grad 1		140-159						90-99
%Hypertonie Grad 2		160-179						100-109
%Hypertonie Grad 3		>= 180						>= 110
%Hypertonie = Bluthochdruck

Als er Madame Pomfrey die Stufen herauf steigen hörte, rief er ihr schon entgegen: \enquote{Puls ist schwach, aber vorhanden Herzschlag bei 75 und Blutdruck 55 zu 95.}

Madame Pomfrey kam durch die Tür und nickte Harry zu. Dann schwang sie ihren Zauberstab und untersuchte weiter. Nach kurzem nahm sie Dumbledore mit und zu Viert verließen sie das Büro. Harry blieb alleine zurück. Unschlüssig sah er sich um. Außer ihm war nur Fawkes im Raum. Sein Blick schweifte durch den Raum und blieb an der Tür hängen, hinter der das Denkarium stand. Harry überlegte kurz und  öffnete die Tür. Nach einem kurzen, entschuldigenden Blick zu Fawkes, nahm er den Gedankenfaden von Snape heraus und leerte ihn in das Denkarium. Das leere Glasröhrchen verschloss er wieder und schob es in seine Tasche. Mit dem Finger rührte er einmal kurz um und tauchte in die Erinnerungen ein.

\begin{traum}
Zuerst hatte er das Gefühl, zu fallen. Doch dieses Mal war er darauf vorbereitet. Er landete zwar immer noch unsanft, konnte sich aber abfangen, indem er in die Hocke ging. Die Umgebung verfestigte sich und eine Tür hinter ihm wurde geöffnet. Ein kleiner Junge von etwa neun oder zehn Jahren kam aus einem Haus.

\enquote{Severus, bleib nicht zu lange fort. Es gibt bald Mittagessen.}

\gedanke{Wahrscheinlich seine Mutter}, dachte Harry. Er konnte eine gewisse Ähnlichkeit feststellen.

Die langen schwarzen Haare hingen glatt aber sauber herunter. Er hatte ein weißes Hemd an, das ihm zwei Nummern zu groß war und darüber eine Jacke. Ebenfalls zu groß. Er lief durch ein paar Straßen eines dreckigen Arbeiterviertels zu einer Wiese. Das Viertel, in dem der junge Severus wohnte, war in einem Teil der Stadt, in dem ein Kohlekraftwerk stand. Die umliegenden Bewohner arbeiteten in eben diesem.

Auf der Wiese angekommen, sah er ein junges Mädchen mit roten Haaren.

\enquote{Hallo Severus. Schön, dass du da bist.}

\enquote{Hi Lily. Ich habe leider nicht viel Zeit, da ich bald zum Mittag muss. Aber eine Stunde ist schon noch drin.}

\gedanke{Lily. Sie heißt wie meine Mum. Und sie sieht ihr auch noch ähnlich.} Er schaute sie begeistert an und interessierte sich nicht für die Worte, die sie wechselten. Es ging hauptsächlich darum, dass Lily etwas über Hogwarts erfahren wollte.

Nach einer Weile kam ein anderes Mädchen zu den beiden. Sie hatte blonde Haare.

\enquote{Hi Tunia. Setz dich zu uns.}

Die Blonde hatte aber kaum Interesse daran. \enquote{Es gibt bald Essen, Lily. Komm mit. Du kannst dich nachher noch mit Severus unterhalten. Reicht es dir nicht, was du schon weißt. Den Rest wirst du schon noch erfahren.}

\enquote{Du hast recht}, sagte Lily und stand auf. \enquote{Bis später, Severus.}

Severus nickte und verabschiedete sich von Lily mit einer kurzen Umarmung. Petunia schüttelte er kurz die Hand und ging dann zurück.

Harry wartete bereits auf den grauen Schleier, der ihn zur nächsten Erinnerung bringen sollte. Doch es passierte nichts. Also folgte er Severus. Vor dem Haus seiner Eltern angekommen, öffnete der Junge die Tür und ging hindurch. Wenn Harry in diesem Moment stofflich gewesen wäre, hätte er die Tür voll auf die Nase bekommen, doch so konnte er einfach hindurchlaufen. Zum ersten Mal sah er etwas Privates von Snape, etwas sehr Privates! Der Gang, in dem er stand, war klein und schmal. Er war zwar anders eingerichtet, aber es erinnerte ihn an das Haus seines Onkels und seiner Tante. Die Tapete war dunkel und schälte sich schon von oben um einige Zentimeter von der Wand herab.

Im kleineren Esszimmer, etwa halb so groß wie zu Hause und durch eine Wand vom Wohnzimmer und der Küche getrennt, saßen nun Severus, seine Mutter und sein Vater. Seine Mutter war eine gutmütig aussehende Frau mittleren Alters. Sie hatte vereinzelt graue Haare, außerdem hatte sie ein paar Pfunde zu viel auf den Hüften, wirkte aber nicht dick.

Sein Vater hatte einen Drei-Tage-Bart und sah aus, als ob er dem Alkohol zugetan war. Er hatte müde Augen. Eventuell hatte er gerade eine Nachtschicht hinter sich gebracht, denn direkt nach dem Essen gab er seiner Frau einen Kuss, drückte seinen Sohn kurz und schaute über seine Hausaufgaben. Dann ging er ins Bett.

Severus half seiner Mutter. Er hatte schon ein paar Zauber drauf, die er aber laut seiner Mutter nicht vor seinem Vater anwenden durfte.

\enquote{Denk an Vater. Pass auf. Er mag es nicht, wenn wir zaubern.}

Der kleine Severus nickte und schwang Mamas Zauberstab.

\enquote{Wann ist es denn so weit?}, fragte er.

\enquote{In vier Wochen. Dann fahren wir in die Winkelgasse. Dein Brief von Hogwarts ist vor einer Stunde angekommen. Er liegt auf deinem Bett unter der Decke.}

Der junge Severus rannte aus dem Zimmer und Harry wollte ihm folgen, doch die Umgebung veränderte sich in einen grauen Schleier.

Es dauerte mehrere Sekunden, bis eine neue Umgebung erschien. Severus saß mit Lily im Hogwarts-Express. Beide saßen in einem Abteil. Sie saßen sich gegenüber. Harry nahm noch eine weitere Person wahr. Sie war aber ganz verschwommen. Man konnte nur eine Silhouette erkennen, die aber an den Rändern schon mit dem Hintergrund zu verschmelzen schien. Die beiden schienen sie nicht zu beachten.

\enquote{Ich bin so aufgeregt}, plapperte Lily, als hinter Harry, der im Abteil stand, die Tür aufging. Er drehte sich um und sah einen Jungen, der eine ähnliche Brille aufhatte wie er selber.

\enquote{Hi junges Fräulein, was willst du denn bei dem Langweiler? Komm doch lieber zu uns. Da ist was los.}

Hinter dem Jungen waren noch zwei weitere; einer mit schwarzem gelocktem Haar und einer mit braunen Haaren und scheinbar dreckiger Oberlippe. Aber wenn man genauer hinsah, bemerkte man schon etwas Bartwuchs. Es dauerte ein paar Sekunden, bis Harry erkannte, dass sein Vater vor ihm stand. Hinter ihm Sirius und Remus.

\enquote{Was ist jetzt. Oder möchtest du weiter bei Schniefelus bleiben?}, fragte er erneut nach und warf einen gehässigen Blick zu Severus.

\enquote{Hau bloß ab. Ich entscheide selber, mit wem ich meine Zeit verbringen. Mit dir aber garantiert nicht.}

Für den Moment abgestraft trollten sich die drei wieder. Es sollte aber nicht das letzte Treffen sein.
\end{traum}

Neben der einen Erinnerung, die Harry von Professor Snape sah, wurde ihm langsam bewusst, dass sein Vater in jungen Jahren genau so war, wie ihn Severus Snape beschrieben hatte, denn schon in der nächsten Szene kam eine weitere Gemeinheit.

\begin{traum}
Die Umgebung verschwand wieder in einem grauen Schleier und offenbarte kurz darauf die Große Halle. Dumbledore, etwas jünger, saß in seinem Stuhl und wartete darauf, dass sich die große Flügeltür öffnete. Nach kurzer Wartezeit tat sie dies auch und hereintraten, hinter Professor McGonagall, die neuen Erstklässler. Harry erkannte seine Mutter, seinen Vater, Remus, Sirius und Severus. Harry sah, wie Severus schweren Herzens zusehen musste, wie der Hut seine Mutter in das Haus Gryffindor brachte, denn für Severus stand es außer Frage, dass er selber in Slytherin landen würde.

Und da war wieder diese Gestalt, die gleich nach seiner Mutter aufgerufen wurde. Professor McGonagalls Lippen bewegten sich zwar, doch er hörte keinen Ton und auch der Mund war wie durch einen Hitzeschleier unkenntlich gemacht, sodass er nicht einmal von den Lippen lesen konnte. Die Person wurde ebenfalls nach Slytherin gebracht. Harry kam das alles sehr merkwürdig vor. Wollte Snape nicht, dass er erfuhr, wer noch mit ihm in derselben Klasse war?

Die Szene mit Severus, der in der Luft bis auf die Unterhose ausgezogen baumelte, übersprang er mit mäßigem Interesse. Er hatte sie schon einmal gesehen. Er wunderte sich zwar, warum er sie ihm zeigte, aber andererseits hatte er sie schon einmal gesehen. Es machte keinen Unterschied. Zumindest aus Sicht von Severus Snape. Harry hätte darauf aber verzichten können.

Der graue Nebel kam ein weiteres Mal. Dann lichtete er sich wieder. Severus war außerhalb des Schlosses. Es war gerade Vollmond. Severus schlich vier Gestalten hinterher. Zum ersten Mal sah er, wie Sirius ihn in den Gang lockte, durch den kurz zuvor Remus gegangen war. Nur durch das beherzte Eingreifen seines Vaters konnte Severus dem sicheren Tod entgehen. Dieser zog ihn mit aller Gewalt von dort weg, was einen Beinbruch bei Severus zur Folge hatte.

Ein erneuter grauer Schleier und die beiden lagen auf der Krankenstation. Lily kam herein und trat zu James Potter heran.

\enquote{Warum hast du ihn gerettet?}, fragte sie ihn.

\enquote{Weil dich sein Verlust sehr traurig gemacht hätte}, gab er ihr als Antwort.

Als sie nach einer kurzen Unterredung mit Severus, der Harry nicht zuhörte, wieder ging, sah sie noch einmal zu James Potter, bevor sie aus dem Blickfeld verschwand.
\end{traum}

\gedanke{Das muss der Zeitpunkt gewesen sein, an dem sich meine Eltern näher gekommen waren}, dachte Harry.

\begin{traum}
Dann folgte die letzte Szene. Es gab einen Streit zwischen Lily und Severus, infolgedessen er sie Schlammblut nannte. Das hinterließ eine tiefe Narbe auf ihrem Gesicht. Zwar keine physische, aber der seelische Schmerz war ihr in diesem Moment genau anzusehen.
\end{traum}

Die Erinnerung war zu Ende und Harry hatte das Gefühl aufzusteigen.

Mechanisch holte er seinen Stab und das Glasröhrchen, entkorkte es und holte die Erinnerungen aus dem Denkarium heraus, legte sie sorgfältig in das Gefäß und verschloss es. Danach räumte er es mit seinem Stab zurück in seine Tasche. Nachdenklich ging er zu Fawkes und streichelte ihn. Der bunte Vogel ließ sich das gefallen und zwitscherte und sang ohne Unterbrechung. Die dunklen Gedanken wurden für den Moment von Harry genommen. Er genoss es, wieder einmal einen dieser seltenen Momente erleben zu dürfen, in denen er einfach nur glücklich war.

Dann spürte er, wie sich eine Person der Pforte näherte. Obwohl ihm dies noch nie zuvor untergekommen war, war er sich sicher, dass es Dumbledore war. Vor der Wand angekommen, bat er um Einlass. Harry musste schmunzeln bei diesem Gefühl. Ihm war so, als ob der Schulleiter nicht mehr in sein eigenes Büro kam. Er ging zur Holztür, öffnete sie und sagte wie auch schon zuvor: \enquote{Öffnen.}

Dann ging er zurück und setzte sich auf einen Stuhl an Dumbledores Schreibtisch. Gedankenverloren sah er zu Fawkes, der ihn mit schrägem Kopf ansah. Keiner der beiden hatte eine Ahnung, was der andere gerade dachte. Aber genau das gab Harry Kraft. Einfach nur dazusitzen und den schönen Vogel anzusehen.

\enquote{Harry?}, fragte Dumbledore ganz erstaunt, als er sein Büro betrat. \enquote{Hast du mir aufgemacht? Was machst du hier?}

Harry sah Dumbledore an. \enquote{Ich halte hier die Stellung, da sonst keiner mehr in das Büro kommt, solange die Wand kein Passwort hat und ich weiß nicht, wie man eines setzt}, gab er ehrlich zur Antwort.

\enquote{Aber, wie bist du hereingekommen?}

Statt eine Antwort zu geben, sah Harry zu Fawkes. Dieser gab nur einen tirilierenden Laut von sich und schaute zurück. Dumbledore sah zwischen beiden hin und her. Er schien die Antwort nicht weiter zu hinterfragen, obwohl es den Anschein erweckte, dass er vermutete, nicht die ganze Wahrheit erfahren zu haben. Zumindest hatte Harry nicht gelogen, denn es war immerhin Fawkes, der ihm damals den Zugang zum Büro ermöglicht hatte. Außerdem würde er ihn nie missbrauchen.

\stimme{Du hast sogar immer Zutritt zum Büro des Schulleiters, Harry. Alle Erben der Gründer haben das. Du musst dich dem Wächter nur offiziell zu erkennen geben. Allerdings wissen die anderen das dann auch. Achte also darauf, falls du in Versuchung kommen solltest}, hörte er Agathas Stimme.

\gedanke{Wissen die das nicht sowieso?}

\stimme{Nein. Bei einigen ist es eine Vermutung, die sie aber für sich behalten. Sonst geben sie sich der Lächerlichkeit preis, falls es nicht stimmen sollte.}

\enquote{Warum bist du noch hier, Harry?}, fragte ihn Albus.

\enquote{Ich wollte dich eigentlich um etwas bitten. \gst}

\enquote{Aber?}

\enquote{Deine temporäre Abwesenheit\abs}, bei diesem Ausdruck musste Albus schmunzeln, \enquote{hat mich veranlasst, mir dein Denkarium kurzfristig auszuleihen}, antwortete er nun ehrlich. \enquote{Ich hoffe, du bist nicht sauer.}

Albus schüttelte nur den Kopf. \enquote{Warum sollte ich? Es steht eh die meiste Zeit nur herum. Außerdem hast du schon Erfahrungen gesammelt. Ich habe schon mal mit dem Gedanken gespielt, es an einem etwas zugänglicheren Ort aufzustellen. Aber der eventuelle Ansturm macht mir noch Angst. \gst Gibt es noch etwas?}

\enquote{Ja. Es geht um Hagrid und sein Verbot, zu zaubern.}

\enquote{Da kann ich dir leider nicht helfen. Es gibt keine Beweise, dass er diese Tat, die ihm zur Last gelegt wird, nicht begangen hat.}

Harry zog das Glasröhrchen mit der Erinnerung heraus und stellte es auf den Tisch. \enquote{Das ist so nicht ganz richtig.}

\enquote{Wessen Erinnerungen sind das?}, fragte Albus nach.

\enquote{Toms Erinnerungen}, antwortete Harry. \enquote{Hagrid hat nicht den Basilisken frei gelassen. Er hat eine Agramantula im Schloss versteckt, welche später in den Wald umgezogen ist. Tom Riddle hat das herausgefunden und die Spinne verscheucht. Zur gleichen Zeit hat der Basilisk ein junges Mädchen umgebracht; Myrte. Er hat danach Hagrid beschuldigt und alle Beweise sprachen gegen ihn. Zwar reichten sie nicht für eine Verurteilung und Askaban aus, aber der Verweis von der Schule und das Zerbrechen seines Zauberstabes taten es als Strafe.}

Albus schien sprachlos zu sein. Vorsichtig, als würde es beim bloßen Ansehen zerbrechen, hob er das Glasröhrchen hoch und betrachtete den schimmernden Faden darin. \enquote{Das hast du dir in meinem Denkarium angesehen?}

\enquote{Nein}, antwortete Harry. \enquote{Aber ich würde mir diese Erinnerung gerne\abs einmal anschauen.} Beinahe hätte er \accentuate{noch einmal} gesagt.

Albus stand auf und holte das Denkarium heraus. Er nahm eine schmale metallene Schale vom steinernen Sockel und ließ sie auf den Tisch schweben. Dann entkorkte er das Glasfläschchen und rührte mit seinem Zauberstab um.

\enquote{Was passiert eigentlich, wenn man das mit seinem Finger tut}, fragte Harry.

\enquote{Nichts. Dann kann man die Erinnerung nicht ansehen. Es muss mit einem Stab gemacht werden.}

\enquote{Hat es schon einmal jemand versucht?}, fragte Harry weiter.

\enquote{Nicht, dass ich wüsste.}

\gedanke{Also könnte die komische Gestalt nur ein Fehler sein, weil ich mit dem Finger umgerührt habe}, ging Harry durch den Kopf.
% Später wird Harry erfahren, dass diese komische Gestalt seine Tante war.

\enquote{Wollen wir?}, fragte Dumbledore.

Harry nickte und beide berührten mit der Nasenspitze die Oberfläche der Flüssigkeit. Beide fielen sie, bis sie im Schloss landeten, Harry fast so elegant wie Dumbledore.

\begin{rueckblick}
Dumbledore war um etwa fünfzig Jahre jünger und Tom Riddle schlich eine Treppe hoch. Er sah, wie eine Trage mit einem Leichentuch und einem Körper darunter vorbeigetragen wurde.

\enquote{Tom? Was machen Sie hier? Wissen Sie nicht, dass eine Ausgangssperre besteht?}

\enquote{Doch Professor Dumbledore, aber ich wollte es mit eigenen Augen sehen, ob die Gerüchte stimmen.}

\enquote{Leider ja, Tom. Aber Sie sollten jetzt zu Bett gehen.}

Tom Riddle nickte, blieb aber noch für eine Frage stehen, bevor er ging. \enquote{Was passiert eigentlich mit dem Verantwortlichen?}

\enquote{Er wird wohl von der Schule fliegen. \gst Wissen Sie vielleicht etwas darüber?}

\enquote{Nein Sir. \gst Gute Nacht.}

\enquote{Gute Nacht, Tom.}

Tom ging die Stufen hinab und Harry und Albus folgten ihm. Sie gingen die Treppen bis ganz nach unten. Dort schlich sich Tom an einen großen Jungen heran. Es war Hagrid. Es folgte ein kleiner verbaler Schlagabtausch, dann ein Blitz und der Deckel einer Kiste, vor der Hagrid stand, flog auf. Heraus kam in einem Tempo, sodass man das Geschöpf fast nicht sehen konnte, eine Spinne und verschwand in den dunklen Gängen.

\enquote{Professor Dippet wird sich sehr freuen, dass du so gefährliche Kreaturen ins Schloss bringst.}

Dann verschwand Tom und die Szene löste sich auf. Sie manifestierte sich wieder vor dem Eingang zur Kammer im Mädchenklo. Der junge Tom sprach Parsel und der Zugang gab eine Röhre frei. Nacheinander rutschten die drei hinunter. Harry und Albus durch den jungen Tom hindurch, da er nach der Landung nicht gleich loslief. Vor dem versiegelten runden Tor angekommen, sprach Tom etwas auf Parsel. Danach ging er durch die sich öffnende Tür und durch den langen Gang bis an das Kopfende. Er sprach unentwegt auf Parsel.

\enquote{Komm zu mir}, übersetzte Harry.

Als der Basilisk herauskam und mit geschlossenen Augenlidern auf Tom traf, wirbelte die Umgebung wieder und wurde grau.
\end{rueckblick}

Danach stiegen beide Besucher auf und landeten in Albus’ Büro.

\enquote{Wird das ausreichen, um Hagrids Unschuld zu beweisen?}

\enquote{Bei dem Zustand im Ministerium? Nur wenn wir den richtigen Mitarbeiter erwischen.} Harry nickte und schaute betrübt zu Boden. \enquote{Woher hast du die Erinnerung eigentlich?}, fragte Albus.

Harry spürte wieder dieses Kratzen. Das Einzige, was ihm einfiel, war Marcel, den er in seine Gedanken projizierte und Albus zeigte. Dadurch irritiert oder aufgeschreckt, rückte er schlagartig nach hinten und fiel fast von seinem Stuhl.

\enquote{Ich muss noch Schulaufgaben machen}, sagte Harry, stand auf und verließ eiligst das Büro.

Den ganzen Weg zurück zum Gemeinschaftsraum musste er in sich hinein grinsen. \gedanke{Das muss ich mir merken. Wer auch immer in meinem Geist herumwühlt, bekommt einen Basilisken zu sehen.}

Gerade eben lief er am Bild von Adriana vorbei, als ihm wieder einfiel, dass er sie schon seit längerem etwas fragen wollte. Er wollte wissen, ob sie es war, die das Buch über die Dementoren geschrieben hatte. Er drehte noch einmal um und lief die wenigen Meter zurück. Dann sah er auf das Bild, an dem er immer wieder vorbeiging. Er sah die Frau mittleren Alters an. Sie schien gerade in einem Stuhl sitzend zu schlafen. \enquote{Mrs De Mimsy-Porpington?}, fragte er. Die Frau öffnete ein Auge. \enquote{Mrs Adriana de Mimsy-Porpington?}, fragte er genauer nach. Die Frau öffnete das andere Auge und richtete sich in ihrem Stuhl auf. Dann nickte sie. \enquote{Haben Sie das Buch über die Dementoren geschrieben?}, fragte er. Adriana hob nur fragend eine ihrer Augenbraue. \enquote{Haben Sie das Buch \buchtitel{Vom Inferi zum Dementoren} geschrieben?}, fragte er genauer nach. Adriana nickte. Harry kam das komisch vor. \enquote{Können Sie nicht sprechen?}, fragte er nach. Adriana nickte erneut. Dann begann sie mit Gebärdensprache. Doch Harry konnte damit nichts anfangen. Resignierend hörte sie wieder auf. Harry wusste jetzt zwar, dass sie die Autorin war, aber er konnte mit ihr keine Unterhaltung führen. Dann fragte Harry sie nach zwei Begriffen, die sie ihm durch Gebärden mitteilte. Harry bedankte sich und nahm sich vor, in der Bibliothek von Hogwarts danach zu suchen. Dann ging er weiter zu seinem Gemeinschaftsraum.

Doch er drehte plötzlich um und wäre fast mit jemandem zusammen gestoßen. \enquote{Tut mir leid, Katharina}, sagte er, als sie in seinen Armen lag, da er instinktiv zulangte. Dann ließ er sie los.

\enquote{Du warst wohl in Gedanken?}, fragte sie.

Er nickte nur. \enquote{Mir ist gerade etwas eingefallen. Ich wollte zur Bibliothek.}

\enquote{Sagst du’s mir, oder ist das Geheim?}

Harry sah sie erst eine Weile an. Er überlegte, ob er ihr vertrauen konnte. Aber die Tatsache, dass er ihr bei ihrem Medusen-Problem geholfen hatte, weil sie ihn darum bat, brachte den Ausschlag. \enquote{Komm mit, wenn du willst.} Zusammen liefen sie zur Bibliothek. \enquote{Ich suche etwas zur Mondbibliothek.}

\enquote{Wozu? \gst Warte mal. Mein Großvater hat diesen Begriff ein paar Mal verwendet. Er konnte mir aber nicht viel dazu sagen.}

\enquote{Jede Information ist mir hilfreich}, antwortete Harry.

\enquote{Ich kann ihn leider nicht mehr fragen}, sagte Katharina, als sie angekommen waren. Sie folgte Harry durch die Gänge. \enquote{Er ist letztes Jahr gestorben.}

\enquote{Tut mir leid}, gab Harry erschrocken und ehrlich betroffen zurück.

\enquote{Das muss es nicht. Er war schon lange Krank.}

Harry nickte und lief weiter. Nach ein paar Biegungen stoppte er und überlegte, wo er suchen könnte.

\enquote{Was suchst du?}, fragte ihn Katharina.

\enquote{Wonach ich suchen muss. Mondbibliothek, Bibliothek oder ein anderer Begriff.} Harry entschied sich, erst einmal, nach dem \accentuate{M} zu suchen. Er ging in den Bereich der Bibliothek, der nicht nach Themen, sondern nach dem Alphabet sortiert war. Hogwarts’ Bibliothek war nämlich zwei geteilt, es gab einen Bereich mit Themen-Sortierung und einen Bereich mit alphabetischer Sortierung. Er suchte im Alphabet die Stelle, in der die Bücher mit \accentuate{M} enthalten waren. Als er sie gefunden hatte, stand er zwischen zwei Regalreihen vor einer Holzpaneele. Sie zeigte wie üblich zwei Buchstaben, die neben einem runden Knopf in der Mitte angebracht waren. \accentuate{L} und \accentuate{M}. Harry drückte das \accentuate{M} und danach den runden Knopf in der Mitte. Nur für ihn teilte sich optisch das Holzpaneele. Er schnappte sich Katharinas Hand und zog sie gerade aus. Im inneren angekommen, ließ er sie los und sah sich nach einem Buch um, das den passenden Titel hatte. Oder eines, das nahe dran war.

Katharina staunte. \enquote{Ich habe gar nicht gewusst, dass es hier versteckte Reihen gibt}, sagte sie.

\enquote{Das weiß ich auch noch nicht so lange. Ehrlich gesagt, habe ich es nur erfahren, weil ich mit Luna\abs Na ja. Auf jeden Fall sind wir unserem Lehrer hinterher und so habe ich erfahren, dass es diese Reihen gibt. Das ist erst das zweite Mal, dass ich hier bin. Die anderen Sachen findet man im normalen Bereich.} \gedanke{Oder in der verbotenen Abteilung}, fügte er gedanklich hinzu.

\enquote{Oder in der verbotenen Abteilung}, sagte Katharina und sah ihn spitzbübisch lächelnd an.

Mit einem Unschuldsgesicht sah er sie an. \enquote{Wo du dich überall herumtreibst}, sagte er.

Ihr Lachen auf ihrem Gesicht brachte ihn ebenfalls zum Lachen. Nachdem sich beide beruhigt hatten, suchten sie diesen Gang ab. Doch weder in diesem, noch im Gang mit dem \accentuate{B} wurden sie fündig.

\enquote{Wie wäre es mit \accentuate{W} wie \accentuate{Wissen}?}, fragte sie.

\enquote{Gut, versuchen wir es}, meinte Harry.

Dort angekommen, hatte Katharina das Tor geöffnet, nachdem ihr Harry gesagte hatte, was sie zu tun habe. Sie machten sich auf die Suche.

\enquote{Wie wäre es hiermit?}, fragte Katharina und zog ein Buch heraus. Sie schlug es auf und überflog das Inhaltsverzeichnis.

Harry kam näher und verdrehte sich, nachdem er sich vorgebeugt hatte, um den Buchtitel vom Umschlag zu lesen. Katharina schlug den Buchdeckel wieder zu, damit Harry aufrecht stehend lesen konnte. \buchtitel{Wissenswertes über den Umgang mit der Magie}

Im Inhaltsverzeichnis stand unter anderem \accentuate{Aufbau der Magie} und \accentuate{Wissenssammlungen}.

Katharina blätterte an die erste Stelle und las etwas über den Aufbau der Magie. Es war eine ausführlichere Beschreibung dessen, was ihnen Professor Elber einmal während einer Vertretungsstunde gesagt hatte. Dann blätterte sie weiter und beide lasen. Das Kapitel war wie ein Bericht eines Forschers geschrieben, der mit jemanden per Brief kommuniziert.

\begin{buch}
Es gibt drei große Sammlungen von Wissen auf der Welt. Wenn man die Bibliothek von Hogwarts hinzunimmt, dann sind es sogar vier. Die eben genannte Bibliothek ist die wohl größte, frei zugängliche Sammlung an Wissen über Magie. Die anderen Sammlungen sind die chinesische Hang-Wu-Schriftensammlung, zu denen nur ein ausgewählter Personenkreis Zugang hat. Sie beinhaltet einen großen Teil der neueren Zauber und Flüche unserer Zeit. Im arabischen Raum gibt es eine Ibn-Al-Flachai genannte Sammlung von Sprüchen aus älteren Tagen. Nicht alle dieser in der, ebenfalls recht restriktiven, jedoch etwas freizügigeren und nach überprüfter Anmeldung zugänglichen, Bibliothek vorhandenen Schriften werden der guten Seite zugeordnet. Etwa die Hälfte der Schriften werden zu den dunklen Künsten gezählt.

Doch die wohl geheimnisvollste ist wohl die Mondbibliothek. Sie soll laut Angaben sämtliches magisches Wissen seit dem Anbeginn der Magie enthalten. Jeden Zauber, Gegenzauber, Fluch und Gegenfluch soll sie enthalten. Sämtliche Tränke sollen dort vorhanden sein. Es ranken sich abenteuerliche Mythen um dieses Machwerk. Eine zuverlässige Quelle berichtet, sie sei von einem schwarzen Magier erschaffen worden. Eine andere zuverlässige Quelle berichtet, es sei die Magie selbst, die sich diesen Ort des Wissens geschaffen hat. Andere ebenso zuverlässige Quellen behaupten, einer der ersten Magier sei so langlebig, dass er sich hier ein Gedächtnis geschaffen hat, mit dem er verbunden sei, um nicht zu vergessen. Er soll sie so verzaubert haben, dass sie alles Wissen sammelt und ihm dann mitteile, wenn er es denn brauche.

Sie sehen also, es ist nicht leicht, etwas darüber zu erfahren. Ich habe es mir zur Aufgabe gemacht, verlässliche Daten zu sammeln und niederzuschreiben. Doch ich habe kaum welche gefunden \gst zumeist in Form von Rätseln. Das aussagekräftigste Rätsel habe ich mir notiert.

Es lautet: \enquote{Um dorthin zu gelangen, nimm den direkten Weg. Wähle den richtigen Zeitpunkt und du wirst dein Ziel erreichen. Übe das Reisen ohne Zeit und du wirst erkennen, wann und wohin du musst. Das Wissen wartet auf dich und wird ständig wachsen. Ein Leben reicht nicht aus, um alle Geheimnisse zu ergründen. Aber hüte dich vor dem Wächter. Wenn du ihn gegen dich hast, hilft auch keine Flucht mehr. Dann hilft nur noch beten und der Übergang in das Danach.}

In einem ersten naiven Versuch an das Rätsel heranzugehen, habe ich erst einmal alles wörtlich genommen. Mondbibliothek. Mir fallen zwei Sachen dazu ein. Eine Bibliothek in einer Kugel. Oder eine Bibliothek auf dem Mond, beziehungsweise unter dessen Oberfläche. Es heißt schließlich: \enquote{\aabs nimm den direkten Weg.} Der nächste Hinweis: \enquote{Wähle den richtigen Zeitpunkt\abs} könnte auf eine bestimmte Mondphase hinweisen. Neumond wäre wohl nicht praktikabel, da man nichts sieht. Vollmond wäre da praktischer. Auf den Versuch, dort hin zu apparieren, könnte der Hinweis \enquote{Übe das Reisen ohne Zeit} deuten. Apparieren braucht ja keine Zeit. Dass man für das ganze Wissen mehr als ein Leben braucht, scheint zu beweisen, oder zumindest darauf hinzudeuten, dass diese Bibliothek schon sehr lange besteht. Und der Hinweis mit dem Wächter könnte auf ein Fabelwesen deuten, das Zauberern und Hexen gefährlich werden könnte. Mir fallen da nur Drachen, Greife, Harpyien und Furien ein. Mit den restlichen gefährlichen Tieren kommt ein durchschnittlich begabter Zauberer zurecht.

Da ich begeisterter Hobbyastronom bin und mich mit dem Mond auskenne, zumindest behaupte ich das von mir, habe ich ihn mal abgesucht, aber nichts gefunden. Entweder ist das mit dem Namen nicht wörtlich zu verstehen, oder aber der Zeitpunkt war falsch. Es könnte aber auch von hier nichts zu sehen sein.

Andere Rätsel waren leider nicht so aussagekräftig wie das eben genannte. Der Vollständigkeit halber führe ich sie einmal auf.

\enquote{Reise direkt, ohne Zeit zu verlieren, dorthin, wo du, Reisender, das Wissen zu suchen gedenkst. Reise nicht über Los. Gib deswegen kein Geld aus.}

\enquote{Betritt den Ort, an dem du Wissen suchst, unbewaffnet und ohne Scheu. Der Wächter wird dich empfangen und in dein Herz blicken. Nur, wenn er dich für würdig hält, wird er dir den ersehnten Zugang gewähren.}

\enquote{Nur der bußfertige Mensch wird sie finden. Die Bibliothek des Wissens, die alle Geheimnisse enthält.}

\enquote{Ein erleuchtetes Wesen musst du sein. Nicht diese rohe Materie. Sie muss dich umgeben, dich mit allem Verbinden. Dem Stein, dem Baum und dem Wasser, das dich umgibt. Allgegenwärtig ist sie.}
\end{buch}

Harry und Katharina schauten sich an. Sie standen wenige Zentimeter nebeneinander. Die Luft von Erotik geladen, doch jeder der beiden wusste, dass sie nicht der anderen Person in diesem Raum galt. Sie lächelten sich an.

\enquote{Weißt du, dass ich dich noch nie geküsst habe?}, fragte Harry.

\enquote{Was war mit unserem Valentiskuss?}, fragte Katharina nach.

\enquote{Der war verabredet und nur Show. Der zählt nicht.}

\enquote{Und der, als du diese Anziehungskraft hattest?}, fragte sie nach.

Harry öffnete seinen Mund, dachte aber noch einmal kurz nach. \enquote{Da haben wir uns nicht geküsst.}

\enquote{Stimmt}, gab sie zurück. \enquote{Na dann.}

\enquote{Aber dann ist der Augenblick des Moments weg}, warf Harry ein.

\enquote{Und wenn du es nicht tust, dann machst du dir bestimmt Vorwürfe.}

\enquote{Und wenn wir ihn verschieben, falls ich mir Vorwürfe mache?}

Katharina dachte kurz nach. \enquote{Und was ist mit mir?} Sie stellte das Buch an ihren Platz zurück und nahm dann Harrys Kopf zwischen ihre Hände. \enquote{Diese Gelegenheit kommt nie wieder, glaube mir.} Sie küsste ihn sanft. Es war ein unschuldiger Kuss. Ein Kuss, den eine Schwester ihrem Bruder geben würde, um ihm zu zeigen, dass sie immer hinter ihm steht, egal was passiert.

Harry lächelte dankbar und umarmte sie. Er sog den Duft ihrer Haare ein, die für wenige Sekunden sich wieder in kleine Schlangen verwandelten, weil sie die Beherrschung verlor. Als er sie wider losließ, sah er gerade noch die Schlangen, die sich wieder in Haare verwandelten. \enquote{Daran musst du noch arbeiten. Es ist gefährlich, wenn du deinen Freund küsst, oder mit ihm schläfst und er dann versteinert, weil du einen unglaublichen\abs}

Sie hielt ihm den Mund zu. \enquote{Ich kann’s mir vorstellen\abs Nein, das will ich besser nicht.} Sie ließ ihn wieder los. \enquote{Aber du hast recht. Sag mal, hast du eine Idee, wie ich das machen könnte?}

\enquote{Übe.}

\enquote{Du weißt dann aber schon, wie hoch die Zahl der Jungs auf der Krankenstation sein wird, wenn ich mit jedem erst einmal schlafen muss. Madame Pomfrey wird sich bei mir in doppelter Hinsicht bedanken.} Harry hob nur eine Augenbraue. \enquote{Erst einmal die Menge an Alraunensaft und die vielen belegten Betten, und dann noch die Menge an Verhütungstränken die ich brauche, wenn ich\abs} Sie brach ab. \enquote{Du hast das anders gemeint, stimmt’s?}

Harry nickte. \enquote{Du musst lernen, deine Gefühle zu beherrschen. Deine Gefühle dürfen deinen Kopf nicht erreichen, sie müssen am Sitz der Gefühle bleiben und nur deinen restlichen Körper erreichen. Deine Kopfhaut muss davon unbeeindruckt bleiben.}

Mit einer leicht angehobenen Augenbraue fragte sie Harry: \enquote{Übst \accentuate{du} mit mir?}

\enquote{Warum fragst du da ausgerechnet einen Gryffindor?}

\enquote{Weil du der einzige bist, dem meine Schlangen nichts ausmachen.}

\enquote{Du meinst, ich soll dich sexuell stimulieren? Würdest du dich einfach so vor mir ausziehen und dich befingern lassen?}, fragte er ungläubig nach.

Katharina wurde rot, was Harry ein Lächeln entlockte. \enquote{Ich dachte eher daran, dass du auch nackt dabei\abs Vergessen wir das. Das kann ich nicht von dir verlangen.}

\enquote{Ist dir das zu peinlich?}

\enquote{Nein, aber es wirft kein gutes Bild auf mich.}

\enquote{Du meinst, die Leute wissen schon davon?}

\enquote{Das nicht, aber\abs}

\enquote{Was ist es dann?}

\enquote{Ich mag dich nicht}, er hob eine Augenbraue an, \enquote{so sehr, dass ich mich dir vollkommen nackt präsentieren könnte.}

Harry trieb das Spielchen weiter. \enquote{Setze einen Hut auf, dann bist du nicht nackt.}

Katharina patschte ihm nun mit der flachen Hand auf seine Brust. \enquote{Du bist unglaublich,\abs unmöglich, weist du das?}

\enquote{Dass ich unglaublich bin, weiß ich und dass ich unmöglich bin auch. Spätestens, seit Hermine das mit mir gemacht hat.}

\enquote{Was mit dir gemacht?}, fragte sie nach. \enquote{Sie hat dich schon mal stimuliert?}

\enquote{Nein, Katharina. Sie hat mir, wie du eben, mit der flachen Hand\abs war das eben eine Retourkutsche?}

\enquote{Kann sein!}, sagte sie und schickte sich an, den Bereich zu verlassen.

Harry nahm das Buch mit und folgte ihr. Kurz vor dem Verlassen der Bibliothek zeigte er Madame Pince den Buchdeckel und ging weiter. Da sie jedes der Bücher in der Bibliothek kannte und mit einem Zauber belegt hatte, konnte man keine der Bücher mitnehmen, ohne sich vorher eingetragen zu haben. Aber da sie von diesen Büchern nichts wusste, konnte man sie so mitnehmen. Sie würden ihren Weg eh wieder selbstständig zurückfinden. Spätestens nach zwei Wochen waren sie wieder an ihrem Platz. Man musste sich also erneut in die Bibliothek aufmachen, falls man es länger brauchen sollte.

\enquote{Mit dem Reisen ohne Zeit könnte apparieren gemeint sein}, schlug Katharina vor.

\enquote{Aber das braucht doch Zeit. Du warst doch in der Stunde dabei, als\abs Nein, du kommst erst nächstes Jahr in den Kurs.}

\enquote{Mir ist schon klar, dass das Zeit braucht. Aber bedenke mal, wie alt das Buch ist. Es ist von Sechzehnhundert-noch-was. Ich glaube nicht, dass man damals schon gewusst hat, dass apparieren, wenn auch wenig, Zeit braucht.}

\enquote{Da könntest du recht haben.}

\enquote{Mal was anderes. Gibt es die DA überhaupt noch?}

\enquote{Ja}, antwortete er, ohne darüber nachzudenken.

\enquote{Darf ich dran teilnehmen?}

\enquote{Du bist schon die zweite Person aus eurem Haus, die mich das fragt.}

\enquote{Wer ist die andere?}, fragte sie nach.

\enquote{Das wirst du übermorgen um fünf herausfinden. Wir sind im alten Tränkezimmer im zweiten Stock.}

Katharina nickte und beide machten sich in verschiedenen Richtungen auf, um ihr Ziel zu erreichen.




\begin{kommentar}
Harry sieht sich die Erinnerungen von Severus an. Dort bemerkt er, dass eine Person ausgeblendet wurde. Zuerst denkt er, dass es daran liegt, dass er die Erinnerungen mit der Hand umgerührt hat. Aber dem ist nicht so. Erst im nächsten Teil erfährt er, dass es sich um seine Tante handelt.
\end{kommentar}

\chapter{Magie sammeln}


Wieder in der Stadthalle von Hogsmeade wurden die Paare für die Apparitionsversuche ausgelost. Die Apparatur spuckte gerade zwei weitere Kugeln aus. \enquote{Harry Potter}, ließ die kleine Kugel verlauten, als sie zerplatzte. Harry trat vor und wartete auf seinen Partner. Eine weitere Kugel schoss hervor und tönte \enquote{Luna Lovegood.} Luna trat ebenfalls aus der Menge hervor.

Harry hörte von Professor Elber ein: \enquote{Das könnte interessant werden. Auf drei \gst Eins. Zwei. Drei.} Luna und Harry apparierten gleichzeitig zur anderen Seite der Halle. Dort angekommen schienen beide verwirrt zu sein und waren etwas wackelig auf den Beinen. Professor McGonagall und Professor Elber liefen sofort zu den beiden, welche sich bereits anschauten und lauthals zu Lachen anfingen.

\enquote{Was ist los, Mister Potter, Miss Lovegood}, rief Professor McGonagall.

\enquote{Ich bin wieder in meinem Körper}, sagten Luna und Harry fast gleichzeitig.

\enquote{Sind Sie sicher?}, fragte Professor McGonagall nach.

\enquote{Aber ja}, antwortete Luna.

Professor Elber zog seinen Zauberstab, richtete ihn auf die beiden und sprach denselben Spruch wie schon einmal, um festzustellen, wie beide miteinander verbunden waren. Dann sagte er sehr ernst. \enquote{Ich muss euch beide enttäuschen. Das hat die Sache nur noch verschlimmert.}

Professor McGonagall schaute ihn nur an. \enquote{Wie meinst du das?}, fragte sie ihn.

\enquote{Die Signale, die vorher vom anderen Körper empfangen wurden, wurden nicht etwa zurückgesetzt, wie es sein sollte. Sie werden jetzt wieder reflektiert. Es scheint nur so, als ob jeder in seinem eigenen Körper ist. \gst Harry, apparieren Sie mal alleine.}

\enquote{Meinen Sie, dass das klug ist Professor?}, fragte Harry.

\enquote{Keine Ahnung}, sprach Professor Elber. \enquote{Aber wenn Sie nicht apparieren können, dann ist der Unterricht so lange für Sie zu Ende, bis Ihr Zustand aufgelöst wurde.}

Harry apparierte nun an das andere Ende der Halle. Doch er blieb in seinem Körper.

\enquote{Und jetzt Sie, Luna.} Luna tat dasselbe und es passierte ebenfalls nichts anderes. \enquote{Nun ja, es scheint, dass sie zumindest alleine apparieren können. Immerhin. Sie müssen sich halt abstimmen. Aber das dürfte für Sie kein Problem sein.} Er grinste.

Er war wieder in seinem Körper und hatte erst einmal Hunger. Luna hatte in seinem Körper anscheinend noch nichts gegessen. Also schlich er sich nach der Stunde erst einmal in die Küche. Er ging den Gang vor der Küche entlang, bis er zum Bild mit der Obstschale kam. Er streichelte die Birne, aus der sofort ein Türgriff wuchs. Er betrat die Küche und stieg die wenigen Stufen hinunter.

Die kleinen Elfen schauten ihn an. \enquote{Wer du sein?}, fragte ihn ein unbekannter Elf.

\enquote{Harry. Harry Potter}, antwortete er ihm.

Die anderen gingen sofort ihrer Arbeit nach. Harry setze sich an den Rand des Tisches und der kleine Elf brachte ihm sofort Essen. Harry nahm sich eine Scheibe Brot, etwas Hähnchen und goss aus einem großen Krug Kürbissaft in seinen Becher. Er stoppte in der Mitte und füllte den Rest mit klarem Wasser auf. Er war gerade mit Essen fertig, als neben ihm Dobby erschien.

Sie schauten sich fragend an.

Dann sagte Harry: \enquote{Ich bin's, Harry.}

\enquote{Harry Potter, Sir}, gluckste der kleine Elf. \enquote{Schön Sie wiederzusehen.} Er nahm ihm gegenüber Platz. \enquote{Es freut Dobby, dass Sie wieder in Ihrem Körper sind.}

\enquote{Danke Dobby}, antwortete Harry. \enquote{Aber unser Problem ist noch immer nicht vollständig gelöst.}

\enquote{Nicht gelöst?}, fragte Dobby. \enquote{Aber Harry Potter ist wieder in seinem eigenen Körper.}

\enquote{Schon Dobby. Aber Professor Elber sagte uns, dass unsere Sinneseindrücke jetzt nicht nur einmal, sondern zweimal reflektiert werden.}

Dobby machte große Augen.

Harry wischte sich seine Hände mit einer Serviette ab und Dobby verschwand, nur um kurz darauf mit einer Schale klaren Wassers, einem Stück Seife und einem Handtuch neben ihm aufzutauchen. Harry wusch sich die Hände und bedankte sich bei Dobby, bevor er die Küche wieder verließ.

Es wurde Zeit, mit Professor Snape über seine Erlebnisse zu reden. Es war gerade noch Unterricht und er könnte eventuell in seinem Büro sitzen. So suchte Harry die geheime Passage auf und machte sich kurz bemerkbar. Er horchte an der Tür und öffnete diese. Das Regal schwang zur Seite und Professor Snape sah auf.

\enquote{Potter. Was verschafft mir die Ehre?}, fragte er.

\enquote{Ich habe mir Ihre Erinnerungen angesehen.}

\enquote{Ah, ja}, gab er emotionslos zurück.

\enquote{Ich habe eine Frage.}

\enquote{Ich werde Ihnen keine Fragen zu Ihrer Mutter, geschweige denn Ihrem Vater, beantworten.}

\enquote{Das will ich auch nicht wissen. Ich möchte etwas anderes wissen. Eigentlich zwei Dinge. Eine Sache zur Erinnerung und eine zu einem ähnlichen Thema. \gst Als Sie mit meiner Mutter im Zug saßen, war da noch eine andere Person anwesend.} Harry drückte es als eine Tatsache aus, da er wusste, dass Snape wie üblich sagen würde, er hätte die Erinnerung doch selbst gesehen, sollte er fragen, ob noch eine weitere Person anwesend gewesen wäre. \enquote{Wissen Sie, wer es war?}

Snape stutzte. Solch eine Frage zu einer seiner Erinnerungen kam ihm doch etwas merkwürdig vor. Er durchforstete sein Gedächtnis, bis er zu dem Zeitpunkt kam, indem er Lily im Zug gegenüber saß. Sofort zog sie ihn in seinen Bann. Er musste sein Gesicht gedanklich abwenden und im Abteil umherschauen. Aber außer einer verzerrten und einfarbigen grauen Masse in Menschengestalt, saß dort niemand.

\enquote{Nein}, antwortete er.

\enquote{Auch nichts Verschwommenes, menschenähnliches?}, fragte er nach.

Snapes Auge zuckte kurz und Harry empfing ein Bild aus seinen Gedanken. Die Figur, so wie Snape sie gesehen hatte, und nicht, wie man sie in einem Denkarium sah. Doch die Konturen waren nicht klarer oder schärfer.

\enquote{Nein}, antwortete er abermals.

Doch Harry wusste es bereits. \enquote{Noch eine Frage, dann sind Sie mich los.}

\enquote{Schön. Dann habe ich endlich Ruhe vor Ihnen}, kam es wieder sarkastisch. Immer wieder brach es bei ihm durch. Er durfte ihn nicht zu sehr an ihn heranlassen. Er hatte immer noch eine Aufgabe.

\enquote{Wissen Sie, ob es einen Unterschied gibt, ob man eine Erinnerung mit dem Finger, oder dem Zauberstab in einem Denkarium umrührt?}

\enquote{Wieso glauben Sie, dass ich so etwas weiß? Immerhin hat unser Schulleiter eines der wenigen noch existierenden Denkarien, weltweit.}

\enquote{Ein einfaches Nein hätte es auch getan. Danke Professor und gute Nacht. \gst Halt, da fällt mir noch was ein. Kurz vor Ihrem Tod hat ihnen da meine Mutter ein Geheimnis anvertraut? Etwas, was Sie sonst niemandem erzählt hatte?} Snape sah ihn einfach nur an. \enquote{Etwas über ein Geschwisterchen?}

\enquote{Sie hatte mal erwähnt, dass sie sich noch ein weiteres Kind wünscht, mehr nicht. \gst Gute Nacht.}

Harry bedankte sich. Er wusste, dass er jetzt gehen musste. Dann verabschiedete er sich und ging. Zurück im Gang sinnierte er: \gedanke{Entweder hatte er bisher keine Zeit, den Gang zu verschließen, oder er konnte es nicht. Oder er möchte, dass ich ständig zu ihm kann, falls ich etwas wissen möchte. Andererseits, wenn ich mich ohne triftigen Grund dort aufhalten sollte, gibt es Strafarbeiten, oder ich fliege von der Schule.}

Nachdem er den Gemeinschaftsraum durchlaufen hatte und in seinem Zimmer war, legte er sich ins Bett und dachte nach. \gedanke{Sprouts Aufsatz kann ich morgen früh machen, den für Snape habe ich schon fertig. Und für Verwandlung habe ich noch kein Thema. Aber darüber kann ich mich morgen mit Hermine unterhalten. Vielleicht fällt ihr etwas ein. Meinen VgddK-Aufsatz werde ich Morgen auch noch beenden. \gst Und was meine Vision von meinen Eltern anbelangt, da bin ich so schlau wie zuvor. Schade, dass keiner weiß, ob meine Mutter schwanger war.} Er stand auf und ging wieder nach unten.

Dean überredete ihn zu einer Runde Spreng-Schnipp-Schnapp. Nach vier verlorenen zwei gewonnen Runden und angesengten Augenbrauen wollte Dean noch seine Schachkenntnisse verbessern und spielte gegen Harry.

Als Harry müde wurde, bedankte er sich und bereitete sich fürs zu Bett gehen vor. Er stieg in sein Himmelbett und ließ die Vorhänge offen.

Ron, der ebenfalls schon im Bett lag, meinte: \enquote{Ich habe mir mal ein paar Punkte von deiner Karte angesehen. Du weißt schon\abs Auf jeden Fall gibt es eine Tür im zweiten Kellergeschoss, die ich nicht aufbringe. Du kannst ja mal schauen.}

Harry nickte, sodass Ron wusste, er habe ihn verstanden. Danach drehte er sich zur Seite und fiel in einen traumlosen Schlaf.

Am nächsten Tag stand er um fünf Uhr bereit, um seine beiden neuen Mitschüler für die DA zu empfangen. Pünktlich kamen sie an und schauten sich erstaunt um. Im Raum lagen Kissen auf dem Boden, Übungspuppen standen herum und ein großer alter Spiegel stand in der Ecke. Poster mit verschiedenen Figuren und Bewegungen hingen an der Wand.

\enquote{Schön, dass ihr da seid, dann können wir ja anfangen.}

\enquote{Wo sind denn die anderen?}, fragte Adrian.

\enquote{Die kommen erst in einer Stunde. Ich habe euch schon jetzt hier herbestellt, damit ich weiß, was ihr könnt.} Damit zog er seinen Zauberstab. \enquote{Und, um euch auf Spur zu bringen. Die ersten Termine werden etwas stressig für euch.}

Dann warf er einen Verwirrungszauber auf die beiden. Er trat um sie herum und wartete. Sichtlich geschockt und verwirrt von der Aktion zogen sie ihre Zauberstäbe und zielten auf den vermeidlichen Harry vor ihnen. Doch der Zauber hatte keine Wirkung. Auch schien sich der Harry vor ihnen kaum zu bewegen. Er machte keine Anstalten auszuweichen, oder sonst eine Aktion gegen sie zu starten.

\enquote{Das ist ein einfacher Verwirrungszauber}, hörten sie plötzlich eine Stimme hinter sich. Da sie die Position mit ihren Ohren recht gut zuordnen konnten, drehten sie sich um und sahen in die Richtung, aus der das Geräusch kam. Langsam wurde er sichtbar und die beiden zögerten nicht und versuchten ihn zu entwaffnen.

Harrys Zauberstab zog nur leicht an seiner Hand, sodass er keine Mühe hatte ihn festzuhalten. Der Expelliarmus war einfach zu schwach. Dann führte Harry ihn aus und hatte kurz darauf die beiden Zauberstäbe in seiner Hand. Er löste den Zauber auf und gab die Stäbe wieder zurück.

\enquote{Den werden wir als Erstes üben. Danach werden wir Vergrößerungs- und Verkleinerungszauber durchnehmen. Die solltet ihr in einer Stunde beherrschen, denn das, was ich heute geplant habe, baut darauf auf. Die anderen haben mit diesen Zaubern schon viel Erfahrung, also strengt euch an.}

Zuerst dauerte es eine Weile, bis die beiden begriffen hatten, wie man die Zauber richtig ausführt, aber dann klappte es recht gut. Adrian war damals zu jung, um in das Inquisitionskommando aufgenommen zu werden; er hatte es nicht einmal versucht, und Katharina wollte nicht. Sie hatte keine Lust auf Streifgänge durchs kalte abendliche Schloss. Kurz vor sechs waren beide erschöpft aber glücklich.

\enquote{Die nächsten drei vier Termine werden wir noch vorher üben.}

Die beiden nickten.

Dann schlug die Uhr Sechs und die ersten der DA kamen herein. Was keiner der Gruppe wusste war, dass die Hausgeister davon Wind bekommen hatten und sich in einer Bücherreihe versteckten. Lediglich deren Augen sah man über den Büchern, falls man wusste, wonach man suchte. Geduldig und mit Spannung beobachteten sie das Treffen.

Katharina und Adrian standen neben Harry und sahen auf den Eingang. Nach und nach kamen die anderen der DA in das Klassenzimmer. Katie und Susan waren die einzigen, die sie anlächelten, was den beiden die Anspannung etwas nahm. Sie würden nachher mit ihnen üben. Der Rest war vor allem eher desinteressiert. Nur wenige zeigten eine noch offene Ablehnung. Darunter auch Zacharias Smith. Der sah immer noch nicht begeistert aus. Rons Gedanken konnte man momentan gar nicht aus seinem Gesicht deuten.

\enquote{Ist es jetzt also so weit, dass man Slytherin aufnimmt?}, fragte Zacharias provokant.

\enquote{Ich finde es beispiellos, wie rassistisch und arrogant Sie gegenüber allen Slytherin sind}, sagte eine Stimme aus Richtung des Bücherregals. Der Geist des blutigen Barons kam hervor und sah Zacharias bedrohlich an. \enquote{Wie können Sie alle Schüler meines Hauses so hassen? Ist es Ihnen egal, wer hinter der Uniform steckt?} Beständig flog er auf ihn zu, was Zacharias dazu veranlasst, langsam rückwärtszugehen. \enquote{Merken Sie sich die Gesichter bei der Auswahlzeremonie und fangen dann an sie zu hassen, sobald der Hut den Namen Slytherin ausruft?} Jetzt stand Zacharias mit dem Rücken an der Wand und der blutige Baron schwebte nur wenige Zentimeter vor ihm. Er hielt seine Hände links und rechts von seinem Gesicht und es sah so aus, als ob er sich an der Wand abstützte. Seine Hände waren dabei nur wenige Zentimeter in der Wand. Die drei anderen Hausgeister kamen nun ebenfalls hervor und sahen ebenso beleidigt zu Zacharias und den anderen, die den beiden Slytherin ablehnend gegenüber standen. \enquote{Vielleicht sollte ich dafür sorgen, dass Sie eine Weile zu den Slytherins gesteckt werden, damit Sie mal am eigenen Leib zu spüren bekommen, wie es ist, von drei Häusern gehasst zu werden.}

Zacharias wollte gerade etwas dagegen sagen, doch der blutige Baron schnitt ihm mit einer Handbewegung das Wort ab. Er gab ihm durch seinen Kopf hindurch eine schallende Ohrfeige. Danach schwebte er durch ihn hindurch und verließ den Raum. Zacharias’ Gesichtsfarbe wurde blauer und es bildeten sich Eiskristalle. Er konnte nicht mehr blinzeln.

\enquote{Geschieht Ihnen recht}, sagten die drei anderen Hausgeister und schwebten neben Zacharias durch die Wand. Hermine und Alicia kamen auf ihn zugestürmt und versuchten, die Erfrierungen zu lindern und Zacharias zu helfen. Dann brachten sie ihn zum Krankenflügel.

\enquote{Und wir machen weiter}, sagte Harry. \enquote{Heute werden wir uns eine andere Schicht der Magie ansehen. Sucht euch einen Platz und jemanden zum Üben.}

Katie und Susan sahen sich kurz an, nickten einander zu und schnappten sich Adrian und Katharina. Die anderen teilten sich entsprechend auf und bildeten mal wieder neue Paarungen, um möglichst verschiedene Angriffstechniken und Bewegungen zu kennen.

Nachdem sich alle hingesetzt hatten, begann Harry sich zu konzentrieren und seine Augen zu schließen. Dann sagte eine Stimme hinter den anderen: \enquote{Illusionen und Täuschungen sind wichtige Techniken.} Alle drehten sich um und sahen Harry. Als sie wieder an die alte Stelle blickten, war er nicht mehr da. Der andere Harry stand auf und lief durch den Raum. \enquote{Tarnen und Täuschen ist wichtig, wenn man sich verteidigen will. Nicht nur für Auroren, auch, wenn man sich zu Hause verteidigen will. Es sind schwere Zeiten, in denen wir sind und denen wir entgegensteuern. Wenn ihr so etwas könnt, dann könnt ihr eure Familie gegenüber den Todessern verbergen. \gst Fangen wir an.}

\trenn

Lucius und Severus saßen in Lucius’ Arbeitszimmer und hatten den Raum magisch abgeschirmt.

\enquote{Und, wie war es bei deiner Familie?}, fragte ihn Severus ohne Umschweife, nachdem sich beide hingesetzt hatten.

Überrascht sah Lucius sein Gegenüber an. Unsicher darüber, was er sagen sollte, fragte er nach: \enquote{Woher glaubst du zu wissen, dass ich meine Familie, die mich verlassen hat, besucht habe?}

\enquote{Frederick}, sagte Severus.

Lucius fiel ein Stein vom Herzen. \enquote{Es war schön, mal wieder alle zu sehen. Weißt du Severus, als mir Tamara von diesem Brauch der Muggel erzählte, musste ich mir einen bösen Kommentar verkneifen. Der Dunkle Lord geht mir langsam\abs} Den Rest des Satzes sprach er nicht mehr aus.

Severus saß emotionslos da und sah ihn an. \enquote{Willst du nicht auch hier raus? Gefangen im eigenen Haus. Wie lange machst du das noch mit?}

\enquote{Der Dunkle Lord lässt mich nicht gehen. Wenn ich weg bin, dann kommen die anderen nicht raus. Als ich die Sachen für den Dunklen Lord besorgen musste, stand ich unter Beobachtung.} Dann stutzte er. \enquote{Wieso ist mir das nicht gleich aufgefallen? Die ganzen vier Stunden war ich praktisch nicht auffindbar. Wieso hat er ihm dann berichtet, dass ich meine Aufgabe erfüllt habe?} Severus streckte einen Zeigefinger in die Luft. \enquote{Du warst das? Was hast du gemacht? Imperius?}

Severus schüttelte den Kopf. \enquote{Falsche Erinnerungen. Den Zauber habe ich in einem Buch gefunden. Man kann zwar, wenn man geschickt ist, herausfinden, dass man manipuliert wurde, aber die originalen Erinnerungen wieder herzustellen ist schwierig. Und da unser Freund keine Veranlassung hat, nachzuschauen\abs}

Lucius grinste. \enquote{Bin ich froh, dass ich dich habe. Wenn ich daran denke, wie ich dich damals angeworben habe. Ich muss mich heute noch bei dir dafür entschuldigen.}

\enquote{Das hast du schon getan. Immerhin durfte ich Dracos Pate werden.}

\enquote{Das lag nicht an mir. Da hatte Narcissa ihre Finger im Spiel. Ich war anfangs dagegen. Aber mittlerweile bin ich auch darüber froh. So hatte er auf Hogwarts immer jemand, dem er vertrauen konnte. Da bin ich dir schon wieder etwas schuldig.}

Severus nickte nur. \enquote{Wie sieht es jetzt aus? Verlässt du das Manor?}

\enquote{Wie denn? Wenn ich versuche zu apparieren, wird der Dunkle Lord mir sofort auf den Fersen sein. Dann verbringe ich die Tage und Nächte in meinem eigenen Verlies.}

\enquote{Darüber solltest du dienstags mal laut nachdenken.}

\enquote{Wieso dienstags? Da ist doch immer mein Schach\aabs Du meinst Frederick? Was kann der schon\abs Der appariert hier einfach rein und raus., er ist ja Tamaras Pate. Auch wieder meine Frau. Du meinst, er könnte mir dabei helfen? Aber wie?}

\enquote{Hast du nicht mitbekommen, was er mit Bellatrix gemacht hat, als sie auf deine Tochter eingewirkt hat?} Lucius schüttelte nur den Kopf. \enquote{Ohne Zauberstab hat er so etwas wie den Cruciatus auf sie geworfen. Kurz danach ist er mit beiden Kindern gegangen. Er hat sie in Sicherheit gebracht.}

\enquote{Dass er sie mitgenommen hatte, weiß ich. Auch, dass es zu ihrer Sicherheit war, aber nicht warum genau. Bellatrix hat mir\abs Klar, wenn sie es selbst betrifft.} Lucius dachte nach. \enquote{Ich habe das Gefühl, dass es bald gegen Hogwarts geht. Wenn es so weit ist, dann werde ich wohl den Entschluss fassen, zu gehen. Alle, die im Manor sind, werden festgehalten. Leider gilt das nicht für den Dunklen Lord.}

\enquote{Das ist wohl wahr! Und ich kenne weder eine Möglichkeit, ihn hier festzuhalten, noch jemand, der es vermöchte. Vielleicht schafft er es auch, diesen Zauber aufzuheben. Dann wären die anderen wieder frei.}

\enquote{Meinst du, Frederick schafft es trotzdem durch? Falls du eingeschlossen sein solltest?}

\enquote{Ich habe ihn das ganze Jahr über in Aktion erlebt. Sein Wissen über Magie ist\abs eigenartig.}

\enquote{Was meinst du mit eigenartig?}

\enquote{Wie soll ich es sagen.} Severus überlegte. \enquote{Weißt du, was er den Siebtklässlern in der ersten Stunde gezeigt hat?} Lucius schüttelte den Kopf. Doch inmitten der Bewegung hielt er inne und nickte. \enquote{Dämonenfeuer. Doch es verhielt sich nicht so, wie du und ich es kennen. Wie wir es gelernt haben.} Severus erschuf mit seinem Stab eine kleine brennende Schlange. Lucius zuckte zuerst, beruhigte sich dann aber wieder, als Severus keinerlei Anstalten machte, sich darum zu kümmern, außer der kleinen Schlange zuzusehen, wie sie über den Tisch kroch. \enquote{Alles eine Frage der Intention, sagte er. Dieses Feuer hier ist harmlos, obwohl es Dämomenfeuer ist. Eigenartig, nicht? Deswegen nennt er es wohl lebendiges Feuer.}

\enquote{Hat er außer dieser Aktion noch etwas gemacht, das das Wort eigenartig rechtfertigt?}

\enquote{Er hat vor der gesamten Schule über die unverzeihlichen Flüche referiert. Er beantwortet Fragen zu dunkler Magie, wo selbst ich keine Antwort geben würde.}

\enquote{Meinst du nicht könnte, Severus?}

\enquote{Das manchmal auch. Also habe ich ihm auch mal einige Fragen gestellt. Entweder er hat sie gleich beantwortet, oder er kam ein paar Stunden später mit einer Antwort. Sowohl dunkle, als auch helle Magie.}

\enquote{Draco und Tamara sagten mir, dass es keine dunkle Magie gibt. Ebenso keine helle. Ehrlich gesagt, habe ich das nicht so recht verstanden. \gst Warte. Es kommt auf die Intention an.}

Severus nickte.

\trenn

Als das heutige Treffen vorbei war, drückte Katharina Harry noch ein Pergament in die Hand, bevor sie, wie die anderen auch, ging. Harry entfaltete es, als er nur noch mit Ron, Hermine und Ginny alleine war.

\begin{brief}
Das ist mein Familienwappen, Harry. Vielleicht hilft es dir bei deiner Suche.
\signumspace
Katharina
\end{brief}

Darunter war das Bild eines Mondes. Er war dreiviertel gefüllt. Darüber lag ein Zauberstab, der nur über der hellen Seite des Mondes zu sehen und quer über dem Mond gezeichnet war. Einen Schritt weiter brachte ihn das vielleicht. Er musste nur die Mondphase herausfinden, dann konnte er sich darauf konzentrieren, einen Zugang zu finden. Er teilte den dreien seine Gedanken mit und auch das, was er herausgefunden hatte.

Dann lief er nachdenklich durch das Schloss.
Er brauchte etwas Zeit für sich. So ging er in den Aufzug und fuhr an dieselbe Stelle, an der er Fawkes getroffen hatte, saß auf dem kleinen Balkon an der Westseite des Schlosses, hatte den Steinboden angewärmt und ließ die Beine durch das Geländer baumeln. Er brauchte Zeit nachzudenken und hatte einen Platz gefunden, der nur ihm bekannt war. Er hatte schon mehrmals diesen Platz aufgesucht, an dem er Ruhe und Entspannung erfahren hatte, hatte seine Augen geschlossen und ließ die Magie durch ihn hindurch fließen. Dann dachte er an die Worte, die noch immer deutlich in seinem Kopf zu hören waren.
% "Nach meiner Grösse beurteilst du mich? Tust du das? Aber das solltest du nicht, denn die Macht ist mein Verbündeter, und ein mächtiger Verbündeter ist Sie. Das Leben erschafft Sie, bringt Sie zur Entfaltung. Ihre Energie umgiebt uns, verbindet uns mit allem. Erleuchtete Wesen sind wir, nicht diese rohe Materie. Du musst Sie fühlen die Macht die dich umgibt, hier, zwischen dir, mir dem Baum, den Felsen dort, allgegenwärtig ja, selbst zwischen dem Sumpf und dem Schiff."

\accentuate{Die Magie ist mein Verbündeter, und ein mächtiger Verbündeter ist sie. Ihre Energie umgibt uns, verbindet uns mit allem. Erleuchtete Wesen sind wir, nicht diese rohe Materie. Sie müssen sie fühlen, die Magie die sie umgibt, hier, zwischen Ihnen, mir, dem Baum, den Felsen dort, allgegenwärtig ja, selbst zwischen dem See und dem Stein auf seinem Grund.}

Er fühlte sich leicht und warm. Spürte, wie ihn ein warmer Energiestrom durchflutete. Er war vollkommen entspannt. Er hörte die Vögel, die am Himmel über ihm flogen, spürte den Wind auf seiner Haut und einen Gecko neben ihm atmen. Er sah durch seine geschlossenen Augen die Vögel auf ihn und auf das Geländer zufliegen, um sich dort, oder in seinem Schoß niederzulassen. Er öffnete die Augen und nahm einen Vogel auf seine Hand. Dieser schien keine Angst vor ihm zu haben. Beide Lebewesen betrachteten sich gegenseitig. Der Vogel wusste wohl, dass Harry ihm nichts tat.

Die Wand hinter ihm ging auf und zwei Gestalten traten erstaunt ins Freie.

\enquote{Hallo Hermine, Ron.}

\enquote{Woher\abs} stammelte Hermine.

\enquote{Woher wusstest du, dass wir das sind}, fragte Ron.

\enquote{Ich habe es gespürt}, sagte Harry. Der Vogel saß noch immer in seiner Hand. Hermine setze sich rechts von Harry neben den Gecko und Ron auf die andere Seite. Einige Vögel flogen davon und auch der Gecko verzog sich wieder, doch der Vogel in seiner Hand und wenige andere blieben und betrachteten die Neuankömmlinge neugierig.

\enquote{Was ist mit den Vögeln los, Harry?}, fragte Ron.

\enquote{Das ist ein Zauber, Ron}, sagte Hermine.

\enquote{Nein, kein Zauber. Magie.}

\enquote{Das meinte ich, Harry}, sagte Hermine.

Jetzt sah er Hermine an, nahm ihre Hand in seine und schaute ihr tief in die Augen.

\enquote{Harry}, sagte sie, als sie leicht rot wurde.

Er legte ihr den Vogel in ihre Hand und lies Hermines Hand wieder los. \enquote{Fühlt sich so ein herbeigezauberter Vogel an?}, fragte er sie.

Hermine sah den Vogel an und strich ihm über sein Gefieder.

\enquote{Nein}, sagte Hermine verwundert.

Dann sah er wieder durch das Geländer in die Ferne. \enquote{Wie seid ihr eigentlich hier hergekommen?}, fragte sie Harry nun.

\enquote{Wir machten uns Sorgen um dich}, sagte Ron. \enquote{Wir wussten nicht, wo du warst, also haben wir auf der Karte nachgesehen. Du warst\abs Harry du warst\abs außerhalb des Schlosses.}

Und Hermine fuhr fort: \enquote{Du warst einfach knapp außerhalb des Schlosses. Westflügel. Dritter Stock.}

Jetzt wusste Harry zumindest, dass ihn die Karte immer noch anzeigte. Er war immer noch entspannt und das warme Gefühl durchdrang ihn immer noch.

\enquote{Wir wollten irgendwie zu dir gelangen, fanden aber keinen Weg. Wir dachten, du seist in Schwierigkeiten. Also sind wir durchs Schloss gelaufen auf der Suche nach einem Lehrer. Wir fragten den Ersten, dem wir begegnet sind. Wir drückten unsere Sorge aus und sagten, dass wir zwar wissen, wo du bist, aber nicht, wie wir zu dir kommen können, oder wie du dort hingelangt bist. Harry, wir hatten uns Sorgen um dich gemacht.}

Und Ron erzählte weiter: \enquote{Wir haben ihm gesagt, dass du im dritten Stock im Westflügel bist, knapp außerhalb des Schlosses. Wir konnten uns nicht erklären, wie du dort hingekommen sein könntest. Der Professor sah uns erst ungläubig an, doch dann überlegte er. Er fing an leicht zu schmunzeln und stand dann auf. Er winkte uns zu und ging mit uns um ein paar Ecken. Dann drückte er einen Stein in der Wand, worauf diese sich teilte. Er gab uns zu verstehen, dass wir in den kleinen Raum eintreten sollten. Dann lehnte er sich herein und drückte auf einen kleinen Knopf. Die Wand verschloss sich, als er sich wieder aufrichtete. Wir beide erschraken, als der Boden zu vibrieren begann. Dann öffnete sich die Wand wieder und du warst da.}

Harry grinste. Er schloss wieder seine Augen und ließ die Magie ihn weiter durchströmen. An keinem anderen Ort im Schloss hatte er dieses Gefühl erhalten, selbst dann nicht, wenn es absolut ruhig war und er sich entspannen konnte. Jetzt endlich wusste er, wozu dieser Ort da war.

\enquote{Harry, die Wand}, schrie Ron plötzlich und schüttelte Harry am Arm. Der erschrak, ließ sich aber sonst nicht aus der Ruhe bringen.

\enquote{Harry}, ermahnte ihn Hermine, doch er reagierte nicht.

\enquote{Wie kannst du nur so ruhig sein, Harry}, brauste Ron auf.

\enquote{Ich nehme an, Harry wird seine Gründe haben}, sagte Hermine, sah Harry dabei an und setzte sich ebenfalls so wie Harry hin, schloss ihre Augen und entspannte.

Doch Ron ließ sich nicht beruhigen. Er stand vor der Wand und versuchte rein zu kommen. Harry seufzte resigniert. Er spürte den Stein fast auf seiner Hand, als er sie leicht nach vorne schob. Der Stein in der Wand versank und kurz darauf öffnete die Wand sich. Ron stolperte einige Schritte vorwärts, da er sich gegen die Wand gelehnt hatte. Im Inneren des Aufzuges wurde durch eine unsichtbare Hand ein Knopf gedrückt und die Wand schloss sich erneut.

Nun herrschte Ruhe.

\enquote{Wo ist Ron hin?}, fragte sich Hermine und schaute sich um.

\enquote{Wieder im Aufzug. Er hatte mich gestört.}

\enquote{Wie? Er hatte dich gestört.} Sie sah auf Harrys Hände. \enquote{Ohne Zauberstab?}, fragte sie ungläubig.

Harry öffnete die Augen und nahm Hermines Hand in seine. \enquote{Wie läuft es zwischen dir und Ron?}, fragte er sie. Hermine wurde warm um ihr Herz. \enquote{Es wird immer besser. So langsam kann er richtig entspannen, wenn wir alleine sind. Seine warmen Lippen fühlen sich gut an.} Dann sah sie auf ihre Hand, die Harry noch immer festhielt. Sie wollte sie zurückziehen, doch Harry hielt sie fest. Er griff an ihr Handgelenk und schloss sanft ihre Hand. Mit ihren Fingern bildete sie einen kleinen Hohlraum. Jetzt lag seine Hand sanft auf der Außenseite ihrer Finger. Dann spürte sie in ihrer Hand etwas Weiches. Harry nahm beide Hände von ihr und legte sie zurück in seinen Schoß.

Hermine öffnete langsam ihre Hand und darin lag ein kleiner Minimuff. \enquote{Harry}, sagte Hermine ganz erstaunt. \enquote{Wie hast du das gemacht?} Sie sah ihn erstaunt an.

Ihre Augen sahen einfach umwerfend aus.

\enquote{Du weißt, dass Ron sich glücklich schätzen kann, dich zu haben?}

\enquote{Lenk' jetzt nicht ab, Harry. Wie machst du das?}

\enquote{Ich weiß es nicht\abs noch nicht. An diesem Ort fällt es mir so leicht, das zu tun. Ich habe es schon in meinem Zimmer versucht, aber dort klappt es nicht annähernd so gut.}

\enquote{Du kannst so etwas in deinem Zimmer? Ohne Zauberstab?}

\enquote{Nicht so gut wie hier}, sagte Harry entschuldigend.

Hermine bekam große Augen. \enquote{Ich habe nicht einmal Dumbledore so etwas ohne Zauberstab machen sehen. Wir sollten zu ihm und es ihm sagen.}

\enquote{Und dann?}, fragte Harry. \enquote{Was machen wir dann? Werde ich Schulleiter?}

\enquote{Harry, das ist nicht witzig.}

\enquote{Nein ist es nicht, aber was will Dumbledore machen?}, fragte er.

\enquote{Lass uns erst einmal zu ihm gehen.}

Hermine stand auf und stand nun vor der Wand. Also blieb Harry nichts anderes übrig. Sie würde ihn physisch so lange traktieren, bis er nachgab. Er stand auf und sah den Stein in der Wand an, der sich nach innen bewegte und so die Wand zum Öffnen brachte. Hermine sah ihn mit leicht offenem Mund an. Im Inneren des kleinen Raumes drückte er den Knopf von Hand und nach einer kurzen Fahrt waren sie noch etwa sechzig Meter vom Büro des Schulleiters entfernt.

\enquote{Woher weißt du von diesen Aufzügen?}, fragte Hermine ihn, als sie unterwegs waren.

\enquote{Professor Elber ging mit Dumbledore zum Astronomieturm, als ich dort hin wollte. Ich lief fast in sie hinein. Also bot Dumbledore mir an, ihn zu begleiten, da ich eh schon spät dran war. Dann bog Professor Elber ab und zog uns mit sich, als wir sagten, dass es da nicht lang ging. Dumbledore war ebenso erstaunt über diese Aufzüge wie ich. Ich habe leider noch nicht alle Wege erkundet. Aber sobald ich alle habe, sage ich Ron und dir Bescheid. \gst Äh, wollte ich Ron und dir Bescheid sagen}, korrigierte er sich. \enquote{Aber sag mal, Hermine, woher weißt du davon?}

\enquote{Ron und ich sind auf Professor Elber gestoßen. Er hat uns einfach in den Aufzug gesteckt und uns direkt zu dir geschickt.}

Sie gaben dem Wasserspeier das aktuelle Passwort, das Harry durch Zufall in Erfahrung gebracht hatte und klopften kurz darauf an die Tür des Schulleiters.

\enquote{Herein}, erklang Dumbledores Stimme durch die Tür. Beide traten ein und sahen bereits Ron und Professor Elber. Ron saß auf einem Stuhl und Professor Elber betrachtete gerade interessiert ein paar Gegenstände in einem der Schränke in Dumbledores Büro.

Dumbledore sah beide fragend an. Dann zog Hermine Harry an seinem Arm und setzte ihn unsanft auf einen Stuhl neben Ron. Nun saß er gegenüber Dumbledores Tisch. \enquote{Professor? Harry möchte ihnen etwas erzählen.}

Professor Elber drehte sich um und lehnte jetzt mit dem Rücken an einem der Schränkchen.

Dumbledore wartete ab, was Harry auf dem Herzen lag. Doch dieser wusste nicht, wie er anfangen sollte.

\enquote{Zeig es ihm}, sagte Hermine. Noch immer kam keine wirkliche Reaktion von Harry. \enquote{Zeig es ihm}, fuhr ihn Hermine jetzt deutlicher an.

Harry nahm langsam seine Hand in seinen Schoß und drehte die Innenseite nach oben. Dann krümmte er seine Finger und schloss sie. Er hatte jetzt eine kleine Kuhle in seiner Hand. Er konzentrierte sich und erfuhr wieder das warme Gefühl. Es schien ihm unglaublich leicht von der Hand zu gehen. Als er sie wieder offen hatte, lag in seiner Hand ein Minimuff. Dann sah er zu Dumbledore. Er hatte einen Gesichtsausdruck, wie ihn Harry noch nie gesehen hatte.

\enquote{Harry}, war alles, was Dumbledore herausbrachte und als er sich danach wieder in seinen Stuhl fallen ließ, den er verlassen hatte, als Harry seine Hand öffnete und Dumbledore den Minimuff sah, musste er schlucken.

Dieser rollte nun in Harrys Hand vergnügt umher und gab surrende und schnurrende Geräusche von sich.

Plötzlich begann Harrys Kopf zu schwirren. \enquote{Mir wird schlecht und schwarz vor Augen}, sagte er noch und sackte dann zusammen. Er spürte noch, wie Hermine ihn aufhielt und verhinderte, dass er mit seinem Kopf auf dem Boden aufschlug. Er konnte nichts mehr sehen, nichts mehr schmecken, nichts mehr riechen und nichts mehr spüren, konnte nur noch hören.

\enquote{Holt Madame Pomfrey.}

\enquote{Nein. Sie kann nichts für ihn tun. Er muss sich von alleine wieder erholen.}

Plötzlich wurde ihm warm an einer Stelle, an der normalerweise seine Brust war. Er spürte keine Hand, aber da musste eine sein. Er merkte nur, wie ein Handflächen-großes Stück auf seinem Bauch warm wurde.

\enquote{Ah, was ich vermutet hatte.} Es war wieder Professor Elber. Er hatte schon verhindert, dass man Madame Pomfrey rief. \enquote{Ja, das ist gut. Sehr gut. Er wird schwächer.}

\enquote{Was soll daran gut sein, dass Harry schwächer wird}, schrie Hermine jetzt. Harry hatte sie noch nie so mit einem Lehrer sprechen sehen. Nein hören, korrigierte er sich.

\enquote{Ich meine nicht Harry. Ich meine\abs}, doch er verstummte. \enquote{Er muss erst einmal wieder seine Kräfte finden. Der abgetrennte Teil der Magie ist rastlos in seinem Körper. Das kann ich spüren.} Die Wärme bewegte sich nun höher auf seinen Brustkorb zu. \enquote{Die Magie muss erst einmal Ruhe finden und sich mit jeder Faser, ja jeder Zelle seines Körpers verbinden. Er wird bald wieder sehen können. Seine Sinne werden wieder zurückkehren. Nicht wahr Harry?}

\gedanke{Ja}, dachte er. \gedanke{Ich beginne langsam wieder meinen Körper zu spüren.}

\enquote{Sie sollten bis zum neuen Schuljahr diesen Ort meiden, Harry. Kehren Sie nicht mehr dorthin zurück}, sprach wieder Professor Elber.

Harrys Wahrnehmungen wurden immer besser. Er konnte jetzt wieder schemenhafte Gestalten erkennen. Dann \gst war er wieder da.

\enquote{Wie geht es dir?}, fragte ihn Hermine, während Ron ihn leicht unsicher ansah.

\enquote{Ganz gut}, sagte er zu Ron gewandt. \enquote{Leichte Kopfschmerzen.}

Dann sah er zu Professor Elber. \enquote{Er ist in mir, richtig? Ein Teil von ihm.} Dann sah er zu Dumbledore. \enquote{So wie das Tagebuch. Ein Teil von Voldemort ist in mir.}

Ron und Hermine sahen Harry ungläubig an. \enquote{Du warst weggetreten, Harry}, sagte Ron.

\enquote{Du fantasierst}, fügte Hermine hinzu.

Harry sah wieder zu Professor Elber und richtete sich auf. Professor Elber ging in die Hocke und sah Harry an. \enquote{Wie kommen Sie darauf?}

\enquote{Ich habe es irgendwie gespürt als ich da saß und die Magie mich durchströmt hat. Ein kleiner Teil von mir blieb kalt, während der Rest sich erwärmte. Ein kleiner Teil, der mich an Voldemort erinnerte, als ich versuchte ihn zu erfassen. Außerdem habe ich schon einmal etwas in mir gespürt. Ich konnte es aber nicht richtig zuordnen.}

Professor Elber zeigte ein sanftes Lächeln. \enquote{Sie wissen also, was mit Ihnen los ist? Dass Voldemort deshalb nicht sterben kann?}

Harry nickte. \enquote{Erst das Tagebuch, dann ich. \gst Meinen Sie er hat noch mehr \gst Seelenabspaltungen vorgenommen?}

Hermine und Ron verschlug es die Sprache. \enquote{Harry, wie kannst du nur so ruhig über solche Sachen reden. Das ist gefährliche schwarze Magie. Woher weißt du von diesen\abs diesen Horkruxen.}

Hermine hätte sich am liebsten selbst auf die Zunge gebissen. Doch Dumbledore schien sie nicht gehört zu haben. Bei Elber hatte sie das Gefühl, dass, falls er es gehört hatte, er einfach darüber weggehen würde. Das war merkwürdig. Er war schwer einzuschätzen. Irgendwie interessierte es ihn nicht, wenn ein Schüler etwas wusste, das er nicht wissen sollte. Das hatte Hermine schon recht bald bemerkt.

\enquote{Wie kann ich ihn bekämpfen?}, fragte Harry nun.

\enquote{Tja}, antwortete Dumbledore.

\enquote{Lassen Sie es einfach über sich kommen, Harry. Sie sind mittlerweile so stark, dass Sie seine Magie anziehen. Wenn Sie die gesamte Magie des Seelenteiles aufgenommen haben, dann wird der Rest absterben und Voldemort wird, sofern er keine weiteren Horkruxe angefertigt hat, sterben können. Dann ist der Zeitpunkt gekommen, ihn zu töten}, machte Professor Elber weiter.

Harrys Kopf fuhr nach oben. Er ignorierte den aufsteigenden Kopfschmerz. Er krabbelte auf seinen Professor zu. Seine Nasenspitze war wenige Zentimeter von ihm entfernt. Wütend schrie er ihn an. \enquote{Ich soll ihn töten, stimmt’s? Es liegt an mir!} Wut spiegelte sich in seinem Gesicht.

\enquote{Was wollen Sie machen? Mich schlagen?}, fragte Professor Elber Harry provozierend und stieß ihm mit seinem Finger in die Brust. Das war für Harry zu viel. Er stürzte sich auf seinen Lehrer und gab ihm eine Ohrfeige.

\enquote{Harry!!!}, schrien Ron und Hermine außer sich vor Entsetzen.

Doch als Harry begriff, was er getan hatte, rollte er sich von ihm, in eine Fötus-Haltung und begann bitterlich zu weinen.

\enquote{Harry, du kannst doch nicht einfach so\abs}, doch Hermine verstummte, als ihr Professor Elber einen Arm auf ihre Schulter legte.

\enquote{Das musste sein, Hermine. Es beschleunigt das Ganze. Er muss seine Wut herauslassen.}

\enquote{Wie meinen Sie das?}, fragten nun Ron und Dumbledore.

\enquote{Der Prozess entzieht Voldemorts Seelenteil gerade eine Menge Magie. Es wird schwächer.}

\enquote{Aber}, begann Hermine wieder. \enquote{Er hat sie geohrfeigt.}

\enquote{Ich habe es auch darauf angelegt, oder etwa nicht?}

\enquote{Sie wollten, dass er das tut?}

\enquote{Das, oder etwas anderes. \gst Albus, ich glaube, wenn Harry sich beruhigt hat, dann schulde ich Hermine, Ron und Harry eine Erklärung.}

Nachdem sich Harry beruhigt hatte, saß er wieder auf einem Stuhl in Dumbledores Büro. Er hatte sich von seinem Zusammenbruch erholt.

Professor Elber fuhr mit seiner Erzählung fort. \enquote{Es ist wichtig, dass Sie das verstehen, Harry. Und es tut mir leid, dass Sie das erfahren müssen. Ich habe zwei Theorien, die ein Grund für ihren Zustand seien können. Zumindest bei der zweiten stimmt Albus mir zu.}

Professor Elber setzte sich nun auf den Stuhl gegenüber Harry. Ron und Hermine standen hinter Harry und legten jeweils eine Hand auf seine Schulter. Harry hielt sie fest.

\enquote{Ich muss dazu sagen, dass es sich um bloße Vermutungen handelt. Und ich mache Ihnen keinen Vorwurf, wenn Sie, nachdem ich fertig erzählt habe, sofort den Raum verlassen, weil sie wütend sind.}

Harry wurde leicht unwohl. Wenn schon sein Professor mit so einem Ausbruch rechnet, wie würde er dann wirklich reagieren?

\enquote{Meine erste Theorie ist, dass Sie ohne eine Spur von magischem Können auf die Welt gekommen sind. Sozusagen als Squib. Erst durch den missglückten Mord an Ihnen, bei dem ja ein Seelenteil Voldemorts unbewusst abgetrennt worden ist und das jetzt in Ihnen verweilt, sind Sie zu einem Zauberer geworden. Sie haben bisher auf die Magie in diesem Seelenteil zugegriffen, als sie zauberten. So langsam aber fängt ihr Körper an, diese Magie in sich aufzunehmen. Ein ganz normaler natürlicher Prozess. Das Seelenteil stirbt dabei ab, obwohl es sich verzweifelt versucht dagegen zu wehren. Deswegen auch die Ohnmachtsanfälle, welche in nächster Zeit noch zunehmen werden.}

Harry wurde ganz bleich im Gesicht. Hermine hielt sich die Hand vor den Mund und zog dabei ihre Hand von Harrys Schulter. Schuldbewusst legte sie sie gleich wieder zurück, um Harry Kraft zu geben, als sie das bemerkte. Ron keuchte nur.

\enquote{Die zweite Theorie, bei der Albus mit mir übereinstimmt, ist, dass Sie als ganz normaler Zauberer auf die Welt gekommen sind. Aber durch den versuchten Mord und dem besagten Seelenteil in Ihnen, haben Sie sich dessen Magie zunutze gemacht, da das Seelenteil Ihre Magie unterdrückt hatte. Erst langsam, jetzt da Ihre eigene Magie stärker wird, da sie sich durch das Unterdrücken des Seelenteiles in ihnen wehren kann, bekämpft es die fremde Magie und zieht sie zu sich herüber.}

Harry warf kurze Blicke zu seinem Schulleiter und sah dessen besorgte Blicke.

\enquote{In beiden Fällen wandert die Magie, egal welche, in Ihren Körper und verbindet sich auf natürliche Art und Weise mit Ihren Zellen.}

Harry würde am liebsten weinen. Wortlos stand er auf und verließ Dumbledores Büro. Er schloss die Tür hinter sich und setzte sich nun gegenüber dem Wasserspeier an die Wand auf den Boden, als er die Wendeltreppe herunter getreten war.

\enquote{Probleme?}, fragte ihn dieser.

Harry erschrak. Es war das erste Mal, dass ihn der Wasserspeier etwas fragte. Sonst kam von ihm nur das monotone \enquote{Passwort}.

Harry hörte leise Stimmen in seinem Kopf. Oder hörte er sie wirklich? \enquote{Er wird gleich wieder kommen, Hermine. Er braucht nur kurz Ruhe, um sich zu sammeln.} Harry meinte, Dumbledore gehört zu haben. \gedanke{Ja}, dachte er, \gedanke{ich brauche einfach nur kurze Ruhe.} Aber er kam nicht so weit.

\enquote{Probleme?}, fragte ihn der Wasserspeier erneut. Harry gab ihm eine kurze Zusammenfassung und sah ihn danach an. Der Wasserspeier schaute kurz nach links und dann nach rechts und ging dann auf Harry zu.

\enquote{Ihr könnt euch bewegen?}, fragte Harry ganz ungläubig.

\enquote{Na na, Harry. Du kannst mich ruhig duzen. Du brauchst mit mir nicht so zu reden, wie man es mit adeligen macht.}

\enquote{Nein}, antwortete Harry. \enquote{Ich habe euch alle gemeint. Alle Figuren Hogwarts, auf denen ein Zauber liegt.}

\enquote{Ja}, antwortete der Wasserspeier, \enquote{wir können reden. Aber wir sind nicht einzelne Geschöpfe, auf denen ein Zauber liegt. Wir sind eins. Wir sind das Schloss, seine Seele. Wir sind Hogwarts, bekommen seit Jahrhunderten die Sorgen und Probleme, aber auch die Freude und die Liebe, sowie den Hass innerhalb dieser Mauern mit. Das hat uns geprägt. Lass mich dir etwas von dem, was deine Mitschüler und du uns im Laufe der Jahrhunderte gegeben habt, zurückgeben.} Der Wasserspeier legte seine Hand auf Harrys Stirn. Sofort wurde ihm warm und sein Zorn verflog zunehmend. Als der Wasserspeier Stimmen hörte, ging er wieder an seinen Platz und erstarrte. Noch einmal bewegte er seinen Kopf und sah zu Harry. \enquote{Kein Wort zu irgendjemand.} Dann sagte der Speier nichts mehr.

Harry stand auf und ging zurück in Dumbledores Büro.

\enquote{Harry, du solltest noch etwas wissen}, fing Dumbledore an. \enquote{Du wirst in nächster Zeit noch öfters solche Ohnmachtsanfälle haben. Immer, wenn das passiert, ziehst du von Voldemorts Seelenteil etwas ab und schwächst es somit bis es endgültig stirbt.} Harry nickte verstehend und sah Dumbledore an, der mit den Fingerspitzen gegeneinander gelehnt in seinem Stuhl saß. \enquote{Aber \gst immer, wenn du nach deinem Anfall wieder aufwachst, musst du dich beherrschen. Du darfst dich nicht aufregen, oder andere starke Gefühle haben. Sonst könntest du unbewusst, während die Magie noch auf der Suche nach einem Platz in dir ist, Zauber bewirken, die du nicht kontrollieren kannst.}

\enquote{Ich sage meinen Kollegen Bescheid}, wandte Professor Elber ein.

Harry nickte und Dumbledore gab ihnen zu verstehen, sie dürfen gehen.

\gedanke{Dann bin ich ja genauso mächtig wie Voldemort vor seinem Sturz war. Oder im zweiten Fall noch mächtiger?}, dachte Harry nach.

\enquote{Du nimmst ihn mit Albus, habe ich recht?}, fragte Professor Elber Dumbledore.

\enquote{Es ist notwendig, ja}, antwortete er.

\enquote{Dann lass ihn den Horkrux spüren. Er hat die Fähigkeit zu spüren, ob ein Gegenstand ein Horkrux ist. Durch Voldemorts Seelenteil.}

Harry, Ron und Hermine schauten wie vom Donner gerührt zu Professor Elber und Dumbledore. Dieser nickte, schaute Harry kurz an und danach stumm auf die Tür. Harry verstand.

Auf dem Weg nach unten fragte er Professor Elber ganz leise: \enquote{Sie haben damals, als Hermine und ich Sie im Schloss gefunden hatten und kurz bevor Sie zusammen gebrochen sind, also bevor Sie für längere Zeit auf der Krankenstation verbracht haben, Ihren eigenen Horkrux zerstört und Ihre Seele repariert.} Elber nickte nur. \enquote{Sind Sie deshalb so alt geworden?}

Er schüttelte den Kopf. \enquote{Das habe ich erst später gemacht. Heute weiß ich, dass das eine Dummheit war. Ich wollte mich so zu sagen absichern.}

Sie verließen zu viert das Büro und machten sich auf den Weg zum Gemeinschaftsraum. Harry dreht sich auf dem Gang noch einmal um und sah, wie Professor Elber vor dem Wasserspeier stand und eine Hand von seiner Schulter nahm. Er hatte den Eindruck, er nicke. Dann drehte er sich um und ging in die entgegengesetzte Richtung. Er drehte sich dabei nicht noch einmal um.

Helena schwebte um die Ecke und Harry hörte nur etwas, was er als \accentuate{Gramps} deutete. Dann ein \enquote{Shhh!} Dann waren die beiden zu weit entfernt. Vermutlich hatte sein Gehirn nur diese Worte gebildet und in Wahrheit bedeuteten sie etwas ganz anderes.

\trenn

Harry dachte nach. \gedanke{Alle Zauber und Flüche sind dort versammelt. \gst In diesem Buch sind sämtliche Sprüche aufgelistet. \gst Grünes Index-Buch \gst Die Magie ist mein Verbündeter.} Er blieb mitten auf den Weg stehen und hing seinen Gedanken hinterher.

\enquote{So nachdenklich?}, hörte er hinter sich.

Er musste schmunzeln, denn Professor Sinistra kam den Gang entlang.

\enquote{Ja Professor.} Dann fiel ihm etwas ein. \enquote{Kann ich Sie etwas fragen?}

\enquote{Wenn’s nichts Unanständiges ist.}

\enquote{Ich habe ein Bild eines Mondes. Kann man die Phase bestimmen? Ich meine, den Tag eines Mondzyklus?}

\enquote{Sicher, wenn die Darstellung des Mondes gut genug ist.}

\enquote{Jetzt gleich?}

\enquote{Kommen Sie in mein Büro.}

Harry folgte seiner Professorin, die gerade dabei war, zu ihrem Büro hoch oben in einem der Türme zu gelangen.

\enquote{Hat Ihnen Professor Dumbledore nicht gesagt, wie Sie schneller zu Ihrem Büro kommen?}, fragte sie Harry, als er merkte, dass sie die Treppen ansteuerte.

Professor Sinistra blieb stehen und wurde leicht rot. \enquote{Woher\abs wollen Sie wissen, ob\abs}

Er lächelte sie an, da er sie ertappt hatte. Sie wusste es, sollte es aber wohl geheim halten. Er ging zu einem der üblichen Plätze und drückte den passenden Stein.

Als sich die Wand teilt und er vor seiner Professorin die kleine Kabine betrat, fragte sie ihn: \enquote{Woher wissen Sie davon, Mister Potter?}

\enquote{Ich war dabei, als es Professor Dumbledore erfahren hatte. Ich war zu einer Ihrer Stunden zu spät dran, als Ihnen Professor Elber Dokumente überreicht und sie\abs Äh, das möchte ich jetzt nicht wiederholen.}

\enquote{Als er heftig mit mir geflirtet hat, meinen Sie?} Harry nickte. \enquote{Dann haben Sie das an Weihnachten gar nicht mitbekommen?}

\enquote{Da bin ich fast hinter Ihnen gestanden. Ich hatte fast den Eindruck, er schläft\abs} Harry brach ab und wurde rot.

Professor Sinistra lachte ihn herzlich an. \enquote{Probleme damit haben, dass er mich Schätzchen nennt, aber mir eiskalt ins Gesicht sagen, ich hätte mit ihm geschlafen}, meinte sie und lachte.

\enquote{So war das nicht gemeint}, wehrte sich Harry.

\enquote{Schon klar}, winkte sie ab. \enquote{Also, was ist jetzt mit dem Mond?}

Harry zog das Pergament aus seiner Tasche und zeigte ihr die Zeichnung.

Sie sah es sich an und dachte nach. Dann zog sie ein Stück Pergament heraus und legte es daneben. \enquote{Setzen Sie sich ruhig neben mich. Dann lernen Sie gleich was. Das nehmen wir nämlich nicht im Unterricht durch. \gst Sehen Sie es als mündliche Note an, falls Sie auf der Kippe stehen sollten und ich Sie danach fragen werde.}

Harry nahm einen Stuhl und setzte sich neben seine Professorin.

Jeden Schritt erklärte sie ihm. Harry fragte ab und an nach, was genau sie machte, verstand aber recht schnell den Gedanken. Aufgrund der Abbildung konnte man um den einundzwanzigsten Tag schätzen. Nach einer kurzen Messung mit dem Lineal und einem Vergleich von Beobachtungskarten und der Zeichnung kam der zweiundzwanzigste heraus.

\enquote{Wofür brauchen Sie das überhaupt?}, fragte sie.

\enquote{Ich bin einem Rätsel auf der Spur.}

\enquote{Lassen Sie mich daran teilhaben?}

\enquote{Mondbibliothek.}

\enquote{Sie jagen dieser Legende nach?}

\enquote{Bisher habe ich keine Hinweise darüber gefunden, dass es eine Legende ist. Ich habe nur spärliche Hinweise gefunden. Zumeist Ideen, wen ich fragen könnte, denn die letzten Hinweise haben mich auf eine Spur gebracht.} Er breitete das Buch aus und zeigte ihr den entsprechenden Abschnitt. \enquote{Das mit den Zaubersprüchen habe ich schon mal gehört. Es gibt ein Buch, in dem alle Zaubersprüche mit Namen und kurzer Erklärung aufgeführt sind. Ich will deshalb mit Professor Elber sprechen. Vielleicht hat er einen Hinweis mehr. Außerdem hat er was über die Magie gesagt. Sie sei mein Verbündeter. Allgegenwärtig ist sie.}

Professor Sinistra dachte eine Weile nach. \enquote{Das mit der Magie hat er uns auch immer wieder gesagt. Und jetzt, wo Sie es erwähnen, dieses grüne Buch, das er da hatte\abs vielleicht weiß er etwas. Zumindest ist es den Versuch wert. Lassen Sie es mich wissen, wenn es etwas Nennenswertes gibt?}

Harry nickte. \enquote{Ich werde vielleicht auf Ihre Hilfe eingehen. Wenn ich den Namen wörtlich nehme, könnte sich diese auf dem Mond befinden.}

\enquote{Aber wie wollen sie dorthin gelangen?}

Harry hob die Schultern. \enquote{Apparieren?}, fragte er leicht ungläubig.

\enquote{Über vierhunterttausend Kilometer? Wohl kaum.}

\enquote{Genau das stört mich daran.}

\trenn

\enquote{Guten Morgen, Luna}, kam es ihm entgegen.

Verschlafen drehte er sich in Richtung der Stimme, öffnete seine Augen und sagte dann: \enquote{Guten Morgen Elisabeth.} Er drehte sich wieder auf die Seite, als ihm plötzlich klar wurde, dass er nicht mehr in seinem Bett lag. Erschrocken setze er sich im Bett auf und schaute sich um.

\enquote{Was ist, Luna?}, fragte ihn Klara, Lunas zweite Zimmergenossin.

Er sah sich um und bemerkte die blauen Bettdecken und Kopfkissen auf den weißen Bettlaken. In den Mädchenschlafsälen der Ravenclaws waren keine Himmelbetten wie in Gryffindor. Er sah Elisabeth wieder in die Augen. Sie hatte lange schwarze, leicht gewellte, schulterlange Haare und grüne Augen. Sie hatte dieselbe Augenfarbe wie Harry. Er stand auf und meinte: \enquote{Danke Elisabeth, aber ich bin Harry.}

\enquote{Du willst uns veralbern}, sagte Klara.

\enquote{Nein Klara}, antwortete Harry.

Jetzt grinste sie. \enquote{Ich glaube dir kein Wort, Luna, Harry kennt meinen Namen nicht.}

\enquote{Doch}, sagte Harry leicht verärgert. \enquote{Luna beschrieb euch mir recht gut}, log er. Er kannte sie, da er sie bereits durch Lunas Augen gesehen hatte und Luna ihm mitteilte, mit wem sie im selben Zimmer schlief. Er stand auf und sah Elisabeth an.

Sie war sehr hübsch. Sie kam auf ihn einen Schritt zu und legte ihre Hände um ihn. \enquote{Also Harry}, sagte sie und kam ihm näher. \enquote{Was machen wir jetzt?}, fragte Elisabeth.

Harry konnte die anderen Mädchen leise kichern hören. Er legte seine Hände auf ihren Oberkörper und hielt sie auf Abstand. \enquote{Lass es, Elisabeth. Ich bin im Körper eines Mädchens und habe daher kein Interesse, eine Beziehung anzufangen.}

Das Gekicher der Mädchen wurde nun lauter.

Elisabeth sah ihn eigenartig an. \enquote{Magst du mich, Harry?}, fragte sie ihn.

\enquote{Natürlich}, gab er zurück. \enquote{Aber ich fühle mich etwas bedrängt}, sagte Harry. \enquote{Du machst mich verlegen.}

Jetzt lächelte ihn Elisabeth an. Ihre Hände immer noch an seiner Seite, seit er sie auf Abstand hielt. Irgendetwas kam ihm komisch vor, als sie so dastanden, doch er wusste nicht was. Seine Hände fingen leichte massierende Bewegungen an. Dann stoppte er abrupt und es dauerte etwas, bis die Nervensignale von seinen Fingern im Gehirn ankamen und dort ausgewertet wurden \gst bis er bemerkte, wo er seine Hände hatte. Sie lagen die ganze Zeit über auf Elisabeths Brüsten. Er versuchte keine Miene zu verziehen, um dieses Gefühl noch etwas länger zu erhalten. Doch er konnte nicht. Er zog seine Hände schnell zurück und krümmte seine Finger ein.

Elisabeth drehte sich zu Klara und den anderen und meinte dann. \enquote{Lasst uns alleine.} Kichernd verließen sie den Raum und schlossen die Tür hinter sich. Elisabeth lief langsam auf Harry zu. Er zog seine Hände weiter zurück, denn er wollte nicht wieder ihre Brüste berühren. Als sie kurz vor ihm stand, dicht an ihm, konnte er seine Hände nicht mehr herunternehmen, da sie sich dicht an ihn gepresst hatte. Und so berührten seine Hände wieder ihre Brüste durch den knappen Stoff.

\enquote{Harry}, flüsterte sie in sein Ohr. \enquote{Wenn du mir einen kleinen Gefallen tust, dann erzähle ich niemandem, wie du meine Brüste berührt hast und es dir gefiel.}

\enquote{Ich habe nicht}, protestiert Harry, \enquote{nicht bewusst}, fügte er hinzu. \enquote{Na ja.}

Elisabeth lachte. Ihre Stimme ließ ihn ein wohliges Gefühl durchleben. Sie entfernte sich etwas von ihm, sodass er seine Hände herunternehmen konnte. Dann presste sie sich wieder an ihn. Er wusste mit seinen Händen nichts weiter anzufangen, also legte er sie ungezwungen um ihre Taille. Sie gab ein leises Schnurren von sich, kurz aber durchdringend. Er wollte sich schon zurückziehen, da sie nicht merken sollte wie sehr es ihn erregte, aber da regte sich nichts. Er war schließlich in Lunas Körper, fühlte nur etwas Angenehmes in seinem Unterkörper und wie sein Schritt leicht feucht wurde. Es war angenehm, als ihre Brüste gegen den Körper drückten, welchen er gerade bewohnte.

Wieder flüsterte sie in sein Ohr. \enquote{Harry, wenn du wieder in deinem Körper bist \gst und ich merke das, falls du wieder den Körper wechseln solltest \gst kommst du zu mir. Ich wollte dich schon lange einmal küssen.}

Er erschrak und versuchte sich aus ihrem Griff zu lösen. Eindringlich sah er in ihre Augen. \enquote{Das ist nicht dein Ernst}, meinte er.

\enquote{Soll ich den anderen erzählen, wie du es genossen hast, meinen Körper zu berühren?}

\enquote{Nein}, stammelte Harry. Und dann, nach einer kleinen Pause meinte er: \enquote{Das ist Erpressung. Also gut.} Er musste ein leichtes Grinsen überspielen, da er den Gedanken, Elisabeth zu küssen, nicht gerade abstoßend fand.

Sie löste sich von ihm und lief zurück zu ihrem Bett. Da sie auch noch ihren Schlafanzug trug, begann sie sich umzuziehen. Harry tat es ihr gleich, nachdem er sich in Lunas Zimmer umgesehen hatte. Immer wieder blickte er zu Elisabeth herüber. Irgendwann erblickte er sie von der Seite. Sie hatte nichts an. Ihre Haut war glatt und vereinzelte Leberflecken zeichneten sich ab. Auf ihrem rechten Oberschenkel konnte er eine klare Form erkennen. Es schien, also ob es wie ein kleines Herz aussah. Er ließ seinen Blick nach oben schweifen und entdeckte ihre Brüste. Sie sahen so aus, wie sie sich anfühlten. Für ihr Alter hatte sie eine beachtliche Oberweite. Als sie zu ihm blickte, warf er seinen Kopf zurück und machte sich daran, sich ebenfalls anzukleiden. Er dachte sich nichts dabei, dass er sich offen umkleidete. Sie war ebenso ein Mädchen wie er.

\gedanke{Wie ich}, dachte er.

\enquote{Gefällt dir mein Körper?}, fragte ihn Elisabeth. Ihm kam wieder Luna in den Sinn, die ihm dieselbe Frage auf der Krankenstation gestellt hatte.

Harry entschloss sich die Wahrheit zu sagen. \enquote{Ja.}

Elisabeth warf ihren Kopf herum und sah Harry erschrocken an. \enquote{Ja?}, antwortete sie leicht unsicher.

\enquote{Ja}, wiederholte Harry und lachte. \enquote{Du hattest wohl ein Nein erwartet.}

Plötzlich wurde ihm wieder schlecht. Er setze sich auf das Bett. Gerade noch rechtzeitig. Er spürte wie ihm schwarz wurde und er auf das Bett zurückfiel.

\enquote{Luna? Alles in Ordnung?}, hörte er, als er wieder seine Augen öffnete. Er war im Gemeinschaftsraum. In seinem Gemeinschaftsraum. Ron stand über ihm und fragte ihn erneut: \enquote{Luna? Alles in Ordnung?}

\enquote{Nein, Harry}, gab er zurück.

\enquote{Oh Mann}, sagte Ron und reichte ihm eine Hand, um ihn zu sich zu ziehen.

Nachdem er aufgestanden war, meinte Harry: \enquote{Luna ist wohl auch zusammengebrochen.}

\enquote{Ja Mann. Was meinst du wie sie mich beim Umziehen angesehen hat? Erst als ich bemerkte, wer sie war\abs}

Harry grinste. \enquote{Sie hat wohl zu viel gesehen}, meinte Harry.

\enquote{Für meinen Geschmack zu viel}, gab Ron zurück.

Er hatte in letzter Zeit öfters mit Luna gewechselt. Einmal sogar während des Unterrichts. Aber zum Glück bekam es keiner mit, da es nur mehrere Minuten gedauert hatte. Harry musste an Elisabeth denken. \gedanke{Und ich merke das, falls du wieder den Körper wechseln solltest}. Das könnte lustig werden.

\trenn

Gemeinsam mit den anderen ging er wieder den Schotterweg hinunter nach Hogsmeade. Sie hatten wieder eine Apparierstunde. Er ließ sich mit Elisabeth etwas zurückfallen und zog sie, nachdem sie in sicherer Distanz zu den anderen waren, in die Büsche. Es war Freitagnachmittag und passenderweise ein Hogsmeade-Wochenende. Sanft drückte er sie an den Baum und kam ihr näher. \enquote{Du weißt, ich löse meine Schulden bald ein. Und wenn, dann mache ich es richtig.} Dann verließ ihn der Mut. Er blieb stehen und sah sie nur an.

Dann passierte etwas, was er nicht erwartet hatte. Sie legte ihre Hände um ihn und drehte sich mit ihm, sodass er mit dem Rücken zum Baum stand. Dann drückte sie ihn gegen den Baum und trat an ihn heran. Er spürte ihre Brüste auf seinem Oberkörper. Automatisch legte er seine Hände um ihre Taille. Sie kam ihm näher, bis sie kurz vor seinem Mund aufhörte. Er konnte ihre Lippen auf seinen spüren, als sie sprach \enquote{Harry, erfülle deine Schulden. Jetzt.}

Also öffnete er seinen Mund und näherte sich ihr. Dann drang er in sie ein. Sobald er ihre Lippen ganz berührt hatte, konnte er sich nicht mehr zurückhalten. Er küsste sie, als gäbe es keinen Morgen mehr. Dann ließ er wieder von ihr ab und sah ihr ins Gesicht.

Sie schaute ihn mit erröteten Wangen sprachlos an. \enquote{Harry}, stammelte sie. \enquote{Was war das?}

\enquote{Ein Kuss, so wie du es wolltest}, sagte er.

\enquote{Nein}, antwortete sie. \enquote{Das war kein Kuss. Das war mehr, war pures Verlangen.} Jetzt übernahm sie die Führung und küsste ihn zurück. Dann löste sie sich von ihm und meinte: \enquote{Schuld mehr als nur eingelöst. Ich glaube, ich schulde dir nun etwas. Lass uns zu den anderen gehen. Du bist spät dran.}

\gedanke{Ein Verlangen liegt jetzt in ihren Augen}, dachte Harry, \gedanke{wie ich es noch nie gesehen habe.}

Sie nahm ihn an der Hand und zog ihn zurück auf den Schotter. Den Rest des Weges liefen sie, ohne ein Wort zu sagen nebeneinander her.

Vor der Stadthalle angekommen, legte sie einen Arm auf seine Schulter und nahm den Türgriff in die Hand. Dann sah sie ihn an und sagte zu ihm: \enquote{Ich habe mir den Fuß gezerrt und du hast mir hier her geholfen.} Dann öffnete sie die Tür und humpelte mit Harry neben sich herein. Er hielt sie in der Zwischenzeit an ihrer Taille und betrat mit ihr den Raum.

\enquote{Entschuldigung für die Verspätung. Sie hat sich den Fuß etwas gezerrt und ich wollte sie nicht den weiten Weg ins Schloss bringen}, sagte er.

\enquote{Harry half mir und stützte mich auf dem Weg hierher}, meinte Elisabeth.

Professor Elber zog seinen Zauberstab und zeigte damit auf Elisabeths Knöchel. Um ihn herum bildeten sich kleine grüne Funken, die ins Innere des Knöchels sickerten. Harry fiel auf, dass Professor Elber seinen Blick zwischen Elisabeths Knöchel und ihrem Gesicht hin-und-her wechselte. Danach schaute er sie an. \enquote{Sie sollten nun wieder normal laufen können}, sagte Professor Elber. So als ob nichts wäre, führte er seinen Unterricht fort.

\enquote{Er weiß es}, flüsterte Harry Elisabeth zu. \enquote{Ich seh' es an seinem Blick.}

\enquote{Was?}, keuchte Elisabeth leise.

\enquote{Dass dein Knöchel nicht verstaucht war. Geh am besten wieder zurück. Und sprich ihn nicht deswegen an.}

Am Nachmittag dann war das große Quidditch-Spiel gegen Slytherin. Heute könnte er eine gute Gelegenheit haben, seinen Sprung zu versuchen. Immer wieder hatte er ihn geübt und Dumbledore wusste Bescheid. Er hatte ihn einmal davon abgehalten zu sehr verletzt zu werden, als ihn Dementoren von seinem Besen holten. Harry hoffte, dass er heute nicht eingreifen müsste. Er hatte mit seiner Mannschaft ein paar Spielzüge auf dem Mini-Quidditch-Brett von Arabella ausprobiert. Die Sicht von außen auf das eigene Spielgeschehen war etwas vollkommen Neues. Aber es war sehr hilfreich. Um seine eigene Mannschaft vorzuwarnen, sagte er ihnen, was er geübt hatte, damit sie sich keine Sorgen machen mussten.

Schließlich standen alle Mannschaftsmitglieder bereit. Auf einen Pfiff von Madame Hooch hin bestiegen sie ihre Besen und hoben ab. Sie umkreisten einmal das Spielfeld, um sich kurz zu akklimatisieren und dann in Formation in der Luft stehenzubleiben. Die Kapitäne der Mannschaft schwebten bis auf Bodennähe und warteten, bis Madame Hooch den Quaffel in die Luft warf.

Der Kapitän der Slytherin-Mannschaft schaffte es, den Ball zu erwischen und warf ihn seiner Mannschaft zu, währenddessen kreisten Harry und Draco über dem Spielfeld. Einige Male standen sie in der Luft dicht beieinander und suchten das Feld ab. Sie nickten sich knapp zu und umrundeten das Feld wieder. Wie schon so oft wunderte sich Harry, dass Slytherin dieses Jahr extrem gut war. Doch er konnte sich nicht auf seinen Gedanken ausruhen. Er erspähte den Schnatz und flog auf ihn zu. Doch viel zu schnell war er auch schon wieder seinem Blick entflohen. Also stieg Harry mit seinem Besen wieder in die Lüfte.

Er versuchte sich von allem anderen unnützen zu befreien und den Schnatz magisch zu erfassen. Doch so einfach es auch in seinen Übungen war; auf dem Spielfeld mit einem kleinen fliegenden Ball, vielen schreienden und tobenden Zuschauern und den Kommandos der Mannschaften, sowie der Durchsage von Lee Jordan; machte es ihm dies nicht gerade leicht, den Schnatz zu fühlen. Seine Konzentration hätte ihm fast zu einer Begegnung mit einem Klatscher verholfen. Er konnte gerade noch ausweichen und musste dabei feststellen, dass Draco anscheinend den Schnatz gesehen hatte. Er hatte nur noch eine Chance. Er musste springen.

Sein Besen stellte sich senkrecht und schob ihn kurz an. Im freien Fall und mit dem Besen hinterher raste er auf den Schnatz zu. Das Publikum hielt den Atem an. Es wurde richtiggehend still. Wenn man einmal von den Kommandos der Mannschaften absah, die nicht wirklich realisierten, was gerade um sie herum passierte. Harry fiel weiter auf den Schnatz zu und wurde von seinem Besen überholt. Da er keine Masse zu beschleunigen hatte, war er schneller unterwegs. Harry fing den Schatz aus der Luft und wurde von seinem Besen, der ihn eine halbe Sekunde vorher überholt hatte, sanft abgefangen. Harry war gerade auf der richtigen Höhe, um mit Draco, der über ihm flog, nicht zusammenzutreffen.

Als Harry den Schnatz in seiner Hand hielt und nach oben in die Luft streckte, löste sich die Anspannung im Publikum und Jubel brach aus. Sie hatten Slytherin mit zehn Punkten Vorsprung geschlagen. Harry atmete noch immer etwas schwer und hörte gar nicht richtig hin, als seine Hauslehrerin mit mahnenden Worten auf ihn zukam.

\enquote{Wollen Sie einen Siegerkuss?}, fragte er sie, bevor ihn seine Mannschaft auf die Schultern hob und davon trug. Das Gesicht, das Professor McGonagall machte, bekam er gar nicht mehr mit. Nur Colins Foto hielt die Szene fest. Dafür würde er sicher noch einen Rüffel kassieren.

\trenn

\enquote{Meinst du, wir schaffen das?}, fragte Harry. \enquote{Ich muss in einer halben Stunde am Bahnhof sein.}

\enquote{Kein Problem, Harry.} Luna zog ihn mit um eine Ecke und in ein leeres Klassenzimmer. Sie verschloss die Tür und forderte endlich sein Versprechen ein. \enquote{Hose runter. Du schuldest mir noch was.}

Eine halbe Stunde später stand die Apparitionsklasse unten am Bahnhof und wartete auf den Zug, der sie nach London in das Ministerium bringen sollte, um ihre Prüfung abzuhalten. Plötzlich kam Professor McGonagall auf den Bahnsteig und ging voller Aufregung auf Professor Elber zu. \enquote{Frederick, es ist etwas passiert. Die Prüfung wird nicht stattfinden.}

\enquote{Was heißt das, nicht stattfinden?}

\enquote{Nun ja, es gibt keinen im Ministerium, der die Prüfung abnimmt. Ich habe es gerade erst erfahren. Das Ministerium ist scheinbar infiltriert worden.}

\enquote{Was?}, schrie Professor Elber. \enquote{Das kann nicht sein. Das Ministerium hat trotz allem die Verpflichtung die Prüfung abzunehmen. Egal wer jetzt die Abteilung leitet. Wir werden trotzdem dorthin fahren. Die Schüler haben ein Recht auf ihre Prüfung und das Ministerium hat die Pflicht diese abzunehmen.}

\enquote{Aber\gst}

\enquote{Kein Aber. Ich werde mich darum kümmern, sobald wir dort sind. Um die Sicherheit kümmere ich mich schon.} Der Zug kam an und Professor Elber zeigte seinen Schülern an, dass sie einsteigen sollen. \enquote{Minerva, steig ein}, sagte Professor Elber zu Professor McGonagall. Leicht irritiert stieg sie ein.

\gedanke{Ihr ist mulmig}, dachte Harry. \gedanke{Aber warum fügt sie sich dann? Sie ist doch sonst nicht so!}

Die Lokomotive setzte sich schwer schnaufend in Bewegung und fuhr Richtung London los. Harry hatte schon so oft diesen Zug benutzt, aber noch nie während des Schuljahres. Es war Freitag-Morgen und gegen Abend würden sie in London angekommen sein, dachte er. Die Fahrt dorthin war recht geruhsam, aber nicht langweilig.

Angekommen in London und durch die Absperrung von Gleis 9 3/4 hindurch, erregte die Gruppe bald eine gewisse Aufmerksamkeit. Professor Elber entging das nicht und er drehte sich plötzlich um und lief rückwärts. \enquote{Also Klasse}, sprach er in einem schottischen Akzent. \enquote{Zusammen bleiben. Ihr wisst, dass der Ausflug nach London ein Privileg ist. Wenn wir mit der U-Bahn nach Westminster fahren, bleibt bitte zusammen.}

\gedanke{Westminster}, dachte Harry. \gedanke{Dort ist das Zaubereiministerium.} Professor Elber drehte sich wieder um und lief nun wieder vorwärts. Man merkte, wie das Aufsehen, welches die Gruppe bis vor kurzem noch erregt hatte, abebbte. Die Muggel hielten sie für eine normale Schulklasse. Sie bewegten sich auf die U-Bahn-Station zu und verschwanden im Untergrund. Professor Elber zog vor den Drehkreuzen eine Karte heraus und ermöglichte durch das Entwerten dieser der Gruppe die Benutzung der U-Bahn.

Nachdem sie an der richtigen Station ausgestiegen waren, begann Professor Elber wieder mit seinem schottischen Akzent. \enquote{So, jetzt geht noch jeder von euch auf die Toilette und dann machen wir weiter.} Keiner widersprach. Professor Elber flüsterte Professor McGonagall noch etwas ins Ohr und führte die Jungs dann auf die Toilette. Keiner der umstehenden Muggel nahm wirklich Notiz von ihnen. Er erklärte auf dem Männerklo den Jungs, dass sie sich die Toilette herunterspülen müssten.

Im Ministerium angekommen war es still. Professor Elber ging voraus und führte die Gruppe durch viele Gänge und Treppen zu ihren Schlafsälen. Danach verriegelte er die große Tür und bereitete sich für die Nacht vor. Es war ein großer Raum, in dem viele Matratzen auf dem Boden lagen. Darauf jeweils ein Kissen und eine Decke. Harry wunderte sich, dass sich sein Professor hier so gut auskannte.




\begin{kommentar}
Katharina und Harry lesen in Hogwarts etwas über die Mondbibliothek. Ein Hinweis ist auch gleichzeitig eine Anspielung auf Monopoly, das Harry mal mit Tamara gespielt hatte. Ein anderer Hinweis auf die Bibliothek stammt von Indiana Jones und der letzte Kreuzzug. Und ein weiterer stammt von Krieg der Sterne.
\end{kommentar}

\chapter{Eine Prüfung die ist (nicht) lustig}


Am nächsten Morgen, als Harry erwachte, war bereits die Hälfte der Klasse wach und wusch sich, die Jungs und Mädchen jeweils hinter spanischen Wänden. Harry wusch sich ebenfalls und zog danach seine Sachen wieder an. Die Tür öffnete sich und Professor Elber kam herein. \enquote{Alle fertig?}, fragte er fröhlich in die Runde. Er lief auf Professor McGonagall zu und unterhielt sich mit ihr. Harry konnte die Unterhaltung nicht hören, war sich aber sicher, dass Professor McGonagall nicht gerade vor Begeisterung sprühte. Sie sah im Gegenteil eher erschrocken aus. Er drehte sich wieder in den Raum und sprach: \enquote{Wir werden uns jetzt in den Prüfungsraum begeben. Dieser befindet sich im untersten Stockwerk.}

\gedanke{Die Gerichtssäle}, schoss es Harry durch den Kopf.

\enquote{Wir gehen nun alle geschlossen dort hin}, sagte Professor Elber. Er öffnete die Tür und ging voraus. Professor McGonagall machte den Abschluss. Wieder ging es durch unzählige Gänge und Treppen bis ganz hinunter in die Gerichtssäle. Dort angekommen, gab Professor Elber der Klasse wortlos zu verstehen, sie sollen sich doch bitte setzen. Danach verließ er den Raum. Professor McGonagall stellte sich jetzt vor die Klasse und erläuterte mit zitternder Stimme den Ablauf der Prüfung. \enquote{Ich werde vor ihnen zum Ziel apparieren. Sie kennen es bereits. Es befindet sich 100~km außerhalb des Ministeriums. Wir sind schon einmal dort gewesen. Die Prüferin wird Ihren Namen aufrufen und Sie treten vor in die Mitte. Danach apparieren Sie dorthin. Die Prüferin wird dann von mir mitgeteilt bekommen, ob Sie dort angekommen sind.} Sie zeigte mit einer Liste in die Höhe. \enquote{Danach ist der nächste dran. Sind alle dort angekommen, geht es den Weg zurück. Alles verstanden?}

Alle nickten. Die Tür ging auf und Professor Elber kam herein, dicht gefolgt von Narcissa Malfoy. Harry sprang sofort auf und zog seinen Zauberstab, doch Professor Elber war schneller und entwaffnete ihn. \enquote{Was denken Sie sich eigentlich dabei, Ihre Prüferin anzugreifen, Harry?}, fragte Professor Elber. Harry blieb das Gesicht stehen. \enquote{Was glauben Sie, wie viel Überzeugung es mich gekostet hat, Narcissa dazu zu überreden?}

\gedanke{Narcissa?}, dachte Harry. Sie stand immer noch dicht hinter Professor Elber, um sich zu schützen. \enquote{Aber Ihr Mann arbeitet für Voldemort.}

\enquote{Mrs Malfoy ist hier in ihrer Eigenschaft als zugelassene Prüferin des Ministeriums. Dass sie, oder jemand aus ihrer Familie, eventuell mit dem Feind im Bunde ist, steht hier nicht zur Diskussion.}

\enquote{Aber}, wand Harry ein.

\enquote{Kein Aber. Haben Sie sich jetzt wieder beruhigt?}

Narcissa Malfoy trat nun vor und sprach in einer liebevollen Art und weiße. \enquote{Ich werde euch nacheinander aufrufen und ihr werdet hier aus der Mitte des Raumes apparieren. Nachdem alle angekommen sind, werdet ihr euch auf den Rückweg machen.} Sie verlas jetzt den ersten Namen.

Der Aufgerufene stand auf und ging sichtlich unwohl in die Mitte des Raumes. Professor McGonagall disapparierte und die Prüfung begann. Der erste Prüfling disapparierte und die Prüferin sah auf ihre Liste. Es erschien ein kleiner Haken, worauf sie den nächsten Namen auf der Liste aufrief. Zunehmend wurde die Klasse weniger. Jetzt war Harry an der Reihe, er stand auf und ging hinab. Professor Elber gab ihm seinen Zauberstab wieder. Er steckte ihm ein, nahm seinen Platz ein und disapparierte. Er tauchte auf der anderen Seite wieder auf. Professor McGonagall sah ihn von allen Seiten an und machte danach einen Haken hinter seinem Namen auf der Liste. Es folgten noch fünf weitere Personen. Danach waren sie wieder komplett.

Es dauerte eine Weile bis Professor Elber erschien. \enquote{Soweit war der erste Teil für sie alle erfolgreich. Nun geht es den Weg wieder zurück. Du rufst die Personen auf, Minerva?} Sie nickte. Professor Elber verschwand wieder und Professor McGonagall rief kurz darauf die erste Person auf.

Währenddessen fand in Hogwarts eine Unterhaltung über den Verbleib von Minerva McGonagall statt.

\enquote{Wo ist eigentlich Minerva?}, fragte Professor Vector. \enquote{Ich habe sie seit heute Morgen nicht mehr gesehen.}

\enquote{Sie ist mit Frederick im Ministerium die Apparitionsprüfung abnehmen}, antwortete Dumbledore.

\enquote{Aber}, protestierte Professor Vector, \enquote{das Ministerium ist doch infiltriert worden.}

\enquote{Ich weiß}, antwortete Dumbledore, \enquote{aber das scheint Frederick nicht zu stören. Ihr kennt ihn ja, wenn er von etwas überzeugt ist, dann lässt er sich nur schwer davon abbringen. Ich wette, er hat seinen Spaß dabei ein paar aus dem Ministerium an der Nase herumzuführen.}

\enquote{Wenn das mal gut geht}, gab Professor Vector als Antwort zurück. \enquote{Sind die Schüler denn sicher?}

\enquote{Du kennst doch Frederick, er wird sie schon heil zurückbringen}, grinste Dumbledore.

Narcissa setzte gerade ihren letzten Haken auf ihrem Brett und sah danach auf die Gruppe. \enquote{Herzlichen Glückwunsch. Ihr habt alle die Prüfung bestanden.} Alle atmeten erleichtert auf. Sie wollte gerade den Raum verlassen, als ihr Professor Elber nachlief und noch etwas mit ihr abklärte. Sie nickte und ging dann voraus. \enquote{Draco? Sie können mit Ihrer Mutter schon vorausgehen.} Malfoy schaute etwas ungläubig, aber ging ihr hinterher. \enquote{Zeit zum Mittagessen. Alle mir nach.} Professor Elber lief Richtung Ausgang und die Gruppe folgte ihm.

Harry schoss es immer noch durch den Kopf. Er konnte es nicht fassen, dass Professor Elber Narcissa Malfoy zu kennen schien. Aber andererseits war da dieser seltsame Traum, in dem er seinen Lehrer gesehen hatte. Automatisch lief er der Gruppe hinterher. Erst als er in der Kantine stand, kam er langsam zur Ruhe. Professor Elber drehte sich zur Gruppe und sprach. \enquote{Setzt euch irgendwo hin, tippt mit eurem Zauberstab den Zettel auf der Stelle an, wo das Essen steht, welches ihr zu euch nehmen wollt und wartet.} Er setzte sich auf einen freien Platz und tippte auf den Zettel. Kurze Zeit später hatte Professor Elber sein Essen auf dem Tisch stehen. Sorgfältig und misstrauisch sah er immer wieder durch den Raum. Harry setzte sich ihm gegenüber, suchte sein Essen aus und sah dann gedankenverloren durch seinen Lehrer hindurch.

\enquote{Professor?}, fragte Harry.

\enquote{Ja.}

\enquote{Darf ich Sie etwas fragen?}

\enquote{Sie dürfen mich alles Fragen, Harry. Aber Sie werden nicht immer eine Antwort bekommen.}

\enquote{Woher kennen Sie Narcissa Malfoy?}

Professor Elber schmunzelte leicht. \enquote{Über Lucius}, war alles, was er sagte.

Die ganze Rückfahrt über dachte Harry über das nach, was er im Ministerium erlebt hatte. Er saß im offenen Abteil des Zuges und betrachtete gedankenverloren Professor Elber, wie er sich mit Malfoy unterhielt. Dann erinnerte er sich wieder an seinen Traum. Er spielte mit Lucius ja Schach. Man war das peinlich, wie er sich im Ministerium aufgeführt hatte. Etwa nach der halben Strecke zurück zum Schloss wurde der Zug plötzlich langsamer und hielt auf offener Strecke an. Harry zog es aus seinen Gedanken. Er ging an das bereits offene Fenster. Hermine hatte es geöffnet und sah nach draußen. Der Lokführer, sowie der Schaffner standen am Kopf der Zugmaschine und schienen sich über etwas aufzuregen. Der Schaffner kam zurück und stieg in Harrys Wagon ein. Er kam auf Professor McGonagall und Professor Elber zu und fragte die beiden: \enquote{Sie sind doch Professoren in Hogwarts! Würden Sie sich bitte mal die Schienen vor unserem Zug anschauen?} Die beiden standen auf und folgten dem Schaffner. Harry beobachtete wie die Drei an den Zugkopf gingen. Nach einiger Zeit kam der Schaffner zurück und verkündete: \enquote{Verehrte Fahrtgäste. Wir haben gerade ein kleines Problem, das vermutlich einige Zeit in Anspruch nehmen wird. Sie können sich außerhalb des Zuges die Beine vertreten. Aber gehen Sie nicht zu weit weg. Bleiben Sie in Hörweite.} Harry verließ den Wagon, um sich die Schienen anzusehen. Hermine begleitete ihn. Vorne angekommen sah er etwas, das ihn erstaunte.

Die Brücke, über welche sie sonst immer gefahren waren, war weg. Es sah so aus, als ob sie weggesprengt worden war.

\enquote{Was meinen Sie?}, fragte der Zugführer Professor Elber. \enquote{Schon zu einer Erkenntnis gelangt?}

\enquote{Die Brücke ist weg}, sagte er.

\enquote{Das sehe ich auch. Was sollen wir jetzt machen?}

\enquote{Den Zug sichern und nachdenken. \gst Wie viele Meter braucht es, damit der Zug seine Höchstgeschwindigkeit erreicht?}

\enquote{Etwa fünfhundertsechzig Meter.}

\enquote{Dann setzen Sie bitte um diese Distanz zurück und geben ein paar Meter dazu.}

\enquote{Was haben Sie vor?}, fragte der Lokführer skeptisch.

\enquote{Wir werden springen.}

\enquote{Springen?}

\enquote{Ja! Wir werden mit dem Zug}, er machte eine ausladende Bewegung mit seiner Hand, \enquote{über die defekte Brücke springen. Mit ein wenig Hilfe durch die Magie natürlich.} Dann drehte er sich zu Professor McGonagall um. \enquote{Darf ich dir assistieren?}, fragte er.

Plötzlich ploppte es hinter ihm und eine Person in schwarzem Umhang und mit einer Maske vor dem Gesicht tauchte auf. Reflexartig drehte sich Elber um und schockte die Gestalt und legte danach mehrere Zauber auf die nähere Umgebung.

\enquote{Wofür sind die denn?}, fragte McGonagall.

\enquote{Damit hier keiner überraschend apparieren kann. Sollte es einer versuchen, wird er neben ihn hier}, er zeigte auf die Gestalt, \enquote{umgeleitet und sofort geklammert.}

McGonagall nickte. \enquote{Dann wollen wir mal. Das habe ich schon lange nicht mehr gemacht}, meinte sie mit kindlicher Freude. Sie ging auf die Lok zu und bestieg das Führerhaus. \enquote{Ich wäre dann so weit}, sagte sie.

Elber nickte und machte mit magisch verstärkter Stimme eine Durchsage. \enquote{Liebe Fahrgäste, wir werden in wenigen Augenblicken wieder Fahrt aufnehmen. Bitte begeben Sie sich alle zurück in den Zug. \gst Wir werden zunächst etwas zurücksetzen und danach einen Sprung durchführen, da die Brücke vor uns leider durch Todesser zerstört wurde. \gst Bitte bewahren Sie Ruhe und denken Sie positiv. Denken Sie daran, dass alles gut gehen wird, das wird dem Zauber helfen zu wirken.} Dann nickte er dem Lokführer zu. \enquote{Setzen Sie bitte zurück. Ich werde während Sie fahren, die letzten Vorbereitungen treffen.} Dann stieg er auf das Dach und legte einige Zauber auf die Wagons.

Der Zug fuhr langsam rückwärts und hielt an, als Elber den letzten Wagon bearbeitet hatte. McGonagall stieg ein paar Stufen hoch, damit sie Elber Bescheid geben konnte. Sie hielt ihren Daumen nach oben, doch Elber fuhr mit seiner flachen Hand vor seinem Kehlkopf hin und her. Da während der kurzen Rückwärtsfahrt des Zuges nur noch zwei weitere Todesser aufgetaucht waren und es ansonsten still war, hatte er ein ganz mieses Gefühl. Er versuchte mit einem Zauber zu erkennen, was hinter der Biegung war, die gleich nach der Brücke war. Ein inneres Gefühl sagte ihm, dass da noch etwas sei, er wusste aber nicht genau, was.

Er ging wieder den Zug entlang nach vorne und blieb in der Mitte stehen. So hatte er eine bessere Sicht nach vorne. Er rief McGonagall zu, dass sie beginnen können. Sie sollte aber die Strecke im Auge behalten, da sie eventuell noch einen weiteren \accentuate{anderen} Sprung durchführen müssten.

Dann nickte Elber, setzte sich auf den Wagon und McGonagall gab dem Zugführer Bescheid loszufahren.

Der Zug nahm Fahrt auf und beschleunigte immer weiter, bis er mit voller Geschwindigkeit auf die Brücke zuraste. Als die Schienen ein Stück nach unten ragten und dann abrupt aufhörten, sank der Zug noch etwas und schwebte dann ruhig in dieser Höhe, bis er auf dem intakten Gleis zurückkehrte, wo sich die Lok und danach die Wagons wieder etwas anhoben. Als der letzte Wagon wieder auf den Schienen stand, fuhr die Lok um die Kurve. Doch es wurde nicht besser, denn die Schienen schienen mit den Holzschwellern verschmolzen zu sein.

McGonagall reagierte wie immer mit den Reflexen einer Katze und brachte den Zug zum Leuchten, was für Elber das Zeichen war, den Sprung einzuleiten. Der Zug verschwand mitsamt den Insassen in einem Wirbel und tauchte stehend etwa hundert Meter hinter dem Bahnhof in Hogsmeade auf, denn die Schienen waren bis zum Bahnhof mit den Schwellern verschmolzen.

Als McGonagall, Elber und der Lokführer ausgestiegen waren, öffnete der Schaffner die Türen der Wagons und half den Fahrgästen, die aussteigen wollten, aus dem Wagon heraus. Die Schüler stiegen ebenfalls aus und warteten, bis McGonagall oder Elber zu ihnen kam.

\enquote{Was meinen Sie?}, fragte der Zugführer Professor Elber. \enquote{Schon zu einer Erkenntnis gelangt?}

\enquote{Nicht direkt}, antwortete Professor Elber. \enquote{Ich vermute, da ist ein Zauber schiefgegangen. Man wollte den Zug anhalten, indem man die Schienen verschwinden ließ, falls wir das Hindernis mit der Brücke überspringen könnten. Oder so etwas in der Art.}

Professor McGonagall kam den Schülern entgegen. \enquote{Wir kehren jetzt alle zurück nach Hogwarts.} Sie führte die Schüler zu den Pferdelosen Kutschen. Harry musste jedes Mal schmunzeln, wenn er daran dachte, dass Thestrale die Kutschen zogen. Sie kamen gerade rechtzeitig zum Abendessen in die Große Halle. Alle staunten, als sie die Gruppe einmarschieren sah.

\enquote{Alle bestanden}, verkündete Professor McGonagall stolz. Harry dachte, er hörte eine gewisse Bitterkeit in ihrer Stimme und führte das auf Mrs Malfoy zurück.

Als Harry am anderen Morgen Richtung Hogsmeade ging, um sich die Schienen anzuschauen, bemerkte er, wie bereits Professor Elber auf dem Gleis stand und sie betrachtete.

\enquote{Guten Morgen Professor}.

\enquote{Guten Morgen Harry. Schon eine Idee?}, fragte ihn Professor Elber.

Harry war erstaunt. \enquote{Nein}, gab er zurück. Er wunderte sich darüber, dass ihn Professor Elber fragte.

\enquote{Guten Morgen Harry \gst Professor}, hörte er hinter sich.

\enquote{Guten Morgen Luna}, antworteten beide gleichzeitig.

\enquote{Mir war danach, hier herzukommen}, sagte sie.

\enquote{Sie beide haben sich wohl aus dem Schloss geschlichen?}

\enquote{Na ja}, antworteten beide unisono. \enquote{Ein bisschen.}

\enquote{Was meinen Sie?}, fragte sie Professor Elber. \enquote{Wie bekommen wir das wieder hin?}

\enquote{Das könnte ein schief gegangener Verwandlungszauber gewesen sein}, antwortete Luna.

\enquote{Interessante Idee. Und wie kommen Sie darauf?}

\enquote{Schlangen. Die Schienen erinnern mich an Schlangen. \gst Man hat wohl versucht, sie in Schlangen zu verwandeln.}

\enquote{Und was}, fragte sie Professor Elber, \enquote{gedenken Sie dagegen zu tun?}

\enquote{Entweder ein Rückführungszauber, oder den Zauber wiederholen. Damit würde ein unvollständig ausgeführter Zauber zu Ende geführt werden.}

\enquote{Guter Einfall, Luna. Aber welcher Zauber wurde angewandt?}, fragte sie Professor Elber.

Harry antwortete. \enquote{Wenn es einer von Voldemorts Leuten war, dann muss es ein Zauber gewesen sein, der zwei große Giftschlangen hervorbringen sollte. Dann würde ich einen Rückführungszauber anwenden.}

\enquote{Netter Einfall. Wer möchte es versuchen?}

Harry und Luna sahen sich nur an.

\enquote{Ich versuche es}, antwortete Luna. Sie nahm ihren Zauberstab und richtete ihn auf die Schienen. Professor Elber kletterte auf den Bahnsteig zurück und wartete auf Luna. \zauber{Revelatio Serpentino.} Es passierte eine Weile lang nichts. Doch dann begann das Metall sich zu sammeln und wieder eine Schiene zu formen. Der Vorgang wurde immer schneller. Bis er sich bei einem Meter Schienen pro Sekunde einpendelte.

\enquote{Gute Arbeit Luna. Wirklich gute Arbeit. Sie verstehen Ihr Handwerk.}

\enquote{Danke Professor}, antwortete sie und wurde leicht rot.

\enquote{Ich denke, ich werde Sie auch in erweiterter Magie unterrichten. Sie sind ein schönes Paar.}

Harry insistierte sofort. \enquote{Wir sind kein Paar mehr.}

\enquote{Das meinte ich auch nicht}, antwortete Professor Elber. \enquote{Ich meinte, bezogen auf die Magie. Zusammen können Sie viel bewegen. Sie scheinen sich gut zu ergänzen.}

Nach ihrer Rückkehr nach Hogwarts, meldeten sich Harry und Luna gewohnheitsmäßig auf der Krankenstation, um ihre Routineuntersuchung über sich ergehen zu lassen. Dieses Mal jedoch meinte Madame Pomfrey: \enquote{Mit Ihnen ist alles in Ordnung. Ich kann keine Interferenzen mehr feststellen. Es scheint so, als ob Sie von Ihrer Körpertauscherei geheilt seien.} Das musste er sofort Ginny erzählen. Harry fühlte sich zu ihr hingezogen, seit er seine Beziehung mit Luna beendet hatte. Er hatte bisher aber noch nicht den Mut gefunden, ihr seine Gefühle zu offenbaren.

\trenn

Es war bereits eine Woche vergangen, seit Harrys Aus- und Zusammenbruch in Dumbledores Büro. Er hatte seinen Rückzugsort nicht mehr besucht, sich aber mit Ron und Hermine über die Aufzüge unterhalten. Zusammen hatten sie schneller als Harry alleine herausgefunden, wohin die Meisten der Knöpfe gingen. Nur die, welche mit einem Passwort versehen waren, konnten sie nicht anfahren. So unter anderem auch Dumbledores Büro. Alle standen bei der letzten Runde im Aufzug und Ron drückte den Knopf, der zu Dumbledores Büro führte. Natürlich leuchtete die Fläche auf und erlosch danach. In der Nähe hörten sie Dumbledore und McGonagall sich über das Essen unterhalten. Es war Essenszeit.

\enquote{Gehen wir Essen}, sagte Ron und verließ den kleinen Raum. Hermine hakte sich bei ihm unter und zusammen verließen sie den Aufzug.

\enquote{Ich komme gleich nach}, sagte Harry, als ihn Hermine und Ron fragend ansahen.

Sie nickten und gingen vor.

Harrys Herz klopfte. Dumbledore ging zum Essen, also konnte er versuchen, ob er immer noch Zugang hatte. Er drückte den Knopf und legte seine Hand auf die leuchtende Fläche. Diese leuchtete grün auf und der Boden fing an zu vibrieren. Harry war auf dem Weg. Er verließ den Aufzug, als die Wand sich vor ihm öffnete und er Fawkes entdeckte. Er saß nicht auf seiner Stange neben Dumbledores Schreibtisch, sondern auf dem Geländer neben der Wand. Freudig ging Harry auf Fawkes zu und streichelte ihn. Nach ein paar Minuten drehte er sich um und sah die Wand an. Ihm wurde leicht mulmig, da er keinen Unterschied sah. Er hatte Angst, den normalen Weg zu gehen. \gedanke{Also gut Harry, du schaffst das.} Er setzte sich auf den Boden, schloss die Augen und konzentrierte sich. Nach wenigen Minuten sah er deutlich vor seinem inneren Auge einen Stein. Er öffnete seine Augen, stand auf und drückte den passenden Stein. Die Wand öffnete sich wieder und Harry betrat den Aufzug. Als er wieder in einem Seitengang war der zur Großen Halle führte, ging er direkt in den Saal zum Essen.

Freudig grinste er Ron und Hermine an. \enquote{Was?}, fragte ihn Ron mit vollem Mund.

\enquote{Ich weiß, wie man in Dumbledores Büro kommt.}

\enquote{Klar}, antwortete Hermine. \enquote{Durch die Tür.}

\enquote{Nein, mit dem Aufzug.}

Ron verschluckte sich fast an seinem Essen. \enquote{Was?}

\enquote{Ich habe einmal, als ich auf diesem Balkon war, Fawkes getroffen. Er hat mich dazu gebracht, diesen Knopf zu drücken. Und als nichts passierte, musste ich es erneut versuchen. Fawkes hat es geschafft, dass ich Zugang zu seinem Büro erhielt. Ich habe es gerade eben nochmal versucht. Fawkes hat mich erwartet und mich freudig begrüßt.} Dann lud er Essen auf seinen Teller.

\enquote{Du meinst, du kommst jederzeit in sein Büro?}, fragte Ron nach.

\enquote{Scheint so. Aber ich werde es nicht missbrauchen. Ich werde diesen Zugang nur verwenden, wenn es absolut nicht anders geht.}

Nach dem Essen wurde er draußen erwartet. Er lief neben seinem Professor her Richtung verbotener Wald. Dort angekommen blieb Professor Elber stehen. Harry musste noch etwa zehn Meter weiter laufen. Er bemerkte nicht, dass Professor Elber einige Schritte rückwärts lief.

\enquote{Was lerne ich heute?}

\enquote{Heute lernen Sie apparieren.}

\enquote{Hatte ich nicht vor Kurzem eine Prüfung?}

\enquote{Ja. Apparieren Sie auf mich zu.}

Harry tat wie ihm geheißen.

\enquote{Uff. Das war aber anstrengender als sonst.}

\enquote{Haben Sie die Fünf-Punkte-Regel angewandt?}

\enquote{Ja. Aber wieso war es dieses Mal etwas schwerer?}

\enquote{Ich habe einen leichten Widerstand erzeugt. Machen Sie weiter. Apparieren Sie zurück.} Harry drehte sich um und begann sich zu konzentrieren. \enquote{Haben Sie in der Zeit, in der Sie unterrichtet wurden, nichts gelernt?}, fragte Elber ihn plötzlich.

Harry drehte sich wieder um und sah ihn an. \gedanke{Mist}, dachte er. \gedanke{Apparieren, nicht umdrehen.} Harry nickte und konzentrierte sich kurz. Dann apparierte er wieder zurück an seinen Anfangspunkt.

Als er wieder am Ausgangspunkt angekommen war, sah er seinen Lehrer ganz kurz neblig. \gedanke{Das geht ganz schön auf die Augen}, dachte er sich.

\enquote{Apparieren Sie jetzt knapp hinter mich.}

\enquote{Was ist, wenn ich sie treffe?}

\enquote{Ich passe schon auf und werde entsprechend ausweichen}, sagte er und grinste ihn an. \enquote{Los.}

Harry tauchte knapp hinter seinem Lehrer auf.

\enquote{In welche Richtung schauen Sie?}

\enquote{Richtung Schloss.}

\enquote{Also von mir weg?}

\enquote{Ja.}

\enquote{Dann wieder zurück und mit dem Gesicht zu mir auftauchen.}

Harry apparierte wieder zurück. Von Mal zu Mal ging es leichter. Das merkte Harry. Er schnaufte ein paar mal durch und apparierte erneut. Jetzt sah er Elber auf den Rücken.

\enquote{Geklappt?}, fragte er knapp.

\enquote{Ja}, antwortete Harry.

\enquote{Also. Noch einmal zurück und dann ist kurz Pause.}

Harry nickte, was sein Lehrer aber nicht sah. Dann apparierte er wieder an seinen Ausgangspunkt. Als er sich umdrehte, verschwand gerade sein Lehrer.

\enquote{Hinter Ihnen}, hörte er.

Harry schreckte zusammen und drehte sich um. Elber hielt einen Korb in der Hand. Er reichte ihn Harry, nachdem er ein Sandwich herausgeholt hatte.

\enquote{Apparieren ist anstrengend}, erklärte er ihm.

\enquote{Das merke ich.}

Nach einer kurzen Pause ging es wieder weiter.

\enquote{Nehmen Sie diesen Beutel mit. Ich möchte sehen, ob es auch mit Gepäck klappt. Halten Sie ihn vor sich. So, als würden Sie einen Wäschekorb tragen.} Er ging wieder von Harry weg und drehte sich dann wieder um. Jetzt stand er jedoch weiter von Harry entfernt. \enquote{Los geht’s.}

Als Harry vor seinem Lehrer auftauchte, war dieser nicht mehr da.

\enquote{Hinter Ihnen}, hörte er aus einigen Metern Entfernung.

Überrascht drehte er sich um und sah seinen Lehrer an seinem Platz stehen.

\enquote{Wie haben Sie das gemacht?}, fragte er ihn.

\enquote{Alles eine Frage des Timings. Passen Sie auf und sehen Sie mir genau zu.} Professor Elber apparierte einen Meter nach links. \enquote{Ist Ihnen etwas aufgefallen?}

\enquote{Nein.}

\enquote{Wie lange hat das gedauert?}

\enquote{Etwa dreihundert Millisekunden}, schätzte Harry.

\enquote{Also langsam genug um selbst zu apparieren?}, fragte ihn Professor Elber.

\enquote{Wenn man darauf gefasst ist und darauf wartet, dann könnte das klappen}, antwortete Harry.

\enquote{Also versuchen wir es. Passen Sie auf. Wenn ich verschwinde und hinter Ihnen auftauchen werden, mit Blick Richtung verbotener Wald, also dieselbe Richtung, in die Sie jetzt schauen, dann werden Sie an meine Stelle apparieren und sich dabei drehen, damit Sie mich anschauen können. Verstanden?}

Harry nickte und Professor Elber apparierte. Fast hätte Harry seinen Einsatz verpasst und so war Professor Elber fast schon hinter ihm, als Harry zu apparieren begann.

\enquote{Etwas langsam, aber für den ersten Versuch gut. Jetzt hinter mich.}

Harry apparierte hinter seinen Lehrer.

\enquote{Gut, laufen wir ein Stück. Den Korb nehmen wir mit.} Elber hielt seine Hand ausgestreckt und der Korb tauchte mit dem Henkel in seiner Hand auf.

\enquote{Was war das?}, fragte Harry.

\enquote{Es nennt sich Distanz- oder Fremd-Apparieren. Man kann Sachen und Dinge, auch Menschen, Fremd-Apparieren. Es funktioniert auf Distanz. Stand in einem sehr interessanten Buch.}

Nach etwa hundert Metern sagte Professor Elber: \enquote{Mir ist das Laufen zu mühselig.} Er packte Harrys Hand. \enquote{Nehmen Sie mich mit?}

\enquote{Was meinen Sie? Soll ich Sie tragen?}

\enquote{Nein, Sie sind noch jung. Das Apparieren strengt Sie nicht so an wie mich.}

\enquote{Aber innerhalb des Schulgeländes kann man nicht apparieren.}

\enquote{Nicht?}

\enquote{Nein, das hat Hermine mir mehrmals gesagt. Das steht auch in \buchtitel{Hogwarts \gst Eine Geschichte.}}

\enquote{So. Tja}, grübelte sein Professor, immer noch Harrys Hand haltend. \enquote{Dann fragte ich mich, was wir die letzte Stunde gemacht haben!}

\enquote{Wir, das heißt ich, bin appariert. Aber außerhalb. Deshalb sind wir doch zum verbotenen Wald gegangen.}

\enquote{Sind Sie sicher? Schon Ihr erster Appariervorgang war von außerhalb des Geländes nach knapp innerhalb. Und im Verlauf der Stunde sind Sie immer weiter auf das Gelände vorgedrungen.}

Harry blieb stehen und zwang so seinen Lehrer es auch zu tun, da er ihn immer noch an der Hand festhielt.

\enquote{Können Sie nicht aus dem Laufen apparieren? \gst Ziel. Wille. Bedacht. Präzision. Bequemlichkeit?}, fragte er Harry.

\enquote{Sie wollen mich doch verarschen.}

\enquote{Trauen Sie mir das zu? Habe ich das jemals getan? Schätzen Sie mich für so fies ein? Harry, ich will Ihnen was beibringen, damit Sie sich gegen Voldemort und seine Todesser wehren können. Glauben Sie ernsthaft, ich will Sie damit hereinlegen? \gst Jetzt machen Sie schon. Ziel. Wille. Bedacht. Präzision. Bequemlichkeit.}

Harry resignierte. Erst glaubte er nicht daran, aber er wollte es zumindest versuchen, als er sich daran erinnerte, dass er es glauben musste. Ein Augenzwinkern später tauchten sie vor den Toren von Hogwarts und direkt hinter Dumbledore auf.

Dieser drehte sich erstaunt um, als er einen \geraeusch{Plopp} hinter sich hörte.

\enquote{Wo kommen Sie denn her?}

\enquote{Ich komme mit Harry. Ihn müssen Sie fragen, wo wir gestartet sind.}

\enquote{Wir kommen vom Schulgelände. Vom Rand des verbotenen Waldes sind wir etwa ein Viertel des Wegs bis hierhergelaufen und dann appariert.}

\enquote{Wie appariert? Ich dachte nicht, dass jemand anders\abs}, doch Dumbledore unterbrach sich.

Professor Elber grinste, was Harry jedoch nicht sah. \enquote{Ich gehe dann mal und trage den Korb in die Küche.} Damit verabschiede er sich und lies Harry mit Dumbledore alleine.

\enquote{Dich hat doch Professor Elber mitgenommen!?}, fragte Dumbledore nach.

\enquote{Nein}, antwortete Harry ehrlich. \enquote{Er bestand darauf, dass ich appariere. Um ehrlich zu sein, hätte ich nicht gedacht, dass es klappt. Aber die ganze letzte Stunde bin ich immer wieder durch das Feld appariert. Zwar nur von knapp außerhalb bis knapp innerhalb. Aber dennoch. Das hat im Nachhinein richtig Spaß gemacht, Albus.}

\enquote{Du erstaunst mich immer wieder.}

\enquote{Ich mich auch. Ich meine, ich bin einfach appariert, ohne mir was zu denken. Und als ich dann hier her apparieren sollte, dachte ich, er will mich veralbern. Dann erfuhr ich jedoch, dass ich die ganze Stunde über durch das Feld appariert bin.}

\enquote{Nimmst du mich eine Runde mit?}, fragte Dumbledore.

Harry griff wortlos zu und tauchte mitten in Hogsmeade vor dem Honigtopf auf.

\enquote{Woher wusstest du, dass ich Brausebonbons wollte?}

\enquote{Ich habe einfach gut geraten. Und offenbar richtig.}

Dumbledore betrat den Laden und holte sich einen neuen Vorrat an Bonbons. Dann ließ er sich von Harry zurückbringen.

\enquote{Hier, Harry.} Er reichte ihm eine kleine Tüte Bonbons. \enquote{Fürs Mitnehmen und wieder bringen.}

\enquote{Danke, Albus.}

Dann ging Dumbledore und Harry brachte seine \accentuate{Beute} grinsend in sein Zimmer. Dort dachte er über das nach, was er bereits gelernt hatte. Er konnte es nicht leugnen, er war mächtig geworden. Doch wieso war ihm das bisher nicht aufgefallen? Er spürte eine Art inneren Drang, nicht weiter darüber nachzudenken. Zuerst wollte er dagegen ankämpfen, aber eine andere Stimme sagte ihm, dass das nicht notwendig war. Ihn beschlich das Gefühl, dass die Magie selbst etwas dagegen hatte. Diese Ahnung lenkte ihn so ab, dass sein Widerstand bröckelte und er verstand, dass er niemandem etwas darüber erzählen konnte.

\trenn

Auf dem Weg zur Großen Halle bemerkte Harry eine Menge Schüler, die vor dem schwarzen Brett standen und sich heftig zu unterhalten schienen. Er kam näher und musste sich strecken, damit er auf den neuen Zettel dort sehen konnte.

\begin{brief}
Am Sonntag, dem 23.05, findet zum ersten Mal ein \enquote{Magie in Konzert} statt. Alle Schüler von Hogwarts werden dazu eingeladen und werden gebeten, festliche Zauberer- oder Hexenkleidung bzw. Schulroben zu tragen. Nähere Informationen am 23. zur Frühstückszeit.
\end{brief}

Harry wunderte sich. \enquote{Was soll das denn?}, fragte Ron.

\enquote{Ich nehme mal an, ein Konzert}, sagte Hermine.

\enquote{Du meinst, wegen des Namens?}, fragte Ron weiter.

\enquote{Ja, was sollte es auch sonst sein. Aber ich frage mich, welche Bands da spielen!}, meinte Hermine.

\enquote{Welche Bands?}, fragte Ron. \enquote{Was meinst du mit: \enquote{Welche Bands?} Das werden die \accentuate{Schwestern des Schicksals} sein, oder die \accentuate{Todesfeen}, so was in der Art.}

Hermine verschränkte ihre Arme \enquote{Und warum schreiben die das dann nicht hin?}, fragte sie. \enquote{Wenn laut deiner Meinung so wichtige und bekannte Bands da sind, warum schreiben sie es dann nicht hin?}

Harry machte sich auf den Weg in die Große Halle und setzte sich neben Ginny. Mit glänzenden Augen schaute sie ihn an. Er erwiderte ihre Blicke und drückte sanft unter dem Tisch ihre Hand, bevor er sie über den Tisch hob, um seinen Teller zu beladen. Es dauerte noch eine Weile, bis Ron und Hermine hereinkamen und sich auch an den Tisch setzten.

\enquote{Warum bist du gegangen?}, wollte Ron wissen.

\enquote{Ich wollte euren ersten offiziellen Streit nicht unterbrechen oder stören}, meinte Harry und zuckte mit den Schultern.

Ein paar Tage später, einen Tag vor dem großen Ereignis, stand Harry wie üblich auf, um seine Jogging-Sachen anzuziehen. Er verließ seinen Raum und wartete im Gemeinschaftsraum auf Ginny, die sich etwas verspätete.

Sie trafen sich mit Luna in der Eingangshalle vor dem großen Tor, öffneten es und begannen ihre übliche Runde zu joggen. Es war bereits seit geraumer Zeit wärmer und die Vögel zwitscherten ihnen begleitend zu. Sie joggten den schmalen Pfad hinunter, welcher zum Quidditch-Feld führte. Als sie näher kamen, bemerkten sie Musik, welche vom Quidditch-Feld zu kommen schien. Fragend schauten sich die drei an.

\enquote{Woher wohl die Musik kommt?}, fragte Luna.

\enquote{Ich nehme an, vom Quidditch-Feld}, sagte Ginny.

\enquote{Vom Quidditch-Feld?}, fragte Harry.

\enquote{Ja, das macht Sinn}, sagte Luna.

\enquote{Aber es kam noch nie Musik vom Quidditch-Feld}, meinte Harry.

Sie joggten ohne Unterbrechung weiter und kamen dem Quidditch-Feld immer näher. Normalerweise umrundeten sie es, oder liefen ein Stück den See entlang und drehten an der einsamen Buche um. Doch heute zog es sie zum Feld. Sie stiegen die Stufen hoch, um auf das Feld zu gelangen. Zuerst sahen sie nur ein paar Personen, die sich in der Mitte des Feldes unterhielten. Erstaunt blieben sie stehen. Harry erkannte Professor Elber, Professor Dumbledore und eine unbekannte Frau, die wie eine Geschäftsfrau mit Rock angezogen war. Sie unterhielten sich und die Frau schrieb sich Notizen auf ihren Block, den sie in ihren Händen hielt.

Professor Elber stand seitlich und blickte kurz herüber. Dann sagte er etwas zu Dumbledore und winkte die drei zu sich. Harry, Ginny und Luna kamen näher und schauten sich um. Überall standen dicke schwere Metallkoffer und Kabel. Stehlampen wie für Fußballübertragungen und Kameras waren über das gesamte Feld verteilt.

\enquote{Was ist das?}, fragten Luna und Ginny.

\enquote{Equipment für Fernsehübertragungen}, sagte Harry.

\enquote{Aber in der Nähe von Hogwarts\abs}, meinte Luna.

Doch Harry stieß ihr seinen Ellenbogen in die Seite und meinte: \enquote{Nicht hier. Die sehen mir nach Muggel aus.} Er zeigte auf einige Personen die wie Arbeiter bei einer Fernsehproduktion aussahen.

Sie kamen den dreien näher und die Frau fragte die beiden: \enquote{Schüler von Ihnen?}

\enquote{Ja}, antwortete Professor Elber. \enquote{Miss Lovegood, Miss Weasley und Mister Potter}.

Die Frau kam näher und schüttelte den dreien die Hände. \enquote{Miss Smith.} Sie zog drei Visitenkarten aus ihrer Tasche und überreichte sie den dreien. \accentuate{Fernsehveranstaltungen \gst Liveübertragungen \gst Gameshows}, stand auf den Kärtchen unter ihrem Namen.

\enquote{Fernsehveranstaltungen?}, fragte Harry.

Luna und Ginny schauten sich nur um und trauten sich nicht zu fragen.

\enquote{Ja}, begann Miss Smith. \enquote{Wir sind bei den Vorbereitungen für eine große Übertragung. Morgen findet hier das \enquote{Magie in Konzert} statt. Ich muss sagen, eine überwältigende Kulisse. Ein ideales Ambiente für ein Konzert.}

\enquote{Konzert?}, fragten alle drei und Luna und Ginny widmeten ihre Aufmerksamkeit wieder Miss Smith.

\enquote{Ja. Wussten Sie das nicht? Hier findet in ein paar Wochen\abs} und sie drehte sich mit erhobenen Händen im Kreis, als ob sie die Kulisse bewunderte, \enquote{ein großes Konzert statt.} Sie drehte sich zu Professor Elber und meinte: \enquote{Obwohl ich immer noch nicht weiß, wo sie so seltsame Gruppen wie die \enquote{Schwestern des Schicksals} herbekommen.}

Professor Elber schaute mit einem kaum merkbaren Lächeln zu Luna und Ginny. \enquote{Die habe ich auf einer meiner Reisen getroffen. Ich fand sie passend.}

\enquote{Welche Bands treten denn auf?}, fragte Harry.

\enquote{Wie? Welche Bands? Wissen Sie nicht\abs?}

\enquote{Wir hatten keine Ahnung}, sagte Ginny.

Professor Elber fügte hinzu: \enquote{Das soll eine Überraschung sein. Behalten Sie es bitte für sich. Dann dürfen Sie am Abend nach dem Konzert hinter die Bühne.}

Harry, Ginny und Luna nickten und schauten sich weiter um. Professor Elber widmete sich wieder Professor Dumbledore und Miss Smith. Harry, Ginny und Luna machten sich auf den Weg um ihren Morgensport fortzusetzen.

\enquote{Das hört sich gut an. Ein Konzert. Aber ich dachte immer, dass Magie sich nicht mit der Elektrizität der Muggel verträgt? Wie kann dann in der Nähe von Hogwarts ein derartiges Konzert stattfinden?}, fragte Luna.

Harry überlegte. Da  fiel ihm etwas ein. \enquote{Professor Elber muss einen Weg gefunden haben. Ich habe ihn vor einiger Zeit dabei beobachtet, wie er an einem Radio herumschraubte. Und es funktionierte. Ich sprach ihn darauf an und er meinte: \inner{Das ist nur der erste Schritt. Ich hoffe, dass ich da weiter komme. Da bin ich schon einige Jahre dabei, einen Weg zu finden unter bestimmten Umständen eine Art Symbiose zu erreichen.} Er hat es wohl erreicht, ein großes Areal so zu verzaubern, dass Elektrizität funktioniert.}

Sie joggten zurück, um sich zu duschen und um sich für das Frühstück umzuziehen. Gespannt warteten sie auf den Abend, an dem das Konzert stattfinden sollte.

Harry wechselte während er duschte mit Luna wieder in seinen Körper zurück. Glücklicherweise duschten die beiden in der Dusche ihrer jeweiligen Geschlechter, in dessen Körper sie gerade steckten. Er duschte also in der Jungendusche, wenn er in seinem Körper war und in der Mädchendusche, wenn er in Lunas Körper war. Sie tat es ebenso. So konnten beide außerdem kaschieren, wer gerade wer war und sich bei den anderen etwas umschauen. Beide hatten das beschlossen, nachdem sie einmal während einer ähnlichen Situation plötzlich ihre Körper getauscht hatten.

\trenn

\enquote{Kreacher}, schrie Harry, (er war wieder in Lunas Körper) als er sich in einem verlassenen Klassenzimmer eingesperrt und einen Schweigezauber auf die Tür gelegt hatte. Es dauerte eine Weile, dann erschien der alte Elf vor ihm. Sein Gesicht zum Boden gesenkt, um sich kurz zu verbeugen. \enquote{Kreacher?}

Der Elf sah zu ihm hoch und stutze. \enquote{Miss Luna, wie konnten Sie\abs}

Doch Harry hob seine Hand und unterbrach damit den Elfen. \enquote{Ich bin Harry. Ich habe auf irgendeine Art und Weise mit Luna den Körper getauscht. Das ist uns die letzten Tage mehrmals passiert. Ich habe dich bisher nur nicht gebraucht, wenn ich in ihrem Körper steckte.}

\enquote{Wie lange?}, fragte der Elf nach. Er sah nicht so aus, als ob er ihm glauben würde.

\enquote{Zwölf Tage etwa}, antwortete Harry.

Der Elf ging langsam und vorsichtig auf ihn zu. \enquote{Wenn mir Miss Luna ihre Hand geben würde?}, meinte er.

Harry reichte ihm seine Hand. Als der Elf seine Hand berührte, zuckte er zusammen und nahm seine Hand zurück. Er verbeugte sich abermals und sagte: \enquote{Verzeiht Kreacher, Sir Harry, dass ich euch nicht geglaubt habe.} Er wollte schon auf eine Wand zulaufen, um seinen Kopf dagegen zu schlagen, doch Harry hielt ihn auf.

\enquote{Kreacher, nein}, sagte er.

Der Elf drehte sich wieder um. \enquote{Wenn Sir Harry einverstanden ist, kann Kreacher mit ein paar anderen Hauselfen dafür sorgen, dass Sir Harry und Miss Luna wieder in ihren Körper zurückkommen.}

Harrys Augen weiteten sich. \enquote{Im Ernst? Geht das? Werden wir wieder Gefahr laufen zu wechseln?}

Kreacher antwortete: \enquote{Ja Sir Harry, Elfen sind dazu in der Lage. Aber nur Elfen, welche eine besondere Beziehung zum\abs} er überlegte kurz, was er sagen sollte, \enquote{getauschten haben, sind dazu in der Lage. Neben mir und Dobby sind noch drei andere Elfen dazu in der Lage. Sie finden Sie interessant. Das sollte genügen, um mit dem Prozess zu beginnen.}

\enquote{Gut, Kreacher, ich werde mit Luna in die Küche kommen. Ich muss erst schauen, wann sie Zeit hat.} \gedanke{Irgendetwas verheimlicht er mir. Aber was soll’s. Solange es mir nicht schadet.}

\enquote{Warten Sie aber nicht zu lange, Sir Harry, sonst könnte der Prozess nicht mehr umkehrbar sein, befürchtet Kreacher. Zumindest nicht für uns Elfen, für Zauberer oder Hexen schon. Dann aber mit Komplikationen. Eventuell großen Komplikationen.}

Harry nickte und winkte Kreacher zu, worauf dieser sich verbeugte und verschwand.

Harry musste wohl oder übel noch einige Stunden Unterricht über sich ergehen lassen, bevor er die Gelegenheit hatte Luna zu treffen. Hagrid hatte ihnen letztes mal erzählt, dass sie dieses Mal etwas besonderes durchnehmen wollten. Also machte er sich auf den Weg hinunter zu Hagrid. Er war einer der ersten, die dort standen. Leider auch Draco Malfoy mit seinen Kumpels. In der Öffentlichkeit waren sie immer bemüht nicht allzu freundlich aufeinander zu wirken, bis auf die Tatsache, dass er ein paar Mal Tamara zu Liebe eine bessere Miene ihm gegenüber an den Tag legte. Ihre Beziehung zueinander war zwar besser geworden, aber warm wurden sie sich deswegen trotzdem nicht. Sie ignorierten einander eher, als dass sie sich beschimpften.

Heute hatte ihnen Hagrid nur graue Theorie beigebracht. Er nahm etwas dran, was sie schon letztes Jahr einmal hatten, und sprach über dies und das. Spannende Tiere (für ihn spannend) und wie sie zu pflegen waren.

\enquote{Am liebsten würd' ich mit euch in'n Wald geh'n}, sagte er. \enquote{Zu den Zentaur'n.} Die Klasse reagierte geschockt. \enquote{Habt ihr 'n Glück. Dumbledore will's nicht. Sagt, das sei zu gefährlich.}

Die Klasse atmete erleichtert auf. Dann beendete Hagrid die Stunde und Harry lief neben Ron und Hermine zurück ins Schloss. Es war schön, ihnen zuzusehen, wie sie noch mit sich in der Öffentlichkeit kämpften.

\enquote{Brauchst du Hilfe bei Granger?}, fragte Draco spöttisch Ron. Harry drehte sich um und sah neben ihm Crabbe und Goyle, sowie Pansy Parkinson. Er wollte schon einen bissigen Kommentar abgeben, doch Ron war schneller.

\enquote{Wenn du, oder einer deiner beiden\abs Dienstboten andeuten wollt, dass\abs} doch Ron verstummte, da er Professor Snape aus seinem Augenwinkel sah. \enquote{Ich Hilfe brauche, dann sag mir wobei, und ich denke darüber nach.} Dann drehte er sich wieder um und zog Hermine sanft ins Schloss.

Harry sah zwischen Draco und Professor Snape hin und her. Er wusste, dass Draco so reagieren musste. Als Snape ihn ansah, trat er ebenfalls ins Schloss und verschwand in dessen Inneren. Er setze sich neben Luna und flüsterte ihr ins Ohr. \enquote{Ich habe heute mit Kreacher gesprochen.} Er belud sich seinen Teller mit einem saftigen Stück Fleisch und einer Portion in Butter geschmorten Maiskolben. Luna sah ihm zu. \enquote{Die sind in Butter geschmort}, sagte sie.

Die um sitzenden Schüler begannen zu kichern. \enquote{Ich habe, besser gesagt Kreacher, hat einen Weg gefunden, wie wir unseren normalen Zustand wieder herstellen können.}

\enquote{Ahm!}, gab sie als Antwort, da sie gerade den Mund voll hatte. Sie strich ihm über seine Wange und nickte ihm zu. Das tat sie immer, wenn ihr die Worte fehlten und sie ihm sagen wollte, sie hatte ihn verstanden.

Nach dem Essen gingen die beiden in die Küche, um Kreacher zu treffen. Dieser gab ihnen zu verstehen, dass sie hier am falschen Platz seien. Die fünf Elfen bildeten mit Harry und Luna einen Kreis und apparierten in ihr Zimmer im Gemeinschaftsraum der Paare.

Dort wies Kreacher die beiden an, sich gegenüberzustellen, die Hände gegeneinander und die Stirn aufeinander zu legen. Als sie sich in die richtige Position begeben hatten, stellten sich die Elfen so hin, dass sie jeden der beiden berührten. Harry und Luna wurde schwarz vor Augen und beide bekamen leichte Kopfschmerzen. Dann fühlte Harry sich, als würde er schweben, sein Geist löste sich von Lunas Körper und bewegte sich auf seinen zu. Dann war alles wieder normal. Die drei kleinen Elfen lösten ihre Hände von ihnen und verschwanden sofort. Kreacher und Dobby blieben noch eine kleine Weile. Kreacher hatte ihnen noch etwas mitzuteilen.

\enquote{Sir Harry, Miss Luna. Kreacher muss Ihnen noch etwas sagen. Sie sind durch die Reintegration geschwächt. Sie sollten die Nacht hier verbringen. Morgen früh sind Sie dann ausgeruht und alles ist wieder normal.} Die beiden Elfen verneigten sich leicht und verschwanden mit einem \geraeusch{Plopp.}

Nachdem sich beide ihre Pyjamas angezogen hatten, stiegen sie ins Bett und zogen die Decke über sich. Ihre Hände lagen über der Decke. \enquote{Es ist so, wie in unserer ersten Nacht hier, Harry.} Luna traf den Nagel auf den Kopf. \enquote{Meinst du}, fuhr sie fort, \enquote{wir haben unsere Körper nur deshalb getauscht, weil wir Sex hatten und unsere seelische Verbindung zueinander so stark ist? Meinst du, das hat zu Interferenzen geführt?}

Luna war sehr scharfsinnig. Schließlich war sie in Ravenclaw. Obwohl sie noch immer von vielen ihrer Mitschüler gemieden wurde, verstand Harry sie immer besser.

\enquote{Das könnte es sein}, antwortete Harry.

Die nächsten Wochen gingen ohne nennenswerte Tauschaktionen dahin. Es schien, als ob der Zauber der Elfen gewirkt hatte.

Als es einige Tage später zum Frühstück ging, war Harry schon gespannt, was passieren würde. Er war sich nicht sicher, was in letzter Zeit im Schloss vor sich ging. Seit einiger Zeit schienen alle Mädchen im Schloss ein eigenartiges Interesse an ihm zu haben. Sein Problem mit Luna war zwar gelöst, aber trotzdem konnte er sich darauf keinen Reim machen. Ihn hatten in letzter Zeit eine Menge Mädchen geküsst. Quer durch alle Häuser hindurch. Susan Bones genauso wie Elisabeth Marlow oder Lavender Brown. Auch Parvati und Padma Patil waren an ihm interessiert.

Er hatte ein Problem gegen ein anderes getauscht.




\begin{kommentar}
Kurz nachdem ich in einem Film einen Zug gesehen hatte, der von einer gesprengten Brücke fiel, hatte ich die Idee, die Rückfahrt vom Ministerium mit einer fehlenden Brücke zu schreiben. Da man immer wieder sieht, wie man in Filmen mit Autos eine große Distanz in der Luft überwinden kann, dachte ich mir, dass es schön wäre, dies mit einem Zug zu tun.
\end{kommentar}

\begin{kommentar}
Als Harry, Ginny und Luna auf dem Quidditch-Feld auftauchen, um zu sehen, woher die Musik kommt, treffen sie auf Miss Smith. Den Namen habe ich aufgrund der Aussprache so gewählt. Wegen des Doppel-s und dem th. Andererseits gibt es in Matrix einen Mr Smith. Eine schöne Doppeldeutigkeit, nicht?
\end{kommentar}

\chapter{Wasserwelten}


Harry war in einem der vielen Gänge und knutschte gerade mit Cho. Er genoss es, wie sich ihre Lippen berührten, als er ihren Mund auf seinem spürte. Er dachte an nichts, als er plötzlich eine weitere Präsenz spürte. Erschrocken löste er sich von Cho und starrte in die Richtung aus der er die Präsenz vermutete.

Er sah Pansy Parkinson. Cho rannte vor Schreck davon. Pansy grinste und wollte schon davonrennen, um es den anderen zu erzählen, vermutete Harry. Er rannte ihr kurz hinterher und umgriff ihr Handgelenk. Abrupt blieb sie stehen. Überrascht drehte sie sich um. Sie machte keine Anstalten mehr, fliehen zu wollen. Harry war sich nicht sicher, ob es nur ein Trick war, damit er in seiner Vorsicht nachließ. Er lockerte seinen Griff etwas, hielt ihr Armgelenk aber immer noch fest, sodass sie sich nicht entwinden konnte.

Pansy stutzte. Ihr Blick veränderte sich leicht. Sie schien weder sauer noch schelmisch zu sein. Sie schaute ihn mit einem erstaunten Blick an. Harry spürte, wie sein Herzschlag schneller wurde. Nun kam sie auf ihn zu und stand ganz dicht vor ihm. Sie hob ihre freie Hand und nahm seinen Kopf in ihre Hand. Langsam zog sie sich an ihn ran und begann ihn zu küssen. Harry war geschockt und wollte schon zurückweichen, aber etwas hinderte ihn. Er löste seinen Griff von ihrem Handgelenk und legt beide Hände um ihre Hüfte, um sie noch näher an sich ran zu ziehen. Sie nahm jetzt ihre andere Hand und umschloss mit ihren Händen seinen Kopf und vergrub ihre Finger in seinem Haar. Harry war außer sich. Noch nie hatte er so weiche und zarte Lippen gespürt.

Sie versanken in einem langen Kuss. Harry genoss es. Er hatte nicht gedacht, dass Pansy derart gut küssen könnte. Und wenn es auch nur einen Moment andauern sollte; ihm war es momentan egal. Er wollte sie nur weiter küssen. Sie brach den Kuss, nur um kurz darauf ihren Mund leicht zu öffnen und ihn wieder zu küssen. Er folgte ihrem Tempo und öffnete seinen ebenfalls. Sie begann sachte mit ihrer Zunge seine Lippen zu umspielen und fuhr seine Oberlippe entlang, nur um kurz darauf an seinen Zähnen entlangzugleiten. Er umspielte mit seiner Zunge ihre, als sie ein leises Stöhnen von sich gab.

Sie brach abermals den Kuss, und drehte ihren Kopf zur Seite. Harry wollte sie schon fragen, ob alles in Ordnung sei, als sie nur sagte: \enquote{Lass uns da hineingehen, Harry.} Harry musste schlucken. Sie sah ihn wieder an und zog ihn zu einer kleinen Tür, die in die Mauer eingelassen war.

Der Raum hatte die Größe einer Besenkammer, wie Harry im Halbdunkeln erkennen konnte. Als Pansy die Tür schloss, war es dunkel. Harry konnte nichts sehen. Im ersten Moment dachte er nur: \gedanke{Verdammt, sie hat mich eingeschlossen, sie spielte nur mit mir.} Er beruhigte sich schnell wieder als er eine weitere Präsenz spürte. Er hörte ein leises Rascheln, gefolgt von einem: \zauber{Lumos.} Sie hatte die Spitze ihres Zauberstabes zum Leuchten gebracht.

Harry schluckte. Er holte seinen Zauberstab ebenfalls heraus und verschloss damit sorgsam die Tür, drehte sich kurz von Pansy weg und murmelte zwei ihr unverständliche Zaubersprüche. Er legte einen Polsterzauber auf einen kleinen Bereich der Kammer und einen Wärmezauber, damit die kalten Bodenfliesen sie nicht erzittern ließen. Dann zog er seine Schulrobe aus und legte sie auf den Boden. Dann drehte er sich wieder zu Pansy um und schaute ihr tief in die Augen. Er trat wieder vor sie, doch sie war schneller und zog ihn fordernd zu sich heran.

\begin{abAchtzehn}

Wieder spürte er ihre zarten Lippen. Harry spürte  wie sie ihre Brüste an ihn drückte. Sie löste den Kuss und ging einen Schritt zurück. Sie schaute ihn an. Harry musste wieder schlucken. Langsam begann sie ihre eigene Robe aufzuknöpfen, um sie fallen zu lassen. Danach widmete sie sich Harrys Hemd.

\enquote{Willst du nicht auch?}, fragte sie.

Harry hob seine Hände und griff an den obersten Knopf ihrer Bluse. Als beide mit nacktem Oberkörper da standen, konnte er zum ersten Mal ihre Brüste sehen. Er senkte seinen Kopf und umspielte ihre Brustwarzen mit seiner Zunge. Sie ließ ein leises Stöhnen hören, während er mit seiner Zunge am Vorhof ihrer linken Brust spielte und mit seiner rechten Hand ihre andere Brust sanft massierte. Er stoppte kurz und strich dann mit seiner Zunge zwischen ihren Brüsten entlang hoch bis zu ihrer Gurgel. Pansy warf den Kopf nach hinten und lies ein leises Gurgeln und Stöhnen erklingen. Er fuhr mit seiner Zunge weiter über ihr Kinn zu ihrem Mund. Sie versank ganz in dem Kuss, den er ihr gab. Seine Hände waren inzwischen an ihrem Nacken angekommen. Sie spürte seine Berührungen und so jagte es ihr einen wohligen Schauer über den Rücken.

Sie fuhr mit ihren Händen seinen Rücken entlang und arbeitete sich an seinem Hosenbund entlang nach vorn, um die Knöpfe an seiner Hose zu öffnen. Er brach den Kuss und machte sich daran, es ihr gleichzutun. Bald standen sie nackt voreinander, im schmalen Schein des Zauberstabes. Er zog sie sanft an die Stelle, die er mit dem Polster- und Wärmezauber belegt hatte und auf der jetzt ihre Schulroben lagen. Sanft drückte er sie an ihren Schultern auf den Boden, auf den sie sich nun legte. Jetzt lag sie vor ihm. Er beugte sich vor und stolperte. Er kam mit seiner Nase kurz vor dem Zentrum ihrer Lust zum Stoppen und hörte ein leises Kichern. \gedanke{Na warte, Pansy}, dachte Harry. Er drückte seine Nase zwischen ihre Schamlippen und fuhr sanft nach oben. Wieder hörte er ein Gurgeln. Sie atmete schnell, aber gleichmäßig. Auch Harrys Puls- und Herzschlag war schon seit einiger Zeit schneller geworden.

Er schaute zu ihr auf. Seine Nasenspitze war nass. Er streckte langsam seine Zunge raus und fuhr wieder zwischen ihre Schamlippen. Ihre Hände bohrten sich in die Schulroben, auf denen sie lag und ein leiser Schrei drang aus ihrer Kehle. \enquote{Harry!} Dann bahnte er sich seinen Weg in leichten schlängelnden Bewegungen nach oben zu ihrem Bauchnabel, wo er kurz verweilte und ihn küsste. Er kam ihren Brüsten langsam näher und beschrieb mit seiner Zunge eine Acht. Seine Hände streichelten an ihrer Seite entlang hoch. Er fand schließlich wieder ihren Mund und beide versanken in einen langen und innigen Kuss. Als er den Kuss kurz unterbrach, leckte sie über seine feuchte Nase, nur um ihn kurz darauf zu sich zu ziehen und ihn erneut zu küssen.

\end{abAchtzehn}

\begin{safedivide}
\fskdivider
\end{safedivide}

\onelineback % Anderenfalls werden 2 Leerzeilen gesetzt

\begin{rueckblick}
Als es Harry zum ersten Mal auffiel, dass sämtliche Mädchen in Hogwarts sich ihm zugeneigt fühlten, sprach er mit Dumbledore darüber. Auch er wusste sich nicht zu helfen, schickte ihn aber mit einem Brief zu Madame Pomfrey. Also machte sich Harry auf den Weg von seinem Büro aus zum Krankenflügel. Er trat ein und klopfte, nachdem er den Krankenflügel durchquerte hatte, an die Tür ihres Büros. Er hörte ein deutliches \enquote{Herein!} und öffnete die Tür.

Er gab ihr den Brief und meinte: \enquote{Der ist von Professor Dumbledore.}

Sie nahm den Brief und öffnete ihn. Ihre Augen verengten sich und sie sah zwischen ihm und dem Brief hin und her. Harry bekam einen Kloß im Hals. \enquote{Kommen Sie mit, Potter}, sagte sie zum ihm und ging durch eine kleine Tür, auf der \accentuate{Apotheke} stand. \enquote{Schließen Sie die Tür hinter sich}, sagte sie, zog ihren Zauberstab und machte ein kleines Feuer unter einem Kessel. Sie entnahm einige Fläschchen und öffnete diverse Fächer an ihrem großen Wandschubladenschrank.

\enquote{Was ist los?}, wollte Harry wissen.

Madame Pomfrey fing an. \enquote{Die weibliche Bevölkerung von Hogwarts hat ein starkes Interesse an Ihnen. Wir wissen nicht warum, aber Professor Dumbledore meinte, dass wir Sie vor, naja, etwaigen Folgen schützen müssen.}

Harry rätselte und schaute Madame Pomfrey verwundert an. Aber so langsam begriff er.

\enquote{Was ich Ihnen gebe, ist ein Trank, der Männer bis auf Weiteres unfruchtbar macht. Es wissen nicht viele davon, da die Zutaten selten sind. Die für den Prophylaxis-Trank sind viel leichter zu beschaffen.} Harry nickte. \enquote{Es gibt ein Gegenmittel, das ich Ihnen Morgen geben werden. Das muss etwas länger köcheln. Heben Sie das Gegenmittel auf, bis sich die Sache mit den Schülerinnen gelegt hat, dann können Sie es nehmen. Denken Sie daran, dass Sie so lange keine Kinder zeugen können.} Sie schöpfte den Trank in einen Becher und meinte: \enquote{Den Kessel müssen Sie austrinken. Zwei Becher voll.} Harry nickte und schluckte seinen Trank hinunter.
\end{rueckblick}

\onelineback % Anderenfalls werden 2 Leerzeilen gesetzt

\begin{abAchtzehn}
Harry und Pansy trennten wieder ihre Münder voneinander.
\end{abAchtzehn}

\trenn

Harry verstand die Welt nicht mehr. Tränen überströmt saß er auf dem kalten Fußboden. Das hatte er verdient. Nie hätte er sich träumen lassen, dass es so weit kommen würde. Seinen Kopf zwischen den Beinen saß er da; vor sich eine Pfütze aus Tränen. Professor Snape kam den Gang entlang auf ihn zu und nachdem er sich vergewissert hatte, dass niemand weit und breit zu sehen und zu hören war, setzte er sich neben Harry auf den Boden, aber so, dass er jederzeit aufstehen konnte, falls er jemanden hören konnte.

\enquote{Was ist los, Potter?}, fragte ihn Snape.

Stille.

\enquote{Potter?}

\enquote{Ich schäme mich. Sie wissen bestimmt, wie es um mich steht!}

\enquote{Ihr momentaner Zustand, der alle Frauen verrückt macht? Nach Ihnen?}

Es war Harry so peinlich, darüber zu reden. Aber irgendjemand brauchte er und Snape war momentan genau der Richtige. Er würde ihn ohne Wenn und Aber zusammen stauchen, aber das hatte er wohl verdient.

\enquote{Professor? Ich habe\abs hätte beinahe\abs} Irgendwie fand er nicht die richtigen Worte.

\enquote{Dann von Anfang an}, sagte Snape. Allerdings nicht bedrohlich, sondern eher so, wie ein fürsorglicher Freund.

Harry sah ihn mit immer noch nassem Gesicht an. \enquote{Kennen Sie die Erstklässler aus Ravenclaw?}

\enquote{Einige!}

\enquote{Auch eine dunkelhaarige, braune, mit kleinen Locken und schulterlangem Haar?}

Snape nickte.

\enquote{Sie rannte einfach in mich hinein. Wir haben uns nicht gesehen. Wir fielen um, sie lag auf mir und sie küsste mich.} Wieder lief eine Träne sein Gesicht herunter.

\enquote{Das ist doch kein Grund\abs}, doch Harry hob die Hand und saß wieder auf den Boden.

\enquote{Sie hat intensiv mit mir geknutscht und ich konnte nicht\abs Jedenfalls überkam es uns und sie zerrte mich in ein Zimmer. Ich widersprach nicht. Bald waren wir\abs} Er sah Snape wieder an. \enquote{Wir hätten beinahe miteinander geschlafen\abs} Doch er kam nicht weiter.

Snape zog überraschend ein Taschentuch aus seiner Tasche heraus und reichte es Harry. Es war noch unbenutzt. Harry trocknete seine Tränen und schniefte einmal feste hinein. Dann gab er es Snape zurück, der es einsteckte.

\enquote{Eine elfjährige}, sagte er noch matt. \enquote{Ich könnte es bei allen ab vierzehn ja noch\abs}, machte er weiter.

Professor Snape stand auf und zog Harry hoch. Dann trug er ihn halb neben sich her, hinunter Richtung Kerker, durch das leere Klassenzimmer für Tränke in sein Büro. Danach setzte er ihn in einen Stuhl, gegenüber seinem Schreibtisch.

\enquote{Ich brauche unbedingt mehr Selbstkontrolle}, sagte Harry. \enquote{So kann es nicht weitergehen. Ich meine, ich habe nicht, wenn mich Fünft-, Sechst- oder Siebtklässlerinnen küssen und mit mir schmusen \gst oder mehr wollen \gst aber alle darunter\abs Bisher habe ich jüngere\abs es ist schrecklich Professor.}

Snape hatte unterdessen ein Feuer unter einem Kessel gemacht und warf einige Kräuter und andere Flüssigkeiten hinein. Dann nahm er ein Fläschchen mit einer Pipette und zählte Tropfen ab. Er goss einen Becher voll und leerte den Kessel mit dem Schwung seines Zauberstabes. Dann stellte er den Becher auf den Tisch und kühlte ihn mit einem anderen Zauber herunter, da er noch gefährlich dampfte.

\enquote{Trinken Sie das, das wird Ihre Sorgen für ein paar Stunden vertreiben. Gehen Sie danach gleich ins Bett. Morgen werden Ihre Sorgen zwar wieder kommen, aber Sie brauchen erst einmal Schlaf.}

Harry nickte, nahm den Becher und trank ihn in einem Zug leer. Dann trottete er die Stufen und Gänge hinauf, bis in sein Zimmer. Er ignorierte alle Leute, die von ihm etwas wissen wollten, zog sich um und ging ins Bett. Sorgenfrei schlief er ein und träumte.

\begin{traum}
Er stand auf einer Wiese und sah sich um. Seine Mutter und sein Vater standen neben ihm. Sie lächelten ihn an. \enquote{Hallo Harry. Du hast heute bewiesen, dass du stark bist}, sagte sein Vater. Harry verstand nicht. \enquote{Du hast dich heute zusammengerissen und nicht mit der Kleinen geschlafen}, sagte seine Mutter stolz.

\enquote{Was?}, stammelte Harry. \enquote{Ihr seid tot. Ihr wisst nicht, was ich gemacht habe.}

\enquote{Ja das stimmt}, sagte seine Mutter und nahm ihn in ihre Arme. \enquote{Aber wir sind immer bei dir, wenn du dein Amulett hältst, während du einschläfst.}

\enquote{Ich halte gerade mein Amulett?}, fragte Harry ganz erstaunt.

\enquote{Ja und nein}, antwortete sein Vater. \enquote{Du träumst gerade.}

\enquote{Immer, wenn du träumst und sorgenfrei einschläfst}, machte seine Mutter weiter, \enquote{dann können wir dir Gesellschaft leisten. Wir bekommen nur das von dir mit, was du uns sagst, oder das, woran du gedacht hast, während du einschliefst.}

Harry verstand so langsam. \enquote{Ihr meint, ihr seid tot, aber könnt euch mit mir durch meine Träume unterhalten?}, fragte Harry.

\enquote{Ja}, antwortete seine Mutter. \enquote{Wir sind erstaunt, dass du so gut aussiehst, Harry. Wir konnten immer wieder mal einen kurzen Blick auf dich werfen und sind stolz auf dich.}
\end{traum}

Dann verschwamm der Traum und Harry drehte sich im Bett um. Am anderen Morgen erinnerte er sich nur noch schemenhaft daran.

Er saß gerade bei Dumbledore und erklärte ihm die Sache mit der elfjährigen und was er beinahe getan hätte. Seine Augen waren kurz davor, zum Tränen anzufangen. Sein Schulleiter sah ihn nur an. \enquote{Tja}, war alles, was er sagte. Dann herrschte eine Weile Stille. Dumbledore stand auf, ging zu seinem Kamin und warf eine kleine Menge Flohpulver hinein. \enquote{Severus Snape}, sagte er.

Es dauerte eine Weile, dann tauchte der Kopf von Professor Snape auf. \enquote{Schulleiter? Was ist los?}

\enquote{Harry braucht ein Mittel, um seinen Willen zu stärken. Er hat mir gerade eben erzählt, was er gestern beinahe getan hätte.}

\enquote{Die kleine Ravenclaw? Braune Haare?}, fragte Snape.

Erstaunt hob Dumbledore eine Augenbraue und sah zu Harry. Dieser blickte ihn an und sagte nur: \enquote{Ich habe es ihm gestern erzählt.}

Dumbledore nickte und drehte sich wieder zum Kamin. \enquote{Wann ist der Trank bereit?}

\enquote{Morgen Nachmittag. Ich arbeite bereits daran.} Dann verschwand Snapes Gesicht aus dem Kamin und Dumbledore setzte sich.

\enquote{Ich schäme mich so, Professor. Ich bin kein guter Mensch.}

Doch Dumbledore unterbrach ihn. \enquote{Doch Harry, du bist ein guter Mensch. Du hast es nicht getan, obwohl es dir nicht leicht gefallen ist. Ich bewundere dich. Du hast sehr viel Selbstkontrolle.}

Harry zweifelte noch immer. \enquote{Aber Sir\abs}

\enquote{Kein Aber, Harry}, sagte Dumbledore und hob erneut seine Hand. \enquote{Du wusstest genau, wo deine Grenze liegt. Ich bin mir sicher, dass nicht viele Schüler in deiner Situation genauso gehandelt hätten. Außerdem\abs} und Dumbledore machte eine kurze Pause, damit Harry wieder zu ihm sah, \enquote{außerdem hat mir unsere Krankenschwester von dem kleinen Vorfall in der Apotheke berichtet. Sie bemerkte das Blitzen und Verlangen in deinen Augen.} Er lehnte sich vor. \enquote{Und Harry, das darfst du niemandem sagen, wenn du ihr Andeutungen gemacht hättest, oder sie auch nur berührt, oder angelächelt hättest, dann hätte sie ihre Beherrschung verloren.}

Jetzt musste Harry schmunzeln und war erleichtert.

Auch Tage später wirbelten Harrys Gedanken immer noch um Pansy und ihr gemeinsames Intermezzo. Hatte er wirklich mit ihr geschlafen? Oder verdrängte er es nur? Snapes Trank hatte seinem Willen eine gewisse Stärke verschafft, sodass er bei Leuten, bei denen er es für unangebracht hielt, nicht weiter ging, als mit ihnen verstohlen oder offen zu knutschen. Cho hatte ihn in der Zwischenzeit angesprochen, um zu erfahren, was er Pansy versprochen hatte, damit ihre kurze Affäre nicht im Schloss bekannt wurde. Sie hatten sich schließlich, nach ihrem flüchtigen Kuss im Raum der Wünsche, nicht großartig aufeinander eingelassen. Er antwortete ihr nur: \enquote{Sie war ganz pflegeleicht.} Weitere Informationen bekam sie nicht aus ihm nicht heraus.

Harry saß wieder mit Pansy auf einer Bank im Schloss und knutschte ein wenig mit ihr. Die Bank stand an der Wand, direkt gegenüber eines Treppenaufganges. Man konnte also jeden sehen, der die Treppe hochkam. Die beiden waren so mit sich beschäftigt, dass sie die Schritte, die die Treppe hochkamen, nicht hörten. Beide erschraken, als sie \enquote{Pansy, Potter} und \enquote{Harry, Parkinson} hörten. Erschrocken schauten sie die beiden an. Da standen Ron Weasley und Draco Malfoy. Harrys Herz schlug wieder schneller, er wollte nicht mit Pansy erwischt werden, sie waren die letzten Wochen sehr vorsichtig gewesen, damit ihre Romanze nicht bekannt wurde. Harrys Gedanken rasten. Er blickte zu Pansy und dann tat er etwas, womit sie nicht gerechnet hatte. Er küsste sie munter weiter, so als ob sie nicht unterbrochen worden wären. Harry konnte praktisch ihre ratlosen Gesichter sehen.

\enquote{Harry!} \enquote{Potter!}, hörte er wieder.

Er unterbrach den Kuss und schaute beide mit aufgesetztem, enttäuschtem Gesichtsausdruck an. Pansy tat das Gleiche. Sie starrten in die entsetzten Gesichter der beiden. Er musste sich ein Lachen verkneifen.

Pansy zog ihn an seinem Ärmel und meinte: \enquote{Lass uns woanders hingehen, wo wir ungestörter sind.}

Er schaute sie an und nickte. Dann standen beide auf und gingen Hand in Hand den Gang entlang um die nächste Ecke. Ab dort beschleunigten sie ihren Gang und jeder lief für sich die geschwungene Treppe hinunter. Unten angekommen setzten sich beide auf die dritt-unterste Stufe. Sie sahen sich an und mussten anfangen zu lachen. Harry hatte eine Hand vor seinem Bauch und die andere vor seinem Mund. Pansy bot sich als sein Spiegelbild an. Nach kurzer Zeit wechselte er von seiner Hand auf seinen Unterarm, um sein Lachen nicht durch das Schloss erklingen zu lassen.

Eine knappe Minute später hörten beide wieder Schritte. Noch immer kicherten und lachten beide, auf den Stufen sitzend. Sie drehten sich um, schauten in die noch erstaunter dreinblickenden Gesichter von Ron Weasley und Draco Malfoy.

Dann sagte Pansy: \enquote{Ihr hättet vorher wirklich eure Gesichter sehen sollen. Als hätte euch der Blitz getroffen.}

Harry nickte und fing nun an laut zu lachen. Pansy stimmte in sein Lachen ein.

\enquote{Das \gst das war \gst war inszeniert?}, fragte Malfoy ganz ungläubig und Ron schaute aus, als wäre ihm übel.

Harry lachte noch immer und nickte. \enquote{Aber sicher doch. Ihr glaubt doch nicht etwa\abs}, und Pansy ergänzte: \enquote{Dass wir beide\abs}, sie zeigte auf Harry und sich, \enquote{etwas miteinander\abs}

\enquote{Nein}, antworteten beide im Gleichklang. Harry und Pansy schauten sich wieder an. Er hob seine Hand und sie schlug ein.

\enquote{Eure dummen Gesichter war es mir wert}, sagten beide fast gleichzeitig. Dann standen sie auf und verschwanden in unterschiedlichen Richtungen im Inneren des Schlosses.

Ron und Draco schauten sich nur verständnislos an.

\trenn

Harry saß bereits im Klassenzimmer und wartete als Erster auf seine Klassenkameraden. Als die Klasse zu etwa einem Drittel anwesend war, kam Professor Elber mit einer Pappschachtel herein. Er stellte sie auf den Tisch am Kopf der Klasse. Es dauerte noch eine Weile bis die Klasse komplett anwesend war. Er öffnete die Schachtel und nahm sie in eine Hand. Mit der anderen verteilte er aus der Schachtel an jeden Schüler einen kleinen schwarzen Würfel mit einer Kantenlänge von etwa 6~cm. Als jeder der Schüler einen Würfel erhalten hatte, lief er wieder nach vorne und legte die Schachtel mit den übrig gebliebenen Würfeln auf ein Sideboard. \enquote{Ihre Hausaufgabe bis Ende des Schuljahres ist es, diesen Würfel zu öffnen. Konzentrieren Sie sich auf den Würfel bis Sie ihn öffnen können. Sollten Sie einen neuen Würfel brauchen, so nehmen Sie sich einen. Wenn keiner mehr da ist, lege ich neue hin \gst Doch nun lassen Sie mich Sie auf Ihre Prüfung vorbereiten.}

Er drehte sich herum und nahm aus der Schublade seines Schreibtisches eine Menge von Zetteln heraus. Nachdem er alle verdeckt herum ausgeteilt hatte, meinte er: \enquote{Sie haben die restliche Stunde über Zeit, Ihre theoretische Testprüfung zu machen. Sie wird in ähnlicher Form in ein paar Wochen abgehalten werden. \gst Die Zeit läuft.}

Er holte aus seiner Tasche eine kleine Uhr heraus, die wie eine Stoppuhr aussah, und betätigte die Krone, um die Zeitmessung zu starten.

Während Harry seinen Zettel ausfüllte, hörte er immer wieder Gekicher und spürte heiße Blicke hinter seinem Rücken. Es dauerte noch, bis Snape seinen Trank fertiggestellt hatte. Er musste sich beherrschen. Wenn er kein \accentuate{Weibchen} sah, ging es. Doch er konnte seine Zuneigung zu den weiblichen Schülern nicht verbergen. Genauso, wie die Schülerinnen nicht von ihm lassen konnten. Er schloss seine Augen und konzentrierte sich. Er benutzte wieder die Okklumentik-Techniken, um seinen Geist zu leeren. Dann öffnete er wieder seine Augen und füllte seinen Zettel aus.

Nach der Stunde betrat er die Bibliothek, da er noch etwas benötigte. Er ging zielstrebig auf die Reihe zu, in der das benötigte Buch stand. Madame Pince stand vor ihm. Er überlegte, ob er sie nicht in Verlegenheit bringen sollte. In seinem Gehirn ratterte es. Verdient hätte sie es, nachdem sie ihn während seiner Strafarbeit nicht unbedingt fair behandelt hatte. Harry fixierte sich auf sie. Er schlich sich von hinten an sie heran und legte seine rechte Hand auf ihre linke Schulter.

\enquote{Entschuldigung, Madame Pince}, sagte Harry.

Sie drehte sich um und sah Harry erschrocken an. Hinter ihr stand Draco Malfoy und nahm ihr gerade ein Buch ab. Sie sah Harry nun direkt in die Augen. Seine Hand lag noch immer auf ihrer Schulter. Abwesend strich er ein paar Fusseln von ihrer Schulter. Mit einem Blick, den er von Luna abgeschaut hatte, sah er sie an. Seine Hand glitt an ihrem Arm hinab. An ihrer Hand angekommen, nahm er sie schließlich in seine. \enquote{Können Sie mir sagen, wo ich ein Buch über schwarze Würfel finde?} Er massierte jetzt ihr Handgelenk mit Daumen und Mittelfinger. Er spürte, wie sie ein Schauer durchzog.

Dann meinte er schließlich: \enquote{Schade, dann werde ich selber suchen.} Er ließ sie los und drehte sich gerade um, als er von hinten zwei Hände spürte, die ihn nach hinten zogen. Nur für eine Sekunde hatte sie ihn an sich gezogen, bevor sie ihn wieder losgelassen hatte und dann davon stürmte.

Harry war bereits auf dem Weg zu einer anderen Reihe um ein Buch zu holen, als ihn Malfoy und Zabini einholten.

\enquote{Klasse, Potter}, lachte Draco Malfoy hinter ihm. Und auch Blaise Zabini konnte sich nicht mehr halten. Harry drehte sich schmunzelnd um.

\enquote{Das war die kleine Rache.}

\enquote{Wofür?}, fragte Blaise ihn, immer noch lachend.

\enquote{Dafür, dass sie mich am Jahresanfang während meiner Strafarbeit ständig genervt hatte.}

Ein Mädchen schlich hinter Harry vorbei und fuhr ihm sanft seinen Rücken entlang. Harry schloss seine Augen und genoss die Berührung. Er sah ihr kurz nach. \gedanke{Romilda Vane, hatte Hermine gesagt.}

\enquote{Die Frauen stehen immer noch auf dich, oder?}, fragte Malfoy.

\enquote{Leider. Manchmal ist es ja angenehm, aber wenn man ständig betatscht wird, oder umarmt wird, oder\abs} er schluckte kurz. \enquote{Ohne Vorwarnung geküsst wird, dann ist es extrem nervig.}

Blaise sah zu Draco und meinte plötzlich: \enquote{Hast du dich inzwischen von deinem Schrecken erholt?}

\enquote{Von welchem Schrecken}, fragte Draco.

\enquote{Dass du Harry und Pansy beim Knutschen gesehen hast.}

Draco Malfoy wurde rot. \enquote{Mann, Potter}, sagte er dann, \enquote{da hast du mich aber schön erwischt. Der Schreck hängt mir immer noch nach.}

Harry grinste spitzbübisch.

\enquote{Sag mal, schwärmt sie irgendwie für dich?}

\enquote{Sagen wir mal so, es hatte mir die Sache wesentlich erleichtert, euch diesen Streich zu spielen. Wir haben euch schon am Fuße der Treppe entdeckt. Dann hatte ich die Idee.}

Er verabschiedete sich von beiden und steuerte einen anderen Bereich der Bibliothek an. Mit Blaise hatte er sich schon immer so weit verstanden, wie man sich als Schüler eigentlich verfeindeter Häuser verstehen konnte. Sie waren einander gegenüber neutral eingestellt.

Die Bibliothek war brechend voll. Die Suche nach der Mondbibliothek sowie dem schwarzen Würfel ergab nichts Neues, aber er hatte etwas über Gebärdensprache gefunden. Nun musste er nur noch lernen. Harry fand noch einen Platz an Lavenders Tisch und machte sich an seine Studien. Lavender hatte ein sehr altes und verschlissenes Buch vor sich liegen und betrachtete ein paar interessante Zaubersprüche. Sie hatte ihren Zauberstab in der Hand und vollführte einige kompliziert aussehende Bewegungen.

\enquote{Was übst du da?}, fragte sie Harry.

\enquote{Weiß ich nicht. Aber die Bewegung finde ich klasse.}

\enquote{Wie heißt der Zauber?}

\enquote{Weiß ich nicht.} Sie vollzog die Bewegung immer und immer wieder. \zauber{Ad unum omnes, induere potens domus}, sagte sie plötzlich, als sie wieder auf das Buch sah.
% sexuell					potens
% anziehen					induere
% haus						domus
% alle						cuncti, omnes
% alle zusammen				cuncti
% alle ohne Ausnahme		ad unum omnes
% alle bis zum Letzten

\enquote{\extase{Neeeiiinnn!}} hallte es plötzlich durch die gesamte Bibliothek. Vor Lavender bildete sich an der Spitze ihres Zauberstabes eine leuchtende Kugel, die größer wurde. Dann gab es einen Knall und eine Welle aus gleißendem Licht durchzog das Schloss. Professor Elber kam mit schnellen Schritten auf sie zu. Er drückte sich zwischen James und Leroy, zwei Freunde von Lavender, die ihr gegenüber saßen, und starrte Lavender mit einem mehr als zornigen Gesichtsausdruck an. \enquote{Was haben Sie getan?}, schrie er sie an. Er ignorierte die Beschwerden der beiden. \enquote{Wenn Sie einen Zauber nicht kennen, oder dessen Wirkung, dann lassen Sie die Finger davon, oder fragen einen, der sich damit auskennt. Fragen Sie einen der Lehrer oder Madame Pince. Frell.}

Die Kugel aus Licht schwebte noch immer vor Lavender. Sie ließ ihren Zauberstab sinken und die Kugel verschwand.

\enquote{Haben Sie eine Vorstellung davon, was Sie getan haben?}, brüllte er sie an. Harry hatte seinen Lehrer noch nie so zornig gesehen. Dann blickte Elber auf das Buch und schlug es zu. Seine Hand stützte er darauf ab. Lavenders Hand wurde eingeklemmt und sie stöhnte leicht auf. Mühsam zog sie ihre Hand wieder heraus. Tränen begannen nun über ihr Gesicht zu laufen. Aber noch immer sah sie Professor Elber wütend an. Sie begann zu weinen und zitterte am ganzen Leib.

Nur langsam beruhigte sich Professor Elber. Er schloss seine Augen und atmete tief ein und aus. Dann lief er um den Tisch herum und ging neben ihr in die Hocke, legte eine Hand auf ihre Schulter und fing an zu erzählen. \enquote{Tut mir leid, aber mit dem Zauber haben Sie uns allen eine Menge Ärger eingehandelt.}

\enquote{Ich\abs ich verstehe nicht Professor}, stammelte Lavender.

\enquote{Dieser Zauber}, fing Professor Elber mit leicht verärgertem Ton an, \enquote{sorgt dafür, dass jeder über jeden herfällt. Sie erinnern sich, dass wir eine ähnliche Situation gerade haben?}

Lavenders Augen weiteten sich vor Schrecken. \enquote{Aber\abs er hat nicht gewirkt}, sagte sie schüchtern.

\enquote{Er wirkt mit Verzögerung}, sagte Professor Elber und schaute auf seine Uhr. \enquote{Nach vierundzwanzig Stunden beginnt hier jeder, jedem hinterherzulaufen. Aber das Schlimmste daran ist, dass Harry hier den Effekt durch seinen Zustand noch verstärken könnte. Ganz davon abgesehen, dass Sie, die den Spruch ausgesprochen haben, besonders darunter zu leiden haben.} Betrübt schaute er sie an. Lavender war nicht in der Lage, etwas zu sagen. \enquote{Sind sie mit Sodom und Gomorrha vertraut?}, fragte er sie. Sie nickte stumm. \enquote{Der Spruch wurde dort von einem schwarzen Magier erschaffen, als er mit Schimpf und Schande aus der Stadt vertrieben wurde. Er hatte sich Rache geschworen und kam ein Jahr später wieder und legte den Spruch auf die Stadt. Dann verschwand er, um sich dem Zauber zu entziehen.}

Lavender war immer noch nervös. \enquote{Kann\abs man\abs gar nicht\abs gar\abs nichts dagegen tun?}, fragte Lavender.

\enquote{Schon}, antwortete Professor Elber, \enquote{aber der Gegentrank muss jedem einzeln eingeflößt werden und braucht achtundvierzig Stunden um fertig zu werden.} Dann fügte er hinzu. \enquote{Sie werden diesen Trank brauen. Sozusagen als Strafarbeit. Sie werden die nächsten zwei Tage kaum zum Schlafen kommen. Wir werden jetzt zusammen ins Lehrerzimmer gehen, wo Sie dem Kollegium ausführlich berichten werden. Wir müssen alle Schüler mit einem Klammerfluch belegen, sodass sich keiner einer Gefahr aussetzt. Selbst die Lehrer. Außerdem wird man ihren Eltern schreiben.}

Lavender lief es wieder eiskalt den Rücken runter. Professor Elber stand auf und Lavender folgte ihm.

Der ganze Tag war wie verhext. Kein Lehrer, egal welcher, konnte sich auf den Unterricht konzentrieren und alle sahen Lavender vorwurfsvoll an. Harry hatte mit ihr irgendwie Mitleid. Sie war zwar manchmal anstrengend, aber das hatte selbst sie nicht verdient. Immer wieder musste er sie gegen die gehässigen Slytherins verteidigen. Als ob das nicht genug wäre, fiel ihm Lavender deshalb jedes Mal dankbar um den Hals und küsste ihn flüchtig, denn Harry litt noch immer unter seinem Zustand.

Er musste sich in sein Bett legen, um in Kürze von seiner Hauslehrerin geklammert zu werden. Er verdrängte alle Gedanken und hatte das Gefühl zu schweben. Professor McGonagall sprach den Klammerfluch \spruch{Petrificus totalus} und Harrys Körper konnte sich nicht mehr bewegen. Doch Harry fühlte sich nicht anders als sonst. Er richtete sich wieder auf, um Professor McGonagall zu sagen, dass der Zauber wohl schiefgegangen sein musste. Doch  Professor McGonagall drehte sich um und verließ sein Zimmer. Er wollte ihr noch hinterherrufen, aber er brachte kein Wort heraus. Er verließ sein Bett und wollte in seine Pantoffeln steigen, als er bemerkte, dass er sich nicht sehen konnte. Er drehte sich um und sah seinen Körper im Bett liegen.

\enquote{Liegt das an der Okklumentik-Technik?}, fragte er sich.

\enquote{Ja, Potter}, hörte er hinter sich. Schlagartig drehte er sich um und sah Professor Snape.

\enquote{Professor?}

Er schaute wieder an sich herunter und bemerkte, dass er sich langsam sehen konnte. Immer fester wurde seine Gestalt.

\enquote{Wir beide sind geklammert, aber da Sie, genau wie ich, Ok"-klu"-men"-tik-Tech"-ni"-ken anwandten, konnten Sie Ihren Geist der Klammerung entziehen. Schweben wir in den Tränke"-keller und schauen ihr zu.} Er ging an Harry vorbei und ließ sich danach durch die Decke sinken. Harry folgte ihm durch Decken und Mauern, auf direktem Weg zum Zauber"-tränke"-keller.

Dort angekommen, sahen sie Lavender vor einem komplizierten Trank liegen. Ihre Haare waren offen und zerzaust, ihr spärliches Make-up war verschmiert und sie hatte Ringe unter ihren Augen. Dösend lag sie auf dem Boden und schlief. Der Wecker rasselte und Lavender mühte sich die nächsten Zutaten in den Kessel zu werfen und umzurühren.

Professor Elber betrat den Raum und meinte: \enquote{Professor McGonagall und Professor Dumbledore waren die letzten. Jetzt sind alle geklammert und können sich und ihnen nichts mehr antun.}

\enquote{Soll ich Sie jetzt auch klammern?}, fragte Lavender unsicher.

\enquote{Nein}, antwortete Professor Elber. \enquote{Ich habe mich unter Kontrolle. Der Wutausbruch, als Sie den Zauber ausführten, dämpften meine Gefühle. Ich sorge mich mehr um Sie. Ich fühle mich zwar zu Ihnen hingezogen, dafür haben Sie gesorgt, aber ich habe mich unter Kontrolle. Jemand muss ja auf Sie aufpassen. \gst Zumindest hoffe ich, dass ich es aushalte.} Er lächelte leicht.

\enquote{Wie darf ich das verstehen?}

\enquote{Falls ich Ihnen zu nahe komme, dann müssen Sie mich auch klammern. \gst Auch wenn Sie den Trank fertig haben und während wir den Schülern und Lehrern ihn einflößen, kommen sie nicht darum herum\abs}, er druckste etwas herum, \enquote{einige sexuelle\abs na ja, Handlungen vorzunehmen.}

Harry war der Meinung, das Lavender leicht übel wurde, doch sie braute tapfer an ihrem Trank weiter.

\enquote{Außerdem werden Sie noch Nachwirkungen spüren. Ich sage das jetzt äußerst ungern, aber Sie und Harry werden wohl übereinander herfallen und sich lieben ohne Ende. Solange bis Sie erschöpft sind. Das dürfte einige Nächte andauern.}

Harry stand mit großen Augen da. Snape konnte sich einen gehässigen Kommentar nicht verkneifen. \enquote{Das hört sich so an, als ob Sie eine Menge Spaß miteinander haben werden.}

\enquote{Und, und das ist das Wichtigste daran: Sie dürfen sich nicht dagegen wehren. Es wird umso stärker und intensiver, wenn Sie sich enthalten. Lassen Sie es einfach geschehen. Zum Glück haben Sie noch etwas Zeit, wenn das Ganze überstanden ist.}

\enquote{Weiß Harry davon?}, fragte sie halb begeistert, halb erschrocken darüber, dass sie mit dem berühmten Potter ganze sinnliche Nächte verbringen würde; oder sollte man sagen: müsste?

\enquote{Nein, noch nicht.} Professor Elber schloss kurz die Augen. \enquote{Obwohl\abs} Er öffnete sie wieder. \enquote{Ich könnte mir durchaus vorstellen, dass er so ein Gefühl diesbezüglich haben könnte.} Er musste nun leicht schmunzeln.

\gedanke{Spürt er meine Anwesenheit?}, schoss es Harry durch den Kopf. \gedanke{Nein, vermutlich kennt er meinen Zustand nur gut. Ja, das wird es sein.}

\enquote{Wissen Sie, ich habe nochmal in dem Buch in der Bibliothek gelesen. Es steht zwar nicht präzise drin, aber das zwischen den Zeilen lässt sich schon lesen, dass Sie und Harry sexuell sehr aktiv\abs}

Als der Trank nach zwei Tagen endlich fertig war, gingen Professor Elber und Lavender in die Gemeinschaftsräume zu den Schülern und Lehrern, die der Einfachheit halber auch dort lagen und flößten ihnen die Tränke ein. Professor Snape hatte es irgendwie geschafft, Lavender derart zu beeinflussen, dass der Trank auch gelang. Lavender schaffte es nicht immer, sich zu beherrschen und berührte ihre noch immer geklammerten Mitschüler und Mitschülerinnen manchmal an intimen Stellen. Glücklicherweise bekamen diese das nur sehr selten mit. Aber am peinlichsten war ihr das bei Professor Snape. Sie wusste, dafür würde sie bezahlen. Sie hatte zwar den Eindruck, dass er das nicht mitbekam, hatte aber das Gefühl, er würde es doch herausfinden.

Damit hatte sie nicht einmal so unrecht, denn er schwebte mit Harry neben ihr, als sie ihm den Trank einflößte.

Harry sagte trocken nun zu seinem Tränkeprofessor: \enquote{Ich verkneife mir jetzt mal einen Kommentar.}

Snape sah ihn nur an.

\trenn

Harry hatte sich im Gemeinschaftsraum in einen Sessel in der Ecke gesetzt. Er zog seine Füße auf den Sessel und drehte sich so, dass man ihn nicht gleich erkennen konnte. Er schloss seine Augen und dachte über ihn und Pansy nach. \gedanke{Unsere Beziehung hat sich gewandelt. Von einer anfänglichen rein sexuellen Beziehung zu etwas wie Liebe. Verliebt. Bin ich tatsächlich in Pansy Parkinson verliebt? Und warum ist Ginny eine der wenigen, die sich nicht an mich ran werfen?}, fragte er sich. Er hatte mit ihr noch nicht darüber gesprochen, aber er war sich sicher, dass Pansy seine Gefühle erwiderte. Ihre Küsse hatten sich verändert. Sie verwendete in letzter Zeit etwas, was ihre Lippen noch zarter machte. Das Porträt öffnete sich und Ron kam herein. Harry hatte seine Augen immer noch geschlossen; er hörte es an seinem Gang.

Er konnte sich seinen Gesichtsausdruck immer noch vorstellen. Er hörte, wie Ron anfing zu erzählen und ließ seine Gedanken gleiten. Er hörte nur mit halbem Ohr hin und öffnete ein Auge, aber nur so weit, dass er Hermine sehen konnte, die er gut im Blick hatte. Er hörte nur Bruchstücke der Unterhaltung, konnte sich aber den Rest lebhaft vorstellen. \enquote{Kleinen Zwist mit Malfoy\abs Harry und Pansy Parkinson\abs Knutschen\abs}, Hermine schluckte und quiekte immer mal wieder, \enquote{nicht unterbrechen\abs kringelten sich vor Lachen\abs hereingelegt.} Er entdeckte Hermine, die immer wieder verstohlen zu ihm hinüberblickte.

\enquote{Ich denke mal, da hat er dir und Malfoy einen schönen Schrecken eingejagt.}

\enquote{Das kann man wohl sagen. Mir läuft es immer noch kalt den Rücken runter, wenn ich mir das Bild vorstelle, wie ich beide zum ersten Mal sah. Auf der Bank sitzend und knutschend. Ich brauche wieder einmal eine heiße Dusche.} Er stand auf und murmelte noch: \enquote{Die Bilder werde ich nie wieder los.} Er schüttelte sich und lief die Stufen zum Gemeinschaftsraum hoch.

Hermine stand auf und kam zu ihm herüber. Da er in einem relativ großen Stuhl saß, setzte sie sich neben Harry, die Füße über seine und eine Armlehne des Stuhles legend. Sie legte ihren Kopf gegen die Rücklehne und schaute ihn mit wenigen Zentimetern Abstand an.

\enquote{Und nun erzähl, was los ist. Aber nicht die Geschichte für Ron und Malfoy, oder für den Rest von Hogwarts.} Sie war ihm dermaßen nahe, dass er sich anstrengen musste, um nicht offensichtlich und schuldbewusst zu schlucken. Misstrauisch schaute er sie an. \enquote{Erzähl mir keine Märchen Harry\abs}, sie küsste sanft seine Nase. \enquote{Ich weiß, dass zwischen euch Zweien was läuft.}

Harry blieb standhaft, doch es schien nicht zu helfen. Sie wurde forscher und kam seinem Gesicht näher.

\begin{rueckblick}
\enquote{Wissen Sie Mister Potter. Auch an uns geht der Effekt, den Sie auf Ihre Mitschülerinnen haben, nicht spurlos vorbei}, sagte Madame Pomfrey, als sie ihm seinen zweiten Becher einschenkte.

Als er ihn trank, fiel sein Blick direkt in Madame Pomfreys Augen. Sie hatte einen Blick an sich, den er noch nie gesehen hatte. \gedanke{Sie muss sich konzentrieren, um mir nicht näherzukommen}, dachte er. In ihm schwelte ein Plan. \accentuate{Sollte er es wagen? Nur ein kleines Intermezzo? Ein kleiner Kuss mit seiner Krankenschwester?} Er schloss seine Augen und schüttelte sich innerlich. Einerseits war der Gedanke faszinierend, eine Art Selkie-Charme zu haben, andererseits bedrückend, da ihm dauernd Mädchen nachliefen und sich an ihn schmiegten oder ihn zu küssen versuchten. Harry hatte bei vielen nichts dagegen und nach einigen anstrengenden Versuchen, sie abzuwehren, resignierte er nach einiger Zeit und wehrte seine Mädels, wie er sie nun nannte, nur noch halbherzig ab. (Selkies sind eine Art Wassermenschen, die ihren Charme spielen lassen können, um ihr Gegenüber sexuell zu erregen.) Harry konnte sich seine Anzugskraft nicht aussuchen. Sie schwankte. Und er konnte sie nicht abstellen. Zu allem Überfluss beherrschten ihn seine Gefühle, sodass er immer wieder die Kontrolle verlor.
\end{rueckblick}

Hermines Gesicht kam seinem Näher, bis sie seine Haut berührte. Ihre Lippen berührten seine und sie gab ihm einen langen Kuss. Vor diesem Moment hatte er sich die vergangenen Wochen am meisten gefürchtet. Einerseits war er stolz auf Hermine, dass sie es so lange aushielt.

\begin{rueckblick}
Die anderen Gryffindor-Mädchen in seinem Jahrgang hatten schon nach zwei oder drei Tagen aufgegeben und waren seinem Charme erlegen. Seitdem ließen sie ihn nicht mehr in Ruhe, bis er ihnen zumindest einmal nachgab und ausgiebig mit ihnen knutschte. Danach war es besser geworden.
\end{rueckblick}

Andererseits bedauerte er es, dass es so lange dauerte, bis Hermine ihn endlich küsste, bzw. Harry endlich Hermine küssen durfte. Er stellte sich schon längere Zeit vor, wie es wäre, Hermine zu küssen, doch er hatte auch etwas Bammel, da sie ja mit Ron zusammen war, seinem besten Freund. Er gab sich ihrem Kuss hin und genoss jede Sekunde. Seinen Zustand hatte Harry bei ihr recht schnell unter Kontrolle. Doch immer wieder wallte er kurz auf. Er brach den Kuss und meinte: \enquote{Hermine, was meinst du.}

\enquote{Ich meine du und Pansy. Ich glaube Ron nicht. Ihr habt nicht nur aus Show geknutscht. Da ist etwas zwischen euch.}

\enquote{Du weißt doch, Hermine, dass mir zurzeit kein Mädchen widerstehen kann.} Er gab ihr einen sanften Kuss auf ihren Mund, den sie umgehend erwiderte.

\enquote{Ja ich weiß}, sagte sie leicht betrübt. Sie war noch immer an ihn geschmiegt. Harry fühlte sich geborgen. So nahe bei Hermine zu sein. Sie schien eine Selbstkontrolle zu haben, die es ihr ermöglichte, selbst unter diesen Umständen eine einigermaßen normale Unterhaltung mit ihm zu führen. \enquote{Aber ich spüre, dass da mehr ist.} Sie blickte ihn mit einem durchbohrenden Blick an. \enquote{Du empfindest etwas für sie, habe ich recht?}

Er konnte nicht anders. Er küsste sie wieder und nickte dann kaum merklich und stumm.

Hermine seufzte leicht und meinte dann: \enquote{Ich habe es mir gedacht. Ihr beide empfindet etwas füreinander. Ich habe euch beobachtet, wenn ihr euch unbeobachtet fühltet. Ihr habt vielsagende Blicke ausgetauscht.}

Harry hob eine Augenbraue. \enquote{Und du bist dir sicher, dass\abs} Er küsste ihre Nase, doch sie zog ihn wieder an sich heran und küsste ihn. Er öffnete leicht seinen Mund.

Sofort begann sie seine Lippen und Zähne zu umspielen. Ihre Hände umschlossen seinen Kopf und griffen in sein Haar. Nach einem langen Zungenkuss löste sie sich wieder von ihm und sagte: \enquote{Du machst es mir nicht leicht, Harry.}

\enquote{Du mir auch nicht, wenn du so nah bei und auf mir sitzt.} Sie grinste ihn an und setzt sich nun neben ihn. \enquote{Mir scheint \gst } fuhr Harry sichtlich wohler fort, \enquote{aber erzähl Ron nichts davon \gst dass Pansy schon vor meinem Zustand etwas für mich empfand. Mein Zustand war nur der Auslöser. Wir sind jetzt schon seit wenigen Wochen zusammen. Seitdem ist sie die Einzige, mit der ich mich wie mit dir unterhalten kann. Sie hat mehr Selbstkontrolle als all die anderen. Noch etwas mehr als du.}

Hermine schaute ihn mit großen Augen an.

\trenn

Eine Woche später nahm sich Harry seinen Würfel an den See. Lavender hatte er die letzte Woche gemieden, wissend, dass er es dadurch noch verschlimmern würde, aber er brauchte die Zeit für sich, um sich darauf vorzubereiten. Die paar Tage würden es, sagte man ihm, nicht sonderlich erschweren. Er tauchte seine nackten Füße in das klare Nass, nahm seinen Würfel aus der Tasche und betrachtete ihn. Er sah den schwarzen Würfel an. Er betrachtete ihn lange. Dann entschloss er sich, sich auf einen flachen Stein zu setzen, der aus dem Wasser ragte.

Stumm sah er zwischen der leicht gekräuselten Wasseroberfläche und dem Würfel hin und her. Plötzlich bemerkte er, dass sich auf der Wasseroberfläche etwas tat. Ein Meereslebewesen streckte seinen Kopf aus dem Wasser und begrüßte Harry. Doch dieser verstand kein Wort.

\enquote{Ich verstehe sie nicht}, sagte Harry.

\enquote{Verzeihung}, antwortete das Wesen. Es schwamm näher. Erst jetzt konnte er es richtig erkennen. Es schien ein Weibchen zu sein. Sie sprach leicht gebrochen, aber Harry verstand sie trotzdem gut. \enquote{Bedrückt Sie irgendetwas?}, fragte sie Harry.

\enquote{Ja, ich habe hier einen Würfel bekommen und soll ihn öffnen, habe aber keine Ahnung wie ich es anstellen soll.} Er wollte nichts von Lavender und sich erzählen.

Sie schwamm noch etwas näher, bis er die kleinen Unregelmäßigkeiten in ihren Augen erkennen konnte. Sie hatte wunderbare Augen. Ihre Iris leuchtete so gelb wie die Sonne. Sie war sehr hübsch, fand Harry.

\enquote{Mein Name ist Chwalla}, sprach sie.

\enquote{Ich heiße Harry}, gab Harry zurück.

Ihr Blick fiel auf seinen Würfel. \enquote{Darf ich mal sehen?}, fragte sie ihn.

\enquote{Ja}, antwortete er und gab ihn ihr in die Hand.

Interessiert betrachtete sie den Würfel. \enquote{So einen habe ich schon mal gesehen. Unser König hat so einen, glaube ich.}

\enquote{Kann er mir helfen, ihn zu öffnen?}, fragte Harry.

\enquote{Keine Ahnung. Komm einfach mit und frage ihn.}

\enquote{Aber ich kann unter Wasser nicht Atmen.}

\enquote{Das hast du aber schon, als du die Prüfung im See absolviert hast.}

\enquote{Damals hatte ich aber Dianthuskraut.}

\enquote{Das ist kein Problem.} Sie verschwand, kam aber kurz darauf mit einem Bündel Dianthuskraut in der Hand zurück. Sie streckte ihm die Pflanzen entgegen und deute ihm an, er möge sie schlucken.

Harry war leicht mulmig, aber er tat wie ihm geheißen. Er kannte das Gefühl. Als er die Pflanzen herunterschluckte, spürte er ein leichtes Kribbeln in Hals. Er entledigte sich seiner Oberbekleidung und ging weiter in den See hinein, um sich zum Tauchen vorzubereiten.

\enquote{Warte}, sagte Chwalla. \enquote{Du kannst so noch nicht zum König gehen.}

\enquote{Soll ich etwa mit meinen Klamotten in den See?}, fragte er sie.

\enquote{Nein, das meinte ich nicht.} Sie kam ihn näher. Er sah ihr wieder in ihre wunderbaren Augen. Jetzt war sie nur noch eine Nasenspitze von ihm entfernt. Sie roch angenehm. Harry begann leicht zu schwitzen. Ihm wurde warm. \enquote{Ich meine, unser König möchte zuerst in seiner Sprache begrüßt werden, bevor er sich mit Außenstehenden unterhält.}

\enquote{Aber ich kann kein Meerisch}, antwortete Harry.

\enquote{Das ist kein Problem, ich bringe es dir bei. Bist du bereit?}

Ganz erstaunt darüber gab er reflexartig: \enquote{Ja}, zur Antwort.

Sie kam ihm noch näher, bis ihre Nasenspitze die seine berührte. Sie drehte ihren Kopf leicht zur Seite und zog seinen Kopf zu sich. Dann küsste sie ihn. Ihre Lippen fühlten sich kalt aber trocken an. Es war richtig angenehm. Sie löste sich wieder von ihm und entfernte sich etwas. Harrys Herz pochte. Sie sagte zu ihm auf Meerisch etwas und Harry antwortete ebenfalls in Meerisch. \meerisch{Siehst du, es geht doch. Komm mit.} Sie drehte sich um und tauchte unter.

Harry folgte ihr. Dank des Dianthuskrauts konnte er unter Wasser atmen. Es dauerte knappe 10~Minuten, bis sie in der Stadt unter Wasser angekommen waren. Auf den Weg dorthin dachte er an Chwallas Kuss. So etwas hatte er noch nie erlebt. Er war immer noch aufgeregt.

Sie führte ihn zum König, der sich gerade mit seiner Frau unterhielt, wie Harry feststellen konnte. \meerisch{Majestät! Ich möchte ihnen etwas vorstellen. Ein Wesen von außerhalb unseres Reiches. Er nennt sich Harry und ist ein Männchen.}

\gedanke{Etwas}, dachte Harry. \gedanke{Sie hat mich erst als sächlich bezeichnet, bevor sie mich mit Namen und dem Geschlecht vorstellte.}

Der König drehte sich zu Harry. Harry war sich nicht sicher, ob er den König ansprechen sollte, entschied sich aber, es nicht zu tun. Aus dem Augenwinkel heraus sah er Chwalla. An ihrem Ausdruck meinte er lesen zu können, dass er warten solle, bis er angesprochen wurde. Harry verbeugte sich vor dem König, um ihm seine Ehrerbietung zu zeigen. So wie er es bei Seidenschnabel gelernt hatte.

Der König schwamm auf Harry zu. \meerisch{Nun, was wünschst du von mir?}, fragte ihn der König. Harry begrüßte den König, wie ihm Chwalla sagte, zuerst in seiner Sprache. Dann erst trug er seinen Wunsch vor.

\meerisch{Majestät, ich wüsste gerne etwas über einen kleinen Würfel. Ich habe einen bekommen und mir wurde gesagt, ich solle ihn öffnen.} Er griff in seine Tasche und zog seinen Würfel heraus.

Der König bekam große Augen. \meerisch{Wachen, haltet ihn fest.} Sofort schwammen weitere Meereslebewesen heran, die Harry mit ihren Speeren in Schach hielten. \meerisch{Hofmeister! Schauen Sie in der Schatzkammer nach, ob der schwarze Würfel noch da ist.} Der Hofmeister kam kurz heran und nickte. Er verschwand. \meerisch{Du bleibst so lange hier, bis mein Hofmeister festgestellt hat, dass du den Würfel nicht gestohlen hast.}

Harry verstand und nickte. Der König nahm ihm den Würfel ab und betrachtete ihn. Kurze Zeit später kam der Hofmeister angeschwommen, den Würfel in der Hand haltend. Er übergab dem König den Würfel und wartete auf weitere Anweisungen. Der König verglich sorgsam alle 6 Seiten der beiden Würfel. Sie sahen identisch aus. Der König gab dem Hofmeister seinen Würfel wieder und gab ihm durch eine Geste zu verstehen, er möge ihn wieder zurückbringen. Die Wache verzog sich und Harry war frei. \meerisch{Komm mit}, sagte der König zu Harry und schwamm Richtung Thron.

\meerisch{Ich sehe du wurdest von Chwalla unterrichtet, mich zuerst in meiner Sprache anzusprechen. Du musst wissen, Chwalla ist meine Tochter.} Harry schluckte. Der König gab Harry seinen Würfel zurück. \meerisch{Wie man ihn öffnet weiß ich auch nicht. Aber ich kann dir sagen, dass der Würfel mit einer Farbschicht überzogen ist. Die musst du zuerst entfernen. \gst Möchtest du noch etwas wissen?}, fragte ihn der König.

\meerisch{Danke Majestät. Das ist alles. Ich hatte mir zwar mehr erhofft, aber ich bin mit dem zufrieden, was ich erfahren habe.} Er verbeugte sich wieder vor dem König und bat ihn in seiner Sprache sich entfernen zu dürfen. Der König lächelte und nickte stumm.

Harry schwamm zurück an die Oberfläche, begleitet von seiner neuen Freundin Chwalla. Mit dem Kopf über Wasser kam ihm Chwalla wieder näher und küsste ihn erneut. \enquote{Wofür war der denn?}, fragte Harry.

\enquote{Jetzt kannst du nicht mehr unsere Sprache sprechen.}

\enquote{Schade}, sagte Harry. \enquote{Ich mochte es. Kann ich diese Fähigkeit nicht behalten?}

Sie lächelte ihn an. \enquote{Gib es doch zu. Du willst nur wieder einen Kuss von mir.}

Harry war der Gedanke nicht ganz unangenehm. \enquote{Zugegeben, das auch. Aber Meerisch zu können hat schon Vorteile. Außerdem kann ich mich dann weiterhin mit eurer Spezies unterhalten.}

Sie lächelte ihn erneut an und kam ihm wieder näher. Harrys Herzschlag erhöhte sich wieder. Sie küsste ihn ein drittes Mal. Doch dieser Kuss war um einiges besser, als die beiden zuvor. Sie ließ wieder von ihm ab. \meerisch{Mach’s gut Harry}, sagte sie und begann unterzutauchen.

\meerisch{Du auch}, gab Harry zurück. Dann war sie unter der Wasseroberfläche verschwunden. Harry sah ihr noch einige Meter nach, bis er sie unter der rauen Oberfläche nicht mehr wahrnahm.

Verträumt sah er auf die Wasseroberfläche. Er nahm nicht wahr, dass sich ihm jemand näherte. \enquote{War das Bad mit deiner neuen Freundin entspannend?}

Erschrocken drehte sich Harry um und sah Professor Dumbledore, der mit seinen nackten Füßen im Wasser stand. \enquote{Meiner neuen? Nein Albus, das ist nicht meine Freundin.}

\enquote{Sah aber so aus, als ihr euch geküsst habt.}

\enquote{Oh das. Da hat sie mir Meerisch beigebracht.} Und Harry sagte Professor Dumbledore auf Meerisch etwas.

Mit erhobenen Augenbrauen und Blick auf das Wasser sagte er: \enquote{Und ich habe das umständlich lernen müssen.}

Harry musste sich ein Grinsen verkneifen.

Harry ging an Dumbledore vorbei, legte seinen Würfel auf einen Stein und fing an, sich wieder anzuziehen, denn eigenartigerweise war seine Unterhose nicht nass. Als er fertig war, hatte Dumbledore bereits den Würfel in der Hand und betrachtete ihn mit wachsendem Interesse. Er gab ihn Harry zurück.

\enquote{Wo hast du denn den her?}, fragte ihn Professor Dumbledore.

\enquote{Den haben wir von Professor Elber erhalten. Wir sollen ihn als Hausaufgabe öffnen.}

\enquote{Ein Rätsel also?}

\enquote{Ja}, antwortete Harry.

\enquote{Ich glaube, ich sollte ihn fragen, ob ich auch einen bekomme}, antwortete Dumbledore.

\enquote{Sie liegen im Klassenzimmer in einer Schachtel. Wenn uns einer kaputtgeht, sollen wir einfach einen neuen holen. Ich denke, du kannst dir einfach einen nehmen.}

\trenn

Er war noch immer mit Pansy zusammen und hatte sich mit Hermine und ihr getroffen, um Pansy zu sagen, dass Hermine es seit einer Weile wisse. Pansy machte erst ein erschrockenes Gesicht, beruhigte sich aber sehr schnell. Er war mit Pansy übereingekommen, sich in Zukunft nicht mehr zurückzuhalten. Er schmiedete mit ihr einen Plan, den er selbst Hermine nicht verraten wollte. Er sagte ihr nur: \enquote{Warte heute das Abendessen ab.}

Sie saß in der Großen Halle mit Blick zur Tür, damit sie, was auch immer kommen mochte, gleich sehen konnte. Harry bog zur genannten Zeit um die Ecke, Hand in Hand mit Pansy. Hermine lächelte leicht und blickte kurz zum Lehrertisch. Sie sah in einige große Augen, die die beiden ebenfalls entdeckt hatten. Sie blickte schnell wieder zurück und hörte auch schon die ersten Stöhner und ein plötzlich anschwellendes Getuschel in der Großen Halle. Als sie die nötige Aufmerksamkeit erreicht hatten, drehten sich beide zueinander, schlangen ihre Arme um den anderen und küssten sich innig. Sie versanken wieder ineinander. Ihre Münder klebten förmlich aneinander. Sie öffnete leicht ihren Mund und Harry spielte mit ihren Zähnen und ihrer Zunge. Er hörte ein Schnaufen und Stöhnen. Er löste sich von ihr und grinste sie an. \enquote{Bis nachher}, sagte er, gab ihr einen kleinen Kuss auf die Wange und setzt sich an seinen Platz gegenüber von Hermine. Er nahm seine Gabel in die Hand und stach zu. So als wäre nichts Besonderes gewesen, unterhielt er sich mit Hermine über seinen Lernplan für das kommende Wochenende. Er genoss die schockierten Blicke der anderen.

\enquote{Du hast gerade eben\abs}, stammelte Neville, \enquote{mit Pansy, Pansy Parkinson geknutscht. Sie ist eine Slytherin.}

Harry drehte sich um und sah Neville an. Aus seinem Augenwinkel heraus sah er viele Mitschüler nickend.

\enquote{Genau}, meinte Dean.

\enquote{Was soll schon dabei sein, seine Freundin zu küssen}, antwortete Harry.

Neville war sprachlos. \enquote{Freundin?}, fragte Dean ganz erstaunt. \enquote{Du kannst doch nicht\abs}

Harry erhob sich und sagte, zu allen an seinem Tisch gewandt: \enquote{Und was ist so schlimm daran, eine Freundin zu haben, die aus Slytherin ist? Wir lieben uns und es ist mir verdammt nochmal egal, aus welchem Haus sie stammt. Sie ist das Beste, was mir in meinen ganzen Jahren hier in Hogwarts passiert ist.} Er blickte zu ihr und sie lächelte ihn an. \enquote{Und ganz im Gegenteil, ich bin froh, dass es jemand aus Slytherin ist. Ich habe es satt: diese ständigen Vorurteile und das Gezanke zwischen unseren Häusern!} Er setzte sich wieder.

Hermine grinste ihn an und sagte dann laut: \enquote{Genauso, Harry}, und fing an zu klatschen.

Nach ein paar Sekunden des Klatschens stimmte Professor Dumbledore am Lehrertisch und einige andere Lehrer ebenfalls mit ein. Es kamen noch vereinzelte Schüler hinzu, die dem Beifall folgten.

Noch niemals hatte neben Umbridge jemand so wenig Applaus bekommen wie Harry. Aber das war ihm egal. Er hatte eine Freundin, die ihn liebte. Und das machte ihn glücklich! Aber war er das mit Pansy wirklich?

Noch am selben Tag nahm ihn sich Ginny zur Seite, um etwas mit ihm zu klären.

\enquote{Was läuft da zwischen dir und Pansy? Ist das etwas Ernstes? Ich dachte, du liebst mich?} Als sie den letzten Satz gesprochen hatte, lief sie rot an.

Harry kam zu ihr, nahm sie zunächst in seine Arme und setzte sich dann mit ihr auf seinem Schoß hin. \enquote{Ja, das tue ich, Ginny. Aber im Moment liebe ich auch Pansy. Ich weiß nicht, was werden wird. Aber das Wichtigste ist für mich, dass ich jemanden habe, mit dem ich glücklich bin.} Harry ließ bewusst offen, wen er genau damit meinte. \enquote{Einerseits bin ich glücklich, andererseits stärkt dies die Beziehungen zwischen den Häusern, aber das Wichtigste ist, dass ich weiß, dass du da sein wirst, wenn das mit Pansy vorbei ist.} Harry hatte das sehr rational zu Ginny gesagt, obwohl sehr viel Gefühl in diesen Sätzen von ihm steckten. Ebenso in seinen Gedanken, die er indirekt herüberbrachte.

Ginny gab sich damit vorläufig zufrieden. Sie würde sich Harry später noch einmal vornehmen, falls diese Geschichte mit Pansy nicht schnell genug vorüber war. Wie es allerdings in Harry aussah, konnte auch sie sich nicht vorstellen.

\trenn

\enquote{Miss Delacour kommt heute mit ihrer Schwester zu Besuch. Sie möchte etwas von Harry wissen und ihm eventuell einen Besuch abstatten}, sagte Professor McGonagall.

\enquote{Fleur Delacour? Halb Veela?}, fragte Professor Elber.

\enquote{Ja, wieso?}

\enquote{Sie darf die Schwelle zu Hogwarts nicht übertreten, solange auf Harry und Lavender noch diese Kraft wirkt. Wenn sie das tut, dann könnte sich die Wirkung vervielfachen \gst oder ins genaue Gegenteil verkehren. \gst Und \gst stellen Sie sich vor, dass ihre jüngere Schwester\abs egal was passiert, es könnte um ein Hundertfaches schlimmer werden. \gst Wann kommen sie an?}

\enquote{Sie sind schon unterwegs.}

Professor Elber drehte sich um und rannte zum Schlosseingang. Er konnte sie gerade noch abfangen und am Betreten des Schlosses hindern. Harry saß in der Großen Halle, als sie zu dritt den Saal betraten. Harry war darüber verwundert.

\enquote{'allo 'arry. Isch freue misch dich zu seh'n. Schlimm, was mit dir passiert ist. Isch 'abe misch, genau wie Gabrielle, unter Kontrolle.}

Dann küsste sie ihn zweimal auf beide Wangen. \enquote{Dein Professor hat mir Bescheid gesagt. Meine Veela-Part 'abe isch unter Kontrolle.}

\enquote{Wie?}, fragte Harry.

\enquote{Wenn isch mich konze'triere, dann kann isch meine Veela-Erbe zurück'alten. Gabrielle kann das nich' so gut, deshalb bin isch ihr immer nahe. Um sie und disch zu schü'zen.}

Ihr Gesicht kam seinem näher. Fleur schloss ihre Augen und konzentrierte sich. Dann hielt sie an und öffnete sie wieder.

\enquote{Es ist schwer. Gabrielle und isch sind noch einige Tage hier. Wir werden später wieder kommen Arry.}

Fleur küsste Harry zum Abschied auf gewohnte Weise. Er wollte zu Gabrielle schauen, die gerade seinem Gesicht nahe kam, und erwischte ihren Mund. Leicht errötend zog sie sich zurück und schaute schuldbewusst zu ihrer großen Schwester. Sie sagte zu ihr etwas, das Harry nicht verstand. Fleur kam seiner Wange nochmals näher und flüsterte ihm ins Ohr: \enquote{Gabrielle möchte sich später bei dir entschuldigen, für das, was gerade passiert ist, und dir nochmals danken, dass du sie gerettet hast. Und ich übrigens auch. Ich habe im Moment das Gefühl, dass ich dir nicht genug gedankt 'abe.} Dann fuhr sie mit ihrer Zungenspitze die inneren Ohrkonturen nach und verließ mit ihrer Schwester die Große Halle.

Ginny sah Harry eigenartig an. Er dachte, dass er einen Anflug von Eifersucht in ihren Augen erkennen konnte.

\trenn

Harry schlug Salazars Tagebuch auf, setzte sich in einen Sessel und begann zu lesen. Es war mehr eine Autobiografie mit persönlichen Gedanken, als ein Tagebuch. Es enthielt nur die wichtigsten Episoden und Gedanken aus seinem Leben.

\begin{buch}
Heute ging mal wieder alles durcheinander. Nicht nur, dass mein Bruder wieder alle nervte, nein, auch meine Schüler meinten, den Tränkeraum mit ihren stickigen Nebelwolken ausfüllen zu müssen, indem sie ihre Tränke explodieren ließen. Dabei könnte man doch meinen, sie müssten es inzwischen begriffen haben. Na ja, der schlimmste von ihnen hängt jetzt mal wieder an seinen Armgelenken im Kerker. Godric meint zwar, das sei eine viel zu schwere Strafe, aber ich sage ihm dann, dass er ein Weichei sei. Ich mag es zwar selber nicht so streng, aber dieser Peeves ist eine Plage.

Gestern hat uns unser Vater verlassen. Nach fünf Jahren, in denen er uns als Hausmeister gedient hat, ist er jetzt fort. Einerseits bin ich traurig darüber, andererseits aber auch froh.

Immer diese Muggelgeborenen. Sie haben es einfach nicht so drauf, wie jene, die aus Zaubererfamilien stammen. Manchmal glaube ich wirklich, dass es an den Genen liegt und die Konzentration der Magie in ihnen bedingt durch ihre Vorfahren nicht so stark ist, wie bei Familien, bei denen mindestens ein Mitglied magische Fähigkeiten hat. Darüber muss ich mal nachdenken. Vielleicht sollte man die Leute anhalten, nicht mehr so viel mit Muggeln zu heiraten.

Die dunklen Künste haben mich schon immer fasziniert. Nicht dass ich sie gerne anwende, oder Menschen damit schade. Es ist vielmehr, dass ich sämtliche Aspekte der Magie gerne kennenlerne und wissen möchte, was möglich ist. Schon die Wahl der Begriffe zeugt von der Engstirnigkeit der magischen Brigade. Als ob Magie eine Farbe hätte. Dabei können solche Zauber auch nützlich sein. Zumindest hat mich noch kein Argument vom Gegenteil überzeugen können.

Endlich Ferien. Zeit, mich meinen Studien zu widmen. Heute bin ich wieder einmal bei Nadine gewesen. Wenigstens sie hört mir zu. Auch wenn ich mich mit ihr nicht so recht unterhalten kann, wie mit anderen. Selbst meine Frau will nichts von den dunklen Bereichen der Magie wissen. Es schmerzt mich, dass ich selbst vor ihr all das geheim halten muss. Nadine meint, dass sie mich durchaus verstehen kann. Sie selbst gehört zu denjenigen, die nur Nachts nach draußen können, weil dann wenige Personen unterwegs sind.

Jetzt habe ich sie fertig. Meine Theorie über die magische Abstammung und Erhaltung unserer Art; wie sich unsere magischen Fähigkeiten stärken lassen können. Meine Versuche an magischen Mäusen hat es bestätigt. Die Verpaarung der besten magischen Fähigkeiten stärkt diese. Wenn wir also überleben wollen, dann müssen wir dafür sorgen, dass wenig Muggel einheiraten. Muggelgeborene Hexen und Zauberer stellen eine Bereicherung dar. Aber mit Muggeln zu heiraten, das würde uns auf Dauer nur schwächen.
\end{buch}


Harry legte das Buch weg. Er schloss kurz seine Augen und griff mit Daumen und Zeigefinger an seine Nasenwurzel. Dann stand er auf und musste sich etwas frisch machen. Nachdem er von der Toilette gekommen war, sich etwas kaltes Wasser über seine Handgelenke laufen ließ und sich mit einem kalten nassen Waschlappen den Nacken Abrieb, kehrte er zurück zu seinem Sessel, nahm das Buch und las weiter.

\begin{buch}
Nadine legte sich mir wieder um den Hals und hörte mir einfach nur zu. Mit ihrer, mit Riechknospen besetzten, Zunge fuhr sie an meiner Backe entlang. Sie beruhigte mich wieder einmal und bat mich, ihr wieder eine meiner Mäuse mitzubringen. Ich fuhr ihr über ihre trockene warme Haut und nickte.

Als ich nach meinen Mäusen sah, traf mich fast der Schlag. Alle waren magisch degeneriert. Ich untersuchte darauf hin alle, aber keine meiner Zöglinge hatte mehr magische Fähigkeiten. Die Generationen-lange Inzucht hatte ihren Tribut gefordert. Sie hatten nicht nur genetische Krankheiten, sondern auch ihre Magie wurde weniger und war verschwunden. Nadine konnte sie alle haben, nachdem ich meine Ergebnisse festgehalten hatte.

Der Rat war empört über meine Korrekturen und wusch mir gehörig den Kopf deswegen. Was mir einfiele, fragten sie mich. Ob ich von allen guten Geistern verlassen sein würde. Sie schlossen mich aus dem Verband aus. Da stand ich nun. Eine Veränderung in Gang gesetzt, die ich nicht mehr aufhalten konnte. Dazu hatte ich nicht mehr die Mittel. Der Schaden war angerichtet.

Meine Frau hatte wenig Verständnis für meine Lage. Sie hatte mich gleich zu Anfang gewarnt. Jetzt musste ich den Preis dafür zahlen. Mein Ansehen war weg. Ich brauchte ein Jahr, bis ich erkannte, dass ich so wenigstens ungeniert leben konnte. Ich hatte nichts mehr zu verlieren.

Ich kehrte also wieder nach Hogwarts zurück. Nach Jahren wieder einmal. Nur ab und an unterrichtete ich. Bei den muggel-stämmigen war ich nicht gerade beliebt, obwohl ich mich nicht anders verhalten habe, als all die anderen Lehrer. Zauberer und Hexen sind so was von nachtragend, da würde ich mir manchmal wünschen, ein Muggel zu sein. Die können scheinbar schneller verzeihen. Zum Glück war es meine Frau nicht.

Heute stellte ich meiner Frau Nadine vor. Nach und nach hatte ich ihr alle meine Geheimnisse enthüllt. Wenigstens mit ihr wollte ich meinen ganzen Frieden machen. Nun war ich bereit zu sterben. Meine Frau litt schon seit längerem an einer Krankheit, für die es keine Heilung gab. Da ich mein Schicksal mit dem ihrem verknüpft hatte, würde auch ich ihr kurze Zeit später nachfolgen.

Da stand ich jetzt, am Grab meiner Frau. In wenigen Stunden würde ich ihr folgen. Freiwillig. Ich sicherte meine Experimente, zerstörte eventuell gefährliche Sachen und verabschiedete mich von Nadine. Die Hauselfen bekamen Anweisungen, was sie zu tun hätten. Dann trat ich meinen letzten Gang an. Ein Hauself begleitete mich. Nur noch wenige Minuten trennten mich von meiner Frau. Ich legte mich neben sie in einen schlichten Holzsarg am Boden der Grube und ließ den Deckel herab schweben. Der Hauself begrub mich unter der dunklen Erde. Ich spürte Müdigkeit und glitt ins Reich der Träume\abs Ich hoffe, dass es sich so abspielt, denn während ich diese Zeilen schreibe, steht schon der Elf neben mir. Wer immer in der Lage ist, das zu lesen, möge mich hoffentlich verstehen.

Salazar Slytherin
\end{buch}

Die nächsten Tage verbrachte er immer wieder abends in Salazars Räumen, da ihn Agatha fragte, was er denn da für ein Buch lese. Als er ihr sagte, dass es um Gebärdensprache ging, bot sie ihm überraschend ihre Hilfe an, da eines ihrer Kinder nicht sprechen konnte und sie somit diese Sprache gelernt hatte. Es war zwar für die damalige Zeit ungewöhnlich, aber das hielt Agatha nicht davon ab.




\begin{kommentar}
Dieses Kapitel steht ganz im Zeichen um Harrys Anziehungskraft. Es existiert nur, weil ich etwas gesucht habe, dass Harry und Pansy für kurze Zeit zusammenbringt. Die Idee dazu kam mir, als ich eine andere Geschichte gelesen hatte.
\end{kommentar}

\begin{kommentar}
Am Ende dieses Kapitels geht Salazar seinem eigenen Tod entgegen. Für manche scheint es so, als ob er in seinem Sarg ersticken würde, aber Salazar selbst wusste genau, wann er sterben würde. Und so konnte er sich hineinlegen und den Deckel schließen, denn wenige Minuten später, die Luft hätte noch länger ausgereicht, starb er.
\end{kommentar}

\chapter{Eine neue Liebe ist wie ein neues Leben}


Diesen Abend und die darauf folgenden verbrachte er mit Lavender in einem abgelegenen Klassenzimmer. Die Hauselfen hatten ein gemütliches Bett und eine romantisch aussehende Umgebung hergerichtet. Harry hätte sich am liebsten nur ein Bett gewünscht und keine romantische Zusammenkunft. Er wollte nicht, wenn er an seine Zeit mit Lavender dachte, an diese Umgebung erinnert werden. Da es aber zwischen ihm und Lavender rein sexuell war, konnte er sich schließlich damit arrangieren.

Den anderen Schülern hatte Snape gesagt, dass sie Strafarbeiten zu leisten hätten. Bei Lavender konnte man das verstehen, so schusselig wie sie war. Und bei Harry hatte man das Gefühl, er müsste etwas ausgefressen haben und Snape behandelte ihn nur etwas härter als alle anderen.

Sie hatten bereits sechs Nächte miteinander verbracht, in denen sie jedes Mal dreimal miteinander geschlafen hatten. Und auch diesen Morgen war er alleine aufgewacht, sie war bereits gegangen. Er fühlte sich irgendwie leicht und unbeschwert. Eigentlich dachte er gar nicht mehr daran mit ihr schlafen zu wollen, aber er wollte die Woche voll bekommen und so wollte er diesen Abend auch wieder in das Klassenzimmer gehen. Außerdem wusste er nicht, ob der Anziehungseffekt noch anhielt, doch zuerst musste er noch mit Fleur und Gabrielle etwas klären.

Sie trafen sich in einem leeren, nicht gebrauchten Klassenzimmer. Fleur und Gabrielle saßen Harry gegenüber \gst ziemlich nahe. Fleur musste ein paar mal durchatmen, bevor sie sich vorbeugte und zu Harry flüsterte: \enquote{Danke 'Arry, dass du meine\abs} Sie öffnete ihre Bluse und zog sie aus. Auch ihre Schwester konnte nicht widerstehen und tat es ihrer großen Schwester gleich. Fleur stand auf und ging zum Fenster. Sie öffnete es und atmete leicht durch. \enquote{Es ist schwer 'Arry.} Er sah zu ihr. \enquote{Ich fühle mich zu dir 'ingezogen.}

Harry bemerkte nicht, dass sich Gabrielle währenddessen auf seinen Schoß gesetzt hatte.

Nur mit ihrem BH auf ihrem Oberkörper bekleidet, saß sie auf ihm und hatte beide Hände auf seiner Brust. Er war noch völlig bekleidet. \enquote{Ich\abs ich 'ätte da un'n ster'bn 'önnen, 'Arry}, sagte Gabrielle. Gedankenverloren spielte sie an seiner Schulrobe herum und öffnete seine Hemdknöpfe. \enquote{Du 'ast mein Le'bn gerett'.} Nun lagen ihre Hände auf seiner Brust. Mit ihren Fingerspitzen kraulte sie leicht seine Brust.

Harry schloss kurz seine Augen, da ihn eine Welle der Behaglichkeit durchlief.

\begin{abAchtzehn}

Fleur kam auf beide zu und stellte sich neben Harry. Ein Bein schwang sie um die Rückenlehne, um hinter ihm zu sitzen. Als Harry das bemerkte, rutschte er etwas vor und ließ sich nach hinten fallen, als sie saß. Er spürte ihre Brüste durch den dünnen BH auf seiner nackten Haut, da er inzwischen seine Robe und sein Hemd ausgezogen bekommen hatte. Er wusste selber nicht, warum er das zuließ.

Gabrielle kam seinem Gesicht näher, sodass sich ihre Nasen berührten. \enquote{Schwer unter Kontrolle\abs}, keuchte Gabrielle. \enquote{Will disch 'aben}, sagte sie mit zittriger Stimme.

Dann hörten die drei einen Knall und ihre Konzentration ließ plötzlich nach. 
\end{abAchtzehn}

%\begin{safedivide}
%\fskdivider
%\end{safedivide}

Dann geschah alles ganz schnell. Fleur verwandelte den Stuhl in einen bequemen Sessel, der nach unten nachgab und sich in eine dicke weiche Decke verwandelte. Fleur lag nun unten, Harry auf ihr und Gabrielle ganz oben. Sie fing an, ihn stürmisch zu küssen, als ihre Schwester sie harsch unterbrach. \enquote{Gabrielle, beherrsche disch.}

\begin{abAchtzehn}

Schuldbewusst zuckte sie zusammen, nickte und ließ sich seitlich von Harry heruntergleiten. Sie nahm ihn etwas mit, was Fleur die Gelegenheit gab, sich seitlich unter ihm herauszuziehen und um sich dann halb auf ihn zu legen. Nun lagen beide Mädchen halb auf ihm. Sie begannen synchron mit seinen Ohren und seinem Hals zu spielen. Sie fuhren mit ihren Zungen die Konturen seiner Ohren nach und bedeckten anschließend seinen Hals mit vielen Küssen.

\end{abAchtzehn}

%\begin{safedivide}
%\fskdivider
%\end{safedivide}

Harry ließ sich in diese Behandlung hineinfallen und verdrängte unbewusst alle anderen Gedanken. Mit geschlossenen Augen ließ er es sich gut gehen. Dann machte sich in seinem inneren ein Gefühl breit, das ihm sagte: \accentuate{Gib nach, dann hast du wieder alles unter Kontrolle.} Er gab diesem Gefühl nach und verdrängte den Rest seiner Gedanken aktiv mit Okklumentik. Immer mehr wurde ihm bewusst, dass er damit seine Gefühle komplett unter Kontrolle bringen, ja sie sogar unterdrücken konnte. Eine knappe Minute später hatte er es geschafft.

Fleur und ihre Schwester hörten abrupt mit ihren Liebkosungen auf, was Harry ein leises Seufzen entlockte. Er öffnet die Augen und hielt Fleur und Gabrielle zurück, indem er sie mit einem Arm um die Hüfte festhielt.

\enquote{Nicht! Sonst wird es peinlich. Wir haben uns lange genug so gesehen und müssen uns nicht schämen. Bleibt einfach liegen und sagt mir, was ihr loswerden wollt.}

Verlegen spielte Gabrielle mit seiner linken Brustwarze, was bei Harry einen leichten Kitzelreiz auslöste. Seine Gefühle hatte er jedoch immer noch unter Kontrolle.

\begin{abAchtzehn}
In seinem Geist hörte er Fleurs Stimme. \stimme{Wie sage ich es ihm, dass ich bis gerade eben mit ihm schlafen wollte, um ihm dafür zu danken, dass er meine Schwester gerettet hatte.} Sie blickte ihn nachdenklich an. \stimme{Eigentlich wollte ich ihm ein besonderes Geschenk machen, indem ich ihm etwas von meinem Veela-Erbe gebe, damit er sich durch Veelas nicht mehr so beeinflusst sieht, aber das war mir gerade eben noch viel zu wenig.}

Harry schaute sie überrascht an. \gedanke{Hast du mir gerade eben deine Gedanken übermittelt, Fleur?}, fragte er sich und konzentrierte sich auf beide Mädchen, da er aus seinem Augenwinkel heraus feststellte, dass ihn Gabrielle auch schon komisch ansah. Fleur wurde augenblicklich rot. Sie versuchte sich loszustrampeln, doch Harry hielt sie fest. Als sie schließlich aufgab, ließ sie sich fallen und vergrub ihren Kopf in seiner Schulter. Sanft strich er über ihr Haar. \gedanke{Scht, Fleur. Alles ist gut.} Und dann, nach einer kleinen Pause: \gedanke{Was meintest du mit: Etwas von meinem Veela-Erbe geben?}
\end{abAchtzehn}

%\begin{safedivide}
%\fskdivider
%\end{safedivide}

Dann hörte er Gabriellas Stimme in seinem Kopf. Klar und deutlich und vollkommen akzentfrei. \stimme{Weißt du, Harry}, er wandte seinen Kopf zu ihr, \stimme{Veelas können jemandem ein besonderes Geschenk machen. Sie geben einen kleinen Teil ihres Veela-Erbes an jemanden, den sie sehr mögen, um ihn vor ihrer \accentuate{Art} zu schützen. Je mehr Veelas das machen, desto besser wird der Schutz.}

Harry zog eine Augenbraue hoch, was Gabrielle veranlasst, leicht zu erröten. Jetzt hörte er wieder Fleur. Harrys Kopf drehte sich, um ihr ins Gesicht zu schauen.

Immer wieder fuhr sie dabei mit ihren Fingern über seine Lippen. \stimme{Es gibt eine spezielle Technik, die Veelas ab Geburt beherrschen. Sie nimmt die Essenz ihres Gegenübers auf und gibt gleichzeitig einen Teil davon an die zu schützende Person weiter.}

\gedanke{Was meinst du mit \accentuate{Essenz?}}

Fleur wurde wieder rot. \stimme{Wie teile ich ihm das mit, dass wir die Flüssigkeiten des anderen aufnehmen müssen, die sie bei einer Erregung bekommen. Am besten mit so viel Körperkontakt wie möglich.} \enquote{Harry}, druckste sie herum. Er zeigte keine Regung. \enquote{Lässt du dich von mir leiten? \gst Ich verspreche dir, es tut dir nicht weh.} Harry nickte nur. \enquote{Vertraust du mir?} Wieder nickte er.

Fleur stand auf und entkleidete sich komplett. Währenddessen öffnete Gabrielle Harrys Hose und zog sie herunter. Dann zog Fleur seine Unterhose herunter. Durch die Nähe des anderen und die sexuelle Spannung waren beide immer noch erregt. Fleur ging auf Harry zu und versuchte so viel Körperkontakt wie möglich herzustellen, indem sie ihn umarmte, was Harry ebenfalls machte, allerdings mit Fleur. Nachdem sie ein paar Minuten so da gestanden waren, zog Fleur mit einer Hand Harrys Vorhaut zurück und fuhr über seine Eichel. Danach leckte sie ihren feuchten Finger ab. Dann sah sie Harry bittend an. Dieser fuhr nach kurzem Zögern Fleur zwischen ihren Beinen und nahm einen Tropfen ihrer Flüssigkeit auf seinen Finger auf, den er danach ableckte. Sie standen nun wieder auf Distanz.

Dann ging ein blaues Leuchten von Fleur aus, wandelte sich in ein cyanfarbenes und begann auf Harry zuzuwandern. Auf Harry angekommen, wandelte es sich in ein goldenes Leuchten um und erhellte den Raum. Dann wurde das Leuchten von ihm aufgesogen und verschwand in Harrys Innerem.

Fleur nickte jetzt Gabrielle zu, was diese ebenfalls mit einem Nicken zurückgab. Harry sah unbehaglich zu Gabrielle. Sie stand auf und reichte Harry die Hand. Unsicher gab er sie ihr und zog sich leicht an ihrer Hand hoch.

\enquote{Ich möchte dir auch danken.} Panik machte sich in Harry breit. Gabrielle war noch extrem jung und Harry wusste nicht, wie es ablaufen sollte. Er sah einmal an ihr entlang hinunter, sie stand noch in Unterwäsche vor ihm. \enquote{Nicht so wie Fleur}, sagte sie, griff sich zwischen ihre Beine und zog ihren Finger, nachdem er nass genug war, wieder zurück und streckte ihn Harry entgegen. Dieser leckte ihn einmal ab. \enquote{Jetzt du 'Arry}, sagte sie und Harry nahm einen Finger und fuhr einmal über seine Eichel. Er hielt einen Tropfen seiner Flüssigkeit an seinem Finger und ließ ihn Gabrielle ablecken. Nun ging ein grünlicher Schimmer von ihr aus, der sich auf Harry zu bewegte und dann zu einem goldenen Ton wandelte. Dieses Mal jedoch dunkler.

Nachdem Harrys Leuchten nachgelassen hatte, fingen Fleur und Gabrielle an, sich anzuziehen.

Harry tat es ihnen gleich. \enquote{Was ist mit Ginny?}, fragte er plötzlich.

\enquote{Deine Freundin?}, fragte Fleur.

Harry war sich unsicher, was er sagen sollte, und nickte daher einfach nur.

Fleur hielt sich eine Hand vor den Mund und machte große Augen.

\enquote{Du musst sie inner'alb eines Tages k'üssen, 'Arry, sonst wird sie ausrasten und es bemerken, das ist ein großer Nachteil, der einzige.} Harrys sah nicht gerade begeistert aus.

Angezogen kam Gabrielle auf ihn zu und umarmte ihn. Sie legte ihren Kopf auf seiner Brust ab und meinte dann: \enquote{'Arry? Isch 'abe mittlerweile das Gefuhl, einen Bruder gefund' zu 'aben.}

Fleur kam ebenfalls auf Harry zu und umarmte ihn. \enquote{Mir geht es ebenso, 'Arry}, sagte sie. \enquote{Ich 'abe das Gefühl, einen kleinen Bruder zu 'aben.}

Harry dachte nach. \enquote{Ist das normal?}, fragte er nach einer Weile.

\enquote{Nein}, gab Fleur zurück. \enquote{Ich weiß nicht, was das auslöst.}

Dann hörte Harry eine Stimme: \stimme{Du hast das trimagische Turnier bestritten und beiden das Leben gerettet. Das hat eine unsichtbare Bindung geschaffen, die zusammen mit dem Veela-Erbe, das du erhalten hast, zu diesem Effekt geführt haben könnte.} Es war Salazar.

\enquote{Wir haben doch das trimagische Turnier zusammen bestritten}, sagte Harry. Fleur und Gabrielle nickte. \enquote{Ich habe euch beiden das Leben gerettet. Das könnte ein Band geschaffen haben, das mit dem Veela-Erbe, das ihr mir überlassen habt, zu diesem Effekt führt.}

\enquote{Ja, 'Arry. Du hast dort meine Schwester aus dem See gerettet und so\abs} Sie brach ab. \enquote{Wann 'ast du mir das Leben gerettet?}, fragte sie.

Harry sah sie an. Dann sagte er: \enquote{Als wir im Irrgarten waren. Du wurdest gerade unter die Hecke gezogen. Ich habe versucht, die Wurzeln von dir zu bekommen. Du warst betäubt. Also habe ich sie mit einem Zauber gekappt und rote Funken nach oben geschossen.}

Fleur bekam große Augen. \enquote{Und ich dachte erst, es wäre ein Sicherungszauber gewesen, bis mir Madame Pomfrey gesagt hat, dass ich wohl noch die Kraft gehabt hatte, dies selber zu tun. \gst Du warst das?}

Harry nickte, worauf ihm Fleur noch einen Kuss auf seine Stirn gab. \enquote{Danke, ich bin dir noch etwas schuldig.} Sie strich über ihr Haar und nahm eines davon heraus.

\enquote{Du bist mir nichts\abs}, doch Fleur legte einen Finger auf seine Lippen.

\enquote{Scht, 'Arry}, sagte sie, rollte das Haar zu einem kleinen Kreis auf, nahm seine Hand und legte das Haar in seine Handfläche. Dann drückte sie die Hand sanft zusammen und tippte mit ihrem Zauberstab einmal darauf.

Harrys Haar wurde kurz warm. Als er sie wieder öffnete, zeichnete sich eine feine, gerollte, rote Linie auf seiner Handfläche ab. Zunehmend verschwand die rote Färbung von seiner Hand.

\enquote{Ein kleines Dankeschön, 'Arry}, sagte sie.

\enquote{Was hast du mir gegeben?}, fragte er.

\enquote{Warte es ab. Du wirst es schon merken}, antwortete Fleur.

\enquote{Ich freue mich, einen großen Bruder auf Hogwarts zu 'aben}, sagte Gabrielle plötzlich und fügte in Gedanken hinzu: \stimme{wenn ich näckstes Jahr 'ier bin.}

Jetzt schaute Harry interessiert zu Gabrielle hinunter. Er ging leicht in die Hocke und hob sie hoch. Sofort schlang sie ihre Beine um seine Hüfte und legte ihre Arme um seinen Hals. So als wäre sie seine Schwester. Sie legte ihren Kopf in seine Halsbeuge und er hörte sie in seinem Geist.

\stimme{Danke, Harry. Das bedeutet mir sehr viel. Auch wenn es nur ein Jahr ist.}

\gedanke{Warum bist du überhaupt hier? Ist nicht Schule in Frankreich?}

\stimme{Wir haben ein paar Feiertage. Und da Fleur sich hier bei der Grundschule in Hogsmeade beworben hatte und heute ihr Vorstellungsgespräch hatte, wollte ich mit, damit ich mich bei dir bedanken kann.}

\enquote{Du willst nächstes Jahr mit Fleur hier in England bleiben?}

Fleur horchte auf. Dann sagte sie: \enquote{Das ist noch geheim Gabrielle.}

\enquote{Upps. \gst 'Arry wird nichts verrat'. Oder 'Arry?}

Harry lächelte sie an und schüttelte den Kopf. Dann gab er ihr einen Kuss auf die Stirn und sagte: \enquote{Nein, Schwesterchen.}

Das brachte Gabrielle dazu zu lachen und Harry einmal kurz, aber festzudrücken. Dann ließ sie von ihm ab und Harry ließ sie wieder nach unten.

Nachdem sie sich von ihm verabschiedet hatten, ging er in den Gemeinschaftsraum, wo Ginny in einem Sessel saß und mit ihren Mitschülerinnen sprach. Er ging zu ihr und sagte dann: \enquote{Ginny, kann ich dich kurz\abs ausleihen?}

Sie sah zu ihm und nickte, stand aber nicht auf, sondern wartete. Harry reichte ihr daher seine Hand, um sie hochzuziehen. Sie ergriff sie und Harry zog sie hoch. Erwartungsvoll sah sie ihn an. Harry musste schlucken. Dann legte er seine Arme um sie und zog sie zu sich heran. Bevor sie begriff was passierte, begann er sie zu küssen. Vor Schreck wich sie zurück, aber Harry zog nach. Doch nach einigen Sekunden schlang sie ihre Arme um ihn, küsste ihn zurück.

Als er wieder von ihr abließ, gab sie keuchend von sich: \enquote{Harry, was war das?}

Doch statt einer Antwort bekam sie einen weiteren Kuss. Dann sagte er: \enquote{Ich liebe dich.}

Sie bekam große Augen. Die Jubelrufe im Gemeinschaftsraum hörte sie nicht, da sie von seinen Taten und seiner Aussage überwältigt war. Er musste sie festhalten, da ihre Knie nachgaben.

\enquote{Ich muss leider heute noch meine Strafe absitzen, aber morgen haben wir alle Zeit der Welt.}

Durch diese Aussage aufgerüttelt, zog sie ihn in eine ruhige Ecke, drückte ihn in einen Sessel und setzte sich auf seinen Schoß.

\enquote{Ihr müsst miteinander schlafen?}, fragte sie leise in sein Ohr. Harry zuckte zusammen. \enquote{Es gibt Gerüchte, dass Lavender und du\abs na ja\abs sie hat den Zauber ausgelöst und ich habe etwas darüber gelesen. Dann dein Zustand. Und letzte Woche bist du ihr aus dem Weg gegangen. Dann habt ihr beide zeitgleich eure Strafe abzusitzen.} Harry antwortete ihr nicht. \enquote{Du kannst es mir sagen, Harry. Ich verkrafte das.}

\enquote{Ich will dich nicht anlügen, Ginny. Sagen wir so. Ab morgen bist du meine Freundin, da ich nicht vorhabe, dich zu betrügen.}

Ginny riss ihre Augen auf und wurde ganz rot. Sie wandte ihr Gesicht ab, was die anderen Gryffindors zu tuscheln oder zu Rufen veranlasste. \enquote{Küssen, küssen, küssen}, riefen sie.

Dadurch sich noch mehr schämend wendete sie ihr Gesicht abermals ab zu Harry, der die Chance ergriff und sie sofort küsste. \enquote{Hu, Harry}, schallte es wieder durch den Raum. Aber auch: \enquote{Ginny, schnapp ihn dir}, konnten sie hören. Wehmütig sah sie ihn an, nachdem sie den Kuss gelöst hatte.

\enquote{Amüsiere dich nicht zu sehr}, sagte sie.

\enquote{Keine Angst, meine liebe. Es ist nur noch heute Nacht, dann bin ich ganz dein.}

Ein strahlendes Lächeln kam über ihre Lippen und erreichte ihre Augen, die zu leuchten begannen.

\enquote{Was ist mit Pansy? Du hast öffentlich gesagt, dass du sie liebst.}

\enquote{Ja, das stimmt. Ich habe mich heute mit ihr getroffen und wir wollten\abs auf jeden Fall war plötzlich alles wieder weg. Ich hatte keine Gefühle mehr. Mitten im Kuss. Pansy schien es genauso zu gehen.}

\enquote{Nicht Parkinson?}

\enquote{Nein. Wir sollten mehr aufeinander zugehen. Damit meine ich die Häuser. Wir haben uns getrennt.}

\enquote{Dann bin ich deine Freundin?}

Harry überlegte kurz. \enquote{Wenn du es so siehst, ja.}

\enquote{Und wann wirst du mich verlassen?}

\enquote{Nie mehr, wenn ich ab Morgen\abs}

Ginny küsste ihn wieder. \enquote{Warum aber bei den anderen?}

\enquote{Na ja, bei Pansy dürfte doch mein Zustand mit Schuld dran sein. Sie schien nie wirklich etwas gegen mich zu haben. Nur durch Draco und unsere offene Feindschaft\abs} und leise, nur für Ginny fügte er hinzu: \enquote{Die sich zu etwas Neutralem gewandelt hat}, und dann wieder normal, \enquote{hat sie sich auf seine Seite gestellt, um die Haustreue zu wahren.}

\enquote{Und Luna?}

\enquote{Luna ist etwas Besonderes. Versteh mich nicht falsch. Ich hab sie immer noch gerne. Zwischen uns war das etwas, was man nicht beschreiben kann. Es ist eine Form der Liebe die\abs nicht romantisch, nicht schwesterlich und nicht familiär ist. Die Art unserer Verbindung ist viel mehr als das. Es scheint so, als ob wir bei Gefahr wie ein Wesen agieren könnten.}

\fluestern{Wie ein Wesen?} Ginny flüsterte jetzt nur noch.

Harry flüsterte zurück. \fluestern{Ja. Ich habe das bisher noch keinem gesagt, aber wir haben die Fähigkeit, uns in den anderen hineinzuversetzen.}

\fluestern{Eure Körpertausch-Aktion?}

\fluestern{Ja. Wir können uns außerdem in den Körper des anderen hineinversetzen, oder ihn beobachten. Erinnerst du dich an die Schachspiel-Runde mit Ron?} Ginny nickte. \fluestern{Da hat Luna zwei Runden mit ihm gespielt. Und eine habe ich mit Luna in meinem Geist gespielt \gst und gewonnen.} Dann küsste er sie wieder. \fluestern{Es gibt noch etwas, was du wissen solltest. Ich gehe ab und an mit Luna schwimmen.}

\fluestern{Da ist doch nichts dabei.}

\fluestern{Nackt.}

Ginny schaute ihn eine Weile lang an. \fluestern{Luna ist wohl die einzige Person neben Hermine, der ich das gestatten würde, mit dir nackt im See zu schwimmen.} Dann schaute sie ihn erschrocken an.

Harry küsste sie wieder. \fluestern{Danke}, war das Einzige, was er dazu sagte. Somit nahm er ihr die Möglichkeit, zurückzurudern und das zuletzt ausgesprochene zu relativieren.

\trenn

Lavender würde wieder auf ihn warten. Er öffnete die Tür, und da lag sie. Im selben dünnen Nachthemd, das er die vergangenen Tage schneller von ihr gerissen hatte, als er Quidditch sagen konnte. Doch heute war ihr Blick nicht so lasziv und scharf, heute war er anders. Harry sah darüber hinweg, denn er wollte ihr gegenüber nicht zugeben, dass sich etwas in ihm geändert hatte. Sie kletterte aus dem Bett und zog ihr Nachthemd aus. Harry zog seinen Morgenmantel aus. Er hatte nichts darunter. Sofort schoss das Blut durch seinen Körper und bahnte sich den Weg zu seinen Lenden.

Kurz darauf lag sie auch schon auf ihm und zog ihn zu einem langen Kuss heran. Sie ließ sich nur durchs Atmen und Reden unterbrechen. \enquote{Harry\abs du\abs hast\abs mir\abs gefehlt}, keuchte sie zwischen jedem Kuss hervor.

Noch nie hatte er ihre Lippen und Küsse als so intensiv empfunden wie heute. Vielleicht lag es auch nur daran, dass es heute ihre letzte gemeinsame Nacht sein würde. Oder es waren Nachwirkungen von Fleurs \accentuate{Behandlung}, dass sie so scharf auf ihn zu sein schien? 

\begin{abAchtzehn}
Sie richtete sich auf und ließ ihn in sich gleiten. Langsam bewegte sie ihr Becken auf und ab, lies es kreisen und wippte nach vorne und wieder zurück. Er spürte ihre Zuckungen und Muskelbewegungen und als sie sich nach vorne fallen ließ um seine Lippen zu empfangen, kam sie. Und Harry in ihr. Als sich beide beruhigt hatten und wieder zu Kräften kamen, drehte sie sich um und lag nun auf dem Rücken. Er steckte noch immer in ihr. Es brauchte eine Weile, bis er sie wieder ausfüllte, aber sie wollte ihn noch nicht verlieren. Langsam glitt er unter vielen Küssen nach unten. Zuerst um ihre Brüste mit seiner Zunge zu umspielen und mit seinem Mund daran zu saugen, dann hinunter zu ihrem Bauchnabel, bis er schließlich genau zwischen ihren Beinen seine Zunge spielen ließ.

Reflexartig schlug sie ihre Beine um ihn und vergrub ihre Hände in der Matratze. Sie wurde immer wilder und ekstatischer. Als sie endlich kam, rief sie nur seinen Namen: \enquote{Harry.}

Nachdem sie sich wieder entspannte und ihren Griff lockerte; ihre Beine öffneten sich wieder; schmeckte er ihren salzigen Saft auf seiner Zunge. Er krabbelte an ihr hoch, legte sich neben sie hin und schlief zusammen mit ihr ein.

Als er mitten in der Nacht erwachte, spürte er Lavender hinter seinem Rücken, einen Arm um ihn gelegt. Er spürte ihren Herzschlag und ihren gleichmäßigen Atem-Rhythmus an seinem Hals. \enquote{Lavender?}, fragte er vorsichtig. \enquote{Bist du wach?}

\enquote{Ja Harry}, antwortete sie.

\enquote{Weißt du, Lavender. Eigentlich war ich heute nicht mehr scharf auf dich, aber\abs}

\enquote{Du wolltest die Woche voll bekommen}, antwortete sie ihm und fiel ihm so ins Wort. Er musste schmunzeln. Sie beugte sich leicht über ihn und küsste seine Wange. \enquote{Geht mir genauso.}

\enquote{Aber, was mir sonst so durch den Kopf geht. Woher wissen wir, dass da nicht noch ein Restfunke ist, der wieder aufblüht?}

\enquote{Wie meinst du das?}

Harry drehte sich um und sah ihr in die Augen. In diesem schwachen Licht sah sie richtig schön aus. \enquote{Na ja, wenn wir die Sache jetzt beenden und es noch nicht vorbei ist, dann\abs werden wir wieder übereinander herfallen.}

Lavender schaute ihn nachdenklich an. \enquote{Was können wir dagegen tun?}

Harry hatte sich das ganze durch den Kopf gehen lassen. Er könnte Hilfe gebrauchen. \enquote{Versprich mir, dass du nicht schreist, wenn ich gleich einen Elfen rufe, um ihn zu fragen, ob er es feststellen kann?} Er konnte sich mittlerweile auf Kreacher verlassen, das wusste er nun. Lavender nickte nur stumm. \enquote{Kreacher? Kommst du? Kreacher, ich\abs wir brauchen dich.}

Es dauerte ein paar Sekunden, dann stand ein leicht verschlafener Elf mit müdem Gesichtsausdruck vor Harry und Lavender. Sie zog sich das Betttuch noch etwas weiter nach oben, um dem Elfen keinen allzu großen Einblick zu gewähren.

\enquote{Sir Harry hat gerufen?}, sagte der Elf und verbeugte sich.

\enquote{Ja Kreacher, tut mir leid dich geweckt zu haben, aber es ist wichtig.}

\enquote{Kreacher immer zu Diensten.}

\enquote{Kreacher, kannst du herausfinden, ob zwischen Lavender und mir noch ein Rest Anziehungskraft vorhanden ist? Ich nehme an, du hast von ihrem \gst unserem Unfall gehört.}

Lavender wurde rot hinter Harry. Das konnte er spüren. \gedanke{Unseren Unfall hatte er es genannt.}

\enquote{Sicher, Kreacher kann das. Sir und Madame müssen Kreacher nur die Hand geben.} Harry streckte Kreacher seine Hand hin. Lavender zögerte etwas, streckte dann aber doch ihre Hand ebenfalls dem Elfen hin. Dieser schaute die beiden lange an und meinte dann: \enquote{Es ist noch ein wenig vorhanden. Einmal sollte noch genügen.} Dann ließ er die Hände los und verbeugte sich.

\enquote{Danke Kreacher. Ich denke, einmal werden wir dich noch brauchen. Heute Nacht.} Der Elf verbeugte sich und verschwand. Harry drehte sich zu Lavender um und meinte dann: \enquote{Sieht so aus, als ob wir noch einmal müssten.}

Er wollte sie schon auf das Bett drücken, um auf sie zu klettern, als sie ihre Hand erhob und meinte. \enquote{Löffelchen-Stellung. Die hatten wir noch nie und Justin würde sie bestimmt auch gefallen.} Dann hielt sich Lavender vor Schreck die Hand vor den Mund. \enquote{Das\abs das wollte ich nicht.}

Doch Harry grinste nur. \enquote{Justin Finch-Fletchley?} Sie nickte stumm. \enquote{Er kann sich glücklich schätzen, dich zu haben.}

Erleichtert drehte sich Lavender um und schmiegte sich mit ihrem Rücken an Harry. Sie zog ein Bein an, um es ihm leichter zu machen, in sie einzudringen. Nachdem sie ein weiteres Mal dem Höhepunkt entgegensteuerten und anschließend Kreacher ihnen bestätigt hatte, dass nichts mehr zu finden sei, Harry sich bedanke und ihm eine gute Nacht wünschte, schliefen sie danach in dieser Position ein.

Am nächsten Morgen lag Lavender noch vor ihm, als er erwachte. Unbewusst fing er an, mit einer ihrer Brüste zu spielen.

\enquote{Harry?}, fragte sie plötzlich.

Er stoppte seine kreisenden Bewegungen und sagte: \enquote{Ja Lavender.}

\enquote{Lass uns ein letztes Mal miteinander schlafen.}

\enquote{Aber Lavender, wenn wir das machen, betrügen wir unsere Partner}, protestierte Harry.

Lavender fuhr mit einer Hand unter die Bettdecke und massierte sich zwischen ihren Beinen, worauf sie Wert darauf legte auch Harry zu liebkosen. \enquote{Ich sehe es nicht als Betrug an, Harry. Wenn wir jetzt aufstehen und uns anziehen würden und uns danach küssen, oder miteinander schlafen würden, dann wäre es Betrug. Jetzt ist es noch Skepsis gegenüber einem alten Elfen.}

Er spürte seine Erregung in ihm steigen. \fluestern{Lavender}, flüsterte er ihr ins Ohr. \fluestern{Was machst du nur mit mir?}

Er hörte ein leises Kichern. Sie ließ von ihm ab und setzte sich wieder auf ihn. Harry nahm noch einmal Lavenders Brüste in die Hand und küsste sie immer und immer wieder leidenschaftlich auf ihren Mund, sobald sie sich vorbeugte. Beide kamen nacheinander, Harry vor Lavender, und sackten zusammen. Harry fing sich wieder und nahm nun seine Finger zu Hilfe. Er fuhr mit einem Finger in ihre Scheide und drückte von außen mit dem Daumen auf ihr Lustknöpfchen. Er wusste nicht, woher er diese Gewissheit nahm, dass es ihr gefallen würde. Innen und außen rieb er jetzt mit seinen Fingern und verpasste Lavender so innerhalb kürzester Zeit einen zweiten und einen dritten Orgasmus. 
\end{abAchtzehn}

\begin{safedivide}
\fskdivider
\end{safedivide}

Danach schwanden ihr die Sinne. Als sie wieder erwachte, hatte Harry ihr bereits einen nassen Lappen auf die Stirn gelegt.

\enquote{Woher?}, stammelte sie.

Harry flüsterte in ihr Ohr: \enquote{Fleur \gst sie hat mir ein paar Tipps gegeben}, sagte er ausweichend.

Mit leuchtenden Augen gab sie ihm noch einen langen Abschiedskuss, bevor sich beide duschten, anzogen und schweigend in die Große Halle zum Frühstücken gingen. Sie würden weder darüber reden, noch dieses Ereignis wiederholen. Lavender war aber diejenige, die am meisten in diesen Nächten lernte, da Harry während seines Zustandes viele Erfahrungen sammeln konnte. Zwar unfreiwillig, aber dennoch\abs

\trenn

Nachdem er kräftig mit Agatha geübt hatte, war er bereit, vor Adriana zu treten und sie zu fragen, was es mit dem Buch über Dementoren zu tun habe. Er ging hoch Richtung Direktoren-Büro und stand dann vor dem Bild. Sie saß wie immer in einem Stuhl und schien zu schlafen. \enquote{Miss de Mimsy-Porpington?}, rief er dem Bild entgegen. Adriana öffnete ihre Augen und sah auf Harry herab, da sie leicht erhöht hing. Harry fing an, sie mithilfe der Gebärdensprache etwas zu fragen. Ganz erstaunt darüber, hob sie beide Augenbrauen und sah ihm zu, wie er sie etwas fragte.

Dann antwortete sie. \enquote{Das ist ganz einfach, Mister Potter\abs}

\enquote{Sie können reden?}, unterbrach er sie.

\enquote{Ja, sicher kann ich reden. Es ist nur mühselig, mit jedem zu reden. Immer wieder kamen Schüler und meinten, sich mit mir unterhalten zu müssen. Da ist mir die Idee mit der Gebärdensprache gekommen. Ich hatte sie gelernt, weil ich in meinem Freundeskreis jemanden hatte, der taub und stumm war. So schrecke ich fast alle ab. Nur Sie haben sich die Mühe gemacht und die Gebärden gelernt. Respekt. \gst Deshalb verdienen Sie es auch, dass ich mich mit Ihnen unterhalte.} Das verblüffte Harry. Er hätte nicht gedacht, dass eine Magierin zu solchen Tricks griff. Er hatte sich solche Mühe gegeben. Und jetzt brauchte er es nicht. Aber andererseits hätte er sich sonst gar nicht mit ihr unterhalten können. \enquote{Nennen Sie mich Adriana.}

\enquote{Dann nennen Sie mich aber Harry.} Adriana nickte und wartete auf Harrys Fragen. \enquote{Ich habe Ihr Buch \buchtitel{Vom Inferi zum Dementoren} gelesen. Haben Sie noch andere Informationen für mich, außer die, die in dem Buch stehen? Wissen Sie, Adriana, Dementoren haben ein gesteigertes Interesse an mir gezeigt.}

\enquote{Ach, das waren Sie?}

\enquote{Ja. Ich würde sie gerne bekämpfen.}

\enquote{Lernen Sie den Patronus-Zauber.}

\enquote{Den kann ich schon seit ein paar Jahren. Ich möchte sie aber nicht nur abwehren können. Ich möchte sie zerstören können. Mit einem Patronus soll das gehen. Aber wie?}

\enquote{Bauen Sie eine Verbindung zu Ihrem Patronus auf.} Sie hörte Geräusche. Eine Gruppe von Personen kam ihnen näher. \enquote{Lesen Sie in der Bibliothek darüber. Gehen Sie. Wenn Sie wieder eine Frage haben, dann wissen Sie ja, wo Sie mich finden.}

Harry nickte und bedankte sich. Dann lief er in die andere Richtung davon, weg von den Geräuschen. Darüber musste er erst einmal nachdenken und setzte sich in eine ruhige Ecke im Schloss. Er griff in seine Tasche und holte seinen Würfel heraus.

Er zog seinen Zauberstab und richtete ihn auf seinen Würfel. Er dachte nach, bevor er den Zauber sprach, der die Farbe vom Würfel entfernen sollte. Er konzentrierte sich und schwang seinen Zauberstab. Da sie das ganze Jahr über ungesagte Zauber bei Professor Flitwick anwandten, war dies eine prima Gelegenheit seine Fähigkeiten zu verbessern.

Die Farbe bröckelte ab und darunter kam der Würfel zum Vorschein. Es schien, dass im Inneren des Würfels ein kleines blaues Licht durch die diffuse Oberfläche leuchtete. An den Kanten des Würfels waren schmale, angelaufene Messingstreifen zu sehen, zusätzlich war auf jeder Seite ein auf der Spitze stehendes Quadrat ebenfalls aus Messingstreifen, und in deren Mitte ein Kreis, ebenfalls aus schmalen Messingstreifen.

Harry betrachtete gebannt den schimmernden Würfel. Er hatte die Farbschicht erfolgreich entfernt, aber noch immer keine Ahnung wie er ihn öffnen konnte. Er drehte ihn zwischen seine Hände hin und her, in der Hoffnung irgendwo einen Hinweis darauf zu erhalten, wie er sich öffnen ließe. Als ihm nichts einfiel, betrachtete er ihn näher.

Die auf der Spitze stehenden Quadrate waren aus zwei dünneren Streifen aufgebaut. Es schien so, als ob man die pyramidenförmigen Ecken des Würfels abnehmen konnte, doch sie bewegten sich nicht. Er legte ihn zurück auf seinen Nachttisch und krabbelte in sein Bett. Seine Zimmergenossen waren noch im Gemeinschaftsraum und unterhielten sich oder spielten. Doch Harry musste etwas alleine sein. Er brauchte nach der Aufregung heute Morgen und der letzten Woche etwas Ruhe. Er drehte sich um und schlief ein.

Harry konnte nach diesem Erlebnis am See, das schon einige Tage zurücklag, keinen klaren Gedanken mehr fassen.

\begin{rueckblick}
Chwalla hatte ihn in seinen Bann gezogen. Er konnte sich bei Flitwick heute nicht konzentrieren. Immer wieder wurde er von ihm ermahnt, was schließlich dazu führte, dass er für den Rest der Stunde als Übungspartner, oder besser gesagt als Versuchskaninchen, herhalten musste. Er musste sich verteidigen und Professor Flitwicks Zauber abwehren. Harry war schon am Ende der Stunde körperlich erschöpft und hatte noch Tränke bei Snape vor sich. Heute sollten sie den Trank der lebenden Toten zubereiten. Ein sehr komplizierter Trank. Harry hatte keinen Erfolg. Sein Trank war eher eine schwarze zähflüssige Masse, die so aussah und so roch, als bestünde sie aus Teer, daher musste er sich wieder Professor Snapes Spott anhören. Besonders Malfoy lachte über ihn. Und auch bei Professor Sprout war es nicht besser. Die fleischfressenden Pflanzen, um die sie sich heute wieder einmal kümmern mussten, bissen sich heute in Harry fest. Und nicht nur im Finger! Es schien so, als ob sie an ihm besonderen Gefallen fanden. Sie bissen ihn an allen möglichen Stellen, sodass er in den Krankenflügel gebracht wurde. Madame Pomfrey war darüber natürlich nicht gerade erfreut. Aber sie behandelte ihn, so wie sie alle Schüler behandelte, voller Hingabe an ihren Job. Und dieses Mal besonders gerne. Sie bemutterte ihn richtig!
\end{rueckblick}

\onelineback % Anderenfalls werden 2 Leerzeilen gesetzt
\trenn

Ron kam ganz aufgeregt in den Gemeinschaftsraum. \enquote{Harry, schon mal auf deinen Stundenplan geschaut?}

Harry schaute ihn erstaunt an. \enquote{Nein, wieso?} Ron zeigte ihm seinen Stundenplan. Das Fach \VgddK war rot gefärbt. Also drehte Harry den Stundenplan um und las.

\begin{brief}
Bringen Sie zu Ihrem nächsten Unterrichtstermin Badesachen mit. Wir treffen uns am See.
\end{brief}

Harry schaute Ron erstaunt an.

\enquote{Ron, Harry}, kam Hermine aufgeregt in den Gemeinschaftsraum. Sofort musste er grinsen. Er stellte sich Hermine gerade in Badesachen vor. Glücklicherweise war dieses Wochenende ein Hogsmeade-Wochenende.

\enquote{Gehen wir dir 'nen Badeanzug kaufen?} Ron konnte sich diese Frage nicht verkneifen.

Hermine schaute ihn böse an und stupste ihn in die Seite. \enquote{Nein Ron, aber hast du eine Badehose?} Ron schaute Hermine mit einem leichten Anflug von Panik an. \enquote{Dann gehen wir gleich nach dem Frühstück nach Hogsmeade und kaufen dir eine}, schlug Hermine vor und zog ihn Richtung Ausgang. Harry hatte bereits seit letztem Jahr eine Badehose, da er ab und an im See schwamm. Er traf auch immer mal wieder auf Luna, die aber immer nackt in den See ging. Seit er mit ihr zusammen war, ging auch er, sofern sie alleine waren, ohne seine Badehose in den See, um mit ihr zu schwimmen. Auch nach ihrer Trennung.

Er war gespannt, was ihn in dieser Stunde erwartete. Doch zunächst war er erst einmal hungrig und folgte Ron und Hermine in die Große Halle.

So einfach wie sich die drei das vorgestellt hatten, für Ron eine Badehose zu kaufen, war es nicht. Der Laden wimmelte nur so von Schülern, die noch Badesachen kaufen mussten. Schließlich fand sich für Ron eine schwarze Badehose mit gelben Streifen und einem grünen Kleeblatt darauf. Harry und Hermine tauschten vielsagende Blicke, als sie zur Kasse liefen. Als der Verkäufer den Preis nannte, stockte Ron. Harry öffnete seinen Geldbeutel und legte den Betrag auf den Tisch. \enquote{Alles Gute zum Geburtstag, Ron}, antworteten Harry und Hermine.

\enquote{Aber, der ist doch erst in zwei Wochen}, protestierte Ron.

\enquote{Dann lassen wir sie eben einpacken, und du kannst erst in zwei Wochen den Unterricht besuchen}, schloss Hermine pragmatisch.

Ron verzog seinen Mund, sagte dann aber glücklich: \enquote{Danke} und umarmte die beiden.

\trenn

Heute fingen sie bei Professor McGonagall mit der Transformation von Menschen an. Ein anspruchsvolles Thema. Harry und Neville taten sich in einem Team zusammen, als es darum ging, Körperteile seines Partners zu verwandeln. Sie mussten sich klar und deutlich einen Affen vorstellen und dem Partner einen Schwanz verpassen. Neville fing an und stellte sich ein Pin\-sel\-ohr\-äff\-chen vor. Er schwang seinen Zauberstab und aus Harrys Hintern wuchs ein Affenschwanz heraus. Harry hatte das Gefühl, ihn wirklich bewegen zu können. Interessiert betrachtete er ihn. Dann war Harry an der Reihe. Er stellte sich seinen Affen vor und wollte gerade seinen Zauberstab schwingen, als ihm wieder Chwalla einfiel. Er führte seine Bewegung aus, als er merkte, dass etwas schieflief. Neville bekam große Augen. Ihm begannen Kiemen zu wachsen. Er schnappte nach Luft. Plötzlich tauchte aus dem Nichts ein großer Eimer mit Wasser zwischen Harry und Neville auf. Er steckte sofort seinen Kopf ins Wasser, um atmen zu können. Harry lief es kalt den Rücken herunter.

\enquote{Woran haben Sie gedacht Mister Potter?}, fuhr ihn Professor McGonagall an. Völlig verstört, drehte er sich ihr zu und sah sie an. \enquote{Woran haben Sie gedacht Mister Potter}, fuhr ihn Professor McGonagall erneut und dieses Mal mit Nachdruck an.

\enquote{Äh, an\abs an\abs Wasserlebewesen. Solche wie im See leben.}

Sie schwang ihren Zauberstab auf Neville gerichtet und zog seinen Kopf aus dem Eimer heraus, welcher darauf hin verschwand. Nevilles Haare waren nass und Professor McGonagall reichte ihm ein herbeigezaubertes Handtuch. Nachdem er seine Haare abgetrocknet hatte, versuchte Harry es erneut. Er konzentrierte sich wieder und dieses Mal klappte es. Aus Nevilles Hintern wuchs nun ebenfalls ein Schwanz heraus. \enquote{Tut mir leid wegen vorhin, Neville. War keine Absicht. Ich bin die letzten Tage nur etwas abgelenkt und bekomme einen Gedanken nicht mehr aus dem Kopf.}

\enquote{Ich verstehe das, Harry. Bei mir ging schon viel mehr schief. Mir geht es gut.} Nachdem jetzt alle durch waren, ging es daran, die Schwänze wieder zu entfernen. Dieses Mal gab es keinen Zwischenfall, sodass Neville und Harry ihre Schwänze wieder verloren. Die hinteren.

Weiter ging es mit den Nasen, die jetzt in einen Rüssel verwandelt werden sollten.

Etwas später saßen die drei im Zaubergetränkekeller und warteten, dass Professor Snape sein Büro verließ und den Unterricht begann. Doch leider betrat Professor Elber den Keller und meinte: \enquote{Schlagt eure Bücher auf Seite 320 auf und mischt euren Trank an.} Er betrat Snapes Büro und kam kurz darauf mit einer Schachtel herein. \enquote{Die fehlenden Zutaten finden sich hier. Fangen Sie an.} Er stellte die Schachtel auf einen kleinen Tisch und setzte sich in den Stuhl hinter dem Schreibtisch, in dem normalerweise Snape saß.

\enquote{Professor, was ist mit Professor Snape?}, fragte Draco Malfoy.

\enquote{Der hat momentan keine Zeit, vielleicht kommt er später kurz vorbei. Ist aber eher unwahrscheinlich}, sagte er kurz angebunden.

Also schlug Harry sein Buch auf und fing mit Neville zusammen an, seinen Trank zu brauen. Ab und an sah er zu seinem Professor auf. Doch immer, wenn er aufsah, dann sah er in ein missmutiges Gesicht. Es schien, dass Professor Elber absolut keine Lust hatte, Professor Snape zu vertreten. Er stand auf und lief durch den Raum, um seinen Schützlingen bei eventuellen Problemen helfen zu können. Leider hellte das seine Stimmung nicht gerade auf.

Plötzlich gab es eine Explosion und schwarzer Rauch erfüllte augenblicklich den ganzen Raum. Er biss in den Augen und im Mund. Doch wenige Sekunden später war der Rauch verschwunden und Professor Elber hatte seinen Zauberstab in der Hand. \enquote{Unfähige Bande. Können Sie nicht besser aufpassen?}, brüllte Professor Elber durch den Raum. Keiner hatte mitbekommen, wie Professor McGonagall und Professor Snape im Türrahmen standen. \enquote{Hat euch Professor Snape nicht beigebracht, besser aufzupassen? Jetzt verstehe ich ihn! Da redet er mehr als fünf Jahre auf euch ein und noch immer passiert so etwas.} Er war richtig in Rage.

\enquote{Professor}, kam es von Lavender. \enquote{Liana liegt am Boden.} Professor Elber drehte sich zu ihr und sah den Kessel immer noch rauchen. Sofort rannte er zur verletzten Schülerin und fühlte ihren Puls. Schlagartig wurde er bleich und ließ seinen Zauberstab über ihr schweben. Danach sprach er einen Zauber, den Harry nicht verstehen konnte. Er konnte zusehen, wie sie innerhalb zweier Sekunden komplett zu Stein wurde. McGonagall und Snape waren zu geschockt über Elbers Reaktion, um etwas zu tun.

Professor Elber stand auf und griff Lavender an beide Schultern. \enquote{Was hattet ihr in diesen Trank getan?} Seine Stimme war zittrig, aber dennoch verärgert und lauter als normal.

\enquote{Wir\abs ich\abs haben\abs habe\abs} dann zeigte sie stumm auf das Kraut, welches neben dem Kessel lag. Professor Elber nahm mit einer Hand das Blatt und sah es sich an.

\enquote{Woher habt ihr das?}, fragte er jetzt zorniger.

\enquote{Frederick}, mischte sich jetzt Professor McGonagall ein. \enquote{Das arme Mädchen.}

Er drehte sich um. Professor Snape sah unterdessen nach der Schülerin.

\enquote{Das arme Mädchen hätte sich beinahe umgebracht, weil sie ein falsches Kraut verwendet hatte}, machte er weiter. Er ignorierte Professor McGonagall wieder und wandte sich wieder zu Lavender. \enquote{Woher habt ihr das?} Er wedelte mit dem Blatt vor ihrer Nase herum.

\enquote{Aus einem Laden aus der Winkelgasse.}

Professor Elber sah erstaunt auf. \enquote{Keiner. Ich wiederhole, keiner verwendet mir dieses Kraut noch. Werft eure Tränke weg. Zerstört sie, sofort. Ich gehe kein weiteres Risiko ein. Das hier ist ein gefährliches Kraut. Es ist dem Plautus-Blatt sehr ähnlich, hat aber eine verehrende Wirkung. Ich kann es nicht fassen, dass der Händler \gst ich hoffe, ihr habt es im üblichen Laden gekauft \gst euch das falsche gegeben hat.}

Lavender bekam große, leicht glasige Augen. \enquote{Ich werde mich später für mein Verhalten bei Ihnen entschuldigen, Miss. Aber jetzt sagen Sie mir bitte, woher genau Sie dieses Blatt haben!} Lavender erkläre nun, woher sie das Blatt hatte. \enquote{Wir müssen den Händler anschreiben. Er soll überprüfen, ob es noch weitere Verwechselungen gab. Die Stunde ist beendet. Bitte bringen Sie Ihre Mitschülerin in den Krankenflügel.}

Dann wandte er sich zu Professor Snape. \enquote{Severus. Ich brauche von Ihnen einen Trank um Liana zu helfen. Noch kann man sie zurückholen. Es bleiben aber nicht mehr viele Sekunden übrig. Deswegen habe ich sie in eine schützende Steinhülle gelegt. Aber der Trank ist kompliziert und ich bin nicht in der Lage, ihn zu brauen. Ich werde Ihnen das Rezept gleich zukommen lassen.}

Danach drehte er seinen Zauberstab in seiner Hand und sah im Raum umher. Als er wieder zu Professor Snape blickte, nickte dieser nur. Professor Elber nickte zurück.

Dann drehte er sich zu Lavender und nahm sie an ihrer Schulter. \enquote{Würden Sie kurz mit nach draußen kommen?} Sie nickte und verließ zusammen mit ihrem Professor den Raum.

Kurz darauf schwebte ein Buch in den Raum und blieb genau vor Professor Snape in der Luft stehen. Es drehte sich noch leicht und klappte danach auf einer bestimmten Seite auf. Professor Snape sah sich das Rezept an und verschwand dann in seinem Büro.

\enquote{Die Stunde ist beendet}, schaltete sich Professor McGonagall jetzt ein. Die Schüler wurden aus ihrem Schock-Zustand gerissen und packten ihre Taschen zusammen.

\enquote{Hast du einen Moment Zeit Harry?}, fragte Dumbledore, der gerade den Raum betrat.

\enquote{Ich habe Hermine versprochen, mit ihr Hausaufgaben zu machen. Das einzige Fach, bei dem sie Hilfe von mir möchte. Geht es noch in einer Stunde, oder soll es gleich sein?}

\enquote{So dringend ist das nicht, Harry. Mach erst deine Aufgaben.}


\trenn

\enquote{Harry, du wunderst dich sicherlich darüber, dass du wieder hier bist.}

\enquote{Ach weißt du Albus, ich\abs äh Professor\abs}

\enquote{Albus ist in Ordnung. Minerva weiß Bescheid.}

\enquote{Ich wundere mich mittlerweile über gar nichts mehr}, meinte McGonagall.

\enquote{Wie geht es mit Professor Elber voran?}, fragte Albus.

\enquote{Sehr gut. Manchmal weiß ich nicht genau, was er von uns will, wenn wir Unterricht haben. Aber auch wenn ich alleine mit ihm unterwegs bin. Aber ich frage mittlerweile gar nicht mehr nach. Er erklärt es mir, wenn wir einen gewissen Stand erreicht haben. Ich habe viel gelernt.}

\enquote{Gut, sehr gut. Ich möchte dich auf eine Reise mitnehmen. Ich habe den Ort eines von Voldemorts Horkruxen erfahren.}

\enquote{Wo?}, war alles, was Harry fragte.

\enquote{Eine Höhle an einer Klippe, mit Zugang zum Meer.}

\enquote{Kenne ich}, antwortete Harry.

Minerva fiel ihre sonst so sorgsam aufgesetzte Miene ab.

\enquote{Wie können Sie diesen Ort kennen, Potter?}, fragte sie fassungslos.

\enquote{Ich habe Träume. Realistische Träume. Ich habe mich dort gesehen, wie ich mit jemand in einem Boot über einen unterirdischen See fuhr. Dann sah ich in eine Schale hinein und entdeckte ein Medaillon. Ich gab meinem \accentuate{Gefährten} den Inhalt des mit einer Flüssigkeit gefüllten Beckens, und legte das Medaillon hinein.}

\enquote{Wie können Sie es erst sehen und dann hineinlegen?}, fragte Minerva.

\enquote{Es war ein Traum Min\abs Professor McGonagall. Eine Vision. Das Medaillon war beim ersten Mal schwach und verwaschen zu erkennen. \gst Dann fuhr ich alleine über den See zurück. \gst Es hat eine Weile gedauert, aber ich habe das Medaillon erkannt. Ich habe meinen Begleiter erkannt. Und ich habe das Medaillon hier im Schloss.}

Er griff in seine Tasche und zog es in ein Taschentuch gewickelt heraus. Dann legte er es auf den Tisch und öffnete das Taschentuch.

\enquote{Mister Potter. Wie leichtsinnig sind Sie eigentlich? Auf eigene Faust ein Versteckt von Du-weißt-schon-wem\abs}

\enquote{Voldemort}, unterbrach sie Harry frech.

\enquote{\aabs Aufzusuchen und auch noch einen Horkrux herauszuholen. Das war mehr als Leichtsinnig.}

Harry sah sie an und wartete, bis sie sich wieder einigermaßen beruhigt hatte.

\enquote{Ich hätte nicht gedacht, dass Sie zu der Sorte von Menschen gehören, die erst schimpfen und sich erst dann erklären lassen, was passiert ist.}

Das hatte gesessen. Professor McGonagall sah ihn vollkommen fassungslos an.

Um die Situation zu entschärfen, sah er sie weiterhin an. \enquote{Ich war nicht dort. Ich habe lediglich davon geträumt. Das Medaillon dort ist eine Fälschung, wie ich herausgefunden habe. Dieses hier auf dem Tisch ist das Echte.}

\enquote{Aber\abs woher\abs?}

\enquote{Kreacher, kommst du?}

Und der alte Elf erschien. \enquote{Sir Harry hat gerufen?}, krächzte er.

\enquote{Sei so gut und erzähle den beiden, was du mir über das Medaillon erzählt hast.} Harry hob das Medaillon noch einmal mit dem Taschentuch vom Tisch, um es Kreacher zu zeigen.

Kreacher nickte und erzählte von Meister Regulus. Wie dieser Kreacher befahl dem Dunklen Lord auf eine Mission zu folgen. Wie er über den See fuhr und von dem schrecklichen Trank trinken musste. Er erzählte, wie er meinte sterben zu müssen. Und dass er zurückkehrte. Ein Schluck frisches Wasser, zu Hause, und der Fluch war gebrochen.

Dumbledore und McGonagall hörten aufmerksam zu. Kreacher wollte sich schon wieder dafür bestrafen, dass er es nicht fertiggebracht hatte, das Medaillon zu zerstören, aber Harry hielt ihn davon ab.

\enquote{Das bringt nichts, Harry. Solange das Medaillon nicht zerstört ist, wird er das immer wieder tun. Selbst deine Anweisungen können dies nicht verhindern.}

\enquote{Dann zerstören wir es eben.}

Dumbledore nickte. Er stand auf und öffnete eine Vitrine. Dann nahm es das Schwert von Gryffindor heraus.

\enquote{Aber Albus, ein einfaches Schwert gegen einen Horkrux?} Seine Gedanken schwammen. \enquote{Ach ja, Kobold-gearbeitet. Es nimmt auf, was es stärkt. \gst Richtig, es ist mit Basiliskengift getränkt, mit jenem Gift, mit dem ich den ersten Horkrux zerstört habe.} Dann wurde ihm etwas bewusst. \enquote{Ich sollte es öffnen, oder?}

Dumbledore nickte nur. \enquote{Außerdem ist noch etwas wichtig.} Er nahm das Medaillon vom Tisch, lief einmal außen herum und legte es auf den Boden. Dann reichte er Kreacher das Schwert.

\enquote{Für Kreacher?}, fragte der Elf.

\enquote{Um das Medaillon zu zerstören, ja. Behalten, nein.}

Der Elf nickt und nahm das Schwert an sich.

\enquote{Wie läuft es ab?}, fragte Harry.

\enquote{Du befiehlst dem Medaillon, sich zu öffnen und dann sticht Kreacher zu, sobald es offen ist.}

Harry nickte. Dann sah er Kreacher an und fragte ihn: \enquote{Bereit?} Sein Elf nickte, also befahl Harry dem Medaillon sich zu öffnen: \parsel{Öffne dich.}

Lautlos öffnete sich das Medaillon und eine schwarze Wolke kehrte hervor. Eine neblige Gestalt manifestierte sich. Sie zeigte Walburga Black, Kreachers ehemalige Herrin. \enquote{Du wirst das nicht tun, Kreacher. Lass das bleiben. Das ist ein Befehl.}

Kreacher begann zu zittern. Bange Sekunden verstrichen. \enquote{Sir Harry ist jetzt mein Herr. Ihr habt mich immer gedemütigt und nicht gut behandelt.} Dann stach er zu und der Nebel verpuffte im Nichts.

Das Medaillon war durchbrochen und zerstört. Harry horchte in sich hinein, aber nichts passierte. Er konnte es immer spüren, wenn mit Voldemort etwas war. Plötzlich blitzten Orte auf, verschwommen, aber dennoch markant um sie zu erkennen, wenn man wusste, wo sich der Ort befindet.

Dumbledore entging das nicht. \enquote{Was ist los Harry?}, fragte er. Er nahm von Kreacher das Schwert entgegen und verstaute es wieder in der Vitrine.

\enquote{Alles in Ordnung, Kreacher?}, fragte Harry seinen Elf. \enquote{Sei ehrlich.}

\enquote{Etwas schlapp.}

\enquote{Dann ruhe dich etwa aus.}

Kreacher nahm dies als Einladung, noch etwas zu bleiben und legte sich auf eine herbeigezauberte Decke vor dem Kamin.

\enquote{Ich habe Orte gesehen, Albus}, antwortete Harry. \enquote{Orte, an denen Horkruxe versteckt sein könnten. Aber sie waren unscharf.}

\enquote{Arbeite weiter daran}, bat Dumbledore.

Harry dachte darüber noch eine Weile nach. Dann löste sich die Versammlung auf. Kreacher war mittlerweile eingeschlafen. Harry besah ihn lächelnd und sorgte mit einem vorsichtigen Schwebezauber dafür, dass die Decke samt Elf hinter ihm her schwebte. Er brachte ihn in die Küche, wo die Elfen gerade beisammen saßen und etwas aßen. Das Abendessen war bereits vorbei und die Küche war sauber. Als die Elfen ihn sahen, sprangen einige auf und holten ihm eine kalte Platte. Sie servierten ihm Reste des Abendessens, die nicht verwendet worden waren. Harry nahm die Speisen dankbar an, nachdem er Kreacher auf seinem Schlafplatz niedergelegt hatte. Während des Essens unterhielt er sich mit drei jungen Elfen, die er als Timmy, Tammy und Tommy kennenlernte. Sie erzählten ihm, dass sie lediglich zur Ausbildung hier seien. Die Familie, zu der sie gehörten, wollten sie nicht nennen. Harry respektierte dies und fragte nicht weiter nach. Nachdem er satt war und genug getrunken hatte, ging er zurück zum Gemeinschaftsraum der Gryffindors und machte sich für die Nacht fertig.

Heute würde Harry gut schlafen können. Er legte sich in sein Bett und schlief augenblicklich ein.




\begin{kommentar}
Der Würfel, den Harry öffnen soll, ist von Star Wars The Clone Wars übernommen. Ihr könnt ja mal nachschauen. Sucht einfach nach »Holocron«. Dann werdet ihr schon fündig.
\end{kommentar}

\chapter{Das Erwachen}


Harry erwachte auf der Krankenstation und sah in Ginnys Augen. \enquote{Was ist los?}, fragte er sie, als er merkte, dass er auf der Krankenstation lag. Er fühlte sich von ihr angezogen.

\enquote{Du bist auf der Krankenstation}, sagte Ginny und lächelte ihn an. Dieses Lächeln war eindeutig nicht schwesterlich. Es war mehr als das.

\enquote{Was ist passiert?}, wollte Harry wissen.

\enquote{Du bist wieder einmal zusammengebrochen. Du lagst seit drei Wochen bewusstlos im Bett.} Und bevor Harry noch etwas sagen konnte. \enquote{Luna geht es gut. Sie ist bereits nach drei Tagen wieder aufgewacht. Wir waren schon in Sorge, dass ihr wieder eure Körper getauscht habt. Jetzt nachdem alles in Ordnung ist zwischen euch.}

\enquote{Seit drei Wochen}, sagte Harry. \enquote{Dann habe ich nur geträumt. Pansy und ich \gst}

\enquote{Hallo Harry}, hörte er plötzlich vom Nebenbett.

Harry erschrak. Es durchfuhr ihn wie ein Blitz. \gedanke{Pansy.} Er drehte seinen Kopf in die Richtung, aus der die Stimme kam, und sah in Pansys schwarze Augen.

Sie lag im Bett neben ihm. Ihr Haar, welches sie normalerweise zu einem Pferdeschwanz zusammengebunden hatte, lag nun offen auf dem Kissen.

\enquote{Hallo Pansy}, gab er zurück. Erst dann realisierte er, dass in seinen Worten kein Hass, keine Ablehnung war.

\enquote{Bist du wieder wach?}, fragte sie ihn. \enquote{Erinnerst du dich?}

Harry war verwirrt. \enquote{Woran sollte ich mich erinnern?}, fragte Harry.

\enquote{An deine Rede in der Großen Halle, in der du allen Leuten sagtest, dass du mich, deine Freundin, lieben würdest und unsere Trennung kurz danach}, sagte Pansy.

Harry riss die Augen auf. \enquote{Das war kein Traum?}, fragte er ungläubig. Er fragte mehr sich selbst, als die anderen. Pansy schüttelte nur den Kopf. Harry ließ seinen Kopf sinken und drehte ihn so, dass er die Decke betrachtete. Ausdruckslos starrte er sie an. Er schloss die Augen und traute sich erst dann zu fragen: \enquote{Dann haben wir zusammen \gst}

\enquote{Hm?}, fragte sie zurück. \enquote{Was haben wir zusammen?}

Harry atmete tief ein. \enquote{Dann haben wir uns in der Großen Halle vor allen Leuten geküsst? Haben vor Draco und Ron geknutscht.}

Sie gab nur ein einfaches \enquote{Ja}, als Antwort.

Harry wurde rot.

\enquote{Ich bin mir sicher, ihr habt noch eine Menge zu besprechen. Ich muss jetzt los zur nächsten Stunde. Bis später Schatz}, sagte Ginny und küsste Harry flüchtig auf den Mund. Eigentlich wollte sie seine Backe erwischen, aber Harry drehte sich gedankenverloren zu ihr und so traf sie seinen Mund. Erschrocken zog sie sich zurück, aber nur so weit, dass sie ihn nicht mehr berührte. Harry musste nur eine Schnute ziehen, um ihre Lippen wieder zu spüren. Er hob seinen Kopf kurz an und küsste sie erneut. Ginny gab nach und als sie sich von ihm löste, wurde sie rot im Gesicht.

\enquote{Wenn schon, dann richtig, Ginny}, war alles, was er herausbrachte. Sie lachte und verschwand aus der Krankenstation. Er wusste nicht mehr, seit wann sie wirklich zusammen waren und ob sie sauer auf Pansy oder ihn war, als sie sich öffentlich küssten. Er wusste nur, er war mit Ginny glücklich. Aber war er das wirklich? Mit Ginny zusammen? Seit wann? Und warum konnte er sich daran nicht mehr erinnern?

Nun waren Harry und Pansy ganz alleine. Sie warf ihre Bettdecke zurück und schlüpfte in ihre Pantoffeln. Ihr Nachthemd war hellgelb mit leichten kaum merkbaren weißen Verzierungen. Sie lief zu ihm hin und setze sich auf seine Bettkante.

\enquote{Harry}, fing sie an. \enquote{Ich glaube nicht, dass ich noch etwas für dich empfinde}, sagte sie.

Harry hatte das Gefühl, sie würde es so taktvoll sagen, wie sie auch konnte, ohne seine Gefühle zu verletzen, falls er doch noch etwas für sie empfinden sollte. \enquote{Mach dir darüber keine Sorgen, Pansy. Ich habe auch kein Verlangen mit dir \gst intim zu werden.}

Sie warf ihren Kopf zurück und lachte aus vollem Herzen. Harry konnte ihren Hals sehen. Sie sah wunderhübsch aus.

\enquote{Weißt du, dass wir diese Unterhaltung schon ein paar Mal geführt haben? Immer dann, wenn du kurz aufgewacht bist!}

\enquote{Nein. Ich kann mich an die letzten Wochen kaum erinnern. Ich glaube zu wissen, dass ich mit Ginny zusammen bin. Und wir zwei haben uns getrennt, richtig?}

Pansy nickte. \enquote{Wir haben uns kurz nach unserem Outing getrennt. Am nächsten Tag warst du mit Ginny zusammen.} Harry hörte aufmerksam zu. \enquote{Aber}, sie stupst ihn auf die Brust, \enquote{dass du mit Lavender geschlafen hast, während wir noch zusammen waren, nehme ich dir übel.}

\enquote{Ich habe dich auch noch lieb}, sagte Harry. \enquote{Willst du ein Bussi? Dann ist wieder alles gut zwischen uns.}

Die Flügeltüren der Krankenstation gingen auf und Draco Malfoy trat ein. Nachdem er sie sorgsam hinter sich verschlossen hatte und sichergestellt hatte, dass sie alleine waren, meinte er nur: \enquote{Guten Morgen Pansy, Harry.}

\enquote{Guten Morgen, Draco}, hörte sich Harry sagen. Es war eigenartig, dass ihn Draco Malfoy mit seinem Vornamen ansprach. Außer im Gemeinschaftsraum der Paare war er für ihn immer nur Potter gewesen. Er blickte zwischen Pansy und Draco hin und her. Draco setzte sich ebenfalls auf Harrys Bettkante neben Pansy und nahm ihre Hand in seine. Harry warf einen Blick darauf. \enquote{Wie hat es Maria aufgefasst?}, fragte er knapp.

\enquote{Sie hat mit mir Schluss gemacht, nachdem ich} und er stockte kurz, \enquote{Pansy mehrmals auf der Krankenstation getroffen hatte.}

Harry zog eine Augenbraue hoch.

\enquote{Nachdem sie gesehen hatte, wie Draco mich küsst}, vollende Pansy den Satz.

Harrys Augen weiteten sich ein weiteres Mal.

Draco fügte hinzu: \enquote{Pansy kam zwei Tage nach dir hier an. Ich habe sie seitdem jeden Tag besucht und vor eineinhalb Wochen oder so, da hat es gefunkt}

\enquote{Neros?}, fragte er Draco.

\enquote{Morgen, dachte ich mir}, sagte er.

Pansy sah ihn fragend an. \enquote{Was Draco?}

Die Tür öffnete sich und Madame Pomfrey kam herein. \enquote{Ah, Mister Potter. Wieder wach?}, fragte sie.

\enquote{Ja}, antwortete Harry. \enquote{Aber warum wieder?}

\enquote{Sie sind schon ein paar Mal aufgewacht und sind nach einigen Minuten wieder bewusstlos geworden.}

Leichte Panik stieg in Harry auf. \enquote{Wie kurz danach}, fragte er nach.

\enquote{So ca. eine viertel Stunde}, antwortete Madame Pomfrey.

Sie trat an sein Bett und Draco verabschiedete sich. \enquote{Muss noch Hausaufgaben machen. Und ihr beide klärt das}, sagte er, küsste Pansy auf den Mund und verschwand.

Madame Pomfrey untersuchte Harry und holte ihm danach ein Stärkungsmittel, welches er austrinken musste. Dann verschwand sie in ihrem Büro. Pansy saß noch immer auf seiner Bettkante und lächelte ihn an. Die Tür zu Madame Pomfreys Büro öffnete sich noch einmal und ihr Kopf schaute heraus. \enquote{Miss Parkinson?} Pansy drehte sich herum. \enquote{Sie können die Krankenstation verlassen.} Der Kopf zog sich zurück und die Tür wurde wieder verschlossen.

Pansy stand auf und fing an die Tasche zu öffnen, welche neben ihrem Bett stand. Sie holte eine Schulrobe, ein Hemd mit Krawatte, eine Hose, einen Schlüpfer und einen BH heraus. Dann zog sie ihr Nachthemd über den Kopf und begann sich anzuziehen.

\enquote{Äh Pansy?}, fragte Harry sie.

\enquote{Ja?}, antwortete sie ohne sich umzudrehen.

\enquote{Was sollen wir klären? Nicht dass es mir etwas ausmachen würde, aber hast du nicht etwas vergessen?}

Sie verschloss gerade ihren BH hinter ihrem Rücken und drehte sich dabei um. Sie hatte nichts als ihre Unterwäsche an. \enquote{Draco glaubt, wir sind noch zusammen. \gst Was sollte ich vergessen haben?}, fragte sie mit leicht unsicherem Gesichtsausdruck.

\enquote{Sag ihm, dass du mich die letzten eineinhalb Wochen mit ihm betrogen hast, das wird ihm gefallen und heute haben wir uns offiziell getrennt. \gst Die Vorhänge zurückzuziehen?}, meinte Harry.

Sie lachte nur. \enquote{Das stimmt, Draco wird das freuen. \gst Harry, wir haben uns nackt gesehen. Deine Nase war zwischen meinen \gst Nun ja. Ich verspüre keine Scham dir gegenüber, falls du das meinst.} Sie zog sich ihre Hose an und danach ihr Hemd samt Krawatte.

\enquote{Wie lange bin ich denn mit Ginny zusammen?}, fragte er sie.

\enquote{Etwa vier Wochen, wenn ich dein Koma mit einbeziehe.}

\gedanke{Vier Wochen}, dachte Harry nach. \enquote{Ich kann mich gar nicht daran erinnern. Ich muss Ginny fragen, wie\abs}

\enquote{Das wird ihr gar nicht gefallen. Liebst du sie überhaupt?}

\enquote{Ja. Schon länger. Ich war nur zu feige, es mir und vor allem ihr einzugestehen.}

\enquote{Dann hast du Ginny bereits betrogen, als du mit mir\abs}

Harry wurde rot. \enquote{Ich weiß nicht.}

Als sie gerade ihre Schulrobe aufnahm, öffnete sich wieder die Tür zum Krankenzimmer und Professor Dumbledore kam herein.

\enquote{Ah, Miss Parkinson, Harry}, begrüßte der Schulleiter die beiden. \enquote{Sie dürfen schon gehen, Miss Parkinson?}, fragte er Pansy.

\enquote{Ja Professor. Madame Pomfrey hatte es mir erlaubt}, sagte sie und zog sich ihre Schulrobe an.

Professor Dumbledore setze sich in der Zwischenzeit auf einen Stuhl neben Harrys Bett und sah Pansy gedankenverloren zu, wie sie ihre Schulrobe zuknöpfte.

Dann sah ihm Pansy in die Augen und blickte danach zu Harry. Sie kam auf ihn zu und gab ihm einen Kuss auf die Wange, nahe seinem Mundwinkel. \enquote{Mach’s gut, Harry. Und eine gute Genesung. Ich sage den anderen Slytherin Bescheid, dass du wach bist.} Dann sagte sie ihm leise ins Ohr: \enquote{Ich würde aber jederzeit wieder mit dir schlafen, falls du so niedergeschlagen und wir beide frei von Partnern sind.} Dann verließ sie den Krankenflügel.

Harry fragte sich, ob ihr das wieder herausgerutscht war. Mit rotem Kopf fing Harry Dumbledores Blick auf. Er konnte ihm nicht lange in die Augen schauen. Nicht nach dem, was gerade passiert war. Also legte er seinen Kopf wieder ins Kissen und entschloss sich, die Decke als passende Alternative anzunehmen. \gedanke{Hat sie das jetzt ernst gemeint? Oder war das Spaß? Sind das Nachwirkungen?} Dumbledore saß nur da und wartete. Harry würde sich schon wieder beruhigen. \enquote{Ist mit mir jetzt wieder alles in Ordnung?}, fragte Harry seinen Schulleiter.

\enquote{Ich nehme es an}, antwortete Dumbledore. \enquote{Es gibt keinerlei Interferenzen mehr und deine sexuelle Anziehungskraft auf die weibliche Bevölkerung ist auch nicht mehr vorhanden.}

\gedanke{Sexuelle Anziehungskraft}, dachte Harry erneut. Er entschloss sich, dem Schulleiter etwas zu sagen. \enquote{Albus}, fing er an.

\enquote{Ja, Harry}, antwortete Dumbledore.

\enquote{Als du mich zu Madame Pomfrey schicktest, um dieses Mittel abzuholen\abs}

Dumbledore antwortete: \enquote{Prophylaxis Reproducta finite.}

\enquote{Als du mich dorthin schicktest und Madame Pomfrey mir diesen Trank braute, sagte sie zu mir, dass es auch an ihnen nicht spurlos vorbeigeht.} Er sah nun Dumbledore direkt in die Augen. \enquote{Es muss für Professor McGonagall, Madame Pomfrey und all die anderen weiblichen Lehrer eine Qual gewesen sein, nicht über mich herzufallen}, sagte Harry.

Es hatte den Anschein, als ob Harry ins Schwarze getroffen hatte. Professor Dumbledore schien leicht verunsichert zu sein. \enquote{Na ja}, druckste er herum. \enquote{Es gab für die Lehrer \gst aber das darfst du niemandem erzählen}, Harry nickte, \enquote{einen Trank, um das Verlangen etwas zu lindern. Aber ich gebe dir recht. Es war nicht einfach für Professor McGonagall}, grinste Dumbledore.

\enquote{Albus?}, fragte Harry.

\enquote{Hmm}, gab Dumbledore zurück.

Harry schluckte. \enquote{Als ich bei Madame Pomfrey war um dieses Mittel zu holen und sie mir gestand, dass es auch an ihnen nicht spurlos vorbeiging, da kam mir der Gedanke, sie einfach zu küssen. Ich stellte mir vor, wie verlegen sie danach sein musste, oder nicht mehr von mir ablassen konnte und sich auf mich warf.} Er pausierte kurz. \enquote{Ich habe mich beherrscht, Albus. Ich habe es nicht getan, trotz des Reizes.}

Jetzt fing Dumbledore herzhaft zu lachen an. Er lachte so sehr, dass Madame Pomfrey die Tür ihres Büros öffnete, um zu sehen, was dort los sei. Harry merkte das und fing nun ebenfalls an zu lachen. Sie konnten sich einfach nicht mehr beherrschen.

\enquote{Sie scheinen sich ja gut zu amüsieren}, rief Madame Pomfrey aus ihrem Büro heraus. \enquote{Sie dürfen gehen, Mister Potter. Da es Ihnen scheinbar schon so gut geht.}

Harry und Dumbledore lachten noch eine Weile weiter, bevor Dumbledore sich zurückzog, damit Harry sich umziehen konnte. Zusammen gingen sie den Weg zur Großen Halle, um zu Abend zu essen. \gedanke{Das tat gut}, dachte Harry. \gedanke{Dumbledore so Lachen zu sehen.} Auf dem Weg dorthin sagte er noch: \enquote{Was würde eigentlich passieren\abs wäre eigentlich passiert, wenn ich mit einer Lehrerin?}

\enquote{Unter diesen Umständen gar nichts, wieso?}

\enquote{Ach, nur so ein Gedanke. Unter den Muggeln gibt es so etwas wie Schutzbefohlene. Da ist es verboten in so etwas wie eine Abhängigkeitssituation zu kommen. Welche Seite in meinem Falle allerdings die Abhängige wäre, müsste man klären.}

Jetzt wusste er, dass die kleine Schmuserei mit Professor Sinistra keine Folgen haben würde. Er setzte sich an einen freien Platz und begann zu essen. Seine Gedanken schweiften zurück und er musste schmunzeln.

\begin{rueckblick}
Harry bog um eine Ecke und stieß mit Sinistra zusammen. Reflexartig griff er zu und hatte sie um ihre Hüfte gepackt. Das machte es ihr sehr schwer, ihm zu widerstehen. Leider wurde durch seinen Zustand sein Verstand zurückgefahren und ihm entwich ein: \enquote{Aurora, tut mir leid.} Dies löste in ihr eine Aktion aus, die Harry vermeiden wollte. Sie zog ihn zu sich und küsste ihn. Mitten in der kleinen Knutscherei bekamen beide ihren vollen Verstand wieder zurück.

\enquote{Kein Wort zu niemand?}, fragte Harry.

\enquote{Richtig, Mister Potter.}

Beide ließen einander los und sammelten ihre Sachen vom Boden auf. Danach gingen sie ihre Wege und verhielten sich so, als ob das nie passiert wäre. Keine roten Backen oder schamhaftes Verhalten während der nächsten Zusammentreffen oder Unterrichtsstunden. Beide konnten dieses Ereignis gut verbergen.
\end{rueckblick}

\onelineback % Anderenfalls werden 2 Leerzeilen gesetzt
\trenn

Mitten in Snapes Unterricht brach Harry wieder zusammen. Das war bereits das dritte Mal, seit er in Dumbledores Büro umgefallen war. Er hatte seine Hausaufgaben abgegeben und war mitten im Brauen eines Trankes. Heute ging es darum, Schrumpfköpfe herzustellen. \gedanke{Ein eigenartiger Trank} dachte sich Harry. Doch viel bekam er nicht mehr mit, denn ihm wurde schwarz vor Augen. Er konnte sich nicht artikulieren und fühlte Nevilles Arm hinter seinem Rücken, als er zu Boden sackte. Dieses Mal konnte er nichts mehr hören, nur fühlen. Er fühlte, wie er auf den Boden gelegt wurde und danach schwebte. Er bekam jede Lageveränderung mit, bis er schließlich nichts mehr fühlte und wegdämmerte.

Neville kümmerte sich weiter um ihren gemeinsamen Trank, während Harry einfach, aber gut gepolstert, in einer Ecke liegen blieb. Madame Pomfrey konnte ihm nicht helfen, da sich die Magie mit ihm verbinden musste. Ein natürlicher Prozess.

Als er wieder erwachte, war die Stunde schon fast um. Snape bewertete Harrys Arbeit gar nicht und die von Neville konnte er nur schwer einschätzen. Aber da der Trank gut geworden war, vermied er es, sich mal wieder auszulassen. Er gab einfach keine Punkte.

Nach dem Ende der Stunde ging es Richtung See.

Professor Elber stand bereits mit den Füßen im Wasser als die Klasse ankam. Er hatte eine kurze Hose und ein kurzärmeliges Hemd an. Die gesamte Klasse baute sich am Rande des Sees auf und wartete gespannt, denn ihr Erscheinen musste ihm aufgefallen sein.

Die Wasseroberfläche bewegte sich und ein Meereslebewesen tauchte auf. Es war wieder Chwalla. Professor Elber begrüßte sie auf Meerisch, was Harry ohne Probleme verstand. \meerisch{Ich grüße euch, junge Königliche Hoheit.}

\meerisch{Ganz meinerseits}, gab sie zurück.

\meerisch{Wir sind vollständig anwesend, denke ich. Sie können sie einweisen}, redete Professor Elber weiter.

\meerisch{Gut, danke Professor.} Sie wandte sich der Gruppe von Schülern zu und begann in deren Sprache zu reden. \enquote{Ich freue mich, dass Sie sich heute einer großen Aufgabe widmen werden. Ich werde sie Ihnen erklären. Nachdem Sie Ihre Badesachen angezogen haben, werden Sie in kleinen Gruppen zu vier oder fünf versuchen in eine unserer bewachten Einrichtungen zu gelangen und den dort befindlichen Gegenstand zu entführen. Die Wachen wissen, dass sie in Kürze von Hexen und Zauberern besucht werden, die versuchen werden sie auszurauben. Für diese Aufgabe haben Sie vierzig Minuten Zeit. Zusätzlich fallen je zehn Minuten Schwimmweg an.}

Die Wasseroberfläche bewegte sich wieder und mehrere Meereslebewesen erschienen an der Wasseroberfläche. In ihren Händen hielten sie Netze, welche grüne algenartige Wasserpflanzen enthielten. Harry dachte, Dianthuskraut zu erkennen. \gedanke{Das konnte was werden, schon wieder Dianthuskraut.} Er hatte die letzten beiden Male noch gut im Gedächtnis. Besonders angenehm war das nicht.

\enquote{Worauf warten Sie? Ziehen Sie sich endlich um}, ermahnte Professor Elber die Klasse.

Die Meereslebewesen mit den Netzen kamen dem Ufer näher.

\enquote{Und wo sollen wir uns umziehen?}, fragte Susan Bones.

Währenddessen knöpfte Luna bereits ihre Schulrobe auf. Darunter sah man eine normale schwarze Hose, eine weiße Bluse und die übliche Ravenclaw-Krawatte. Sie löste den Krawattenknoten und zog sie ab. Dann nahm sie ihren Zauberstab und zauberte um sich herum eine kleine Bretterkabine, in der sie sich umzog.

\enquote{Nehmen Sie sich ein Beispiel an Ihrer Schulkameradin, sie hat es ziemlich schnell begriffen.}

Die Meereslebewesen mit den Netzen waren nun am Ufer angekommen und standen auf. Sie standen noch mit den Füßen im Wasser. Eine seltsame blaue Wasserblase befand sich um ihren Hals herum und lief mit einem dicken Schlauch den Rücken entlang herunter bis zur Wasseroberfläche. Damit mussten sie wohl atmen, denn es dauerte, bis alle fertig waren. Chwalla tauchte mehrmals kurz unter. Harry hatte bereits heute Morgen seine Badehose angezogen und konnte daher ohne Umkleide seine Schulrobe ausziehen. Professor Elber zog unterdessen seinen Zauberstab heraus und verzauberte eine kleine Einstiegsschneise in den See hinein. Wasserlianen kamen an die Oberfläche und bildeten so praktische Stufen. Die Wasseroberfläche färbte sich sehr leicht ins Rötliche.

Die Pflanzenträger öffneten ihre Netze und legten kleine Portionen von Dianthuskraut auf die flachen Steine, welche aus dem Wasser ragten. Eine gaben sie dem Professor. Danach verschwanden sie mit den leeren Netzen wieder unter Wasser. Es dauerte noch wenige Minuten, bis sich die Klasse restlos umgezogen hatte. Die Umkleiden verschwanden nacheinander wieder und alle standen in Badesachen da.

Professor Elber hob die Hand, in der er das Dianthuskraut hielt und fing an. \enquote{Sie sehen hier in meiner Hand, und ebenso auf den flachen Steinen dort, Dianthuskraut. Sobald Sie diese Pflanze geschluckt haben, sind Sie in der Lage eine Stunde lang unter Wasser zu atmen. Ihre Aufgabe ist klar. Schwimmen Sie zu den Ihnen zugewiesenen Stellen, die Ihnen Chwalla zeigen wird.} Chwalla schwamm näher an das Ufer heran. \enquote{Spätestens in einer Stunde müssen Sie zurück sein.}

\meerisch{Haben Sie Ihre Gruppen schon gebildet?} Die andern hielten sich ihre Ohren zu, denn Meerisch über Wasser war nicht zu ertragen. \enquote{Oh, Verzeichnung. Ich meinte natürlich: Haben Sie Ihre Gruppen schon gebildet?}, fragte sie.

\enquote{Ja}, antwortete die Klasse.

\enquote{Gut, dann finden Sie sich bitte zusammen, damit der Professor Ihnen entsprechend farbige Bändchen geben kann.}

Professor Elber zog wieder seinen Zauberstab und wartete, bis sich die Gruppen deutlich herauskristallisiert hatten. Er schwang seinen Zauberstab und die einzelnen Gruppen färbten sich.

\enquote{Was ist mit meiner Haut passiert?}, kamen schreie aus der Klasse, denn es hatten sich keine Bändchen an den Handgelenken gebildet. Die Hautfarbe der Gruppen hatte sich ins Blaue, ins Grüne oder anderen Farben verwandelt.

\enquote{Ups}, gab Professor Elber mit leichtem Lächeln zurück. \enquote{Bändchen sieht man nicht so gut unter Wasser. So ist es besser. War aber trotzdem nicht beabsichtigt. Manchmal macht die Magie Sachen, die praktischer sind, wenn man es nicht so genau nimmt.}

Leicht pikiert lief die Klasse nun ins Wasser und nahm sich eine Portion Dianthuskraut.

\enquote{Aber Professor}, beschwerte sich ein Schüler. \enquote{Wir können uns unter Wasser gar nicht unterhalten.}

\enquote{Das ist richtig}, entgegnete ihm Professor Elber. \enquote{Sie werden sich unter Wasser verständigen müssen. Es ist eine Prüfung, auf die sie sich nicht vorbereiten können. Viel Spaß.}

Chwalla erklärte ihnen, wohin sie schwimmen mussten, nachdem sie sich getrennt hatten. Dann tauchte sie unter und wartete. Jeder nahm das Kraut in den Mund und tauchte Sekunden später freiwillig ins Wasser, nachdem sich Kiemen bildeten und die Schüler über Wasser nicht mehr atmen konnten.

Chwalla schwamm voran. Unter Wasser war alles leicht bläulich. Je tiefer man kam, desto weniger sah man, doch die Wachen, welche Chwalla begleiteten, hatten kleine Leuchtquellen dabei, damit sie besser sehen konnten. Die Meereslebewesen brauchten sie nicht, sie sahen auch so genug. Es ging immer weiter hinunter in das Dunkle. Nach endlosen Minuten hielt Chwalla an und drehte sich um. Die Wächter schwammen näher und gaben einem Mitglied aus der Gruppe die Leuchtquelle. Es war eine kleine Kugel, die ein behagliches Licht ausströmte. Chwalla nahm von einer der Wachen einen Dreizack entgegen und stieß ihn in die Höhe. Kleine verschiedenfarbige Kugeln kamen aus der Spitze und breiten sich zu einer Straße aus, der man folgen musste. Jede Gruppe hatte eine andere Straße aus leuchtenden Punkten, denen sie folgen mussten. Das Spiel konnte also beginnen.

Neville wurde in Harrys Gruppe nun als Anführer bestimmt. Sie mussten sich mithilfe einer Zeichensprache verständigen, die sie vorher vereinbart hatten. Es waren nur wenige Zeichen, denn die Zeit reichte für mehr nicht aus. Dann folgten sie der Spur aus Licht, allen voran Neville. Als sie ihrem Ziel Nähe kamen, waren dort überall Lichter, damit sie besser sehen konnten und die Wachen nicht wussten, wann sie angreifen würden. Diesen Vorteil würden sie bei einem realen Angriff nicht haben. Neville schwamm hinter einen Busch, um sich in dessen Deckung heranzuschleichen. Die anderen folgten ihm.

Es musste genau geplant werden. Sie waren zu fünft, konnten aber nur vier Wachen zählen, also musste einer ins Innere schwimmen, während die anderen die Wachen ablenkten.

Nachdem der Weg frei war, schwamm Harry in das Innere, doch das war gar nicht so einfach, zuerst musste er nach unten schwimmen, und dort erwartete ihn dann ein Tintenfisch. Noch schlief er, aber seine Arme waren ihm im Weg. Er konnte unter Wasser keinen Zauberspruch sagen, also musste er sich konzentrieren. Er nahm seinen Zauberstab, zeigte auf die Arme des Fisches und ohne ein Wort zu sagen, bewegten sich die Arme des Tintenfisches vom Gang, den sie versperrten, weg. Dahinter konnte er in einigen Metern Entfernung eine Tür erkennen, doch er konnte nicht mehr dorthin schwimmen, denn der Tintenfisch wurde wach und griff mit seinen Armen nach ihm. In Panik schwamm er zurück und der Tintenfisch verfolgte ihn. Am Eingang des Versteckes angekommen, warteten schon die anderen auf ihn. In Schach gehalten von den Wachen. Sie hatten verloren und keine Zeit mehr für einen erneuten Angriff. Die Wache gab ihnen zu verstehen, dass sie zurück an die Wasseroberfläche schwimmen sollten und gab ihnen eine kleine schwarze Platte mit, welche sie abgeben sollten. Dies stand auf einem geschützten Zettel, den eine Wache ihnen zeigte.

Als Harrys Gruppe wieder mit dem Kopf über Wasser atmen konnte, sahen sie bereits am Ufer Tische in den Farben ihrer Gruppen. Ebenso viele Grills mit leckeren Würsten, sowie Professor Elber, der eine Grillzange in der Hand hielt und die Würste ab und an umdrehte. Neben ihm unterhielten sich angeregt Professor Dumbledore und Professor Flitwick, außerdem stand an jedem Tisch ein kleiner Elf und schien auf irgendetwas zu warten. Harry folgte den anderen an den Rand des Sees, um ins Trockene zu gelangen.

Sofort kam der Elf heran und fragte jeden, was er auf seinen Hotdog haben möchte. \enquote{Was möchten sie auf ihre Hotdogs haben Sir? Wir haben Kraut, Zwiebeln, Senf und Ketchup. Suchen Sie sich was aus.}

Harry bestellte seinen mit Senf und wenig Zwiebeln, Neville nahm seinen mit Ketchup und die anderen aus seiner Gruppe mit allem. Der Elf poppte hinter den Grill und nahm vom Tisch daneben ein Brötchen, schnitt es auf, nahm mit einer Zange eine Wurst von Grill und tat die bestellten Sachen darauf. Dann poppte er zurück und überreichte den ersten Hotdog. Harry konnte gar nicht so schnell schauen, wie der Elf ihnen fünf das Essen brachte. Außer ihnen war anscheinend noch niemand zurück. Sie liefen auf Professor Dumbledore und Professor Flitwick zu, um sich danebenzustellen und der Unterhaltung beizuwohnen. Dabei aßen sie ihre Hotdogs.

Plötzlich hörten sie vom Wasser ein Gekeuche und Gehuste. Die nächste Gruppe war gerade aufgetaucht. Und kurz darauf die nächste. Die beiden Elfen, welche bislang hinter ihren Tischen standen, liefen auf die Gruppen zu und nahmen die Bestellungen entgegen.

Harry ging ein paar Schritte zurück, um sich an seinen Tisch zu lehnen. Von hier konnte er der Unterhaltung ebenso gut beiwohnen. Das Schwimmen und Tauchen hatte ihn hungrig gemacht. Er sah sich um, aber es gab nichts zu Trinken.

Er sah den kleinen Elfen an und fragte ihn: \enquote{Gibt es auch was zu trinken?}

\enquote{Ja Sir}, antwortete der Elf. \enquote{Im Schloss! Was möchten Sie haben, Sir?}

\enquote{Sag mir erst, wie du heißt}, sagte Harry.

\enquote{Wonkes}, antwortete der kleine Elf.

\enquote{Dann bring mir bitte ein Glas Wasser, Wonkes}, sagte Harry mit leichtem Lächeln zu dem Elfen. Der verschwand und stand keine zwei Sekunden danach wieder mit einem Glas frischem kalten Wasser in der Hand neben ihm. Er reichte es ihm und Harry bedankte sich.

\enquote{Ah, das tat gut}, sagte Harry, als er sein Glas mit einem Zug ausgetrunken hatte. Die anderen drehten sich erstaunt um und fragten Harry, wo er denn das Glas herhabe. \enquote{Von Wonkes}, sagte Harry und zeigte auf den kleinen Elfen. Die anderen bestellten gleich etwas, denn das Schwimmen hatte auch sie durstig gemacht.

Neville gab seine schwarze Platte ab und machte sich mit den anderen auf den Weg zurück ins Schloss. Morgen würden sie darüber noch diskutieren.

\trenn

Es war schon spät und Harry entspannte sich von den Strapazen seines Quidditch-Trainings. Langsam wurden die Schüler im Gemeinschaftsraum weniger. Ginny saß alleine auf einem Sofa. Die Letzte ihrer Mitschülerinnen war vor wenigen Minuten aufgestanden und zu Bett gegangen. Harry stand also auf und setzte sich neben Ginny auf das Sofa. Vorsichtig begann er.

\enquote{Ginny, Schatz?}, fragte er zögerlich.

\enquote{Ja Harry, was ist?}

\enquote{Ich\abs ich weiß nichts mehr. Ich meine, wie und wann\abs}, er schluckte einmal, \enquote{seit wann sind wir zusammen?}

Ginny sah ihn komisch und mit leicht glasigen Augen an.

\enquote{Ich meine, ich liebe dich Ginny, aber ich habe komplett vergessen, wie wir\abs} Er sah betreten zu Boden. \enquote{Ich hätte dir meine Gefühle schon viel früher offenbaren sollen, aber ich war so ein Feigling. Und als du mit anderen Jungs\abs Na ja, ich war eifersüchtig und wusste nicht, ob du mich noch magst.}

\enquote{Natürlich mag ich dich.}

Jetzt war es Harry, der sie komisch ansah. Ginny bemerkte das und küsste ihn zur Bestätigung, dass es mehr als nur mögen war. Dies nahm ihm eine schwere Last von seiner Seele.

\enquote{Ich liebe dich, Harry. Es war nur besonders schwer seit Beginn des letzten Schuljahres. Und nachdem du mich als eine Schwester bezeichnet hast, da\abs}

\enquote{Das war dumm von mir, Ginny. Ich bemerkte deine Tränen und ruderte zurück, falls du nicht mehr für mich empfinden würdest.} Auch Harrys Augen wurden langsam feucht.

Ginny bemerkte dies lächelnd und küsste ihn erneut. Dann fing sie an. \enquote{Es begann vor fünf Wochen. Du hast dich gerade von deinen Strapazen erholt und hast mir gesagt, dass du nun bereit wärst. Ich habe dich natürlich komisch angesehen. Dann hast du meine Hände genommen und mir gesagt, dass du die ganze Zeit ein Idiot warst und dir schon länger klar war, dass du mich liebst. Ich habe deinem Werben schließlich nachgegeben, nach etwa anderthalb Wochen, wollte aber wissen, warum du dann mit Luna zusammen warst. Du hast mir auch etwas über Pansy erzählt. Du hast gemeint, du brauchst noch etwas Zeit, das zu verarbeiten, dann bist du ins Koma gefallen.}

Harry nickte. \enquote{Weißt du}, begann er, \enquote{das war eher eine Art Zufall. Ich weiß nicht genau, aber für Luna empfinde ich zwar etwas, aber nicht so stark und nicht das gleiche wie für dich. Es ist so, als ob meine Gefühle für sie anderen Ursprungs sind. Es sind nicht mehr die Gefühle für einen Partner, die ich einige Wochen lang hatte. Es ist vielmehr eine Art der seelischen Verbindung.}

\enquote{Du meinst eine platonische Liebe?}

\enquote{Nein, etwas anderes. Es ist eine Art Zuneigung, die sich im Laufe der Zeit gewandelt hat zu etwas, mit dem wir beide voll und ganz zufrieden sind.} Er senkte seinen Kopf, da er Ginny nicht länger in die Augen schauen konnte. \enquote{Weißt du, es ist mir unangenehm darüber zu reden, aber\abs} Er nahm sie ganz fest in die Hand, schaute sie an und sagte dann ganz schnell, bevor er seinen Blick wieder von ihr abwendete: \enquote{Wenn du mit jemanden schläfst und sich aufgrund einer besonderen seelischen Kombination nicht nur der Körper, sondern auch der Geist verbindet, dann kann man das nicht mit etwas vergleichen. Das körperliche Verlangen ihr gegenüber ist gleich null, aber\abs} Dann wandte er seinen Kopf ab und machte weiter. \enquote{Ich hatte einfach Angst, dass du mich nicht mehr haben wolltest, dass du der Meinung seist, du könntest mir nie mehr genug geben, dass ich dich immer mit Luna vergleichen würde. Ich dachte, du fühlst dich immer in ihrem Schatten.}

Das gab Ginny erst einmal zu denken. Als sie nach minutenlangem Nachdenken immer noch keine Reaktion zeigte, verabschiedete sich Harry von ihr und ging zu Bett.

\trenn

\enquote{Sie müssen besser aufpassen, Mister Potter. Diese Klatscher können einem ganz schön zusetzen. Ich behalte Sie über Nacht hier und dann können Sie morgen wieder neuen Taten entgegensehen. \gst Aber dieses Mal etwas vorsichtiger}, ermahnte ihn Madame Pomfrey.

Harry nickte und blieb im Bett liegen.

Gegenüber des Ganges war ein Sichtschutz aufgebaut und man hörte ein leises Atmen und Wimmern.

\enquote{Alles in Ordnung?}, fragte Harry vorsichtig, nachdem Madame Pomfrey den Raum verlassen hatte, um zu Abend zu essen.

\enquote{Ja}, kam zaghaft hinter dem Sichtschutz hervor.

\enquote{Wie heißt du?}, fragte Harry.

\enquote{Alina Meyers, Slytherin}, kam schwach von der anderen Seite. \enquote{Und du?}

\enquote{Harry Potter\abs}

\enquote{Gryffindor}, antwortete sie. \enquote{Du bist im sechsten Jahr, richtig? Ich bin im Zweiten.}

\enquote{Ja!}, antwortete Harry. \enquote{Wie siehst du aus? Damit ich weiß, ob du mir schon einmal aufgefallen bist.} Er hörte ein Kichern als Antwort. \enquote{Was ist?}, fragte er nach.

\enquote{Natürlich bin ich dir aufgefallen. Ich habe dich gleich nach deiner Ankunft umgerannt. Tut mir leid.} Harry überlegte kurz und kramte in seinem Gedächtnis. \enquote{Was ist?}, fragte Alina nach einiger Zeit.

\enquote{Nichts. Ich habe nur kurz überlegt, ob ich dich noch in Erinnerung habe. Aber ich glaube, ich weiß es jetzt wieder.}

Harry hatte durch seine verstärkten Bemühungen und der Wiederaufnahme seiner Ok"-klu"-men"-tik-Übungen das Bild gefunden und sah die Szene nun wieder deutlich vor sich.

\begin{rueckblick}
\enquote{Ist das nicht toll, Harry}, fragte Hermine.

Harry wollte gerade antworten, da stürmten zwei Schülerinnen um die Ecke und warfen ihn fast um.

\enquote{Oh, Entschuldigung. War keine Absicht}, sagten die beiden Mädchen und rannten davon.
\end{rueckblick}

\enquote{Lange braune Haare, graue Augen und leichte Sommersprossen auf der Nase}, sagte Harry dem Vorhang.

\geraeusch{Ieks.} \enquote{Das weißt du noch? Wie kannst du dir das alles merken?}

\enquote{Der Mensch kann sich viel mehr merken, als es uns bewusst ist. Wir müssen nur dafür Sorge tragen, dass wir Zugriff darauf haben. Ich habe mir sehr mühsam diese Technik erarbeiten müssen. Und es klappt nicht bei jedem.}

Das Mädchen nickte, doch Harry konnte es nicht sehen. Die Tür zur Krankenstation ging auf und Professor Elber kam herein. Er legte, als er Harry sah, einen Finger auf seine Lippen und ging leise in Richtung Büro. Er klopfte kurz an und trat dann ein.

\enquote{Madame Pomfrey ist beim Essen}, antwortete Harry ihm.

Nach einer Weile kam er wieder aus ihrem Büro heraus und hatte einen kleinen Tigel in der Hand. Er sah wieder zu Harry und legte abermals einen Finger auf seine Lippen. Er wollte gerade die Krankenstation verlassen, da kam ein Wimmern und Weinen durch den Vorhang. Professor Elber blieb stehen und ging zum verhüllten Bett.

\enquote{Alles in Ordnung?}, fragte er.

\enquote{Madame Pomfrey}, kam es abgehackt hinter dem Vorhang hervor.

\enquote{Die ist beim Essen}, antwortete Professor Elber.

Es herrschte kurz Ruhe, dann kam ein: \enquote{Hilfe.}

Professor Elber legte eine Hand an den Vorhang, schien sich aber doch eines Besseren zu besinnen und fragte, die Hand immer noch am Vorhang: \enquote{Darf ich Ihnen versuchen zu helfen?}

\enquote{Ja}, kam es leise zurück.

\enquote{Dazu müsste ich aber den Vorhang zur Seite schieben.}

\enquote{Wenn es sein muss.}

Er schob den Vorhang ein paar Zentimeter zur Seite und machte erst einmal einen halben Schritt zurück \gst und wurde bleich. Er schluckte ein paar Mal und trat dann hinter den Sichtschutz, zog sich einen Hocker heran und sprach mit Alina ganz leise. Dann hörte Harry etwas.

\enquote{Harry, wenn Sie wollen und Sie es verkraften, dann möchte Alina, dass Sie zu ihr kommen}, sagte Professor Elber.

\enquote{Ich würde gerne, Professor. Aber ich soll nicht aufstehen}, antwortete Harry.

\enquote{Kein Problem}, hörte er und ehe er noch einmal blinzeln konnte, bewegte sich sein Bett auf das von Alina zu, das Bett daneben zur Seite und tauschte mit seinem den Platz.

Als Harry in ihr Sichtfeld kam, verschlug es ihm fast die Sprache. Nachdem er sich gefasst hatte, sagte er zu Alina: \enquote{Du siehst immer noch hübsch aus.}

Sie errötete leicht. Professor Elber strich ihr über die Backe und meinte: \enquote{Nicht rot werden, das hinterlässt hässliche Flecken auf der Haut.}

Ihr Gesicht war wie früher und hatte sich kaum verändert, doch die Haut war nicht mehr rosig und zart, sondern ledrig und härter. Ihre Fingernägel wurden länger und die Anordnung hatte sich leicht verändert. Sie wurden mehr und mehr zu Klauen. Dann durchlief sie ein weiterer Schub. Ihre Augen veränderten sich in ein helles Gelb und die Haare wurde grauer.

\enquote{Harry wird heute bei Ihnen bleiben und Sie trösten.}

Dankbar nickte sie ihm zu und sah danach mit einem sehnsüchtigen Lächeln zu Harry.

\enquote{Was ist mit ihr passiert?}, fragte Harry.

\enquote{So weit bin ich noch nicht gekommen}, antwortete sein Professor.

Also erzählte Alina von ihrer Übungsstunde mit ihrer Freundin Bessi. Dem Üben des Schwebezaubers und des missglückten Färbezaubers auf der Feder. Sie erzählte, dass sie sich gegenüber standen und sie vom Zauber getroffen wurde. Professor Elber runzelte die Stirn. Er überlegte.

Plötzlich schrie Alina wieder auf und fasste sich an ihren Po. \enquote{Es tut weh}, schrie sie, als aus ihrem Hintern ein Schwanz mit einer laubförmigen Spitze zu wachsen schien. Als der Schwanz voll ausgebildet war, begann sie kurz zu fauchen.

Professor Elber legte seinen Kopf in seine Hände und sah gar nicht glücklich aus. \enquote{Weiß Madame Pomfrey was Ihnen fehlt?}, fragte er.

\enquote{Ja}, antwortete Alina.

\enquote{Hat sie es Ihnen gesagt?}

\enquote{Nein, Professor.} Dann nach einer kurzen Pause. \enquote{Was fehlt mir?}

Er hob seinen Kopf und sah Alina an. \enquote{Verkraften Sie es? Denn das, was ich Ihnen sage, könnte Ihnen mehr Angst einjagen\abs}

Alina nickte nur energisch mit dem Kopf. \enquote{Ich habe gesehen wie meine beiden Brüder starben. Einen habe ich mit sieben verloren. Den anderen kurz vor meiner Einschulung nach Hogwarts. Mich kann so leicht nichts schocken.}

Harry war beeindruckt, wie stark dieses Mädchen doch sein musste. Andererseits brauchte sie wohl jemanden, der sie tröstet oder ihr einfach nur Gesellschaft leistet.

\enquote{Also gut. Haben Sie schon einmal etwas von Harpyien gehört?}, fragte er.

Alina sah ihn an. Dann nahm sie den kleinen Handspiegel von ihrem Nachtkästchen und betrachtete ihr Gesicht. Nachdem sie den Spiegel zurückgelegt hatte, nahm sie ihren Schwanz in die Hand und betrachtete ihn, sowie ihre Klauen. \enquote{Ich werde zu einer!}, stellte sie mehr fest als dass es eine Frage war und schaute Harry und Professor Elber an.

Dieser nickte und meinte dann: \enquote{Ich weiß nicht wieso. Ich werde mit ihrer Mitschülerin sprechen. Sie soll uns an einem Dummy demonstrieren, was sie gemacht hat und dann sehen wir weiter. Ich habe so etwas noch nie erlebt. Ich habe schon über die irrsinnigsten Verwandlungen gelesen oder von ihnen gehört.} \gedanke{Oder sie selber erlebt.} \enquote{Aber so etwas ist mir neu. \gst Ich werde jetzt gehen und Ihre Mitschülerin mitbringen.} Er stand auf und verließ den Raum.

Harrys Bett rückte noch ein bisschen näher und Alina versuchte nach Harrys Hand zu greifen. Dieser ließ es zu und dachte bei sich: \gedanke{Wie die Klauen von Hedwig.}

\enquote{Woran denkst du, Harry?}, fragte Alina.

\enquote{An Hedwig. Ihre Klauen fühlen sich genauso an wie deine Hände.}

Alinas Augen wurden feucht.

\enquote{Nicht doch, Alina. Nicht weinen.}

\enquote{Fällt mir aber schwer.}

Ein paar Minuten später kam Madame Pomfrey herein und schaute in Richtung des Bettes in dem Harry liegen müsste. Dann fuhr ihr Kopf nach einem kleinen Räuspern herum und sie entdeckte ihn, wie er in seinem Bett neben Alina lag. Ihre Klaue in seiner Hand.

\enquote{Was machen Sie da, Mister Potter?}, fragte die Krankenschwester.

\enquote{Ich tröste Alina}, gab Harry knapp zurück.

\enquote{Wie kommen Sie dazu\abs}, begehrte Madame Pomfrey auf.

\enquote{Alina hat mich gebeten\abs}

\enquote{Mister Potter\abs} Harry fuhr lasziv mit seiner Zunge über seine Lippen und sah Madame Pomfrey dabei direkt an. Sie verstummte augenblicklich. \enquote{Also gut}, antwortete sie und ging in ihr Büro.

\enquote{Wie hast du das gemacht?}, wollte Alina wissen.

\enquote{Mein früherer Zustand der weiblichen Bevölkerung gegenüber, Alina}, sagte Harry.

\enquote{Oh, das.}

Es dauerte ca. eine halbe Stunde, da standen Professor Elber und Bessi vor den beiden Betten. Die Tür zur Krankenstation war gesichert und Madame Pomfrey stand mit gezogenem Zauberstab da. Professor Elber beschwor eine Feder hervor und legte sie auf ein Nachtkästchen, welches er in die Mitte des Raumes gezogen hatte. Bessi vollführte den Zauber und die Feder wurde in eine verwandelt, die von einer Harpyie stammte. Professor Elber und Madame Pomfrey beobachteten sie genau. Er legte die Feder beiseite und beschwor eine neue herauf. Dann vollzog er denselben Zauber wie Bessi. Die Feder verfärbte sich wunschgemäß. Nach einer neuen Feder nahm Professor Elber Bessis Zauberstab und versuchte es erneut. Und wieder wurde die Feder die einer Harpyie. An einer erneuten Feder durfte Bessi mit Professor Elbers Zauberstab es erneut versuchen. Dieses Mal klappte der Zauber.

Ein paar weitere Tests mit Bessis Zauberstab und verschiedene Zauber brachten als Ergebnis nur einen Defekt beim Färbezauber hervor. Sonst schien der Zauberstab ordnungsgemäß zu funktionieren.

\enquote{Behalten Sie ihn, Bessi, aber färben Sie damit nichts ein}, erklärte Professor Elber ihr. \enquote{Ich werde Mister Ollivander schreiben, er möge doch bitte herkommen. Oder noch besser, ich besuche ihn und versuche ihn mitzubringen.} Mit diesen Worten verließ er die Krankenstation. Madame Pomfrey ging ebenfalls durch ihr Büro in ein kleines Zimmer, in dem sie immer schlief, wenn sie solche Patienten hatte, um näher bei ihnen zu sein und ihnen schneller helfen zu können.

Nachdenklich setzte sich Bessi auf Harrys Bett, um Alina anzuschauen. Dicke Tränen liefen ihr übers Gesicht. \enquote{Es tut mir leid, Alina. Ich wollte das nicht.}

\enquote{Ist schon gut, Bessi. An dir liegt es nicht. Jemand hat deinen Stab manipuliert.}

Bessi legte sich auf die Seite und die beiden Mädchen unterhielten sich noch etwas. Doch viel zu schnell waren sie eingeschlafen. Harry zog Bessi vorsichtig zu sich unter die Decke, um sie vor dem Auskühlen zu schützen. Er versuchte das Bett zu verlassen, aber sein gebrochenes Bein schmerzte zu sehr. Als sich Bessi auch noch herumdrehte und sich an ihn schmiegte, ließ er den Versuch bleiben und legte sich resigniert zurück.

Als er am nächsten Morgen aufwachte, schmerzte sein Bein nicht mehr. Er stand vorsichtig auf und humpelte \gst ohne sein Bein zu sehr zu belasten \gst zum Nachbarbett. Dort legte er sich auf die Decke und schlief nochmals kurz ein.

Sanft wurde er durch ein Bussi auf eine Wange geküsst. Es war Bessi. \enquote{Danke}, sagte sie, \enquote{dass ich die Nacht bei dir verbringen durfte. Ich weiß nicht warum, aber bei dir habe ich mich wohlgefühlt. Jetzt weiß ich auch was Tamara meinte, als sie sagte: \inner{Mein Vertrauen zu ihm war einfach da.}} Sie schlug die Hand vor den Mund.

Harry lächelte sie an, nahm eine ihrer Hände in seine und sagte: \enquote{Ich werde ihr nichts davon erzählen.}

Bessi umarmte ihn erneut und verließ eilig die Krankenstation.

Nun hatte er das Herz von vier Slytherin-Mädchen gewonnen. Neben Katharina und Pansy, verstand er sich jetzt auch mit Alina und Bessi recht gut.

Harry zog sich gerade an, als Professor Elber mit Mr Ollivander und Bessi den Raum betrat.

\enquote{Ah, Mister Potter. Schön Sie wiederzusehen.}

\enquote{Ebenfalls, Mr Ollivander}, antwortete Harry.

Mr Ollivander nahm sich Bessis Zauberstab und setzte sich auf einen Stuhl, der durch einen Wink von ihm herankam und hinter ihm stoppte. Dann untersuchte er Bessis Zauberstab sehr genau und gründlich. Er nahm auch Alinas Zauberstab und Harrys als Referenz her. Dann vollführte er mit allen drei Zauberstäben denselben Zauber und nur bei Bessis Zauberstab kam die Veränderung der Feder in eine Harpyien-Feder. Mr Ollivander reichte ihr einen anderen Zauberstab, der ebenso elf Zoll lang war, aus Haselnussholz und im Kern ein Schwanzhaar eines Hippogreifes hatte.

Bessi vollführte den Zauber ohne Probleme und war zufrieden.

Mr Ollivander zog aus seiner Jacke eine Schachtel heraus, die wie eine Zauberstab-Schachtel aussah, aber etwas kleiner war und ziemlich zerschlissen aussah. Ein Gummiband hielt die Schachtel zusammen, welches er nun entfernte und die Schachtel öffnete. Er holte zwei Zauberstäbe heraus und legte einen auf einen kleinen Tisch, den er heranzog. Der Zauberstab wurde hinten unterlegt, sodass er genau auf die Feder zeigte. Bessi hatte gesagt, dass Alinas Zauberstab ebenfalls auf die Feder gezeigt hatte, als sie den verheerenden Zauber ausführte.

Mr Ollivander verlangte einen Dummy, den Madame Pomfrey normalerweise für ihre Erste-Hilfe-Kurse hatte. Er stellte den Dummy hinter dem Zauberstab auf und zielte mit Bessis altem Stab auf die Feder in der Mitte. Er sprach den Färbezauber und anstatt der Feder verwandelte sich der Dummy in eine Puppe mit dem Aussehen einer Harpyie. Nachdenklich sah Mr Ollivander den Zauberstab in seiner Hand an. Dann nahm er den zweiten Zauberstab aus seiner Schachtel und begann Bessis alten Zauberstab zu untersuchen. Nachdem er damit fertig war, holte er seinen zweiten auf dem Stuhl liegenden Zauberstab und untersuchte ihn ebenfalls.

\enquote{Kann ich den Zauberstab hier irgendwo einem Test unterziehen, ohne dass jemand Schaden davon trägt?}, fragte Mr Ollivander.

\enquote{Kommen Sie mit, Hogwarts hat einen eigenen Raum dafür. Wir sind hier bestens ausgerüstet}, sprach Professor Elber.

Bessi wurde zum Frühstück geschickt und versprach ihrer Freundin, ihr die Hausaufgaben zu bringen und den Stoff mitzuschreiben. Mr Ollivander und Professor Elber verließen die Krankenstation und Harry ging zum Frühstück, nachdem Madame Pomfrey ihn entlassen hatte.

Auf dem Weg dorthin fragte er sich, wie man einen Zauberstab verändern konnte, sodass er nur bei einem Zauber ein fehlerhaftes Verhalten zeigt.

\stimme{Das ist äußerst kompliziert, Harry. Das geht weit in den Bereich der schwarzen Magie}, meldete sich Salazar.

\gedanke{Könnte Voldemort es schaffen?}, fragte Harry.

\stimme{Ja, ihm dürfte es möglich sein. Nur frage ich mich was er damit\abs Wenn es ihm gelingt, Bessi unter den Imperius zu setzen, dann könnte er sie dazu veranlassen, die Zauberstäbe anderer Schüler zu entwenden und denselben Zauber an ihnen zu vollziehen und sie dann zurücklegen. \gst Das dürfte sie zwar umbringen, aber das macht ihm nichts aus. Dann hätte er eine Armee von Harpyien, die er unter seiner Kontrolle hat und dann kann er das Schloss aus dem Inneren angreifen.}

\chapter{Verwandlungen}


Nach dem Frühstück ging Harry wie immer zu seiner Unterrichtsstunde in \VgddK. Am Klassenzimmer angekommen, fand er nur einen Zettel an der Tür. Darauf stand: \accentuate{Die Klasse findet heute im Innenhof mit dem Feenbrunnen statt. F. Elber}

Also machte sich die Klasse auf und ging zum Hof mit dem besagten Brunnen. Dort wartete bereits ihr Lehrer. Aber auch Flitwicks Fünftklässler, die er jetzt hatte, waren anwesend. Die Klasse betrat den Hof und Professor Elber fing an.

\enquote{Schön, schön. Sie haben hergefunden. Professor Flitwick lässt sich durch Madame Pomfrey entschuldigen. Er ist auf der Krankenstation beschäftigt. Deshalb habe ich kurzerhand Ihre Mitschüler mitgenommen, damit Sie gemeinsam etwas lernen. Das eigentliche Unterrichtsthema wird heute also anders aussehen. Sie werden sich in Gruppen von zwei oder vier Personen zusammenfinden. Aber klassenübergreifend bitte. Also je ein Fünft- und ein Sechstklässler, oder eben zwei aus jeder Jahrgangsstufe. \gst Ihre Aufgabe ist es, diesen Innenhof zu untersuchen. Lassen Sie sich nicht davon abbringen, wenn ich irgendwo sitze oder stehe. Vertreiben Sie mich, wenn Sie die Stelle, an der ich mich befinde, untersuchen wollen. \gst Folgendes Szenario stellen Sie sich bitte vor: Sie sind diesen Gang\abs}, er zeigte kurz auf einen Gang, der auf den Hof zulief und in einer Biegung dann auf den Innenhof führte, \enquote{\aabs entlang gesprungen und verfolgen zusammen mit Ihren Kollegen einen Dieb. Dieser hatte, bevor er geflohen war, einen Beutel in der Hand und bog in diesen Hof ein. Knapp eine halbe Minute später kamen Sie auch dort an und der Dieb saß auf einer Bank, hatte aber keine Beute mehr bei sich. Einer Ihrer Kollegen nahm ihn bereits mit zum Verhör. Der Rest \gst also Sie \gst müssen nun versuchen, das Versteck aufzuspüren. Fangen Sie an.}

Er trat zurück und setzte sich auf eine Bank in der Nähe.

Alle Schüler standen erst einmal perplex herum, bevor sie sich fingen und zaghaft kleine Gruppen bildeten. Harry tat sich mit Neville aus seiner Klasse, sowie mit Sirin und Klaus aus der Jahrgangsstufe unter ihm zusammen. Es waren Hufflepuffs. Er kannte beide leidlich, und stellte sich daher vor \gst genauso wie Neville \gst und lernte dadurch ihre Namen (wieder), da sie sich ebenfalls vorstellten.

Er fragte sich, wie die jüngeren Zauberer und Hexen dabei helfen konnten. Während er eine Seite des Hofes untersuchte, sahen sich die anderen aus seiner Gruppe ihre Seiten an. Nach zehn Minuten trafen sie sich wieder und fingen an sich auszutauschen.

\enquote{Nichts gefunden}, merkte Klaus an.

\enquote{Bei mir auch nichts}, sagte Neville.

Sirin nickte nur. Ebenso Harry.

Sirin sah sich erneut im Hof um. Dann wanderte ihr Blick nach oben. Auf einem Vorsprung sah sie etwas, das so aussah, als ob es dort fremd war. Sie schwang ihren Zauberstab und sprach: \zauber{Wingardium Leviosa.} Ein Koffer hob sich leicht an und schwebte sanft auf die Erde.

Dies weckte die Aufmerksamkeit einiger anderer Schüler. Neugierig kamen sie etwas näher, behielten aber respektvollen Abstand.

\zauber{Alohomora}, sprach sie und der Deckel des Koffers sprang auf.

Eine spärlich gekleidete Frau mit vier Armen und zwei Beinen stieg aus dem Koffer und ging auf Sirin zu. Es war Shiva, eine hinduistische Göttin. Mit vor Schreck bleichem Gesicht stolperte sie rückwärts, bis sich Harry zwischen sie und Shiva schmiss und sie sich zu verwandeln begann. Nach kurzem erschien ein Dementor und Harry beschwor instinktiv einen Patronus hervor. Dieser galoppierte auf den vermeintlichen Dementor zu. Dann besann sich Harry eines besseren und sprach \zauber{Riddikulus} und der Dementor löste sich in einer kleinen Rauchwolke auf, die auf den Koffer zu schwebte, dessen Deckel sich schloss.

Jetzt meldete sich wieder Professor Elber zu Wort. \enquote{Nette Idee. Fünf Punkte für jedes der beiden Häuser. Aber der Koffer war es nicht, er ist zu groß.} Der Koffer schwebte wieder auf den Vorsprung zurück und die Suche ging weiter.

Harry dachte nach. Da meldete sich Klaus. \enquote{Wir haben noch nicht die Bank und den Brunnen untersucht}, stellte er sachlich fest.

\gedanke{Der Brunnen. Natürlich}, dachte Harry. \gedanke{Sonst fließt doch immer Wasser durch die Hörner, welche die männlichen Feen halten.}

Er winkte seine Gruppe zu sich und flüsterte ihnen seine Erkenntnis zu.

\enquote{Das macht Sinn}, merkte Neville an. \enquote{Es könnte unter dem Brunnen sein. Sollen wir ihn anheben?}, fragte er.

\enquote{Vielleicht ist er gesichert}, meinte Sirin. \enquote{Wir sollten ihn erst untersuchen.}

So wurde es auch beschlossen und die Gruppe besah sich den Brunnen genauer. Sie untersuchten ihn intensiv und kamen zu dem Schluss, dass der Brunnen an hebbar sei. Harry und Neville wurde die Aufgabe zugetragen, den Brunnen anzuheben. Nachdem dieser in der Luft geschwebt hatte, holte Klaus den Stoffbeutel zu sich, der darunter in einer Kuhle im Boden lag. Dann schwebte der Brunnen wieder nach unten und das Wasser floss durch die Füllhörner, welche die Feen hielten.

Der Beutel wurde magisch geöffnet und ein kleines bronzenes Kästchen kam zum Vorschein. Klaus hielt es in seinen Händen und wollte es gerade öffnen, doch Sirin hielt ihn davon ab. \enquote{Halt, Klaus. Denk an den Koffer. Stell es lieber hin}, sagte sie.

Klaus stellte das Kästchen auf den Boden und öffnete es magisch. Es lag eine Pergamentrolle darin. Neville bückte sich und holte sie heraus. Dann entrollte er sie und las vor.

\begin{brief}
Liebes Gewinnerteam,

herzlichen Glückwunsch zum Sieg.

Als kleine Belohnung dürft ihr euch an einem bestimmen Ort im Schloss eine Erholung gönnen. Diesen Sonntag könnt ihr euch vergnügen. Genießt euren Preis und denkt daran: Fasst alle das Pergament an, damit ihr von dem Zauber, den es umgibt, geleitet werdet.
\end{brief}

Sofort fassten drei weitere Hände das Pergament an, welches nach einigen Sekunden anfing zu glühen und dann im Nichts verschwand.

\enquote{Der Unterricht ist beendet}, sagte Professor Elber und lächelte. Nach zehn Minuten klingelte es und zeigte das Ende der Stunde und somit die Vorbereitung für die nächste an.

\trenn

Die Tage vergingen ohne Ergebnis für Alina. Sie wurde noch immer zu einer Harpyie. Mr. Ollivander konnte den Zauberstab reparieren und auch den Zauber ausfindig machen, der die Verwandlung verursachte, aber alle Mühen in der Bibliothek \gst das stundenlange Durchsuchen der Bücher \gst brachte nichts.

\stimme{Harry, du bist doch von m\aabs Professor Elber unterrichtet worden?}, fragte ihn Salazar.

\gedanke{Ja}, antwortete Harry während der abendlichen Astronomievorlesung.

\stimme{Du könntest es schaffen, die Harpyie zu bezwingen, sie von Voldemorts Fluch zu befreien. Du bist derjenige, der eine Verbindung zu ihm hat, verfügst über einen Teil seiner Macht. Zusammen mit deiner eigenen Magie wird es dir möglich sein, den Fluch zu brechen. Du musst allerdings allein mit der Harpyie sein und darfst den Zauber erst anwenden, wenn du von den anderen abgeschirmt bist. Die Querschläger könnten andere verletzen oder sogar töten.}

\enquote{Mister Potter, nicht träumen. Markieren Sie die Planetenpositionen auf Ihrer Karte}, ermahnte ihn Professor Sinistra.

\enquote{Ja Ma'am.}

Also begann Harry durch das Fernrohr zu schauen und die Positionen der Planeten zu erkunden. Er trug Planet um Planet ein, schaute in Tabellen nach einer Referenz für deren Bedeutung, während sie an bestimmten Positionen standen, und zeichnete die Gesichter der Planeten.

Eine einzelne Sternschnuppe durchquerte seine Sicht. Er wünschte sich etwas.

\trenn

Sonntagnachmittag umgab Harry ein eigenartig unbeschwertes Gefühl der Leichtigkeit. Er hatte das Gefühl, als stünde er unter dem Imperius. Doch er sah keinen Grund dagegen anzukämpfen, da er wusste, dass ihn ein Zauber heute leiten würde. Er verabschiedete sich aus dem Gemeinschaftsraum und lief in einen Bereich des Schlosses, der unbenutzt war. Auf seinem Weg lief er an staubigen Rüstungen und ziemlich alt aussehenden Bildern vorbei. Er begegnete einigen Hauselfen, die beschäftigt waren die staubigen Rüstungen zu säubern, damit sie nicht gänzlich unter einer Staubschicht verschwanden.

Die Elfen wollten gerade verschwinden, da sie Harry sahen, aber er hielt sie mit einem: \enquote{Lasst euch nicht stören}, davon ab. Er ging vor den Elfen in die Hocke und sagte dann: \enquote{Ich möchte euch danken, dass ihr uns täglich helft. Unsere Betten macht, unsere Kleidung wascht, bügelt und sauber in unsere Zimmer legt. Danke, dass ihr uns im Winter Wärmekissen in das Bett legt und die Feuer in den Kaminen am Brennen haltet. Danke, dass ihr für uns kocht und putzt.} Dann streckte er seine Hand aus.

Die Elfen sahen verschreckt aus. Noch nie hatte sich jemand bei ihnen bedankt. Einigen von ihnen liefen sogar Tränen die Wangen herunter.

Harry griff in seine Tasche und fing an, den Elfen die Tränen von ihren Gesichtern zu wischen. Jetzt bewegten sie sich noch weniger.

\enquote{Alles in Ordnung mit euch?}, fragte Harry vorsichtig nach.

Eine junge Elfe kam auf ihn zu und meinte: \enquote{Danke, Sir. Noch nie hat jemand so etwas zu uns gesagt. Wir Elfen fühlen uns geehrt, von so einem jungen und mächtigen Zauberer beachtet zu werden.}

Harry lächelte die Elfen an.

\enquote{Wir müssen wieder an unsere Arbeit, Sir}, sagte ein anderer älterer männlicher Elf.

Harry nickte, stand auf und lief weiter. Der Zauber führte seinen Weg und Harry drehte sich noch einmal um, um den Elfen zuzusehen.

Nach guten zehn Minuten stand er vor einem Gemälde, das sich scheinbar nicht bewegte und der Zauber, welcher ihn führte, fiel von ihm ab. Er hatte wieder das Gefühl, ganz Herr seiner Sinne zu sein. Er stellte sich die Frage, ob er unter dem Zauber so regiert hatte. Doch er kam zu dem Entschluss, dass dieser Anblick den Zauber kurz zurückweichen ließ und ihm seinen Willen \gst sich bei den Elfen zu bedanken \gst ließ.

Das Gemälde zeigte eine große Blumenvase mit vier Blumen an. Jede der Blumen war gleich gemalt. Nur waren ihre Blütenblätter in unterschiedlichen Farben gemalt. Sie waren in grün, blau, gelb und rot gehalten. Die Vase schimmerte, als wäre sie aus echtem Zinn. Die Farbe hatte metallische Pigmente, die sie schimmern ließ. Jetzt bekam er wieder einen Impuls, der ihm zuflüsterte: \stimme{Drücke die blaue Blume.} Doch das war leichter gesagt als getan. Selbst wenn er hochspringen würde, konnte er die Blütenblätter nicht erreichen. Und selbst wenn er den Hinweis bis zum äußersten ausreizte und nur den untersten Teil des Stils erreichen würde, der aus der Vase herausragte, er konnte ihn nicht erreichen. Er zog seinen Zauberstab in der Absicht, ihn auf die Pflanze zu werfen, als ihn wieder die Stimme in seinem Kopf unterbrach. \stimme{Doofer Zauberer}, hörte er in seinem Kopf.

\fluestern{Zauberer}, murmelte Harry vor sich hin. Er richtete seinen Zauberstab auf die blaue Blume und führte einen ungesagten und leichten \spruch{Protego} aus, der auf die Blume drückte.

Er hörte ein leises \geraeusch{Klack} und das Bild bewegte sich auf einer Seite wenige Millimeter von der Wand weg, als ob ein Verschluss aufsprang. Er öffnete das Bild und stieg hindurch. Das Gemälde fiel hinter ihm sofort wieder ins Schloss. Er stand in einem Gang, zu dessen linker Seite Spinde standen. Einer davon stand offen. Harry ging auf ihn zu und entdeckte einen Kleiderbügel und einige Fächer. Er zog seine Schulrobe aus und hängte sie vorsichtig auf den Kleiderbügel. Dann öffnete er seine Krawatte und legte sie in eines der Fächer. Danach knöpfte er sein Hemd auf und entledigte sich auch seiner restlichen Kleidung. Ohne etwas an seinem Körper ging er den Gang weiter und bog nach links ab.

Staunend blieb er stehen, als er das üppige Bad sah. Es war fast doppelt so groß, wie das der Vertrauensschüler, in dem er in seinem vierten Schuljahr einmal war. Neville und Klaus waren bereits im Wasser und hatten die Augen geschlossen. Harry sah sich um. Er ging an den Rand der in den Boden eingelassenen Wanne und stieg ins Wasser. Klaus und Neville entdeckten ihn und lächelten ihm zu. Harry nahm gegenüber Neville Platz, Klaus war zu seiner Rechten. Der Platz am Eingang war noch frei. Die beiden verwickelten Harry in ein Gespräch über Quidditch.

Sie hörten ein Geräusch und Harry sah nach links. Dort kam gerade Sirin auf sie zu. Auch sie war vollkommen nackt, wie die anderen. Harry besah sich ihren jungen Körper, spürte aber weder Peinlichkeit noch sonderliches Verlangen oder Scham. Erst heute Abend würden ihm die Erinnerungen daran noch einmal zu schaffen machen.

Sirin war am Becken angekommen und stieg in die Wanne. Sie setzte sich auf den freien Platz und begann in das wieder aufgenommene Gespräch einzusteigen.

\enquote{Nächstes Jahr möchte ich auch ins Team. Ich hoffe, ich schaffe es, da ja dann wieder jemand wegfällt.}

Jetzt war Harrys Neugierde geweckt. \enquote{Dann spielst du nur zwei Jahre, wenn du es denn schaffen solltest!}

\enquote{Ja, Harry.}

\enquote{Warum hast du es nicht früher versucht?}, fragte Neville dazwischen.

\enquote{Dazu fühlte ich mich noch nicht bereit}, sagte sie. \enquote{Ich musste noch auf dem Besen üben und auch Quidditch war mir nicht so \gst ich konnte es nicht, weil ich kaum Zeit fand, es richtig zu lernen.}

\enquote{Als was möchtest du denn spielen?}, fragte Neville.

Sirin setzte sich in einem Schneidersitz hin und legte ihre Arme auf dem Beckenrand ab. Die Wellenbewegungen im Becken nahmen nach ihren Bewegungen wieder ab, bis sie fast zum Stillstand kamen.

\enquote{Ich weiß noch nicht genau, das überlege ich mir über die Ferien. Ich dachte an Treiber, oder aber Sucher.} Sie sah Harry kurz an, wurde rot und sagte dann schnell: \enquote{Reservesucher natürlich. Äh, ich meine, Sucher, anstelle von Erika}, haspelte sie.

Harry lächelte ihr zu. \gedanke{Hast wohl vergessen, dass wir in verschiedenen Häusern sind}, dachte Harry.

Dann wandte sich Sirin an Neville und fragte ihn, warum er nicht in der Mannschaft spielte; immerhin hatte sie ihn schon ein paar Mal beobachtet, wie er sich mittlerweile auf dem Besen bewegte.

\begin{abAchtzehn}
Harrys Blick wanderte durch das ruhige Wasser über Sirins Körper. Sie hatte die gleiche Hautfarbe, wie die Patil-Zwillinge, da ihre Vorfahren auch aus Indien kamen. Ihre schulterlangen, schwarzen Haare sahen frisch gewaschen aus. Sein Blick wanderte tiefer und er sah, dass sie ihre Scham rasiert hatte.

Als er vor Wochen einmal die Gelegenheit hatte, sie spärlich bekleidet zu sehen, konnte er durch ihre Wäsche noch ihre Schambehaarung erkennen; für einen kurzen Moment sogar fühlen. Sein Blick wanderte zu Klaus und Neville, und dann wieder auf Augenhöhe, wo er sich schließlich wieder in das Gespräch einfand.
\end{abAchtzehn}

\begin{safedivide}
Harry döste vor sich hin und hörte nur mit halbem Ohr zu.
\end{safedivide}

Nach einer Weile verabschiedeten sich die drei und Harry genoss noch etwas das Alleinsein.

\gedanke{Salazar?}

\stimme{Ja, Harry.}

\gedanke{Was passiert, wenn ich mein Amulett ablege?}

\stimme{Das kommt darauf an, was du damit bezweckst.}

\gedanke{Ich frage mich, was passiert, wenn ich jetzt mein Amulett ablege. Malfoy\abs Draco hatte beide Male, genau wie ich, Albträume, als ich mal mein Amulett nicht anhatte.}

\stimme{Solange du wach bist, passiert nichts. Erst, wenn du schläfst und träumst.}

\gedanke{Aber ich habe doch auch sonst Albträume, wenn ich mein Medaillon anhabe. Nur wenn ich es nicht anhatte, habe ich von Draco geträumt.}

Es herrschte eine Weile Stille. Harry wollte gerade wieder nachfragen, als Salazar ihm antwortete: \stimme{Es könnte in Verbindung mit eurer weitläufigen Verwandtschaft und dem Amulett zusammenhängen. Und eventuell durch den Seelensplitter in dir.}

Harry dachte nach. Dann schimmerte die Luft und Salazar kam zum Vorschein. Er saß ihm gegenüber im Wasser und genoss es sichtlich, sich im warmen Wasser aufzuhalten.

\enquote{Ich dachte, Geister spüren nichts}, sagte Harry.

\enquote{Normalerweise ist das so, Harry, aber hier}, und er lächelte selig, \enquote{wirken sehr alte Zauber, die uns ein gewisses Gespür geben. Wenn sich hier ein sterblicher aufhält, dann können wir hier auch rein und die Annehmlichkeiten für wenige Augenblicke genießen.} Er schloss die Augen und legte den Kopf zurück. Dann sagte er: \enquote{Ich muss wieder los, Harry. Aber wir können uns ja} \stimme{gedanklich unterhalten.}

Harry lächelte in sich hinein. Dann schloss auch er seine Augen und blieb noch eine Weile im warmen Wasser.

Nach einem entspannten Nickerchen hörte er ein Glucksen und wusste, dass es Myrte war. Er öffnete seine Augen und sah sie einen Meter über seinem Gesicht schweben. Durchsichtig, traurig und in ihrer Schuluniform. Sie sah Harry an.

\enquote{Ich bin immer so allein, Harry. Warum kommst du mich nicht besuchen?}

\enquote{Ich habe viel zu tun, Myrte.}

\enquote{Na ja, wenigstens sind wir jetzt beisammen.} Sie wirkte nun etwas fröhlicher. \enquote{Ich habe dich gesehen, wie du mit anderen Mädchen Sex hattest. \gst Weißt du, ich hatte noch nie welchen. Und das, wo ich schon über sechzig bin.}

Harry musste schmunzeln. \enquote{Myrte, du bist erst vierzehn, oder so.}

\enquote{Ich bin über sechzig. Ich bin vielleicht jung gestorben, aber als Jungfrau und das vor sehr langer Zeit.} Sie wirkte wieder etwas trauriger. Dann schwebte sie tiefer und nun direkt über ihm. Sie bildete seine Konturen nach, hielt aber etwa dreißig Zentimeter Abstand zu ihm. \enquote{Ich möchte auch mal mit jemandem schlafen. Ich habe schon viele Paare dabei beobachtet und mir vorgestellt, wie es sein würde. Du bist der Erste, mit dem ich es mir seit langer Zeit einmal vorstellen könnte.} Wieder kam sie ihm etwas näher.

Harry versuchte aus dieser Situation das Beste zu machen und versuchte sie davon abzubringen. \enquote{Du hast ja immer noch etwas an, Myrte}, sagte er.

Myrte sah an sicher herunter und sagte: \enquote{Oh. Du hast recht.}

Sie schloss ihre Augen und konzentrierte sich. Sie begann zu verschwimmen, und als sie wieder klar zu erkennen war, hatte sie nichts mehr an und war vollkommen nackt. Sie entfernte sich von Harry etwas und meinte dann: \enquote{Und, Harry, nimmst du mich so?}

Myrte hatte etwas an sich, das Harry nicht beschreiben konnte. Als sie vollkommen nackt vor ihm schwebte und er in diesem Raum mit verminderter Scham war, regte sich etwas an ihm und er stellte sich tatsächlich vor mit ihr zu schlafen. Würde er Ginny mit einem Geist überhaupt betrügen können? Wäre Sex mit einem Geist, so er denn möglich wäre, Betrug am Partner?

Harry hatte den Eindruck, dass sie etwas weniger zu durchschauen war als sonst. \gedanke{Sie hat etwas mehr Substanz}, dachte er. Mit ihrer Zunge fuhr sie über ihre Lippen und senkte ihren Körper und ihren Kopf zu Harry hinunter. Dieser versuchte ihr auszuweichen und drückte sich gegen den Beckenrand und schob sich langsam aus dem Wasser auf den Boden davor. Er blieb mit seinen Kniekehlen hängen und seine Beine baumelten im Wasser, als Myrte mit ihren Lippen seine traf.

\begin{abAchtzehn}
Er spürte nur einen warmen Lufthauch, mehr nicht. Sie grinste ihn an und schwebte an ihm herunter. Sein Penis streckte sich ihr entgegen und sie nahm ihn in den Mund. Zumindest versuchte sie es. Wieder nur spürte er einen warmen Lufthauch. Er umschloss seinen Penis und fing an sich zu stimulieren. Mit seiner Hand fuhr er immer wieder durch Myrtes Kopf. Sie schien es nicht zu stören.

Dann glitt sie immer höher schwebend auf seiner Brust entlang. Dort wo ihr Mund knapp unterhalb seiner Haut war, konnte er einen warmen Schauer spüren. Als sie an seinem Mund ankam, zuckten kleine Blitze durch ihn. Myrte hielt kurz inne und meinte: \enquote{Das ist mir noch nie passiert.}
\end{abAchtzehn}

\gedanke{Hier herrscht sehr alte Magie}, kam Harry wieder in den Sinn.

Mit seiner freien Hand fuhr er Myrte in ihren Nacken und machte in der Luft kraulende Bewegungen, was sie sich scheinbar gefallen ließ. Sie ließ sich tiefer sinken, sodass ihr Körper leicht in seinen hinein glitt. Das war ein unglaubliches Gefühl. 

\begin{abAchtzehn}
Er war halb mit einem Geist verbunden und stimulierte sich selbst. Myrte drehte nur leicht ihre Lippen durch seine und schien glücklich zu sein.

Dann richtete sie sich auf, öffnete ihre Beine, spreizte mit ihren Fingern ihre Schamlippen und sank langsam auf Harry nieder. Harry sog zischend die Luft ein. Dieses Gefühl, das ihn durchfuhr, konnte er mit nichts anderem vergleichen. Er konnte es nicht beschreiben. Er hatte das Gefühl sich wirklich mit ihr zu vereinen. Myrte brachte nur ein \geraeusch{Ahhhhhh!} heraus. Sie warf ihren Kopf zurück und keuchte. Harry hatte das Gefühl, sich schwerer zu tun. Er nahm seine Hand von sich und umklammerte Myrtes Hüften.

Er konnte nun nicht mehr so einfach durch sie hindurch greifen. Vorsichtig hob und senkte er sie, als wäre sie sehr zerbrechlich. Doch mit der Zeit wurde sie für ihn immer fester, bis er sie ganz sah und spürte; auf sich, bei sich, in ihr.

Myrte schien davon nicht allzu viel mitbekommen zu haben. Sie hatte ihre Augen immer noch geschlossen und ihren Kopf immer noch in ihrem Nacken. Harry nahm jetzt seine Hände von ihren Hüften und fing an, ihre Brüste zu kneten.

\enquote{Harry}, kreischte Myrte. \enquote{Ja, knete sie durch, bearbeite sie, massiere sie}, rief sie. Sie öffnete ihre Augen und sah Harry an. Sie senkte ihren Oberkörper hinunter und küsste ihn. Ihr Blick war noch immer jenseits von Zeit und Raum. Er spürte ihre Säfte an seinem Schaft entlang herunterfließen, über seine Hüften und Schenkeln auf den warmen Steinboden.

Es ließ sich nicht vermeiden, dass er dieses Jahr an Erfahrungen sammelte, aber auf das, was er gleich erleben würden, konnte man sich auch mit Training nicht vorbereiten. Er spürte, wie sich Myrtes Orgasmus näherte und kam seinem Höhepunkt auch immer näher. Dann riss sie ihn mit sich. Doch sie konnte kein \enquote{\extase{Harry!}} schreien, da er ihren Mund mit einem tiefen und innigen Kuss versiegelte. So spürte er den Druck ihres Schreis in seinen Lungen, die sich mit warmer Luft füllten. Er spritze einen Teil seines Saftes durch sie hindurch in die Luft.
\end{abAchtzehn}

\begin{safedivide}
Beide blieben eines Weile so und genossen das Gefühl.
\end{safedivide}

Als sie sich wieder beruhigt hatte und Harry sich von ihr löste, sah sie ihn mit glücklichen, glasigen Augen an.

\begin{abAchtzehn}
Und schon wieder spürte er Myrtes Unterleib zucken. Und erneut riss sie ihn in einen unglaublichen Orgasmus hinein. Dieses Mal blieb sein Sperma in ihr. Als er sie von sich drücken wollte, riss es ihn wieder in den Himmel und Myrte mit ihm. Also blieben sie nach ihrem dritten Mal einfach liegen. Als sie sich beruhigt hatten, schwebte Myrte etwas höher und richtete sich in der Luft auf. Sie schwebte nun in einem Schneidersitz direkt über seinem Gesicht und gab ihm einen unglaublichen Anblick.

Er hob seinen Kopf und bearbeitete ihre Schamlippen mit seiner Zunge und ließ dabei auch ihre Klitoris nicht außen vor. Ihr Unterleib begann wieder zu zucken und dann ergoss sich ein dicker Strahl über Harrys Gesicht, nachdem sie komplett die Beherrschung verloren hatte, als ihr Harry einen Finger in ihre Scheide steckte und mit dem Daumen über ihr Lustknöpfchen rieb.

Myrte schloss ihre Augen und genoss dieses Gefühl noch einmal. Sie küsste Harry zum Abschied, fuhr sich kurz zwischen ihren Beinen mit den Fingern durch und sein Sperma fiel auf den Boden.
\end{abAchtzehn}

\begin{safedivide}
\fskdivider
\end{safedivide}

Dann begann sie wieder zu verschwimmen, und als sie wieder scharf zu sehen war, hatte sie wieder ihre alte Schuluniform an.

\begin{abAchtzehn}
\enquote{Ich will kein Risiko eingehen}, sagte sie. \enquote{Ich weiß nicht einmal, ob ich überhaupt schwanger werden könnte.} 
\end{abAchtzehn}

%\begin{safedivide}
%\fskdivider
%\end{safedivide}

Dann warf sie ihm noch eine Kusshand zu und verschwand mit einem seligen Lächeln in den Abflussrohren.

Harry duschte sich und zog sich danach an. Er verließ den Raum und sah Myrte, die scheinbar auf ihn wartete.

\enquote{Wiederholen wir das noch einmal, Harry? Irgendwann?}, fragte sie, mit einer gewissen Sehnsucht in ihrer Stimme.

\enquote{Irgendwann}, antwortete Harry, dessen Scham langsam zurückkehrte.

Auch Myrte war anzusehen, dass sie in ihre normale Art mit Harry umzugehen, zurückzufallen schien. Glücklich schwebte sie davon.

\trenn

Langsam schlief Harry ein und begann zu träumen. Das Amulett hatte er mit der Hand umfasst. Dann, mitten in der Nacht, lag er wieder neben Draco. Wie immer nackt. Beide hielten sich an den Händen.

\enquote{Bist du wach?}, fragte er Harry.

\enquote{Nein, Draco, ich träume. Genau wie du.}

\enquote{Weißt du, was merkwürdig ist?}, fragte er. \enquote{Wir liegen hier nackt nebeneinander und ich habe keine Gefühle für dich.} Harry musste leicht schmunzeln. \enquote{Ich meine, ich empfinde keinen Hass oder Ablehnung dir gegenüber.} Dann stieg er aus dem Bett und setzte sich auf den Fußboden. Die Beine angewinkelt und seine Hände außen herum geschlungen.

\enquote{Weiß jemand von diesen Träumen?}, fragte Harry Draco, stieg ebenfalls aus dem Bett und setzte sich mit gleicher Haltung neben ihn.

\enquote{Außer Tamara? Nein!}, antwortete Draco.

\enquote{Wieso deine Schwester?}, fragte Harry nach.

\enquote{Du hast es ihr doch erzählt}, antwortete Draco.

\enquote{Oh. Stimmt.}

Dann schwiegen beide lange Zeit.

\enquote{Irgendwie beneide ich dich}, begann Draco dann nach langer Zeit das Gespräch wieder. \enquote{Du hast Freunde gefunden.}

\enquote{Du hast doch Vincent und Gregory.}

\enquote{Das sind keine Freunde, die sind wie Schoßhündchen. Ich gebe ihnen Anweisungen und sie führen sie aus.}

\enquote{Blaise?}

\enquote{Der kommt einem Freund am nächsten.}

\enquote{Theodore?}

\enquote{Eher nicht.}

\enquote{Millicent?}

\enquote{Nein.}

\enquote{Maria und Pansy?}

\enquote{Die eine war, die andere ist meine Freundin.}

\enquote{Ich?}

\enquote{Wir können uns nicht leiden und bekriegen uns ständig.}

\enquote{Sieht momentan aber nicht danach aus.}

Draco sah ihn an. \enquote{Ich möchte nicht immer über oder unter dir aufwachen und etwas von dir in meinem Mund haben, wenn mir bewusst wird, dass ich träume.}

Harry lächelte leicht. \enquote{Daran können wir arbeiten.} Er konzentrierte sich und die Umgebung änderte sich. Harry trug nun ordentliche Kleidung, Draco war immer noch nackt.

Skeptisch sah ihn der Blonde an. \enquote{Wieso kannst du so etwas und ich nicht?}, fragte er ihn.

\enquote{Hast du es versucht?}, fragte Harry zurück. Draco schüttelte den Kopf, dachte kurz nach und hatte dann ebenfalls Kleidung an. \enquote{Siehst du.}

Draco nickte und begann vorsichtig ein paar Sachen, die ihm auf der Seele lagen, zu erzählen.

Harry tat dasselbe, obwohl er schon mit Ron darüber gesprochen hatte.

Erst am nächsten Morgen stellten beide fest, dass sie darüber nicht reden konnten. Sie wussten viel vom jeweils anderem, konnten aber nichts davon erzählen.

\stimme{Das liegt an mir}, hörte Harry in seinem Geist sagen und lächelte vor sich hin, als er am Frühstückstisch saß.

Er hatte begonnen, Draco zu verstehen. Sein Leben wurde immer von anderen bestimmt. Er konnte bislang keine eigenen Entscheidungen treffen.

\trenn

Harry ging auf die Krankenstation, um Alina etwas Gesellschaft zu leisten. Er betrat sie und fand nur Madame Pomfrey vor, die gerade aufräumte. Als sie zu ihm sah, winkte er nur ab und zeigte auf den abgeschirmten Bereich, in dem Alina lag. Stumm nickte sie ihm zu und kümmerte sich wieder um ihre Arbeit.

Harry trat an die Vorhänge und Stellwände um Alinas Bett heran und sagte leise: \enquote{Alina? Ich bin’s, Harry. Darf ich zu dir?}

\enquote{Ja}, kam schwach von der anderen Seite.

Harry betrat den abgeschirmten Bereich und musste sich erst einmal an den Anblick gewöhnen. Dort war fast eine voll ausgebildete junge Harpyie zu sehen. Nur das Gesicht hatte noch menschliche Züge und auf ihren Oberschenkeln, von denen einer unter der Bettdecke heraus schaute, waren noch Hautpartien zu sehen, die menschlich waren. Harry setzt sich auf einen Stuhl an ihrem Bett.

\enquote{Darf ich?}, fragte er und deutete auf ihre Hand.

Sie nickte nur stumm und sah ihn traurig an. \enquote{Keiner kann mir helfen. Mr. Ollivander konnte zwar den Zauber isolieren und beheben, aber keiner hat bisher ein Gegenmittel gefunden.}

Harry griff nach ihrer Hand. Sofort durchzogen ihn Bilder. Er sah eine kleine Schlange, die ihre Zähne in Alinas Oberschenkel schlug.

\stimme{Daran hatte ich gar nicht gedacht. Das könnte sogar funktionieren.}

\gedanke{Was könnte funktionieren?}

Harry bemerkte Alinas sorgenvollen Blick nicht, da er zu sehr mit sich selbst beschäftigt war.

\stimme{Basiliskengift. Es könnte mit dem richtigen Zauber zusammen ihren Zustand beheben. Das wäre sogar die bessere Methode.}

\enquote{Harry? Alles in Ordnung mit dir?}, fragte Alina.

\gedanke{Aber, wie kommen diese Bilder plötzlich in meinen Geist?}

\stimme{Das liegt an deinen drei Magie-Quellen, die sich in dir vereinen. Deine eigene Magie, die Magie Voldemorts und meine.}

\gedanke{Dadurch schaffe ich es, solche Bilder zu sehen?}

\stimme{Dadurch schaffst du es, Zugang zur Magie zu haben, die sonst gewöhnlichen Zauberern verschlossen ist. Du dringst hier in Bereiche der Magie vor, die nur wenigen offenbart werden.}

\gedanke{Kennst du jemanden, an den ich mich wenden kann?}

\stimme{Du wirst doch schon unterrichtet.}

\enquote{Harry, alles in Ordnung?}, fragte Alina erneut.

\enquote{Ja, Alina. Mir ist gerade eine Möglichkeit eingefallen, dir zu helfen.}

\enquote{Wie meinst du das?}

\enquote{Ich hatte eine Vision. Und meine Visionen waren in letzter Zeit sehr präzise und stellten sich als wahr heraus.}

\enquote{Was muss ich tun?}

\enquote{Du müsstest dich von einer Schlange}, jJetzt klingelte in seinem Kopf etwas, \gedanke{das war keine Schlange. Das war etwas Tödlicheres. Das war ein Basilisk}, \enquote{beißen lassen. In Zusammenarbeit mit einem Zauber wirst du dann geheilt.}

\enquote{Und das hilft mir?}

\enquote{Das kann ich dir nicht versprechen, aber ich bin der Meinung, dass es das tut.}

Der Vorhang ging auf und Madame Pomfrey betrat den abgeschirmten Bereich. \enquote{Wie kommen Sie darauf, die Lösung zu haben, wenn alle Professoren und Heiler keine Lösung finden können?}, fragte sie ihn.

\enquote{Weil ich Quellen habe, auf die andere nicht zugreifen können.}

\enquote{Welche Quellen?}

Harry holte sein Amulett hervor. \enquote{Ich habe durch das Amulett hier viel gelernt dieses Jahr. Außerdem kann ich auf}

\stimme{Vorsicht.}

\enquote{andere Magiequellen zugreifen. Ich kann nicht sagen, welche, aber sie sind sehr machtvoll. Wenn Sie es zulassen, dann bitte ich Sie nur um eines: Erzählen Sie niemandem davon und vertrauen Sie mir.}

Madame Pomfrey sah ihn skeptisch an. \enquote{Professor Dumbledore sollte dabei sein. Wenn er Ihnen vertraut, dann stimme ich zu.}

\enquote{Danke, Madame Pomfrey, ich werde alles vorbereiten. Sagen Sie dem Direktor Bescheid?}

Sie nickte und Harry verschwand von der Krankenstation.

\stimme{Nimm die Aufzüge, Harry. Das Symbol mit dem grünen Ring. Halte mein Amulett und danach deine Hand auf die Prüffläche, dann erhältst du Zugang und kommst in den versteckten Gängen vor meinen Räumen heraus. Die Tür kannst du so öffnen. Sie erkennt dich.}

Harry nickte, obwohl er kein gegenüber hatte und ging zu einem der Torbögen des Schlosses vor der Krankenstation. Er drückte den Stein und trat in das Innere des Aufzuges. Dann nahm er sein Amulett ab und drückte den Knopf mit dem grünen Ring. Er hielt zuerst das Amulett und danach seine Hand auf die Prüffläche. Die Türen schlossen sich und der Boden vibrierte. Als die Türen sich wieder öffneten und er heraustrat, musste er sich erst einmal orientieren.

Dann trat er die wenigen Meter zur Tür und öffnete sie. Er stand in Salazars privaten Räumen.

\parsel{Marcel? Bist du hier?}

Es dauerte eine Weile, dann tauchte der kleine Basilisk auf, schlich zwischen seinen Beinen hindurch und schlängelte sich an seinem Hosenbein entlang hoch zur Hand. Dann legte er sich wie ein Schal um Harrys Hals und sah ihn an.

\parsel{Wass gibt ess?}

\parsel{Ich brauche deine Hilfe um einer Mitschülerin zu helfen.}

\parsel{Wass mussss ich tun?}

\parsel{Du musst sie nur beißen.}

\parsel{Dass sschaffe ich.}

Harry streichelte ihm über seinen Kopf und die kleine Schlange hielt dagegen. Sie ließ sich das Streicheln gefallen. Harry verließ nach einem Blick und einem dankbaren nicken zu Salazar und seiner Frau die Räumlichkeiten und kehrte über die Aufzüge zurück in die Krankenstation.

Dumbledore war bereits da und von Madame Pomfrey im Groben darüber aufgeklärt, was sie erfahren hatte. Harry schloss hinter sich die Türen und sicherte sie.

\enquote{Du meinst, dass du ihr helfen kannst?}, fragte er.

\enquote{Ja}, antwortete Harry. \enquote{Ich kann Ihnen nicht sagen, woher ich das weiß, aber ich bin sehr sicher, dass es klappt.}

Dumbledore sah ihn lange an. Nachdem Harry langweilig wurde, ging er auf Alina zu und nahm Marcel von seinem Hals herunter, um ihn Alina zu zeigen.

\enquote{Dies hier ist Marcel. Er wird dich beißen.}

\parsel{Ich kann das nicht.}

Harry sah zu Marcel. \parsel{Wie? Du kannst das nicht?}

\parsel{Ich komme nicht durch die Haut mit meinen kleinen Zähnen. Du musst ihr das Gift mit etwas anderem als meinen Zähnen injizieren.}

Harry sah zu Madame Pomfrey. \enquote{Haben sie eine Spritze und ein Reagenzglas mit einem Gummihandschuh darüber?}

Madame Pomfrey sah ihn erst fragend und dann erstaunt an. \enquote{Das ist keine Schlange, Mister Potter. Oder? Das Gift, das Sie mir gegeben haben, das andere meine ich, stammt von \accentuate{ihm}?} Sie wollte den Begriff Basilisk nicht in den Mund nehmen, da sie die Antwort zu sehr ängstigen würde.

\enquote{Da haben Sie recht.}

\enquote{Warum leben wir dann immer noch?}

\enquote{Poppy?}, fragte Dumbledore nach.

\enquote{Weil ich mit einem Zauber und einem Trank dafür gesorgt habe, dass seine Blicke nicht mehr tödlich sind.}

\enquote{Reden wir hier von einem Basilisken, Harry?}

\enquote{Ja, Professor. Sein Name ist Marcel. \gst Bekomme ich jetzt die Spritze und das Reagenzglas für das Gift? Ich möchte Alina gerne helfen.} Harry nervte es, dass er immer wieder aufgehalten wurde. \enquote{Tut mir leid}, sagte er dann, da er merkte, dass es doch respektlos war, was er gesagt hatte.

\enquote{Ich hole Ihnen die Sachen}, sagte Madame Pomfrey und verschwand.

Dumbledore war noch immer nicht richtig überzeugt. \enquote{Warum hast du mir nichts davon erzählt, Harry?}

\enquote{Du erzählst mir ja auch nicht alles. Außerdem hatte ich Angst um meinen kleinen Freund.}

\enquote{Aber wie kommst du an diese Zauber? Ich kenne keine Möglichkeit, einem Basilisken diese Fähigkeit zu nehmen.}

\enquote{Du hast dich mit diesen Tieren aber auch nicht beschäftigt. Aber jemand anderes sehr wohl. Ich habe viel über diese Geschöpfe gelernt. Wenn man ihren tödlichen Blick im Griff hat, dann sind sie wie normale Schlangen. Nur ihr Gift ist überaus wirkungsvoll. Es übertrifft die der giftigsten Schlange auf Erden um ein Vielfaches.}

Madame Pomfrey war bereits mit den gewünschten Sachen zurück und gab sie Harry. \enquote{Bitte, Mister Potter, fangen Sie an.}

\enquote{Ich bin noch nicht ganz davon überzeugt, Poppy.}

\enquote{Dich hat auch keiner gefragt. Das ist meine Patientin und ich bin für sie verantwortlich. Wenn ich eine Behandlung für richtig halte, dann ist das ausschließlich meine Sache. Da hältst du dich raus.} Dumbledore sah sie erstaunt an. So hatte er sie noch nie erlebt. \enquote{Also, hilfst du uns, oder gehst du?}, meinte sie jetzt noch energischer.

Es brauchte eine Weile, bis Dumbledore wieder zu sich fand. \enquote{Ich werde von dort aus zusehen}, sagte er endlich und entfernte sich, um sich auf einen Hocker zu setzen und das Ganze aus der Ferne zu betrachten.

Harry nahm die Gegenstände von Madame Pomfrey entgegen und setzte sich wieder an Alinas Bett. Er zog den Gummihandschuh über das Reagenzglas, als ihn Marcel unterbrach.

\parsel{Ich denke, ich schaffe das auch ohne. Ich weiß ja, was du von mir erwartest.}

Harry nickte und entfernte den Handschuh wieder. Dann hielt er seinem kleinen Freund das Glas hin, worauf dieser hineinbiss und sein Gift abgab. Dann nahm Harry eine kleine Menge davon in die Spritze auf und fragte Alina, ob sie bereit sei. Nach kurzem Nicken stach er zu und gab das Gift über die Kanüle in die Muskeln frei. \enquote{Schützt euch bitte vor Querschlägern}, sagte Harry zu Madame Pomfrey und Dumbledore. Er zog die Spritze zurück, stand auf legte sie beiseite. Dann wartete er ein paar Sekunden, bevor er seinen Zauberstab zog und ihn auf Alina richtete. Gerade als sie anfing, das Gift zu bemerken, legte er los.

\enquote{Das Gift brennt, Harry}, sagte Alina.

Ihm flogen die passenden Sprüche in seinem Geist zu. Nacheinander wendete er jeden dieser Sprüche auf Alina an: Einen um das Gift zu wandeln, einen um die Regeneration der Zellen anzuregen und einen um die Verwandlung zu transformieren. Zuerst sah es so aus, als ob sich die Transformation schneller entwickeln würde und Dumbledore stand schon auf um näher zu kommen. Doch Harry erschuf einen Schild um die zwei herum und machte weiter. Als Alina vollständig eine Harpyie war und keine menschlichen Züge mehr an sich trug, begann die Rückverwandlung. Durch die Sprüche verhinderte er, dass sich ihr Geist veränderte. Aber die körperliche Transformation musste sie zuerst vollenden, bevor er sie zurückverwandeln konnte.

Als er fertig war und das schützende Feld wieder abbaute, legte er sich erschöpft auf das Bett daneben und schlief ein. Nach drei Stunden erwachte er wieder. Dumbledore und Madame Pomfrey waren verschwunden, nur Alina saß noch an seinem Bett.

\enquote{Danke}, sagte sie und gab ihm einen Kuss auf die Stirn. \enquote{Ich bin wieder ich selber. Wie kann ich dir das zurückzahlen?}

\enquote{Lerne und bleib auf der richtigen Seite.}

\enquote{Welche ist das?}

\enquote{Das ist manchmal schwer zu beantworten, zumal jede Seite sagt, ihre sei die richtige.}

\enquote{Das würde die andere Seite wohl nicht sagen.}

\enquote{Vermutlich nicht.}

\enquote{Dann werde ich auf deiner Seite bleiben.} Sie lächelte ihn an, drückte kurz seine Hand und meinte dann: \enquote{Ich werde jetzt in mein Haus gehen. Die anderen haben mich sicherlich schon vermisst. Und ich werde meinen Eltern schreiben.}

\enquote{Tu das}, sagte Harry und ruhte sich weiter aus. Er spürte, dass er Voldemorts Seelenteil wieder einen großen Teil Magie entzog.

Außerdem musst er sich noch überlegen, wie er Ginny das mit Myrte beibrachte. Sollte er es ihr sagen?

\chapter{Leben und leben lassen}


Harry saß in dem grünen Ledersessel mit dem schwarzen Holz in Salazars Räumen und sah zu den verzauberten Fenstern hinaus. Vor ein paar Tagen erst hatte er entdeckt, wie man die Aussicht ändert und es einmal versucht. Doch nach wenigen Minuten kehrte er wieder zur alten Ansicht zurück, da er sich den Ort sehr genau vorstellen musste. Und das einzige, was ihm dabei einfiel, war das Haus seiner einzigen Verwandten. Er überlegte, was er Dobby sagen würde. \gedanke{Traupaar-Führer}, ging ihm wieder durch den Kopf.

Er wusste jetzt, dass seine Rolle der eines Priesters entsprach und er die Trauung vornehmen sollte. Dobby hatte ein paar Reden beigefügt, damit Harry eine Ahnung davon hatte, aber eigentlich musste man keine vorgegebene Regeln beachten. Das einzige, was ihm nichts sagte, war der Teil mit der magischen Bindung.

\enquote{Dobby}, rief er. \enquote{Es geht um deine Trauung.}

Es dauerte ein paar Sekunden, dann tauchte er mit seiner Braut Winky auf.

\enquote{Sir Harry hat gerufen?}, sagte Dobby.

\enquote{Ja Dobby. Hallo Winky}, begrüßte er die beiden Elfen. \enquote{Kommt, setzt euch.} Dobby und Winky sahen sich an, dann setzten sie sich schließlich auf den Boden. \enquote{Eigentlich meinte ich auf den Sessel. Aber egal.} Harry setzte sich ebenfalls auf den Fußboden und fragte Dobby über die genauen Umstände aus.

\enquote{Soweit ich verstanden habe, soll ich euch trauen.} Die beiden Elfen nickten. \enquote{Und ihr beide wollt, dass ich das mache.} Dabei sah er Winky an; diese nickte ein paar Mal in schneller Abfolge. \enquote{Bis dahin habe ich es verstanden und durch deine beigefügten Reden habe ich auch eine Ahnung, was ich sagen möchte.} Damit hatte er schon mal implizit zugestimmt, wenn man die offenen Punkte beiseite ließ.

\enquote{Was mich aber irritiert, beziehungsweise was ich nicht ganz verstehe, ist der Teil mit der magischen Bindung.}

Winky begann zu erzählen. \enquote{Die magische Bindung dient dazu, dass sich die Ehepartner trotz der Schutzzauber, die die Elfen zusätzlich über die Häuser ihrer Meister legen, damit fremde Elfen nicht in die Häuser apparieren können, sehen und besuchen können. Sonst müssten wir uns immer gegenseitig anmelden. Aber durch die Bindung können wir zusammen wohnen, auch wenn unsere Meister räumlich getrennt sein sollten. Die Schutzzauber erlauben das.}

\enquote{Und geht das automatisch, oder muss ich da etwas tun?}

\enquote{Ihr müsst einen Zauber sprechen, Sir Harry}, antwortete Dobby. \enquote{Dobby wird ihn euch mitteilen, wenn Ihr zustimmen solltet. Ihr dürft ihn aber niemandem verraten.}

Harry nickte. \enquote{Aber warum soll ich das übernehmen?}

Dobby sah Harry mit gesenkten Ohren an. Dann wandte er einen Blick Richtung Boden. Er war nicht sein Meister, deshalb konnte er ihn auch anlügen. Er dachte nach, was er sagen sollte. Winky stieß ihn von der Seite an und nickte kaum merkbar.

\enquote{Dobby und Winky hoffen, später einmal von Sir Harry aufgenommen zu werden}, sagte der kleine Elf schließlich ziemlich leise und zurückhaltend.

Das musste Harry erst einmal verdauen. Er verstand den Zusammenhang noch immer nicht. \enquote{Ist das der Grund?}

\enquote{Nicht ganz}, sagte Winky. \enquote{Durch den Zauber, den Sie, Sir Harry, sprechen, sind Sie in kleinem Maße Teil des Zaubers. Es ist ein intuitives Gespür, falls es einem von uns schlecht gehen sollte.}

Und Dobby ergänzte: \enquote{Die Aufgabe des Traupaar-Führers geht über die normale Trauung hinaus. Er begleitet die beiden zu Trauenden zum Zeremonialplatz. Will heißen: Er appariert mit ihnen hinter die Gäste und läuft dann hinter dem Brautpaar nach vorne, umrundet sie und führt die Zeremonie. Dann bringt er sie zu den Feierlichkeiten, sollten sie an einem anderen Ort sein.}

Harry dachte noch eine kleine Weile nach. Im Prinzip hatte er eh schon zugestimmt. \enquote{Also gut. Ich mache es. Wann findet die Trauung statt?}

\enquote{Dobby und Winky dachten, Mitte des nächsten Jahres. An einem Sonntag-Nachmittag hier in Hogwarts. Die Erlaubnis des Direktors haben Dobby und Winky schon.}

Jetzt musste Harry schmunzeln. \enquote{Wer ist noch dabei?}, fragte er nach.

\enquote{Alle Elfen von Hogwarts; ein paar Freunde, andere Elfen; Sie, Sir Harry und der Schulleiter. Sonst niemand.}

Dann hatte Harry doch noch eine Frage. \enquote{Wird die Zeremonie im Stehen durchgeführt, oder ist das egal?}

\enquote{Der Traupaar-Führer leitet die Zeremonie}, sagte Winky ganz verständnislos.

Das war für Harry das Zeichen, dass er sie gestalten konnte, wie er wollte. Er würde, wie in japanischen Filmen, auf seinen Beinen auf dem Boden sitzen. Ebenfalls die Elfen. Die Gäste bekamen Kissen. Er musste also die Halle schmücken. Aber bis dahin war noch eine Menge Zeit. Es konnte sich noch viel ändern.

\trenn

\enquote{Schnell, rufen Sie Ihren VgddK-Lehrer}, hörte Harry die Stimme von Professor McGonagall. Kurz danach hörte er schnelle Schritte auf sich zukommen und um die Ecke bog ein kleines Mädchen, das ihn fast umrannte. Er kannte sie zwar vom Sehen, hatte aber keine Ahnung wie sie hieß. Nach ein paar Schritten kam er um die Ecke und sah, wie Professor McGonagall verzweifelt versuchte, ein sehr lebhaftes Feuer zu löschen. Es schien sich zu verändern und umherzuwandern. Aber das war nicht alles, so etwas hielt Harry zwar für ungewöhnlich, aber er hatte sich angewöhnt über nichts mehr zu staunen, was in Hogwarts passierte.

Erst als er den Grund für Professor McGonagall mehr als verzweifelte Versuche sah, entglitten Harrys Gesichtszüge. Hinter dem Feuer in einen kleinen Winkel zurückgedrängt waren zwei Schüler und zitterten und schrien. McGonagall konnte die zwei schließlich beruhigen, sodass sie nur noch zitterten, aber nicht mehr schrien. Das Feuer schien den Zaubern auszuweichen. Es schien so, als ob das Feuer leben würde und immer, wenn es ihr zu gelingen schien, das Feuer einzudämmen, flammte es kurz darauf wieder zu seiner vollen Größe wieder auf.

Da sonst nur jüngere Schüler da waren, beschleunigte Harry seine Schritte, zog seinen Zauberstab und fragte Professor McGonagall: \enquote{Brauchen Sie Hilfe Professor? Kann ich irgendetwas tun?}

Professor McGonagall drehte ihren Kopf. \enquote{Ja Mister Potter. Versuchen Sie das Feuer in Schach zu halten, oder zu löschen.} Harry nickte. \enquote{Haben Sie schon einmal mit Dämonenfeuer zu tun gehabt?} Harry schüttelte den Kopf. \enquote{Versuchen Sie einfach es in Schach zu halten, bis Professor Elber kommt. Hoffentlich weiß er, wie man es löscht, denn es entzieht sich meinen Versuchen. Es scheint irgendwie anders zu sein.}

Zusammen mit seiner Verwandlungslehrerin versuchte er das Feuer in Schach zu halten. Es konnte sich nur noch in Grenzen bewegen, aber die beiden Schüler konnten noch immer nicht entkommen. Zumindest hörten die beiden auf zu zittern und begannen sich dann zusammengekauert auf den Boden zu setzen.

Wasser war zwecklos, erkannte Harry, nachdem er einen Schwall darüber fließen gelassen hatte. Es ging einfach durch das Feuer hindurch und benässte den Boden. Dann versuchte er es mit Wind wegzublasen. Zuerst vertrieb er das Feuer, doch als einer der Schüler aufstand um durch den schmalen Spalt zu entwischen, bewegte sich das Feuer wieder auf seine Ausgangsposition zurück. Das einzig Gute daran war, dass das Feuer sich nicht vermehrte oder sonst irgendwie ausbreitete.

Dann erinnerte er sich, was Professor Elber über das lebendige Feuer gesagt hatte. Er dachte kurz nach und steckte seinen Zauberstab ein. Mit geschlossenen Augen und ruhigen Gedanken hielt er das Feuer nur mit seinem Geist in Schach; an Ort und Stelle. Außer bei seinen privaten Stunden und ein paar wenigen in Professor Elbers Unterricht zauberte er immer mit Zauberstab. Deshalb entfiel ihm immer, dass er es auch ohne beherrschte.

Jetzt hörte er wieder Schritte hinter sich. Er drehte sich kurz um und sah die Kleine mit Professor Elber hinter sich herlaufen. \enquote{Minerva! Was?} Dann stockte er. \enquote{Lebendiges Feuer? Das können die Siebtklässler doch löschen? Weshalb bin ich hier Minerva? Du kannst das doch löschen.}

\enquote{Eben nicht}, gab sie sauer zurück und drehte sich um. \enquote{Es widersetzt sich sämtlichen bekannten Versuchen.}

Harry hatte das Feuer gut unter Kontrolle. Nur löschen konnte er es nicht. Er spürte förmlich die Blicke seines Lehrers hinter sich.

\enquote{Lässt du Harry jetzt ganz alleine dagegen ankämpfen?}, fragte Elber Professor McGonagall.

\enquote{Nein, du sollst dich darum kümmern.}

Harry hörte Schritte. Dann das Rascheln eines Umhanges und spürte danach eine Hand auf seiner Schulter. \enquote{Sie können aufhören}, hörte er. Dankbar ließ er gedanklich los, lies seine Hände sinken und sah Professor Elber zu.

Professor Elber schlenkerte mit seinem Zauberstab, um das Feuer zu testen. \enquote{Hmmm. \gst Du hast nicht übertrieben, Minerva. Eigenartiges Feuer.}

\enquote{Sagte ich doch.}

Er blieb stehen und hielt das Feuer in Schach, während er nachdenklich sein Kinn rieb. Er versuchte noch ein paar Zauber, doch Harry hatte den Eindruck, dass er nicht besonders entspannt war und locker wie bisher darüber hinwegging, das Problem löste und sich wieder seinen Aufgaben widmete. Harry bemerkte nur, wie er seinen Zauberstab einsteckte. Das Feuer baute sich danach zu seiner vollen Größe wieder auf und loderte breiter als vorher und etwa bis zu Harrys Bauchnabel.

\enquote{Frederick, was tust du?}

Er gab ihr keine Antwort. Harry stand entgeistert neben ihm und beobachtete ihn. Er schloss seine Augen und begann eine Beschwörungsformel zu sprechen. Harry hatte den Eindruck, dass die Luft um ihn herum zu flimmern anfing. Gerade so, als würde er in der Wüste stehen und an den Horizont blicken, oder im heißen Sommer die Asphalt-Straße beobachten.

Professor Elbers Haare wurden weiß und er öffnete seine Augen wieder. Dann blickte er kurz zu Harry und trat auf das Feuer zu. Genau an den Stellen, an denen er sich bewegte, verzog sich das Feuer, hatte Harry den Eindruck. Als er durch das Feuer hindurch war, ging er in die Hocke und sprach mit den beiden Schülern. \enquote{Ich nehme euch jetzt an meiner Seite hoch. Legt eure Beine um meinen Bauch und schmiegt euch so fest ihr könnt an mich. \gst Euer Leben hängt davon ab. Lasst nicht los, solange ich es euch nicht sage. Verstanden?}

Die beiden nickten und standen auf. Professor Elber legte je einen Arm um sie und zog sie hoch. Sofort schlangen sie ihre Beine um ihn und schmiegten sich fest an ihn. Ihre Köpfe vergruben sie in seinen Haaren, sodass sie nichts sahen.

\gedanke{Sie müssen furchtbare Angst haben}, dachte Harry.

Langsam aber sicher, Schritt um Schritt, trat Professor Elber durch das Feuer, welches jetzt stärker loderte. Es versuchte, die beiden zu verbrennen, kam aber nicht an sie heran. Etwas schien das Feuer abzuhalten. Als er das Feuer um einige Meter hinter sich gelassen hatte, ging er wieder in die Hocke. \enquote{Ihr könnt jetzt loslassen}, sagte er.

Die beiden lösten sich von ihm und umarmten ihn noch einmal kurz. \enquote{Danke}, sagten beide fast gleichzeitig.

Er sah zu ein paar anderen Mitschülern. \enquote{Ihr bringt die beiden direkt in den Krankenflügel.} Die angesprochenen nickten und nahmen sie mit.

Professor Elber stand wieder auf und zog seinen Zauberstab. Er fuhr an sich hinunter und das eigenartige flimmern verschwand. Dann wandte er sich wieder dem Feuer zu. Er schwang seinen Zauberstab über seinen Kopf und beschwor ebenfalls ein Feuer hervor und ließ es auf das andere zufließen. Beide Feuer vereinten sich und führten einen atemberaubenden Tanz auf. Die Flammen versuchten sich gegenseitig zu unterdrücken und zu übertrumpfen und die anderen Flammen zu verdrängen und auszulöschen.

Dann schwang er wieder seinen Zauberstab und das Feuer klatschte mit Gewalt an die dahinter liegende Wand, an der zuvor noch die beiden Zweitklässler gesessen hatten. Jetzt war das Feuer verschwunden, aber die Wand war mit mehreren Millimetern Ruß überzogen.

Nachdenklich stand Harrys Lehrer da und sah ihn an, sprach aber zu McGonagall: \enquote{Minerva? Tust du mir einen Gefallen?}

\enquote{Welchen?}

\enquote{Treibe alle Schüler in der Großen Halle zusammen. Ich möchte eine Ansprache halten.} Er drehte sich zu Professor McGonagall um. \enquote{Das war kein normales lebendiges Feuer. Und ich möchte wissen, ob irgendein Schüler dafür verantwortlich ist.}

\enquote{Zusammentreiben?}, fragte Professor McGonagall aufgeregt.

\enquote{Dann eben zusammen rufen.}

\enquote{Ok, Frederick.}

Alle Schüler, mit Ausnahme der beiden im Krankenflügel, standen nun in der Großen Halle. Die Tische und Bänke schwebten an der Decke der Halle, sodass genug Platz war. Professor Elber stand mit einem grimmigen Gesichtsausdruck auf der Empore und schritt auf und ab. Dann drehte er sich und begann.

\enquote{Es dürften noch nicht alle hier darüber Bescheid wissen, was heute im fünften Stock vor dem Ge\-schichts\-kun\-de-Klas\-sen\-zim\-mer stattfand.} Dann lief er wieder auf und ab. Professor McGonagall machte ein betretenes Gesicht und beobachtete Professor Elber genau. Alle anderen Lehrer waren ebenfalls da.

Er drehte sich wieder zu den Schülern und nahm seinen Zauberstab heraus. Dann ging er in die Mitte der Empore und beschwor Flammen hervor, welche er über seiner Hand schweben ließ.

\enquote{Das hier ist ein ganz normales Feuer. Es nährt sich von brennbaren Materialien, oder von der Menge an Magie, die sie ihm zuteilen, damit es nicht erlischt.} Er schloss seine Hand und das Feuer verschwand wieder.

Er schwang wieder seinen Zauberstab. \enquote{Das hier ist lebendiges Feuer. Oder auch Dämonenfeuer.} Eine kleine Flamme schwebte wieder über seiner Hand. Aber man konnte ihr ihre Gefährlichkeit nicht ansehen. Er drehte seine Hand und das Feuer fiel zu Boden. Sofort breitete es sich auf die dreifache Größe aus und begann zu wachsen. Mit seinem Auge auf dem Feuer begann er zu erzählen. \enquote{Ich hatte Ihnen schon einmal davon erzählt. Aber dieses Mal verwende ich den Ausdruck Teufelsfeuer. \gst Dieses Feuer war heute im fünften Stock zu sehen. Es schloss zwei Schüler ein. Zweitklässler. Die sind jetzt im Krankenflügel.} Dann wurde er deutlich zorniger. \enquote{Jeder von Ihnen wird sich einer kleinen Überprüfung unterziehen müssen. Keiner verlässt mir die Große Halle, bevor nicht der Letzte in diesem Raum seinen Zauberstab hat kontrollieren lassen.} Leichte Panik durchflutete den Raum. \enquote{Diejenigen, die Ihre Zauberstäbe nicht bei sich haben, werden sich konzentrieren und laut, aber deutlich, Ihnen befehlen, hierherzukommen.}

Das Feuer breitete sich unterdessen immer weiter unter den wachsamen Augen Professor Elbers aus.

Jetzt sagte er zornig: \enquote{Sollte ich feststellen, dass irgendein Schüler dieses Feuer}, und er fügte leiser hinzu, \enquote{oder irgendein Lehrer}, und nun wieder in normaler Lautstärke, \enquote{wird derjenige, diejenige, oder diejenigen von mir etwas bekommen. Und zwar eine Reise nach Hause. Ich denke, Ihr Schulleiter wird Sie dafür hinauswerfen.}

Das Gemurmel, welches mittlerweile die Halle erfüllte, wurde lauter und leichte Panik stieg auf.

\enquote{Kommen Sie einzeln hervor und treten Sie durch das Feuer. Es wird Ihnen nichts tun.} Schlagartig breitete sich das Feuer auf die gesamte Breite der Großen Halle aus. Man musste hindurch, um sich der Überprüfung zu unterziehen. \enquote{Strecken Sie mir Ihren Zauberstab mit dem Griff von sich weg entgegen. Ich werde ihn kurz berühren und Ihnen dann sagen, ob Sie gehen können.}

Keiner der Anwesenden bewegte sich. Also machte Harry den Anfang, da er wusste, ihm würde nichts geschehen. Selbstbewusst trat er vor und holte seinen Zauberstab aus seinem Umhang. Er blieb kurz vor dem Feuer stehen und drehte seinen Zauberstab in der Hand um. Nun hielt er die Spitze in der Hand. Das Teufelsfeuer bildete eine schmale Gasse und er hatte das Gefühl, dass das Feuer zu ihm züngelte.

Er atmete noch einmal kurz durch und schritt ohne zu zögern durch die Flammen. Sie fühlten sich warm und weich an. Keine Spur von übermäßiger Hitze oder Angriffsgedanken. Er streckte seinen Zauberstab seinem Lehrer entgegen. Dieser berührte ihn einmal sachte und tippte ihn danach mit seinem Zauberstab an. Kleine grüne Funken kamen aus seiner Spitze hervor. Professor Elber nickte und bat Harry an die Seite.

Bei jedem Schüler vollzog er dieselbe Prozedur. Bei einigen Slytherins war Harry gespannt, ob sie den Test bestehen würden! Doch Malfoy zeigte keine Spur von Unsicherheit, als er auf seinen Lehrer zuging. Aus seinem Zauberstab stoben ebenfalls grüne Funken. Grinsend ging er an Harry vorbei und stellte sich neben Pansy und hinter Maria.

Einmal stutzte Professor Elber, als er Crabbes Zauberstab überprüfte. Er nahm ihn aus seiner Hand und betrachtete ihn nochmals. Dann sah er ihm in die Augen und meinte, da noch ein knappes Dutzend Schüler übrig waren: \enquote{Bleiben Sie hier neben mir stehen. Ich behalte ihn noch kurz bei mir.} Harry konnte in Crabbes Gesicht Panik erkennen. Nachdem die restlichen Schüler und auch Lehrer überprüft worden waren und auch Professor Dumbledores Zauberstab überprüft wurde, widmete Professor Elber sich wieder Crabbe.

Noch einmal untersuchte er Crabbes Zauberstab und anschließend Crabbe selber. Er flüsterte einen komplizierten Zauber und Crabbes Gesicht veränderte sich. Kleine Wolken kamen aus seinen Ohren und verbanden sich über seinem Kopf. Eine unheimliche Stimme drang daraus hervor. \enquote{Imperio. Du wirst ein Dämonenfeuer nächste Woche im Schloss legen. Pass aber auf, dass niemand dabei zu Schaden kommt. Übe jetzt einmal.} Dann verschwand die Rauchwolke.

\enquote{Mister Crabbe}, fing Professor Elber an. \enquote{Ich muss Sie bitten, Professor Snape zu folgen.} Professor Elber blickte zu Snape und meinte: \enquote{Nehmen Sie ihn nicht zu hart ran. Ich habe das Gefühl, dass er erst nächste Woche das Feuer legen sollte. Für das heute war jemand anderes verantwortlich.}

\enquote{Aber wer?}, schaltete sich Dumbledore ein. \enquote{Wer ist denn übrig? Filch, Madame Pomfrey, die beiden im Krankenflügel und Sie. Sonst war es jemand, der ins Schloss eingebrochen ist.}

Professor Elber gab seinen Zauberstab an Dumbledore, der ihn überprüfte. Es dauerte eine halbe Minute, bis er ihn zurückerhielt und allen bestätigte, dass auch er es nicht gewesen war.

\enquote{Wenn keiner etwas dagegen hat}, machte Professor Elber weiter, \enquote{dann werde ich die letzten Vier überprüfen.}

\enquote{Mister Filch können Sie auslassen}, meinte Professor Flitwick.

\enquote{Sicher? Ich meine zu wissen, dass man unter gewissen Umständen unter dem Imperius auch \gst Nicht-Magiern \gst eine gewisse Menge an Magie geben kann, um den Auftrag auszuführen.}

Dann ging er auf Argus Filch zu und untersuchte ihn. \enquote{Ok. Dann werde ich mal die Letzten drei im Schloss angehen.}

Als er zwanzig Minuten später wieder kam, standen immer noch alle in der Halle herum. Das Feuer hatte sich inzwischen ausgebreitet und die Große Halle mächtig angewärmt.

\enquote{Upps.} Er schlenkerte seinen Zauberstab und das Feuer wurde weniger, bis es schließlich verschwand. \enquote{Negativ bei Poppy}, grummelte er. \enquote{Aber die beiden Schüler waren positiv. Eindeutige Anzeichen vom Imperius und anderen bewusstseinsverändernden Mitteln und Zaubern.}

Die Tische und Bänke kamen langsam von der Decke und Professor Elber setze sich auf eine der Bänke.

\enquote{Ich verstehe nicht, wie so etwas möglich ist!}, sagte Professor McGonagall.

\enquote{Man hat die Schüler wohl während ihrer Ferien unter den Imperius gestellt, um hier Chaos zu verursachen. Leider, oder sollte ich sagen zum Glück, haben sich die beiden nur selber verletzt.}

\enquote{Nein}, intervenierte Dumbledore. \enquote{Was sie meinte war: Wie kommen Sie zu der Ansicht, dass sie unter dem Imperius standen \gst stehen?}

\enquote{Es gibt ein paar Zauber, um das herauszufinden. Und nein, ich glaube kaum, dass ein Auror diese anwenden würde. Es kostet eine Menge Kraft und wird im Allgemeinen der dunklen Seite zugeordnet. Es ist selber eine leichte Form des Imperius, aber legal. Ich teste damit den Willen und ob er gelenkt wird.} Nachdenklich sahen die anderen Lehrer ihn an. \enquote{Ich muss mich erst einmal ausruhen. Meine Möglichkeiten waren begrenzt und auch das Dämonenfeuer zu löschen hat mich sehr beansprucht, die ganze Aktion hat mich sehr mitgenommen.}

Dann stand er auf und verließ mit schweren Schritten die Große Halle.

\enquote{Warum hast du ihn gewähren lassen?}, fragte McGonagall Dumbledore leise.

\enquote{Weil er mehr über Hogwarts und Magie weiß, als wir beide zusammen.} Dann ging auch Dumbledore.

Das brachte Harry zum Nachdenken. Er schlenderte durch das Schloss, begleitet von Ginny. Und wieder einmal spürte er eine Lampe, die nicht wie die anderen war. Er entfernte sie und ersetzte sie durch eine dauerhaft funktionierende. Ginny half ihm dabei, da er ihr den Zauber schon vor Wochen gezeigt hatte. Zusammen hatten sie bereits mehr als sechzig Lampen repariert.

In Salazars Räumen angekommen, kuschelten sich beide aneinander. Das letzte Schuljahr lief vor Harrys geistigem Auge vorbei. Bei der Unterhaltung mit Dumbledore über die Bilder von Hogwarts blickte er automatisch zu Salazar.

\stimme{Ein interessanter Einblick, den du uns da gewährst}, tönte es in seinem Geist. \stimme{Ich finde es nur Schade, dass man euch heutzutage nicht mehr das lehren kann, was wir damals gelehrt haben.}

\gedanke{Wie meinst du das?}

\stimme{Ich meine die alte Magie. Man hat über die Jahrhunderte scheinbar viel vergessen. Zumindest du bist in diese Richtung unterwiesen worden.}

\gedanke{Du meinst unseren \VgddK-Lehrer!}, folgerte Harry.

\stimme{Nenn ihn ruhig beim Namen \gst Frederick. Er hat dir die Sachen beigebracht, die man früher gelehrt hat.}

\gedanke{Du meinst die alte Magie?}

\stimme{Ja.}

\gedanke{Woher kennt er die?}

\stimme{Auf die Frage muss ich dir eine Antwort schuldig bleiben. Kümmere dich lieber um deine Freundin.}

Harry nickte und griff Ginny unter ihr Kinn, um sie vom Lesen abzuhalten. Seit sie hier waren, hatte sie ununterbrochen gelesen. Zuerst sanft, dann etwas wilder begann er sie zu küssen, bis sie schließlich eng umschlungen auf dem Sofa lagen. Harry unten und Ginny auf ihm.

\enquote{Ich liebe dich, Harry}, flüsterte Ginny.

\enquote{Ich liebe dich auch, Ginny}, antwortete Harry. \enquote{Wie viele Kinder wolltest du später mal?}

\enquote{Drei mindestens. Du weißt, ich komme aus einer kinderreichen Familie. Ich gebe mich mit einem nicht zufrieden.}

Harry grinste sie an und küsste sie dafür.

\trenn

Es war der Tag des letzten Quidditch-Spieles. Der Tag, an dem sie gegen Hufflepuff spielen würden. Dieses Jahr hatte Hufflepuff eine gute Saison hinter sich gebracht und mussten nur noch Gryffindor schlagen, damit sie den Quidditch-Pokal gewinnen würden. Harry war auf seinem Besen in der Luft und suchte das Spielfeld sowie den Himmel ab. Er zog kleine Kreise, um in Bewegung zu bleiben. In der gegnerischen Hälfte des Spiels blieb er irgendwann stehen und blickte wieder über das Feld.

\begin{rueckblick}
\enquote{Ist dir auch aufgefallen, dass Myrte in letzter Zeit fröhlicher wirkt?}, fragte Hermine ihn.

\enquote{Nein}, antwortete Harry, \enquote{nimm mich halt nächstes Mal mit, wenn du dort auf die Toilette gehst.}

Hermine streckte ihm die Zunge raus und verzog ihr Gesicht.

Er wusste genau, wovon sie sprach, aber er konnte ihr doch nicht sagen, dass er mit Myrte unglaublichen Sex gehabt hatte. Sie würde ihm entweder nicht glauben, oder aber würde ihn für krank halten, ihn sogar der Nekrophilie beschuldigen.
\end{rueckblick}

%Der Begriff Nekrophilie bezeichnet eine Sexualpräferenz, die auf Leichen gerichtet ist. Nekrophilie ist im ICD-10-Verzeichnis der psychischen Störungen unter „Sonstige Störungen der Sexualpräferenz“ (F65.8) als Paraphilie klassifiziert.

Dann sah er plötzlich den Schnatz und innerhalb weniger Augenblicke durchzuckten ihn verschiedene Gedanken. \gedanke{Wenn ich jetzt den Schnatz fange, dann haben wir sechsmal hintereinander den Quidditch-Pokal gewonnen. Sechsmal. Jedes Mal, seit ich im Team spiele. Hufflepuff hatte seit mehr als dreißig Jahren keinen Pokal mehr gewonnen. Wenn aber Lisa jetzt den Schnatz fangen würde, \abs}

Er folgte dem Schnatz mit seinen Augen. Die der gegnerischen Sucherin waren auf seine gerichtet. \gedanke{Los, komm schon Lisa, dreh dich um und fang ihn}, durchzog es Harry. \gedanke{Ich werde ihn jetzt noch nicht nehmen.} Seine Augen folgten unablässig dem Schnatz. Die Hufflepuff-Sucherin folgte für einen kurzen Moment seinem Blick, um ihm danach wieder in die Augen zu sehen. Dann registrierte sie, dass sie den Schnatz erblickt hatte. Blitzschnell drehte sie sich um und raste auf ihn zu. \gedanke{Na also, jetzt nur nichts anmerken lassen.} Harry gab Gas und flog ihr hinterher. Nun jagten beide den Schnatz.

\enquote{Und Harry Potter und Lisa Dervall jagen den Schnatz. Wollen wir hoffen, das der richtige Sucher ihn fängt}, trällerte Lee Jordan über das Spielfeld. Professor McGonagall warf ihm einen bösen Blick zu. \enquote{Ich habe keine Namen genannt}, sagte er daraufhin ebenso laut. \enquote{Jetzt hängt alles davon ab, wer den Schnatz fängt und somit den Quidditch-Pokal gewinnt. Hufflepuff \gst \extase{oder} \gst Gryffindor.} Das Gryffindor schrie er mit mehr Begeisterung heraus.

\gedanke{Kein Wunder, ist ja auch sein Haus}, dachte Harry. Beide Sucher waren jetzt auf gleicher Höhe. Vierzig Meter über dem Boden. Beide waren gleich auf. Kaum hatte Lisa den Schnatz in der Hand, wurde sie auch schon von einem Klatscher getroffen und fiel vom Besen Richtung Boden. Ohne lange zu überlegen, stürzte Harry ihr hinterher. Er bekam nicht mehr bewusst mit, wie Lee verkündete, dass Hufflepuff das Spiel gewann und somit auch den Quidditch-Pokal.

Er bekam sie an ihrer Quidditch-Robe zu packen und hielt wenige Zentimeter über dem Boden mit ihr in der Luft an. \enquote{Lisa, stütze dich am Boden ab, ich kann dich nicht mehr lange\abs} Lisa tat, was Harry ihr sagte. Kaum berührte sie den Boden, ließ Harry sie auch los und stieg von seinem Besen. Er ließ ihn senkrecht in der Luft schweben und kniete neben Lisa hin.

\enquote{Danke, Harry}, sagte sie glücklich und hielt den Schnatz in ihrer Hand. Sie öffnete sie leicht damit Harry ihn auch sehen konnte.

\enquote{Herzlichen Glückwunsch zum Sieg}, sagte Harry und strich ihr mit seiner Hand über den Rücken.

\enquote{Komm nach dem Spiel nochmal auf die Tribüne, ich möchte dir noch was sagen}, sagte Lisa.

Harry nickte. Mehr konnten sie nicht mehr besprechen, da der Trubel zu laut und die auf sie stürmenden Leute zu nah waren.

Sie hoben Lisa als ihre Gewinnerin in die Luft und trugen sie vom Spielfeld. Harry konnte nur innerlich lachen. Ihm war es egal, dass sie dieses Jahr den Pokal nicht gewannen. Sie hatten die letzten fünf Jahre McGonagall den Pokal für ihr Büro beschert. \gedanke{Würde sie es verstehen, wenn sie es herausfinden würde?}, fragte sich Harry. Er schob den Gedanken beiseite und wartete erst einmal, ob seine Kameraden überhaupt etwas davon mitbekamen.

\enquote{Harry, du brauchst dich nicht von uns zu trennen. Wir machen dir keinen Vorwurf, dass du den Schnatz nicht rechtzeitig geschnappt hast. Hufflepuff hatte dieses Mal einfach mehr Glück und sie haben den Pokal genauso verdient, wie jede andere Mannschaft}, sagte Ginny, die im Umkleideraum stand und ihre Jäger-Sachen verstaute. \enquote{Eigentlich ist es sogar gut. Das gibt den Hufflepuffs mal wieder richtig Mut. Vielleicht werden die Spiele gegen sie dann etwas herausfordernder. Bisher waren sie doch recht langweilig, meine ich.}

\enquote{Ich weiß}, antwortete Harry, \enquote{ich brauche nur kurz etwas Zeit für mich alleine. Es hat nichts mit dem Schnatz zu tun.} Er verließ die Umkleiden und setzte sich auf die Ränge des Spielfeldes. Er wartete auf Lisa.

Mittlerweile hatte sich das Feld geleert und nur noch Harry saß auf den Rängen. Er wartete, ließ sich verschiedenen Dinge durch den Kopf gehen und begann schließlich mit seinen Okklumentik-Übungen. \gedanke{Lisa wird schon noch kommen}, sagte sich Harry immer wieder. \gedanke{Hufflepuffs sind zuverlässig.} Nach zwanzig Minuten tauchte sie schließlich auf und setzte sich sehr nah neben Harry. Gerade so, dass sie ihn nicht berührte, er aber immer wieder ihren Atem spürte, wenn sie ihn ansah.

\enquote{Danke, Harry}, sagte sie sanft.

\enquote{Das hätte jeder gemacht}, antwortete Harry.

\enquote{Nein, das glaube ich nicht.}

\enquote{Doch, jeder andere hätte dich auch aufgefangen, als du auf den Boden zugerast bist.}

Lisa schaute ihn an. Er spürte ihren Atem auf seiner Backe. Sie rückte ein Stück näher an ihn heran. Nun berührte sie ihn wirklich.

\enquote{Ich meine nicht, dass du mich aufgefangen hast.} Erstaunt sah Harry jetzt Lisa an, ihr direkt in die Augen. Sie hatte eine schwarze Iris mit einer Spur gelb, wie Harry es noch nie gesehen hatte. \enquote{Ich meinte damit, dass du uns den Pokal geschenkt hast.}

Harry spielte den erstaunten. \enquote{Ich habe euch nicht\abs Du hast mich doch gesehen. Ich bin dir hinterher und war gleich auf mit dir.}

\enquote{Tu nicht so, Harry. Ich kann dich verstehen. Du hast den Schnatz schon lange vor mir gesehen. Du hättest einfach nur auf ihn zu fliegen müssen, hättest ihn weit vor mir haben können. Außerdem ist dein Besen schneller als meiner. Du hättest mich bei der Jagd überholen können. Aber du tatest es nicht.}

\enquote{Ich war in Gedanken. Ja, ich habe den Schnatz gesehen, war aber in meinen Gedanken vertieft}, sagte Harry entschuldigend.

\enquote{Aber du hättest mich überholen können und hast es trotzdem nicht getan.} Harry sah wieder geradeaus und sagte nichts mehr. \enquote{Harry?}

\enquote{Hm!}

\enquote{Darf ich dir dafür, als kleines Dankeschön, einen Kuss geben?}

Er sah Lisa wieder an, betrachtete ihr gewelltes kurzes dunkelbraunes Haar, welches sie zu einem Pferdeschwanz zusammengebunden hatte.

\enquote{Du hast mich doch schon geküsst. Erinnerst du dich nicht mehr daran?}, fragte Harry ausweichend.

\enquote{Das schon, aber da stand ich unter fremdem Einfluss. Da hattest du eine Wirkung an dir, die ja alle anderen Mädchen auch zu spüren bekamen. Und ich möchte wenigstens einmal einen Kuss von dir, ohne dass ich dazu\abs gezwungen werde.}

Harry betrachtete nun ihre wenigen Sommersprossen auf ihrer Nase und unter ihren Augen. Schließlich öffnete er leicht seinen Mund und legte seinen Kopf schief. Langsam kam sie ihm näher. Wenige Millimeter vor seinem Mund, er konnte ihre Lippen beim Sprechen fast schon spüren, sagte sie: \enquote{Danke.}

Dann küsste sie ihn kurz aber intensiv. Harrys ganzer Körper empfing nun Lisas Kuss. Dann sah sie ihn wieder an.

\enquote{Ich werde keinem davon erzählen}, sagte sie. \enquote{Nicht davon, dass du uns den Sieg einfach gemacht hast, und auch nicht von unserem Kuss.}

\enquote{Dein Kuss}, korrigierte sie Harry.

Lisa lächelte und ging. Harry saß noch eine Weile und betrachtete die Umgebung, als er Schritte neben sich hörte.

\enquote{Hi, Harry.}

\enquote{Hi, Ginny, bist du gerade gekommen?}

\enquote{Nein, ich war schon vor Lisa da und habe euch dann beobachtet.}

Harry sah jetzt mit Schrecken in Ginnys Gesicht, doch diese rutschte näher an ihn heran. \enquote{Ich weiß es}, sagte sie. Harry wurde ganz bleich im Gesicht. \enquote{Ich weiß, dass du Hufflepuff den Sieg fast geschenkt hast und dass Lisa sich bei dir\abs bedankt hatte.} Ihre letzten Worte, dachte Harry, waren nicht allzu freundlich. Um dem Ganzen die Spannung zu nehmen, küsste er seine Freundin. Diese stieß ihn jedoch nicht weg, wie Harry zuerst vermutete, sondern ließ ihn einfach.

\enquote{Interessanter Lippenbalsam, dass Lisa da verwendet}, sagte Ginny, als sie sich von Harry löste. Dann sagte sie in einem strengen Ton zu Harry: \enquote{Dass mir so etwas nicht noch einmal passiert, Harry James Potter.}

Harry schluckte. \enquote{Ja Ma'am}, sagte er schuldbewusst, im Klaren darüber, dass er von seiner Freundin erwischt worden war und senkte seinen Kopf.

\enquote{Ich hab dich viel zu gern, Harry}, sagte sie und zog ihn zu einem langen Kuss zu sich ran, \enquote{als dass ich dir dafür lange böse sein könnte. Aber mach so etwas ja nicht noch einmal}, ermahnte sie ihn wieder. \enquote{Ich kann ihre Beweggründe verstehen und auch, dass du sie gewähren hast lassen. Aber\abs}

Und Harry verstand. Sie schmusten noch eine Zeit lang, bis es schließlich beide zurück ins Schloss zog. Dort angekommen sah er Professor Elber, der sich mit Sirin unterhielt.

\enquote{Und, wie gefiel Ihnen Ihr Wohlfühltag?}, fragte er.

\enquote{Gut. Vor allem der Egalisierungs-Zauber hat es mir angetan. Bringen Sie uns den auch mal bei?}

\enquote{Was bezeichnen Sie als Egalisierungs-Zauber?}

\enquote{Na ja, den Zauber, der die Peinlichkeit und die Scham minderte. Wir waren dort alle ohne etwas an und hatten weder Scham noch etwas anderes. Das ist mir erst hinterher in den Sinn gekommen. Das ist ein toller Zauber.}

\enquote{Ach, diesen meinen Sie. Ja, der ist schon praktisch. Er liegt schon lange auf dem Raum. Jetzt, da Sie vier Kenntnisse davon haben, können Sie jederzeit dort hinein. Es wird Ihnen immer wieder möglich sein, dort zu baden, zu duschen, oder\abs andere Sachen zu machen.} Beim letzten Absatz grinste er sie leicht an.

Sirin wurde leicht rot.

\enquote{Was glauben Sie, was ich in diesem Raum früher für Mädchen\abs}, wieder grinste er leicht. Sirin wurde knallrot bei diesem Gedanken. Jetzt lachte er aus voller Kehle. \enquote{Sie müssen aufpassen, dass Sie nicht zu früh in eine peinliche Situation kommen. Vierundzwanzig Stunden sollten sonst schon dazwischen liegen. Denn sonst wird es umso peinlicher. Beim ersten Mal braucht es bei manchen noch etwas länger.} Er strich ihr sanft über die Wange. \enquote{Aber je peinlicher die Situation, desto schneller vergeht sie und desto weniger wird es die nächsten Male. Es tritt ein Gewöhnungseffekt ein.}

Da hatte Harry einen Gedanken, den er beizeiten mit Ginny ausleben wollte.

\enquote{Aber wie kommen Sie darauf, dass ich ihn kennen würde?}

\enquote{Was meinen Sie?}

\enquote{Den Zauber. Sie fragten mich, ob ich Ihnen den beibringen würde.}

\enquote{Na ja}, sagte Sirin, \enquote{ich schätze Sie so ein, dass Sie solche Sachen wissen.}

\enquote{Haben Sie Zeit?}, fragte er.

\enquote{Ja, warum?}

\enquote{Dann kommen Sie mit. Dumbledore wollte den Zauber auch lernen. Zwar aus einem anderen Grund, aber es interessiert ihn. Wissen Sie, es ist leichter einen Zauber von jemandem zu lernen, als aus einem Buch.}

Damit verschwanden beide in einem der Gänge des Schlosses.

Am nächsten Tag waren Prüfungen. Heute hatte die gesamte Schule Prüfung im Fach \VgddK.

\gedanke{Seine letzte Tätigkeit als Lehrer}, ging Harry durch den Kopf.

Die gesamte Schule versammelte sich auf einer großen Freifläche und starrte auf eine große milchige Halbkuppel, die auf einer Backsteinmauer mit einer Höhe von einem halben Meter stand. In der Mitte war ein Durchgang aus Holz zu sehen, dessen Mitte komplett dunkel war. Es war fast so, als ob da nichts wäre. Es war nicht einfach schwarz, da war einfach nichts, so hatte man den Eindruck.

\enquote{Ich mache es kurz}, sagte der Prüfer, der die Prüfung abnahm. \enquote{In diese Kuppel gehen Sie mit etwa drei Sekunden Abstand. Keine Sorge, Sie passen alle da rein. Im Inneren werden Ihnen verschiedene Aufgaben gestellt werden, die Sie mehr oder weniger erfolgreich absolvieren werden. Mit der Zeit wird es immer schwieriger, bis irgendwann der Punkt erreicht ist, an dem Sie aufgeben müssen. Es wird Ihr Fortschritt innerhalb der Kuppel bewertet, sowie Ihre Lösungen, beziehungsweise Lösungsansätze.} Er pausierte kurz. \enquote{Dann fangen wir mit der siebten Jahrgangsstufe an. Bitte treten Sie vor, drei Schritte Abstand und laufen Sie langsam in das Innere der Kuppel. Sie werden einander bis zum Ausgang nicht mehr begegnen.}

Die Siebtklässler nahmen ihre Stellung ein und schritten nacheinander durch den Bogen in das Innere. Als Nächstes war Harrys Jahrgangsstufe dran.

Mit den Dementoren am Eingang war er gleich fertig. Er schickte ihnen einen Patronus entgegen. Dann musste er durch ein Gewässer tauchen, da ihm ein unbekanntes Feld den direkten Weg mit dem Boot verwehrte. Harry überlegte kurz und belegte seine Kleidung mit einem wasserabweisenden Zauber und tauchte. Doch er hatte nicht mit Wasserlebewesen gerechnet, die ihn unter Wasser behalten wollten und nach unten zogen. Es dauerte eine Weile, bis er wieder wusste, wie er sich dagegen wehren konnte. Einige konnte er durch ein paar Brandflecken mit seinem Zauberstab auf deren Haut abwehren, anderen versuchte er die Finger zu brechen, damit sie von ihm abließen.

So kämpfte er sich durch die schwerer werdenden Aufgaben.

Dann wurde es zunehmend kälter. Harry zauberte sich eine warme Jacke herbei und zog sie an, doch sie spendete ihre Wärme nicht lange. Auch Wärmezauber schienen bei weitem nicht so effektiv zu sein. \gedanke{Wenn ich doch nur dieses ewige Feuer richtig könnte}, ging ihm durch den Kopf. Dann kam ihm etwas in den Sinn. \gedanke{Lebendiges Feuer. Das ist doch die gleiche Art der Magie. Deshalb haben die Siebtklässler das durchgenommen. Es wird Teil ihrer Prüfung sein; durch die Kälte mit diesem Feuer zu kommen.} Harry grinste in sich hinein. Leicht fröstelnd ließ er die kalte Passage hinter sich und widmete sich der nächsten Aufgabe. Er musste durch einen Dschungel. Er hatte während des Schuljahres viele Wesen durchgenommen, die im Dschungel wohnten, davon viele magische Schlangenarten, und auch diese komischen Wesen, die wie Äste aussahen.

Und kürzlich hatten sie erst diese blauen kleinen Wesen durchgenommen: Medusoner wurden sie genannt. Sie waren normalerweise harmlos, doch wenn man ihr Revier ohne Einladung betrat, konnten sie ungemütlich werden und angreifen. Harry versuchte sich zu erinnern, woran man deren Reviergrenzen erkannte. Als es ihm wieder einfiel, bemerkte er, dass er vor wenigen Schritten an einem Hinweis vorbeigegangen war.

Er blieb stehen und lief die wenigen Schritte rückwärts und betrachtete den Hinweis, der die Form eines kunstvoll aus Lianen gewundenen Pfahls hatte. Harry betrachtete den Pfahl eine Weile und überlegte, in welche Richtung er am besten gehen sollte. Beide Wege der Abzweigung, die nur zehn Meter zurücklag, konnten ein Umweg von mehreren Kilometern sein. \gedanke{Einer der beiden ist bestimmt doppelt so lang wie der andere}, dachte Harry.

Dann hörte er etwas rascheln. Er drehte seinen Kopf in die Richtung, aus der das Geräusch kam und fragte in Richtung des Busches: \enquote{Sind sie ein Medusoner? Wenn ja, hätte ich gern die Erlaubnis, das Gebiet ihres Stammes zu durchqueren.}

Ein blauer Kopf mit angewachsenem Lamellenpilz-artigem Hut schaute hervor.

Harry sah das Wesen und stellte sich vor. \enquote{Mein Name ist Harry Potter und ich erbitte die Erlaubnis, dieses Gebiet zu durchqueren.}

\enquote{Was bekommen wir dafür?}

\enquote{Was hättet ihr denn gern?}

\enquote{Basiliskenschuppen.}

\enquote{Habe ich nicht bei mir. Wie wäre es mit etwas zu essen?}

\enquote{Ja, das geht auch}, sagte der Medusoner, als wäre es nichts Besonderes so etwas Wertvolles gegen etwas zu Essen einzutauschen.

\enquote{Was hätten Sie denn gern?}

\enquote{Schnecken und Früchte.}

Harry zauberte einen großen Früchtekorb herbei, den er mit einem Zauber wenige Millimeter über dem Boden schweben ließ.

Ein zweiter Kopf kam aus dem Gebüsch heraus. Die beiden blauen Wesen schauten sich an und nickten einander zu. Einer kam auf den Früchtekorb zu und nahm ihn mit. Der andere kam auf Harry zu und verlangte, dass er ihn während der Durchquerung des Stammesgebietes zu tragen habe, damit keiner angreifen würde.

Harry akzeptierte die Bedingung und nahm das blaue Wesen auf die Schulter.

Während der Durchquerung entwickelte sich eine kleine Unterhaltung zwischen den beiden. Er fragte ihn zu seiner Prüfung aus. Doch mehr als ein: \enquote{Wir melden das Verhalten der Prüflinge. Alles andere liegt bei Ihrem Prüfer}, bekam er nicht zu hören.

Nachdem Harry das Stammesgebiet durchquert hatte, setzte er seinen kleinen Begleiter ab, verabschiedete sich von ihm und setzte seine Prüfung fort. Dann kam er an ein Labyrinth. Er fühlte sich an das trimagische Turnier erinnert. Inzwischen kannte er mehr Zauber, so auch einen Zauber, den er in einem Buch während seiner Strafarbeit in der Bibliothek gefunden hatte. Er wandte ihn an und wartete ein paar Minuten, bis sich der Zauber entwickelt hatte. Dann sah er an jeder Abzweigung und Kreuzung unterschiedlich dicke Linien. Harry folgte der jeweils dicksten Linie und fand so den Weg aus dem Labyrinth. Der Zauber war ein sogenannter Schleimpilzzauber. Diese Pilze konnten sehr schnell und effektiv den besten Weg zu einer Nahrungsquelle finden. Der Zauber funktionierte ähnlich. Er fand den schnellsten Weg zum Ausgang.

Am Ende des Labyrinthes stand er vor einer Felswand. Er sah sich um, entdeckte aber keinen Weg, der drumherum führte, er musste also klettern. Doch eine Kletterausrüstung konnte er sich nicht herbeizaubern. Auch jeder andere Zauber versagte. Also musste er von Hand klettern und machte sich an den Aufstieg. Doch kurz vor dem Ziel rutschte er ab und fiel.

Er landete weich. Als er aufstand, war kein Labyrinth mehr vorhanden. Nur eine weite Wiese mit einem Torbogen. Als er näher kam, erklang eine Stimme. \stimme{Prüfung abgeschlossen. Sie erfahren Ihr Ergebnis draußen.}

Harry trat durch den Torbogen und fand sich außerhalb der Kuppel wieder. Einige seiner Mitschüler, die nach ihm hineingegangen waren, sah er bereits. Sie mussten schon vorher ihre Grenzen erreicht haben.

Harry hatte ein Ohnegleichen in diesem Fach und war vollauf mit sich zufrieden.

\stimme{Warum hast du mich bei dem Feuer nicht gefragt, wie der Zauber geht, Harry?}, fragte ihn Salazar in seinem Geist.

\gedanke{Das wäre den anderen gegenüber nicht fair gewesen.}

\stimme{Nicht fair? Junge, das ist ein Teil von dir. Dieses Wissen hast du bereits. Hättest du dich etwas konzentriert, dann wäre es dir eingefallen.}

\gedanke{Warum sagst du mir das erst jetzt?}

\stimme{Das weißt du schon länger. Das habe ich dir schon früher gesagt.}

\gedanke{Tut mir leid, habe ich wohl vergessen.}

In einem unbeobachteten Moment auf dem Rückweg zum Schloss ließ er, nachdem er angestrengt nachgedacht hatte, eine kleine Flamme auf seiner Handfläche erscheinen, sie dann unter seinem Ärmel nach oben gleiten und dann ganz zart über seinen Brustkorb und den Rücken verteilen. Er ließ sich etwas wärmen, bis der Zauber endete. Er hielt nicht lang, aber er schaffte es doch.

\trenn

Harry hatte seine letzte Prüfung hinter sich gebracht und lief gerade in die Große Halle, um zu Abend zu Essen. Mitten im Raum schwebte eine große Version des komischen Würfels, den sie zum Öffnen bekommen hatten. Dahinter der gleiche Würfel, nur ohne Farbe. Und wiederum dahinter die geöffnete Version. Die pyramidalen Ecken mit den dreieckigen Grundflächen hatten sich um 60 Grad gedreht und das Innere der Kreise an den sechs Seiten war entfernt worden. Im Inneren schwebte eine kleine Kugel, die das bläuliche Licht ausstrahlte. Die Große Halle war bereits zu drei Vierteln gefüllt. Harry sah sich um. Ihm fiel nur ein Würfel auf, der in der Luft schwebte und normale Größe hatte. Er war geöffnet. Harry musste ein paar Schritte weitergehen, um zu sehen, über wem der Würfel schwebte.

\gedanke{Bringen Sie am Abend des letzten Prüfungstages Ihren Würfel mit. Egal ob Sie ihn öffnen konnten, oder nicht}, hatte Professor Elber sie angewiesen.

Von seiner neuen Position aus konnte Harry erkennen, wer unter dem Würfel saß. Es war Luna. Sofort ärgerte er sich, dass ihm Luna nichts darüber erzählt hatte. Doch im nächsten Moment war ihm klar, dass das doch Betrug gewesen wäre. Er griff in seine Tasche und besah sich seinen Würfel.

\gedanke{Konzentrieren Sie sich auf den Würfel, bis Sie ihn öffnen können.}

\gedanke{Warum bin ich nicht früher darauf gekommen}, dachte Harry. Plötzlich war alles klar. Das ganze Jahr über hatten sie darauf hin gearbeitet. Sie übten Zauberstabslose Magie. Sie ließen zwar nur ihre Zauberstäbe und Besen auf sich zukommen, aber das war ein Schritt auf dem Weg. Und vor allem, was Professor Elber sagte. \gedanke{Konzentrieren Sie sich auf den Würfel, bis Sie ihn öffnen können.} Es war so einfach. Er warf den Würfel vor sich in die Luft und konzentrierte sich darauf, dass er schweben blieb. Der Würfel fiel noch einige Zentimeter runter, bis ihn Harry vor seinen Augen stabilisieren konnte. Dann schloss er seine Augen und stellte sich sehr genau vor, wie er den Würfel öffnen würde. Er öffnete die Augen und sah, wie die pyramidalen Ecken des Würfels abhoben, das Innere der Kreise langsam verschwand und die Ecken sich langsam zu drehen begannen. Dann kamen die Ecken zurück und rasteten hörbar ein.

Mit zufriedenem Gesichtsausdruck setzte er sich zum Abendessen hin. Der Würfel folgte ihm in sicherem Abstand über seinem Kopf. Nun schwebten zwei kleine Würfel in der Großen Halle. Die großen Würfel drehten sich immer noch langsam unter der Wolken-behangenen verzauberten Decke. Er fing Dumbledores Blick auf, dessen Würfel verschlossen, aber von Farbe befreit vor ihm lag. Harry musste sich beherrschen, seinen Kürbissaft nicht über dem Tisch zu verteilen. Er hätte erwartet, dass zumindest Dumbledore von allen Lehrern es schaffen würde. Denn inzwischen lagen vor sämtlichen Lehrern ebenfalls solche Würfel.

Doch Harry war nicht vorbereitet, nicht auf das, was gleich passieren würde, denn er verteilte seinen Kürbissaft dennoch über den Tisch. Zum Glück saß ihm niemand gegenüber, als Professor Elber die Halle betrat, lief er bis zum oberen Ende der Tische. Er blieb stehen und schaute auf Dumbledores Würfel. Dumbledore fing seinen Blick auf und fuhr danach mit seiner Hand vor dem Würfel vorbei. Dieser hob leicht vom Tisch ab und die Ecken entfernten sich, das Innere der Kreise verschwand und die Ecken rasteten gedreht wieder ein. Danach schwebte der Würfel wenige Zentimeter über dem Tisch.

Harry hustete, als er sah, wie beiläufig Dumbledore den Würfel öffnete. Harry wusste nicht mehr, was er während seines Hustenanfalls herausbrachte, aber kurz darauf erschien Kreacher. Er sah Harry an und danach die Sauerei, die er hinterlassen hatte. Er verschwand, um kurz darauf wiederzukommen; mit einem Putzlappen in der Hand. In Windeseile wischte er den verschütteten Kürbissaft auf und erneuerte die angespuckten Speisen und Getränke. Danach verschwand er wieder. Harry war froh, dass er sich mit Kreacher nun besser verstand. Seine Hetzreden waren nicht mehr zu hören. Er war ihm gegenüber jetzt vollständig loyal geworden. Auch nannte er ihn seit geraumer Zeit nicht mehr Meister, sondern nur noch \accentuate{Sir Harry.}

Harry sah sich immer wieder um, doch außer seinem Würfel und dem von Luna, war keiner geöffnet. Abgesehen von dem Würfel Dumbledores. Die Tore der Großen Halle schlossen sich, nachdem der letzte Schüler die Halle betreten hatten und signalisierte somit, dass jetzt alle Anwesend seien. Nachdem alle gegessen und getrunken hatten, stand Dumbledore auf und trat hinter sein Pult. Die Flügel der Eule breiteten sich aus und so als ob jemand einen Schweigezauber auf den Saal gelegt hätte, verstummte das Gemurmel.

\enquote{Ein weiteres Jahr haben wir jetzt hinter uns gelassen. Ein Jahr voller Aufregung und ein Jahr des Lernens. Obwohl die Bedrohung dort draußen beständig gewachsen ist, haben wir uns bislang gut geschlagen. Ihr seht hier über euch an der Decke drei Würfel. Diese Würfel mussten einige von euch bis zum Jahresende öffnen. Aber ich denke, dass dieses Rätselfieber sich bereits ausgebreitet und auch uns Lehrer erwischt hat. Ich muss zugeben, es war nicht leicht, dahinterzukommen, wie diese Würfel zu öffnen sind.} Er ließ seinen Blick durch die Halle schweifen. \enquote{Wie ich sehe, haben es nur zwei Schüler geschafft, ihre Würfel zu öffnen.} Er drehte sich um und blickte zum Lehrertisch, um sich danach wieder den Schülern zuzuwenden. \enquote{Und kein Lehrer.} Er grinste leicht. \enquote{Ich möchte nun Professor Elber bitten!}

Professor Elber stand auf und während er auf das Pult zulief, trat Dumbledore zurück. \enquote{Danke, Albus}, sagte Professor Elber und stellte sich vor das Pult mit der Eule.

\enquote{Wie Professor Dumbledore bereits sagte, sind es nur zwei Schüler, die begriffen hatten, was ich damit meinte, als ich euch sagte: \enquote{Konzentrieren Sie sich auf den Würfel, bis Sie ihn öffnen können.} Dies gilt noch immer. Aber um Ihnen den ersten Schritt zu erleichtern: Entfernen Sie einfach mal die Farbschicht von Ihrem Würfel.}

Jetzt erfüllte allgemeines Rascheln die Große Halle und viele kleine Wolken aus Schwarz verpufften im Nichts.

\enquote{Konzentrieren Sie sich auf den Würfel, bis Sie ihn öffnen können}, sagte Professor Elber erneut. Er zog einen Würfel aus seiner Tasche. Die Farbe war bereits entfernt. Er vergrößerte ihn mit seinem Zauberstab und lies ihn vor sich schweben.

\enquote{Stellen Sie sich vor Ihrem geistigen Auge den Würfel vor. Schauen Sie ihn sich genau an. Dann schließen Sie die Augen und stellen sich vor, dass der Würfel sich öffnen würde.} Er schloss die Augen und der Würfel begann sich zu öffnen. Dann öffnete er wieder seine Augen und meinte: \enquote{Bitte, versuchen Sie es alle. Sollte es nicht klappen, haben Sie die Ferien über Zeit zum Üben. Den Würfel können Sie behalten. Sie können ihn jederzeit öffnen und schließen. Sogar kleine Dinge lassen sich in ihm verstecken, wenn sie die Lichtkugel entfernen.} Er drehte sich wieder um, lief zurück und setzte sich.

Harry war erstaunt, wie um ihn herum viele Würfel langsam zu schweben begannen, teilweise zusammenstießen, aber sich dennoch öffneten. Es schafften nicht alle. Kaum einer aus den ersten drei Schuljahren hatte seinen Würfel öffnen können. Er lächelte Tamara an, die erschöpft, aber glücklich aussah, als ihr Würfel vor ihr schwebte. \enquote{Eine echte Malfoy}, sagte er zu ihr.

Sie schaute an Harry vorbei zu ihrem Bruder und meinte: \enquote{Draco hat es auch geschafft.}

\enquote{Die unteren Jahrgangsstufen könnten damit eventuell Probleme haben. Lassen Sie sich nicht entmutigen. Versuchen Sie es beständig}, sagte Professor Dumbledore und ermutigte die Schüler.

\trenn

\enquote{Warum sind wir hier, Harry?}, fragte Ginny, als sie wieder diesen Gang mit den angestaubten Rüstungen entlang liefen.

Harry hatte mittlerweile geübt, den \spruch{Protego}-Zauber ungesagt und ohne Zauberstab auszuführen. Er antwortete nicht gleich und sagte nach einigen Metern: \enquote{Warte es ab, Ginny. Hast du deine Badesachen?}

Sie bejahte und lief neben ihm her. Sie nahm seine Hand in ihre und beide liefen an den Rüstungen vorbei. Die wenigen Elfen, die sie trafen, verbeugten sich kurz und arbeiteten dann weiter. Sie verschwanden nicht mehr nach Harrys Rede.

\gedanke{Das könnte nächstes Jahr interessant werden}, dachte sich Harry, \gedanke{wenn die Elfen im Gemeinschaftsraum sind, wenn ich alleine noch arbeite, oder sonst irgendwas tue.}

Sie waren mittlerweile bei dem Porträt angekommen. Harry sah hinauf zur blauen Blume und dachte: \gedanke{Protego}. Sie hörten ein leises \geraeusch{Klick}-Gearäusch und der Bilderrahmen sprang wenige Millimeter vor.

Harry öffnete den Rahmen und ließ Ginny den Vortritt. Sie lief voraus und Harry folgte ihr. Als sie um die Ecke sah, blickte sie sich erst einmal vor Erstaunen um. Dann drehte sie sich Richtung Ausgang, öffnete einen der Spinde und zog ihre Kleidung aus; auch ihren Badeanzug. Dann lief sie in das Becken und setzte sich so hin, dass sie jeder sehen konnte, der auf das Becken zuging.

Harry hatte sich auf diese Situation vorbereitet und zog sich ebenfalls aus. Dann trat er um die Ecke, lief auf das Becken zu und verschwand ebenfalls bis zum Bauchnabel im Wasser. Er setzte sich Ginny gegenüber auf den Sitz unterhalb der Wasseroberfläche und sah Ginny an.

Beide betrachteten sich nun von Kopf bis Fuß. \gedanke{Selbst wenn sie ihren Badeanzug anbehalten hätte, würde es mir nichts ausmachen.}

Ginny bewegte sich nicht, da sie Harrys Körper betrachten wollte. Harry tat es ihr gleich.

Dann schwamm sie zu ihm herüber, setzte sich auf seine Knie und legte ihre Stirn gegen seine. \enquote{Warum sind wir hier?}, fragte sie.

\enquote{Ich muss mit dir reden, Ginny.}

\enquote{So?}, fragte sie ihn.

\enquote{Du hast angefangen. \gst Ja, so.}

Sie lächelte leicht. \enquote{Was sagte deine Freundin dazu?}

\enquote{Ich hab mich doch von Pan\aabs Ähh\abs Es scheint ihr nichts auszumachen. Zumindest hat sie sich nicht negativ geäußert.}

\enquote{Hast du sie gefragt?}, kam jetzt mit etwas Nachdruck.

\enquote{Kann ich kurz machen. \gst Ginny, hast du was dagegen?}

Statt einer Antwort küsste sie ihn.

\enquote{Tut mir leid, aber ich habe keine eindeutige Antwort erhalten. Sie hat mir nonverbal mit einem Kuss geantwortet. Reicht dir das?}

Ginny nickte leicht. \enquote{Wirst du dich von ihr trennen?}, fragte sie nun.

\enquote{Weißt du noch, was ich dir in den Ferien gesagt habe?}

Ginny ging das Bild auf Harrys Geburtstagsfeier durch den Kopf. Ihre Augen wurden leicht feucht.

Harry erschrak und küsste ihre feuchten Augen. \enquote{Weißt du, das hat sich grundlegend geändert. Ich empfinde mittlerweile mehr für dich. \gst Viel mehr.}

Jetzt glitt ein Lächeln über Ginnys Gesicht. \enquote{Das spüre ich}, antwortete sie ihm.

Erst jetzt bemerkte Harry seine Erektion, die gegen Ginnys Pobacken und die äußeren Schamlippen drückten. Er lächelte mit einem verführerischen Gesichtsausdruck zurück und meinte: \enquote{Du bist aber auch schon ganz feucht.}

Sie patschte ihm mit der flachen Hand auf die Brust, was einen Wasserschwall auf beide Gesichter spritzte. Harry zog sie näher zu sich heran und küsste sie ausgiebig. Gern ließ sich Ginny diese Behandlung gefallen, spürte den intensiven Druck zwischen ihren Beinen, seine Brust auf ihrer, seine Hände an ihrem Rücken und seine Lippen auf den ihren, ihrem Kinn, dem Hals und der Stirn.

\enquote{Ich bin noch nicht bereit dafür, Harry}, sagte Ginny.

Harry verstand und sagte: \enquote{Ich weiß, deshalb habe ich dich nicht hierher geholt. Sondern für das, was wir gemacht haben.} Er überlegte kurz. \enquote{Ich habe mir vielleicht etwas weniger vorgestellt, aber ich richte mich ganz nach dir.}

\enquote{Ich habe Angst, Harry. Du bist schon\abs hast schon mit so vielen\abs}

\enquote{Ginny, ich wünschte, ich könnte diese Erfahrungen verdrängen\abs Aber ich glaube kaum, dass sich das\abs}

Sie küsste ihn wieder. \enquote{Du bist so lieb.}

\gedanke{Außer ich verwende Okklumentik. Aber wie lange kann ich dieses Wissen unterdrücken? Und wenn ich die Konzentration verliere, dann bricht es über mich herein und ich könnte mit Ginny etwas machen, was ich hinterher bereue}, ging ihm durch den Kopf. \enquote{Aber ich werde dir, wenn du so weit bist, jeden Wunsch erfüllen, dass unser gemeinsames erstes Mal für dich unvergesslich wird.}

\enquote{Woher willst du\abs}

\enquote{Ich werde dir keine Namen nennen, Ginny.}

Sie schmollte und schaute ihn mit Dackelaugen an.

\enquote{Es waren auch Slytherins dabei. Mehr sage ich nicht}, lenkte Harry ab.

\enquote{Lehnst du dich zurück, Harry?}, forderte sie von ihm.

Harry rutschte mit seinem Becken etwas nach vorne, damit er schräg mit durchhängendem Rücken auf dem Steinsitz saß. Ginny küsste ihn noch einmal. Dann richtete sie ihren Oberkörper auf, stieg links und rechts von ihm auf die kleine Mauer rings um das Becken und stieg über ihn hinweg. Harry blieb die Luft bei diesem Anblick weg. Er saß mehrere Sekunden starr da und sog das Bild in sich auf, das er eben gesehen hatte. Es brannte sich für immer in sein Gedächtnis ein.

Dann stand er auf und folgte Ginny. Nachdem sich beide angezogen hatten, gingen sie Hand in Hand und glücklich zurück zum Gemeinschaftsraum.

\trenn

Die Prüfungszeit war vorbei und es waren nur noch wenige Tage, bis der Hogwarts-Express die Schüler nach Hause bringen konnte. Alle saßen gemeinsam in der Großen Halle und waren beim Essen. Nach einer Weile stand Professor McGonagall auf und lief hinter dem Tisch herum und zog dann mit Professor Snape, der neben Professor Elber saß, diesen hoch. Dumbledore schwang seinen Zauberstab und in der Mitte erschien ein kleiner Stuhl mit dem sprechenden Hut. Die drei waren auf dem Weg zu ihm, als ihn Professor Elber entdeckte und begann sich zu wehren. \enquote{Nein, nein} und stemmte sich gegen sie um zu verhindern, dass sie ihn weiter Richtung Stuhl zogen. Doch es half nichts. Unaufhaltsam zerrten Snape und McGonagall an ihm, bis er auf dem Stuhl saß. Ergeben legte er seine Hände gegeneinander und klemmte sie mit gesenktem Kopf zwischen seine Beine. Harry war der Meinung, in seinen Augen Tränenflüssigkeit zu sehen. Dann hob McGonagall den Hut vom Boden auf und Professor Elber wollte schon wieder aufstehen, als er von Professor Snape wieder heruntergedrückt wurde. Dann setze sie Elber den Hut auf.

Sofort spürte Harry eine Fröhlichkeit in sich aufsteigen und eine innere Zufriedenheit machte sich in ihm breit. Doch den anderen ging es scheinbar genauso. Ginny nahm direkt seine Hand in ihre und warf ihm einen kurzen aber intensiven Blick zu.

Dem Hut entwich nur ein \enquote{Oh}, als er den Kopf unter sich spürte. Professor Elber nahm die Hände vors Gesicht und senkte es noch weiter.

Stille.

Der Hut sagte eine Weile nichts. Langsam konnte Harry Flüssigkeit zwischen den Händen seines Professors feststellen. Nach und nach tropfte sie an ihm herunter auf den Boden. Dann sagte der sprechende Hit nur noch ein Wort: \enquote{Magistri.} Die Halle wurde noch stiller.

Professor Elber sprang vom Stuhl auf und verließ mit schnellen Schritten und Tränen überströmtem Gesicht die Große Halle. Währenddessen warf er den Hut von seinem Kopf, direkt auf den Boden. Als er die Klinke der großen Holztür in der Hand hielt, drehte er sich noch einmal kurz um und schrie in die Große Halle. \enquote{Ich hoffe, ihr seid jetzt zufrieden.} Dann öffnete er die Tür und schlug sie beim hinausgehen mit voller Wucht zu.

Schlagartig sank die Stimmung in der Großen Halle.

\enquote{Da haben wir wohl übertrieben}, murmelte Professor McGonagall.

Harry fühlte sich plötzlich schlecht. Er sah sich in der Halle um und merkte, dass keiner der Anwesenden mehr sich seinem Essen widmete. McGonagall ging mit gesenktem Blick zu ihrem Platz zurück und setzte sich ebenso wie Snape. Nur mit dem Unterschied, dass man dessen Gefühle nicht lesen konnte. Keiner der Lehrer aß mehr, oder sah durch die Halle. Keiner rührte mehr sein Essen an.  Alle sahen nur mit gesenktem Kopf auf den Teller vor sich.

Die Stimmung war auf einem Tiefpunkt, wie ihn Harry noch nie erlebt hatte.

Durch Harrys Kopf klangen noch einmal die Worte des sprechenden Hutes. \accentuate{Oh.} Und im selben Abstand wie der Hut brauchte: \accentuate{Magistri.}

Schließlich stand er auf und verließ die Große Halle, als er noch einmal kurz seinen Blick schweifen ließ. Dracos Blick traf seinen, und er stand auf um nun ebenfalls die Halle zu verlassen. Sein Blick blieb an Luna hängen, die als Einzige versuchte, etwas zu essen. Langsam, aber dennoch unsicher, schob sie einen Bissen nach dem anderen in sich hinein. Harry lächelte leicht in sich hinein und verließ neben Draco die Halle. Schweigend gingen sie nebeneinander her, bis sich ihre Wege trennten. Sie blieben kurz stehen und nickten einander zu, bevor jeder seinen Weg ging.

Harry hatte sich bereits für die Nacht zurechtgemacht und sah durch das Fenster in die dunkle Nacht. Er hörte Donner grollen und dachte: \gedanke{Klasse, das wird eine tolle Nacht.}

Die Tür öffnete sich und ein Blitz erhellte die Nacht. Er schlug in der Nähe des Schlosses ein. Harry zuckte kurz zusammen. Ron und Dean, sowie Seamus und Neville kamen herein und bereiteten sich ebenfalls für die Nacht vor. Als die vier wieder hereinkamen, lag Harry schon im Bett. Die Hände hinter seinem Kopf gelegt, starrte er an die Decke. Das Gewitter war in der Zwischenzeit etwas lauter geworden. Harry hatte das Gefühl, mit jemandem reden zu wollen. \enquote{Gute Nacht}, sagte er zu seinen Freunden und schloss mit einer gelangweilten Geste die Vorhänge zu. \enquote{Wow, Harry}, kam von der anderen Seite durch den schweren Stoff. \enquote{Das muss ich auch mal versuchen.}

Harry grinste in sich hinein. Er nahm kurz sein Amulett in die Hand und schloss für einen kurzen Moment die Augen. Als er sie wieder geöffnet hatte, schwebte Salazar vor ihm. Er verwickelte ihn in eine gedachte Diskussion.

\gedanke{Ich brauche deinen Rat, Salazar.}

\stimme{Wobei, Harry?}

\gedanke{Es geht um einen meiner Lehrer und sein komisches Verhalten beim Essen.}

\stimme{Hat er sich beklagt, dass es ihm nicht schmeckt?}, gluckste Salazar.

\gedanke{Nein. Professor McGonagall und Professor Snape haben ihn zu einem Stuhl gezogen und ihm den sprechenden Hut aufgesetzt.}

Salazar zuckte kaum merkbar zusammen. \stimme{Was ist passiert?}, fragte er so unbeeindruckt, wie er konnte.

\gedanke{Zuerst hat der Hut nur \accentuate{Oh} gesagt. Und dann, nach einer Weile: \accentuate{Magistri}.} Wieder zuckte Salazar unmerklich zusammen. \gedanke{Was hältst du davon?}, fragte ihn Harry nun.

Salazar überlegte kurz. \stimme{Nun}, sprach er, \stimme{ich denke, der sprechende Hut hat etwas Besonderes in ihm erkannt.}

Ein lauter Blitz schlug jetzt in der Wand vor Harry ein er fuhr hoch. Mit dem Gesicht mitten in Salazar lauschte er in die dunkle Nacht hinein. Als ihm wieder einfiel, wo er seinen Kopf hatte, legte er sich wieder hastig hin.

\gedanke{Was denkst du?}, fragte Harry weiter.

\stimme{Er dürfte wohl etwas Besonderes sein. Der Hut wollte ihm wohl sagen, dass er sich sein Haus aussuchen konnte.}

\gedanke{Er ist aber mit Tränen in den Augen sofort rausgerannt, als der Hut \accentuate{Magistri} zu ihm sagte.}

\stimme{Er war wohl zu geschockt}, sagte Salazar und sah dabei aber nicht überzeugend aus.

\gedanke{Ich denke eher, er hatte eine Ahnung. Denn als er da saß, nahm er direkt seine Hände zwischen seine Beine und senkte seinen Kopf. Und als der Hut dann \accentuate{Oh} sagte, schlug er seine Hände vors Gesicht und fing an zu weinen. \gst Ist so etwas schon einmal passiert?}

\gedanke{Soweit ich weiß, einmal. Warte kurz.} Salazar verschwand und kam eine Minute später wieder. Genau, als wieder ein Blitz ins Schloss einschlug. Harry zuckte wieder zusammen. Dann hörte er ein Rascheln.

\stimme{Dies ist erst das zweite Mal}, schloss Salazar und verschwand.

Die Vorhänge um Harrys Bett wurden aufgezogen und Neville stand davor. Er sah ihn an und meinte dann: \enquote{Ach, du kannst also auch nicht schlafen?}

\enquote{Nein}, gab Harry matt zurück und stand auf. Zusammen gingen sie zu den Anderen und dann zusammen in den Gemeinschaftsraum. Dort war es brechend voll, weil keiner der Gryffindors schlafen konnte. Und wieder ging ein greller Blitz vor dem Turm der Gryffindors nieder und schlug in die Schlossmauern ein. Das Donnern und Grollen wurde bedrohlicher. \gedanke{An Schlaf ist wohl heute nicht mehr zu denken.} Ginny drückte sich durch die Masse, um sich an Harry zu kuscheln. Er nahm sie in seine Arme und gab ihr einen flüchtigen Kuss. Doch sie zog ihn zu sich und so standen sie da, bis sich Ron neben ihnen räusperte. Unwillig löste sich Ginny von Harry und legte ihren Kopf an seine Schulter.

\enquote{Ron?}, fragte Harry.

\enquote{Ja?}

\enquote{Holst du mit Hermine zusammen McGonagall?}

\enquote{Warum?}

\enquote{Weil du und Hermine Vertrauensschüler seid und keiner von uns schlafen kann. Und wenn wir versuchen, hier einen Schutzzauber auf die Wände zu legen, damit es nicht mehr so laut ist, dann kann es sein, dass er entweder nicht richtig ist, oder zu sonst irgendwelchen Problemen führen kann.}

\enquote{Ah!} Ron sah dabei nicht sehr intelligent aus.

Ron suchte in dem Gewirr Hermine und begab sich dann durch das Porträt-Loch nach draußen. Nach einer gefühlten Stunde, aber laut seiner Uhr nur einer viertel Stunde, kamen sie zu dritt wieder. \enquote{Meine Güte, hier auch?}, fragte Professor McGonagall ganz entsetzt. \enquote{Das ganze Schloss bebt schon.}

\enquote{Wie meinen Sie das, Professor?}, fragte Martina.

\enquote{In den anderen Gemeinschaftsräumen ist es laut den andern Hauslehrern auch so. Und nicht nur dort. Auf das ganze Schloss prasseln Blitze ein.}

Stille.

Erst jetzt nahm Harry die leicht dunklere Farbe der Mauersteine wahr, die sich seit dem Vorfall in der Großen Halle verändert hatte. Er ließ seinen Blick durch den Raum schweifen und blieb an der Stelle hängen, an der Professor Elber für die Eltern an Weihnachten einen zusätzlichen Raum erschaffen hatte. Er drückte sich durch vereinzelte Schüler und setzte sich auf eine Treppenstufe.

Er ließ sein Schuljahr noch einmal vor seinem geistigen Auge Revue passieren und dachte an die vielen Extra-Stunden, die er genossen hatte und die ihm auf seinem Weg weiter helfen sollten. Seine Hand suchte ein Stück Mauer, das er berührte. Langsam keimte in ihm ein Verdacht. Er schloss seine Augen und lehrte seinen Geist von allen störenden Gedanken. McGonagalls Worte hatte er schon vor geraumer Zeit vollkommen ausgeblendet. Er sah nun die Blitze vor seinem geistigen Auge und zwang sie, weniger zu werden. Er konzentrierte sich auf deren Intensität, um sie zu vermindern. Leider konnte er dem Donner keinen Respekt abtrotzen. Aber die Blitze wurden weniger, verloren etwas an Intensität und schlugen jetzt nicht mehr ins Schloss ein. Danach wurde ihm schwarz vor Augen.

Er erwachte erst wieder, als er etwas an seinen Lippen fühlte. Er legte seine Hände um den Nacken der Person und zog sie zu sich. Bedauerlicherweise hörte er kurz darauf ein entrüstetes \enquote{Harry.}

Er kannte die Stimme genau. \gedanke{Das war Ginny. Aber wieso war sie so weit weg? Und wieso kann sie reden, wenn ich sie küsse?} Er löste den Kuss und öffnete seine Augen. Er blickte in zwei wunderbare Augen, konnte aber wegen seiner schlechten Augen nicht viel sehen, ließ die Person los und suchte nach seiner Brille. Als er sie aufgesetzt hatte, lief ihm ein kalter Schauer über den Rücken.

Er wollte sich gerade entschuldigen, aber er entschied sich im letzten Moment um und so entfuhr ihm ein einfaches: \enquote{Danke.}

Die angesprochene presste ihre Lippen kurz aufeinander, um den leichten Schimmer von Harrys Speichel zu entfernen, bevor er sie anlächelte und dann zu Ginny sah und ihr seine Arme entgegenstreckte. Als sie in seinen Armen lag, fasste Harry doch noch ein Herz und meinte: \enquote{Entschuldigung, Madame Pomfrey. Ich dachte, Sie wären Ginny.} Wortlos drehte sie sich um und ging in ihr Büro.

\enquote{Die hast du jetzt aber geschafft}, flüsterte sie in sein Ohr.

\enquote{Ich dachte, sie wär du. Sie hat sich nach dir angefühlt.}

\enquote{Muss ich ihretwegen jetzt eifersüchtig sein?}

Harry antwortete ihr nicht, sondern gab ihr einen fordernden Kuss. \enquote{Darf ich schon gehen? Was meinst du?}

\enquote{Auf jeden Fall. Zieh dich an.}

Sie wollte sich gerade umdrehen, als sie Harry festhielt.

\enquote{Wenn wir weiter zusammen sein wollen, dann sollten wir langsam anfangen, uns zu vertrauen.} Er schlug die Decke zurück und zog sein Nachthemd aus. Nur in seiner Unterhose stand er vor ihr und begann, sich seine Sachen anzuziehen. Als er fertig war, gab er Ginny wieder einen Kuss und meinte: \enquote{Ich rede kurz mit Madame Pomfrey. Wartest du hier so lange?}

Ginny nickte und Harry verschwand in Madame Pomfreys Büro. Als er wieder kam, verließen sie Hand in Hand den Krankenflügel.

Noch immer donnerte und blitzte es draußen. Und das sollte die nächsten Tage noch anhalten.

\trenn

Zu seinem vorletzten Termin saß Harry in einem bequemen Stuhl und sah an die Decke der privaten Räumlichkeiten Professor Elbers in Hogwarts.

\enquote{Dieses Jahr haben Sie viel gelernt, Harry. Ich habe Ihnen so viel beigebracht, wie ich konnte. Es gibt aber noch etwas, was Sie wissen müssen. \gst Ich weiß nicht, ob es Ihnen schon klar geworden ist, oder ob Sie es nur vermuten. Vielleicht haben Sie auch nur das unterbewusste Wissen? \gst Vielleicht verstehen Sie das auch nicht sofort in seiner vollen Tragweite, was ich Ihnen jetzt sagen werde. Vielleicht hören Sie nur die Worte und verstehen den oberflächlichen Sinn dahinter. Ich will aber, dass Sie den tiefen Sinn dahinter verstehen. \gst Magie muss gepflegt werden. \gst Sie beherrschen doch den Patronus-Zauber. \gst Wie oft haben Sie ihn in den letzten Jahren erscheinen lassen? \gst Meinen Sie nicht, dass es für Ihre Beziehung untereinander besser wäre, ihn öfter zu rufen? Er muss nicht unbedingt eine Aufgabe haben. Rufen Sie ihn einfach. Spielen Sie mit ihm. \gst Das stärkt Ihre Bindung zueinander. Mit einer guten Bindung haben Sie viel mehr Möglichkeiten. \gst Stärken Sie sie.}

Harry war wie vom Donner gerührt. Vor seinem geistigen Auge sah er ein paar Monate zurückblickend Professor Elber auf einer Wiese mit seinem Patronus herumalbern. Damals hatte er nur seinen Kopf geschüttelt und es nicht verstanden, zumal er kurz darauf im Krankenflügel einige Zeit verbringen musste. Doch so langsam ergab es einen Sinn. Man konnte mit einem Patronus Dementoren vernichten, hatte er gelesen. \gedanke{Dazu muss man eine enge Bindung mit seinem Patronus eingehen. Wie er wohl reagieren wird, wenn ich ihn jetzt erschaffe?} Kurz darauf stand ein silbern-blau leuchtender Hirsch im Zimmer. Und wiederum ein paar Sekunden später ein Schwarm geflügelter Insekten. Diese formierten sich und bildeten ebenfalls einen Hirsch. Nun sprangen beide Tiere im Zimmer umher und in die angrenzenden Räume und spielten miteinander.

\enquote{Fragen Sie mich nicht, warum die einen Hirsch gebildet haben. Von mir haben sie das nicht. Und wenn ich ehrlich bin, habe ich das nie so richtig verstanden}, sagte sein Lehrer und schloss seine Augen. \enquote{Bauen Sie eine Verbindung zu Ihrem Patronus auf.}

Harry versuchte es, wurde aber eine viertel Stunde später durch gleichmäßige Atemgeräusche und ab und an einen schnarchenden Laut gestört. Er musste schmunzeln, schaffte es aber doch, zumindest die Präsenz seines Patroni und den seines Lehrers zu spüren. Ob letzteres etwas über seine Verbindung zu ihm aussagte, wusste er nicht. Er wusste aber, dass zwei Menschen, die sich nahe stehen, gleiche Patroni hatten.

Nachdem er seinen Patronus verschwinden hatte lassen, weckte er seinen Professor. Dann fragte Harry ihn, ob er etwas über die Mondbibliothek wisse. Doch leider half ihm seine Antwort nur bedingt weiter.

\enquote{Ja, die gibt es. Aber ich werde Ihnen noch nichts darüber erzählen. Für Sie ist es noch zu früh.}




\begin{kommentar}
Die Prüfung in VgddK beginnt. Diese findet in einer großen Kuppel statt, deren Eingang ein schwarzes 'Nichts' ist. Genauso, wie das 'Nichts' in der unendlichen Geschichte.
\end{kommentar}

\chapter{Abschiede}


Am nächsten Tag in der Früh fuhr wieder der Hogwarts-Express, um alle Schüler nach Hause zu bringen. Das schlechte Wetter war genauso schnell verschwunden, wie es gekommen war. Harry lief im Trainingsanzug einige Runden auf der sandigen Piste auf dem Quidditchgelände. Er fragte sich, warum er noch so kurz vor Schuljahres-Ende eine Trainingseinheit absolvieren sollte. Und vor allem, wer würde sein Gegner sein. Er hatte gerade eine erneute Runde beendet, als Dumbledore hereinkam. Harry stutzte, denn er hatte eine altertümliche Robe in Schwarz an. Er stellte sich in den Kreis, der die Arena abgrenzte, und wartete auf Harry.

\enquote{Wie jetzt?}, fragte dieser. \enquote{Du? \gst Sie?}

\enquote{Frederick weiß Bescheid.}

\enquote{Oh\abs okay. Du bist mein Gegner?}

\enquote{’N Problem damit?}

\enquote{Nein. Ich bin nur überrascht.} Harry ging auf den Kreis zu und atmete einmal kräftig durch. Dann trat er hinein.

Die beiden Kontrahenten verbeugten sich und der Kampf begann. Harry war zuerst zögerlich und blockte vorwiegend ab, anstatt anzugreifen. Doch nach und nach kamen ihm die vergangenen Ereignisse wieder in den Sinn. Der Kampf Voldemort mit Dumbledore, der Kampf Dumbledore mit Elber und schließlich die vielen Trainingseinheiten und die Erinnerungen von Salazar, die er ihm zeigte.

Harry entschied sich jetzt aktiver zu sein. Zug um Zug griff er mehr und stärker an. Solange, bis Dumbledore ihm einen mächtigen Zauber schickte, den er mit einem Schild blocken konnte. Beständig floss die magische Energie aus Dumbledores Zauberstab heraus und auf Harrys Schild zu. So konnte er natürlich nicht mehr angreifen. Doch er hatte eine Idee. Er nahm seine andere Hand und hielt sie so, als ob er etwas blocken wollte. Dann führte er seinen Zauberstab auf seine Hand und diese begann kurz zu leuchten. Der Zauber wurde übertragen und Harry konnte nun angreifen, obwohl er einen Schild hatte.

Dumbledore war darüber so erstaunt, dass er seinen Stab nicht mehr halten konnte. Er flog auf Harry zu, der ihn zielsicher wie ein Sucher fing. Er musste sich ein Grinsen verkneifen, da er Dumbledore besiegt hatte. Zwar zweifelte er daran, dass er in einem echten Kampf eine Chance gehabt hätte, aber dieser Überraschungsmoment war ihm doch gelungen.

Harry verbeugte sich abermals und ging dann auf Dumbledore zu. \enquote{Ich glaube, das ist deiner}, sagte er und reicht ihm seinen Stab. Dumbledore lächelte und klopfte Harry anerkennend auf die Schulter. Schwatzend liefen sie nebeneinander zurück zum Schloss und ließen den Kampf noch einmal Revue passieren.

\trenn

Es war der letzte Abend im Schloss. Die Halle war wie immer voll und alle Schüler aßen zu Abend. Nachdem das Essen wieder von der Tafel verschwunden war, wollte gerade Dumbledore seine Abschlussrede halten, doch Professor Elber stand bereits und hielt sie seinerseits.

\enquote{Ich danke Ihnen allen, dass ich dieses Schuljahr hier auf Hogwarts verbringen durfte. Dass ich Sie am Jahresanfang verschreckt habe, tut mir leid. Aber dafür haben Sie ja dieses tolle Konzert als Entschädigung erhalten. Ich möchte Ihnen aber noch etwas auf den Weg mitgeben, etwas, dass Sie immer an Ihre Schulzeit erinnert, etwas, dass Ihnen sagt: \inner{Übe, und du wirst ein wahrer Meister.} Zusammen mit ein paar Schülern habe ich eine kleine Demonstration vorbereitet. Wenn diese einmal zu mir kommen würden?}

Es dauert wenige Sekunden, als alle gleichzeitig vor ihm erschienen. Aus dem Nichts. Sie waren lediglich von ihren Bänken aufgestanden, hatten sich einen Schritt in den Gang entfernt und apparierten scheinbar nach vorne, vor den Lehrertisch. Nun standen Draco Malfoy, Hermine Granger, Ronald Weasley, Susan Bones und Ernie McMillian vorne. Das löste tumultartige Zustände im Raum aus. Die Lehrer standen fassungslos auf; von überall hörte man: \enquote{Das ist unmöglich. Man kann im Schloss und auf den Ländereien nicht apparieren.}

Es verstrichen einige Sekunden und alle schauten wie gebannt zu den Lehrern nach vorne, die genauso ratlos aussahen. Nur einer stand da und rührte sich nicht. Snape. Dumbledore sah mit kindlichem Interesse zu. Nachdem er den ersten Schreck überwunden hatte, erinnerte er sich an das, was ihm sein Kollege erzählte. \accentuate{Virtuelles apparieren.} Dann musste er ein Schmunzeln unterdrücken.

\enquote{Die Schüler hier sind natürlich nicht wirklich appariert. Es sah nur so aus. Sie haben ein Trugbild von sich projiziert. In Wahrheit stehen sie immer noch an ihren alten Stellen. Ihre Mitschüler mögen es bitte einmal probieren, sie anzufassen.}

Und tatsächlich. An den Stellen, an denen Ron und Hermine verschwunden waren, spürte man immer noch einen Widerstand. Und auch die Schüler der anderen Häuser konnten die Körper ihrer Kameraden spüren.

\enquote{Ich hoffe, diese kleine Demonstration bleibt Ihnen noch eine Weile in Erinnerung. Vielleicht sehen wir uns irgendwann noch einmal. Bis dahin, lernen Sie fleißig und üben Sie. Denn das Dunkle wartet nicht, bis sie gerüstet sind. Es ist da draußen und wartet nur auf einen passenden Zeitpunkt.}

Dann öffneten sich die Türen der Großen Halle und die Schüler begannen, sich für den nächsten Morgen vorzubereiten, den Tag der großen Abreise.

Die Sonne ging langsam unter und der Himmel begann sich rot zu färben. Harry stand im Torbogen zum Eingang von Hogwarts und sah nach draußen. Von hinten kamen Professor Elber und Professor Dumbledore näher. Er drehte sich nicht um, hörte aber deren Unterhaltung mit.

\enquote{Also, Albus, es war schön hier. Wir sehen uns.} Harry hörte, wie sich die zwei die Hände gaben und sie sich schüttelten.

Dann wurde er durch eine andere Unterhaltung abgelenkt.

\enquote{Kann ich die Ferien über bei dir bleiben, Astoria?}, fragte Pansy ihre Mitschülerin. \enquote{Ich habe gerade einen Brief von meinen Eltern erhalten}, schluchzte sie weiter. \enquote{Sie haben mich hinausgeworfen, weil ich Dracos Freundin bin. Ich kann nirgendwo hin.}

\enquote{Deine Großeltern, oder andere Verwandte?}, fragte sie nach.

\enquote{Väterlicherseits sind sie schon im ersten Krieg gestorben, mütterlicherseits kenne ich sie nicht. Ich habe keine Ahnung, ob noch jemand lebt. Sonstige Verwandte sind mir nicht bekannt. \gst Doch, ein Cousin in Australien. Aber der ist ständig unterwegs.}

\enquote{Tut mir leid, Pansy, aber wir sind die erste Woche in Frankreich, bei meinem Onkel.}

\enquote{Der, den ihr nicht leiden könnt?}

\enquote{Meine Mutter und mein Vater können ihn nicht leiden. Ich finde ihn gar nicht mal so schlecht. Aber meine Eltern sagten sich: \inner{Er bezahlt uns immerhin die Reise und die Unterkunft. Und eine Woche Frankreich. \gst Paris sehen und so. \gst Das lassen wir uns nicht entgehen.} Danach werden wir nach Amerika gehen.} Harry drehte sich jetzt um und sah Pansy richtig bedrückt. \enquote{Was ist mit Draco?}

\enquote{Vergiss es. Sein Vater\abs} Sie sprach nicht mehr weiter.

Astoria nickte nur. \enquote{Und andere Mitschüler?}

\enquote{Sieht schlecht aus. Ich habe noch Millicent und Lemia nicht gefragt. Aber ich habe kaum Hoffnung. Vermutlich muss ich unter irgendeiner Brücke schlafen.}

Harry hatte plötzlich Mitleid mit Pansy und fühlte mit ihr. Dann konnte er nicht mehr zu ihr sehen und sah zu Dumbledore und Elber, dessen besorgtes Gesicht auf Pansy gerichtet war. Als sich Pansy ihm zu wandte, blickte er schnell zu Dumbledore. Harry drehte sich wieder um, sah nach draußen und genoss weiterhin das Wetter.

Dann spürte er eine Hand auf seiner Schulter. \enquote{Harry, alles Gute weiterhin in der Schule.}

\enquote{Danke, Professor}, sagte Harry und lächelte. Professor Elber ging an ihm vorbei und lief den Weg nach Hogsmeade hinunter. In der Hand hatte er einen kleinen Koffer, der Harry an eine Aktentasche erinnerte, und einen Regenschirm, der so schmal wie ein Spazierstock war. Harry sah ihm nach.

\enquote{Professor?}, fragte er Dumbledore. \enquote{Haben Sie schon einen neuen Lehrer für nächstes Jahr?}

\enquote{Ja.}

\enquote{Professor?}, fragte er Dumbledore.

\enquote{Ja, Harry}, gab Dumbledore zurück.

\enquote{Woher kennen Sie Professor Elber?}, fragte Harry.

Dumbledore schaute nun ebenfalls den Pfad entlang hinunter. Nach einer Weile fing er zu erzählen an.

\enquote{Weißt du, Harry, bevor Voldemort deine Eltern ermordet hat und verschwunden ist, gab es einige Prozesse gegen Todesser. Ich wohnte damals allen Verhandlungen als Zuhörer bei. Mit Ausnahme der von Snape und Karkaroff. Aber Karkaroffs Verhandlung hast du bereits gesehen.} Harry nickte. \enquote{Auf jeden Fall, als es dann zu den Verurteilungen kam, fragte der zuständige Vorsitzende die Jury. \inner{Irgendjemand für eine Verurteilung?} Und Professor Elber war der Einzige, der seine Hand erhob. Er saß in der ersten Reihe. Die anderen aus der Jury saßen alle hinter ihm. Er konnte sie nicht sehen. Der Vorsitzende fragte weiter: \inner{Irgendjemand für einen Freispruch?} Keiner hob die Hand. \inner{Irgendwelche Enthaltungen?}, fragte der Vorsitzende. Alle Anderen hoben die Hand.}

Harry sah Dumbledore erstaunt an.

\enquote{Ich muss dir nicht sagen, dass die Todesser natürlich heftig protestierten. Aber sie waren verurteilt worden. Einer hatte damals Professor Elber geschworen, dass der Dunkle Lord ihn persönlich zur Strecke bringen würde. Das hat ihn eigenartigerweise nicht besonders interessiert. Er stand auf und sagte klar und deutlich: \inner{Wenn Lord Voldemort etwas von mir will, dann soll er selber kommen.} Danach setze er sich wieder.}

Harry sah wieder Richtung Pfad.

\enquote{So ist es bei jeder Verhandlung gelaufen. Er war der Einzige, der für Schuldig gestimmt hat. Alle Anderen enthielten sich.} Jetzt musste Dumbledore lachen. \enquote{Der Vorsitzende hatte jedes Mal danach gesagt: \inner{Einstimmig verurteilt.}}

Dumbledore drehte sich jetzt zu Harry. \enquote{Er ist nach jeder Verhandlung immer gleich verschwunden. Durch ein paar Kontakte konnte ich schließlich erfahren, wo er sich eingemietet hatte. Eines Abends bin ich dann zu ihm gegangen. Ich wollte ihm gratulieren, dass er immer dafür war. Als ich dann im Dunklen die Gasse entlang ging, sah ich von weitem einen Fremden in das Haus gehen. Ich dachte schon ein Todesser und beschleunigte meine Schritte. Aus einem Fenster sah ich grüne Blitze. Danach tat es einen Schlag und der Fremde brach durch die Haustür. Professor Elber kam mit gezogenem Zauberstab heraus und sagte zu ihm etwas in der Art: \enquote{Wenn Voldemort etwas von mir will, dann soll er gefälligst selber kommen.} Das hatte mich beeindruckt.}

Dumbledore drehte sich nun um und lief langsam in das Innere des Schlosses, da es durch aufkommende Wolken kälter wurde. Harry lief neben ihm her.

\enquote{Er ging danach zurück und die Haustür nahm ihren früheren Platz ein. Als ich dann klopfte, musterte er mich eine Zeit lang, bis er mich hereinbat. Seit der Zeit kennen wir uns. Und ich konnte ihn endlich überreden, hier zu unterrichten. Das wollte ich schon letztes Jahr. Aber du weißt ja selber.}

\enquote{Umbridge}, entgegnete ihm Harry. Dumbledore nickte. Sie betraten die Große Halle und Dumbledore setzte sich Harry gegenüber an seinen Haustisch. \enquote{Meinst du, dass er Nicolas Flamel kennt?}, fragte Harry seinen Schulleiter.

\enquote{Wie kommst du darauf Harry?}

\enquote{Weil er mir gegenüber einige Andeutungen gemacht hat, beziehungsweise er sich versprochen hat, dass mich das vermuten lässt. Zudem hat er dich unten in der Kammer ganz schön schwitzen lassen. Er sah keinen Zauber als ein Problem an. Er hat uns über die unverzeihlichen Flüche aufgeklärt, als wäre es das Normalste der Welt. Und er hat mir Sachen beigebracht, die weit über das hinausgehen, was man hier normalerweise lernt.} Erst bei seinem letzten Satz merkte er, was er gesagt hatte. \enquote{Entschuldigung, Albus.}

Doch Dumbledore lachte. \enquote{Nein Harry, mir ist nicht bekannt, dass er Nicolas und Perenelle kennt. Wie kommst du nur auf diesen Gedanken?}

\enquote{So wie er auftritt und vor allem sein Wissen über die Magie. Vor allem, wie er Bellatrix behandelt hat.}

Dumbledore, der sich in der Zwischenzeit in einen Hähnchenflügel biss, legte ihn beiseite, schluckte seinen Bissen runter und fragte dann: \enquote{Wie meinst du das?}

\enquote{Na ja, ich hatte einen Traum, oder eine Vision. In der spielte er mit Lucius Malfoy Schach, als Draco ganz aufgeregt durch das halbe Schloss schrie. Sofort ist er in den Saal gesprungen und sah, wie Bellatrix Dracos Schwester Tamara mit dem Cruciatus-Fluch belegte.}

Dumbledores Augen wurden größer.

\enquote{Er entwaffnete sie mit einer Handbewegung und sagte Draco, er solle seine Schwester nach oben tragen und dort warten. Als sie alleine waren, versuchte Bellatrix ihn mit dem Cruciatus-Fluch zu belegen, da er sich kurz umdrehte. Er zuckte nur kurz zusammen und entwaffnete sie abermals. Dann dachte ich, dass seine Haare weiß wurden und länger, seine Augen rot und seine Fingernägel leicht wuchsen. Dann zuckten aus seinen Fingern blau-violette Blitze, die er auf Bellatrix schleuderte. Nachdem er sie gefoltert oder bestraft hatte, während dessen ich dachte, Teile ihrer Haut würden verbrennen, hörte er auf. Er nahm seinen Zauberstab heraus und ließ sie wie eine bewusstlose Puppe in einen Sessel schweben und legte einen Finger auf ihre Stirn. Sie sah danach irgendwie anders aus und er sagte so etwas wie, dass sie eine halbe Stunde für sich hätte.}

Dumbledore sah in ausdruckslos an. Er biss wieder in seinen Flügel, kaute langsam und blickte nachdenklich Harry an.

\enquote{Vielleicht\abs hat er einen eigenen Stein. \gst Wie alt schätzt du ihn?}

\enquote{Na ja, er scheint zumindest Schüler zu kennen, die ungefähr dreißig Jahre vor dir zur Schule gingen. Und so wie er immer darüber erzählt, als ob er selbst dabei gewesen wäre. Ich denke, er ist mindestens ein paar hundert Jahre alt.} Jetzt biss auch Harry in einen Chicken-Wing. \enquote{Darf ich Ihnen eine etwas eigenartige Frage stellen, Professor? Und bitte, fragen Sie mich nicht, wie ich darauf komme}, fragte Harry, nachdem sich die Schüler in ihrer Umgebung mehrten und sie nicht mehr wirklich alleine waren.

\enquote{Sicher Harry. Ich halte es da wie mein Kollege. Du darfst mich alles Fragen, aber du erhältst nicht immer eine Antwort.}

\enquote{Es geht um Geister.}

\enquote{Ah ja.}

\enquote{Gibt es gewisse Zauber \gst oder Umstände \gst wissen Sie, normalerweise kann man ja durch einen Geist durchgehen, seine Hand in ihn halten \gst Aber gibt es eine Möglichkeit, dass man einen Geist anfassen kann, dass er zeitlich begrenzt anfassbar, also materiell, wird?}

Dumbledore brach seinen Bissen, den er nehmen wollte, ab und sah Harry an. Dieser wurde rot.

\enquote{Ist dir das schon mal passiert?}, fragte er nach.

\enquote{Ich sagte doch: \inner{Fragen Sie mich bitte nicht}}, gab er ausweichend zurück.

Dumbledore überlegte. \enquote{Nun ja, dass Geister scheinbar materiell werden können, davon habe ich schon einmal gehört. Es bedarf dazu, soweit ich mich erinnern kann, sehr starker Gefühlsregungen. Ein einfaches Verlangen reicht dazu nicht aus.}

Harry nickte. \enquote{Wenn also ein Geist erregt ist, im positiven oder negativen Sinne, dann kann das passieren?}

\enquote{Ja. Das muss aber eine sehr starke Erregung sein.}

\enquote{Kann man mit Zaubern nachhelfen, also die Erregungsschwelle senken?} Harry kam sich langsam komisch vor, bei seinen Fragen.

Dumbledore dachte nach und sah gedankenverloren in die Runde. \enquote{Sir Nicolas, würden Sie mal herüber schweben?}

Der Geist schwebte zu den beiden und Dumbledore erzählte ihm kurz, was Harrys Frage war und wo beide jetzt standen.

Dieser kratzte sich am Ohr und dachte nach. \enquote{Bestimmte Zauber\abs}, murmelte er vor sich hin. Er rieb nun mit der anderen Hand an den inneren Konturen seines Ohres herum und sah auf den gedeckten Tisch. Er roch an einem einzelnen Pommes und kaute dann in der Luft.

Dann verschränkte er seine Finger ineinander, legte sie auf dem Tisch ab und legte die Daumen gegeneinander. Er sah zu Harry hoch und meinte dann: \enquote{Es gibt Zauber, die haben auf Geister durchaus eine Wirkung. Aber keine so starke, dass sie so etwas auslösen würden.} Er dachte kurz nach. \enquote{Eine Kombination von Zaubern könnte diesen Effekt etwas anheben \gst aber der Geist braucht noch immer eine gewisse Menge an Konzentration und Gefühlsregungen.} Er fuhr sich mit seinem Finger die Lippen entlang. \enquote{So etwas wie Ihr Zustand vor einigen Wochen könnte so etwas hervorrufen. Es ist eine Art der Gefühlsaufladung. Möglich, dass es im Zusammenhang mit Zaubern, die auf einem Ort wirken, auf Geister so eine Auswirkung hat. \gst Aber das ist nur eine Vermutung.} Ohne dass Dumbledore es merkte, zwinkerte er ihm zu.

Jetzt durchzuckten Harry mehrere Gedanken. Er erinnerte sich an Salazars Statue und wie er ihm die Möglichkeit zur Stofflichkeit gab. Ihm viel erst jetzt richtig auf, dass während des Geschlechtsverkehrs mit Myrte seine erste Ladung Sperma durch sie hindurch flog, dann aber ein feiner Nebel sich auf sie zu bewegte und so Myrte zu kurzer Stofflichkeit verhalf. \gedanke{Biologische Materie, na klar.} Am liebsten hätte er sich vor den Kopf geschlagen. Jetzt musste er leicht Schmunzeln. Es war so einfach. Geister konnten durch Übergießen mit biologisch angereicherten Flüssigkeiten und einer starken Gefühlsregung und unter Zuhilfenahme von Zaubern, die Gefühls-fördernd sind, für begrenzte Zeit Stofflichkeit erhalten.

Vor allem reizte ihn die Möglichkeit, seinen Verwandten Salazar einmal vorzustellen. Mit diesem Gedanken aß er weiter.

\trenn

Am nächsten Morgen wartete Harry auf seine Freunde um mit ihnen zum Zug zu laufen, als ein Hauself vor Pansy erschien und ihr einen Brief in die Hand drückte. \enquote{Hier Miss, für Sie.}

Pansy nahm den Brief entgegen und las ihn vor, da ihre Freundin ihr gegenüber stand und er nichts Privates zu enthalten schien. Harry konnte den Monolog mit anhören.

\begin{brief}
Liebe Pansy,

ich habe erfahren, dass dich deine Eltern hinausgeworfen haben.

Als dein Pate bin ich mir der Verantwortung bewusst, die ich jetzt auf mich nehme. Wenn du die Ferien woanders verbringen möchtest, dann sag meinem Elfen einfach Bescheid. Er wird dann wieder gehen. Falls du noch Optionen offen hast und lieber die Zugfahrt abwarten möchtest, dann kannst du das ebenso. Mein Elf wird dich dann am Bahnhof erwarten und deine Entscheidung entgegennehmen. Etwaige Fragen an mich können wir persönlich klären, wenn du deine Ferien bei mir verbringen möchtest.
\signumspace
Liebe Grüße

Dein Pate
\end{brief}

Pansy ließ den Brief in ihren Händen sinken und sah nachdenklich den Elfen an, der ihren Blick wartend erwiderte. \gedanke{Er steht einfach da und wartet. Keine Spur von Unterwürfigkeit. Er schaut mich selbstbewusst an und wartet.} \enquote{Ich warte noch die Zugfahrt ab. Bitte erwarte mich am Bahnhof und gehe davon aus, dass ich mitkomme. Allerdings kann sich während der Zugfahrt noch was ändern.}

Der Elf nickte und wartete noch einen kurzen Moment. Dann fragte er nach. \enquote{War’s das Miss?}

Pansy nickte und der Elf verschwand.

In der Zwischenzeit waren Harrys Freunde angekommen und sie liefen zusammen zum Zug. Er hing noch seinen Gedanken nach, bis sie endlich im Zug saßen.

Die Pfeife des Zuges pfiff und der Zug setzte sich in Bewegung. Kurz bevor der Zug losfuhr, poppte es und ein kleiner Elf tauchte im Abteil auf. Neugierig sah er Harry an. Der Zug fuhr bereits, war aber immer noch im Bahnhof. Ein weiterer Elf poppte auf dem Bahnsteig auf. Nachdem der kleine Elf seinen Vater entdeckt hatte, winkte er ihm aus dem Zug zu.

Harry sah kurz zwischen beiden hin und her. Dann öffnete er das Fenster und rief hinaus: \enquote{Soll ich Ihnen Ihren Sohn hinausgeben, oder holen Sie ihn in London ab? Ich passe dann so lange auf ihn auf, dass er keinen Unsinn macht.}

\enquote{Der Zug ist schon zu schnell. Das schaffen Sie nicht mehr. Ich hole ihn in London ab}, sagte der Elfenvater.

Harry nickte und schloss das Fenster wieder. Dann setzte er sich wieder und nahm den kleinen Elfen auf seinen Schoß, da er ihn wartend ansah.

Ginny saß neben Harry im Abteil und hatte ihre Hand auf seinem Fuß. Seine Hand lag darüber. Ihre Köpfe aneinander gelehnt sahen sie Ron und Hermine zu. Hermine hatte vor Aufregung gestern kaum geschlafen und lag mit ihrem Kopf in Rons Schoß. Verträumt fuhr er durch Hermines Haar. Neville und Luna saßen sich gegenüber an der Fensterreihe und sahen nach draußen auf den Bahnhof. Hagrid blickte noch durchs Fenster und winkte. Der Zug fuhr aus dem Bahnhof und um die erste Ecke. Heute Abend würden sie wieder Zuhause sein. \gedanke{Zu Hause}, dachte Harry. Die ganzen Ferien über wäre er wieder bei seinem Onkel und seiner Tante. Aber für den Moment lag es noch Stunden entfernt. Jetzt hatte er Ginny neben sich und war glücklich.

Er schloss die Augen und lauschte Lunas Geist. Er sah das Konzert vor sich, das er verpasst hatte, weil er in der Krankenstation war. Er war hinter der Bühne mit den Künstlern. Und er drehte sich und tanzte unbeschwert zur Musik. Er fühlte sich, als ob er es selber erlebt hatte, aber es waren doch nur Lunas Erlebnisse. Stumm lächelte er in sich hinein.

Nach diesem kurzen Überblick fing das Ereignis an.

\begin{rueckblick}
Professor Elber stand vor dem Lehrertisch und hielt wieder eine seiner ungeliebten Reden.

\enquote{Heute Abend ist das große Ereignis, das lediglich als \accentuate{Magie in Konzert} ausgeschrieben wurde. Hierbei handelt es sich, wie der Name schon sagt, um ein Konzert. Allerdings spielt nicht eine Band, sondern mehrere. Es ist eine Art Benefizveranstaltung, da Aufnahmen gemacht und verkauft werden. Diese Veranstaltung ist mit Technik der Muggel aufgemacht. Also keinerlei Zauberei, wenn ich bitten darf. \gst Da Sie das ganze Jahr über sehr gut mitgearbeitet haben und stetig Interesse gezeigt hatten, dachte ich mir, ich revanchiere mich damit. Es gaben mir sogar viele Schüler Rückmeldung, was sie vom Unterricht gehalten haben und was sie noch lernen möchten. Deswegen, als großes Danke, habe ich dieses Ereignis, mit Erlaubnis unseres Schulleiters, organisiert. Außer den Schülern werden auch eine gewisse Anzahl an Muggeln dem Ereignis beiwohnen, Muggel, die Sie kennen. Allgemein gesagt, Leute, die schon Weihnachten hier waren. \gst Die \accentuate{Veranstalter} sind alles Muggel, die hinterher ein paar Details vergessen werden. Die werden sich aber an einen großen Teil erinnern. Dafür sollten eigentlich Auroren sorgen. Da diese aber}, er pausierte kurz, \enquote{nicht zur Verfügung stehen, wird das Lehrerkollegium diese Aufgabe übernehmen. Passen Sie trotzdem auf, was Sie in Gegenwart des Personals sagen. Die Festivalteilnehmer dürften unproblematischer sein. Alle haben in gewisser Weise mit der magischen Gemeinschaft zu tun. In acht Stunden geht es los. Eine Stunde vorher ist Einlass.}

Jetzt begann aufgeregtes Geschnatter in der Großen Halle. Unzählige Diskussionen entstanden in der Großen Halle. Harry bekam viele Diskussionsfetzen mit. Mehr, als wenn er selbst in der Halle gesessen hätte. Diese Erfahrung war für ihn neu.
\end{rueckblick}

Von dem quengelnden Elfen wieder in die Realität zurückgekehrt, überlegte er, was er dem Elfen als Beschäftigung geben könnte. Er zauberte eine Rassel hervor, die den kleinen für eine gute Stunde beschäftigte.

Die Rassel wirkte auf ihn beruhigend, sodass Harry wieder seine Augen schloss und seinen Kopf zurücklehnte. Luna übersprang die sieben Stunden und fing wieder an, als es auf das Quidditch-Feld ging.

\begin{rueckblick}
Ein leichtes Kribbeln war zu spüren, als Luna durch den Eingang auf das Feld trat. Auf beiden schmalen Seiten der Arena waren Bühnen aufgebaut. In etwa einen Meter und fünfzig standen Verstärker und Lautsprecher. Im hinteren Teil der Bühne war etwas zu sehen, was an Orgelpfeifen erinnerte. Sie suchte sich einen guten Platz in der Mitte, da Luna vermutete, dass auf beiden Seiten Auftritte waren. Ein Teil der Schüler ging näher an die Bühne heran. Luna sah sich um. Sie drehte sich einmal im Kreis. Es hatte sich außer den beiden Bühnen an den Stirnseiten nichts geändert. Vom Eingang her kam ihr Vater auf sie zu. Sie lächelte ihn an.
\end{rueckblick}

Dann gab es einen Schnitt. Die Ansicht wechselte und Professor Dumbledore stand auf der hinteren Bühne und fing an zu erzählen. Vermutlich war die Unterhaltung zu privat.

\begin{rueckblick}
\enquote{Ich freue mich, dass wieder einmal viele Eltern, Verwandte und Bekannte hierhergekommen sind. Erneut gibt es etwas hier auf diesem schönen Feld zu feiern. Auch wenn es abseits gelegen ist. Aber das kann nur von Vorteil sein. Wir können nicht wegen Ruhestörung oder zu lauter Musik belästigt werden. Ich will euch nicht lange aufhalten. Ihr wartet sicher schon alle gespannt auf den ersten Gig. Damit gebe ich die Bühne frei. Hinter euch fängt nun die erste Gruppe an, während auf dieser Bühne aufgebaut wird. Die Lieder finden also abwechselnd auf beiden Bühnen statt. Viel Spaß.}

Dann begann die Musik und Luna drehte sich mit dem Rest der Konzertteilnehmer um.

\enquote{Hallo Freunde}, begann der Sänger der \accentuate{Wirbelnden Kröten}. \enquote{Wir freuen uns, dass wir zu so einem bedeutenden Ereignis eingeladen wurden. Deshalb haben wir jetzt erstmalig für euch unser neustes Lied. \gst Federleicht.}

Die Musik begann mit einem Schlagzeug, das das gesamte Lied über denselben eingängigen Bolero-artigen Takt vorgab. Doch dieses Lied wurde von der Gitarristin gesungen. Mit heller Stimme fing sie an.

\begin{lied}
Wie eine Feder

getragen durch Magie

schweb’ ich durch die Welten

doch ich finde nie

das, das was ich suche

liegt doch klar vor mir, doch mein Ziel erreiche

ich wohl doch nie\abs
\end{lied}

Das Lied war leicht, fand Luna. Sie sah zu ihrem Vater, der selig lächelnd der Musik zuhörte und einen Blick hatte, der scheinbar die Umgebung ausblendete und nur auf die Bühne gerichtet war.
\end{rueckblick}

Immer abwechselnd auf beiden Bühnen kamen die Lieder dran. Aber eines davon beeindruckte Harry besonders. Es wurde extra angekündigt, damit die Bühnen-Crew, die das gesamte Konzert mitfilmte, nicht verwundert schaute.

Harry warf einen Blick auf Frodo, der immer noch mit der Rassel spielte und ab und an zum Fenster auf die vorbeiziehende Landschaft schaute.

\begin{rueckblick}
\enquote{Das nächste Lied ist etwas Besonderes. Wie ich einigen von euch schon mitgeteilt hatte, testen wir heute unter realen Bedingungen ein neues Stück Technologie, das realistische Bilder projizieren kann.} Wieder kam ein Lied, das mit einem Schlagzeug anfing. Dann kam von unter der Bühne ein kreisrunder Stempel hervor mit einem kleinen Geschöpf darauf. Es war ein Medusoner. Mit hoher Stimmlage begann sie. \enquote{Das Lied nennt sich Stardust.} Das Schlagzeug begann und eine Gruppe weiterer Medusoner trat auf die Bühne. Jeder im Publikum wusste, dass die Ansprache von Dumbledore nur dazu diente, die Crew zu beruhigen.

\begin{lied}
We are waiting for the sun

We are waiting for the sun

to come out and play

all these rainy days

I getting all

we are waiting for the stars

we are waiting for the stars\abs
\end{lied}
\end{rueckblick}

Das hatte Harry beeindruckt. So ein kleines Wesen, dass sich traute, vor so einem großen Publikum aufzutreten. Die nachfolgenden Lieder hörte er überwiegend in seinem Geist, da er etwas Bewegung brauchte. Ginny unterhielt sich mittlerweile mit Hermine und Ron aß gerade. Luna schien Neville zu beobachten, der vor sich hin döste. Er wollte gerade umdrehen und Luna Bescheid geben, dass sie kurz auf Frodo aufpassen solle, als er ihn schon an seiner Hand spürte, wo er sich festhielt.

\enquote{Frodo mitgehen}, sagte der Elf, der seine Hand hielt und ihn fröhlich ansah.

Harry sah lächelnd zu ihm herab und nickte. \enquote{Also gut, gehen wir ein wenig.}

Durch seine Größe fiel Frodo niemandem auf, der in den Abteilen saß. Nur als er ein Großraum-Abteil durchquerte, erregten sie Aufmerksamkeit. Wieder einmal gab es Getuschel. Frodo störte das nicht und Harry ignorierte es. Er war das gewohnt. Als sie der Hexe mit dem Süßigkeiten-Wagen begegneten, kaufte Harry etwas für Frodo. Er war ein wenig wie ein kleines Kind.

Erst als er Katharina und Alina entdeckte, die gerade ihre Köpfe zusammen gesteckt hatten, realisierte er, dass er im Slytherin-Abteil angekommen war.

Beide sahen zu ihm auf und begrüßten ihn. \enquote{Hallo, Harry. Wen hast du denn mitgebracht?}

\enquote{Das hier ist Frodo. Er hat sich eingeschlichen. Ich passe auf ihn auf, bis ihn sein Vater in London abholt.}

Beide rückten etwas zusammen und machten Harry Platz. Er setzte sich auf die Gang-Seite und nahm Frodo, der ihn bettelnd ansah, auf seinen Schoß.

\enquote{Und was macht ihr?}, fragte Harry.

\enquote{Wir verabreden uns. Wir wollen uns in den Ferien mal treffen. Nachdem sie von der Krankenstation wieder gekommen war, habe ich ihr beim Schulstoff geholfen. Da haben wir uns angefreundet. Jetzt besuchen wir uns in den Ferien.}

Dann begann der Zug zu bremsen. Es war fast eine Notbremsung. Innerhalb hundertfünfzig Metern hielt er an. Katharina öffnete das Fenster und sah nach draußen. Vor den Schienen schien etwas zu liegen. Eine Sperre. In der Ferne sah man Dementoren auf den Zug zufliegen und eine Menge schwarzer Gestalten standen da und warteten.

Dann ertönte ein Pfiff und eine Menge Elfen tauchte auf und verteilte sich rings um den Zug. Sie warteten auf die Dementoren. Einige sahen sich um warfen ihren Blick auf den Zug und ihre Insassen. Eine der weiblichen Elfen entdeckte Harry und poppte in das Innere des Zuges.

\enquote{Mister Potter, Sir. Würden Sie bitte die Dementoren mit einem Patronus vertreiben? Um die anderen werden wir uns kümmern. Nur können wir nicht gleichzeitig gegen beide vorgehen.}

Harry verstand. \enquote{Passt ihr so lange auf Frodo auf?}, fragte er Katharina und Alina. Beide nickten. \enquote{Gut, ich komme mit.}

Die Elfe nahm ihn bei der Hand und poppte zurück zu den anderen.

\enquote{Mister Potter wird sich um die Dementoren kümmern}, sagte die Elfe.

Harry wartete, bis er die übliche Kälte spürte. Dann stelle er sich die schönsten Szenen vor. Er hätte erwartet, Luna zu sehen. Doch dann kam ihm Ginny in den Sinn und vor sein geistiges Auge. Er sah klar und deutlich die Szene im Bad vor sich, als sie über ihn stieg. Dann schwenkte sein Verstand zu Luna und anschließend zu Lavender. Und dann zu Myrte. Diese Kombination gab ihm die notwendige Kraft. Er zog seinen Zauberstab und heilt ihn nur noch nach oben. Sofort löste sich eine gewaltige blau schimmernde Welle und breitete sich kuppelartig aus. Welle um Welle trennte sich vom Zauberstab und näherte sich den Dementoren. Nach vergeblichen Versuchen durchzubrechen kehrten sie um und verließen den Schauplatz.

Die Elfe brachte Harry wieder in den Zug zurück, wo sofort Frodo wieder bei ihm war.

Die Angreifer, die bislang nur dastanden und hofften, dass die Dementoren die Zuginsassen so weit schwächten, dass sie nur noch den Zug durchstreifen mussten, waren für wenige Sekunden verwirrt. Dies nutzten die Elfen und fingen an, die ersten mit Seilen zu fesseln und am Apparieren zu hindern. Sie hatten rund ein Drittel gefesselt, als sich der Rest anfing zu wehren. Es war ein ungleicher Kampf. Die Angreifer hatten gegen die Elfen kaum eine Chance. Rund ein Drittel des Restes starb durch zurückprallende oder umgelenkte Flüche, oder weil die Elfen sich duckten oder zur Seite sprangen und der Zauber einen Kollegen traf.

Der Rest disapparierte. Die Elfen warteten noch einige Minuten. Dann begann ein Teil von ihnen die Schienen freizuräumen und scheinbar dem Zugführer Bescheid zu sagen. Bilbo tauchte auf und Harry freute sich, da er die Verantwortung für Frodo abgeben konnte.

\enquote{Danke, dass Sie auf meinen Sohn aufgepasst haben.}

\enquote{Keine Ursache, Mister Bilbo.}

Die Zugpfeife ertönte und Bilbo verabschiedete sich. \enquote{Wir sehen uns, Mister Potter.} Dann poppte er und war verschwunden.

Der Zug setzte sich wieder in Bewegung und Harry setzte sich neben Katharina.

\enquote{Wer wohl die Elfen geschickt hat?}, fragte Katharina.

\enquote{Weiß nicht}, meinte Harry und hob seine Schultern. \enquote{Vielleicht war es Professor Elber, ich bin mir aber nicht sicher.}

\enquote{Hast du wieder eine Freundin? Nachdem du dich ja von Luna und Pansy getrennt hast\abs}

\enquote{Ja.}

\enquote{Und wen?}

\enquote{Ginny.}

\enquote{Weasley?}

\enquote{Ja.}

\enquote{Und wie läuft es?}

\enquote{Ganz gut. Wir haben uns mehrmals ausgesprochen. Das liegt wohl daran, dass ich nach meinem dreiwöchigen Aufenthalt im Krankenflügel an Erinnerungslücken litt. Ich musste mir erst einmal sagen lassen, was ich vergessen habe. Deshalb habe ich mich mit Ginny unterhalten. Dadurch ist unsere Verbindung enger geworden, finde ich. Wir sind aneinander gewachsen.}

\enquote{Du meinst wohl ineinander, Potter}, meldete sich plötzlich Draco Malfoy. Bisher hatte er sich ruhig verhalten, aber er muss ja seiner Rolle gerecht werden.

\enquote{Du weißt doch, dass ich es manchmal mit den Worten nicht so genau nehme, Malfoy}, konterte er.

Das sollte ihm genug Zeit verschaffen, sich zu verabschieden.

\enquote{Wir sehen uns}, sagte er zu Katharina und Alina. Dann ging er zurück in sein Abteil.

\enquote{Wo ist der Kleine?}, fragte Luna sofort, da er ohne zurückkam.

\enquote{Sein Vater hat ihn abgeholt, als der Zug angehalten hat.}

\enquote{Was ist denn passiert? Wir haben nur mitbekommen, dass der Zug angehalten hat. Dann dauerte es etwas. Dann kamen Dementoren, die durch einen Patronus vertrieben wurden.} Harry zeigte auf sich und nickte. \enquote{Dann gab es einen ungleichen Kampf zwischen den Elfen und den Todessern? Oder was waren die? Dann ging es wieder weiter. Habe ich was vergessen?}

\enquote{Nein, Luna. Das war alles.}

\enquote{Dann ist ja gut.} Sie widmete sich wieder Neville, neben dem sie jetzt saß.

Ginny stand auf und deutete Harry an, er möge sich auf ihren Platz setzen. Dann setzte sie sich auf seinen Schoß.

Harry schloss wieder die Augen, nachdem Ginny sich an ihn geschmiegt hatte und lauschte Luna Erlebnissen.

\begin{rueckblick}
Dann wurde eine Wand am Bühnenende gesenkt und ein eigenartiges Instrument kam zum Vorschein. Ein alter Mann trat mit Anzug auf die Bühne. Er begann mit einer sehr kurzen Erklärung seines Instrumentes. \enquote{Dies hier ist ein Trautonium, entwickelt von Professor Trautwein um 1930. Das folgende Lied heißt \accentuate{Canon Caprice 2}.} Dann setzte er sich und begann zu spielen.\\
Es war ein eigenartiges Lied. Elektronische Musik hatte bisher keiner der Zauberer, die nicht bei Muggeln aufgewachsen waren, gehört. Und selbst die anderen hatten derartige Klänge noch nie gehört.
\end{rueckblick}

Der Zug hielt an und Harry stieg aus. Er verabschiedete sich von Ron und Hermine, begrüßte Mr. und Mrs. Weasley, blieb mit Ginny zurück und küsste sie zum Abschied noch einmal. Er sah Pansy, wie sie mit dem Hauselfen sprach, der sie auch schon vor Hogwarts angesprochen hatte, ihm ihre Hand gab und dann mit ihm verschwand. \gedanke{Sie geht wohl zu ihrem Paten}, dachte Harry. Dann durchschritt er die Absperrung zur Muggelwelt, wo ihn schon seine Tante erwartete. Harry sah seinen Onkel und auch seinen Cousin nicht. Sie nahm ihn mit zum Parkplatz wo er seinen Koffer einladen musste. Danach fuhr sie mit ihm schweigend nach Hause.

Etwas hatte sich mit seiner Tante verändert, dachte er noch, als er in seinem Bett im Ligusterweg 4 einschlief und zu träumen begann. Doch zuvor bekam er noch den Abschluss des Konzerts mit, den Luna ihm noch übersandt hatte und den er bis jetzt ignorierte.

\begin{rueckblick}
Seitlich durch den Haupteingang kam eine Gruppe von Dudelsack-Spielern. Sie fingen an mit einem Riff, den man eher von Gitarren gewohnt war.\\
Während die eine Hälfte den Rhythmus spielten, dudelte die andere Hälfte die Melodie.
\end{rueckblick}

Harry musste grinsen. Er kannte das Lied. \liedinline{Thunderstruck}. Auf dem Dudelsack hörte sich das einfach unglaublich an.

\trenn

Während Harrys Fahrt nach Hause, apparierte der Elf mit Pansy in ein kleines Anwesen. \enquote{Das hier ist ihr Zimmer, Miss}, sagte der Elf und zeigte auf die Tür vor ihnen. \enquote{Morgen früh wird sich ihr Pate um Sie kümmern. Ich wünsche Ihnen eine gute Nacht.} Dann verschwand der Elf. Pansy stand noch staunend vor der Tür und griff schließlich an die Klinke, um sich ihr Zimmer anzusehen. Sie ließ ihre kleine Tasche fallen, die sie noch in der Hand hielt, und ihr Mund klappte auf. Das war etwas ganz anderes als zu Hause, oder in Hogwarts. Zu Hause hatte sie ein mittelgroßes Zimmer mit einem Schreibtisch, einem großen Schrank für ihre Kleidung, ein Bett und ein großes Regal, in das sie ihre Schulbücher legte und andere Bücher zum Lesen oder Studieren hatte. In Hogwarts musste sie ihr Zimmer mit anderen Mitschülern teilen und konnte ihre Hausaufgaben im Gemeinschaftsraum machen.

Doch dieses Zimmer sah anders aus. Sie fand ein ähnliches Bett wie ihres in Hogwarts vor. Nur war es etwas größer. Das Zimmer war doppelt so breit und anderthalbmal so lang wie ihr Zimmer bei ihren Eltern. Eine ganze Wand bedeckte ein Regal, in dem nicht nur ihre gesamten Schulbücher lagen, sondern auch noch jede Menge andere Bücher, die sehr wichtig aussahen. In die Regale integriert war etwas, das nach einem Wandschrank aussah.

Sie betrat den Raum, ging auf den Schrank zu und öffnete ihn. Im Inneren waren ihre sämtlichen Kleidungsstücke, die sie bei ihren Eltern lassen musste. Betäubt durch die Eindrücke schloss sie die Schranktür wieder und sah ganz erstaunt auf den Elfen neben sich, er war aus dem Nichts aufgetaucht, der noch ein paar Bücher in das Regal stellte. Es waren Pansys Lieblingsbücher.

\enquote{Sind etwa meine gesamten Sachen aus meinem Zimmer hier?}, fragte Pansy ganz erstaunt.

\enquote{Dies \accentuate{hier ist} ihr Zimmer, Miss. Aber ja, alle Sachen aus ihrem \accentuate{alten} Zimmer sind nun hier.}

\enquote{Aber, wie bist du\abs}

\enquote{Der Elf der Familie Parkinson war sehr hilfsbereit. Er muss Sie sehr mögen.}

Pansy taumelte nach hinten und ließ sich auf ihr Bett fallen.

\enquote{Alles gut, Miss?}, fragte der Elf besorgt nach.

Pansy nickte nur, immer noch nicht fassend, was gerade passiert war. \gedanke{Meine Eltern haben mich hinausgeworfen. Mein Vater konnte mich ja noch nie richtig leiden. Aber jetzt bin ich bei meinem Paten, den ich nicht einmal kenne, und habe ein Zimmer, das schöner ist als alle Zimmer, die ich bisher gesehen habe. Es passt zu mir.} Der Elf war wieder verschwunden und Pansy zog sich aus und danach ihr Nachthemd an, sie verschwand noch einmal ins Bad und stieg danach ins Bett um grübelnd einzuschlafen.

Als sie am nächsten Tag erwachte, ging ihr erst einmal ein Gedanke durch den Kopf. \gedanke{Das war ein schöner Traum gestern. Ich habe in einem bequemen Bett in einem Zimmer\abs}, dann schlug sie die Augen auf und blickte sich um. \gedanke{Kein Traum. Ich bin wirklich hier.} Sie sah sich um und entdeckte ihren Hogwarts-Koffer. \gedanke{Habe ich den gestern gar nicht mehr gesehen, oder ist der in der Nacht hereingekommen?}, fragte sie sich. Unsicher und noch leicht schläfrig stand sie auf und ging zu ihrem Schrank. Dort waren nun mehr Kleider und Anziehsachen, als am Abend zuvor. Nachdenklich schloss sie den Schrank wieder und richtete sich für das Frühstück, das ihr Magen lautstark erwartete. Sie stieg die Treppe hinunter in das Erdgeschoss und lief die Spur aus einzelnen Blumenblättern nach. Es waren Lilienblüten. \enquote{Sind das nicht auch die Lieblingsblumen von Harrys Mutter?}, fragte sie sich. Dann hatte sie die Schiebetüre erreicht, die durch zwei auf zierlichen Holztischchen stehenden Blumenvasen eingerahmt waren und öffnete sie. Der Tisch war bereits gedeckt und Pansy blickte durch die Fensterwand auf der gegenüberliegenden Seite des Raumes hinaus in den Garten.

Dann setzte sie sich im weiß gestrichenen Zimmer auf einen Stuhl und begann sich ihren Teller zu beladen. \enquote{Nur meine Lieblingssachen, die ich gerne Frühstücke}, ging es ihr erneut durch den Kopf. \enquote{Das Himmelbett hatte dieselbe Zeichnung auf dem Baldachin wie in der Schule. Das kann doch kein Zufall sein. Was weiß mein Pate alles über mich, was ich nicht weiß? Wieso hat er sich nicht mehr seit meinem dritten Geburtstag gemeldet? Wieso kann ich mich nicht mehr an sein Gesicht erinnern?}

\enquote{Guten Morgen, Miss}, schreckte sie der Hauself aus ihren Gedanken. \enquote{Das hier ist für Sie.}

Pansy nahm den Brief, den ihr der Hauself gab, und las ihn sofort. Sie registrierte nicht die kleinen tippelschritte und das Knarzen des Stuhles, da sie so in ihrem Brief fixiert war.

\begin{brief}
Liebe Pansy,

tut mir leid, dass ich nicht zum Frühstück da sein kann, aber ich muss unbedingt noch vorher etwas erledigen.

Bin gegen 10 Uhr \gst
\end{brief}

Pansy sah auf ihre Uhr. Es war kurz nach 8 Uhr.

\begin{brief}
Bin gegen 10 Uhr wieder da und leiste dir ab da dann Gesellschaft. Ich werde dir dann deine Fragen, die dir sicher schon auf deiner Zunge brennen, beantworten.
\signumspace
Liebe Grüße

Onkel Fred
\end{brief}

\gedanke{Onkel Fred}, ging ihr durch den Kopf. \gedanke{Wenigstens weiß ich jetzt seinen Namen, seinen Vornamen.} Sie legte den Brief beiseite und biss wieder in ihr Brötchen, als sie den Elfen ihr gegenüber sitzend erblickte und kurz zusammenzuckte. Stumm und ausdruckslos, mit dem Brötchen im Mund, sah sie ihn an. Der Elf blickte kurz zurück und sah dann verständnislos an sich herunter. Dann sah er wieder zu ihr und schließlich über den Tisch, ob auch alles in Ordnung sei, bis es ihm schließlich dämmerte und er meinte zu wissen, was los sei.

\enquote{Oh, Verzeihung Miss, Remmy vergaß, dass Sie es offensichtlich nicht gewöhnt sind, mit Hauselfen zu frühstücken.}

Pansy biss den Teil des Brötchens in ihrem Mund ab und schüttelte dann ihren Kopf.

\enquote{Remmys Chef meint, dass die Familie, wenn möglich, zusammen essen sollte. Da Remmy normalerweise mit ihrem Chef und den anderen zusammen frühstückt, die aber nicht da sind, ist es Remmys Aufgabe, dies für Sie zu tun.}

\enquote{Du musst aber nicht\abs}, sagte sie, als sie ihren Bissen herunterschluckte, \enquote{mit mir frühstücken\abs}

\enquote{Remmy hatte heute selbst noch nichts. \gst Ich genieße das gemeinsame Frühstück mit der Familie}, fügte der Elf hinzu, nachdem er durch das Fenster in den Garten sah.

\gedanke{Ich genieße das}, dachte Pansy. \gedanke{Seit wann reden Elfen denn in der Ich-Form von sich?}

\enquote{Remmy?}, fragte Pansy vorsichtig nach und der Elf sah sie wieder an. \enquote{Du hast gerade \inner{Ich genieße das} gesagt.} Der Elf bekam große schuldbewusste Augen und wollte gerade etwas erwidern, doch Pansy unterbrach ihn. \enquote{Mich stört das nicht, nur ist es mir gerade aufgefallen.}

Remmy zog den Kopf leicht ein und meinte dann leise: \enquote{Unser Chef legt keinen Wert auf solche Unterwürfigkeit. Wir sollen uns nicht selbst beim Namen nennen. Ich tat das nur, um sie nicht zu sehr zu irritieren.} Und dann, nach einer kleinen Pause, fügte er leicht beschämend hinzu: \enquote{Er würde uns sogar Kleidung geben, wenn wir welche wollen}, doch er unterbrach sich, schlug sich die Hand vor den Mund und eilte hinaus. \enquote{Hab noch viel zu tun, Miss.}

Grinsend sah sie ihm nach. \gedanke{Hast wohl zu viel über deinen Herrn verraten}, ging ihr durch den Kopf.

Als sie fertig war mit dem Frühstücken, stand sie auf, öffnete die Fensterwand und trat einige Schritte auf die Terrasse hinaus. Nachdenklich schaute sie in den Garten, der aus allerlei verschiedenen Pflanzen bestand. Es schien keine Pflanze zu geben, die hier nicht wachsen würde.

Es klingelte an der Haustür und Pansy hörte ein leises \geraeusch{Plopp}. Die Haustür wurde geöffnet und eine Stimme sagte: \enquote{Ich habe hier ein Paket für E\agst}

\enquote{Schon gut, ich nehme es}, sagte der Elf. Kurz darauf fiel die Haustür ins Schloss und es war wieder still.

Dann, so gegen 10 Uhr, ging die Haustür auf und eine Stimme sagte: \enquote{Ich bin wieder da.} Pansy kam die Stimme bekannt vor, aber sie wusste nicht, woher sie sie kannte. Dann hörte sie Schritte. Sie traute sich nicht, sich umzudrehen, bis sie eine Hand auf ihrer Schulter spürte. \enquote{Es freut mich, dass du nicht unter einer Brücke schlafen musst. Dir wäre es vielleicht lieber gewesen, bei einer deiner Freundinnen oder einer Schulkameradin oder -kameraden zu sein, als bei mir.} Jetzt lief ihr ein kalter Schauer über den Rücken. Diese Stimme kannte sie genau. \enquote{Es tut mir leid, dass ich dich das ganze Schuljahr über getriezt und unfair behandelt habe, aber ich kann dir erklären, warum.}

Sie drehte sich leicht zitternd um und sah Frederick Elber in die Augen, ihrem ehemaligen Lehrer in \VgddK. \enquote{Sie\gst}, begann sie.

\enquote{Du, Pansy. Nenn mich Frederick, oder Fred oder Onkel Fred, aber siez mich nicht. Setz dich. Ich denke, ich bin dir eine Erklärung schuldig.}

Er nahm sie bei der Hand und führte sie in eine etwa zehn Metern entfernte eiche-lasierte Holzlaube und setzte zuerst Pansy hin und danach sich ihr gegenüber. Er sah auf die Tischplatte und begann zu erzählen.

\enquote{Es war kurz vor deiner Geburt, als deine Mutter zu mir kam und mich bat, deine Patenschaft zu übernehmen. Dein Vater war nicht sonderlich begeistert, stimmte aber zu. Dann, einen Tag vor deiner Geburt, musste er dringend geschäftlich weg. Also war ich an seiner Stelle da und stand deiner Mutter bei. Ja, ich war da}, sagte er, als er kurz aufsah und Pansys Blick einfing. \enquote{Ich war bei deiner Geburt dabei und sah dich. Klein und schmutzig.} Er lächelte sie an. Dann sah er in ihre Augen und fuhr fort. \enquote{Du hast große Ähnlichkeiten mit deiner Großmutter. Auf jeden Fall kam dein Vater einen Tag danach wieder zurück und nahm dich glücklich in seine Arme. Ich bot ihm an, in einem Denkarium deine Geburt zu sehen, doch er lehnte ab.}

\enquote{Was ist ein Denkarium?}, fragte Pansy nach.

\enquote{In ein Denkarium legt man Gedanken, die man sich noch einmal ansehen will. Man ist noch einmal am selben Ort, den man schon einmal erlebt hat. Es ist eine Art Illusion, die aus den eigenen Gedanken erzeugt wird. Man kann sich so auf Details konzentrieren, die man sonst leicht übersieht.}

Pansy nickte.

\enquote{Dann kam dein erster Geburtstag, dein zweiter Geburtstag und schließlich dein dritter. Dann brach der Kontakt ab und ich erhielt später einen Brief, dass ich nicht mehr kommen brauche. Ich bin persönlich bei ihm gewesen und habe ihn gefragt, doch er warf mich raus. \gst Ich schrieb dann ein paar Tage später einen Brief, in dem ich nachfragte, was ich falsch gemacht hatte, warum ich nicht mehr zu euch kommen durfte. \gst Als Antwort bekam ich einen Heuler, in dem er mir androhte, dir etwas anzutun, wenn ich dir zu nahe kommen würde und mich mit dir anfreunden würde. \gst Deshalb bin ich in der Schule immer so gemein zu dir gewesen. Ich hatte Angst um dich. Angst, dass er dir etwas tun würde, wenn in einem Brief mein Name fallen würde und du zu sehr Gutes von mir berichten würdest.}

Es sah Pansy wieder genau in die Augen, nachdem er während seines Vortrages ihr Gesicht abgesucht hatte.

\enquote{Und das soll ich dir glauben?}, fuhr sie ihn an.

\enquote{Schreib nach Hause, Pansy, wenn du mir nicht glaubst. Einen Brief an deinen Vater und einen an deine Mutter. Dann warte ab. Du kannst meine Eulen nehmen, oder deine Briefe an meinen Hauselfen geben.}

\enquote{Das werde ich auch tun}, sagte sie zornig, stand auf und ging in ihr Zimmer, um ihren Eltern zu schreiben.

Sie begann zwei identische Briefe aufzusetzen, die sich nur in der Anrede unterschieden.

\begin{brief}
Hi Mum,

ich sitze gerade bei einer Freundin, bei der ich die kommende Woche verbringen werde. Wir haben uns gestern Abend noch ein paar alte Fotos angesehen, als sie mich fragte, warum denn mein Pate keine Zeit für mich habe. Da fiel mir ein, dass ich ihn schon lange nicht mehr gesehen hatte. Was macht er überhaupt? Warum hat er sich von uns zurückgezogen?

Bitte antworte mir.
\signumspace
Pansy
\end{brief}

Sie rief nach Remmy und fragte ihn, ob er die Briefe übermitteln könnte. Er nickt, nahm die Briefe an sich und verschwand.

Der Rest des Tages verlief recht schweigsam, da Pansy die meiste Zeit in ihrem Zimmer verbrachte und sich mit ihren Hausaufgaben ablenkte. Sie kam erst gegen Abend zum Essen herunter und schaufelte daher ihre Portion gierig in sich hinein.

Ihr Pate saß mit seinem Elfen still am Tisch und sprach nichts mit ihr. \gedanke{Sie wird sich schon melden, wenn sie ihre Antwort hat}, dachte er.

Mitten im Essen tauchte eine Eule auf, die einen Heuler mitbrachte. Er war für Pansy. Sie band ihn los und wollte damit in ihr Zimmer rennen, doch er qualmte bereits verdächtig. Als sie die Vorhalle bereits zur Hälfte durchquert hatte, riss sich der Heuler los und explodierte. Sie hörte die Stimme ihres Vaters.

\extase{Dein sauberer Pate ist tot. Dieser Wahnsinnige hat sich selbst umgebracht, als er an neuen Tränken forschte. Er war ein Arsch. Frage ja nie wieder nach ihm und mich überhaupt nichts mehr du Todesser-Hure. Du bist für mich gestorben, du Flittchen.}

Geschockt stand Pansy nun da und schaute ängstlich und mit Tränen in den Augen zurück. Frederick hob seinen Kopf und schaute sie an, als ob er fragen würde: \accentuate{Wirst du weglaufen, wenn ich dir zu helfen versuche?}

Pansy kniff ihre Augen zu und die Tränen bahnten sich ihren Weg über ihr Gesicht. \gedanke{Todesser-Hure hat er mich genannt. Mein eigener Vater.} Doch sie realisierte sehr bald, dass das nur eine weitere Ausprägung seiner wütenden Art war. Sie spürte, wie sich ein Stück Stoff über ihre Wangen bewegte. Der Spur der Tränen aufwärts folgend. Dann fühlte sie Finger an ihrem Kinn, das leicht angehoben wurde. In ihrem Gesicht spürte sie einen warmen Atem, bevor ihre Stirn geküsst wurde. Dann spürte sie noch, wie etwas in ihre Tasche gesteckt wurde und wie sich jemand einen halben Schritt weit entfernte.

Langsam öffnete sie ihre tränenden Augen und sah zu ihrer Tasche. Dort ragte ein Stoff-Taschentuch heraus. Sie nahm es und erstickte neue aufkommende Tränen im Keim. Eine Eule flatterte durch das Fenster und setzt sich auf Pansys Schulter. Pansy nahm ihr den Brief ab und ihr Pate sagte der Eule, dass auf dem Tisch etwas Speck für sie liegen würde. Sie könnte sich auch in dem Eulenhäuschen eine Weile ausruhen, falls sie es denn wolle. Die Eule schuhute und flog in das Nebenzimmer, wo sie einen Streifen Schinken nahm und in die Nacht verschwand. Wohin, das konnte man nicht mehr sehen.

Pansy öffnete ihren Brief und wischte sich noch einmal die Tränen fort, bevor sie las. Ihr Pate stand immer noch vor ihr.

\begin{brief}
Liebe Pansy,

du weißt, dass ich gegen die Entscheidungen und Taten deines Vaters nichts ausrichten kann. Mir droht aber keine Gefahr. Er hat sich tierisch aufgeregt, als er erfahren hatte, wer dein Freund ist. Ich hoffe, du hast die anderen Wochen auch einen geeigneten Unterschlupf.

Von deinem Paten habe ich schon lange nichts mehr gehört. Sein Name ist aber, falls du ihn brauchen solltest, Frederick Elber. Dein Vater hat mir nie gesagt, warum er den Kontakt zu ihm abgebrochen hat. Aber du kannst froh sein, dass er sich darüber gefreut hat, wie er dich behandelt, als du uns geschrieben hast, sonst hätte er dich vermutlich verprügelt.

Ich habe unseren Elfen angewiesen, dass, falls du etwas brauchen solltest, er es dir bringen soll. \gst Er steht gerade neben mir und sagt mir, dass deine Sachen bereits abgeholt wurden. Ich beauftrage ihn, deine restlichen Sachen \gst sollten noch welche übrig sein \gst dir zu schicken. Und ein paar Erinnerungsfotos werde ich ihm auch noch mitgeben. Das dauert allerdings noch ein paar Tage.
\signumspace
In der Hoffnung, dass es dir gut geht\gst

Deine dich liebende Mum
\end{brief}

Erneut liefen Pansy Tränen über ihr Gesicht. Sie sah ihren Paten stumm an und er schien ihre Sehnsucht zu begreifen. Er ging auf sie zu und nahm sie fest in den Arm.

Als sie am nächsten Morgen aufwachte, erinnerte sie sich nur noch, dass sie nach dem Lesen des Briefes ihrer Mutter an der Schulter ihres Paten geheult hatte und dann wohl eingeschlafen war. Sie zog sich um und ging dann nach unten um zu frühstücken. Freundlich erwartete sie bereits Onkel Fred, der ihr zulächelte. Sie gab ein mattes Lächeln zurück, traurig, dass sie verstoßen wurde, aber doch glücklich, scheinbar eine neue Heimat gefunden zu haben.

\enquote{Wie ist dein Verhältnis zu Harry Potter?}, fragte sie ihr Pate während des Frühstücks.




\begin{kommentar}
Das erste Lied des Konzerts habe ich mir selber ausgedacht. Dann kommt das Lied, das ein Medusoner singt. Später wird es Lena Mayer-Landruth singen. Es wird die Titelmelodie zum Film Jesus liebt mich.
\end{kommentar}

\begin{kommentar}
Und zu den Dudelsack-Bläsern: Sucht einfach auf Youtube nach Thunderstruck und Dudelsack.
\end{kommentar}

\begin{kommentar}
Ich finde, der letzte Satz ist eine sehr schöne Überleitung zur nächsten Geschichte.
\end{kommentar}


\vspace{\baselineskip}
\begin{center}
\gst \block{Das Ende} \gst
\end{center}

\includekommentare{}
%\input{tex/Reviews.tex}

\end{document}
