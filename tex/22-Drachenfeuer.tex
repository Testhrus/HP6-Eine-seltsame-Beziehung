\chapter{Drachenfeuer}


Die beiden landeten in Rumänien und wurden schon freudig von Nepomuk und den anderen Drachenhütern erwartet. Frederick machte sich sofort an die Arbeit und setzte sich vor Nepomuk. Dieser legte sich nun hin und hielt ihm seine Schnauze entgegen. Beide schlossen ihre Augen und die Heilung begann. Währenddessen unterhielten sich die beiden Drachen Tabaluga und Nepomuk über Tabalugas Ausflug. Die Idee der Feuerprobe gefiel Nepomuk. Darüber musste er unbedingt mit Frederick sprechen.

Während der Heilung wurde sein Geruchssinn immer besser.

Nachdem sie fertig waren, fragte Nepomuk nach. \stimme{Habe ich das richtig verstanden? Ihr wollt eine Feuerprobe bei einem eurer Schüler vornehmen?}

Frederick sah auf und dem Drachen direkt in sein Gesicht. Dann blickte er kurz zu Tabaluga. Schließlich sagte er einfach nur: \enquote{Ja!}

\stimme{Da würde ich gerne der zweite Drachen sein.}

Frederick hob seine Augenbrauen. \gedanke{Bist du dazu in der Lage? Verstehe mich nicht falsch. Aber den Feuerzauber an einem Menschen zu vollziehen schafft nicht jeder.}

\stimme{Ja, das schaffe ich. Meine Mutter hat mir das beigebracht. Sie meinte, es sei alte Familientradition, das zu können.} Wie zum Beweis hob er seinen Kopf und schuf einen Feuerwirbel, der sich wie ein Korkenzieher in die Höhe schraubte.

Fredericks Augen wurden größer. \gedanke{Von Tabaluga wusste ich es, aber dass auch du diese Kunstfertigkeit besitzt\abs}

\enquote{Wollt ihr noch bei uns zu Mittag Essen?}, fragte Charlie.

\enquote{Wir haben leider nicht viel Zeit. Noch eine knappe viertel Stunde.}

\enquote{Das schaffen wir. Bei uns gibt es heute Bohneneintopf. Wir haben für\abs zwanzig Lebewesen gekocht.} Dabei sah er die beiden Drachen an.

\enquote{Also gut}, sagte Frederick und die beiden Drachen nickten.

Nach einem nicht sehr üppigen Mahl, das dennoch gut schmeckte und sättigte, verschwanden die beiden Drachen mit Frederick.

\enquote{Wer unter euch stellt sich für eine kleine Mutprobe zur Verfügung? Diese beinhaltet, sich in die Mitte dieses Kreises zu stellen und sich von Tabaluga hier mit Feuer umfließen zu lassen. Wir machen das, wenn die Mittagspause beendet ist.} Diese Frage stellte Hagrid vor der Mittagspause.

Draco Malfoy meldete sich, worauf Hagrid erstaunt war und nur nickte. Draco stand bereits auf der Plattform in der Mitte des Kreises und wartete.

Als spürten sie es, erschienen die beiden Drachen links und rechts von ihm. Auf der einen Seite stand Tabaluga, auf der anderen Seite Nepomuk.

\stimme{Ist er das?}, fragte Nepomuk.

\gedanke{Ja, das scheint der Freiwillige zu sein. Er hat sich wohl während unserer Abwesenheit gemeldet.}

Hagrid entfernte sich und setzte sich auf den Boden. Er war fasziniert, dass nun zwei Drachen zur Verfügung standen. Aber im Gegensatz zu den anderen war er ganz ruhig. Die Stimmung im Publikum konnte man am ehesten als gespannt bezeichnen. Draco hingegen wusste, dass er dem Paten seiner Schwester vertrauen konnte.

Zeitgleich warfen Tabaluga und Nepomuk einen Feuerschwall seitlich an Draco vorbei. Es entstand eine Feuersäule, die in den Himmel empor stieg. Beide Korkenzieher-Feuerwände vereinten sich und schlossen die sichtbare Lücke. Dann änderten sich die Farben und Strähnen in vielen Farben durchzogen das ansonsten in Gelb und Rot gehaltene Feuer. Blau, Grün und Violett färbten sich Streifen aus Feuer und wirbelten um einen Mittelpunkt herum. Es sah fantastisch aus.

Nachdem das Feuer verstummt war, referierte Hagrid noch etwas über das Feuer der Drachen. Er erzählte, dass Drachen ihre Beute niemals roh essen würden. Auch wenn das so aussähe. Wenn sie ihre Beute in den Mund nähmen, würde sie intern gebraten. Schnell und zuverlässig. Drachen seien außerdem sehr kultiviert. Doch kaum einer glaubte dies.

Nach einer weiteren Stunde war der Unterricht beendet und die Drachen konnten von den Schülern von Nahem gesehen werden. Ein paar mutige konnten sich sogar auf die Drachen setzen. Den beiden gefiel das.

\enquote{Wollen wir mal starten?}, fragte Nepomuk, der aufgrund der Nähe zu Hogwarts und den vielen Zauberern vorübergehend ihre Sprache beherrschte.

Erstaunt nickten diese. Dann hob Nepomuk ab und drehte eine Runde über das Schloss. Der Start verursachte einiges an Aufregung. Lehrer, wie Schüler riefen aufgeregt umher, da sie nicht wussten, was passierte. Nur drei Personen standen da und grinsten sich eins.

\enquote{Das war abgemacht Professor, nich?}, fragte Hagrid.

\enquote{Zwischen den Drachen und den Schülern vielleicht. Ich wusste davon nichts.}

\enquote{Ich würde jetzt auch gerne da oben sein}, meinte Draco.

\enquote{Beherrsche dich. Aber wenn du willst, frag Tabaluga. Er nimmt dich bestimmt mit.}

Draco nickte und wartete kurz. Dann trat er an den Drachen heran und fragte ihn, ob er auf ihm reiten könne. Dieser nickt nur und lies Draco aufsitzen. Dann startete auch er.

Bis in den frühen Abend dauerte es, bis alle Schüler die wollten, einmal auf einem Drachen fliegen konnten. Als letzter wollte noch Dumbledore.

Als er den Drachen verlassen hatte, erklärte Hagrid noch, dass Drachen normalerweise nicht so gutmütig seien und jeden auf ihrem Rücken reiten ließen. \enquote{Wir ham hier zwei außergewöhnliche Exemplare}, sagte er noch. Dann löste sich die Schar langsam auf und es ging zum Abendessen.

Zurück blieben die beiden Drachen und Frederick. Die hölzerne Bühne sank magisch ab und verschwand dann. Nun standen die drei auf dem unveränderten Quidditch-Feld auf dem sandigen Boden.

\enquote{Was ist jetzt mit dem Drachenstein? Was weißt du darüber?}, fragte Nepomuk. \enquote{Und keine Angst. Ich werde dir schon nichts tun}, ergänzte er.

Frederick setzte sich auf den sandigen Boden und Tabaluga legte sich ebenfalls auf den Boden. Frederick erschuf ein wärmendes Feuer, da es merklich kühler wurde. Nepomuk blieb mit seinen Vorderpfoten stehen und senkte lediglich sein Hinterteil ab.

Frederick atmete einmal schwer durch. Dann begann er. \enquote{Sagt euch der Name Mantigru etwas?} Beide Drachen nickten erschrocken. \enquote{Er stand in meinen Diensten. Bevor ich der wurde, der ich heute bin.}

\enquote{Der Schrecken aller Drachen}, sagte Nepomuk. \enquote{Ich hatte einmal das Vergnügen mit ihm. Zum Glück ist er besiegt worden.} Dann stutzte er. \enquote{Wieso kanntest du ihn?}, fragte er nach.

\enquote{Er stand in meinen Diensten. Wisst ihr, ich war nicht immer der ausgeglichene und neutrale Zauberer, also der, den ihr kennengelernt habt. Früher war ich phasenweise brutal und dunkel. Nachdem Mantigru gescheitert ist, erschuf ich den Drachenstein. Dann erkannte ich, dass es falsch war, was ich tat. Auf die genaueren Umstände möchte ich nicht eingehen. Ich entschied damals, den Drachenstein so gut es mir möglich war zu sichern und zu verstecken. Er hatte eine Menge Macht in sich. Man konnte sie auch für das Gute verwenden. Leider hat ihn vor kurzem jemand gefunden und euch damit geschadet. Zwar konnte es abgewendet werden\abs} Er brach ab.

\enquote{Du fühlst dich schuldig?}, fragte Nepomuk mit eigenartigem Belag in der Stimme nach. Eine Weile herrschte stille. Dann sagte Nepomuk: \enquote{Du musst dafür bestraft werden.}

\enquote{Glaube mir, das werde ich gerade.} Die beiden Drachen sahen ihn skeptisch an. Vorsichtig zog er seinen Zauberstab, errichtete um die Gruppe ein Bronze-schimmerndes Feld, das am Boden einen kreisrunden Rand aufwies. Dann legte er seinen Zauberstab zwischen sich und dem Feuer ab. Er atmete noch einmal durch und sagte schließlich: \enquote{Prüft es ruhig nach.} Dann schloss er seine Augen und öffnete seinen Geist, sodass die Drachen seine Gedanken, Wünsche und Erfahrungen sehen konnten. Seine Vergangenheit, sowie seine erwartete Zukunft. Er legte ihnen sein Leben offen.

Ruhig, aber tief atmete er, während die beiden Drachen vorsichtig und zurückhaltend seinen Geist sondierten. Sie drangen immer tiefer in seine Vergangenheit ein. Ihre Mienen spiegelten die ganze Palette an Gesichtsausdrücken, die die Drachen hatten, wider. Wut, Zorn, Freude, Liebe, Abneigung und Hass, Freundschaft und Zuneigung. Über eine Stunde, für die Drachen aber eine Strapaze von mehreren Tagen, dauerte die Prozedur.

Dann zogen sich die Drachen wieder zurück. Frederick verschloss wieder seinen Geist und nahm danach vorsichtig seinen Zauberstab, löste die Barriere auf und steckte ihn ein. Währenddessen sortierten die Drachen ihre Gedanken. Dann herrschte mehrere Minuten Ruhe. Jeder hing seinen Gedanken nach. Die Drachen verarbeiteten das, was sie erfahren hatten, und auch Frederick musste sich ausruhen, denn es kostete ihn eine Menge Kraft, gleich zwei Drachen in seinen Gedanken zu haben.

\enquote{Ich hab mich schon gewundert, was für ein rot-schimmerndes Licht vom Quidditch-Feld kommt}, sagte Dumbledore, der neben Hagrid am Eingang stand und zu den dreien um das Feuer sitzenden sah. \enquote{Dürfen wir uns zu euch setzen?}, fragte er.

Die beiden Drachen schauten sie erst an, dann nickte Tabaluga. Dumbledore und Hagrid kamen näher und setzten sich um das Feuer.

\enquote{Was macht ihr hier? Es ist schon ungewöhnlich, dass Menschen und Drachen so friedlich beieinander sitzen.}

\enquote{Wir erzählten uns Geschichten aus unserer Vergangenheit.}

\enquote{Ich vermisse Norbert}, sagte Hagrid.

\enquote{Norberta}, korrigierte Tabaluga.

\enquote{Wie?}

\enquote{Sie ist ein Weibchen.}

\enquote{Geht es ihr gut?}, fragte Hagrid ganz aufgeregt.

\enquote{Ja. Sie ist eine der zahmeren. Sie erinnert sich noch an dich. Erst vor wenigen Tagen hat sie von dir gesprochen. Immerhin ist sie ja auf dich geprägt.}

Jetzt war Hagrid glücklich und zog ein Taschentuch, in das er kräftig schnäuzte und dabei das Feuer kurzfristig höher flackern ließ.

Dann sahen alle wieder eine Weile in die Flammen, bis Tabaluga meinte: \enquote{Wollen wir wieder zurück?} Nepomuk nickte und erhob sich. Tabaluga tat es ihm gleich. \enquote{Bringst du uns?}

Frederick schüttelte lächelnd den Kopf, stand auf und zog seinen Zauberstab. \enquote{Gute Reise}, sagte er und schwang ihn.

Dann waren die beiden Drachen verschwunden.

\enquote{Wo sind sie hin?}, fragte Dumbledore.

\enquote{Zurück in Rumänien.}

\enquote{Wie haben Sie das gemacht?}

\enquote{Fernapparition. Eine neue Methode.}

\enquote{Ach Professor. Ich habe Ihnen hier ein Sandwich vom Abendessen mitgebracht.}

\enquote{Danke Hagrid.} Professor Elber nahm das Sandwich und wickelte es aus. Dann biss er hinein und setzte sich wieder. Die drei übrig gebliebenen Zauberer saßen noch eine Weile, bevor sie das Feuer löschten und sich auf den Weg zum Schloss, oder der Jagdhütte machten.

Gerade als Dumbledore das Schlosstor durchquerte, wurde er von Harry abgefangen.

\enquote{Auf ein Wort, Professor.}

\enquote{Gerne Harry. In meinem Büro?}

Harry nickte und so gingen sie in Dumbledores Büro und saßen kurze Zeit später auf einem gemütlichen Sofa.

\enquote{Was ich dich fragen wollte\abs wie ist das mit den magischen Bildern? Ist das nur ein Zauber? Ein aktuelles Abbild an Wissen und Charakter? Oder kann sich das Bild weiter entwickeln? Wie sieht es aus, wenn die gemalte Person noch lebt? Sind beide verbunden? Wie ist es, wenn die Person schon gestorben ist? Besteht dann eine Verbindung zu einem Leben danach? Davon abgesehen, dass sie keine Fragen dazu beantworten können.}

Dumbledore dachte eine Weile nach. \enquote{Das sind ziemlich viele Fragen, Harry. Ich weiß nicht, ob ich sie dir alle beantworten kann. Aber eines kann ich dir mit Gewissheit sagen, die Bilder der Direktoren wurden alle nach deren Tod erschaffen. Diese sind aber auch etwas Besonderes. Dieser Zauber gelingt nur der amtierenden Schulleiter-Person.} Dann grinste er und Harry verstand, dass er Frauen wie Männer damit meinte. \enquote{Ich habe mir diese Fragen auch schon mal gestellt. Ist es nur ein Abbild? Ein Abbild dessen, was wir glauben, dass diese Person ist oder ausmacht? Oder ist es wirklich ein Zauber, der die Essenz einer Person kopiert. \gst Eines kann ich dir jedoch schon sagen: Zumindest die Direktoren in Hogwarts haben eine Verbindung zu ihrem Leben danach. Sie können zwar nichts darüber erzählen, oder etwas über andere Personen erzählen, aber sie können zumindest Nachrichten und Grüße in eine Richtung weitergeben.}

\enquote{Das heißt, ich könnte über einen der Direktoren hier meine Eltern grüßen lassen.}

\enquote{Falls die Person das auch tatsächlich macht. Wir haben keine Garantie, dass sie es auch tut, oder vielleicht nur versucht. Es kann sogar sein, dass sie deine Eltern gar nicht finden, oder nicht zu ihnen durchdringen. Es könnte mehrere Ebenen der Existenz im Leben danach geben. Oder Fremde, die sich nicht miteinander unterhalten dürfen oder können. Wir wissen nichts darüber.}

\enquote{Was würde passieren, wenn ich von mir ein Bild anfertigen lassen würde? Könnte man diesen Effekt, dass beide miteinander verbunden sind, dann nicht ausprobieren?}

\enquote{Hmm.} Dumbledore dachte nach. \enquote{Was ist, wenn ein Zauber auf dem Bild verhindert, dass er die Wahrheit darüber sagen kann? Wenn er nein sagt, obwohl ihr miteinander verbunden seid?}

Das brachte Harry zum Nachdenken. Er griff in die Schale mit den Lakritz-Schnappern und kaute auf einem herum. \enquote{Und wenn das Bild zerstört wird, während man noch lebt? Geht das Wissen des Bildes dann in einem auf?}, fragte er mehr sich selbst und sah zu Dumbledore. Dieser hob nur seine Schultern. \enquote{Danke Albus. \gst Auch wenn du mir nicht wirklich weiterhelfen konntest.}

\enquote{Tut mir leid. Ich wünschte, ich hätte andere Informationen für dich.}

Dann wurde Harry wieder einmal schwarz vor Augen. Er schaffte es noch, sich auf den Boden zu legen und Albus ein Zeichen zu geben. Dann war er wieder weg. Als er wieder erwachte, saß Albus neben ihm auf dem Boden. Fawkes saß auf seiner Stange und pfiff einmal kurz, um auf sich aufmerksam zu machen. Harry lächelte ihn an und ging, nachdem er aufgestanden war, auf ihn zu. Der Phönix ließ sich von ihm streicheln. Dann flatterte er kurz mit seinen Flügeln und marschierte über Harrys Arm an seine Schulter.

\stimme{Nimm mich heute Nacht mit}, hörte Harry in seinen Gedanken.

Harry sah erstaunt zu Fawkes, der nur einmal kurz seinen Kopf senkte. Harry verstand und strich noch einmal durch sein Gefieder. Dann verließ er das Büro und ging durch das Schloss. \enquote{Nehmen wir den kurzen Weg?}, fragte er das Tier. Fawkes knabberte leicht an seinem Ohr, was Harry als Bestätigung auffasste und in den nächsten Aufzug stieg. Kurz vor dem Gryffindor-Gemeinschaftsraum stieg er aus und trat auf das Porträt zu.

\enquote{Passwort?}, fragte die fette Dame im Bild, doch das, welches ihr Harry nannte, akzeptierte sie nicht.

Es war schon spät und keiner reagierte auf Harrys Klopfzeichen. Er sah Fawkes an und grinste ihn an.

\enquote{Dann eben anders}, sagte er und lag kurze Zeit später in Salazars Privaträumen im Bett. Fawkes hielt sich an Harrys Fußende fest und schlug seine Augen zu.

Doch nach kurzem hörte Harry die bekannte Stimme, die er sonst nur Punkt zwölf mittags hörte. \stimme{Komm zu mir.} Noch immer konnte er sie nicht genau orten. Sie schien jedes Mal von woanders zu kommen. Schien zu wandern. Doch andererseits schlief er um Mitternacht meist, oder war zu beschäftigt.

Fawkes öffnete seine Augen und meinte nur: \stimme{Mondbibliothek.} Dann schloss er seine Augen wieder und schlief erneut ein.

\gedanke{Er hat es also auch gehört. Dann hat es definitiv etwas mit dieser Bibliothek zu tun.}

Beim Frühstück am Tag darauf blätterte er in einem von Salazars Büchern und fand den Zauber, der die Lampen im Schloss mit Leben erfüllte. Er übte den Zauber ein paar mal und fragte bei Salazar nach, ob er es denn auch richtig mache. Dieser war stolz auf ihn, dass einer seiner Enkel sich um das Schloss kümmern wollte und half ihm nach Kräften.

Am nächsten Tag hörte Harry eine Unterhaltung zwischen zwei Slytherin ein Jahr unter ihnen mit.

\enquote{Also jetzt weiß ich, was ich mache, wenn ich mal groß bin. Ich werde Drachenhüter.}

\enquote{Aber Drachen sind doch wild. Du hast doch Hagrid zugehört, oder?}

\enquote{Ja, habe ich. Aber ich will trotzdem Drachenhüter werden.}

\enquote{Die nächste Drachenkolonie ist aber in Rumänien. Willst du jeden Abend oder jedes Wochenende nach Hause zu deiner Frau oder Freundin apparieren? Oder zieht sie mit um?}

\enquote{Ne, Amalia mag das nicht. Das habe ich nicht bedacht.}

\enquote{Amalia?}, fragte nun sein Mitschüler. \enquote{In unserem Haus gibt es keine Amalia.}

Jetzt musste Harry grinsen. Er kannte eine Amalia. Es war die Einzige auf Hogwarts. Er erinnerte sich noch genau an die Auswahlzeremonie. Eine kleine Brünette mit ein paar Sommersprossen wurde nach Hufflepuff gesteckt. Er kannte den jungen Adrian nur vom sehen, aber die beiden anderen hatten kaum Verständnis für ihn. Im Gegensatz zu dem Jungen, der keinesfalls an die Reinheit des Blutes glaubt. Er dachte nach, ob er ihm helfen konnte.

\enquote{Welche Amalia?}, fragte einer der beiden Begleiter nach. \enquote{Es gibt hier keine Amalia!}

\enquote{Wie geht es der kleinen Beauxbaton?}, fragte Harry.

Geistesgegenwärtig sagte Adrian: \enquote{Sehr gut. Leider sehen wir uns nicht so häufig.}

Harry nickte und lief weiter.

\enquote{Woher kennst du sie denn, Potter?}, fragte der größere der beiden um Adrian stehenden.

\enquote{Persönlich nicht, aber ihre Schwester hatte über sie\abs Warum erzähle ich euch das eigentlich?} Dann lief er weiter.

Nach dem Essen fing ihn Adrian ab. \enquote{Danke Potter, Harry. Äh.}

\enquote{Schon gut. Adrian, richtig?}

Dieser nickte. \enquote{Wieso?}, fragte er nach.

\enquote{Ich glaube kaum, dass deine Mitschüler Verständnis dafür hätten, dass ein Slytherin eine Hufflepuff zur Freundin hat. Zumindest ein großer Teil.}

Adrian wurde bleich. \enquote{Woher?}

\enquote{Es gibt nur eine Amalia. Und die ist in Hufflepuff. \gst Gratuliere. Du hast einen\abs Ihr habt einen guten Geschmack.}

\enquote{Danke.} Nach einer Weile fragte er nach. \enquote{Sag mal. Du hast doch, habe ich gehört, letztes Jahr versucht ein paar Mitschüler zu unterrichten, als wir Umbridge hatten.}

\enquote{Ja}, antwortete er vorsichtig.

\enquote{Machst du das wieder? Ich meine, ich würde mich gerne verteidigen können.}

Harry sah ihn eine Weile an. Adrian wurde mittlerweile leicht mulmig. Dann fragte er: \enquote{Würdest du dich denn in dieser Gruppe überhaupt wohlfühlen? Es waren damals ja keine Slytherin dabei. Du wärst dann quasi der Einzige.}

Adrian dachte eine Weile nach. \enquote{Na ja. Ich schätze, anfangs wird es schwer werden. Es würde sicherlich dauern, bis mich die anderen akzeptieren werden.}

\enquote{Hmm. Nehmen wir mal an, dass es diese Gruppe, immer noch gibt. Dann müsste ich zuerst mit denen reden. Ich melde mich in drei Tagen. Ist das in Ordnung?}

Adrian dachte kurz nach. \enquote{In Ordnung. Bis dann. \gst Warte mal. Was ist, wenn noch mehr aus meinem Haus mitmachen möchten?}

\enquote{Du meinst, ein Slytherin würde sich von einem Gryffindor etwas sagen lassen? Dich mal ausgenommen!} Dann ging Harry und lies einen nachdenklichen Adrian stehen.

\trenn

\enquote{Du siehst so bedrückt aus}, meinte Marietta, als sie wieder im Raum der Wünsche übten.

\enquote{Ich denke nach}, antwortete Harry.

\enquote{Sagst du auch, worüber?}

\enquote{Über einen möglichen Neuzugang zu unserer Gruppe.}

\enquote{Vertrauenswürdig?}

\enquote{Ich denke schon. Er ist mir jedenfalls nicht negativ aufgefallen. Ich bin mir nur unsicher, ob er\abs}

\enquote{In die Gruppe passt?}, fragte Ron, der gerade eine Pause machte und sich neben Harry setzte.

Kurz darauf quetschte sich Ginny zwischen Marietta und Harry. Sie konnte es nicht ertragen, dass sich jemand an \accentuate{ihren} Harry heranzumachen versuchte.

\enquote{Wer ist es denn?}

\enquote{Adrian Montague.}

\enquote{Kenne ich nicht}, sagte Ginny. \enquote{Aus unserem Haus ist der nicht.}

\enquote{Aus meinem auch nicht}, meinte Marietta.

\enquote{Vielleicht Hufflepuff?}, überlegte Ron.

\enquote{Nein}, sagte Hannah Abbott, die in der Nähe stand und mit halbem Ohr mitgehört hatte.

\enquote{Moment mal}, meinte Ron. \enquote{Er ist weder in Gryffindor, noch in Ravenclaw oder Hufflepuff. Wie geht das denn?}

Marietta und Ginny mussten lachen.

\enquote{Man, Ron}, meinte Ginny. \enquote{Als ob Hogwarts nur drei Häuser hätte. Wenn er in keinem der drei Häuser ist, dann kommt er wahrscheinlich aus Slytherin.}

Marietta und Ginny lachten weiter. Dann plötzlich verstummten sie und Harry musste sich ein Grinsen verkneifen.

Ginny sah Harry an. \enquote{Entweder willst du uns alle verarschen, oder du überlegst dir wirklich, dass wir einen\abs Sag, dass das nicht wahr ist.} Jetzt verstummte die gesamte Gruppe und es war plötzlich still. \enquote{Sag mir, dass du nicht überlegst einen Slytherin bei uns aufzunehmen.}

\enquote{Doch, genau das}, antwortete Harry. \enquote{Ich bin mir nur nicht sicher, wie er aufgenommen würde. Ich halte ihn für einen guten jungen Zauberer.}

\enquote{Spinnst du jetzt völlig? Einen Slytherin?}, antwortete Ron erregt.

\enquote{Nein, ich spinne nicht}, gab Harry laut zurück. \enquote{Aber du solltest dir mal überlegen, warum es immer noch diesen Hass zwischen den Häusern gibt, wenn du einer derjenigen bist, die diesen Hass ständig anstacheln.} Dann sah er auf und in die erschrockenen Gesichter seiner Mitschüler. \enquote{Wenn ihr ihn nicht wollt und diesen Hass der Häuser weiter aufrechterhalten wollt, von mir aus. Ich werde ihn vermutlich unterrichten. Auch wenn es dann Einzelunterricht sein wird. Ich habe von dieser Hass-Sache endgültig die Schnauze voll.}

Jetzt war Harry wieder wohler. Seit zwei Tagen hatte sich dieser Frust wegen der Ablehnung eines Slytherin angestaut. Nicht zu vergessen, der monatelange Kampf gegen seine Windmühlen. Manchmal kam er sich wie Don Quijote vor.

\enquote{Wer ist es denn?}, fragte Susan Bones nach.

\enquote{Adrian Montague}, sagte Harry erneut.

Susan überlegte eine Weile. \enquote{Von mir aus}, sagte sie und begann sich wieder ihren Übungen zu widmen.

Jetzt entbrannte eine heftige Diskussion darüber, ob man Slytherin aufnehmen durfte oder nicht. Nach zehn Minuten stand Harry auf und verließ den Raum. Er trottete niedergeschlagen wegen des Streits und erleichtert darüber, dass er seinem Frust Luft gemacht hatte, durch das Schloss.

Nach einer Weile lief Professor Flitwick neben ihm her. \enquote{Probleme?}, fragte er.

Harry nickte. \enquote{Ja. Wir haben gerade einen Streit.}

\enquote{Wer ist wir?}

\enquote{Die DA. Ich würde gerne jemanden neues hinzunehmen, aber ein großer Teil der DA nicht. Vermutlich.}

\enquote{Sagen Sie mir, wer?}

\enquote{Adrian Montague.} Dann erst war Harry wieder klar bei Verstand. Er sah sich um, entdeckte aber niemanden.

\enquote{Hier unten}, gluckste der kleine Zauberer.

Harry sah nach unten und lief rot an. \enquote{Entschuldigen Sie Professor Flitwick. Das ist normalerweise nicht meine Art.} Er lief ein paar Schritte weiter zu einer Bank und setzte sich.

Professor Flitwick kam zu ihm und setzte sich neben ihn. \enquote{Wissen Sie, es gibt Zeiten, da muss man sich dem Druck der Gemeinschaft beugen.} Keiner der beiden bekam mit, dass sich Teile der DA näherten, aber außer Sichtweite blieben. \enquote{Und es gibt Zeiten, da muss man gegen den Strom schwimmen. Hogwarts ist so ein Ort. Wir haben immer wieder bewiesen, dass man manchmal mutig genug sein muss und es der Gemeinschaft beweisen muss, dass sie falsch lag. Wenn die Aktion nach hinten losgeht, hat man zwar die Häme der anderen, weiß aber, dass man falsch lag. Und die anderen wissen zumindest, dass sie definitiv richtig lagen.} Dann überlegte er eine Weile. \enquote{Werden Sie ihn trotzdem unterrichten?}

\enquote{Ich denke schon.}

\enquote{Dann werde ich Ihnen helfen!}

\enquote{Wie?}

\enquote{Ich werde Ihr Co-Lehrer. Wir können gemeinsam etwas vorführen. Eins-zu-Eins Duelle, Zwei-zu-Eins Duelle und wenn Pomona mitmacht, dann können wir auch Drei-zu-Eins Duelle machen.}

Harry gefiel die Idee. \enquote{Ich warte noch die Entscheidung meiner\abs der DA ab. Aber ich komme auf Ihr Angebot zurück. Gilt das nur im Falle des Einzelunterrichtes?}

\enquote{Wenn Sie mir anbieten, die DA mal zu besuchen, nehme ich gerne an.} Harry dachte eine Weile nach. \enquote{Kommen Sie mit? Ich gebe eine Tasse Tee aus.}

\enquote{Gerne Professor.} Die beiden standen auf und liefen den Gang entlang. \enquote{In Ihrem Büro?}

\enquote{Nein, wir gehen ins Lehrerzimmer.}

Die Gruppe der DA kam um die Ecke und sahen den beiden zu, wie sie schwatzend Richtung Lehrerzimmer im Inneren des Schlosses verschwanden.

\enquote{Meint ihr, er macht es wahr?}, wollte Zacharias Smith wissen.

\enquote{Schwer zu sagen. Er könnte es auch nur so gesagt haben, um uns umzustimmen. Oder, weil er richtig sauer war}, meinte Cho.

\enquote{Was meint ihr dazu?}, fragte Katie Bell an Ron und Hermine gewandt. \enquote{Ihr kennt ihn länger und besser als wir.}

\enquote{Ich denke, wenn er davon überzeugt ist, dass er ungefährlich ist, dann wird er es auch tun. Da ist er dickköpfig.}

\enquote{Auch, wenn wir dagegen sind?}

\enquote{Dann wird er ihn, sofern der junge Mann will, mitnehmen, oder, wie du gehört hast, ihn privat unterrichten.}

\enquote{Das ist doch bestimmt eine Falle, damit Du-weißt-schon-wer erfährt, was Harry alles kann}, meinte Zacharias. \enquote{Immerhin ist er ein Slytherin.}

\enquote{Ich glaube, das hat Harry gemeint}, warf Katie ein.

\enquote{Was hat er gemeint?}, fragte Ron nach.

\enquote{Dass wir ihn ablehnen, nur weil er aus dem falschen Haus stammt.}

\enquote{Spinnst du?}

\enquote{Genau das ist es! Du bist auch einer von denen, die ihren Kindern den Hass auf Slytherins übertragen. Und die dann auf deren Kindern. Und das geht über Generationen so weiter. Und nicht nur auf unserer Seite.}

\enquote{Bist du jetzt komplett durchgedreht?}

\enquote{Nein, aber du, Ron. Überleg mal, wie sich die Slytherins wohl fühlen.}

\enquote{Das interessiert mich nicht.}

\enquote{Genau darum geht es. Sie werden einfach in eine Ecke gedrängt, obwohl sie nichts dafür können. Der Hut verteilt sie. Und jetzt überlege mal, wie sich Harry fühlen würde, wenn ihn der Hut nach Slytherin gesteckt hätte. Denn das hat er ihm ja schließlich angeboten. Oder hast du da auch nicht aufgepasst?} Ron schwieg, wie der Rest der Gruppe. \enquote{Dann hätten wir ihn alle gehasst und für Vold\aabs Du-weißt-schon’s Liebling gehalten und ihn genau so abgelehnt. Er hätte ihn trotzdem zu bekämpfen versucht, wäre aber vermutlich schon an seinen Hauskameraden gescheitert. Harry hat recht! Wird Zeit, dass sich da was ändert! Ich habe jetzt Hunger.} Dann ging sie. Sie drehte sich noch einmal um und sagte: \enquote{Falls du weiter reden oder streiten willst, ich bin in der Großen Halle.}

Es dauert eine Weile, bis sich die Schüler gefasst hatten. Nach und nach löste sich die Gruppe auf und ging.

\trenn

Harry und Professor Flitwick kamen am Lehrerzimmer an und der kleine Koboldmischling öffnete die Tür. Im Zimmer war nur noch Professor Sinistra, die am Fenster stand und hinaussah. Als die beiden das Zimmer betraten, drehte sie sich um. \enquote{Hallo Filius.}

\enquote{Hallo Aurora.}

\enquote{Hat Mister Potter was ausgefressen?}

\enquote{Nein.}

\enquote{Nicht? Na dann}, meinte sie und drehte sich wieder um.

Flitwick bot Harry einen Sitzplatz an. Dieser setzte sich. Kurz darauf hatten beide ihren Tee und etwas Gebäck vor sich stehen.

Harry sah eine Weile stumm auf seine Tasse. Dann fing er an zu erzählen. \enquote{Ich mag nicht mehr! Dieser Hass auf das andere Haus. Ich meine, ich habe nichts gegen die Hausmeisterschaft und die Hauspunkte. Ein bisschen Ansporn sollte schon sein. Oder beim Quidditch den Pokal. Aber dass man das Haus Slytherin so ausgrenzt, wegen etwas, was so viele Jahre vergangen ist\abs Ich meine, ich verstehe mich mit Draco mittlerweile so, dass wir uns nicht mehr richtig anfeinden. Ich mag ihn zwar immer noch nicht besonders, oder würde so weit gehen und sagen, dass ich ihn respektiere, aber\abs Ich denke, ich bin ihm gegenüber einfach neutral geworden.}

\enquote{Er aber anscheinend nicht mit ihnen. Oder es ist ihm entgangen, dass\abs} Er brach ab, da ihn Harry eigenartig ansah.

\enquote{Das bleibt jetzt aber unter uns, verstehen wir uns?} Professor Flitwick musste schlucken, da Harry in dem Moment eine Autorität ausstrahlte, wie er sie sonst nur von Dumbledore kannte. Daher nickte er nur einfach. Professor Sinistra, die die beiden ansah, nickte auf Harrys Blick hin auch nur. \enquote{Wir streiten uns nur noch, wenn wir uns öffentlich treffen. Falls wir uns alleine, oder mit den entsprechenden Leuten um uns herum begegnen, dann belassen wir es dabei mit dem Kopf zu nicken und weiterzugehen. Wir nutzen nicht mehr jede Gelegenheit, um uns die Hölle heiß zu machen.}

Dann wurde es still im Zimmer. Professor Sinistra setzte sich zu ihnen und nach einer Weile kam Dumbledore herein. \enquote{Oh, Harry, hast du was ausgef\aabs}

Harrys eigenartiger, beleidigter Blick ließ ihn verstummen. Dumbledore wusste selbst nicht, was gerade mit ihm passierte. Er spürte so etwas wie Peinlichkeit. Harry hingegen war über seine Reaktion mehr als überrascht. Umso überraschter war er über die Reaktion seines Schulleiters, der sich abwandte, sich eine Tasse voll Tee aus einer Kanne goss und sich dann eine Weile an ein Fenster stellte und hinaussah, bevor er sich zu ihnen setzte. Der Direktor wusste in diesem Moment selbst nicht, warum er Harry nicht darauf hinwies, dass er auf seinem Stuhl saß und er sich doch bitte einen anderen nehmen solle.

Als dann nach einer Weile noch Professor McGonagall den Raum betrat, stutze und sich ebenfalls eine Tasse Tee holte, stand Harry auf, bedankte sich bei Professor Flitwick für den Tee, verabschiedete sich von den Lehrern und verließ das Zimmer.

\enquote{Ich bin mehr als überrascht, Albus, dass Sie Mister Potter nicht von Ihrem Stuhl verscheucht haben}, warf Professor Flitwick in den Raum.

\enquote{Ich auch Filius, ich auch}, sagte Dumbledore gedankenversunken und starrte immer noch die Tür an, durch die eben Harry gegangen war. Dann sah er Filius wieder an. \enquote{In dem Moment fand ich es einfach unpassend. Ich hatte das Gefühl, ich sitze als Schüler vor meinen Direktor.} Dann lächelte er leicht. \enquote{Es war irgendwie ein tolles Gefühl.}

\enquote{Ihnen gefällt das? Wenn man vor dem Schulleiter als Schüler sitzt?}

\enquote{Das meinte ich nicht. Es war schön, das mal wieder erfahren zu haben. So wird man als Lehrer wieder daran erinnert, welche Verantwortung man gegenüber den Schülern hat. Es war beängstigend, solange es dauerte, hat mich aber wieder daran erinnert, dass sich unsere Schüler manchmal ebenso fühlen. Wir sollten diese Gefühle nicht noch verstärken.}

\enquote{Hat Sie Potter also in die Knie gezwungen}, kam sarkastisch aus einer Ecke des Lehrerzimmers. Severus Snape, den bisher keiner beachtet hatte, kam zu ihnen und setzte sich.

\enquote{Wie Sie sich gefühlt haben, brauche ich Sie ja wohl nicht zu fragen}, kam leicht bissig von Aurora Sinistra.

Snape schwieg, doch in seinem innersten fühlte er genau so.

\enquote{Mister Potter scheint einige interessante Fähigkeiten zu entwickeln, nicht wahr Albus?}, kam jetzt von McGonagall.

\enquote{In der Tat, das tut er.}

\enquote{Anscheinend wird er gut ausgebildet.}

Sinistra verschluckte sich fast an ihrem Tee.

Flitwick sah sie daraufhin lange an. Dann meinte er: \enquote{Hat man Ihnen etwa gesagt, dass Sie nicht darüber reden sollen?}

Sie nickte nur.

\enquote{Mir aber auch}, meinte McGonagall und Dumbledore nickte nur.

\enquote{Dann hat man jedem von Ihnen und wahrscheinlich dem gesamten Lehrerkollegium dasselbe gesagt}, gab nur Snape zum Besten. \enquote{Ja, er ist halt doch ein Schlitzohr.}

\enquote{Wer?}, fragte Sinistra nach.

\enquote{Wenn Sie das noch immer nicht wissen, dann werde ich es Ihnen garantiert nicht sagen.} Dann stand Snape auf und verließ das Lehrerzimmer Richtung Kerker.

Sinistra war eine der wenigen, vielleicht auch die einzige, die Severus manchmal verstand. Daher fühlte sie sich auch nicht beleidigt. Sie dachte nach. Der Einzige, von dem sie wusste, dass er Harry unterrichtete, war ihr Kollege Frederick. Dann wusste sie, was Severus meinte.

Harry war unterdessen an seinem Ziel angekommen. Auf seinem Weg hatte er noch eine der Lampen ersetzt, die einer Auffüllung bedurften. Er setzte sich auf das grüne gemütliche Sofa. Kurz darauf rutschte er darauf herum und legte sich hin. Wieder änderte er seine Position. Dann stand er auf und lief zu einem Sessel, in den er sich setzte. Er drehte sich wieder und lag zwischen den Armlehnen quer auf dem Sessel. Er legte seinen Kopf gegen die Rücklehne und schloss die Augen.

\enquote{Er ist irgendwie aufgewühlt, Salazar}, sagte seine Frau Agatha leise zu ihm. Harry konnte sie nicht verstehen.

\enquote{Er hat sich mit seiner Gruppe gestritten.}

\enquote{Die DA, von der du mir erzählt hast?}

\enquote{Ja.}

\enquote{Weswegen?}

\enquote{Er will einen Schüler aus meinem Haus in seiner Gruppe unterrichten. Diese hat wohl was dagegen.}

\enquote{Was wird er tun?}

\enquote{Vermutlich wird er ihn trotzdem unterrichten.}

\enquote{Also so ein Dickkopf wie du?}

\enquote{Sieht so aus. Lass uns auch schlafen.}

Am nächsten Morgen erwachte Harry gerädert im Sessel und schaute verschlafen auf das Bild. Er hatte von einem rosa Regenschirm geträumt. Wieder einmal. Er rieb sich die Hände vorm Gesicht und sah hoch zu Salazar und seiner Frau Agatha. \enquote{Guten Morgen ihr zwei.}

Salazar und Agatha grüßten zurück.

\enquote{Du siehst müde aus}, sagte Agatha und sah ihn mitleidig an.

\enquote{Ja, das ist gestern nicht so gut gelaufen mit meiner Gruppe. Und außerdem hatte ich wieder diesen Traum. Ich weiß einfach nicht, wie ich ihm helfen soll.}

\enquote{Deinem Traum?}

\enquote{Nein, Hagrid. Einem meiner Freunde.}

\enquote{Was hat er denn für ein Problem?}

\enquote{Er darf nicht zaubern, weil man ihn vor fünfzig Jahren verdächtigt hat ein gefährliches Monster in der Schule frei zu setzen. Daraufhin hat man ihn hinausgeworfen. Er ist aber noch hier, als Wildhüter und nun auch als Lehrer.}

\enquote{Und wie möchtest du ihm helfen?}

\enquote{Ich möchte seine Unschuld beweisen. Das Problem ist, ich hatte einen Beweis. Das ist mir erst dieses Jahr klar geworden. Gedanken in einem Tagebuch.}

\enquote{Wie können Gedanken in einem Tagebuch stecken? Du meinst geschrieben?}

\enquote{Nein. Das Tagebuch war ein Horkrux. Ich habe ihn in meinem zweiten Jahr vernichtet. Davon habe ich euch erzählt.} Beide nickten. \enquote{Das war der einzige Beweis. Ich habe sonst keine andere Chance. Außer es gibt einen Weg in der Zeit zurückzureisen und die Gedanken aus dem Tagebuch zu extrahieren, damit man sie in einem Denkarium ansehen kann. Natürlich, ohne dass es auffällt. Besonders dem Seelenteil. Dann muss man wieder zurück in die Gegenwart. Aber außer einem Zeitumkehrer kenne ich nichts. Und die hat das Ministerium allesamt unter Verschluss.} Dann sah Harry zu den beiden hoch. \enquote{Außer ihr wisst etwas.}

Salazar setzte ein nachdenkliches Gesicht auf. Er schritt aus dem Bild und kehrte kurz danach wieder zurück, durchlief es und verschwand auf der anderen Seite, von der er nach einigen Sekunden wieder kam. Das passierte mehrere male hintereinander. Er lief praktisch im Bild umher und darüber hinaus. Dann blieb er stehen und sah Harry an. \enquote{Der Zeitumkehrer wäre nicht das Problem. Ich habe einen in meinen Beständen. Irgendwo. Das Problem, oder eher die Probleme, sind vielmehr die lange Zeit, die überbrückt werden muss. Hin und zurück, sowie die Maßnahme der Extraktion von Gedanken bei einem sich wehrenden Horkrux. Dann musst du ihn auch noch mit einem Gedächtniszauber belegen.}

Harry stand nun ebenfalls auf und lief durch den Raum. Ein Elf brachte Frühstück und verschwand wieder. Marcel kam zu ihm und wollte auf seinen Arm. Er hatte schon länger keine menschliche Wärme mehr gespürt. Harry nahm ihn auf und trug ihn herum.

\enquote{Du hast doch einen direkteren Zugriff auf diese Informationen}, sagte Salazar.

\enquote{Welche?}

\enquote{Voldemort selbst. Du hast doch eine direkte Verbindung zu ihm. Durchforste sein Gedächtnis und kopiere dir die Erinnerungen. Man wird feststellen können, dass sie echt sind. Natürlich musst du vorsichtig sein. Voldemort ist extrem gut. Mache einen Schritt nach dem anderen. Wenn du bereit bist und das Risiko eingehen möchtest, werde ich dir helfen.}

Harry war dankbar und verabschiedete sich nach dem Frühstück von beiden. Darüber musste er erst einmal nachdenken.

\trenn

\enquote{Haben Sie Zeit?}, wurde Harry von Elber gefragt, der gerade mit Ron und Hermine vom Essen kam.

\enquote{Ja}, antwortete er. Und an Ron und Hermine gewandt fragte er: \enquote{Ihr wisst euch sicherlich die Zeit alleine zu vertreiben.} Dabei lächelte er beide süffisant an. Harry folgte seinem Lehrer hinaus auf das freie Feld.

Hinter einer Bäume- und Büsche-Kombination versteckt und vom Schloss aus kaum einsehbar, stellten sich beide gegenüber auf und zogen ihre Zauberstäbe. Beide nickten einander zu und der Tanz der Kontrahenten begann. Zauber um Zauber flog umher und beleuchtete die aufkommende Abenddämmerung.

Während Harry und Professor Elber ihr Duell vollzogen, genossen Ron und Hermine die Zweisamkeit. Nur spärlich bekleidet lagen sie eng aneinander gekuschelt in ihrem Raum. Immer wieder fuhren sie mit ihren Händen am Körper des anderen auf und ab. Mit unzähligen Küssen bedeckten sie sich, bis sie irgendwann einfach so entspannt waren, dass sie einnickten.

Etwa vierzig Minuten später näherte sich eine Gruppe aus Lehrern und Schülern den eigenartigen Lichtern, die sich auf dem Gelände zeigten. Harry warf gerade einen Zauber auf seinen Lehrer, als die Gruppe in ihr Sichtfeld schwenkte. Elber knickte an seinen Knien nach hinten ab, blieb aber stehen. Der Zauber verfehlte ihn knapp über seinem Bauch. Er sah nach oben und bemerkte, dass der Zauber direkt auf die Gruppe treffen würde. Er warf einen Zauber mit solch einer Geschwindigkeit hinterher, dass dieser Harrys Zauber überholte und einen Schild erschuf, um die Zaungäste zu schützen.

Gerade als er wieder Richtung Harry sah und sich aufrichten wollte, um einen erneuten Angriff zu starten, erwischte ihn der nächste Zauber Harrys frontal, schleuderte ihn durch die Luft und warf ihn gegen das Kraftfeld. Professor Elber verlor seinen Stab und fiel bewusstlos zu Boden. Professor McGonagall wollte schon zu ihm treten, doch das Kraftfeld hinderte sie. Harry rannte auch auf ihn zu, kam aber an ihn heran, da er innerhalb des Feldes stand.

Während Professor McGonagall versuchte das Feld zu neutralisieren, untersuchte Harry seinen Lehrer, wie er es von Madame Pomfrey gelernt hatte. \gedanke{Nur Bewusstlos und Kopfschmerzen}, dachte er sich.

\enquote{Würden Sie bitte das Feld aufheben, Mister Potter?}, verlangte Professor McGonagall, die es bis dahin nicht geschafft hatte das Feld aufzuheben.

\enquote{Wie stellen Sie sich das vor, Professor?} Er nahm seinen Stab hoch. \enquote{Soll ich ihn einfach\abs}, er bewegte ihn auf das Feld zu, \enquote{\aabs auf das Feld halten\abs} Er berührte mit seiner Spitze das Feld, \enquote{\aabs und es löst sich\abs}, das Feld verschwand, \enquote{\aabs auf.} Er stutzte. \enquote{Wie ging das jetzt?}, fragte er mehr sich selbst.

Professor McGonagall besah sich ihren Kollegen und sah dann vorwurfsvoll auf Harry. \enquote{Was haben Sie sich dabei gedacht, Mister Potter.}

\enquote{Was ich mir\abs? Na hören Sie mal, wir haben hier trainiert, bis Sie uns gestört haben. Professor Elber hat das Feld nur zu Ihrem Schutz erschaffen. Leider habe ich meinen zweiten Zauber nicht schnell genug zurücknehmen können.}

\enquote{Bin ich jetzt etwa Schuld?}, echauffierte sich Professor McGonagall.

\enquote{Hilft es denn?}, fragte Harry genervt nach. Dann beschwor er eine Trage herauf, ließ seinen Lehrer darauf schweben und machte sich auf den Weg zurück zum Schloss.

Im Krankenflügel angekommen legte er ihn auf ein freies Bett.

Madame Pomfrey kam heran und untersuchte ihn. \enquote{Ihre Meinung, Mister Potter.}

\enquote{Bewusstlos und Kopfschmerzen}, antwortete er knapp.

\enquote{Wieso so gereizt?}, fragte sie auf dem Rückweg zur Apotheke, wo sie einige Tränke holte.

Als sie wieder da war, antwortete Harry: \enquote{Wir hatten ein kleines Duell. Zur Übung. Ich warf gerade einen Zauber auf ihn und setzte noch einen nach, als Zaungäste auftauchten. Professor Elber bemerkte sie rechtzeitig und warf einen Zauber der meinen überholte und unsere Besucher schützte. Leider konnte ich meinen zweiten Zauber nicht mehr aufhalten und Professor Elber war nicht schnell genug. Er erwischte ihn Frontal und schleuderte ihn gegen das Feld. Dann hat mir Professor McGonagall auch noch Vorwürfe gemacht. Besser gesagt, machen wollen. Ich bin dann etwas ungehalten gewesen und habe wohl leicht über reagiert}

\enquote{Leicht überreagiert?}, fragte Professor McGonagall, die gerade hereinkam.

\enquote{Das wäre nicht passiert, wenn Sie mich nicht gleich so angepflaumt hätten}, antwortete er, ohne zu ihr zu sehen. Dann sagte er zu Madame Pomfrey: \enquote{Sehen Sie was ich meine?}

Diese nickte nur und meinte: \enquote{Warten Sie kurz. Sie bekommen von mir etwas. Das lindert Ihren Reizzustand. Außerdem ist es beruhigend und schlaffördernd.} Sie kam mit zwei Kelchen und gab einen davon Harry, den anderen Professor McGonagall.

\enquote{Wieso gibst du mir einen?}

\enquote{Runter damit, Minerva}, befahl sie. \enquote{Du hast den genauso verdient.}

Widerstandslos tranken beide die Flüssigkeit und legten sich danach in je ein Bett. Zufrieden kümmerte sich Madame Pomfrey um ihren Patienten und ging dann. Sie ließ ihre Gäste alleine und machte das Licht aus.

\trenn

\enquote{Dürfen Sie sich denn schon wieder duellieren?}, fragte Harry, als er mit Professor Elber erneut über die Ländereien lief.

Dieser schüttelte seinen Kopf. \enquote{Nein, aber wir werden heute nichts dergleichen machen. Warten Sie es ab.}

Sie liefen Richtung Hagrid, dann an seiner Hütte vorbei und in Richtung einer kleinen Baumgruppe. Davor waren etwa zwanzig Baumstümpfe zu sehen. Hagrid wartete bereits.

\enquote{Schön, dasser da seid}, sagte er.

\enquote{Gerne, Hagrid}, dann zu Harry gewandt: \enquote{Diese Baumstümpfe müssen entfernt werden. Die Hälfte mit Zauberstab, die andere Hälfte ohne. Und jeder Stumpf muss auf andere Art und Weise entfernt werden.}

Harrys Kiefer klappte auf. \enquote{Wie?}

\enquote{Egal wie. Wichtig ist nur, immer auf andere Art.}

\enquote{Sie meinen, ich soll die Baumstümpfe entfernen?} Professor Elber nickte. \enquote{Was bringt mir das?}

\enquote{Das liegt an Ihnen, es herauszufinden. Anderenfalls sage ich es Ihnen in etwa drei Wochen.}

Harry blieb erst einmal stumm stehen und sah abwechselnd zwischen Hagrid, seinem Lehrer und den Baumstümpfen hin und her. Dann schließlich begann er. Er besah sich einen der Baumstümpfe und versuchte es mit einem \spruch{Wingardium Leviosa}. Er strengte sich an, bis er eine Reaktion des Stumpfes spürte und sah. Er spannte seine Arm- und Handmuskeln immer mehr an, sodass sie zu schmerzen begannen. Gerade als der Baumstumpf aus der Erde flog, bekam er einen Krampf in seiner Hand und ließ seinen Zauberstab fallen, um seine Hand zu halten.

\enquote{Ah!}, schrie er. Seine linke Hand umklammerte sein rechtes Handgelenk und er hoffte, so den Schmerz zu vermindern. Als er die Hand nach einigen Sekunden wieder losließ, bemerkte er eine kleine Schwellung.

Professor Elber kam auf ihn zu und fuhr mit seinem Zauberstab über sein Handgelenk in immer kürzer werdenden pendelnden Bewegungen, bis er ganz still stand und ihn nach oben wegzog.

\enquote{Danke!}, keuchte er. \enquote{Woher kennen Sie so etwas?}

\enquote{Unsere Medi-Hexe.}

\enquote{Wie?}

\enquote{Ich war bei ihr und habe ihr geschildert, was wir heute vorhaben, dann hat sie mir gezeigt, mit welchen zu erwartenden Verletzungen ich zu rechnen hätte. Die passenden Zauber hat sie mir gleich verraten. Sie sollten trotzdem am Ende der Stunde zu ihr gehen. Machen Sie weiter.}

\enquote{Gleich?}

\enquote{Es können auch ein paar Minuten Pause dazwischen sein.}

\enquote{Woll’n se was z’ trink’n?}

\enquote{Gerne Hagrid.}

\enquote{Dann komm’n se.}

Damit verschwanden beide in Hagrids Hütte und ließen Harry alleine. Als dieser die Hälfte der Stümpfe entfernt hatte, versuchte er es ohne Zauberstab. Doch er hatte keinen großen Erfolg. Also behalf er sich mit dem überwiegenden Teil der restlichen mit seinem Zauberstab. Als er fertig war, klopfte er an die Tür und wurde kurz darauf von Hagrid reingebeten. Nach einer Tasse Tee und etwas Geplauder verließen die beiden Hagrid und gingen zurück zum Schloss.

\enquote{Alles gut verlaufen?}, wurde Harry gefragt.

\enquote{Ja}, gab er knapp zurück.
