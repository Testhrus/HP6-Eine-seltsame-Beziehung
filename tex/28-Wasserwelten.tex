\chapter{Wasserwelten}


Harry war in einem der vielen Gänge und knutschte gerade mit Cho. Er genoss es, wie sich ihre Lippen berührten, als er ihren Mund auf seinem spürte. Er dachte an nichts, als er plötzlich eine weitere Präsenz spürte. Erschrocken löste er sich von Cho und starrte in die Richtung aus der er die Präsenz vermutete.

Er sah Pansy Parkinson. Cho rannte vor Schreck davon. Pansy grinste und wollte schon davonrennen, um es den anderen zu erzählen, vermutete Harry. Er rannte ihr kurz hinterher und umgriff ihr Handgelenk. Abrupt blieb sie stehen. Überrascht drehte sie sich um. Sie machte keine Anstalten mehr, fliehen zu wollen. Harry war sich nicht sicher, ob es nur ein Trick war, damit er in seiner Vorsicht nachließ. Er lockerte seinen Griff etwas, hielt ihr Armgelenk aber immer noch fest, sodass sie sich nicht entwinden konnte.

Pansy stutzte. Ihr Blick veränderte sich leicht. Sie schien weder sauer noch schelmisch zu sein. Sie schaute ihn mit einem erstaunten Blick an. Harry spürte, wie sein Herzschlag schneller wurde. Nun kam sie auf ihn zu und stand ganz dicht vor ihm. Sie hob ihre freie Hand und nahm seinen Kopf in ihre Hand. Langsam zog sie sich an ihn ran und begann ihn zu küssen. Harry war geschockt und wollte schon zurückweichen, aber etwas hinderte ihn. Er löste seinen Griff von ihrem Handgelenk und legt beide Hände um ihre Hüfte, um sie noch näher an sich ran zu ziehen. Sie nahm jetzt ihre andere Hand und umschloss mit ihren Händen seinen Kopf und vergrub ihre Finger in seinem Haar. Harry war außer sich. Noch nie hatte er so weiche und zarte Lippen gespürt.

Sie versanken in einem langen Kuss. Harry genoss es. Er hatte nicht gedacht, dass Pansy derart gut küssen könnte. Und wenn es auch nur einen Moment andauern sollte; ihm war es momentan egal. Er wollte sie nur weiter küssen. Sie brach den Kuss, nur um kurz darauf ihren Mund leicht zu öffnen und ihn wieder zu küssen. Er folgte ihrem Tempo und öffnete seinen ebenfalls. Sie begann sachte mit ihrer Zunge seine Lippen zu umspielen und fuhr seine Oberlippe entlang, nur um kurz darauf an seinen Zähnen entlangzugleiten. Er umspielte mit seiner Zunge ihre, als sie ein leises Stöhnen von sich gab.

Sie brach abermals den Kuss, und drehte ihren Kopf zur Seite. Harry wollte sie schon fragen, ob alles in Ordnung sei, als sie nur sagte: \enquote{Lass uns da hineingehen, Harry.} Harry musste schlucken. Sie sah ihn wieder an und zog ihn zu einer kleinen Tür, die in die Mauer eingelassen war.

Der Raum hatte die Größe einer Besenkammer, wie Harry im Halbdunkeln erkennen konnte. Als Pansy die Tür schloss, war es dunkel. Harry konnte nichts sehen. Im ersten Moment dachte er nur: \gedanke{Verdammt, sie hat mich eingeschlossen, sie spielte nur mit mir.} Er beruhigte sich schnell wieder als er eine weitere Präsenz spürte. Er hörte ein leises Rascheln, gefolgt von einem: \zauber{Lumos.} Sie hatte die Spitze ihres Zauberstabes zum Leuchten gebracht.

Harry schluckte. Er holte seinen Zauberstab ebenfalls heraus und verschloss damit sorgsam die Tür, drehte sich kurz von Pansy weg und murmelte zwei ihr unverständliche Zaubersprüche. Er legte einen Polsterzauber auf einen kleinen Bereich der Kammer und einen Wärmezauber, damit die kalten Bodenfliesen sie nicht erzittern ließen. Dann zog er seine Schulrobe aus und legte sie auf den Boden. Dann drehte er sich wieder zu Pansy um und schaute ihr tief in die Augen. Er trat wieder vor sie, doch sie war schneller und zog ihn fordernd zu sich heran.

\begin{abAchtzehn}

Wieder spürte er ihre zarten Lippen. Harry spürte  wie sie ihre Brüste an ihn drückte. Sie löste den Kuss und ging einen Schritt zurück. Sie schaute ihn an. Harry musste wieder schlucken. Langsam begann sie ihre eigene Robe aufzuknöpfen, um sie fallen zu lassen. Danach widmete sie sich Harrys Hemd.

\enquote{Willst du nicht auch?}, fragte sie.

Harry hob seine Hände und griff an den obersten Knopf ihrer Bluse. Als beide mit nacktem Oberkörper da standen, konnte er zum ersten Mal ihre Brüste sehen. Er senkte seinen Kopf und umspielte ihre Brustwarzen mit seiner Zunge. Sie ließ ein leises Stöhnen hören, während er mit seiner Zunge am Vorhof ihrer linken Brust spielte und mit seiner rechten Hand ihre andere Brust sanft massierte. Er stoppte kurz und strich dann mit seiner Zunge zwischen ihren Brüsten entlang hoch bis zu ihrer Gurgel. Pansy warf den Kopf nach hinten und lies ein leises Gurgeln und Stöhnen erklingen. Er fuhr mit seiner Zunge weiter über ihr Kinn zu ihrem Mund. Sie versank ganz in dem Kuss, den er ihr gab. Seine Hände waren inzwischen an ihrem Nacken angekommen. Sie spürte seine Berührungen und so jagte es ihr einen wohligen Schauer über den Rücken.

Sie fuhr mit ihren Händen seinen Rücken entlang und arbeitete sich an seinem Hosenbund entlang nach vorn, um die Knöpfe an seiner Hose zu öffnen. Er brach den Kuss und machte sich daran, es ihr gleichzutun. Bald standen sie nackt voreinander, im schmalen Schein des Zauberstabes. Er zog sie sanft an die Stelle, die er mit dem Polster- und Wärmezauber belegt hatte und auf der jetzt ihre Schulroben lagen. Sanft drückte er sie an ihren Schultern auf den Boden, auf den sie sich nun legte. Jetzt lag sie vor ihm. Er beugte sich vor und stolperte. Er kam mit seiner Nase kurz vor dem Zentrum ihrer Lust zum Stoppen und hörte ein leises Kichern. \gedanke{Na warte, Pansy}, dachte Harry. Er drückte seine Nase zwischen ihre Schamlippen und fuhr sanft nach oben. Wieder hörte er ein Gurgeln. Sie atmete schnell, aber gleichmäßig. Auch Harrys Puls- und Herzschlag war schon seit einiger Zeit schneller geworden.

Er schaute zu ihr auf. Seine Nasenspitze war nass. Er streckte langsam seine Zunge raus und fuhr wieder zwischen ihre Schamlippen. Ihre Hände bohrten sich in die Schulroben, auf denen sie lag und ein leiser Schrei drang aus ihrer Kehle. \enquote{Harry!} Dann bahnte er sich seinen Weg in leichten schlängelnden Bewegungen nach oben zu ihrem Bauchnabel, wo er kurz verweilte und ihn küsste. Er kam ihren Brüsten langsam näher und beschrieb mit seiner Zunge eine Acht. Seine Hände streichelten an ihrer Seite entlang hoch. Er fand schließlich wieder ihren Mund und beide versanken in einen langen und innigen Kuss. Als er den Kuss kurz unterbrach, leckte sie über seine feuchte Nase, nur um ihn kurz darauf zu sich zu ziehen und ihn erneut zu küssen.

\end{abAchtzehn}

\begin{safedivide}
\fskdivider
\end{safedivide}

\onelineback % Anderenfalls werden 2 Leerzeilen gesetzt

\begin{rueckblick}
Als es Harry zum ersten Mal auffiel, dass sämtliche Mädchen in Hogwarts sich ihm zugeneigt fühlten, sprach er mit Dumbledore darüber. Auch er wusste sich nicht zu helfen, schickte ihn aber mit einem Brief zu Madame Pomfrey. Also machte sich Harry auf den Weg von seinem Büro aus zum Krankenflügel. Er trat ein und klopfte, nachdem er den Krankenflügel durchquerte hatte, an die Tür ihres Büros. Er hörte ein deutliches \enquote{Herein!} und öffnete die Tür.

Er gab ihr den Brief und meinte: \enquote{Der ist von Professor Dumbledore.}

Sie nahm den Brief und öffnete ihn. Ihre Augen verengten sich und sie sah zwischen ihm und dem Brief hin und her. Harry bekam einen Kloß im Hals. \enquote{Kommen Sie mit, Potter}, sagte sie zum ihm und ging durch eine kleine Tür, auf der \accentuate{Apotheke} stand. \enquote{Schließen Sie die Tür hinter sich}, sagte sie, zog ihren Zauberstab und machte ein kleines Feuer unter einem Kessel. Sie entnahm einige Fläschchen und öffnete diverse Fächer an ihrem großen Wandschubladenschrank.

\enquote{Was ist los?}, wollte Harry wissen.

Madame Pomfrey fing an. \enquote{Die weibliche Bevölkerung von Hogwarts hat ein starkes Interesse an Ihnen. Wir wissen nicht warum, aber Professor Dumbledore meinte, dass wir Sie vor, naja, etwaigen Folgen schützen müssen.}

Harry rätselte und schaute Madame Pomfrey verwundert an. Aber so langsam begriff er.

\enquote{Was ich Ihnen gebe, ist ein Trank, der Männer bis auf Weiteres unfruchtbar macht. Es wissen nicht viele davon, da die Zutaten selten sind. Die für den Prophylaxis-Trank sind viel leichter zu beschaffen.} Harry nickte. \enquote{Es gibt ein Gegenmittel, das ich Ihnen Morgen geben werden. Das muss etwas länger köcheln. Heben Sie das Gegenmittel auf, bis sich die Sache mit den Schülerinnen gelegt hat, dann können Sie es nehmen. Denken Sie daran, dass Sie so lange keine Kinder zeugen können.} Sie schöpfte den Trank in einen Becher und meinte: \enquote{Den Kessel müssen Sie austrinken. Zwei Becher voll.} Harry nickte und schluckte seinen Trank hinunter.
\end{rueckblick}

\onelineback % Anderenfalls werden 2 Leerzeilen gesetzt

\begin{abAchtzehn}
Harry und Pansy trennten wieder ihre Münder voneinander.
\end{abAchtzehn}

\trenn

Harry verstand die Welt nicht mehr. Tränen überströmt saß er auf dem kalten Fußboden. Das hatte er verdient. Nie hätte er sich träumen lassen, dass es so weit kommen würde. Seinen Kopf zwischen den Beinen saß er da; vor sich eine Pfütze aus Tränen. Professor Snape kam den Gang entlang auf ihn zu und nachdem er sich vergewissert hatte, dass niemand weit und breit zu sehen und zu hören war, setzte er sich neben Harry auf den Boden, aber so, dass er jederzeit aufstehen konnte, falls er jemanden hören konnte.

\enquote{Was ist los, Potter?}, fragte ihn Snape.

Stille.

\enquote{Potter?}

\enquote{Ich schäme mich. Sie wissen bestimmt, wie es um mich steht!}

\enquote{Ihr momentaner Zustand, der alle Frauen verrückt macht? Nach Ihnen?}

Es war Harry so peinlich, darüber zu reden. Aber irgendjemand brauchte er und Snape war momentan genau der Richtige. Er würde ihn ohne Wenn und Aber zusammen stauchen, aber das hatte er wohl verdient.

\enquote{Professor? Ich habe\abs hätte beinahe\abs} Irgendwie fand er nicht die richtigen Worte.

\enquote{Dann von Anfang an}, sagte Snape. Allerdings nicht bedrohlich, sondern eher so, wie ein fürsorglicher Freund.

Harry sah ihn mit immer noch nassem Gesicht an. \enquote{Kennen Sie die Erstklässler aus Ravenclaw?}

\enquote{Einige!}

\enquote{Auch eine dunkelhaarige, braune, mit kleinen Locken und schulterlangem Haar?}

Snape nickte.

\enquote{Sie rannte einfach in mich hinein. Wir haben uns nicht gesehen. Wir fielen um, sie lag auf mir und sie küsste mich.} Wieder lief eine Träne sein Gesicht herunter.

\enquote{Das ist doch kein Grund\abs}, doch Harry hob die Hand und saß wieder auf den Boden.

\enquote{Sie hat intensiv mit mir geknutscht und ich konnte nicht\abs Jedenfalls überkam es uns und sie zerrte mich in ein Zimmer. Ich widersprach nicht. Bald waren wir\abs} Er sah Snape wieder an. \enquote{Wir hätten beinahe miteinander geschlafen\abs} Doch er kam nicht weiter.

Snape zog überraschend ein Taschentuch aus seiner Tasche heraus und reichte es Harry. Es war noch unbenutzt. Harry trocknete seine Tränen und schniefte einmal feste hinein. Dann gab er es Snape zurück, der es einsteckte.

\enquote{Eine elfjährige}, sagte er noch matt. \enquote{Ich könnte es bei allen ab vierzehn ja noch\abs}, machte er weiter.

Professor Snape stand auf und zog Harry hoch. Dann trug er ihn halb neben sich her, hinunter Richtung Kerker, durch das leere Klassenzimmer für Tränke in sein Büro. Danach setzte er ihn in einen Stuhl, gegenüber seinem Schreibtisch.

\enquote{Ich brauche unbedingt mehr Selbstkontrolle}, sagte Harry. \enquote{So kann es nicht weitergehen. Ich meine, ich habe nicht, wenn mich Fünft-, Sechst- oder Siebtklässlerinnen küssen und mit mir schmusen \gst oder mehr wollen \gst aber alle darunter\abs Bisher habe ich jüngere\abs es ist schrecklich Professor.}

Snape hatte unterdessen ein Feuer unter einem Kessel gemacht und warf einige Kräuter und andere Flüssigkeiten hinein. Dann nahm er ein Fläschchen mit einer Pipette und zählte Tropfen ab. Er goss einen Becher voll und leerte den Kessel mit dem Schwung seines Zauberstabes. Dann stellte er den Becher auf den Tisch und kühlte ihn mit einem anderen Zauber herunter, da er noch gefährlich dampfte.

\enquote{Trinken Sie das, das wird Ihre Sorgen für ein paar Stunden vertreiben. Gehen Sie danach gleich ins Bett. Morgen werden Ihre Sorgen zwar wieder kommen, aber Sie brauchen erst einmal Schlaf.}

Harry nickte, nahm den Becher und trank ihn in einem Zug leer. Dann trottete er die Stufen und Gänge hinauf, bis in sein Zimmer. Er ignorierte alle Leute, die von ihm etwas wissen wollten, zog sich um und ging ins Bett. Sorgenfrei schlief er ein und träumte.

\begin{traum}
Er stand auf einer Wiese und sah sich um. Seine Mutter und sein Vater standen neben ihm. Sie lächelten ihn an. \enquote{Hallo Harry. Du hast heute bewiesen, dass du stark bist}, sagte sein Vater. Harry verstand nicht. \enquote{Du hast dich heute zusammengerissen und nicht mit der Kleinen geschlafen}, sagte seine Mutter stolz.

\enquote{Was?}, stammelte Harry. \enquote{Ihr seid tot. Ihr wisst nicht, was ich gemacht habe.}

\enquote{Ja das stimmt}, sagte seine Mutter und nahm ihn in ihre Arme. \enquote{Aber wir sind immer bei dir, wenn du dein Amulett hältst, während du einschläfst.}

\enquote{Ich halte gerade mein Amulett?}, fragte Harry ganz erstaunt.

\enquote{Ja und nein}, antwortete sein Vater. \enquote{Du träumst gerade.}

\enquote{Immer, wenn du träumst und sorgenfrei einschläfst}, machte seine Mutter weiter, \enquote{dann können wir dir Gesellschaft leisten. Wir bekommen nur das von dir mit, was du uns sagst, oder das, woran du gedacht hast, während du einschliefst.}

Harry verstand so langsam. \enquote{Ihr meint, ihr seid tot, aber könnt euch mit mir durch meine Träume unterhalten?}, fragte Harry.

\enquote{Ja}, antwortete seine Mutter. \enquote{Wir sind erstaunt, dass du so gut aussiehst, Harry. Wir konnten immer wieder mal einen kurzen Blick auf dich werfen und sind stolz auf dich.}
\end{traum}

Dann verschwamm der Traum und Harry drehte sich im Bett um. Am anderen Morgen erinnerte er sich nur noch schemenhaft daran.

Er saß gerade bei Dumbledore und erklärte ihm die Sache mit der elfjährigen und was er beinahe getan hätte. Seine Augen waren kurz davor, zum Tränen anzufangen. Sein Schulleiter sah ihn nur an. \enquote{Tja}, war alles, was er sagte. Dann herrschte eine Weile Stille. Dumbledore stand auf, ging zu seinem Kamin und warf eine kleine Menge Flohpulver hinein. \enquote{Severus Snape}, sagte er.

Es dauerte eine Weile, dann tauchte der Kopf von Professor Snape auf. \enquote{Schulleiter? Was ist los?}

\enquote{Harry braucht ein Mittel, um seinen Willen zu stärken. Er hat mir gerade eben erzählt, was er gestern beinahe getan hätte.}

\enquote{Die kleine Ravenclaw? Braune Haare?}, fragte Snape.

Erstaunt hob Dumbledore eine Augenbraue und sah zu Harry. Dieser blickte ihn an und sagte nur: \enquote{Ich habe es ihm gestern erzählt.}

Dumbledore nickte und drehte sich wieder zum Kamin. \enquote{Wann ist der Trank bereit?}

\enquote{Morgen Nachmittag. Ich arbeite bereits daran.} Dann verschwand Snapes Gesicht aus dem Kamin und Dumbledore setzte sich.

\enquote{Ich schäme mich so, Professor. Ich bin kein guter Mensch.}

Doch Dumbledore unterbrach ihn. \enquote{Doch Harry, du bist ein guter Mensch. Du hast es nicht getan, obwohl es dir nicht leicht gefallen ist. Ich bewundere dich. Du hast sehr viel Selbstkontrolle.}

Harry zweifelte noch immer. \enquote{Aber Sir\abs}

\enquote{Kein Aber, Harry}, sagte Dumbledore und hob erneut seine Hand. \enquote{Du wusstest genau, wo deine Grenze liegt. Ich bin mir sicher, dass nicht viele Schüler in deiner Situation genauso gehandelt hätten. Außerdem\abs} und Dumbledore machte eine kurze Pause, damit Harry wieder zu ihm sah, \enquote{außerdem hat mir unsere Krankenschwester von dem kleinen Vorfall in der Apotheke berichtet. Sie bemerkte das Blitzen und Verlangen in deinen Augen.} Er lehnte sich vor. \enquote{Und Harry, das darfst du niemandem sagen, wenn du ihr Andeutungen gemacht hättest, oder sie auch nur berührt, oder angelächelt hättest, dann hätte sie ihre Beherrschung verloren.}

Jetzt musste Harry schmunzeln und war erleichtert.

Auch Tage später wirbelten Harrys Gedanken immer noch um Pansy und ihr gemeinsames Intermezzo. Hatte er wirklich mit ihr geschlafen? Oder verdrängte er es nur? Snapes Trank hatte seinem Willen eine gewisse Stärke verschafft, sodass er bei Leuten, bei denen er es für unangebracht hielt, nicht weiter ging, als mit ihnen verstohlen oder offen zu knutschen. Cho hatte ihn in der Zwischenzeit angesprochen, um zu erfahren, was er Pansy versprochen hatte, damit ihre kurze Affäre nicht im Schloss bekannt wurde. Sie hatten sich schließlich, nach ihrem flüchtigen Kuss im Raum der Wünsche, nicht großartig aufeinander eingelassen. Er antwortete ihr nur: \enquote{Sie war ganz pflegeleicht.} Weitere Informationen bekam sie nicht aus ihm nicht heraus.

Harry saß wieder mit Pansy auf einer Bank im Schloss und knutschte ein wenig mit ihr. Die Bank stand an der Wand, direkt gegenüber eines Treppenaufganges. Man konnte also jeden sehen, der die Treppe hochkam. Die beiden waren so mit sich beschäftigt, dass sie die Schritte, die die Treppe hochkamen, nicht hörten. Beide erschraken, als sie \enquote{Pansy, Potter} und \enquote{Harry, Parkinson} hörten. Erschrocken schauten sie die beiden an. Da standen Ron Weasley und Draco Malfoy. Harrys Herz schlug wieder schneller, er wollte nicht mit Pansy erwischt werden, sie waren die letzten Wochen sehr vorsichtig gewesen, damit ihre Romanze nicht bekannt wurde. Harrys Gedanken rasten. Er blickte zu Pansy und dann tat er etwas, womit sie nicht gerechnet hatte. Er küsste sie munter weiter, so als ob sie nicht unterbrochen worden wären. Harry konnte praktisch ihre ratlosen Gesichter sehen.

\enquote{Harry!} \enquote{Potter!}, hörte er wieder.

Er unterbrach den Kuss und schaute beide mit aufgesetztem, enttäuschtem Gesichtsausdruck an. Pansy tat das Gleiche. Sie starrten in die entsetzten Gesichter der beiden. Er musste sich ein Lachen verkneifen.

Pansy zog ihn an seinem Ärmel und meinte: \enquote{Lass uns woanders hingehen, wo wir ungestörter sind.}

Er schaute sie an und nickte. Dann standen beide auf und gingen Hand in Hand den Gang entlang um die nächste Ecke. Ab dort beschleunigten sie ihren Gang und jeder lief für sich die geschwungene Treppe hinunter. Unten angekommen setzten sich beide auf die dritt-unterste Stufe. Sie sahen sich an und mussten anfangen zu lachen. Harry hatte eine Hand vor seinem Bauch und die andere vor seinem Mund. Pansy bot sich als sein Spiegelbild an. Nach kurzer Zeit wechselte er von seiner Hand auf seinen Unterarm, um sein Lachen nicht durch das Schloss erklingen zu lassen.

Eine knappe Minute später hörten beide wieder Schritte. Noch immer kicherten und lachten beide, auf den Stufen sitzend. Sie drehten sich um, schauten in die noch erstaunter dreinblickenden Gesichter von Ron Weasley und Draco Malfoy.

Dann sagte Pansy: \enquote{Ihr hättet vorher wirklich eure Gesichter sehen sollen. Als hätte euch der Blitz getroffen.}

Harry nickte und fing nun an laut zu lachen. Pansy stimmte in sein Lachen ein.

\enquote{Das \gst das war \gst war inszeniert?}, fragte Malfoy ganz ungläubig und Ron schaute aus, als wäre ihm übel.

Harry lachte noch immer und nickte. \enquote{Aber sicher doch. Ihr glaubt doch nicht etwa\abs}, und Pansy ergänzte: \enquote{Dass wir beide\abs}, sie zeigte auf Harry und sich, \enquote{etwas miteinander\abs}

\enquote{Nein}, antworteten beide im Gleichklang. Harry und Pansy schauten sich wieder an. Er hob seine Hand und sie schlug ein.

\enquote{Eure dummen Gesichter war es mir wert}, sagten beide fast gleichzeitig. Dann standen sie auf und verschwanden in unterschiedlichen Richtungen im Inneren des Schlosses.

Ron und Draco schauten sich nur verständnislos an.

\trenn

Harry saß bereits im Klassenzimmer und wartete als Erster auf seine Klassenkameraden. Als die Klasse zu etwa einem Drittel anwesend war, kam Professor Elber mit einer Pappschachtel herein. Er stellte sie auf den Tisch am Kopf der Klasse. Es dauerte noch eine Weile bis die Klasse komplett anwesend war. Er öffnete die Schachtel und nahm sie in eine Hand. Mit der anderen verteilte er aus der Schachtel an jeden Schüler einen kleinen schwarzen Würfel mit einer Kantenlänge von etwa 6~cm. Als jeder der Schüler einen Würfel erhalten hatte, lief er wieder nach vorne und legte die Schachtel mit den übrig gebliebenen Würfeln auf ein Sideboard. \enquote{Ihre Hausaufgabe bis Ende des Schuljahres ist es, diesen Würfel zu öffnen. Konzentrieren Sie sich auf den Würfel bis Sie ihn öffnen können. Sollten Sie einen neuen Würfel brauchen, so nehmen Sie sich einen. Wenn keiner mehr da ist, lege ich neue hin \gst Doch nun lassen Sie mich Sie auf Ihre Prüfung vorbereiten.}

Er drehte sich herum und nahm aus der Schublade seines Schreibtisches eine Menge von Zetteln heraus. Nachdem er alle verdeckt herum ausgeteilt hatte, meinte er: \enquote{Sie haben die restliche Stunde über Zeit, Ihre theoretische Testprüfung zu machen. Sie wird in ähnlicher Form in ein paar Wochen abgehalten werden. \gst Die Zeit läuft.}

Er holte aus seiner Tasche eine kleine Uhr heraus, die wie eine Stoppuhr aussah, und betätigte die Krone, um die Zeitmessung zu starten.

Während Harry seinen Zettel ausfüllte, hörte er immer wieder Gekicher und spürte heiße Blicke hinter seinem Rücken. Es dauerte noch, bis Snape seinen Trank fertiggestellt hatte. Er musste sich beherrschen. Wenn er kein \accentuate{Weibchen} sah, ging es. Doch er konnte seine Zuneigung zu den weiblichen Schülern nicht verbergen. Genauso, wie die Schülerinnen nicht von ihm lassen konnten. Er schloss seine Augen und konzentrierte sich. Er benutzte wieder die Okklumentik-Techniken, um seinen Geist zu leeren. Dann öffnete er wieder seine Augen und füllte seinen Zettel aus.

Nach der Stunde betrat er die Bibliothek, da er noch etwas benötigte. Er ging zielstrebig auf die Reihe zu, in der das benötigte Buch stand. Madame Pince stand vor ihm. Er überlegte, ob er sie nicht in Verlegenheit bringen sollte. In seinem Gehirn ratterte es. Verdient hätte sie es, nachdem sie ihn während seiner Strafarbeit nicht unbedingt fair behandelt hatte. Harry fixierte sich auf sie. Er schlich sich von hinten an sie heran und legte seine rechte Hand auf ihre linke Schulter.

\enquote{Entschuldigung, Madame Pince}, sagte Harry.

Sie drehte sich um und sah Harry erschrocken an. Hinter ihr stand Draco Malfoy und nahm ihr gerade ein Buch ab. Sie sah Harry nun direkt in die Augen. Seine Hand lag noch immer auf ihrer Schulter. Abwesend strich er ein paar Fusseln von ihrer Schulter. Mit einem Blick, den er von Luna abgeschaut hatte, sah er sie an. Seine Hand glitt an ihrem Arm hinab. An ihrer Hand angekommen, nahm er sie schließlich in seine. \enquote{Können Sie mir sagen, wo ich ein Buch über schwarze Würfel finde?} Er massierte jetzt ihr Handgelenk mit Daumen und Mittelfinger. Er spürte, wie sie ein Schauer durchzog.

Dann meinte er schließlich: \enquote{Schade, dann werde ich selber suchen.} Er ließ sie los und drehte sich gerade um, als er von hinten zwei Hände spürte, die ihn nach hinten zogen. Nur für eine Sekunde hatte sie ihn an sich gezogen, bevor sie ihn wieder losgelassen hatte und dann davon stürmte.

Harry war bereits auf dem Weg zu einer anderen Reihe um ein Buch zu holen, als ihn Malfoy und Zabini einholten.

\enquote{Klasse, Potter}, lachte Draco Malfoy hinter ihm. Und auch Blaise Zabini konnte sich nicht mehr halten. Harry drehte sich schmunzelnd um.

\enquote{Das war die kleine Rache.}

\enquote{Wofür?}, fragte Blaise ihn, immer noch lachend.

\enquote{Dafür, dass sie mich am Jahresanfang während meiner Strafarbeit ständig genervt hatte.}

Ein Mädchen schlich hinter Harry vorbei und fuhr ihm sanft seinen Rücken entlang. Harry schloss seine Augen und genoss die Berührung. Er sah ihr kurz nach. \gedanke{Romilda Vane, hatte Hermine gesagt.}

\enquote{Die Frauen stehen immer noch auf dich, oder?}, fragte Malfoy.

\enquote{Leider. Manchmal ist es ja angenehm, aber wenn man ständig betatscht wird, oder umarmt wird, oder\abs} er schluckte kurz. \enquote{Ohne Vorwarnung geküsst wird, dann ist es extrem nervig.}

Blaise sah zu Draco und meinte plötzlich: \enquote{Hast du dich inzwischen von deinem Schrecken erholt?}

\enquote{Von welchem Schrecken}, fragte Draco.

\enquote{Dass du Harry und Pansy beim Knutschen gesehen hast.}

Draco Malfoy wurde rot. \enquote{Mann, Potter}, sagte er dann, \enquote{da hast du mich aber schön erwischt. Der Schreck hängt mir immer noch nach.}

Harry grinste spitzbübisch.

\enquote{Sag mal, schwärmt sie irgendwie für dich?}

\enquote{Sagen wir mal so, es hatte mir die Sache wesentlich erleichtert, euch diesen Streich zu spielen. Wir haben euch schon am Fuße der Treppe entdeckt. Dann hatte ich die Idee.}

Er verabschiedete sich von beiden und steuerte einen anderen Bereich der Bibliothek an. Mit Blaise hatte er sich schon immer so weit verstanden, wie man sich als Schüler eigentlich verfeindeter Häuser verstehen konnte. Sie waren einander gegenüber neutral eingestellt.

Die Bibliothek war brechend voll. Die Suche nach der Mondbibliothek sowie dem schwarzen Würfel ergab nichts Neues, aber er hatte etwas über Gebärdensprache gefunden. Nun musste er nur noch lernen. Harry fand noch einen Platz an Lavenders Tisch und machte sich an seine Studien. Lavender hatte ein sehr altes und verschlissenes Buch vor sich liegen und betrachtete ein paar interessante Zaubersprüche. Sie hatte ihren Zauberstab in der Hand und vollführte einige kompliziert aussehende Bewegungen.

\enquote{Was übst du da?}, fragte sie Harry.

\enquote{Weiß ich nicht. Aber die Bewegung finde ich klasse.}

\enquote{Wie heißt der Zauber?}

\enquote{Weiß ich nicht.} Sie vollzog die Bewegung immer und immer wieder. \zauber{Ad unum omnes, induere potens domus}, sagte sie plötzlich, als sie wieder auf das Buch sah.
% sexuell					potens
% anziehen					induere
% haus						domus
% alle						cuncti, omnes
% alle zusammen				cuncti
% alle ohne Ausnahme		ad unum omnes
% alle bis zum Letzten

\enquote{\extase{Neeeiiinnn!}} hallte es plötzlich durch die gesamte Bibliothek. Vor Lavender bildete sich an der Spitze ihres Zauberstabes eine leuchtende Kugel, die größer wurde. Dann gab es einen Knall und eine Welle aus gleißendem Licht durchzog das Schloss. Professor Elber kam mit schnellen Schritten auf sie zu. Er drückte sich zwischen James und Leroy, zwei Freunde von Lavender, die ihr gegenüber saßen, und starrte Lavender mit einem mehr als zornigen Gesichtsausdruck an. \enquote{Was haben Sie getan?}, schrie er sie an. Er ignorierte die Beschwerden der beiden. \enquote{Wenn Sie einen Zauber nicht kennen, oder dessen Wirkung, dann lassen Sie die Finger davon, oder fragen einen, der sich damit auskennt. Fragen Sie einen der Lehrer oder Madame Pince. Frell.}

Die Kugel aus Licht schwebte noch immer vor Lavender. Sie ließ ihren Zauberstab sinken und die Kugel verschwand.

\enquote{Haben Sie eine Vorstellung davon, was Sie getan haben?}, brüllte er sie an. Harry hatte seinen Lehrer noch nie so zornig gesehen. Dann blickte Elber auf das Buch und schlug es zu. Seine Hand stützte er darauf ab. Lavenders Hand wurde eingeklemmt und sie stöhnte leicht auf. Mühsam zog sie ihre Hand wieder heraus. Tränen begannen nun über ihr Gesicht zu laufen. Aber noch immer sah sie Professor Elber wütend an. Sie begann zu weinen und zitterte am ganzen Leib.

Nur langsam beruhigte sich Professor Elber. Er schloss seine Augen und atmete tief ein und aus. Dann lief er um den Tisch herum und ging neben ihr in die Hocke, legte eine Hand auf ihre Schulter und fing an zu erzählen. \enquote{Tut mir leid, aber mit dem Zauber haben Sie uns allen eine Menge Ärger eingehandelt.}

\enquote{Ich\abs ich verstehe nicht Professor}, stammelte Lavender.

\enquote{Dieser Zauber}, fing Professor Elber mit leicht verärgertem Ton an, \enquote{sorgt dafür, dass jeder über jeden herfällt. Sie erinnern sich, dass wir eine ähnliche Situation gerade haben?}

Lavenders Augen weiteten sich vor Schrecken. \enquote{Aber\abs er hat nicht gewirkt}, sagte sie schüchtern.

\enquote{Er wirkt mit Verzögerung}, sagte Professor Elber und schaute auf seine Uhr. \enquote{Nach vierundzwanzig Stunden beginnt hier jeder, jedem hinterherzulaufen. Aber das Schlimmste daran ist, dass Harry hier den Effekt durch seinen Zustand noch verstärken könnte. Ganz davon abgesehen, dass Sie, die den Spruch ausgesprochen haben, besonders darunter zu leiden haben.} Betrübt schaute er sie an. Lavender war nicht in der Lage, etwas zu sagen. \enquote{Sind sie mit Sodom und Gomorrha vertraut?}, fragte er sie. Sie nickte stumm. \enquote{Der Spruch wurde dort von einem schwarzen Magier erschaffen, als er mit Schimpf und Schande aus der Stadt vertrieben wurde. Er hatte sich Rache geschworen und kam ein Jahr später wieder und legte den Spruch auf die Stadt. Dann verschwand er, um sich dem Zauber zu entziehen.}

Lavender war immer noch nervös. \enquote{Kann\abs man\abs gar nicht\abs gar\abs nichts dagegen tun?}, fragte Lavender.

\enquote{Schon}, antwortete Professor Elber, \enquote{aber der Gegentrank muss jedem einzeln eingeflößt werden und braucht achtundvierzig Stunden um fertig zu werden.} Dann fügte er hinzu. \enquote{Sie werden diesen Trank brauen. Sozusagen als Strafarbeit. Sie werden die nächsten zwei Tage kaum zum Schlafen kommen. Wir werden jetzt zusammen ins Lehrerzimmer gehen, wo Sie dem Kollegium ausführlich berichten werden. Wir müssen alle Schüler mit einem Klammerfluch belegen, sodass sich keiner einer Gefahr aussetzt. Selbst die Lehrer. Außerdem wird man ihren Eltern schreiben.}

Lavender lief es wieder eiskalt den Rücken runter. Professor Elber stand auf und Lavender folgte ihm.

Der ganze Tag war wie verhext. Kein Lehrer, egal welcher, konnte sich auf den Unterricht konzentrieren und alle sahen Lavender vorwurfsvoll an. Harry hatte mit ihr irgendwie Mitleid. Sie war zwar manchmal anstrengend, aber das hatte selbst sie nicht verdient. Immer wieder musste er sie gegen die gehässigen Slytherins verteidigen. Als ob das nicht genug wäre, fiel ihm Lavender deshalb jedes Mal dankbar um den Hals und küsste ihn flüchtig, denn Harry litt noch immer unter seinem Zustand.

Er musste sich in sein Bett legen, um in Kürze von seiner Hauslehrerin geklammert zu werden. Er verdrängte alle Gedanken und hatte das Gefühl zu schweben. Professor McGonagall sprach den Klammerfluch \spruch{Petrificus totalus} und Harrys Körper konnte sich nicht mehr bewegen. Doch Harry fühlte sich nicht anders als sonst. Er richtete sich wieder auf, um Professor McGonagall zu sagen, dass der Zauber wohl schiefgegangen sein musste. Doch  Professor McGonagall drehte sich um und verließ sein Zimmer. Er wollte ihr noch hinterherrufen, aber er brachte kein Wort heraus. Er verließ sein Bett und wollte in seine Pantoffeln steigen, als er bemerkte, dass er sich nicht sehen konnte. Er drehte sich um und sah seinen Körper im Bett liegen.

\enquote{Liegt das an der Okklumentik-Technik?}, fragte er sich.

\enquote{Ja, Potter}, hörte er hinter sich. Schlagartig drehte er sich um und sah Professor Snape.

\enquote{Professor?}

Er schaute wieder an sich herunter und bemerkte, dass er sich langsam sehen konnte. Immer fester wurde seine Gestalt.

\enquote{Wir beide sind geklammert, aber da Sie, genau wie ich, Ok"-klu"-men"-tik-Tech"-ni"-ken anwandten, konnten Sie Ihren Geist der Klammerung entziehen. Schweben wir in den Tränke"-keller und schauen ihr zu.} Er ging an Harry vorbei und ließ sich danach durch die Decke sinken. Harry folgte ihm durch Decken und Mauern, auf direktem Weg zum Zauber"-tränke"-keller.

Dort angekommen, sahen sie Lavender vor einem komplizierten Trank liegen. Ihre Haare waren offen und zerzaust, ihr spärliches Make-up war verschmiert und sie hatte Ringe unter ihren Augen. Dösend lag sie auf dem Boden und schlief. Der Wecker rasselte und Lavender mühte sich die nächsten Zutaten in den Kessel zu werfen und umzurühren.

Professor Elber betrat den Raum und meinte: \enquote{Professor McGonagall und Professor Dumbledore waren die letzten. Jetzt sind alle geklammert und können sich und ihnen nichts mehr antun.}

\enquote{Soll ich Sie jetzt auch klammern?}, fragte Lavender unsicher.

\enquote{Nein}, antwortete Professor Elber. \enquote{Ich habe mich unter Kontrolle. Der Wutausbruch, als Sie den Zauber ausführten, dämpften meine Gefühle. Ich sorge mich mehr um Sie. Ich fühle mich zwar zu Ihnen hingezogen, dafür haben Sie gesorgt, aber ich habe mich unter Kontrolle. Jemand muss ja auf Sie aufpassen. \gst Zumindest hoffe ich, dass ich es aushalte.} Er lächelte leicht.

\enquote{Wie darf ich das verstehen?}

\enquote{Falls ich Ihnen zu nahe komme, dann müssen Sie mich auch klammern. \gst Auch wenn Sie den Trank fertig haben und während wir den Schülern und Lehrern ihn einflößen, kommen sie nicht darum herum\abs}, er druckste etwas herum, \enquote{einige sexuelle\abs na ja, Handlungen vorzunehmen.}

Harry war der Meinung, das Lavender leicht übel wurde, doch sie braute tapfer an ihrem Trank weiter.

\enquote{Außerdem werden Sie noch Nachwirkungen spüren. Ich sage das jetzt äußerst ungern, aber Sie und Harry werden wohl übereinander herfallen und sich lieben ohne Ende. Solange bis Sie erschöpft sind. Das dürfte einige Nächte andauern.}

Harry stand mit großen Augen da. Snape konnte sich einen gehässigen Kommentar nicht verkneifen. \enquote{Das hört sich so an, als ob Sie eine Menge Spaß miteinander haben werden.}

\enquote{Und, und das ist das Wichtigste daran: Sie dürfen sich nicht dagegen wehren. Es wird umso stärker und intensiver, wenn Sie sich enthalten. Lassen Sie es einfach geschehen. Zum Glück haben Sie noch etwas Zeit, wenn das Ganze überstanden ist.}

\enquote{Weiß Harry davon?}, fragte sie halb begeistert, halb erschrocken darüber, dass sie mit dem berühmten Potter ganze sinnliche Nächte verbringen würde; oder sollte man sagen: müsste?

\enquote{Nein, noch nicht.} Professor Elber schloss kurz die Augen. \enquote{Obwohl\abs} Er öffnete sie wieder. \enquote{Ich könnte mir durchaus vorstellen, dass er so ein Gefühl diesbezüglich haben könnte.} Er musste nun leicht schmunzeln.

\gedanke{Spürt er meine Anwesenheit?}, schoss es Harry durch den Kopf. \gedanke{Nein, vermutlich kennt er meinen Zustand nur gut. Ja, das wird es sein.}

\enquote{Wissen Sie, ich habe nochmal in dem Buch in der Bibliothek gelesen. Es steht zwar nicht präzise drin, aber das zwischen den Zeilen lässt sich schon lesen, dass Sie und Harry sexuell sehr aktiv\abs}

Als der Trank nach zwei Tagen endlich fertig war, gingen Professor Elber und Lavender in die Gemeinschaftsräume zu den Schülern und Lehrern, die der Einfachheit halber auch dort lagen und flößten ihnen die Tränke ein. Professor Snape hatte es irgendwie geschafft, Lavender derart zu beeinflussen, dass der Trank auch gelang. Lavender schaffte es nicht immer, sich zu beherrschen und berührte ihre noch immer geklammerten Mitschüler und Mitschülerinnen manchmal an intimen Stellen. Glücklicherweise bekamen diese das nur sehr selten mit. Aber am peinlichsten war ihr das bei Professor Snape. Sie wusste, dafür würde sie bezahlen. Sie hatte zwar den Eindruck, dass er das nicht mitbekam, hatte aber das Gefühl, er würde es doch herausfinden.

Damit hatte sie nicht einmal so unrecht, denn er schwebte mit Harry neben ihr, als sie ihm den Trank einflößte.

Harry sagte trocken nun zu seinem Tränkeprofessor: \enquote{Ich verkneife mir jetzt mal einen Kommentar.}

Snape sah ihn nur an.

\trenn

Harry hatte sich im Gemeinschaftsraum in einen Sessel in der Ecke gesetzt. Er zog seine Füße auf den Sessel und drehte sich so, dass man ihn nicht gleich erkennen konnte. Er schloss seine Augen und dachte über ihn und Pansy nach. \gedanke{Unsere Beziehung hat sich gewandelt. Von einer anfänglichen rein sexuellen Beziehung zu etwas wie Liebe. Verliebt. Bin ich tatsächlich in Pansy Parkinson verliebt? Und warum ist Ginny eine der wenigen, die sich nicht an mich ran werfen?}, fragte er sich. Er hatte mit ihr noch nicht darüber gesprochen, aber er war sich sicher, dass Pansy seine Gefühle erwiderte. Ihre Küsse hatten sich verändert. Sie verwendete in letzter Zeit etwas, was ihre Lippen noch zarter machte. Das Porträt öffnete sich und Ron kam herein. Harry hatte seine Augen immer noch geschlossen; er hörte es an seinem Gang.

Er konnte sich seinen Gesichtsausdruck immer noch vorstellen. Er hörte, wie Ron anfing zu erzählen und ließ seine Gedanken gleiten. Er hörte nur mit halbem Ohr hin und öffnete ein Auge, aber nur so weit, dass er Hermine sehen konnte, die er gut im Blick hatte. Er hörte nur Bruchstücke der Unterhaltung, konnte sich aber den Rest lebhaft vorstellen. \enquote{Kleinen Zwist mit Malfoy\abs Harry und Pansy Parkinson\abs Knutschen\abs}, Hermine schluckte und quiekte immer mal wieder, \enquote{nicht unterbrechen\abs kringelten sich vor Lachen\abs hereingelegt.} Er entdeckte Hermine, die immer wieder verstohlen zu ihm hinüberblickte.

\enquote{Ich denke mal, da hat er dir und Malfoy einen schönen Schrecken eingejagt.}

\enquote{Das kann man wohl sagen. Mir läuft es immer noch kalt den Rücken runter, wenn ich mir das Bild vorstelle, wie ich beide zum ersten Mal sah. Auf der Bank sitzend und knutschend. Ich brauche wieder einmal eine heiße Dusche.} Er stand auf und murmelte noch: \enquote{Die Bilder werde ich nie wieder los.} Er schüttelte sich und lief die Stufen zum Gemeinschaftsraum hoch.

Hermine stand auf und kam zu ihm herüber. Da er in einem relativ großen Stuhl saß, setzte sie sich neben Harry, die Füße über seine und eine Armlehne des Stuhles legend. Sie legte ihren Kopf gegen die Rücklehne und schaute ihn mit wenigen Zentimetern Abstand an.

\enquote{Und nun erzähl, was los ist. Aber nicht die Geschichte für Ron und Malfoy, oder für den Rest von Hogwarts.} Sie war ihm dermaßen nahe, dass er sich anstrengen musste, um nicht offensichtlich und schuldbewusst zu schlucken. Misstrauisch schaute er sie an. \enquote{Erzähl mir keine Märchen Harry\abs}, sie küsste sanft seine Nase. \enquote{Ich weiß, dass zwischen euch Zweien was läuft.}

Harry blieb standhaft, doch es schien nicht zu helfen. Sie wurde forscher und kam seinem Gesicht näher.

\begin{rueckblick}
\enquote{Wissen Sie Mister Potter. Auch an uns geht der Effekt, den Sie auf Ihre Mitschülerinnen haben, nicht spurlos vorbei}, sagte Madame Pomfrey, als sie ihm seinen zweiten Becher einschenkte.

Als er ihn trank, fiel sein Blick direkt in Madame Pomfreys Augen. Sie hatte einen Blick an sich, den er noch nie gesehen hatte. \gedanke{Sie muss sich konzentrieren, um mir nicht näherzukommen}, dachte er. In ihm schwelte ein Plan. \accentuate{Sollte er es wagen? Nur ein kleines Intermezzo? Ein kleiner Kuss mit seiner Krankenschwester?} Er schloss seine Augen und schüttelte sich innerlich. Einerseits war der Gedanke faszinierend, eine Art Selkie-Charme zu haben, andererseits bedrückend, da ihm dauernd Mädchen nachliefen und sich an ihn schmiegten oder ihn zu küssen versuchten. Harry hatte bei vielen nichts dagegen und nach einigen anstrengenden Versuchen, sie abzuwehren, resignierte er nach einiger Zeit und wehrte seine Mädels, wie er sie nun nannte, nur noch halbherzig ab. (Selkies sind eine Art Wassermenschen, die ihren Charme spielen lassen können, um ihr Gegenüber sexuell zu erregen.) Harry konnte sich seine Anzugskraft nicht aussuchen. Sie schwankte. Und er konnte sie nicht abstellen. Zu allem Überfluss beherrschten ihn seine Gefühle, sodass er immer wieder die Kontrolle verlor.
\end{rueckblick}

Hermines Gesicht kam seinem Näher, bis sie seine Haut berührte. Ihre Lippen berührten seine und sie gab ihm einen langen Kuss. Vor diesem Moment hatte er sich die vergangenen Wochen am meisten gefürchtet. Einerseits war er stolz auf Hermine, dass sie es so lange aushielt.

\begin{rueckblick}
Die anderen Gryffindor-Mädchen in seinem Jahrgang hatten schon nach zwei oder drei Tagen aufgegeben und waren seinem Charme erlegen. Seitdem ließen sie ihn nicht mehr in Ruhe, bis er ihnen zumindest einmal nachgab und ausgiebig mit ihnen knutschte. Danach war es besser geworden.
\end{rueckblick}

Andererseits bedauerte er es, dass es so lange dauerte, bis Hermine ihn endlich küsste, bzw. Harry endlich Hermine küssen durfte. Er stellte sich schon längere Zeit vor, wie es wäre, Hermine zu küssen, doch er hatte auch etwas Bammel, da sie ja mit Ron zusammen war, seinem besten Freund. Er gab sich ihrem Kuss hin und genoss jede Sekunde. Seinen Zustand hatte Harry bei ihr recht schnell unter Kontrolle. Doch immer wieder wallte er kurz auf. Er brach den Kuss und meinte: \enquote{Hermine, was meinst du.}

\enquote{Ich meine du und Pansy. Ich glaube Ron nicht. Ihr habt nicht nur aus Show geknutscht. Da ist etwas zwischen euch.}

\enquote{Du weißt doch, Hermine, dass mir zurzeit kein Mädchen widerstehen kann.} Er gab ihr einen sanften Kuss auf ihren Mund, den sie umgehend erwiderte.

\enquote{Ja ich weiß}, sagte sie leicht betrübt. Sie war noch immer an ihn geschmiegt. Harry fühlte sich geborgen. So nahe bei Hermine zu sein. Sie schien eine Selbstkontrolle zu haben, die es ihr ermöglichte, selbst unter diesen Umständen eine einigermaßen normale Unterhaltung mit ihm zu führen. \enquote{Aber ich spüre, dass da mehr ist.} Sie blickte ihn mit einem durchbohrenden Blick an. \enquote{Du empfindest etwas für sie, habe ich recht?}

Er konnte nicht anders. Er küsste sie wieder und nickte dann kaum merklich und stumm.

Hermine seufzte leicht und meinte dann: \enquote{Ich habe es mir gedacht. Ihr beide empfindet etwas füreinander. Ich habe euch beobachtet, wenn ihr euch unbeobachtet fühltet. Ihr habt vielsagende Blicke ausgetauscht.}

Harry hob eine Augenbraue. \enquote{Und du bist dir sicher, dass\abs} Er küsste ihre Nase, doch sie zog ihn wieder an sich heran und küsste ihn. Er öffnete leicht seinen Mund.

Sofort begann sie seine Lippen und Zähne zu umspielen. Ihre Hände umschlossen seinen Kopf und griffen in sein Haar. Nach einem langen Zungenkuss löste sie sich wieder von ihm und sagte: \enquote{Du machst es mir nicht leicht, Harry.}

\enquote{Du mir auch nicht, wenn du so nah bei und auf mir sitzt.} Sie grinste ihn an und setzt sich nun neben ihn. \enquote{Mir scheint \gst } fuhr Harry sichtlich wohler fort, \enquote{aber erzähl Ron nichts davon \gst dass Pansy schon vor meinem Zustand etwas für mich empfand. Mein Zustand war nur der Auslöser. Wir sind jetzt schon seit wenigen Wochen zusammen. Seitdem ist sie die Einzige, mit der ich mich wie mit dir unterhalten kann. Sie hat mehr Selbstkontrolle als all die anderen. Noch etwas mehr als du.}

Hermine schaute ihn mit großen Augen an.

\trenn

Eine Woche später nahm sich Harry seinen Würfel an den See. Lavender hatte er die letzte Woche gemieden, wissend, dass er es dadurch noch verschlimmern würde, aber er brauchte die Zeit für sich, um sich darauf vorzubereiten. Die paar Tage würden es, sagte man ihm, nicht sonderlich erschweren. Er tauchte seine nackten Füße in das klare Nass, nahm seinen Würfel aus der Tasche und betrachtete ihn. Er sah den schwarzen Würfel an. Er betrachtete ihn lange. Dann entschloss er sich, sich auf einen flachen Stein zu setzen, der aus dem Wasser ragte.

Stumm sah er zwischen der leicht gekräuselten Wasseroberfläche und dem Würfel hin und her. Plötzlich bemerkte er, dass sich auf der Wasseroberfläche etwas tat. Ein Meereslebewesen streckte seinen Kopf aus dem Wasser und begrüßte Harry. Doch dieser verstand kein Wort.

\enquote{Ich verstehe sie nicht}, sagte Harry.

\enquote{Verzeihung}, antwortete das Wesen. Es schwamm näher. Erst jetzt konnte er es richtig erkennen. Es schien ein Weibchen zu sein. Sie sprach leicht gebrochen, aber Harry verstand sie trotzdem gut. \enquote{Bedrückt Sie irgendetwas?}, fragte sie Harry.

\enquote{Ja, ich habe hier einen Würfel bekommen und soll ihn öffnen, habe aber keine Ahnung wie ich es anstellen soll.} Er wollte nichts von Lavender und sich erzählen.

Sie schwamm noch etwas näher, bis er die kleinen Unregelmäßigkeiten in ihren Augen erkennen konnte. Sie hatte wunderbare Augen. Ihre Iris leuchtete so gelb wie die Sonne. Sie war sehr hübsch, fand Harry.

\enquote{Mein Name ist Chwalla}, sprach sie.

\enquote{Ich heiße Harry}, gab Harry zurück.

Ihr Blick fiel auf seinen Würfel. \enquote{Darf ich mal sehen?}, fragte sie ihn.

\enquote{Ja}, antwortete er und gab ihn ihr in die Hand.

Interessiert betrachtete sie den Würfel. \enquote{So einen habe ich schon mal gesehen. Unser König hat so einen, glaube ich.}

\enquote{Kann er mir helfen, ihn zu öffnen?}, fragte Harry.

\enquote{Keine Ahnung. Komm einfach mit und frage ihn.}

\enquote{Aber ich kann unter Wasser nicht Atmen.}

\enquote{Das hast du aber schon, als du die Prüfung im See absolviert hast.}

\enquote{Damals hatte ich aber Dianthuskraut.}

\enquote{Das ist kein Problem.} Sie verschwand, kam aber kurz darauf mit einem Bündel Dianthuskraut in der Hand zurück. Sie streckte ihm die Pflanzen entgegen und deute ihm an, er möge sie schlucken.

Harry war leicht mulmig, aber er tat wie ihm geheißen. Er kannte das Gefühl. Als er die Pflanzen herunterschluckte, spürte er ein leichtes Kribbeln in Hals. Er entledigte sich seiner Oberbekleidung und ging weiter in den See hinein, um sich zum Tauchen vorzubereiten.

\enquote{Warte}, sagte Chwalla. \enquote{Du kannst so noch nicht zum König gehen.}

\enquote{Soll ich etwa mit meinen Klamotten in den See?}, fragte er sie.

\enquote{Nein, das meinte ich nicht.} Sie kam ihn näher. Er sah ihr wieder in ihre wunderbaren Augen. Jetzt war sie nur noch eine Nasenspitze von ihm entfernt. Sie roch angenehm. Harry begann leicht zu schwitzen. Ihm wurde warm. \enquote{Ich meine, unser König möchte zuerst in seiner Sprache begrüßt werden, bevor er sich mit Außenstehenden unterhält.}

\enquote{Aber ich kann kein Meerisch}, antwortete Harry.

\enquote{Das ist kein Problem, ich bringe es dir bei. Bist du bereit?}

Ganz erstaunt darüber gab er reflexartig: \enquote{Ja}, zur Antwort.

Sie kam ihm noch näher, bis ihre Nasenspitze die seine berührte. Sie drehte ihren Kopf leicht zur Seite und zog seinen Kopf zu sich. Dann küsste sie ihn. Ihre Lippen fühlten sich kalt aber trocken an. Es war richtig angenehm. Sie löste sich wieder von ihm und entfernte sich etwas. Harrys Herz pochte. Sie sagte zu ihm auf Meerisch etwas und Harry antwortete ebenfalls in Meerisch. \meerisch{Siehst du, es geht doch. Komm mit.} Sie drehte sich um und tauchte unter.

Harry folgte ihr. Dank des Dianthuskrauts konnte er unter Wasser atmen. Es dauerte knappe 10~Minuten, bis sie in der Stadt unter Wasser angekommen waren. Auf den Weg dorthin dachte er an Chwallas Kuss. So etwas hatte er noch nie erlebt. Er war immer noch aufgeregt.

Sie führte ihn zum König, der sich gerade mit seiner Frau unterhielt, wie Harry feststellen konnte. \meerisch{Majestät! Ich möchte ihnen etwas vorstellen. Ein Wesen von außerhalb unseres Reiches. Er nennt sich Harry und ist ein Männchen.}

\gedanke{Etwas}, dachte Harry. \gedanke{Sie hat mich erst als sächlich bezeichnet, bevor sie mich mit Namen und dem Geschlecht vorstellte.}

Der König drehte sich zu Harry. Harry war sich nicht sicher, ob er den König ansprechen sollte, entschied sich aber, es nicht zu tun. Aus dem Augenwinkel heraus sah er Chwalla. An ihrem Ausdruck meinte er lesen zu können, dass er warten solle, bis er angesprochen wurde. Harry verbeugte sich vor dem König, um ihm seine Ehrerbietung zu zeigen. So wie er es bei Seidenschnabel gelernt hatte.

Der König schwamm auf Harry zu. \meerisch{Nun, was wünschst du von mir?}, fragte ihn der König. Harry begrüßte den König, wie ihm Chwalla sagte, zuerst in seiner Sprache. Dann erst trug er seinen Wunsch vor.

\meerisch{Majestät, ich wüsste gerne etwas über einen kleinen Würfel. Ich habe einen bekommen und mir wurde gesagt, ich solle ihn öffnen.} Er griff in seine Tasche und zog seinen Würfel heraus.

Der König bekam große Augen. \meerisch{Wachen, haltet ihn fest.} Sofort schwammen weitere Meereslebewesen heran, die Harry mit ihren Speeren in Schach hielten. \meerisch{Hofmeister! Schauen Sie in der Schatzkammer nach, ob der schwarze Würfel noch da ist.} Der Hofmeister kam kurz heran und nickte. Er verschwand. \meerisch{Du bleibst so lange hier, bis mein Hofmeister festgestellt hat, dass du den Würfel nicht gestohlen hast.}

Harry verstand und nickte. Der König nahm ihm den Würfel ab und betrachtete ihn. Kurze Zeit später kam der Hofmeister angeschwommen, den Würfel in der Hand haltend. Er übergab dem König den Würfel und wartete auf weitere Anweisungen. Der König verglich sorgsam alle 6 Seiten der beiden Würfel. Sie sahen identisch aus. Der König gab dem Hofmeister seinen Würfel wieder und gab ihm durch eine Geste zu verstehen, er möge ihn wieder zurückbringen. Die Wache verzog sich und Harry war frei. \meerisch{Komm mit}, sagte der König zu Harry und schwamm Richtung Thron.

\meerisch{Ich sehe du wurdest von Chwalla unterrichtet, mich zuerst in meiner Sprache anzusprechen. Du musst wissen, Chwalla ist meine Tochter.} Harry schluckte. Der König gab Harry seinen Würfel zurück. \meerisch{Wie man ihn öffnet weiß ich auch nicht. Aber ich kann dir sagen, dass der Würfel mit einer Farbschicht überzogen ist. Die musst du zuerst entfernen. \gst Möchtest du noch etwas wissen?}, fragte ihn der König.

\meerisch{Danke Majestät. Das ist alles. Ich hatte mir zwar mehr erhofft, aber ich bin mit dem zufrieden, was ich erfahren habe.} Er verbeugte sich wieder vor dem König und bat ihn in seiner Sprache sich entfernen zu dürfen. Der König lächelte und nickte stumm.

Harry schwamm zurück an die Oberfläche, begleitet von seiner neuen Freundin Chwalla. Mit dem Kopf über Wasser kam ihm Chwalla wieder näher und küsste ihn erneut. \enquote{Wofür war der denn?}, fragte Harry.

\enquote{Jetzt kannst du nicht mehr unsere Sprache sprechen.}

\enquote{Schade}, sagte Harry. \enquote{Ich mochte es. Kann ich diese Fähigkeit nicht behalten?}

Sie lächelte ihn an. \enquote{Gib es doch zu. Du willst nur wieder einen Kuss von mir.}

Harry war der Gedanke nicht ganz unangenehm. \enquote{Zugegeben, das auch. Aber Meerisch zu können hat schon Vorteile. Außerdem kann ich mich dann weiterhin mit eurer Spezies unterhalten.}

Sie lächelte ihn erneut an und kam ihm wieder näher. Harrys Herzschlag erhöhte sich wieder. Sie küsste ihn ein drittes Mal. Doch dieser Kuss war um einiges besser, als die beiden zuvor. Sie ließ wieder von ihm ab. \meerisch{Mach’s gut Harry}, sagte sie und begann unterzutauchen.

\meerisch{Du auch}, gab Harry zurück. Dann war sie unter der Wasseroberfläche verschwunden. Harry sah ihr noch einige Meter nach, bis er sie unter der rauen Oberfläche nicht mehr wahrnahm.

Verträumt sah er auf die Wasseroberfläche. Er nahm nicht wahr, dass sich ihm jemand näherte. \enquote{War das Bad mit deiner neuen Freundin entspannend?}

Erschrocken drehte sich Harry um und sah Professor Dumbledore, der mit seinen nackten Füßen im Wasser stand. \enquote{Meiner neuen? Nein Albus, das ist nicht meine Freundin.}

\enquote{Sah aber so aus, als ihr euch geküsst habt.}

\enquote{Oh das. Da hat sie mir Meerisch beigebracht.} Und Harry sagte Professor Dumbledore auf Meerisch etwas.

Mit erhobenen Augenbrauen und Blick auf das Wasser sagte er: \enquote{Und ich habe das umständlich lernen müssen.}

Harry musste sich ein Grinsen verkneifen.

Harry ging an Dumbledore vorbei, legte seinen Würfel auf einen Stein und fing an, sich wieder anzuziehen, denn eigenartigerweise war seine Unterhose nicht nass. Als er fertig war, hatte Dumbledore bereits den Würfel in der Hand und betrachtete ihn mit wachsendem Interesse. Er gab ihn Harry zurück.

\enquote{Wo hast du denn den her?}, fragte ihn Professor Dumbledore.

\enquote{Den haben wir von Professor Elber erhalten. Wir sollen ihn als Hausaufgabe öffnen.}

\enquote{Ein Rätsel also?}

\enquote{Ja}, antwortete Harry.

\enquote{Ich glaube, ich sollte ihn fragen, ob ich auch einen bekomme}, antwortete Dumbledore.

\enquote{Sie liegen im Klassenzimmer in einer Schachtel. Wenn uns einer kaputtgeht, sollen wir einfach einen neuen holen. Ich denke, du kannst dir einfach einen nehmen.}

\trenn

Er war noch immer mit Pansy zusammen und hatte sich mit Hermine und ihr getroffen, um Pansy zu sagen, dass Hermine es seit einer Weile wisse. Pansy machte erst ein erschrockenes Gesicht, beruhigte sich aber sehr schnell. Er war mit Pansy übereingekommen, sich in Zukunft nicht mehr zurückzuhalten. Er schmiedete mit ihr einen Plan, den er selbst Hermine nicht verraten wollte. Er sagte ihr nur: \enquote{Warte heute das Abendessen ab.}

Sie saß in der Großen Halle mit Blick zur Tür, damit sie, was auch immer kommen mochte, gleich sehen konnte. Harry bog zur genannten Zeit um die Ecke, Hand in Hand mit Pansy. Hermine lächelte leicht und blickte kurz zum Lehrertisch. Sie sah in einige große Augen, die die beiden ebenfalls entdeckt hatten. Sie blickte schnell wieder zurück und hörte auch schon die ersten Stöhner und ein plötzlich anschwellendes Getuschel in der Großen Halle. Als sie die nötige Aufmerksamkeit erreicht hatten, drehten sich beide zueinander, schlangen ihre Arme um den anderen und küssten sich innig. Sie versanken wieder ineinander. Ihre Münder klebten förmlich aneinander. Sie öffnete leicht ihren Mund und Harry spielte mit ihren Zähnen und ihrer Zunge. Er hörte ein Schnaufen und Stöhnen. Er löste sich von ihr und grinste sie an. \enquote{Bis nachher}, sagte er, gab ihr einen kleinen Kuss auf die Wange und setzt sich an seinen Platz gegenüber von Hermine. Er nahm seine Gabel in die Hand und stach zu. So als wäre nichts Besonderes gewesen, unterhielt er sich mit Hermine über seinen Lernplan für das kommende Wochenende. Er genoss die schockierten Blicke der anderen.

\enquote{Du hast gerade eben\abs}, stammelte Neville, \enquote{mit Pansy, Pansy Parkinson geknutscht. Sie ist eine Slytherin.}

Harry drehte sich um und sah Neville an. Aus seinem Augenwinkel heraus sah er viele Mitschüler nickend.

\enquote{Genau}, meinte Dean.

\enquote{Was soll schon dabei sein, seine Freundin zu küssen}, antwortete Harry.

Neville war sprachlos. \enquote{Freundin?}, fragte Dean ganz erstaunt. \enquote{Du kannst doch nicht\abs}

Harry erhob sich und sagte, zu allen an seinem Tisch gewandt: \enquote{Und was ist so schlimm daran, eine Freundin zu haben, die aus Slytherin ist? Wir lieben uns und es ist mir verdammt nochmal egal, aus welchem Haus sie stammt. Sie ist das Beste, was mir in meinen ganzen Jahren hier in Hogwarts passiert ist.} Er blickte zu ihr und sie lächelte ihn an. \enquote{Und ganz im Gegenteil, ich bin froh, dass es jemand aus Slytherin ist. Ich habe es satt: diese ständigen Vorurteile und das Gezanke zwischen unseren Häusern!} Er setzte sich wieder.

Hermine grinste ihn an und sagte dann laut: \enquote{Genauso, Harry}, und fing an zu klatschen.

Nach ein paar Sekunden des Klatschens stimmte Professor Dumbledore am Lehrertisch und einige andere Lehrer ebenfalls mit ein. Es kamen noch vereinzelte Schüler hinzu, die dem Beifall folgten.

Noch niemals hatte neben Umbridge jemand so wenig Applaus bekommen wie Harry. Aber das war ihm egal. Er hatte eine Freundin, die ihn liebte. Und das machte ihn glücklich! Aber war er das mit Pansy wirklich?

Noch am selben Tag nahm ihn sich Ginny zur Seite, um etwas mit ihm zu klären.

\enquote{Was läuft da zwischen dir und Pansy? Ist das etwas Ernstes? Ich dachte, du liebst mich?} Als sie den letzten Satz gesprochen hatte, lief sie rot an.

Harry kam zu ihr, nahm sie zunächst in seine Arme und setzte sich dann mit ihr auf seinem Schoß hin. \enquote{Ja, das tue ich, Ginny. Aber im Moment liebe ich auch Pansy. Ich weiß nicht, was werden wird. Aber das Wichtigste ist für mich, dass ich jemanden habe, mit dem ich glücklich bin.} Harry ließ bewusst offen, wen er genau damit meinte. \enquote{Einerseits bin ich glücklich, andererseits stärkt dies die Beziehungen zwischen den Häusern, aber das Wichtigste ist, dass ich weiß, dass du da sein wirst, wenn das mit Pansy vorbei ist.} Harry hatte das sehr rational zu Ginny gesagt, obwohl sehr viel Gefühl in diesen Sätzen von ihm steckten. Ebenso in seinen Gedanken, die er indirekt herüberbrachte.

Ginny gab sich damit vorläufig zufrieden. Sie würde sich Harry später noch einmal vornehmen, falls diese Geschichte mit Pansy nicht schnell genug vorüber war. Wie es allerdings in Harry aussah, konnte auch sie sich nicht vorstellen.

\trenn

\enquote{Miss Delacour kommt heute mit ihrer Schwester zu Besuch. Sie möchte etwas von Harry wissen und ihm eventuell einen Besuch abstatten}, sagte Professor McGonagall.

\enquote{Fleur Delacour? Halb Veela?}, fragte Professor Elber.

\enquote{Ja, wieso?}

\enquote{Sie darf die Schwelle zu Hogwarts nicht übertreten, solange auf Harry und Lavender noch diese Kraft wirkt. Wenn sie das tut, dann könnte sich die Wirkung vervielfachen \gst oder ins genaue Gegenteil verkehren. \gst Und \gst stellen Sie sich vor, dass ihre jüngere Schwester\abs egal was passiert, es könnte um ein Hundertfaches schlimmer werden. \gst Wann kommen sie an?}

\enquote{Sie sind schon unterwegs.}

Professor Elber drehte sich um und rannte zum Schlosseingang. Er konnte sie gerade noch abfangen und am Betreten des Schlosses hindern. Harry saß in der Großen Halle, als sie zu dritt den Saal betraten. Harry war darüber verwundert.

\enquote{'allo 'arry. Isch freue misch dich zu seh'n. Schlimm, was mit dir passiert ist. Isch 'abe misch, genau wie Gabrielle, unter Kontrolle.}

Dann küsste sie ihn zweimal auf beide Wangen. \enquote{Dein Professor hat mir Bescheid gesagt. Meine Veela-Part 'abe isch unter Kontrolle.}

\enquote{Wie?}, fragte Harry.

\enquote{Wenn isch mich konze'triere, dann kann isch meine Veela-Erbe zurück'alten. Gabrielle kann das nich' so gut, deshalb bin isch ihr immer nahe. Um sie und disch zu schü'zen.}

Ihr Gesicht kam seinem näher. Fleur schloss ihre Augen und konzentrierte sich. Dann hielt sie an und öffnete sie wieder.

\enquote{Es ist schwer. Gabrielle und isch sind noch einige Tage hier. Wir werden später wieder kommen Arry.}

Fleur küsste Harry zum Abschied auf gewohnte Weise. Er wollte zu Gabrielle schauen, die gerade seinem Gesicht nahe kam, und erwischte ihren Mund. Leicht errötend zog sie sich zurück und schaute schuldbewusst zu ihrer großen Schwester. Sie sagte zu ihr etwas, das Harry nicht verstand. Fleur kam seiner Wange nochmals näher und flüsterte ihm ins Ohr: \enquote{Gabrielle möchte sich später bei dir entschuldigen, für das, was gerade passiert ist, und dir nochmals danken, dass du sie gerettet hast. Und ich übrigens auch. Ich habe im Moment das Gefühl, dass ich dir nicht genug gedankt 'abe.} Dann fuhr sie mit ihrer Zungenspitze die inneren Ohrkonturen nach und verließ mit ihrer Schwester die Große Halle.

Ginny sah Harry eigenartig an. Er dachte, dass er einen Anflug von Eifersucht in ihren Augen erkennen konnte.

\trenn

Harry schlug Salazars Tagebuch auf, setzte sich in einen Sessel und begann zu lesen. Es war mehr eine Autobiografie mit persönlichen Gedanken, als ein Tagebuch. Es enthielt nur die wichtigsten Episoden und Gedanken aus seinem Leben.

\begin{buch}
Heute ging mal wieder alles durcheinander. Nicht nur, dass mein Bruder wieder alle nervte, nein, auch meine Schüler meinten, den Tränkeraum mit ihren stickigen Nebelwolken ausfüllen zu müssen, indem sie ihre Tränke explodieren ließen. Dabei könnte man doch meinen, sie müssten es inzwischen begriffen haben. Na ja, der schlimmste von ihnen hängt jetzt mal wieder an seinen Armgelenken im Kerker. Godric meint zwar, das sei eine viel zu schwere Strafe, aber ich sage ihm dann, dass er ein Weichei sei. Ich mag es zwar selber nicht so streng, aber dieser Peeves ist eine Plage.

Gestern hat uns unser Vater verlassen. Nach fünf Jahren, in denen er uns als Hausmeister gedient hat, ist er jetzt fort. Einerseits bin ich traurig darüber, andererseits aber auch froh.

Immer diese Muggelgeborenen. Sie haben es einfach nicht so drauf, wie jene, die aus Zaubererfamilien stammen. Manchmal glaube ich wirklich, dass es an den Genen liegt und die Konzentration der Magie in ihnen bedingt durch ihre Vorfahren nicht so stark ist, wie bei Familien, bei denen mindestens ein Mitglied magische Fähigkeiten hat. Darüber muss ich mal nachdenken. Vielleicht sollte man die Leute anhalten, nicht mehr so viel mit Muggeln zu heiraten.

Die dunklen Künste haben mich schon immer fasziniert. Nicht dass ich sie gerne anwende, oder Menschen damit schade. Es ist vielmehr, dass ich sämtliche Aspekte der Magie gerne kennenlerne und wissen möchte, was möglich ist. Schon die Wahl der Begriffe zeugt von der Engstirnigkeit der magischen Brigade. Als ob Magie eine Farbe hätte. Dabei können solche Zauber auch nützlich sein. Zumindest hat mich noch kein Argument vom Gegenteil überzeugen können.

Endlich Ferien. Zeit, mich meinen Studien zu widmen. Heute bin ich wieder einmal bei Nadine gewesen. Wenigstens sie hört mir zu. Auch wenn ich mich mit ihr nicht so recht unterhalten kann, wie mit anderen. Selbst meine Frau will nichts von den dunklen Bereichen der Magie wissen. Es schmerzt mich, dass ich selbst vor ihr all das geheim halten muss. Nadine meint, dass sie mich durchaus verstehen kann. Sie selbst gehört zu denjenigen, die nur Nachts nach draußen können, weil dann wenige Personen unterwegs sind.

Jetzt habe ich sie fertig. Meine Theorie über die magische Abstammung und Erhaltung unserer Art; wie sich unsere magischen Fähigkeiten stärken lassen können. Meine Versuche an magischen Mäusen hat es bestätigt. Die Verpaarung der besten magischen Fähigkeiten stärkt diese. Wenn wir also überleben wollen, dann müssen wir dafür sorgen, dass wenig Muggel einheiraten. Muggelgeborene Hexen und Zauberer stellen eine Bereicherung dar. Aber mit Muggeln zu heiraten, das würde uns auf Dauer nur schwächen.
\end{buch}


Harry legte das Buch weg. Er schloss kurz seine Augen und griff mit Daumen und Zeigefinger an seine Nasenwurzel. Dann stand er auf und musste sich etwas frisch machen. Nachdem er von der Toilette gekommen war, sich etwas kaltes Wasser über seine Handgelenke laufen ließ und sich mit einem kalten nassen Waschlappen den Nacken Abrieb, kehrte er zurück zu seinem Sessel, nahm das Buch und las weiter.

\begin{buch}
Nadine legte sich mir wieder um den Hals und hörte mir einfach nur zu. Mit ihrer, mit Riechknospen besetzten, Zunge fuhr sie an meiner Backe entlang. Sie beruhigte mich wieder einmal und bat mich, ihr wieder eine meiner Mäuse mitzubringen. Ich fuhr ihr über ihre trockene warme Haut und nickte.

Als ich nach meinen Mäusen sah, traf mich fast der Schlag. Alle waren magisch degeneriert. Ich untersuchte darauf hin alle, aber keine meiner Zöglinge hatte mehr magische Fähigkeiten. Die Generationen-lange Inzucht hatte ihren Tribut gefordert. Sie hatten nicht nur genetische Krankheiten, sondern auch ihre Magie wurde weniger und war verschwunden. Nadine konnte sie alle haben, nachdem ich meine Ergebnisse festgehalten hatte.

Der Rat war empört über meine Korrekturen und wusch mir gehörig den Kopf deswegen. Was mir einfiele, fragten sie mich. Ob ich von allen guten Geistern verlassen sein würde. Sie schlossen mich aus dem Verband aus. Da stand ich nun. Eine Veränderung in Gang gesetzt, die ich nicht mehr aufhalten konnte. Dazu hatte ich nicht mehr die Mittel. Der Schaden war angerichtet.

Meine Frau hatte wenig Verständnis für meine Lage. Sie hatte mich gleich zu Anfang gewarnt. Jetzt musste ich den Preis dafür zahlen. Mein Ansehen war weg. Ich brauchte ein Jahr, bis ich erkannte, dass ich so wenigstens ungeniert leben konnte. Ich hatte nichts mehr zu verlieren.

Ich kehrte also wieder nach Hogwarts zurück. Nach Jahren wieder einmal. Nur ab und an unterrichtete ich. Bei den muggel-stämmigen war ich nicht gerade beliebt, obwohl ich mich nicht anders verhalten habe, als all die anderen Lehrer. Zauberer und Hexen sind so was von nachtragend, da würde ich mir manchmal wünschen, ein Muggel zu sein. Die können scheinbar schneller verzeihen. Zum Glück war es meine Frau nicht.

Heute stellte ich meiner Frau Nadine vor. Nach und nach hatte ich ihr alle meine Geheimnisse enthüllt. Wenigstens mit ihr wollte ich meinen ganzen Frieden machen. Nun war ich bereit zu sterben. Meine Frau litt schon seit längerem an einer Krankheit, für die es keine Heilung gab. Da ich mein Schicksal mit dem ihrem verknüpft hatte, würde auch ich ihr kurze Zeit später nachfolgen.

Da stand ich jetzt, am Grab meiner Frau. In wenigen Stunden würde ich ihr folgen. Freiwillig. Ich sicherte meine Experimente, zerstörte eventuell gefährliche Sachen und verabschiedete mich von Nadine. Die Hauselfen bekamen Anweisungen, was sie zu tun hätten. Dann trat ich meinen letzten Gang an. Ein Hauself begleitete mich. Nur noch wenige Minuten trennten mich von meiner Frau. Ich legte mich neben sie in einen schlichten Holzsarg am Boden der Grube und ließ den Deckel herab schweben. Der Hauself begrub mich unter der dunklen Erde. Ich spürte Müdigkeit und glitt ins Reich der Träume\abs Ich hoffe, dass es sich so abspielt, denn während ich diese Zeilen schreibe, steht schon der Elf neben mir. Wer immer in der Lage ist, das zu lesen, möge mich hoffentlich verstehen.

Salazar Slytherin
\end{buch}

Die nächsten Tage verbrachte er immer wieder abends in Salazars Räumen, da ihn Agatha fragte, was er denn da für ein Buch lese. Als er ihr sagte, dass es um Gebärdensprache ging, bot sie ihm überraschend ihre Hilfe an, da eines ihrer Kinder nicht sprechen konnte und sie somit diese Sprache gelernt hatte. Es war zwar für die damalige Zeit ungewöhnlich, aber das hielt Agatha nicht davon ab.




\begin{kommentar}
Dieses Kapitel steht ganz im Zeichen um Harrys Anziehungskraft. Es existiert nur, weil ich etwas gesucht habe, dass Harry und Pansy für kurze Zeit zusammenbringt. Die Idee dazu kam mir, als ich eine andere Geschichte gelesen hatte.
\end{kommentar}

\begin{kommentar}
Am Ende dieses Kapitels geht Salazar seinem eigenen Tod entgegen. Für manche scheint es so, als ob er in seinem Sarg ersticken würde, aber Salazar selbst wusste genau, wann er sterben würde. Und so konnte er sich hineinlegen und den Deckel schließen, denn wenige Minuten später, die Luft hätte noch länger ausgereicht, starb er.
\end{kommentar}
