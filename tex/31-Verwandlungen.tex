\chapter{Verwandlungen}


Nach dem Frühstück ging Harry wie immer zu seiner Unterrichtsstunde in \VgddK. Am Klassenzimmer angekommen, fand er nur einen Zettel an der Tür. Darauf stand: \accentuate{Die Klasse findet heute im Innenhof mit dem Feenbrunnen statt. F. Elber}

Also machte sich die Klasse auf und ging zum Hof mit dem besagten Brunnen. Dort wartete bereits ihr Lehrer. Aber auch Flitwicks Fünftklässler, die er jetzt hatte, waren anwesend. Die Klasse betrat den Hof und Professor Elber fing an.

\enquote{Schön, schön. Sie haben hergefunden. Professor Flitwick lässt sich durch Madame Pomfrey entschuldigen. Er ist auf der Krankenstation beschäftigt. Deshalb habe ich kurzerhand Ihre Mitschüler mitgenommen, damit Sie gemeinsam etwas lernen. Das eigentliche Unterrichtsthema wird heute also anders aussehen. Sie werden sich in Gruppen von zwei oder vier Personen zusammenfinden. Aber klassenübergreifend bitte. Also je ein Fünft- und ein Sechstklässler, oder eben zwei aus jeder Jahrgangsstufe. \gst Ihre Aufgabe ist es, diesen Innenhof zu untersuchen. Lassen Sie sich nicht davon abbringen, wenn ich irgendwo sitze oder stehe. Vertreiben Sie mich, wenn Sie die Stelle, an der ich mich befinde, untersuchen wollen. \gst Folgendes Szenario stellen Sie sich bitte vor: Sie sind diesen Gang\abs}, er zeigte kurz auf einen Gang, der auf den Hof zulief und in einer Biegung dann auf den Innenhof führte, \enquote{\aabs entlang gesprungen und verfolgen zusammen mit Ihren Kollegen einen Dieb. Dieser hatte, bevor er geflohen war, einen Beutel in der Hand und bog in diesen Hof ein. Knapp eine halbe Minute später kamen Sie auch dort an und der Dieb saß auf einer Bank, hatte aber keine Beute mehr bei sich. Einer Ihrer Kollegen nahm ihn bereits mit zum Verhör. Der Rest \gst also Sie \gst müssen nun versuchen, das Versteck aufzuspüren. Fangen Sie an.}

Er trat zurück und setzte sich auf eine Bank in der Nähe.

Alle Schüler standen erst einmal perplex herum, bevor sie sich fingen und zaghaft kleine Gruppen bildeten. Harry tat sich mit Neville aus seiner Klasse, sowie mit Sirin und Klaus aus der Jahrgangsstufe unter ihm zusammen. Es waren Hufflepuffs. Er kannte beide leidlich, und stellte sich daher vor \gst genauso wie Neville \gst und lernte dadurch ihre Namen (wieder), da sie sich ebenfalls vorstellten.

Er fragte sich, wie die jüngeren Zauberer und Hexen dabei helfen konnten. Während er eine Seite des Hofes untersuchte, sahen sich die anderen aus seiner Gruppe ihre Seiten an. Nach zehn Minuten trafen sie sich wieder und fingen an sich auszutauschen.

\enquote{Nichts gefunden}, merkte Klaus an.

\enquote{Bei mir auch nichts}, sagte Neville.

Sirin nickte nur. Ebenso Harry.

Sirin sah sich erneut im Hof um. Dann wanderte ihr Blick nach oben. Auf einem Vorsprung sah sie etwas, das so aussah, als ob es dort fremd war. Sie schwang ihren Zauberstab und sprach: \zauber{Wingardium Leviosa.} Ein Koffer hob sich leicht an und schwebte sanft auf die Erde.

Dies weckte die Aufmerksamkeit einiger anderer Schüler. Neugierig kamen sie etwas näher, behielten aber respektvollen Abstand.

\zauber{Alohomora}, sprach sie und der Deckel des Koffers sprang auf.

Eine spärlich gekleidete Frau mit vier Armen und zwei Beinen stieg aus dem Koffer und ging auf Sirin zu. Es war Shiva, eine hinduistische Göttin. Mit vor Schreck bleichem Gesicht stolperte sie rückwärts, bis sich Harry zwischen sie und Shiva schmiss und sie sich zu verwandeln begann. Nach kurzem erschien ein Dementor und Harry beschwor instinktiv einen Patronus hervor. Dieser galoppierte auf den vermeintlichen Dementor zu. Dann besann sich Harry eines besseren und sprach \zauber{Riddikulus} und der Dementor löste sich in einer kleinen Rauchwolke auf, die auf den Koffer zu schwebte, dessen Deckel sich schloss.

Jetzt meldete sich wieder Professor Elber zu Wort. \enquote{Nette Idee. Fünf Punkte für jedes der beiden Häuser. Aber der Koffer war es nicht, er ist zu groß.} Der Koffer schwebte wieder auf den Vorsprung zurück und die Suche ging weiter.

Harry dachte nach. Da meldete sich Klaus. \enquote{Wir haben noch nicht die Bank und den Brunnen untersucht}, stellte er sachlich fest.

\gedanke{Der Brunnen. Natürlich}, dachte Harry. \gedanke{Sonst fließt doch immer Wasser durch die Hörner, welche die männlichen Feen halten.}

Er winkte seine Gruppe zu sich und flüsterte ihnen seine Erkenntnis zu.

\enquote{Das macht Sinn}, merkte Neville an. \enquote{Es könnte unter dem Brunnen sein. Sollen wir ihn anheben?}, fragte er.

\enquote{Vielleicht ist er gesichert}, meinte Sirin. \enquote{Wir sollten ihn erst untersuchen.}

So wurde es auch beschlossen und die Gruppe besah sich den Brunnen genauer. Sie untersuchten ihn intensiv und kamen zu dem Schluss, dass der Brunnen an hebbar sei. Harry und Neville wurde die Aufgabe zugetragen, den Brunnen anzuheben. Nachdem dieser in der Luft geschwebt hatte, holte Klaus den Stoffbeutel zu sich, der darunter in einer Kuhle im Boden lag. Dann schwebte der Brunnen wieder nach unten und das Wasser floss durch die Füllhörner, welche die Feen hielten.

Der Beutel wurde magisch geöffnet und ein kleines bronzenes Kästchen kam zum Vorschein. Klaus hielt es in seinen Händen und wollte es gerade öffnen, doch Sirin hielt ihn davon ab. \enquote{Halt, Klaus. Denk an den Koffer. Stell es lieber hin}, sagte sie.

Klaus stellte das Kästchen auf den Boden und öffnete es magisch. Es lag eine Pergamentrolle darin. Neville bückte sich und holte sie heraus. Dann entrollte er sie und las vor.

\begin{brief}
Liebes Gewinnerteam,

herzlichen Glückwunsch zum Sieg.

Als kleine Belohnung dürft ihr euch an einem bestimmen Ort im Schloss eine Erholung gönnen. Diesen Sonntag könnt ihr euch vergnügen. Genießt euren Preis und denkt daran: Fasst alle das Pergament an, damit ihr von dem Zauber, den es umgibt, geleitet werdet.
\end{brief}

Sofort fassten drei weitere Hände das Pergament an, welches nach einigen Sekunden anfing zu glühen und dann im Nichts verschwand.

\enquote{Der Unterricht ist beendet}, sagte Professor Elber und lächelte. Nach zehn Minuten klingelte es und zeigte das Ende der Stunde und somit die Vorbereitung für die nächste an.

\trenn

Die Tage vergingen ohne Ergebnis für Alina. Sie wurde noch immer zu einer Harpyie. Mr. Ollivander konnte den Zauberstab reparieren und auch den Zauber ausfindig machen, der die Verwandlung verursachte, aber alle Mühen in der Bibliothek \gst das stundenlange Durchsuchen der Bücher \gst brachte nichts.

\stimme{Harry, du bist doch von m\aabs Professor Elber unterrichtet worden?}, fragte ihn Salazar.

\gedanke{Ja}, antwortete Harry während der abendlichen Astronomievorlesung.

\stimme{Du könntest es schaffen, die Harpyie zu bezwingen, sie von Voldemorts Fluch zu befreien. Du bist derjenige, der eine Verbindung zu ihm hat, verfügst über einen Teil seiner Macht. Zusammen mit deiner eigenen Magie wird es dir möglich sein, den Fluch zu brechen. Du musst allerdings allein mit der Harpyie sein und darfst den Zauber erst anwenden, wenn du von den anderen abgeschirmt bist. Die Querschläger könnten andere verletzen oder sogar töten.}

\enquote{Mister Potter, nicht träumen. Markieren Sie die Planetenpositionen auf Ihrer Karte}, ermahnte ihn Professor Sinistra.

\enquote{Ja Ma'am.}

Also begann Harry durch das Fernrohr zu schauen und die Positionen der Planeten zu erkunden. Er trug Planet um Planet ein, schaute in Tabellen nach einer Referenz für deren Bedeutung, während sie an bestimmten Positionen standen, und zeichnete die Gesichter der Planeten.

Eine einzelne Sternschnuppe durchquerte seine Sicht. Er wünschte sich etwas.

\trenn

Sonntagnachmittag umgab Harry ein eigenartig unbeschwertes Gefühl der Leichtigkeit. Er hatte das Gefühl, als stünde er unter dem Imperius. Doch er sah keinen Grund dagegen anzukämpfen, da er wusste, dass ihn ein Zauber heute leiten würde. Er verabschiedete sich aus dem Gemeinschaftsraum und lief in einen Bereich des Schlosses, der unbenutzt war. Auf seinem Weg lief er an staubigen Rüstungen und ziemlich alt aussehenden Bildern vorbei. Er begegnete einigen Hauselfen, die beschäftigt waren die staubigen Rüstungen zu säubern, damit sie nicht gänzlich unter einer Staubschicht verschwanden.

Die Elfen wollten gerade verschwinden, da sie Harry sahen, aber er hielt sie mit einem: \enquote{Lasst euch nicht stören}, davon ab. Er ging vor den Elfen in die Hocke und sagte dann: \enquote{Ich möchte euch danken, dass ihr uns täglich helft. Unsere Betten macht, unsere Kleidung wascht, bügelt und sauber in unsere Zimmer legt. Danke, dass ihr uns im Winter Wärmekissen in das Bett legt und die Feuer in den Kaminen am Brennen haltet. Danke, dass ihr für uns kocht und putzt.} Dann streckte er seine Hand aus.

Die Elfen sahen verschreckt aus. Noch nie hatte sich jemand bei ihnen bedankt. Einigen von ihnen liefen sogar Tränen die Wangen herunter.

Harry griff in seine Tasche und fing an, den Elfen die Tränen von ihren Gesichtern zu wischen. Jetzt bewegten sie sich noch weniger.

\enquote{Alles in Ordnung mit euch?}, fragte Harry vorsichtig nach.

Eine junge Elfe kam auf ihn zu und meinte: \enquote{Danke, Sir. Noch nie hat jemand so etwas zu uns gesagt. Wir Elfen fühlen uns geehrt, von so einem jungen und mächtigen Zauberer beachtet zu werden.}

Harry lächelte die Elfen an.

\enquote{Wir müssen wieder an unsere Arbeit, Sir}, sagte ein anderer älterer männlicher Elf.

Harry nickte, stand auf und lief weiter. Der Zauber führte seinen Weg und Harry drehte sich noch einmal um, um den Elfen zuzusehen.

Nach guten zehn Minuten stand er vor einem Gemälde, das sich scheinbar nicht bewegte und der Zauber, welcher ihn führte, fiel von ihm ab. Er hatte wieder das Gefühl, ganz Herr seiner Sinne zu sein. Er stellte sich die Frage, ob er unter dem Zauber so regiert hatte. Doch er kam zu dem Entschluss, dass dieser Anblick den Zauber kurz zurückweichen ließ und ihm seinen Willen \gst sich bei den Elfen zu bedanken \gst ließ.

Das Gemälde zeigte eine große Blumenvase mit vier Blumen an. Jede der Blumen war gleich gemalt. Nur waren ihre Blütenblätter in unterschiedlichen Farben gemalt. Sie waren in grün, blau, gelb und rot gehalten. Die Vase schimmerte, als wäre sie aus echtem Zinn. Die Farbe hatte metallische Pigmente, die sie schimmern ließ. Jetzt bekam er wieder einen Impuls, der ihm zuflüsterte: \stimme{Drücke die blaue Blume.} Doch das war leichter gesagt als getan. Selbst wenn er hochspringen würde, konnte er die Blütenblätter nicht erreichen. Und selbst wenn er den Hinweis bis zum äußersten ausreizte und nur den untersten Teil des Stils erreichen würde, der aus der Vase herausragte, er konnte ihn nicht erreichen. Er zog seinen Zauberstab in der Absicht, ihn auf die Pflanze zu werfen, als ihn wieder die Stimme in seinem Kopf unterbrach. \stimme{Doofer Zauberer}, hörte er in seinem Kopf.

\fluestern{Zauberer}, murmelte Harry vor sich hin. Er richtete seinen Zauberstab auf die blaue Blume und führte einen ungesagten und leichten \spruch{Protego} aus, der auf die Blume drückte.

Er hörte ein leises \geraeusch{Klack} und das Bild bewegte sich auf einer Seite wenige Millimeter von der Wand weg, als ob ein Verschluss aufsprang. Er öffnete das Bild und stieg hindurch. Das Gemälde fiel hinter ihm sofort wieder ins Schloss. Er stand in einem Gang, zu dessen linker Seite Spinde standen. Einer davon stand offen. Harry ging auf ihn zu und entdeckte einen Kleiderbügel und einige Fächer. Er zog seine Schulrobe aus und hängte sie vorsichtig auf den Kleiderbügel. Dann öffnete er seine Krawatte und legte sie in eines der Fächer. Danach knöpfte er sein Hemd auf und entledigte sich auch seiner restlichen Kleidung. Ohne etwas an seinem Körper ging er den Gang weiter und bog nach links ab.

Staunend blieb er stehen, als er das üppige Bad sah. Es war fast doppelt so groß, wie das der Vertrauensschüler, in dem er in seinem vierten Schuljahr einmal war. Neville und Klaus waren bereits im Wasser und hatten die Augen geschlossen. Harry sah sich um. Er ging an den Rand der in den Boden eingelassenen Wanne und stieg ins Wasser. Klaus und Neville entdeckten ihn und lächelten ihm zu. Harry nahm gegenüber Neville Platz, Klaus war zu seiner Rechten. Der Platz am Eingang war noch frei. Die beiden verwickelten Harry in ein Gespräch über Quidditch.

Sie hörten ein Geräusch und Harry sah nach links. Dort kam gerade Sirin auf sie zu. Auch sie war vollkommen nackt, wie die anderen. Harry besah sich ihren jungen Körper, spürte aber weder Peinlichkeit noch sonderliches Verlangen oder Scham. Erst heute Abend würden ihm die Erinnerungen daran noch einmal zu schaffen machen.

Sirin war am Becken angekommen und stieg in die Wanne. Sie setzte sich auf den freien Platz und begann in das wieder aufgenommene Gespräch einzusteigen.

\enquote{Nächstes Jahr möchte ich auch ins Team. Ich hoffe, ich schaffe es, da ja dann wieder jemand wegfällt.}

Jetzt war Harrys Neugierde geweckt. \enquote{Dann spielst du nur zwei Jahre, wenn du es denn schaffen solltest!}

\enquote{Ja, Harry.}

\enquote{Warum hast du es nicht früher versucht?}, fragte Neville dazwischen.

\enquote{Dazu fühlte ich mich noch nicht bereit}, sagte sie. \enquote{Ich musste noch auf dem Besen üben und auch Quidditch war mir nicht so \gst ich konnte es nicht, weil ich kaum Zeit fand, es richtig zu lernen.}

\enquote{Als was möchtest du denn spielen?}, fragte Neville.

Sirin setzte sich in einem Schneidersitz hin und legte ihre Arme auf dem Beckenrand ab. Die Wellenbewegungen im Becken nahmen nach ihren Bewegungen wieder ab, bis sie fast zum Stillstand kamen.

\enquote{Ich weiß noch nicht genau, das überlege ich mir über die Ferien. Ich dachte an Treiber, oder aber Sucher.} Sie sah Harry kurz an, wurde rot und sagte dann schnell: \enquote{Reservesucher natürlich. Äh, ich meine, Sucher, anstelle von Erika}, haspelte sie.

Harry lächelte ihr zu. \gedanke{Hast wohl vergessen, dass wir in verschiedenen Häusern sind}, dachte Harry.

Dann wandte sich Sirin an Neville und fragte ihn, warum er nicht in der Mannschaft spielte; immerhin hatte sie ihn schon ein paar Mal beobachtet, wie er sich mittlerweile auf dem Besen bewegte.

\begin{abAchtzehn}
Harrys Blick wanderte durch das ruhige Wasser über Sirins Körper. Sie hatte die gleiche Hautfarbe, wie die Patil-Zwillinge, da ihre Vorfahren auch aus Indien kamen. Ihre schulterlangen, schwarzen Haare sahen frisch gewaschen aus. Sein Blick wanderte tiefer und er sah, dass sie ihre Scham rasiert hatte.

Als er vor Wochen einmal die Gelegenheit hatte, sie spärlich bekleidet zu sehen, konnte er durch ihre Wäsche noch ihre Schambehaarung erkennen; für einen kurzen Moment sogar fühlen. Sein Blick wanderte zu Klaus und Neville, und dann wieder auf Augenhöhe, wo er sich schließlich wieder in das Gespräch einfand.
\end{abAchtzehn}

\begin{safedivide}
Harry döste vor sich hin und hörte nur mit halbem Ohr zu.
\end{safedivide}

Nach einer Weile verabschiedeten sich die drei und Harry genoss noch etwas das Alleinsein.

\gedanke{Salazar?}

\stimme{Ja, Harry.}

\gedanke{Was passiert, wenn ich mein Amulett ablege?}

\stimme{Das kommt darauf an, was du damit bezweckst.}

\gedanke{Ich frage mich, was passiert, wenn ich jetzt mein Amulett ablege. Malfoy\abs Draco hatte beide Male, genau wie ich, Albträume, als ich mal mein Amulett nicht anhatte.}

\stimme{Solange du wach bist, passiert nichts. Erst, wenn du schläfst und träumst.}

\gedanke{Aber ich habe doch auch sonst Albträume, wenn ich mein Medaillon anhabe. Nur wenn ich es nicht anhatte, habe ich von Draco geträumt.}

Es herrschte eine Weile Stille. Harry wollte gerade wieder nachfragen, als Salazar ihm antwortete: \stimme{Es könnte in Verbindung mit eurer weitläufigen Verwandtschaft und dem Amulett zusammenhängen. Und eventuell durch den Seelensplitter in dir.}

Harry dachte nach. Dann schimmerte die Luft und Salazar kam zum Vorschein. Er saß ihm gegenüber im Wasser und genoss es sichtlich, sich im warmen Wasser aufzuhalten.

\enquote{Ich dachte, Geister spüren nichts}, sagte Harry.

\enquote{Normalerweise ist das so, Harry, aber hier}, und er lächelte selig, \enquote{wirken sehr alte Zauber, die uns ein gewisses Gespür geben. Wenn sich hier ein sterblicher aufhält, dann können wir hier auch rein und die Annehmlichkeiten für wenige Augenblicke genießen.} Er schloss die Augen und legte den Kopf zurück. Dann sagte er: \enquote{Ich muss wieder los, Harry. Aber wir können uns ja} \stimme{gedanklich unterhalten.}

Harry lächelte in sich hinein. Dann schloss auch er seine Augen und blieb noch eine Weile im warmen Wasser.

Nach einem entspannten Nickerchen hörte er ein Glucksen und wusste, dass es Myrte war. Er öffnete seine Augen und sah sie einen Meter über seinem Gesicht schweben. Durchsichtig, traurig und in ihrer Schuluniform. Sie sah Harry an.

\enquote{Ich bin immer so allein, Harry. Warum kommst du mich nicht besuchen?}

\enquote{Ich habe viel zu tun, Myrte.}

\enquote{Na ja, wenigstens sind wir jetzt beisammen.} Sie wirkte nun etwas fröhlicher. \enquote{Ich habe dich gesehen, wie du mit anderen Mädchen Sex hattest. \gst Weißt du, ich hatte noch nie welchen. Und das, wo ich schon über sechzig bin.}

Harry musste schmunzeln. \enquote{Myrte, du bist erst vierzehn, oder so.}

\enquote{Ich bin über sechzig. Ich bin vielleicht jung gestorben, aber als Jungfrau und das vor sehr langer Zeit.} Sie wirkte wieder etwas trauriger. Dann schwebte sie tiefer und nun direkt über ihm. Sie bildete seine Konturen nach, hielt aber etwa dreißig Zentimeter Abstand zu ihm. \enquote{Ich möchte auch mal mit jemandem schlafen. Ich habe schon viele Paare dabei beobachtet und mir vorgestellt, wie es sein würde. Du bist der Erste, mit dem ich es mir seit langer Zeit einmal vorstellen könnte.} Wieder kam sie ihm etwas näher.

Harry versuchte aus dieser Situation das Beste zu machen und versuchte sie davon abzubringen. \enquote{Du hast ja immer noch etwas an, Myrte}, sagte er.

Myrte sah an sicher herunter und sagte: \enquote{Oh. Du hast recht.}

Sie schloss ihre Augen und konzentrierte sich. Sie begann zu verschwimmen, und als sie wieder klar zu erkennen war, hatte sie nichts mehr an und war vollkommen nackt. Sie entfernte sich von Harry etwas und meinte dann: \enquote{Und, Harry, nimmst du mich so?}

Myrte hatte etwas an sich, das Harry nicht beschreiben konnte. Als sie vollkommen nackt vor ihm schwebte und er in diesem Raum mit verminderter Scham war, regte sich etwas an ihm und er stellte sich tatsächlich vor mit ihr zu schlafen. Würde er Ginny mit einem Geist überhaupt betrügen können? Wäre Sex mit einem Geist, so er denn möglich wäre, Betrug am Partner?

Harry hatte den Eindruck, dass sie etwas weniger zu durchschauen war als sonst. \gedanke{Sie hat etwas mehr Substanz}, dachte er. Mit ihrer Zunge fuhr sie über ihre Lippen und senkte ihren Körper und ihren Kopf zu Harry hinunter. Dieser versuchte ihr auszuweichen und drückte sich gegen den Beckenrand und schob sich langsam aus dem Wasser auf den Boden davor. Er blieb mit seinen Kniekehlen hängen und seine Beine baumelten im Wasser, als Myrte mit ihren Lippen seine traf.

\begin{abAchtzehn}
Er spürte nur einen warmen Lufthauch, mehr nicht. Sie grinste ihn an und schwebte an ihm herunter. Sein Penis streckte sich ihr entgegen und sie nahm ihn in den Mund. Zumindest versuchte sie es. Wieder nur spürte er einen warmen Lufthauch. Er umschloss seinen Penis und fing an sich zu stimulieren. Mit seiner Hand fuhr er immer wieder durch Myrtes Kopf. Sie schien es nicht zu stören.

Dann glitt sie immer höher schwebend auf seiner Brust entlang. Dort wo ihr Mund knapp unterhalb seiner Haut war, konnte er einen warmen Schauer spüren. Als sie an seinem Mund ankam, zuckten kleine Blitze durch ihn. Myrte hielt kurz inne und meinte: \enquote{Das ist mir noch nie passiert.}
\end{abAchtzehn}

\gedanke{Hier herrscht sehr alte Magie}, kam Harry wieder in den Sinn.

Mit seiner freien Hand fuhr er Myrte in ihren Nacken und machte in der Luft kraulende Bewegungen, was sie sich scheinbar gefallen ließ. Sie ließ sich tiefer sinken, sodass ihr Körper leicht in seinen hinein glitt. Das war ein unglaubliches Gefühl. 

\begin{abAchtzehn}
Er war halb mit einem Geist verbunden und stimulierte sich selbst. Myrte drehte nur leicht ihre Lippen durch seine und schien glücklich zu sein.

Dann richtete sie sich auf, öffnete ihre Beine, spreizte mit ihren Fingern ihre Schamlippen und sank langsam auf Harry nieder. Harry sog zischend die Luft ein. Dieses Gefühl, das ihn durchfuhr, konnte er mit nichts anderem vergleichen. Er konnte es nicht beschreiben. Er hatte das Gefühl sich wirklich mit ihr zu vereinen. Myrte brachte nur ein \geraeusch{Ahhhhhh!} heraus. Sie warf ihren Kopf zurück und keuchte. Harry hatte das Gefühl, sich schwerer zu tun. Er nahm seine Hand von sich und umklammerte Myrtes Hüften.

Er konnte nun nicht mehr so einfach durch sie hindurch greifen. Vorsichtig hob und senkte er sie, als wäre sie sehr zerbrechlich. Doch mit der Zeit wurde sie für ihn immer fester, bis er sie ganz sah und spürte; auf sich, bei sich, in ihr.

Myrte schien davon nicht allzu viel mitbekommen zu haben. Sie hatte ihre Augen immer noch geschlossen und ihren Kopf immer noch in ihrem Nacken. Harry nahm jetzt seine Hände von ihren Hüften und fing an, ihre Brüste zu kneten.

\enquote{Harry}, kreischte Myrte. \enquote{Ja, knete sie durch, bearbeite sie, massiere sie}, rief sie. Sie öffnete ihre Augen und sah Harry an. Sie senkte ihren Oberkörper hinunter und küsste ihn. Ihr Blick war noch immer jenseits von Zeit und Raum. Er spürte ihre Säfte an seinem Schaft entlang herunterfließen, über seine Hüften und Schenkeln auf den warmen Steinboden.

Es ließ sich nicht vermeiden, dass er dieses Jahr an Erfahrungen sammelte, aber auf das, was er gleich erleben würden, konnte man sich auch mit Training nicht vorbereiten. Er spürte, wie sich Myrtes Orgasmus näherte und kam seinem Höhepunkt auch immer näher. Dann riss sie ihn mit sich. Doch sie konnte kein \enquote{\extase{Harry!}} schreien, da er ihren Mund mit einem tiefen und innigen Kuss versiegelte. So spürte er den Druck ihres Schreis in seinen Lungen, die sich mit warmer Luft füllten. Er spritze einen Teil seines Saftes durch sie hindurch in die Luft.
\end{abAchtzehn}

\begin{safedivide}
Beide blieben eines Weile so und genossen das Gefühl.
\end{safedivide}

Als sie sich wieder beruhigt hatte und Harry sich von ihr löste, sah sie ihn mit glücklichen, glasigen Augen an.

\begin{abAchtzehn}
Und schon wieder spürte er Myrtes Unterleib zucken. Und erneut riss sie ihn in einen unglaublichen Orgasmus hinein. Dieses Mal blieb sein Sperma in ihr. Als er sie von sich drücken wollte, riss es ihn wieder in den Himmel und Myrte mit ihm. Also blieben sie nach ihrem dritten Mal einfach liegen. Als sie sich beruhigt hatten, schwebte Myrte etwas höher und richtete sich in der Luft auf. Sie schwebte nun in einem Schneidersitz direkt über seinem Gesicht und gab ihm einen unglaublichen Anblick.

Er hob seinen Kopf und bearbeitete ihre Schamlippen mit seiner Zunge und ließ dabei auch ihre Klitoris nicht außen vor. Ihr Unterleib begann wieder zu zucken und dann ergoss sich ein dicker Strahl über Harrys Gesicht, nachdem sie komplett die Beherrschung verloren hatte, als ihr Harry einen Finger in ihre Scheide steckte und mit dem Daumen über ihr Lustknöpfchen rieb.

Myrte schloss ihre Augen und genoss dieses Gefühl noch einmal. Sie küsste Harry zum Abschied, fuhr sich kurz zwischen ihren Beinen mit den Fingern durch und sein Sperma fiel auf den Boden.
\end{abAchtzehn}

\begin{safedivide}
\fskdivider
\end{safedivide}

Dann begann sie wieder zu verschwimmen, und als sie wieder scharf zu sehen war, hatte sie wieder ihre alte Schuluniform an.

\begin{abAchtzehn}
\enquote{Ich will kein Risiko eingehen}, sagte sie. \enquote{Ich weiß nicht einmal, ob ich überhaupt schwanger werden könnte.} 
\end{abAchtzehn}

%\begin{safedivide}
%\fskdivider
%\end{safedivide}

Dann warf sie ihm noch eine Kusshand zu und verschwand mit einem seligen Lächeln in den Abflussrohren.

Harry duschte sich und zog sich danach an. Er verließ den Raum und sah Myrte, die scheinbar auf ihn wartete.

\enquote{Wiederholen wir das noch einmal, Harry? Irgendwann?}, fragte sie, mit einer gewissen Sehnsucht in ihrer Stimme.

\enquote{Irgendwann}, antwortete Harry, dessen Scham langsam zurückkehrte.

Auch Myrte war anzusehen, dass sie in ihre normale Art mit Harry umzugehen, zurückzufallen schien. Glücklich schwebte sie davon.

\trenn

Langsam schlief Harry ein und begann zu träumen. Das Amulett hatte er mit der Hand umfasst. Dann, mitten in der Nacht, lag er wieder neben Draco. Wie immer nackt. Beide hielten sich an den Händen.

\enquote{Bist du wach?}, fragte er Harry.

\enquote{Nein, Draco, ich träume. Genau wie du.}

\enquote{Weißt du, was merkwürdig ist?}, fragte er. \enquote{Wir liegen hier nackt nebeneinander und ich habe keine Gefühle für dich.} Harry musste leicht schmunzeln. \enquote{Ich meine, ich empfinde keinen Hass oder Ablehnung dir gegenüber.} Dann stieg er aus dem Bett und setzte sich auf den Fußboden. Die Beine angewinkelt und seine Hände außen herum geschlungen.

\enquote{Weiß jemand von diesen Träumen?}, fragte Harry Draco, stieg ebenfalls aus dem Bett und setzte sich mit gleicher Haltung neben ihn.

\enquote{Außer Tamara? Nein!}, antwortete Draco.

\enquote{Wieso deine Schwester?}, fragte Harry nach.

\enquote{Du hast es ihr doch erzählt}, antwortete Draco.

\enquote{Oh. Stimmt.}

Dann schwiegen beide lange Zeit.

\enquote{Irgendwie beneide ich dich}, begann Draco dann nach langer Zeit das Gespräch wieder. \enquote{Du hast Freunde gefunden.}

\enquote{Du hast doch Vincent und Gregory.}

\enquote{Das sind keine Freunde, die sind wie Schoßhündchen. Ich gebe ihnen Anweisungen und sie führen sie aus.}

\enquote{Blaise?}

\enquote{Der kommt einem Freund am nächsten.}

\enquote{Theodore?}

\enquote{Eher nicht.}

\enquote{Millicent?}

\enquote{Nein.}

\enquote{Maria und Pansy?}

\enquote{Die eine war, die andere ist meine Freundin.}

\enquote{Ich?}

\enquote{Wir können uns nicht leiden und bekriegen uns ständig.}

\enquote{Sieht momentan aber nicht danach aus.}

Draco sah ihn an. \enquote{Ich möchte nicht immer über oder unter dir aufwachen und etwas von dir in meinem Mund haben, wenn mir bewusst wird, dass ich träume.}

Harry lächelte leicht. \enquote{Daran können wir arbeiten.} Er konzentrierte sich und die Umgebung änderte sich. Harry trug nun ordentliche Kleidung, Draco war immer noch nackt.

Skeptisch sah ihn der Blonde an. \enquote{Wieso kannst du so etwas und ich nicht?}, fragte er ihn.

\enquote{Hast du es versucht?}, fragte Harry zurück. Draco schüttelte den Kopf, dachte kurz nach und hatte dann ebenfalls Kleidung an. \enquote{Siehst du.}

Draco nickte und begann vorsichtig ein paar Sachen, die ihm auf der Seele lagen, zu erzählen.

Harry tat dasselbe, obwohl er schon mit Ron darüber gesprochen hatte.

Erst am nächsten Morgen stellten beide fest, dass sie darüber nicht reden konnten. Sie wussten viel vom jeweils anderem, konnten aber nichts davon erzählen.

\stimme{Das liegt an mir}, hörte Harry in seinem Geist sagen und lächelte vor sich hin, als er am Frühstückstisch saß.

Er hatte begonnen, Draco zu verstehen. Sein Leben wurde immer von anderen bestimmt. Er konnte bislang keine eigenen Entscheidungen treffen.

\trenn

Harry ging auf die Krankenstation, um Alina etwas Gesellschaft zu leisten. Er betrat sie und fand nur Madame Pomfrey vor, die gerade aufräumte. Als sie zu ihm sah, winkte er nur ab und zeigte auf den abgeschirmten Bereich, in dem Alina lag. Stumm nickte sie ihm zu und kümmerte sich wieder um ihre Arbeit.

Harry trat an die Vorhänge und Stellwände um Alinas Bett heran und sagte leise: \enquote{Alina? Ich bin’s, Harry. Darf ich zu dir?}

\enquote{Ja}, kam schwach von der anderen Seite.

Harry betrat den abgeschirmten Bereich und musste sich erst einmal an den Anblick gewöhnen. Dort war fast eine voll ausgebildete junge Harpyie zu sehen. Nur das Gesicht hatte noch menschliche Züge und auf ihren Oberschenkeln, von denen einer unter der Bettdecke heraus schaute, waren noch Hautpartien zu sehen, die menschlich waren. Harry setzt sich auf einen Stuhl an ihrem Bett.

\enquote{Darf ich?}, fragte er und deutete auf ihre Hand.

Sie nickte nur stumm und sah ihn traurig an. \enquote{Keiner kann mir helfen. Mr. Ollivander konnte zwar den Zauber isolieren und beheben, aber keiner hat bisher ein Gegenmittel gefunden.}

Harry griff nach ihrer Hand. Sofort durchzogen ihn Bilder. Er sah eine kleine Schlange, die ihre Zähne in Alinas Oberschenkel schlug.

\stimme{Daran hatte ich gar nicht gedacht. Das könnte sogar funktionieren.}

\gedanke{Was könnte funktionieren?}

Harry bemerkte Alinas sorgenvollen Blick nicht, da er zu sehr mit sich selbst beschäftigt war.

\stimme{Basiliskengift. Es könnte mit dem richtigen Zauber zusammen ihren Zustand beheben. Das wäre sogar die bessere Methode.}

\enquote{Harry? Alles in Ordnung mit dir?}, fragte Alina.

\gedanke{Aber, wie kommen diese Bilder plötzlich in meinen Geist?}

\stimme{Das liegt an deinen drei Magie-Quellen, die sich in dir vereinen. Deine eigene Magie, die Magie Voldemorts und meine.}

\gedanke{Dadurch schaffe ich es, solche Bilder zu sehen?}

\stimme{Dadurch schaffst du es, Zugang zur Magie zu haben, die sonst gewöhnlichen Zauberern verschlossen ist. Du dringst hier in Bereiche der Magie vor, die nur wenigen offenbart werden.}

\gedanke{Kennst du jemanden, an den ich mich wenden kann?}

\stimme{Du wirst doch schon unterrichtet.}

\enquote{Harry, alles in Ordnung?}, fragte Alina erneut.

\enquote{Ja, Alina. Mir ist gerade eine Möglichkeit eingefallen, dir zu helfen.}

\enquote{Wie meinst du das?}

\enquote{Ich hatte eine Vision. Und meine Visionen waren in letzter Zeit sehr präzise und stellten sich als wahr heraus.}

\enquote{Was muss ich tun?}

\enquote{Du müsstest dich von einer Schlange}, jJetzt klingelte in seinem Kopf etwas, \gedanke{das war keine Schlange. Das war etwas Tödlicheres. Das war ein Basilisk}, \enquote{beißen lassen. In Zusammenarbeit mit einem Zauber wirst du dann geheilt.}

\enquote{Und das hilft mir?}

\enquote{Das kann ich dir nicht versprechen, aber ich bin der Meinung, dass es das tut.}

Der Vorhang ging auf und Madame Pomfrey betrat den abgeschirmten Bereich. \enquote{Wie kommen Sie darauf, die Lösung zu haben, wenn alle Professoren und Heiler keine Lösung finden können?}, fragte sie ihn.

\enquote{Weil ich Quellen habe, auf die andere nicht zugreifen können.}

\enquote{Welche Quellen?}

Harry holte sein Amulett hervor. \enquote{Ich habe durch das Amulett hier viel gelernt dieses Jahr. Außerdem kann ich auf}

\stimme{Vorsicht.}

\enquote{andere Magiequellen zugreifen. Ich kann nicht sagen, welche, aber sie sind sehr machtvoll. Wenn Sie es zulassen, dann bitte ich Sie nur um eines: Erzählen Sie niemandem davon und vertrauen Sie mir.}

Madame Pomfrey sah ihn skeptisch an. \enquote{Professor Dumbledore sollte dabei sein. Wenn er Ihnen vertraut, dann stimme ich zu.}

\enquote{Danke, Madame Pomfrey, ich werde alles vorbereiten. Sagen Sie dem Direktor Bescheid?}

Sie nickte und Harry verschwand von der Krankenstation.

\stimme{Nimm die Aufzüge, Harry. Das Symbol mit dem grünen Ring. Halte mein Amulett und danach deine Hand auf die Prüffläche, dann erhältst du Zugang und kommst in den versteckten Gängen vor meinen Räumen heraus. Die Tür kannst du so öffnen. Sie erkennt dich.}

Harry nickte, obwohl er kein gegenüber hatte und ging zu einem der Torbögen des Schlosses vor der Krankenstation. Er drückte den Stein und trat in das Innere des Aufzuges. Dann nahm er sein Amulett ab und drückte den Knopf mit dem grünen Ring. Er hielt zuerst das Amulett und danach seine Hand auf die Prüffläche. Die Türen schlossen sich und der Boden vibrierte. Als die Türen sich wieder öffneten und er heraustrat, musste er sich erst einmal orientieren.

Dann trat er die wenigen Meter zur Tür und öffnete sie. Er stand in Salazars privaten Räumen.

\parsel{Marcel? Bist du hier?}

Es dauerte eine Weile, dann tauchte der kleine Basilisk auf, schlich zwischen seinen Beinen hindurch und schlängelte sich an seinem Hosenbein entlang hoch zur Hand. Dann legte er sich wie ein Schal um Harrys Hals und sah ihn an.

\parsel{Wass gibt ess?}

\parsel{Ich brauche deine Hilfe um einer Mitschülerin zu helfen.}

\parsel{Wass mussss ich tun?}

\parsel{Du musst sie nur beißen.}

\parsel{Dass sschaffe ich.}

Harry streichelte ihm über seinen Kopf und die kleine Schlange hielt dagegen. Sie ließ sich das Streicheln gefallen. Harry verließ nach einem Blick und einem dankbaren nicken zu Salazar und seiner Frau die Räumlichkeiten und kehrte über die Aufzüge zurück in die Krankenstation.

Dumbledore war bereits da und von Madame Pomfrey im Groben darüber aufgeklärt, was sie erfahren hatte. Harry schloss hinter sich die Türen und sicherte sie.

\enquote{Du meinst, dass du ihr helfen kannst?}, fragte er.

\enquote{Ja}, antwortete Harry. \enquote{Ich kann Ihnen nicht sagen, woher ich das weiß, aber ich bin sehr sicher, dass es klappt.}

Dumbledore sah ihn lange an. Nachdem Harry langweilig wurde, ging er auf Alina zu und nahm Marcel von seinem Hals herunter, um ihn Alina zu zeigen.

\enquote{Dies hier ist Marcel. Er wird dich beißen.}

\parsel{Ich kann das nicht.}

Harry sah zu Marcel. \parsel{Wie? Du kannst das nicht?}

\parsel{Ich komme nicht durch die Haut mit meinen kleinen Zähnen. Du musst ihr das Gift mit etwas anderem als meinen Zähnen injizieren.}

Harry sah zu Madame Pomfrey. \enquote{Haben sie eine Spritze und ein Reagenzglas mit einem Gummihandschuh darüber?}

Madame Pomfrey sah ihn erst fragend und dann erstaunt an. \enquote{Das ist keine Schlange, Mister Potter. Oder? Das Gift, das Sie mir gegeben haben, das andere meine ich, stammt von \accentuate{ihm}?} Sie wollte den Begriff Basilisk nicht in den Mund nehmen, da sie die Antwort zu sehr ängstigen würde.

\enquote{Da haben Sie recht.}

\enquote{Warum leben wir dann immer noch?}

\enquote{Poppy?}, fragte Dumbledore nach.

\enquote{Weil ich mit einem Zauber und einem Trank dafür gesorgt habe, dass seine Blicke nicht mehr tödlich sind.}

\enquote{Reden wir hier von einem Basilisken, Harry?}

\enquote{Ja, Professor. Sein Name ist Marcel. \gst Bekomme ich jetzt die Spritze und das Reagenzglas für das Gift? Ich möchte Alina gerne helfen.} Harry nervte es, dass er immer wieder aufgehalten wurde. \enquote{Tut mir leid}, sagte er dann, da er merkte, dass es doch respektlos war, was er gesagt hatte.

\enquote{Ich hole Ihnen die Sachen}, sagte Madame Pomfrey und verschwand.

Dumbledore war noch immer nicht richtig überzeugt. \enquote{Warum hast du mir nichts davon erzählt, Harry?}

\enquote{Du erzählst mir ja auch nicht alles. Außerdem hatte ich Angst um meinen kleinen Freund.}

\enquote{Aber wie kommst du an diese Zauber? Ich kenne keine Möglichkeit, einem Basilisken diese Fähigkeit zu nehmen.}

\enquote{Du hast dich mit diesen Tieren aber auch nicht beschäftigt. Aber jemand anderes sehr wohl. Ich habe viel über diese Geschöpfe gelernt. Wenn man ihren tödlichen Blick im Griff hat, dann sind sie wie normale Schlangen. Nur ihr Gift ist überaus wirkungsvoll. Es übertrifft die der giftigsten Schlange auf Erden um ein Vielfaches.}

Madame Pomfrey war bereits mit den gewünschten Sachen zurück und gab sie Harry. \enquote{Bitte, Mister Potter, fangen Sie an.}

\enquote{Ich bin noch nicht ganz davon überzeugt, Poppy.}

\enquote{Dich hat auch keiner gefragt. Das ist meine Patientin und ich bin für sie verantwortlich. Wenn ich eine Behandlung für richtig halte, dann ist das ausschließlich meine Sache. Da hältst du dich raus.} Dumbledore sah sie erstaunt an. So hatte er sie noch nie erlebt. \enquote{Also, hilfst du uns, oder gehst du?}, meinte sie jetzt noch energischer.

Es brauchte eine Weile, bis Dumbledore wieder zu sich fand. \enquote{Ich werde von dort aus zusehen}, sagte er endlich und entfernte sich, um sich auf einen Hocker zu setzen und das Ganze aus der Ferne zu betrachten.

Harry nahm die Gegenstände von Madame Pomfrey entgegen und setzte sich wieder an Alinas Bett. Er zog den Gummihandschuh über das Reagenzglas, als ihn Marcel unterbrach.

\parsel{Ich denke, ich schaffe das auch ohne. Ich weiß ja, was du von mir erwartest.}

Harry nickte und entfernte den Handschuh wieder. Dann hielt er seinem kleinen Freund das Glas hin, worauf dieser hineinbiss und sein Gift abgab. Dann nahm Harry eine kleine Menge davon in die Spritze auf und fragte Alina, ob sie bereit sei. Nach kurzem Nicken stach er zu und gab das Gift über die Kanüle in die Muskeln frei. \enquote{Schützt euch bitte vor Querschlägern}, sagte Harry zu Madame Pomfrey und Dumbledore. Er zog die Spritze zurück, stand auf legte sie beiseite. Dann wartete er ein paar Sekunden, bevor er seinen Zauberstab zog und ihn auf Alina richtete. Gerade als sie anfing, das Gift zu bemerken, legte er los.

\enquote{Das Gift brennt, Harry}, sagte Alina.

Ihm flogen die passenden Sprüche in seinem Geist zu. Nacheinander wendete er jeden dieser Sprüche auf Alina an: Einen um das Gift zu wandeln, einen um die Regeneration der Zellen anzuregen und einen um die Verwandlung zu transformieren. Zuerst sah es so aus, als ob sich die Transformation schneller entwickeln würde und Dumbledore stand schon auf um näher zu kommen. Doch Harry erschuf einen Schild um die zwei herum und machte weiter. Als Alina vollständig eine Harpyie war und keine menschlichen Züge mehr an sich trug, begann die Rückverwandlung. Durch die Sprüche verhinderte er, dass sich ihr Geist veränderte. Aber die körperliche Transformation musste sie zuerst vollenden, bevor er sie zurückverwandeln konnte.

Als er fertig war und das schützende Feld wieder abbaute, legte er sich erschöpft auf das Bett daneben und schlief ein. Nach drei Stunden erwachte er wieder. Dumbledore und Madame Pomfrey waren verschwunden, nur Alina saß noch an seinem Bett.

\enquote{Danke}, sagte sie und gab ihm einen Kuss auf die Stirn. \enquote{Ich bin wieder ich selber. Wie kann ich dir das zurückzahlen?}

\enquote{Lerne und bleib auf der richtigen Seite.}

\enquote{Welche ist das?}

\enquote{Das ist manchmal schwer zu beantworten, zumal jede Seite sagt, ihre sei die richtige.}

\enquote{Das würde die andere Seite wohl nicht sagen.}

\enquote{Vermutlich nicht.}

\enquote{Dann werde ich auf deiner Seite bleiben.} Sie lächelte ihn an, drückte kurz seine Hand und meinte dann: \enquote{Ich werde jetzt in mein Haus gehen. Die anderen haben mich sicherlich schon vermisst. Und ich werde meinen Eltern schreiben.}

\enquote{Tu das}, sagte Harry und ruhte sich weiter aus. Er spürte, dass er Voldemorts Seelenteil wieder einen großen Teil Magie entzog.

Außerdem musst er sich noch überlegen, wie er Ginny das mit Myrte beibrachte. Sollte er es ihr sagen?
