\chapter{Erste Annäherung, oder doch nur eine Strafe?}


Harry ging zum Unterricht. Auf dem Weg dorthin kam ihm Pansy entgegen und stellte sich ihm in den Weg. Er hatte keine Chance vorbeizukommen. Also entschied er sich zu einem ungewöhnlichen Schritt. Er schritt auf Pansy zu und bewegte seinen Kopf zu ihrem Ohr. Leider kam in diesem Moment Professor Snape um die Ecke. \enquote{Potter. Nachsitzen wegen Belästigen von Schülern. Melden Sie sich nach dem Unterricht bei Madame Pince.} Sauer trat Harry an Pansy vorbei und setzte sich schwer schnaufend neben Ron, nachdem er das Klassenzimmer betreten hatte. \enquote{Man könnte meinen, dass sich Parkinson und Snape absprechen, um mir eins reinzuwürgen.}

\enquote{Wir werden uns heute dem Euphemos-Trank widmen. Schlagen sie ihr Buch auf und suchen sie nach dem Trank. Dann bereiten sie ihn zu. Die fehlenden Zutaten holen sie aus dem Schrank für allgemeinen Bedarf. Wenn sie mich brauchen, ich bin in meinem Büro.} Harry maß die Flüssigkeiten ab, während Ron die anderen Zutaten, wie Kräuter, klein schnippelte. Snape kam hin und wieder aus seinem Büro, die Tür hatte er immer noch offen, und begutachtete die Tränke der Schüler.

Harry wusste, dass ihr Trank perfekt war, und so füllte er am Ende der Stunde ein Glas ab, um es zur Bewertung abzugeben. Nach dem Essen ging er in die Bibliothek, um sich bei Madame Pince für sein Nachsitzen zu melden.

\enquote{Mister Potter, Professor Snape hat sich bereits bei mir gemeldet. Folgen Sie mir.} Sie schritt durch die Gänge der Bibliothek und inspizierte die Reihen. Am Ziel angekommen, lagen vor ihm Bücher auf den Tischen und auf den Boden.

\enquote{Ihren Zauberstab bitte}, sagte Madame Pince und streckte ihm ihre Hand entgegen.

\enquote{Wie?}, fragte Harry ganz erstaunt.

\enquote{Sie werden die Bücher von Hand sortieren. Dort liegt die Liste.} Sie zeigte auf einen Stapel Pergament.

Harry übergab ihr seinen Zauberstab und blickte wütend auf die Pergamente. Seinen Zorn auf Snape auslebend, sortierte Harry die Bücher in der Bibliothek. Seinen Zauberstab hatte er abgegeben und würde ihn erst wieder zurückerhalte, wenn er mit seiner heutigen Arbeit fertig war. \gedanke{Das wird drei Tage á 4 Stunden dauern}, dachte sich Harry. Also machte er sich an die Arbeit. Er ging die Bücherliste durch und fing an, die ersten Bücher in das Regal zu stellen. Hin und wieder warf er einen Blick in ein Buch, dessen Titel sich interessant anhörte. So fand er unter anderem heraus, wie man aufgrund von Körperflüssigkeiten oder Hautzellen den zugehörigen Träger feststellen konnte. Die Probe musste erst mit einem Zauber behandelt werden, um sie danach in einem Trank aufzulösen. Danach musste wieder ein Zauber angewandt werden, um ein plastisches Bildnis zu erhalten. Die restliche halbe Stunde für heute sortierte er die anderen Bücher. Unter anderem einige, die Schwebezauber behandelten. Harry hatte dieses bereits in seinem ersten Jahr gehabt, es war also vollkommen uninteressant. Doch eines erregte seine Aufmerksamkeit, er würde es sich ausleihen. Nach getaner Arbeit nahm er seinen Zauberstab zurück, lieh das Buch aus und ging in sein Bett, während Madame Pince seine Arbeit noch kurz begutachtete und danach die Bibliothek verschloss.

Hundemüde schleppte er sich zuerst ins Bad und nach getaner Abendtoilette ins Bett. Harry lag etwa eine viertel Stunde im Bett, als er ein kleines \geraeusch{Plopp} hörte. Harry öffnete \gst immer noch liegend \gst einen Vorhang und sah nach draußen. Vor seinem Bett stand ein junger männlicher Elf und sah sich ängstlich um.

\enquote{Wer bist du denn?}, fragte Harry den jungen Elf.

Dieser drehte sich etwas und sah ihn an. \enquote{Frodo kalt}, sagte er.

Harry bemerkte, dass der Elf nur eine kleine Serviette anhatte, die das nötigste verdeckte. Klar, dass ihm kalt war. Und er, Harry, lag in einem warmen Bett. Harry dachte darüber nach, ihm einen warmen Schlafplatz anzubieten. Er überlegte, aber außer seinem Bett hatte er nichts. \gedanke{Das muss einer der Elfen aus der Küche sein}, ging ihm durch den Kopf.

Harry hob seine Decke an und bot dem jungen Elf einen Schlafplatz an, doch dieser sah ihn nur an. Harry wurde es langsam zu viel. Da er in Reichweite stand, rückte er etwas auf seinem Bett herum und griff dann vorsichtig mit seinen Händen nach dem Elf und hob ihn hoch. Dann zog er ihn zu sich und steckte ihn unter die Decke. Beide Köpfe schauten nun unter der Decke hervor. Jetzt schloss Harry wieder den Vorhang und legte sich auf die Seite, den kleinen Elf vor sich mit dem Rücken zu ihm.

\enquote{Frodo warm}, meinte der Elf. Dann sanken seine Ohren auf sein Gesicht und er schlief ein.

Auch Harry versank in einen erholsamen Schlaf. Diesen hatte er bitter notwendig. Doch der Schlaf hielt nicht lange. Er wurde wieder durch einen \geraeusch{Plopp} geweckt. Dann hörte er leise Rufe in einer ihm unbekannten Sprache. Vorsichtig hob er seinen Vorhang an und spähte unten durch. Vor seinem Bett bewegte sich etwas.

Dieses etwas drehte sich und sah ihn aus großen Augen an. \enquote{Verzeihen Sie die Störung. Ich suche jemand. Bitte schlafen Sie weiter}, antwortete der Elf.

Harry hob den Vorhang etwas weiter an und fragte dann: \enquote{Wenn suchen sie Sir?}

Der Elf bekam erst große Augen und meinte dann unsicher: \enquote{Meinen Sohn.}

\enquote{Wie heißt er?}, fragte Harry.

\enquote{Frodo.}

Harry ließ den Vorhang wieder herunter und schob ihn kurz darauf beiseite. Dann zog er seine Bettdecke etwas herunter. \enquote{Ist er das, Sir?}

Der Elfenvater bekam große Augen und nickte nur, da er scheinbar aufgrund des Schocks sprachlos war.

Harry hob seine Bettdecke an und hob ihn vorsichtig aus dem Bett und reichte ihn seinem Vater. Dieser nahm ihn dankbar an. Dann sah er Harry ehrfürchtig an.

\enquote{Danke Mister Potter. Wie hat sich mein Sohn verhalten?}

\enquote{Sie kennen mich?}, fragte Harry nach.

\enquote{Fast alle Elfen kennen sie, Mister Potter.}

Harry nickte. \enquote{Nun, er war sehr brav. Tauchte hier auf und schaute mich an. Dann meinte er, ihm sei kalt. Also habe ich ihm einen warmen Schlafplatz angeboten. Morgen hätte ich ihn in der Küche abgegeben, damit sich jemand seiner Art um ihn kümmern kann, während seine Familie gesucht wird.}

\enquote{Hat er sich gewehrt, ins Bett zu kommen?}

\enquote{Er hat gar nichts gesagt. Nur \enquote{Frodo warm}, als er unter der Decke lag.}

\enquote{Sie haben gut daran getan, ihn bei sich zu haben.} Dann verschwand er mit seinem Jungen auf dem Arm und einem leisen \geraeusch{Plopp}.

Harry war zu Müde, um weiter darüber nachzudenken. Er schloss seinen Vorhang, deckte sich wieder zu und schlief ein. Als er am nächsten Morgen aufstand, hatte dieser Vorfall für ihn keine Bedeutung mehr.

Während des Frühstücks sagte er Ron und Hermine, was er in der Nacht erlebt hatte. Danach ging er mit Ron und Hermine Richtung Wald, um mit den Dementoren zu üben. Kaum hatte er das Schloss verlassen und war auf dem Weg zum Wald, traf er Professor Dumbledore, der mit Professor Flitwick und Professor Sprout vor ihnen her liefen.

\enquote{Guten Morgen, die Professoren}, grüßten die drei.

\enquote{Guten Morgen, Harry, Hermine, Ron}, grüßte Dumbledore und auch die anderen Professoren sagten: \enquote{Guten morgen, Mister Potter. Miss Granger, Mister Weasley.}

\enquote{Wohin geht ihr?}, fragte Dumbledore, der nun zwischen den dreien lief, die anderen Lehrer vor ihnen, aber ihre Ohren nach hinten gerichtet.

\enquote{Richtung Wald, für den Unterricht üben. Es ist kurz nach acht, also wird es für uns Zeit.}

\enquote{Ich halte das für gefährlich. Laut meinen Informationen sollen sich hier Dementoren aufhalten.}

\enquote{Wo?}, fragte Harry scheinheilig nach.

\enquote{In diesem Waldstück}, sagte Dumbledore und zeigte auf den Waldrand, der knappe fünfzehn Meter entfernt war.

\enquote{Du musst deine Bewegung präziser machen}, hörte man aus dem Wald. \enquote{Lass mich mal.} Es folgten einige Sekunden der Stille. \zauber{Expecto Patronum}, sagte eine Stimme, die Harry eindeutig Blaise Zabini zuordnen konnte. \enquote{Siehst du, so geht das. Die Bewegung nicht so unsauber, dann klappt es auch.}

Harry fing an zu grinsen, was Dumbledore nicht entging, da er zu Harry sah und wissen wollte, wie er auf die Nachricht mit den Dementoren reagieren würde. Sie hatten den Rand des Waldes erreicht, als die anderen Lehrer bereits ihre Zauberstäbe zogen.

\enquote{Das wird nicht nötig sein, die sind hier um mit uns zu üben}, warf Hermine ein.

\enquote{Wie, zum Üben hier?}, fragte Professor Flitwick.

\enquote{Na ja, wir sollten heute nochmals mit den Dementoren üben, wenn wir demnächst beigebracht bekommen, wie wir mit Patroni Nachrichten übermitteln wollen.}

\enquote{Sie haben schon mal mit Dementoren\abs?}, fragte Professor Flitwick nach.

\enquote{Sicher}, antwortete Ron. \enquote{Vor ein paar Tagen.}  Dann lief er mit Harry und Hermine in den Wald hinein.

\enquote{Was wollt ihr denn hier?}, fragte Zabini nach.

\enquote{Üben}, antwortete Harry.

\enquote{Aber nicht hier.}

\enquote{Von mir aus. Dann nehmen wir einen Dementoren etwas abseits mit.}

\enquote{Geht bloß weiter.}

Die drei winkten einen Dementoren zu sich her und wanderten ein paar Schritte nach nebenan. Dumbledore, Flitwick und Sprout standen erst einmal perplex da. Die Zauberstäbe noch immer in der Hand und abwartend, um ihre Schüler zu schützen. Doch scheinbar hielten sich die Dementoren zurück und griffen nur an, wenn die Schüler bereit waren, und hörten auf, sobald der Patronus zu schwach wurde, oder abebbte.

Nach einigen Minuten kamen weitere Schüler schwatzend hinzu und begrüßten die Professoren. Sie suchten sich einen freien Dementoren, oder schlossen sich einer Gruppe an und übten ihre Patroni.

\enquote{Solch zahme Dementoren habe ich noch nie gesehen}, meinte Professor Sprout nachdenklich.

\enquote{Aber wie\abs?}, fragte sich Professor Dumbledore.

Nach einer Weile meinte Professor Flitwick: \enquote{Also ich gehe jetzt. Falls noch jemand dableiben möchte. \gst Soll ich jemand etwas mitbringen?}

Gedanken versunken schüttelten die beiden die Köpfe. Nach einer Weile, es waren bereits mehrere Gruppen gekommen und gegangen, kam auch Professor Elber vorbei und schaute dem Treiben zu.

\enquote{Erstaunlich. Diese Dementoren}, meinte Dumbledore.

\enquote{Ja, finde ich auch}, entgegnete Elber. \enquote{Eine neue Sorte?}

Dumbledore sah ihn mit hochgezogener Augenbraue an. Elber lächelte ihn kurz an, drehte sich um und ging dann wieder.

Später, nach dem Essen, schwang das Porträt zurück und ein ziemlich fertiger Harry kam herein.

\enquote{Was ist passiert?}, fragte Hermine erstaunt.

\enquote{Ich bin mausetot. Gestern war es nur geistig, aber heute\abs wie die verrückten haben sie auf mich eingezaubert.}

\enquote{Die?}, fragte sie nach.

\enquote{Professor Flitwick war heute auch dabei. Wir haben alle Kombinationen durchgemacht. Als Letztes haben sich beide gegen mich gestellt und mich ordentlich schwitzen lassen. \gst Ich kann mich keinen Millimeter mehr bewegen.}

\begin{rueckblick}
\enquote{Hallo Harry}, sagte Professor Elber. \enquote{Ich habe heute Professor Flitwick mitgebracht. Zusammen werden wir Duelle üben. Das ist etwas, wovon ihre Gruppe profitieren wird.} Harry nickte. \enquote{Wir beginnen ganz klassisch. Zuerst schauen sie zu. Dann wird sich einer von uns gegen sie stellen und sich mit ihnen alleine duellieren. Dann kommt der andere dazu. \gst Verstanden?}

Harry nickte erneut.

\enquote{Also, Mister Potter}, sagte Professor Flitwick. \enquote{Sehen Sie zu und lernen Sie.}

Harry wusste, das Professor Flitwick ein guter Duellant war. Schon während seiner Schulzeit hatte er fast alle Duelle gewonnen. Er sah den beiden zu, wie sie sich gegenüber aufstellten und verbeugten. Er fühlte sich an sein zweites Schuljahr erinnert, in dem er von Lockhart einmal solche Stunden genossen hatte. Es schüttelte ihn innerlich, als er an diesen Blender dachte.

Beide Kontrahenten umkreisten nun einander. Nacheinander warfen sie sich Zauber entgegen. Mal war es ein blauer, dann wieder ein gelber, oder ein purpurner. Doch Harry wusste, dass er sich von den Farben nicht faszinieren lassen durfte. Er zog vorsichtig seinen Zauberstab und hielt ihn in der Hand. Er war darauf vorbereitet, angegriffen zu werden.

Immer mehr und stärker warfen sie sich die Zauber um die Ohren, bis plötzlich Professor Flitwick einen in Richtung Harry warf. Er konnte ihn gerade noch mit einem Schild-Zauber abblocken und zurückwerfen. Nun musste er sich mit dem kleinen Zauberkünstler duellieren. Dieser war flink, das konnte Harry nicht leugnen. Wie ein Wiesel sprang er umher, machte Saltos in der Luft und landete hinter Harry, nachdem er versucht hatte einen Zauber von oben auf Harry loszulassen.

Harry hatte alle Hände voll zu tun, die Zauber abzuwehren, selber anzugreifen und auszuweichen. Er spürte ein eigenartiges Kribbeln hinter sich und konnte gerade noch zur Seite springen. Aber der Zauber, den ihm Professor Elber gegen seinen Rücken warf, streifte noch seinen Arm. Erst langsam und nur wenig, dann aber immer schneller und heftiger spürte er viele kleine Krabbler auf seinem Körper. Er hatte das Gefühl, dass hunderte von Ameisen unter seiner Kleidung waren. Durch Harrys heftige Reaktion und wildes um sich schlagen, wurden die beiden Professoren nachlässig.

Harry bemerkte dies rechtzeitig und simulierte noch eine Weile, nachdem er sich des Zaubers entledigt hatte. In schneller Folge warf er je einen Klammerzauber auf beide Professoren, die sie mehr oder weniger gut wegsteckten. Professor Flitwick konnte noch einen Arm bewegen und sich aus der Umklammerung befreien und Professor Elber hatte es nur eine Körperhälfte gelähmt.

Als Harry stutze, meinte er: \enquote{Ein kleiner, einfacher Trennzauber, damit man nicht vollkommen getroffen wird. Er hilft nur bei einfachen Flüchen.}

Dann griffen beide wieder an. Harry kam ganz schön ins Schwitzen. Er selber konnte auch einige Treffer landen. Als er ziemlich fertig  aussah, beendete Professor Flitwick die Trainingseinheit. Dann bedankte er sich und verließ mit Harry das Gelände, eine einfache Wiese. Professor Elber machte sich in die andere Richtung davon.
\end{rueckblick}

Langsam fing er an, seine Schuhe mit den Füßen auszuziehen. Tamara saß in einem Sessel ihm gegenüber und sah ihn nur über ihre Zeitung hinweg an. Sein Kopf fiel fast automatisch nach hinten und seine Augen schlossen sich. Doch er war immer noch wach, dachte er.

Als er wieder zu sich kam, spürte er ein Kribbeln an seinen Füßen. Seine Beine wurden gerade durch geknetet und seine Schultern massiert. Ein wohliges Gefühl sickerte durch seinen Körper in seine Gedanken. Dann wurde ihm schwarz vor Augen. Zumindest hatte er das Gefühl, da er seine Augen immer noch geschlossen hielt. Nach wenigen Minuten wurde er wieder klar im Kopf. Er öffnete seine Augen und sah nach oben. Parvati lächelte ihn an. Er lächelte zurück und schloss kurz seine Augen. Dann hob er seinen Kopf und sah Ginny und Hermine, die sich um seine Beine kümmerten. Jetzt war er bei vollem Verstand.

Ruckartig zog er sich zusammen. \enquote{Mädels, hört auf.} Doch noch reagierten sie nicht. \enquote{Mädels bitte}, sagte er aufgewühlt.

Etwas hatte sich verändert, spürte er. Er versuchte sie abzuschütteln. Zeitgleich schlugen die Flammen im Kamin höher. Dadurch aufgeschreckt, ließen sie von ihm ab.

Tamara grinste nur. Harry grinste zurück. \enquote{Ist das bei allen Jungs so?}, fragte sie keck.

\enquote{Was meinst du?}

\enquote{Dass die Mädchen einen massieren, wenn man körperlich fertig ist.}

Er wusste, dass er sich für die folgende Aussage armknuffe einfangen würde, aber das war es ihm wert. \enquote{Bei mir ist es jedenfalls so. Mir kann halt kein Mädchen widerstehen.}

\enquote{Was bist du wieder eingebildet}, antwortete Ginny.

Hermine stand auf, schnappte sich ein Buch und verließ den Raum. Ron folgte ihr kurz darauf.

Und Parvati reagiert, wie es Harry nicht erwartet hätte. Sie beugte sich vorne über und kam seinem Gesicht näher. \enquote{Du siehst auch zum Anbeißen aus.}

Harry schreckte zur Seite und fielt vom Sofa herunter. Parvati begann schallend zu lachen. \enquote{Erst einen auf Macho machen und wenn es ernst wird, den Schwanz einziehen.} Dann ging sie zu Lavender, um mit ihr Hausaufgaben zu machen.

Tamara sah ihn immer noch an. \enquote{Hilfst du mir bei meinen Hausaufgaben? Ich habe da ein Problem.}

\enquote{Nur, wenn ich nichts tun muss, sondern nur denken.} Dann ließ er einfach seinen Oberkörper auf den Teppich sinken und schloss seine Augen. \enquote{Dann frag mich was. Vielleicht kann ich dir helfen.}

\trenn

Harry aß gedankenverloren an einem Apfel, blätterte gerade durch das Buch aus der Bibliothek und schaute immer mal wieder auf den großen See am Fuße Hogwarts. Plötzlich stutze er. Er sah jemanden auf der Wasseroberfläche stehen. Zumindest dachte er das. Das Buch musste warten. Er ging durch die Flure des Schlosses, um der Sache auf den Grund zu gehen.

Kurz vor dem Ausgang des Schlosses traf er auf Ron.

\enquote{Hi Harry. Wohin gehst du?}

\enquote{Ich gehe zum See. Ich habe etwas Komisches gesehen: Jemand der auf dem Wasser läuft.}

\enquote{Du willst mich verarschen!}, sagte Ron.

\enquote{Nein. Ich habe es gesehen. Ich weiß nicht, ob ich mir das nur eingebildet habe oder nicht. Deshalb möchte ich dem nachgehen. Und ein Spaziergang wäre nicht verkehrt.}

\enquote{Ich begleite dich, Harry.}

Und so liefen Harry und Ron den schmalen Weg zum See hinunter.

\enquote{Hey, das sieht gerade so aus, als ob jemand auf dem Wasser steht und sich bewegt.}

\enquote{Genau das habe ich gemeint.}

Noch immer waren sie nicht nahe genug, um etwas zu sehen. Erst als sie vor dem See standen, konnten sie die Gestalt durch ein Loch im umgebenden Buschwerk erkennen. Sie stand tatsächlich auf dem Wasser. Der Kleidung nach war es ihr Lehrer in VgddK. Er hielt in einer Hand einen langen Stab, an dessen einem Ende eine Art schmale Schaufel befestigt war und an dessen anderem Ende eine Art Elektroschocker war. Zumindest sah es so aus. Er bewegte sich wie ein Martial Arts-Kämpfer mit seinem Stab und schien jemanden anzugreifen. Ron und Harry konnten aber aufgrund der Wasserpflanzen den angegriffenen nicht sehen, aber durch einen schmalen Spalt in der Wasserbepflanzung, mit wem er kämpfte. Plötzlich bewegten sich die Wasserpflanzen und bogen ihre Triebe zur Seite um Platz zu machen. Erstaunt sah Harry zu Ron in der Erwartung, dass er ebenso erstaunt sei. Doch Ron grinste leicht und steckte seinen Zauberstab wieder ein.

Jetzt hatten sie eine bessere Sicht. Beide waren erstaunt, dass sich ihr Professor scheinbar mit Parvati und Padma Patil duellierte. Beide hatten ebenfalls so einen Stab. Professor Elber schleuderte mit der Schaufel immer wieder eine Menge Wasser auf die beiden Zwillinge. Durch das Wasser leicht abgelenkt konnte er mit einer Vorwärtsbewegung seiner Hand, die beiden zurückdrängten. Es war so, als ob sie ein Luftstoß zurück blies. Plötzlich kam von der Rückseite ein weiterer Wasserstoß auf sie zu. Professor Snape kam ins Bild und griff nun ebenfalls an. Harry und Ron hatten ihn vorher nicht gesehen. Umso überraschter waren sie, dass sie jetzt zu viert kämpften.

Verwirrt schauten sich die beiden kurz an und gingen dann zum Gegenangriff über. Abwechselnd schleuderten sie eine Menge Wasser auf ihre beiden Angreifer. Professor Elber und Professor Snape mussten die Wassermassen mit ihren Schaufeln abwehren. Während dessen schleuderten die beiden Patil-Zwillinge Stromstöße von der anderen Seite der Stäbe. Professor Elber warf es zurück und er kam mit dem Rücken zuerst auf der Wasseroberfläche auf. Nach wenigen Sekunden, in denen die Zwillinge einen zufriedenen Eindruck machten, sank er in den See ein. Es dauerte eine Weile bis den beiden mulmig wurde und sie sich aufmachten ihn zu suchen. Doch er kam hinter ihnen wieder hoch und schleuderte ihnen je einen Stromstoß in den Rücken. Jetzt standen beide Angreifer vor ihnen. Nach ein paar Sekunden schickte er wieder ein paar Stöße, die die Zwillinge aber gekonnt mit der Spitze ihrer Kampfstäbe abwehrten. Professor Snape schickte mit einer Drehung eine Menge an Wasser hinterher, was die beiden Zwillinge mit einem Schild-Zauber parierten.

Professor Elber stieß mit seinem Stab kurz auf der Wasseroberfläche auf, wodurch sich eine kreisförmige Welle ausbreitete und vom Schild-Zauber abgehalten wurde. Kurz danach begann sich das Innere des Schildes von unten her mit Wasser zu füllen. Unaufhaltsam stieg das Wasser im Inneren des Schildes hoch. Professor Elber richtete die spratzelnde Seite seines Kampfstabes auf die beiden Zwillinge. Er schien nur darauf zu warten, dass sie den Zauber fallen ließen, weil sie Angst hatten zu ertrinken. Als es nicht mehr ging, warfen die Patil-Zwillinge ihre Stäbe nach außen und sofort begann der Wasserstand sich innerhalb des Schildes zu verringern, bis er die Seeoberfläche wieder erreicht hatte.

Der Schild löste sich auf und die beiden Zwillinge hoben ihre Stäbe wieder auf. Danach gingen alle zum Rand des Sees, wo Harry und Ron standen.

\enquote{Was war das denn?}, fragte Ron.

\enquote{Magi-Fu}, antwortete Professor Elber. \enquote{Eine spezielle magische Kampfsportart. Ich habe sie entwickelt, als ich einige Zeit in Asien unterwegs war und mich mit den dortigen Kampfsportarten auseinandergesetzt habe.}

\enquote{Können wir das auch lernen?}, fragte Ron begeistert.

\enquote{Wir?} Harry war entsetzt.

\enquote{Nur, wenn sie sich mit asiatischen Kampfsportarten auskennen. Die beiden Zwillinge hier beherrschen sie, deswegen habe ich auch angefangen sie zu unterrichten.} Und er fügte leise für Ron und Harry hinzu. \enquote{Und um Übungspartner zu haben.}

Dann schaute Harry fragend zu Professor Snape. \enquote{Mein Vater war begeisterter Martial-Arts Kämpfer}, war das Einzige, was Professor Snape sagte, bevor er an ihnen vorbeiging.

\trenn

Schweigend übergab Harry am Abend seinen Zauberstab der geierartigen Hexe in der Bibliothek und ging wortlos zu seinem Regal. Wieder sortierte er die Bücher ein. Und wieder fand er interessante Bücher, die er durchblätterte und einige Sachen fand, die er später einmal durchlesen wollte. Nachdem er für heute seine Strafe abgearbeitet hatte, nahm er eines der Bücher zu Madame Pince mit vor, um es sich auszuleihen. Als sie den Titel sah, zog sie nur eine Augenbraue hoch, sagte aber nichts.

Nach dem Essen nahm Harry Ron und Hermine mit in ein leeres Klassenzimmer. Auf dem Weg dorthin hörten sie mehrmals etwas dumpf aufschlagen und nach jedem dritten Mal eine schnelle, hohe, aber melodiöse Tonfolge. Die drei folgten den Geräuschen. Sie schlichen sich an eine Ecke ran, da sie den Geräuschen immer näher kamen. In einem kleinen Gang, der in einer Sackgasse endete und nur etwa fünf Meter lang war, standen der Schulleiter und ein Lehrer und warfen Pfeile auf eine Scheibe mit farbigen Ringen und einem Metallgitter darüber, welche auf Augenhöhe angebracht war. In etwa zwei Metern Höhe waren hölzerne Plaketten angebracht, auf denen der Name des Spielers und darunter die Punkte standen. Harry las: \accentuate{Dumbledore} und \accentuate{Elber}.

Auf Dumbledores Schild waren 140 Punkte und auf dem Schild von Elber 150 Punkte.

\gedanke{Anscheinend führt Elber}, dachte Harry.

Dumbledore warf einen Pfeil, traf und die Anzeige zog sofort die Punkte ab. Jetzt standen nur noch 60 Punkte auf Dumbledores Schild.

\gedanke{Ups, die werden abgezogen}, dachte sich Harry und sah kurz zu Ron und Hermine, welche gebannt zusahen.

Dumbledore warf die restlichen Pfeile, worauf noch zehn Punkte übrig blieben. Er ging nach vorne und zog die Pfeile aus der runden Scheibe und ging dann zurück hinter die gemalte Linie.

Dann war Elber dran. Er warf eine 100, und zweimal eine 25. Dann standen auf seinem Schild Null Punkte und er gewann.

Er ging nach vorne und zog seine Pfeile heraus. Als er wieder hinter der Linie stand, begann eine neue Runde und Dumbledore begann zu werfen. Währenddessen entwickelte sich eine Unterhaltung zwischen den beiden.

\enquote{Sagen Sie mal Frederick\abs Ich habe gehört, dass sie die Schüler auch teilweise in die Geheimnisse der dunklen Künste einweihen.}

\geraeusch{Klonk}, machte es und ein Pfeil von Professor Elber flog daneben. Leicht erregt und mit zusammen gepressten Zähnen sagte er: \enquote{Es gibt keine dunkle Magie. Es gibt nur die Intention dessen, der sie anwendet \gst Herrschaft.}

\enquote{Ich habe ein ungutes Gefühl dabei.}

\enquote{Wieso? Was ist daran schlimm? Man muss wissen, womit man es zu tun hat. Die Schüler sollen sich wehren können.} Er warf eine 80, danach eine 320.

\enquote{Die Schüler haben Angst vor ihnen.} Es klingelte und für Dumbledore wurden 240 Punkte abgezogen.

\enquote{Guter Wurf, Albus.}

\enquote{Danke.}

\enquote{Wieso haben die Schüler Angst vor mir?}, fragte Professor Elber weiter.

\enquote{Weil sie die dunklen\abs Sie wissen, was ich meine. Sie bringen ihnen Magie bei, die Voldemort und seine Todesser verwenden. Viele glauben, sie sind ein Todesser. Das glauben auch viele meiner Kollegen.}

\geraeusch{Klonk}. Wieder ging ein Wurf daneben. Zornig warf er beide Pfeile auf die Scheibe und die Anzeige verlöschte und es gab, erschien: \accentuate{ungültiger Wurf}, samt passenden Geräuschen.

Tief und langsam durchatmend sah er Dumbledore an. \enquote{Ich werde mal mit den Kollegen reden müssen. Vielleicht war ich nicht deutlich genug. \gst Aber dass die Schüler es nicht begriffen haben? Ich habe ihnen doch in der ersten Stunde alles erklärt.}

\enquote{Offenbar nicht deutlich genug. Aber mir wäre es lieber, wenn sie nochmal vor allen\abs}

\enquote{Vor der gesamten Schule? Sie wissen doch, wie ungern ich das mache.}

\enquote{Das sah letztes Mal aber anders aus.}

\enquote{Da musste ich. Und dann soll es nicht aussehen, als hätte ich Angst.}

\enquote{Dann los, bald ist Abendessen.}

\enquote{Heute? Ausgeschlossen, dienstags habe ich immer einen wichtigen Termin nach dem Essen. Was glaubst du, warum ich immer so schnell weg bin? Morgen, Albus.}

\enquote{Also gut.}

Dumbledore kam an die Wand nach vorne und steckte seine Pfeile in den kleinen Vorsprung, während Professor Elber sie aus der Scheibe zog und ebenfalls ablegte. Dann drehten sich beide herum und kamen aus dem Gang. Hermine, Harry und Ron schreckten zurück und liefen dann normal nach vorne. Die Scheibe nicht beachtend grüßten sie ihre Lehrer und gingen normal weiter.

\enquote{Haben Sie bemerkt, dass sie uns belauscht haben?}, hörte Harry Dumbledore sagen. Dann waren sie zu weit weg, um noch etwas zu verstehen.

Nachdenklich gingen die drei in den Gemeinschaftsraum, wo sie darüber sinnierten. Etwas später holte Harry sein Buch und begann endlich zu lesen.

\begin{buch}
Neben vielen anderen Orten, wo Wissen gesammelt wird, gibt es einen, der das magische Wissen unserer Gesellschaft enthält. Die Mondbibliothek. In diesem Werk habe ich alle Informationen gesammelt, die ich finden und herausfinden konnte. Leider sieht man schon an der Dicke des Buches, dass es nicht allzu viele Informationen sind, die ich habe. Selbst den Namen herauszufinden hatte mich eine Weile beschäftigt. Bedauerlicherweise konnte, oder wollte, mir keiner sagen, wo sich diese geheimnisvolle Bibliothek befindet. Wenn ich nur an alle die Bücher, sofern es welche gibt, denke, dann läuft mir schon das Wasser\abs aber ich schweife ab. Ich habe ein Rätsel gefunden, dass ich bislang nicht lösen konnte. Es lautet: \enquote{Um dorthin zu gelangen, nimm den direkten Weg. Wähle den richtigen Zeitpunkt und du wirst dein Ziel erreichen. Übe das Reisen ohne Zeit und du wirst erkennen, wann und wohin du musst. Das Wissen wartet auf dich und wird ständig wachsen. Ein Leben reicht nicht aus, um alle Geheimnisse zu ergründen. Aber hüte dich vor dem Wächter. Wenn du ihn gegen dich hast, hilft auch keine Flucht mehr. Dann hilft nur noch beten und der Übergang in das Danach.}
\end{buch}

Auf den nachfolgenden Seiten fand Harry denselben Text in mehreren Sprachen. Es waren auch Bilder vorhanden. Die Bibliothek, wie sie sich der Autor wohl vorstellt. Es waren Kupferstiche. Einer von vielen Gängen mit einer endlos scheinenden Buchreihe war zu sehen, denn es schien, als laufe man durch die Bücherreihen hindurch. Ein anderes Bild zeigte eine leuchtende Kugel inmitten eines Raumes.

Dann las er noch unterhalb des Bildes:

\begin{buch}
Nur, wer den richtigen Blick auf die Bibliothek hat, wird sie finden. Das Wissen zu erlangen kann mühselig oder leicht sein. Suche nicht, finde sie.
\end{buch}

Da Hermine immer wieder sehr interessiert schaute, was Harry las, gab er ihr das Buch, nachdem er fertig war.

Nun dachte er nach. Er schloss die Augen und merkte nicht, wie sich jemand neben ihn setzte. Irgendwann spürte er nur eine Berührung. \gedanke{Mondbibliothek. Vermutlich wegen der hellen Gänge. Das Holz der Regale in den Reihen sah auf dem Bild sehr hell aus. Nein, das war vielleicht nur die Vorstellung des Künstlers. Mondbibliothek. Der Mond ist unser Trabant. Etwa vierhundert-tausend Kilometer entfernt. Er bremst die Erde. Er sorgt für Ebbe und Flut. Er verdunkelt alle wie viele Jahre die Sonne? Der richtige Zeitpunkt. Eine Sonnenfinsternis? Warum Mond?}

Er öffnete die Augen und sah aus dem Augenwinkel heraus jemand neben ihm sitzen. Es war Ginny. Sie sah ihn an und lächelte. Dann wandte sie ihren Blick ab. Harry nutze die Gelegenheit und gab ihr einen Kuss auf die Wange. Überrascht sah sie ihn an.  \enquote{Danke}, war das einzige, was Harry sagte.

\enquote{Wofür?}, fragte sie nach.

\enquote{Denk nach} und an Hermine gewandt: \enquote{Bist du fertig mit dem Buch?}

Diese nickte und gab ihm das Buch zurück. Harry kopierte die Seite und die Bilder auf beide Seiten eines Pergaments und schob es dann in seine Tasche.

\enquote{Ich gehe dann mal nach oben}, sagte er und verschwand.

\enquote{Für das Geburtstagsgeschenk?}, rief ihm Ginny noch nach.

\enquote{Nein.}

Auf seinem Weg nach oben grinste er in sich hinein. \gedanke{Mal sehen, wann Ginny darauf kommt.}

Nachdem er sich schlafen gelegt hatte, begann er zu träumen. Er träumte von letztem Jahr, als er bei Umbridge gerade Nachsitzen hatte.

\begin{traum}
In ihrem hässlichen Rosa stand sie da und starrte aus dem Fenster, während sich Harry die Hand wund schrieb. Plötzlich kam Hagrid mit seinem rosa Schirm herein. Harrys Blick fiel auf ihn. \gedanke{Rosa}, dachte er.
\end{traum}

Dann schreckte er hoch. Irgendwo musste er sich das ganze notieren, damit er es nicht vergaß. Sein neues Tagebuch-Set kam ihm gerade recht. Er führte zwar kein Tagebuch, aber als Notizbuch wusste er es zu schätzen. Interessante oder auch komische Träume hielt er fest. Er schrieb auch auf, was Hagrid sagte. \gedanke{Manchmal muss man seine Träume auch wörtlich nehmen. Hagrids Rosa Schirm. Er hat eine Bedeutung. Wenn mir nur einfallen würde, welche!} Dieser Traum war vollkommen daneben, aber in einem Traum ist alles möglich. Er kann durchaus eine Bedeutung haben.

Er schlief wieder ein und erwachte erst am nächsten Morgen. Er ging seine Notizen noch einmal durch und überlegte, was es mit dem rosa Schirm auf sich hatte. \gedanke{Hagrid hat einen. Darin hat er seinen Zauberstab. Er hat ihn geklebt. Illegal.} Das war es. Sein Unterbewusstsein wollte ihm sagen, dass er Hagrid helfen sollte. Aber wie? Der einzige Beweis, den er je hatte, war Voldemorts Tagebuch. Selbst, wenn es noch intakt wäre; Voldemorts jüngeres selbst würde keinem anderen zeigen, was passiert war. Wie also Hagrid helfen?

Doch er wurde von Ron unterbrochen und ging mit ihm, um zu frühstücken.

\trenn

Harry befand sich im Krankenflügel der Schule, nachdem ihn aufgrund seiner Unaufmerksamkeit eine Pflanze gebissen hatte. Seine linke Hand bis hinauf zu seiner Beuge schwoll auf das Doppelte an.

\enquote{Es ist doch jedes Jahr dasselbe. Immer sind ein oder zwei dabei, die es schaffen sich beißen zu lassen}, sagte Madame Pomfrey die Augen rollend.

\enquote{Es ist nicht so, dass ich darauf gewartet habe oder mich absichtlich habe beißen lassen, Madame Pomfrey}, antwortete Harry.

Er hatte nun das erste Mal das Gefühl, dass sie lächelte. Sonst sah er sie immer nur sehr ernst. Sie verschwand kurz in ihrem Büro. Währenddessen setzte sich Harry auf ein freies Bett und entfernte den Rest seiner schon reißenden, spannenden Kleidung, da sein Arm geschwollen war, mit seinem Zauberstab.

Zu allem Überfluss hatten sie heute auch noch Heilkunde, zusammen mit den Ravenclaws. Die Glocke läutete erneut und somit begann der Unterricht. Die Klasse war vollständig versammelt, als Madame Pomfrey wieder hereinkam. \enquote{Ah ja}, sagte sie. \enquote{Gut gut.} Sie lief zu Harry und nahm ihn gleich als Demonstrationsobjekt her.

\enquote{Kommen Sie ruhig näher. Als Heiler dürfen sie keine Angst vor dem Patienten haben. In diesem Fall ist es nur eine simple Schwellung einer Todarm-Wurzel. Wenn sie nicht rechtzeitig behandelt wird, dann stirb der Arm ab. Aber das dauert. Sie können mit bis zu einem Monat Verzögerung die Heilsalbe aufbringen. Nun gut, wenn sie bis dahin die ständig wachsenden Schmerzen aushalten.} Sie öffnete die Salbendose und nahm einen ordentlichen Schwung heraus. Dann strich sie Harrys Hand damit ein.

Harry war überrascht, dass die Salbe nicht kalt war. Aber trotzdem lief ihm ein Schauer über den Rücken. Er genoss die Berührung von Madame Pomfrey. Und das irritierte ihn.

\enquote{Wer will auch mal?}, fragte sie in die Runde. Harry riss seine Augen auf. Eine einzelne Ravenclaw trat hervor und Madame Pomfrey reichte ihr die Salbe.

\enquote{Nehmen sie nur eine ordentliche Portion auf ihre Hand und streichen sie damit den Arm ein. Bedecken Sie ihn richtig damit. Es sollte schon ein halber Millimeter sein. Das zieht recht schnell in die Haut ein. Sehen Sie Miss Elfwood, die Hand ist schon wieder frei von der Salbe.}
% Linda Elfwood

Sie nickte und nahm eine ordentliche Portion heraus. Dann verteilte sie die Salbe auf Harrys Arm. \enquote{Etwas weniger}, korrigierte sie Madame Pomfrey. Sie verteilte die Menge, die sie genommen hatte nun über den ganzen Arm, anstatt noch einmal nachzufassen. Binnen zehn Minuten war die Schwellung komplett verschwunden.

Dann dachte Harry, er höre nicht richtig, als Madame Pomfrey sagte: \enquote{Und wenn sie zu ihrem Patienten besonders nett sind und ihn auf die Wange küssen, dann wirkt es noch schneller.} Sie tat wie geheißen und küsste Harry auf die Wange. Danach drehte sie sich zu Madame Pomfrey um, um ihre Meinung zu hören, ob es so richtig gewesen sei.

Madame Pomfrey zog ihre Augenbraue hoch und sagte mit gewisser Heiterkeit. \enquote{Das war eigentlich als Scherz gedacht. Ich hatte nicht erwartet, dass sie das auch wirklich tun.} Die Schülerin, Linda hieß sie, wie ihm Luna mal gesagt hatte, wurde sofort rot.

Madame Pomfrey fuhr fort. \enquote{Es gibt nur zwei mir bekannte Verletzungen, deren Heilung durch einen Kuss beschleunigt werden, aber diese gehört nachweislich nicht dazu. \gst Trösten Sie sich Miss Elfwood. Mit mir hat man während meiner Ausbildung denselben Scherz getrieben.} Dann wandte sie sich zur Klasse und meinte: \enquote{Lassen sie sich also nicht veralbern, falls sie einmal den Beruf des Heilers antreten wollen.}

\trenn

Harry hatte seine zweite Strafarbeit hinter sich und machte sich auf den Weg zum dritten Stock im Westflügel. Er klopfte an das Porträt und Luna ließ ihn kurz darauf herein.

\enquote{Wie bist du\abs?}

\enquote{Shh!}, gab Luna zurück und küsste ihn. Dann zog sie ihn zu sich und nahm ihn mit in ihren Raum. Nach einem gemütlichen Gespräch im  Bett, begannen sich beide umzudrehen und einzuschlafen.

Mitten in der Nacht wachte Harry auf und hob seinen Kopf. Am Bettende sah er einen kleinen Elfen herumlaufen und aufräumen. Harry schaute ihm eine kleine Weile zu und erkannte Kreacher. Er legte seinen Kopf wieder auf das Kissen, schloss die Augen und horchte in die Nacht hinein. \gedanke{Kreacher scheint mich doch irgendwie zu mögen, sonst würde er hier nicht aufräumen.} Als Kreacher endlich fertig war, stand Harry auf um auf die Toilette zu gehen. Als er sich wieder ins Bett legte und zu Luna drehte, drehte sie sich ebenfalls in seine Richtung. Ihr Atem war ruhig und gleichmäßig. Sie schien zu schlafen. Vorsichtig reckte er seinen Kopf nach vorne und gab ihr einen Kuss auf die Stirn. Sie ließ einen wohligen Schnaufer erklingen und kam ihm entgegen. Er lächelte leicht und gab ihr nun einen richtigen Kuss. Wieder gab sie ein wohliges Geräusch von sich. Harry grinste und drehte sich herum. Langsam tastete sie sich, scheinbar immer noch schlafend, an ihn heran und legte einen Arm um seine Hüfte. Ihr warmer Atem ließ seine Nackenhärchen aufstehen. In ihm breitete sich ein langsamer erholsamer Schlaf aus.

Am Morgen nach seiner vierten gemeinsamen Nacht mit Luna, welche ähnlich ablief wie die davor, hatte es Harry geschafft, sich noch ein bisschen hinzulegen, bevor es zum Frühstück ging und so eine weitere Unterredung mit Hermine oder Ron vermieden. Abermals in der Großen Halle angekommen, traf sein Blick den von Luna und er wusste genau was sie dachte. Mitten während des Frühstückes war es wieder so weit und die Posteulen kamen. Harry schaute zu ihnen hinauf und ihm fielen zwei seltsam klein wirkende, gänzlich grau aussehende Eulen auf. Es waren Schuleulen, die sonst sehr selten Post brachten. Eine flog auf Harry zu und landete beinahe auf seinem Teller. Sie hatte einen Brief in ihrem Schnabel und auf dem ungewöhnlich grau aussehenden Umschlag stand in krakeliger Handschrift:

\begin{brief}
Harry Potter,\\
Gryffindor\\
Große Halle
\end{brief}

Harry nahm der Eule den Brief ab, gab ihr etwas Schinken vom Frühstückstisch und ließ sie wieder fliegen. Er drehte den Brief um, um nach dem Absender zu schauen. \accentuate{Dobby} stand dort. Harry drehte sich um und schaute zu Luna, die scheinbar auch einen Brief bekommen hatte. Er steckte den Brief in seine Tasche und frühstückte weiter. Die Fragen seiner Banknachbarn ignorierte er und verschwand nach dem Essen. Er erzählte Ron und Hermine, dass er noch etwas Quidditch-Training benötige und danach im Gemeinschaftsraum seine Hausaufgaben machen wollte. Luna wusste bereits, was er vorhatte und stand auf, um in den dritten Stock zu gehen. Dort traf sie auch schon auf Harry, den sie die letzten Schritte begleitete. Im Gemeinschaftsraum der Paare öffneten Harry und Luna ihre Briefe und Harry las seinen vor.

\begin{brief}
Hallo Harry Potter, Sir.

Dobby freut sich ihnen mitzuteilen, dass er nunmehr bereit ist, mit den anderen Hauselfen die anderen Paare in den fünften Gemeinschaftsraum zu lassen. Entsprechende Briefe, sie hier nächsten Sonntag zu treffen, werden morgen rausgehen. Bitte weisen Sie und Miss Luna die anderen Paare ein, so wie Dobby es mit ihnen gemacht hat.
\end{brief}

Luna las ihren Brief mit und sagte Harry, dass bei ihr dasselbe drinnen stand. Harry blickte auf, lächelte Luna leicht gequält an, worauf sie aufstand, sich in seinen Schoß setzten und ihm zu spüren gab, dass es schon klappen würde. Harry dachte immer wieder daran, dass er ihre Gedanken lesen konnte, wenn er sie nicht unterdrückte und bemerkte seit einigen Tagen, dass er, wenn er sich besonders anstrengte, auch durch ihre Augen sehen und durch ihre Ohren hören konnte. Luna ging es nicht anders und sie fragte Harry, ob die nächste Partie Schach nicht sie gegen Ron spielen dürfte. Das vertrieb die sorgenvolle Miene von Harrys Gesicht und er gab ihr einen Kuss, den sie erwiderte. Harry ging wie Luna  mit gemischten Gefühlen dem Treffen entgegen, trafen sie dort doch Leute, die in ihren eigenen Häusern waren. Doch die Tatsache, dass dieser Raum ihnen allen eine Zufluchtsstätte bot, gab ihm Trost, sie den Fängen von Mister Filch und den anderen patrouillierenden Lehrern zu entreißen, gab ihm Mut das Ganze durchzustehen. Nach den üblichen Sticheleien Malfoys den Rest der Woche, dem langweiligen Unterricht von Professor Binns und dem übermäßigen Enthusiasmus von Professor Flitwick, kam so langsam aber sicher der Freitag.

Während der Stunde bei Professor Trelawney, erinnerte er sich wie Dean einen grauen Brief von Dobby bekommen hatte. Dean öffnete ihn, stutzte und zeigte mit seinem Zauberstab, seinen Namen nennend, auf das Stückchen Papier. Er schaute erstaunt, als er Dobbys Zeilen las, schaute zu Harry, grinste frech und tippte seinen Brief wieder an, welcher sich begann aufzulösen und mit einem lauten \geraeusch{Plopp} verschwand. Harry hörte an diesem Tag mehrere Plopp-Geräusche, die das übliche Grundrauschen während des Frühstücks übertönten. Harry war gar nicht wohl gewesen, als Dean seinen Brief bekommen hatte. Bekam aber merkwürdigerweise nichts zu hören. Keiner schien ihn darauf anzusprechen, oder schief anzuschauen. Harry fiel erst jetzt auf, dass er keine Ahnung hatte, was Dobby in den Briefen erwähnte und ob er überhaupt seinen oder Lunas Namen nannte. Plötzlich stieß ihn Ron mit seinem Ellenbogen und Harry war gerade wieder bei der Sache, als Professor Trelawney vor ihnen erschien und Harry aufforderte, Ron die Tarot-Karten zu legen. Er ließ Ron die Karten mischen und begann auszuteilen. Er murmelte irgendwas von einer Veränderung in der Liebe und dem Umstand, dass er bald Vater werden würde. Professor Trelawney beglückwünschte ihn auf eine Art und Weise, dass es die ganze Klasse mitbekam. Ron verzog sein Gesicht und blickte Harry böse an. So als ob es seine Schuld gewesen sei.

Die Stunde nach dem Mittagessen bei Professor Snape war auch nicht gerade die Beste, obwohl sich Harry bei seinen Hausaufgaben zu bessern schien. Snape behandelte ihn immer noch äußerst ungerecht und zog ihm bei jeder Kleinigkeit Punkte ab, die er aber scheinbar im Mittel immer wieder durch seine Hausaufgaben ausgleichen konnte. Harry strengte sich zwar etwas mehr an als sonst, aber dass sich das dermaßen auf seine Noten bei Snape auswirken würde hatte er nicht gedacht. Vielleicht liegt es auch nur daran, was er alles durchgemacht hatte, während der letzten Schuljahre (Harry, nicht Snape). Er hatte gegen Voldemort im ersten Schuljahr gekämpft. Im zweiten einen Teil von Voldemorts Seele und einen Basilisken zur Strecke gebracht. Im dritten seinem Paten zur Flucht verholfen. In seinem vierten den Trimagischen~Pokal gewonnen und war knapp Voldemort entkommen und im fünften seinen Paten verloren. Ron weckte ihn wieder aus seinen Gedanken. Gerade noch rechtzeitig, damit Harry seinem Trank noch ein paar Kräuter hinzufügen konnte. Am Ende der Stunde füllte er eine Probe seines Trankes in ein Glas und gab es Snape zur Überprüfung.

\enquote{Potter}, hörte er Snape sagen, als er sich gerade umgedreht hatte. \enquote{Sie haben vergessen ihren Namen draufzuschreiben. Sie haben drei Sekunden Zeit.}

Harry wusste genau, dass er es nicht schaffen würde, seinen Namen auf ein Blatt Papier zu schreiben, um es mit einem Stückchen Faden an sein Glas zu binden. Er drehte sich blitzschnell um, so schnell, dass Snape erschrak, zeigte mit seinem Zauberstab auf das Glas und sprach:

\enquote{Gravure, Harry Potter.}

Er schaute Snape in die Augen während er seinen Zauberstab einsteckte. Dann drehte er sich um und verließ den Raum. Snape nahm sich das Glas und schaute es missmutig an. In das Glas waren die Worte \accentuate{Harry Potter} eingraviert und Snape rief ihm wütend nach: \enquote{Fünf Punkte Abzug für Gryffindor wegen Beschädigen von Schuleigentums.} Harry schmunzelte im Hinausgehen Ron und Hermine zu, denn er war sich sicher, dass er die Punkte mit seinen Hausaufgaben wieder ausgleichen konnte, Snape vielleicht sogar beeindruckt hatte. Er hatte ihn verblüfft! Dabei hatte er den Zauberspruch nur zufällig gefunden, als er in der Bibliothek lustlos in einem Buch blätterte.

In der Zwischenzeit hatte er bereits Professor Sprout und Professor Snape nach einer Erlaubnis für die abgesperrte Abteilung gefragt, war aber immer abgeblitzt. Sogar Professor Dumbledore hielt nichts davon.

\trenn

Als Harry im Gemeinschaftsraum der Gryffindors ankam, war dieser leer, wie ausgestorben. Harry legte seinen Besen vom Quidditch-Training unter sein Bett und ihm kam ein Einfall. Er öffnete seinen Koffer und suchte nach der Karte des Rumtreibers. Als er sie gefunden hatte, ging er zu seiner Tür, verschloss sie mit einem Zauberspruch und setzte sich auf sein Bett. Er tippte die Karte an und sprach: \enquote{Ich schwöre feierlich, ich bin ein Tunichtgut.} Die Karte begann Hogwarts zu zeigen und Harry klappte sie auf. Er suchte nach dem dritten Stock im Westflügel, doch da war nichts. Plötzlich fiel Harry an der Stelle wo das Porträt an der Wand hing, ein kleiner Punkt auf, der sich kaum merkbar von der Karte abhob. Harry stutzte und versuchte ihm mit seinem Zauberstab anzutippen, doch der Punkt färbte sich kurz rot, nur um dann wieder seine ursprüngliche Farbe anzunehmen. Harry überlegte kurz, tippte den Punkt abermals an und sprach dann: \zauber{Aqua Neros!}

Er sah, wie sich aus der Karte ein weiterer Abschnitt erhob und den Gemeinschaftsraum der Paare abbildete. Er sah kleine Fußspuren darin umherlaufen. Kleinere als sonst. Und eine Schrift daneben. Es waren Dobby und Kreacher. \gedanke{Anscheinend räumen die beiden gerade auf.} Da fiel ihm ein, dass er noch nie die Küche auf der Karte gesehen hatte. Er tippte wieder auf das neue Kartenstück, welches verschwand und schaute an die Stelle, wo die Küche war. Da war auch ein kleiner Punkt, den Harry sofort antippte. Das Porträt, welches sonst den Zugang zur Küche versperrte, wurde auf einem weiteren Kartenstück sichtbar. Harry rieb an der Stelle der Birne auf dem Papier herum und die Karte verformte sich. Es erschien ein genauer Umriss der Küche und Harry sah viele kleine Füße umherlaufen. Neben jeder Fußspur ein kleiner Name. Viele davon kannte er nicht aber der Name Winky fiel ihm auf. Harry tippte wieder auf die Karte und das Papier, welches die Küche darstellte, verschwand wieder. Es war so, also wurde das Papier von der Karte aufgesogen. Jetzt wurde Harrys Neugier geweckt und er suchte auf der Karte nach weiteren Punkten. Als er einen fand, nahm er seine Karte von Hogwarts \gst er musste immer noch schmunzelnd, als er daran dachte, wie er sie bekommen hatte \gst und verzeichnete den Punkt. So verfuhr er bei jedem Punkt, den er fand. Sorgfältig trug er sie ein. Ebenso schrieb er die Positionen der Punkte auf ein Pergament, denn die Karte konnte er nicht immer mitnehmen.
Als er alle gefunden hatte, war die Freistunde auch schon wieder vorbei. Harry tippte die Karte mit seinem Zauberstab an und sagte: \enquote{Missetat begangen.}

\begin{rueckblick}
Harry bemerkte, dass er seine Karte des Rumtreibers verloren hatte. Sofort drehte er sich um und ging den Weg zurück. Als er in seinem Klassenzimmer ankam, hatte die Professor Elber bereits in der Hand und saß hinter seinem Schreibpult im Klassenzimmer.

Mit leicht zusammengekniffenen Augen und skeptischem Blick fuhr er mit einer Hand über das dicke Pergament. Dabei murmelte er immer wieder. \enquote{Interessante Schutzzauber\abs keine herkömmlichen\abs was ist das für\abs? Komisch\abs}

Langsam näherte sich Harry seinem Lehrer.

Dieser sah auf und sah ihn fragend an. \enquote{Ja bitte?}

Harry druckste etwas herum und sah immer wieder verstohlen auf das Pergament.

Darauf hin wechselte Professor Elbers Blick ebenfalls zwischen Harry und dem Pergament hin und her. \enquote{Ihres?}, fragte er.

Harry nickte.

\enquote{Beweisen Sie es}, forderte Professor Elber mit leichtem schmunzeln im Gesicht.

Harrys Gesicht versteinerte. \enquote{Ich\abs äh\abs}

Professor Elber gab ihm die Karte.

\enquote{Danke}, sagte Harry und wandte sich ab.

\enquote{Versprechen sie mir etwas!}, forderte er. Harry drehte sich noch einmal um. \enquote{Wenn dieses Dokument schwarze Magie enthält, verwenden Sie es nicht. Ich werde es merken. \gst Und noch etwas. Ich habe keinerlei Kenntnis vom Inhalt.}

Harry nickte und ging. Kurz vor der Tür drehte er sich noch einmal um. \enquote{Sagen Sie mal\abs Ich habe sie vor einiger Zeit mit einer Karte von Hogwarts gesehen, wie sie mit Professor Flitwick und Professor Sprout einen Innenhof freigelegt hatten. Haben Sie da noch eine?}, fragte er schüchtern.

Professor Elber nickte und stand auf. Er verschwand in seinem Büro und kam nach einer guten Minute wieder zurück. In der Hand hatte er eine große Rolle Pergament. Er gab sie an Harry. Dieser bedankte sich artig und ging.
\end{rueckblick}

Die Karte faltete sich wieder zusammen und die Schrift verschwand. Harry legte die Karte zurück in seinen Koffer und faltete sein neues Pergament zusammen, um es in die Tasche zu stecken. Er nahm sich sein Schreibzeug und beschloss, in die Bibliothek zu gehen, um seine Hausaufgaben zu machen. In einigen Tagen konnte er beginnen, sich auf die Suche nach den geheimen, ominösen Punkten der Karte machen.

Auf dem Weg dorthin traf er Professor Elber, der verwirrt um sich schaute, ansonsten war der Gang leer.

\enquote{Sehen Sie auch alles grau?}, fragte er Harry.

\enquote{Die Steinwände sind doch immer grau Professor.}

\enquote{Ich meine nicht die Steinwände.} Er lief zu einem Fenster und sah nach draußen. \enquote{Wenn ich mich umsehe}, begann er und Harry stellte sich neben ihn, \enquote{dann sehe ich grau; graues Gras, grauer Himmel, graue Bäume, alles grau. Selbst sie kommen mir vor, als seinen sie aus einem alten Schwarz-Weiß-Film.}

Harry sah ihn an, als wüsste er nicht, ob er versuchte ihn zu veralbern, oder ob er verrückt geworden war. Oder er sah wirklich nur noch grau-stufig.

\enquote{Ich sehe keinen Unterschied zu früher}, gab Harry zur Antwort.

\enquote{Na ja, danke Harry, ich werde dann mal zu Poppy gehen. Vielleicht weiß sie etwas.} Er drehte sich um und ging.

Harry sah noch ein paar Minuten verwirrt drein, bevor er sich wieder Richtung Bibliothek absetzte.

Auf der Suche nach einem Buch für seine Hausaufgaben sah er eines mit dem Titel: \buchtitel{Grauschleier in praktischer und theoretischer Natur.} Jetzt war sein Interesse geweckt und er nahm es heraus. Darin fand er eine Beschreibung und ein paar Zauber, die diverse Wirkungen hatten. Ein Zauber entfernte die Farbe eines unbewegten Gegenstandes. Ein anderer von lebendem Gewebe, wieder einer von toten Lebewesen. Ein anderer nahm einem die farbige Sicht\abs Das war es doch, was er suchte. Es gab diese Zauber mit zeitlicher Begrenzung, aber auch dauerhafte. Er blätterte weiter und fand noch passende Gegenzauber.

Dann packte er seine Sachen zusammen und nahm das Buch mit auf die Krankenstation. Dort zeigte er es Madame Pomfrey, die sofort einen passenden Gegenzauber versuchte. Scheinbar schlug er fehl.

\enquote{Ich sehe immer noch alles in Grau}, antwortete Professor Elber.

Madame Pomfrey runzelte ihre Stirn und sah erneut in das Buch. Dann fand sie bei den Nebenwirkungen, dass es unter besonderen Umständen bis zu einer Stunde dauern könnte, bis die Wirkung wieder nachließ.

Professor Elber sah sich das Buch an und meinte: \enquote{Ein einfacher Grau-Zauber also\abs} Er musste noch eine Weile warten, bis er die Krankenstation verlassen durfte, also verließ Harry mit dem Buch in der Hand die Krankenstation und machte sich wieder auf den Weg in die Bibliothek. Er stellte das Buch zurück und widmete sich wieder seinen Hausaufgaben.




\begin{kommentar}
Als Harry seine Strafarbeit von Professor Snape in der Bibliothek absitzen muss, liest er zum ersten Mal etwas über Zauber, mit denen man aufgrund von Körperflüssigkeiten oder Hautzellen den Träger identifizieren kann. Das sind die ersten Schritte hin zu forensischen Methoden, die im nächsten Teil der Geschichte wieder aufgegriffen werden, als Harry Unterricht bei Elber erhält.
\end{kommentar}

\begin{kommentar}
Als Harry dann in seinem Bett liegt, erscheint ein kleiner Elf. Frodo. Harry nimmt ihn mit in sein Bett, damit er nicht frieren muss. Damit hat er unbewusst etwas mit seiner Magie angestellt und er beginnt, die Magie der Elfen in sich aufzunehmen. Dadurch hat er nun eine dritte Quelle der Magie. Neben seiner eigenen und der durch Voldemorts Seelenteil auch noch die der Elfen.
\end{kommentar}

\begin{kommentar}
Etwas später duelliert sich Harry mit Flitwick zum ersten Mal. Der kleine Zauberer ist erstaunlich flink und wuselt und springt umher. Eine kleine Hommage an Yoda und sein Duell mit Count Dooku.
\end{kommentar}

\begin{kommentar}
Als ich die Magi-Fu-Szenen auf dem Wasser geschrieben habe, dachte ich an Kung-Fu. Ich wollte es etwas spektakulärer machen, als der klassische Kampfsport. Und auf dem Wasser hat man zusätzliche Möglichkeiten.
\end{kommentar}

\begin{kommentar}
Und wieder hat Harry eine Strafarbeit in der Bibliothek. Er muss Bücher manuell sortieren. Eine schöne versteckte Gelegenheit, dass Snape ihm einige Bücher vorschlagen kann, die Harry interessieren dürften.
\end{kommentar}

\begin{kommentar}
Um noch leichte Unsicherheit zu schüren, fand ich es durchaus angemessen, dass Elber sich mit Malfoy Senior trifft. Einmal in der Woche spielen sie Schach. Sogar während Malfoy in Askaban saß.
\end{kommentar}
