\chapter{Auf und ab}


Er war auf dem Weg zum Astronomieturm und leider viel zu spät, um pünktlich zum Unterricht zu erscheinen. Der Weg hoch zum Turm war lang und anstrengend. Fast wäre er von hinten in Professor Elber und Professor Dumbledore gerannt. \enquote{Entschuldigung} sagte Harry, als er sie sah.

\enquote{Wohin so schnell?}, fragte Dumbledore ihn.

\enquote{Zum Astronomieturm}, antwortete Harry.

\enquote{Wir auch}, entgegnete ihm Dumbledore. \enquote{Lust ein Stück mit uns zu laufen? Wäre außerdem eine gute Entschuldigung fürs zu spät kommen!} Sein Schulleiter blitze ihn an.

\gedanke{Warum nicht}, dachte sich Harry und ging neben seinen beiden Professoren her.

\enquote{Du hast dich von deiner Freundin getrennt, habe ich gehört?}

Harry fragte sich, woher Dumbledore das schon wieder wusste. \enquote{Ja, es läuft gerade nicht so gut zwischen uns. Etwas Abstand tut uns ganz gut.}

Professor Elber bog plötzlich ab. Dumbledore und Harry sahen ihn nur staunend an.

\enquote{Zum Astronomieturm geht es da lang, Frederick}, sagte ihm Dumbledore.

Professor Elber drehte sich um und entgegnete dann: \enquote{Ich laufe doch nicht die ganzen Treppen, nur um Aurora diese Dokumente zu geben.}

Erst jetzt bemerkte Harry die große Mappe, welche Professor Elber in der Hand hatte. Professor Elber drehte sich wieder um und ging weiter. Dumbledore und Harry folgten ihm. Kurz bevor sie unter einem der vielen Torbogen, welche über das ganze Schloss verteilt waren, ankamen, blieb Professor Elber stehen und drehte sich nach rechts.

Harry kannte diese Torbogen genau. Für ihn erfüllten sie immer nur stilistische Zwecke. Wenn man unter so einem Torbogen stand und sich nach rechts drehte, waren in etwa 1,60 m Höhe beidseitig Steine, die aus der normalen Mauer herausragten und die gleiche Höhe zum Torbogen hatten. Auf der linken Seite war ein schmaler Stein und auf der rechten Seite ein breiter Stein. Drehte man sich im Torbogen um, dann war der schmale Stein natürlich auf der anderen Seite.

Professor Elber zählt vom breiten Stein fünf weiter und drückte diesen dann in die Wand. Er gab nach und nach ca. 5 Sekunden öffnete sich die Wand. Er schritt in den quadratischen Raum von ca. 4 Quadratmetern Grundfläche, drehte sich um und sah wieder nach draußen. Harry und Dumbledore waren überrascht, aber folgten brav und drehten sich ebenfalls um. Im Inneren waren nur glatte Steinwände zu sehen.

Nachdem sich Harry aber umgedreht hatte, bemerkte er auf etwa Augenhöhe einen großen glasierten Stein, der ein merkwürdiges Symbol zeigte. Auf Handhöhe war ein Mosaik abgebildet. Es bestand aus vielen kleinen bunten Steinen mit ebenso vielen unterschiedlichen Symbolen darauf. Es stellte keinerlei Figur oder Form dar. Professor Elber suchte kurz und drückte dann auf das Mosaik, welches Harry verdächtig an einen Turm mit nebenstehendem Fernrohr erinnerte. Der Stein versank ein Stück und blieb gedrückt in der Wand. Die Mauer schloss sich auf die gleiche Art und Weise wie sie sich schon zuvor geöffnet hatte. Harry erinnerte sich an seinen ersten Besuch mit Hagrid in der Winkelgasse zurück.

Die Wand hatte sich nun geschlossen und die drei standen schweigend im nun schwach erleuchteten Raum. Harry konnte keine Lichtquelle ausmachen, aber man konnte alles klar und deutlich erkennen. Der Boden fing leicht an zu vibrieren und das große Symbol auf Augenhöhe an der Wand begann seine Form, seine Farbe und sein Symbol beständig zu verändern.

Als es so aussah, wie eine vergrößerte Form des Symbols welches Professor Elber gedrückt hatte, hörte das Vibrieren des Bodens auf. Die Wand öffnete sich wieder und Professor Elber stieg aus und bog sofort nach links ab. Immer noch völlig erstaunt schauten sich Dumbledore und Harry an. Harry hatte diesen Zustand schneller überwunden und verließ ebenfalls den Raum, gefolgt von Dumbledore.

Nach wenigen Metern bog Professor Elber abermals nach links und jetzt eröffnete sich der runde Astronomieturm. Sie waren kurz vor dem Klassenzimmer angelangt und sahen nach unten. Von dort konnten sie die Stimmen der Schüler erkennen, die sich auf den Weg nach oben machten. Professor Elber öffnete auf der linken Seite die Tür und trat ein. Harry und Dumbledore folgten ihm. Harry warf seine Schultasche auf seinen Platz und schaute Professor Elber zu.

\enquote{Aurora \gst Aurora.} Nichts geschah. Dann sagte er nach einer kleinen Pause etwas, was Harry die Sprache verschlug. \enquote{Aurora Schätzchen.}

Es dauerte keine 5 Sekunden, da stand auch schon Professor Sinistra vor ihm und meinte mit leicht erröteten Wangen. \enquote{Frederick, Sie alter Charmeur.} Er überreichte ihr die Mappe, worauf sie sich bedankte und wieder verschwand. Professor Dumbledore betrachtete das ganze mit kindlichem Interesse. Professor Elber drehte sich zu Harry und meinte nur. \enquote{Wenn Sie nicht reagieren, dann muss man dem Ganzen etwas Nachdruck verleihen.} Dann verließ er den Raum und ging. Professor Dumbledore stellte sich an eines der Fenster des Turmes und sah verträumt nach draußen. Harry verließ das Klassenzimmer wieder und schaute die Wendeltreppe entlang nach unten, lehnte sich auf das Geländer und sah seinen Klassenkameraden zu, die sich noch immer die Treppe hoch quälten. Er musste schmunzeln. Noch hatte er keine Ahnung, dass er keine Minute zu früh von den Hogwarts-Aufzügen erfahren hatte. Zumindest wusste er jetzt von einem.

Die nächste Zeit würde er kein Wort darüber gegenüber seinen Klassenkameraden verlieren. Er wollte zuerst die Ziele erkunden.

Ron kam oben an und sah erstaunt in Harrys Gesicht. \enquote{Du warst doch hinter mir. Wie bist du\abs}, fragte er.

\enquote{Ich habe dich auf halber Strecke überholt Ron. Nicht mitbekommen?}, grinste Harry ihn an.

Ron sah ihn nur erstaunt an. Harry ging in das Klassenzimmer und setzt sich. Professor Dumbledore verließ den Raum, um den Unterricht nicht zu stören. Sie mussten wieder die Bahnen der Planeten berechnen. Dieses Mal die des Uranus. Alles verlief ruhig und die Schüler zeichneten ihre Bahnen und sahen immer mal wieder durch ihre Fernrohre. Plötzlich fiel Parvati von ihrem Stuhl und krümmte sich. Schaum quoll aus ihrem Mund. Sie zuckte, als hätte sie einen Anfall. \enquote{Professor}, rief Lavender. \enquote{Kommen Sie schnell.} Professor Sinistra kam angerannt und sah Parvati voller entsetzen an. Lavender sah nun durch Parvatis Fernrohr und fing kurz darauf ebenfalls an, dieselben Symptome zu zeigen. Auch sie fiel zusammen und begann mit Schaum vor dem Mund zu zucken.

\enquote{Jemand möge Madame Pomfrey holen}, schrie Professor Sinistra durch das Zimmer. \enquote{Sofort!}

\enquote{Ich gehe und hole sie}, rief Harry. Er wusste, dass er sie schneller als alle anderen erreichen würde.

Er stürmte nach draußen und schloss die Tür hinter sich. Dann bog er zweimal nach rechts ab und drückte den richtigen Stein neben dem Torbogen. Die Wand ging wieder auf und Harry trat ein. Er suchte nach einem bestimmten Symbol. Schließlich fand er ein rotes Kreuz neben einem roten Halbmond auf weißem Untergrund. Sofort drückte Harry darauf und die Wand schloss sich. Wenige Augenblicke danach öffnete sich die Wand und er verließ den Raum.
% Das Rote Kreuz auf weißen Grund ist in der westlichen Welt das Zeichn für Erste Hilfe. In der muslimischen Welt ist dies der Rote Halbmond

Harry blickte sich kurz um und rannte nach rechts, auf die großen Flügeltüren des Krankensaales zu. Hastig öffnete er sie und lief zu Madame Pomfreys Büro. Er riss förmlich die Tür auf.

In ihrem Büro saß Madame Pomfrey und ihr gegenüber Professor Dumbledore.

\enquote{Potter, etwas mehr Disziplin bitte}, maßregelte ihn Madame Pomfrey.

\enquote{Keine Zeit, Madame Pomfrey}, entgegnete er. \enquote{Wir haben einen Notfall im Astronomieturm. Lavender Brown und Parvati Patil liegen mit epileptischen Zuckungen und Schaum vor dem Mund auf dem Boden. Eventuell eine Vergiftung. Es fing an, nachdem sie durch das Fernrohr gesehen hatten. Erst Parvati, danach Lavender, als sie durch Parvatis Fernrohr schaute.}

Madame Pomfrey schnappte sich ihre Notfall-Tasche und verließ, ohne sich von Dumbledore zu verabschieden, die Krankenstation.

Sie wollte schon den üblichen Weg einschlagen, als ihr Harry geschwind hinterherlief und sie am Handgelenk packte. Er zog sie Richtung Torbogen, von dem er gerade eben gekommen war.

\enquote{Was tun Sie da Mister Potter}, reagierte Madame Pomfrey genervt. \enquote{Zum Astronomieturm geht es da lang.}

Harry ignorierte sie und griff nur umso fester zu. Er drückte den richtigen Stein und wartete kurz, während er ihr sagte: \enquote{Abkürzung.} Die Wand öffnete sich und Harry zog Madame Pomfrey hinter sich her. Harry drückte den Knopf für den Astronomieturm und die Wand begann sich zu schließen. Nach kurzer Fahrt waren die beiden auch schon oben abgekommen. Harry hatte sie noch immer an ihrem Handgelenk gefasst und zog sie nun um die Ecke zum Klassenzimmer. Als er die Tür geöffnet hatte, ließ er sie los und ihr den Vortritt.

Nachdem Madame Pomfrey beiden einen Bezoar gegeben hatte, beruhigten sich die beiden. In der Zwischenzeit hatte Professor Sinistra das entsprechende Fernrohr durch einen Schildzauber abgeschirmt. Die Unterrichtsstunde war zu Ende und die Klasse wurde zu Bett geschickt.

Am nächsten Morgen saß Harry zwischen Lavender und Parvati auf der Krankenstation. In einer Hand hielt er Parvatis, in der anderen Lavenders Hand.

Lavender wachte als erste auf und drückte Harrys Hand, da er abwesend durch den Raum blickte. \enquote{Oh Lavender}, sagte Harry. \enquote{Alles in Ordnung?}

\enquote{Ja}, meinte sie. \enquote{Du warst wahnsinnig schnell, hat man mir erzählt, als ich kurz aufgewacht bin. Danke Harry.}

Harry lächelte sie nur an. Dann spürte er einen Druck an seiner anderen Hand und drehte sich zu Parvati hin.

\enquote{Hallo Harry}, sagte sie sanft. \enquote{Danke für die Rettung.} Sie blickte kurz zu Lavender und gab Harry durch Handzeichen zu verstehen, sie möge ihm doch Bitte näher kommen.

Harry folgte der Aufforderung. \gedanke{Sie will mir wohl etwas sagen, das Lavender nicht hören darf}, dachte er.

Doch Parvati zog ihn zu sich und gab ihm einen Kuss auf seine Wange nahe seinem Mundwinkel. Harry war erstaunt. \enquote{Ein kleines Dankeschön}, flötete sie und lächelte ihn an.

\enquote{Harry}, kam es ihm von hinten entgegen.

\gedanke{Lavender, natürlich. Das lässt sie nicht auf sich sitzen}, dachte Harry. Er drehte sich um und sah, wie ihm Lavender die Arme entgegenstreckte. \gedanke{Das könnte interessant werden}, dachte Harry. Er ging zur ihr hin. Als sie ihn in ihre Hände bekam, zog sie ihn zu sich und gab ihm einen langen und feuchten Kuss auf den Mund. Dann ließ sie von ihm ab und funkelte wild zu Parvati hinüber.

Harry blickte zwischen beiden hin und her. Sie tauschten böse Blicke. Dann hatte Harry einen Einfall. Er ging wieder einen Schritt auf Parvati zu und meinte dann: \enquote{Willst du nachziehen, Parvati?}

\enquote{Was?}, gab sie erstaunt zurück, wobei sie ihn anschaute.

\enquote{Na ja}, antwortete Harry, \enquote{du hast mich \accentuate{fast} geküsst, Lavender ging einen Schritt weiter und \accentuate{hat} mich geküsst. \gst Möchtest du nachziehen, oder einen Schritt weiter gehen?} Nun schauten ihn beide Mädchen an. \enquote{Ich meine, so wie ihr euch angeschaut habt, könnte man doch meinen, ihr habt eine Wette am Laufen. Wer traut sich bei Harry Potter am meisten.} Damit, so hoffte er, hatte er entweder eine Lawine losgetreten, was ihm gefallen würde; andererseits könnte es aber den gegenteiligen Effekt haben. Das wäre Harry dann auch recht. Auf jeden Fall hatte er seinen Spaß.

Parvati griff nach ihrem Kissen und warf es Harry entgegen. Er fing es auf und begann zu lachen. Dann mussten Parvati und Lavender ebenfalls loslachen.

Harry gab Parvati ihr Kissen zurück. Sie legte es hinter ihren Kopf und setzte sich dann auf. Sie drehte sich und ließ ihre Beine vom Bett herunterhängen. Harry setzte sich neben sie auf die Bettkante. Lavender verließ ihr Bett und setzte sich neben Harry.

Stumm saßen sie nun zu dritt nebeneinander. Harry hielt wieder ihre Hände. Nach einigen Minuten sagte Harry dann: \enquote{Ich gehe dann mal wieder. Ich habe noch Unterricht.} Er drehte sich zu Lavender und gab ihr einen Kuss auf ihre Wange. Sie wurde leicht rot. Er drehte sich zu Parvati und wollte sie ebenfalls auf die Wange küssen. Doch sie drehte sich kurz vorher zu ihm und so traf er ihren Mund. Ihre Lippen waren zarter, stellte Harry mit Genugtuung fest.

Dann lehnte sich Parvati nach vorne und meinte zu Lavender: \enquote{Jetzt sind wir quitt.}

Harry stand auf und meinte nur: \enquote{Ich gehe dann mal, bevor das hier noch ausartet.}

Jetzt mussten beide Mädchen wieder lachen. Madame Pomfrey kam gerade aus ihrem Büro, als Harry auf dem Weg zur Tür war. Dann verließ er die Krankenstation und machte sich auf den Weg zur nächsten Stunde. Auf dem Weg dorthin kam er an einer Tür vorbei, die leicht geöffnet war. Professor Elber saß hinter einem Tisch und hielt ein Amulett in der Hand. Er konnte seinen Professor von der Seite aus sehen. Er blieb interessiert stehen. Hermine trat von hinten an ihn heran, um zu sehen, was Harry so interessant fand.

Dann sagte Professor Elber: \enquote{Ich bereue zutiefst, gib dich frei.} Es kam eine kleine blaue Kugel heraus, die in seinen Professor eindrang und verschwand. Dann brach er bewusstlos auf dem Tisch zusammen. Harry und Hermine rannten zur Tür, doch sie verschloss sich vor ihnen.

\enquote{Hermine, du versuchst die Tür zu öffnen, ich hole Madame Pomfrey}, sagte er zu Hermine. Er rannte um die nächste Ecke zu einem Torbogen und drückte den Stein. Die Wand öffnete sich und Harry merkte sich das angezeigte Symbol. Nachdem er den Stein für den Krankenflügel gedrückt hatte, suchte er das passende Symbol, um zurückzukehren. Doch die Strecke war nicht sehr lang.

Mit Madame Pomfrey angekommen, führte er sie in das Zimmer. Hermine hatte die Tür in der Zwischenzeit geöffnet und Madame Pomfrey untersuchte ihn. \enquote{Brauchen Sie uns noch?}, fragte Hermine, \enquote{wir haben Unterricht.}

\enquote{Nein}, entgegnete ihnen Madame Pomfrey. \enquote{Sie können gehen.}

\trenn

Auf dem Weg zur Großen Halle kam von hinten Katharina Chapel aus Slytherin an ihn heran. \enquote{Harry? Darf ich dich was fragen?}

Harry war erstaunt, dass eine Slytherin normal mit ihm sprechen konnte. Sie war attraktiv, hatte mittelblonde schulterlange Haare und war nur ein paar Zentimeter kleiner als er. \enquote{Ja. \gst Katharina, richtig?}, fragte er.

\enquote{Genau. \gst Ich\abs äh\abs wollte\abs}.

Harry ging der Valentinstag durch den Kopf. Wollte sie ihn vielleicht fragen, ob er mit ihr\abs? Aber das erschien ihm doch etwas zu abwegig.

\enquote{Ich\abs} sie atmete noch einmal durch. \enquote{Ich wollte dich fragen, ob du mit mir am Valentinstag nach Hogsmeade gehen willst!}

\gedanke{Also doch. Aber wie sage ich es ihr.} \enquote{Hör mir bitte zu Katharina.}

\enquote{Du willst nicht?} Ihre Augen wurden leicht feucht.

Harry griff sofort ein Taschentuch und wischte ihr leicht über die Augen damit. \enquote{Das habe ich nicht gesagt. Ich möchte nur nicht, dass es zu einem Missverständnis kommt.} Jetzt sah sie ihn erstaunt an. \enquote{Ich gehe gerne mit dir dorthin. Aber mehr als diesen Ausflug kann ich dir nicht bieten. \gst} Harry sah sich kurz um. \enquote{Sag das bitte niemandem, aber Luna und ich haben uns wieder versöhnt. Nur wollen wir noch etwas Abstand haben und es langsamer angehen. Ich bin also offiziell nicht gebunden.} Dann lächelte er sie an und fuhr über ihre Wange. \enquote{Ehrlich gesagt freue ich mich darauf. Das wird eine Menge Aufsehen erregen.}

\enquote{Das reicht mir schon}, antwortete sie. Harry gab ihr einen Kuss auf die Wange und wollte gerade gehen, als sie ihn am Handgelenk festhielt. \enquote{Du weißt schon, dass wir unser \accentuate{Date} mit einem echten Kuss beenden müssen?}

Harry zog eine Augenbraue hoch, begann dann aber zu lächeln. \enquote{Das bekommen wir dann hin, wenn es so weit ist. Wir haben ja am vierzehnten genügend Zeit zur Planung}, grinste Harry leicht. Katharina lächelte zurück und ging dann fröhlich weiter. Harry war der Meinung, ihr Grinsen zu sehen. Und das, obwohl sie ihm den Rücken zeigte.

\stimme{Braver Junge}, hörte er in seinem Kopf.

\gedanke{Luna, ich\abs}

\stimme{Geh ruhig hin mit ihr. Ich mag das Valentinsgehabe eh nicht. Und wenn es dir Spaß macht\abs}

Mitte der nächsten Woche, kurz nach dem Essen, als Hermine mit Ron und Harry im Schlepptau aus der Großen Halle kam, schrie plötzlich eine Schülerin aus Slytherin: \enquote{Schnell, holt jemand sofort Madame Pomfrey, Professor Elber ist schwer verletzt.} Ein wildes und hektisches Durcheinander herrschte Sekunden später im Vorraum und der Eingangshalle. Hermine sprang sofort zu Professor Elber und ging vor ihm auf die Knie. Kaum hatte ihn Madame Pomfrey von seinem Ohnmachtsanfall kuriert, lag er schon wieder verletzt im Schloss.

\enquote{Hermine}, keuchte Professor Elber, \enquote{Bibliothek\abs zwischen den Regalen\abs N-P\abs Posessium\gst Avada Kedavra.} Dann brach er zusammen. Hermine stand auf und rannte sofort zur Bibliothek, wo sie Madame Pomfrey begegnete, die auf dem Weg in die Eingangshalle war. Harry beobachtete, wie Madame Pomfrey, nachdem sie angekommen war, ihren Zauberstab zog und ein paar Worte murmelte. Funken sprühten aus ihrem Zauberstab und fingen an, den reglosen Körper Professor Elbers zu bedecken. Sie färbten sich grünlich. Madame Pomfrey zuckte zusammen und fing an zu zittern.

\enquote{Bringen Sie ihn bitte in die Krankenstation.} Leicht zitternd ging sie voraus, während einige Schüler, darunter auch Harry, Professor Elber in den Krankenflügel schleppten. Oben angekommen, legten sie ihn auf ein Krankenbett und traten zurück. Langsam bewegte sich Professor Elber und brachte nur ein \enquote{Severus, Vertretung, mein Buch, Büro} heraus, um gleich danach wieder regungslos liegenzubleiben.

\enquote{Was ist mit ihm?}, fragte Harry.

\enquote{Wenn ich das wüsste}, antwortete Madame Pomfrey.

\enquote{Ich hab es, Professor}, schrie Hermine, als sie in die Krankenstation stürmte. Abrupt blieb sie stehen, als sie bemerkte, dass er bewusstlos da lag. \enquote{Was ist mit ihm?}, fragte Hermine.

\enquote{Ich weiß es nicht}, antwortete Madame Pomfrey abermals.

Hermine betrachtete das Buch und schlug es auf. \enquote{Er hatte irgendwas von \inner{Avada Kedavra} gesagt.} Sie schlug im Inhaltsverzeichnis nach und bleib bei einer Zeile mit dem Eintrag \enquote{Wenn man von Avada Kedavra gestreift wird und überlebt hat\abs} stehen. Hastig schlug sie die Seite auf und gab ein leises Quieken von sich.
Sie überflog die Seite und gab das Buch dann an Madame Pomfrey.

\enquote{Hier, Madame Pomfrey, ich habe noch nie so einen komplizierten Trank gesehen.}

Madame Pomfrey nahm das Buch an sich und las aufmerksam die ganze Seite durch. \enquote{Oh weh. Das schaffe ich nicht alleine. Potter. Würden Sie bitte Professor Snape holen, ich brauche seine Hilfe.}

Harry verließ die Krankenstation und machte sich auf den Weg zum Kerker.

\enquote{Professor Snape\abs}, sagte Harry, als er dort angekommen war.

\enquote{Jetzt nicht Potter, ich habe zu tun.}

\enquote{\aabs Madame Pomfrey braucht Ihre Hilfe. Es geht um Professor Elber.}

\enquote{Was hat er denn angestellt?}

\enquote{Er liegt bewusstlos auf der Krankenstation und Madame Pomfrey braucht dringend ihre Hilfe bei einem Heiltrank.}

Misstrauisch beäugte Professor Snape Harry. Schließlich stand er auf und ging um seinen Schreibtisch herum. \enquote{Sie gehen voraus, Potter}, sprach er. Er verschloss sorgsam sein Büro und folgte Harry in den Krankenflügel.

Oben angekommen standen schon Professor Dumbledore und Professor McGonagall am Krankenbett.

\enquote{Worum geht es, Poppy?}, fragte Professor Snape. \enquote{Wie kann ich Ihnen helfen?} Madame Pomfrey reichte ihm das Buch und Professor Snape las sich aufmerksam die Seite durch. \enquote{Das dürfte eine ganze Weile dauern. Ich schätze so etwa eine Woche, bis wir alle Zutaten beisammen haben. Dann nochmal etwa eineinhalb Wochen bis der Trank fertig ist. Und was steht hier? Drei Wochen Genesungszeit mit 2/3 der Zeit totale Amnesie?}

\enquote{Kriegen Sie das hin?}, fragte Professor Dumbledore.

\enquote{Ich denke gemeinsam werden wir das schon schaffen}, sprach Professor Snape und dreht sich zu Madame Pomfrey. \enquote{Wo haben sie eigentlich das Buch her?}

\enquote{Das hat mir Miss Granger gegeben.}

Professor Snape drehte sich zu Hermine und schaute sie mit hochgezogener Augenbraue an.

\enquote{Bevor er zusammengebrochen ist, hat er mir gesagt, was ich aus der Bibliothek holen sollte.}

Und Harry fügte hinzu: \enquote{Er hat danach nur noch ihren Namen, Vertretung und etwas von einem Buch in seinem Büro gesagt.}

Snapes Gesicht war wie immer unergründlich. Genauso gut hätte Harry gegen eine Wand sprechen können.

Plötzlich fiel Harry der Samstagskurs ein. \enquote{Was ist mit unserem Samstagskurs?} Harry biss sich auf die Zunge, denn das wollte er eigentlich nicht sagen, denn seit einigen Wochen übten sie jeden Samstag in der Kammer ihre Zaubersprüche.

\enquote{Samstagskurse?}, fragte Snape mit einer Gleichgültigkeit, die nur von ihm kommen konnte.

Auch Professor Dumbledore und Professor McGonagall drehten sich um.

\enquote{Welche Samstagskurse?}, fragte Professor Dumbledore nach.

\enquote{Ich dachte, das hätte Professor Elber mit Ihnen ausgemacht?}, sagte Hermine. \enquote{Wir üben seit einiger Zeit Verteidigung gegen dunkle Künste in der Kammer. Während der normalen Unterrichtsstunden holen wir den restlichen Stoff nach, der uns seit einigen Jahren fehlt. Wir sind fast ganz durch mit aufholen, meinte Professor Elber.}

\enquote{In welcher Kammer?}, fragte Professor Dumbledore nach. Doch dann kam ihm die Erleuchtung. \enquote{Ihr meint doch nicht etwa die Kammer des Schreckens?}

\enquote{Doch}, antwortete Harry. \enquote{Wir haben sogar zusammen die ganzen Flüche und Schutzzauber entfernt und berichtigt.}

\enquote{Ich wusste schon, warum ich ihn als Lehrer für dieses Jahr wollte. Er beherrscht sein Handwerk}, meinte Professor Dumbledore.

\enquote{Severus, übernehmen Sie seine Stunden?}, fragte Professor Dumbledore.

\enquote{Kann ich machen, Schulleiter, aber was ist mit den Samstagsstunden?}

\enquote{Die werde dann ich übernehmen, falls Sie keine Zeit haben}, antwortete Dumbledore.

\enquote{Und, wie war's?}, fragte Ron, als er Harry auf einem der Gänge traf.

\enquote{Snape vertritt Elber in VgddK}, antwortete Harry. \enquote{Und bei dir?}

\enquote{Ich habe mein Valentinsdate klargemacht.}

\enquote{Hermine?}

\enquote{Ja. \gst Und du? Wen hast du gefragt?}

\enquote{Niemand.}

\enquote{Also gehst du nicht nach Hogsmeade?}

\enquote{Doch}, antwortete Harry und grinste Ron an.

\enquote{Hä. Aber wenn du niemanden gefragt hast, wie\abs? Oh.} Harrys grinsen wurde breiter. \enquote{\accentuate{Dich} hat jemand gefragt?}, fragte er dann. \enquote{Wer? Luna?}

Harrys Grinsen wurde nur noch breiter. \enquote{Das siehst du dann, wenn es so weit ist.}

Ron bettelte noch eine Weile, doch Harry ließ sich nicht erweichen. Natürlich würde es die Gerüchteküche anheizen, wenn Harry nicht mit seiner Freundin nach Hogsmeade ginge, sondern mit einer Slytherin. Doch ihm war es egal. Er erregte sowieso aufsehen. Egal, was er tat.

\trenn

Die Doppelstunde bei Professor McGonagall verlief angenehm entspannt, da die Tipps von Professor Elber, welche er vor seinem komatösen Zustand verbreitete, es allen leichter machten. Professor McGonagall war erstaunt, dass die Fortschritte der Klasse nun größer waren als sonst. Dieses Mal sollten sie Katzen verschwinden und an anderer Stelle wieder auftauchen lassen. Bereits nach der Hälfte der Stunde waren drei Viertel so weit und nach eineinhalb Stunden konnten alle ihre Katzen verschwinden lassen. Professor McGonagall war beeindruckt. Alle übten noch eine Weile weiter, bis die Glocke das Ende der Stunde einläutete. Harry packte, wie alle, seine Sachen zusammen und machte sich mit Ron und Hermine auf zur nächsten Stunde \fach{Verteidigung gegen die dunklen Künste}. Was würde sie alle die nächsten Wochen bei Snape wohl erwarten? Mit leicht flauem Magen ging er in das Klassenzimmer und wartete auf Professor Snape. Mit dem Läuten der Glocke kam auch schon Professor Snape herein und schritt mit schnellen Schritten durch die Reihen.

\enquote{Es mag sich vielleicht noch nicht bei allen herumgesprochen haben, aber Professor Elber ist für die nächsten Wochen unpässlich und er hat mich gebeten, seine Stunden zu übernehmen. Machen Sie weiter mit dem, was sie letztes Mal gemacht haben, ich werde in Kürze bei Ihnen sein.}

Er lief Richtung Büro und öffnete die Tür, nur um darin zu verschwinden und die Tür hinter sich zu schließen. Die Klasse übte währenddessen ihren Zauberstab über ihrer Hand schweben zu lassen, während er leicht rotierte.

Nach einiger Zeit kam Professor Snape wieder in die Klasse zurück und baute sich vorne auf.

\enquote{Die nächsten acht Wochen wird Ihr Unterrichtsstoff etwas anders sein. Ich sehe Sie haben bereits alle relevanten Dinge durchgenommen. Ich habe Professor Elbers Absichten für die nächste Zeit studiert und werde sie fortführen. Die samstäglichen Übungskurse wird Professor Dumbledore leiten.}

Ein Murmeln und flüstern ging durch die Reihen. Vereinzelt fielen Zauberstäbe herunter und prallten auf die Tische und den Boden.

\enquote{Würden Sie Ihre Zauberstäbe wieder aufheben? Wir brauchen sie für Ihre nächste Übung}, fuhr Professor Snape sie an.

Es herrschte ein Stimmen Wirrwarr im Raum und viele \accentuate{Auf}'s klangen durch das Zimmer. Nachdem alle wieder ihre Zauberstäbe in ihren Händen hielten (keiner von ihnen hatte sich gebückt), fuhr Professor Snape fort.

\enquote{Bevor ich es vergesse. Morgen wird Ihre Unterrichtsstunde auf dem Schlossvorplatz stattfinden.} Abermals ging ein Murmeln und Staunen durch die Klasse, doch Professor Snape ließ sich dadurch nicht beeindrucken und aus der Ruhe bringen. \enquote{Halten Sie Ihre Zauberstäbe bereit und machen Sie mir die folgende Bewegung nach.} Professor Snape vollführte eine knappe und kurze Handbewegung.

\enquote{Wichtig ist, dass Sie dabei an das zu schützende Objekt denken müssen. Also meistens sich selbst. Wir nehmen heute einen Illusionierungszauber durch}, sprach Professor Snape, schwang seinen Zauberstab und war kaum noch zu sehen. Er lief durch den Raum und man konnte schemenhafte Umrisse erkennen. \enquote{Dieser Zauber verbirgt nicht bewegliche Teile komplett. Deshalb müssen Sie ruhig stehen, wenn Sie sich mit diesem Zauber tarnen wollen. Üben Sie alle mal. Tun Sie sich in Gruppen von zwei Personen zusammen und prüfen dann das Ergebnis Ihres Gegenübers. Der Spruch lautet: \spruch{Desuillusio.}}

Die Schüler drehten sich so, dass sie ihren Partner ansehen konnten, und sprachen den Zauber auf sich selbst. Harry war fast sofort verschwunden. Es dauert nur etwas länger, bis er durchsichtig war, aber sein Schutz war dem Professor Snapes ebenbürtig. Es machte eine Menge Spaß, sich so zu tarnen. Leider durfte man sich nicht bewegen, sonst wurde man entdeckt. Man musste statuenhaft dasitzen. Harry kam eine Idee und er belegte sich mit einem Zauber, den er gedanklich sprach und der seinen Körper vor versehentlichen Bewegungen schützte. Seine Kleidung wurde über seinem Bauch fester, sodass er sich nicht durch Atembewegungen verraten konnte. Das einzige, was man noch sehen konnte, waren zwei kleine Verzerrungen, in Augenhöhe, wenn er blinzelte. Ron teilte ihm das mit, worauf hin Harry daran dachte, seine Augendeckel transparent zu machen. So konnte er weiterhin blinzeln, aber nicht mehr gesehen werden. Er war nun komplett unsichtbar. Ron stach ihm in die Backe, als er ihn suchte.

\enquote{Au}, kam aus dem Nichts und Harry wurde wieder sichtbar. \enquote{Nicht so grob, Ron}, meckerte er.

Kurz vor dem Ende der Stunde behielt Professor Snape, Harry und Ron bei sich. \enquote{Sie kommen morgen früh zu mir ins Büro. Sie werden mir helfen, die benötigten Gegenstände auf den Vorplatz zu schaffen.} Harry und Ron nickten nur.

Abends im Gemeinschaftsraum fragte Hermine Harry: \enquote{Meinst du, er ändert den Lehrplan? Oder hält er sich an den Professor Elbers?}

\enquote{Zumindest hat er das mal gesagt}, meinte Ron, der gerade von oben herunterkam.

Harry stimmte ihm zu. \enquote{Ich meine auch, dass er ihn in großen Teilen übernimmt. Und scheinbar verläuft der Unterricht bei ihm besser, als Zaubertränke.} Hermine und Ron nickten nur. \enquote{Aber was mich am meisten interessieren wird, sind die Samstagskurse bei Professor Dumbledore.}

Hermine nickte nur wieder.

\enquote{Meinst du, er weiß, wo der Eingang ist, oder sollen wir ihn hinführen?}, fragte Ron.

\enquote{Fragen wir ihn doch Morgen beim Frühstück}, meinte Hermine.

\trenn

Als Hermine am nächsten Tag mit dem Frühstück fertig war, stand sie auf und ging vor zum Lehrertisch. Harry und Ron folgten ihr. \enquote{Professor Dumbledore?}

\enquote{Ja, Hermine.}

\enquote{Sollen wir Sie am Samstag nach dem Frühstück abholen und zur Kammer führen?}

Einige um sitzende Lehrer bekamen große Augen und konnten nicht mehr schlucken.

\enquote{Gerne, ich habe nämlich keine Ahnung, wie ich da hinkomme.}

\enquote{Dann bis Samstag}, sagte Harry und verabschiedete sich. Er schnappte nur noch ein: \enquote{Welche Kammer?}, von Professor Sprout auf, bevor er außer Hörweite kam.

Harry, Ron und Hermine grinsten.

Am nächsten Tag, in Professor Snapes Büro angekommen, stand dieser auch schon auf und gab Ron und Harry zu verstehen, ihm zu folgen. Sie gingen weiter hinunter als sonst und schienen einer fast endlos langen Treppe zu folgen. Unten angekommen führte ein Gang einige Meter geradeaus. Doch plötzlich war er zu Ende.

\gedanke{Sackgasse, na toll. Snape hat sich verirrt}, dachte Harry.

Professor Snape griff in seine Tasche und zog einen Zettel heraus. Er murmelte einige Worte und zeigte mit seinem Zauberstab auf die Wand vor ihm. Dann fuhr er wie auf einem Muster die Ziegelsteinfugen ab. Die Wand teilte sich und gab eine Tür frei. Professor Snape öffnete sie und trat ein. Harry und Ron folgten ihm. Er drehte sich um und meinte dann: \enquote{Sie beide sind mir für das Arbeitsgerät verantwortlich. Sie bringen es, falls wir es brauchen, zum Unterricht und räumen es hinterher wieder auf.} Er drehte sich wieder um und suchte eine Kiste. Als er sie fand, zeigte er darauf und deutete den beiden an, sie mitzunehmen. Er verschwand und ließ Harry und Ron alleine.

Harry näherte sich und las die Aufschrift: \accentuate{Konzentrations- und Vertrauensscheiben.}

\enquote{Was das wohl heißen mag?}, fragte Ron.

\enquote{Keine Ahnung}, entgegnete Harry. Er zog seinen Zauberstab und rief: \enquote{Locomotor Kiste}. Sie fing an, leicht zu schweben. Harry griff sich eine Seite der Kiste und Ron die andere. Ohne Mühe trugen sie die Kiste hinaus und schlossen die Tür.

Als sie oben ankamen, warteten schon alle Schüler und Professor Snape fing mit dem Unterricht an.

\enquote{Heute werden Sie sich mit diesen Teilen auseinandersetzen.} Er zeigte auf die Kiste und Ron öffnete sie.

\enquote{Sie stellen sich nebeneinander auf und jeder nimmt sich eine Scheibe aus der Kiste. Es sind gleich viele schwarze wie weiße.}

Jeder der Schüler trat vor die Kiste und nahm sich eine der Scheiben heraus. Professor Snape machte weiter.

\enquote{Wenn Sie sich nachher auf Ihre Scheiben stellen, werden diese etwas vom Boden abheben und zu schweben beginnen. Sie werden automatisch Ihren Partner finden. Er wird also keine bewusste Auswahl geben. Treten Sie nun alle auf Ihre Vertrauensscheiben.}

Alle Schüler traten auf ihre Übungsscheiben, worauf diese zu schweben begannen. Nacheinander machten sich die Scheiben auf und suchten ihren Partner unter der Menge. Auch Harrys Scheibe fing an sich zu bewegen und machte vor Dracos Scheibe halt. Beide bekamen nur große Augen und machten einen ziemlich unglücklichen Gesichtsausdruck.

\enquote{Wenn Sie inzwischen alle Ihre Partner gefunden haben, dann fangen wir jetzt an.} Er zog wieder seinen Zettel aus der Tasche, nahm seinen Zauberstab und sagte: \enquote{Macrone. Astate. Lewioss.} Die Scheiben färbten sich grün und begannen etwas höher zu schweben. \enquote{Wenn ich den Startschuss gebe, fangen Sie an sich zu konzentrieren, um die Scheibe ihres Partners auf etwa gleicher Höhe wie Ihre eigene zu halten. Sie dürfen maximal um einen dreiviertel Meter abweichen, sonst verlieren beide ihren Kontakt und stürzen zu Boden. Halten Sie sich bereit. \gst Amigosa!}

Einige Scheiben zuckten gefährlich und es schien so, als ob einige die gefährliche Grenze des Abstandes erreichten. Draco fing an zu Grinsen, aber Harry kümmerte das wenig.

\enquote{Du stürzt genauso ab wie ich. Falls du also etwas vorhast, lass es lieber bleiben.}

Dracos Augen verengten sich, aber er verstand. Notgedrungen mussten beide diese Stunde gemeinsam überstehen und konzentrierten sich auf die Scheibe des anderen, um den Abstand nicht unnötig groß zu halten. Von Zeit zu Zeit ging ein Impuls durch die Scheiben und sie bewegten sich auseinander. Mal in der Ebene voneinander weg; Mal in der Höhe.

\enquote{Passen Sie auf, wenn sich die Farbe Ihrer Scheibe verändert heißt das, dass Sie den Abstand immer kleiner halten müssen. Es gibt noch die Farben gelb (50 cm), blau (10 cm) und rot (2 cm). Konzentrieren Sie sich und sprechen Sie sich ab. Hier ist Teamfähigkeit gefordert. Wir werden das die nächsten paar Male durchführen. Sie werden dann zusätzlich ihre Augen verbunden bekommen. Ach und noch eines. Ihre Partner werden wohl die gleichen bleiben.} Professor Snape schaute noch einmal auf seinen Zettel.

Harry war gar nicht wohl. Am liebsten hätte er sich jetzt in der Krankenstation gemeldet. Aber da musste er wohl durch. Noch ein paar Stunden mit Malfoy.

Der Rest der Stunde verlief nicht gerade angenehm, da Malfoy ihn immer wieder versuchte aus seiner Konzentration zu werfen, doch als die Farbe seiner Scheibe sich plötzlich ins Gelbe zu verändern anfing, riss er sich zusammen und verringerte schnell den Abstand von Harrys Scheibe zu seiner.

Diese Scheiben hatten etwa einen Meter im Durchmesser und eine Dicke von ca. drei Zentimetern. Es schien so, als wären sie aus einer Art flüssigem Metall gemacht, denn sie schimmerten und gaben leicht nach, wenn man sie bestieg. Es war so, als würde man auf Wasser laufen. Jeder Tritt gab einen kleinen Ring von sich, der mit dem des anderen Fußes Interferenzen bildete.

Die Glocke läutete und alle machten sich auf zur nächsten Stunde, nachdem sie ihre Scheiben auf den Boden schweben gelassen hatten. Harry und Ron verstauten noch den Koffer mit den zurückgelegten Scheiben und trafen zur nächsten Stunde 5 Minuten verspätet ein.

Die Kräuterkunde-Stunde bei Professor Sprout verlief dieses Mal nicht ganz so gut, denn Harry konnte sich irgendwie nicht mehr Konzentrieren. Es schien so, als ob er seine ganze Konzentrationskraft während der Stunde bei Snape aufgebraucht hatte. Dauernd schnitt er bei seinem Setzling daneben, worauf dieser sich lautstark beschwerte. \enquote{Tut mir leid}, beschwichtigte Harry ihn immer wieder, doch das half auch nicht weiter. Am Ende der Stunde war sein Setzling total verhunzt und Madame Sprout musste ihn bis nächstes Mal wieder aufpäppeln. Harry hatte erst mal Hunger und setzt sich nach dem Essen auf seinen Platz bei Professor Binn. Er wartete, bis er durch die Tafel schwebte und nahm eine Schlafposition ein. Er fühlte sich so, als könnte er neue Kraft schöpfen. Doch die Stunde war schnell wieder vorbei. Ron weckte ihn und so ging er hinter ihm und Hermine her, um zu Hagrid zu gehen. Heute hatte er wenig zu tun, da sich sein Pflegetier bester Gesundheit freute.
Ron unterhielt sich etwas mit Hagrid und Hermine wohnte dem Gespräch bei.

\enquote{Rosalie. Nein}, rief Harry, als sich das Tier Malfoy näherte. Er hatte das Gefühl, sie würde ihn schwer verletzen.

Draco Malfoy drehte sich um und erschrak, als er eine warme feuchte Zunge auf seiner Backe spürte. Draco fasste sich schnell wieder und sagte dann: \enquote{Rosalie, magst du mich?} Zuerst nahm er sie nur leicht schemenhaft wahr, doch schließlich wurde ihre Gestalt immer deutlicher. \enquote{Ein nettes Haustier hast du da. Nimmst du es mit nach Hause, damit es deiner Familie das letzte Hemd wegfrisst?} Scheinbar verstand Rosalie, dass es eine Beleidigung für Harry war, denn sie biss ihn leicht ins Ohr, welches anfing zu bluten. \enquote{Au}, rief Malfoy. Hagrid kam mit wenigen Schritten herüber, sah sich das Ohr kurz an und machte einen Verband daran. Draco konnte gar nicht so schnell reagieren.

\trenn

Nervös ging Harry am Freitagabend zu Bett. Morgen war es so weit. Gleich nach dem Frühstück würden sie wieder in der Kammer des Schreckens ihre VgddK-Übungen haben. Und dieses Mal würde Professor Dumbledore sie führen und anweisen.

Die Meisten der anderen waren mit dem Frühstück schon fertig, als Hermine zu Harry sagte: \enquote{Iss auf, dann nehmen wir Professor Dumbledore mit.}

\enquote{Ja, ist gut}, antwortete Harry und schluckte seinen letzten Bissen Toast herunter. Er trank seinen Becher aus und zu zweit gingen sie vor zum Lehrertisch. Professor Dumbledore bemerkte sie und stand ebenfalls auf. Ron war noch kurz zur Bibliothek gelaufen, um ein Buch für seine Hermine zu holen, da sie es noch unbedingt haben wollte, und war vorausgelaufen. Dumbledore umrundete den Tisch und kam ihnen entgegen.

\enquote{Bereit?}, fragte Hermine.

\enquote{Um ehrlich zu sein, bin ich etwas nervös}, antwortete Dumbledore. \enquote{Aber sagt es nicht den anderen.}

Sie führten Professor Dumbledore in den dritten Stock wo sich das Mädchenklo mit der maulenden Myrte befindet. Hermine öffnete die Tür und ging voran, dicht gefolgt von Professor Dumbledore.

\enquote{Uhh, Harry}, kam es ihm wieder entgegen. \enquote{Heute werde ich dir so richtig einheizen}, meinte Myrte. \enquote{Frederick hat mir neulich ein paar tolle Tricks und Kniffe gezeigt. Ich werde euch ALLE dran kriegen.} Plötzlich fiel ihr Blick auf Professor Dumbledore. \enquote{Ach, was wollen Sie denn hier?}

\enquote{Er wird uns heute begleiten, Myrte. Professor Elber ist auf der Krankenstation. Er kann die nächsten Male nicht.}

\enquote{Uaaah}, entfuhr es aus Myrte. \enquote{Das kann er mir doch nicht antun. Ich sollte heute doch meinen großen Auftritt haben.}

\enquote{Ich denke, es läuft alles so, wie es sich Professor Elber gedacht hat}, meinte Professor Dumbledore. \enquote{Und wenn das beinhaltet, dass Sie sich heute mit den Schülern beschäftigen, dann wird das wohl so sein. Außerdem habe ich keinerlei Informationen, was ihr hier bisher gemacht habt. Ich kann euch momentan lediglich zusehen und etwas anleiten, beziehungsweise euch helfen}, fügte er hinzu.

\enquote{Dann kommen Sie mal, Professor}, sagte Hermine, als sie vor dem Zugang stand, sich umdrehte und verschwand.

Ron ging ebenfalls in Richtung Eingang und folgte Hermine, da er gerade zur Tür hereinkam.

\enquote{Ich bleibe dicht hinter dir, Harry}, meinte Myrte. Harry schmunzelte etwas verkrampft und schwang sich in die Rohre. Unten angekommen, kam lange Zeit nichts. Als sich die vier schon zu wundern begannen, hörten sie plötzlich etwas.

\enquote{Huuiiii. Klasse.} Unten angekommen meinte Professor Dumbledore nur: \enquote{Hat irrsinnig Spaß gemacht. Hier sollten wir öfter mal herkommen.}

\enquote{Wir sind hier jede Woche}, meinte Ron.

Dumbledore folgte Harry und Ron durch die Gänge in die Kammer. Hermine war bereits vorausgelaufen, um die Tür zu öffnen. Sie musste die entsprechenden Parselworte lernen, um dies zu tun.

Drinnen angekommen meinte Professor Dumbledore nur: \enquote{Beeindruckend.} Er folgte Harry und Ron an den Kopf der Kammer, sah sich weiter um und sagte dann. \enquote{Tja, dann machen Sie mal. Ich schaue Ihnen zu.}

Er zauberte sich einen bequemen Sessel her und setzt sich darauf, seinen Zauberstab in einer Hand festhaltend. Die Schüler schauten sich nur verwundert an, als Hermine die Initiative ergriff und zu Harry meinte: \enquote{Was machen wir jetzt?}

\enquote{Was schaust du mich so an?}, fragte Harry.

\enquote{Du hast uns doch letztes Jahr unterrichtet, also mach was draus. Du hast von uns allen am meisten Erfahrung von uns.}

\enquote{Äh ja}, sagte Harry.

Myrte schwebte von hinten an Harry heran, legte ihren Kopf auf seiner Schulter ab und machte ein glückliches Gesicht. Die anderen Schüler schmunzelten.

\enquote{Ihr wisst doch sicher alle, was Professor Elber das letzte Mal gesagt hatte. Wir sollen heute den Stoff vom letzten mal wiederholen. Nächstes Mal fangen wir dann mit den Sachen an, die wir am Mittwoch im neuen Kapitel lesen. Stellt euch alle zu kleinen Gruppen zusammen.}

Alle Schüler stellten sich auf und machten sich bereit.

Harry zog seinen Zauberstab und fing an bunte wunderbar leuchtende Blitze auf die Gruppen zu schleudern. Die kleinen Gruppen, deren Mitglieder mit den Rücken zueinander gewandt waren, bildeten einen Schild um sie herum und lenkten so die Blitze ab. Als die Blitze die ersten Gruppen trafen und von den Schilden abprallten, hörte Harry auf und gesellte sich zu seiner Gruppe. Dieses Mal hatte er Neville und Dean Thomas in seiner Gruppe.

Die Blitze prallten von den Schilden der Gruppen und von den Wänden der Kammer ab. Sie sausten kreuz und quer durch die Kammer und kamen praktisch von allen Seiten. Ab und zu beobachtete Harry Professor Dumbledore, der begeistert in seinem Sessel saß und immer mal wieder einen bunten, schwächer werdenden Blitz zurückschleuderte, der auf ihn zukam. Langsam klangen die Blitze ab und verschwanden mit der Zeit. Die Gruppen lösten sich wieder auf und alle bildeten automatisch einen Kreis.

\enquote{Malfoy?}, fragte Harry. \enquote{Fängst du an?}

Malfoy antwortete nicht, sondern fing nur an einen weißen, hellen Blitz auf Harry abzufeuern. Dieser nahm ihn mit seinem Zauberstab auf, lenkte ihn um und schickte ihn weiter an Millicent Bulstrode. Diese wiederum schickte ihn weiter an Parvati Patil, welche ihn auf Neville losließ. Als der Blitz anfing schwächer zu werden, nahm ihn der nächste an und lenkte ihn in ein kleines Loch in der Decke. Kurz darauf erzeugte er einen neuen und das Spiel ging von vorne los. Nach einigen Runden fragte Harry Professor Dumbledore: \enquote{Wollen Sie nicht doch auch mal mitmachen?}

\enquote{Ach \gst warum nicht}, antwortete Dumbledore, stand auf und meinte: \enquote{Ich habe noch einen kleinen Tipp für euch.} Er trat auf die Gruppe zu und sagte zu Harry: \enquote{Schleudere nochmal so einen Blitz auf mich zu und ihr anderen, passt auf und haltet eure Zauberstäbe bereit.}

Harry begab sich in Position und schleuderte einen Blitz auf Professor Dumbledore. Dieser wollte ihn aufnehmen, doch es schob ihn zurück und schmiss ihn fast um.

Alle lachten, besonders Malfoy.

\enquote{Upps, tut mir leid Professor}, meinte Harry.

\enquote{Schon in Ordnung}, entgegnete ihm Dumbledore, \enquote{ich war nur nicht darauf vorbereitet. Was war das denn für ein Blitz?}, fragte Professor Dumbledore.

\enquote{Ein Pulsierender}, meinte Harry lapidar. \enquote{Hat uns Professor Elber beigebracht. Er meinte, wenn wir den abwehren können, sind wir gut bedient. Damit schaffen wir 90 Prozent aller unserer Gegner.}

\enquote{Da hat er wohl leicht untertrieben. Das ist hohe Magie. Also nochmal.}

\enquote{Wie meinen Sie das}, fragte Draco Malfoy nach.

\enquote{Nun ja, diese Art von Blitze können nur von sehr mächtigen Magiern erzeugt werden. Anscheinend ist Harry\abs}, doch plötzlich hörte er auf und begann seine Stirn zu runzeln. \enquote{Habt ihr diese Blitze vorher auch schon gehabt, als ihr im Kreis gestanden seid?}

\enquote{Ja}, antwortete Hermine. \enquote{Jeder von uns musste schon einmal einen erzeugen. Zuerst fiel es uns schwer, aber mittlerweile hat es jeder von uns drauf.}

Professor Dumbledore sagte nichts. \enquote{Äh ja. Ich war vorhin bei einer kleinen Demonstration. Also Harry, noch einmal bitte.}

Harry begab sich in Position und schleuderte wieder einen Blitz auf Professor Dumbledore. Der nahm ihn auf, teilte ihn und schleuderte ihn auf zwei verschiedene Schüler, die ihn sofort neutralisierten.

\enquote{Ich dachte mir, dass ich euch das heute noch zeige und nächsten Samstag könnt ihr das dann wiederholen und euren neuen Stoff aus dem Buch dazu machen. Kann ich mir von jemand das mal ausleihen?}, fragte Professor Dumbledore.

\enquote{Ja, von mir}, sagte Parvati. \enquote{Ich lerne sowieso immer mit Lavender zusammen. Da können wir eines eine Weile entbehren.}

\enquote{Und wann komm' ich dran?}, fragte Myrte.

\enquote{Jetzt!}, antwortete Harry.

Darauf schien Myrte gewartet zu haben. Sie stieg hoch und wollte bereits anfangen, als sie mitten in ihrer Bewegung stehen blieb und aus ihrer Hand kleine Kügelchen auf die Schüler losließ. Plötzlich stürzte sie herab und fing an durch die sterblichen hindurchzufliegen. Das war alles andere als angenehm. Es war schon so nicht angenehm, wenn einen ein Geist durchquerte, aber zu erleben, wie eine ekstatische Myrte durch einen glitt, toppte alles. Es war so, als ob man fror und gleichzeitig einer großen Hitze ausgesetzt war. Nach einer viel zu anstrengenden Stunde war Harry zu müde zum Abendessen.

Bevor er ins Bett ging, bemerkte Harry eine Eule an einem der Fenster. Ron war schneller und öffnete es. Die Eule flog zielstrebig zu Harry und ließ den Brief vor Harry fallen. Danach drehte sie gleich wieder ab und flog durch das Fenster hinaus in den dunklen Nachthimmel. Ron schloss das Fenster und ging wortlos ins Bett. Harry öffnete den Brief, nachdem er sichergestellt hatte, dass ihm keiner zusehen konnte.

\begin{brief}
Lieber Harry,

ich hoffe, du bist mir nicht böse, aber ich habe vor wenigen Minuten mit jemand geredet, der die Menge, die Morgen sicher am Eingang wartet und sehen will, mit wem du dich triffst, schocken wird. Sie ist eine gute Freundin und wird auf jeden Fall dicht halten. Warte bitte pünktlich zum vereinbarten Zeitpunkt am Haupteingang zum Schloss. Pansy wird sich kurz neben dich stellen, bevor sie dann zu ihrem Date geht und ich dann komme.
\signumspace
Liebe Grüße und Bussi

Katharina
\end{brief}


Harry ließ beinahe den Brief fallen. \gedanke{Pansy? Sie hat es Pansy erzählt?} Harry ließ sich auf sein Bett sinken. \enquote{Ausgerechnet sie.} Er schloss den Brief in seinen Koffer ein und legte sich unruhig ins Bett.

\begin{traum}
Harry lief gedankenverloren durch das leere Schloss. Es störte ihn nicht, dass er alleine war. Er setzte sich in die Große Halle und begann zu Essen, als sich ihm gegenüber Salazar Slytherin setzte und mitaß. Plötzlich war die Große Halle voll von frühstückenden Schülern und selbst der Lehrertisch war wieder besetzt. Salazar sah ihn lange an und meinte dann zu Pansy Parkinson, welche neben ihm saß: \enquote{Und, was meinst du? Hat er Angst?}

\enquote{Ja, ganz eindeutig. Er hat Angst.}

\enquote{Wovon redet ihr überhaupt}, wollte Harry wissen.

\enquote{Von dir, Harry.}

\enquote{Lass mich in Ruhe, Parkinson.}

\enquote{Na na na, wie redest du denn mit deiner weitschichtigen Verwandtschaft.}

Harry verschluckte sich an seinem Bissen und Pansy musste ihm auf den Rücken klopfen, damit der Speisebrocken hinunter in seinen Magen rutschte.

\enquote{Danke \gst Pansy.}

\enquote{Keine Ursache, Harry. Für dich mache ich das doch gerne.}

Jetzt erst sah er zu Pansy hinüber und war nicht im Entferntesten darüber erstaunt, dass sie eine Gryffindor-Uniform trug. Da in der Halle niemand außer ihnen dreien war, besah er sie eine Weile. \gedanke{Ich mag sie}, dachte sich Harry. \enquote{Was steht heute auf dem Stundenplan?}

\enquote{Ihre beide habt jetzt Tränke, Wahrsagen, Verwandlung, Zauberkunst und heute Abend Astronomie.}

Harry nickte und frühstückte vorsichtig weiter.
\end{traum}

Danach schlief Harry entspannter weiter und wurde durch das Zurückziehen seiner Vorhänge durch Kreacher geweckt.

\enquote{Guten Morgen, Sir Harry}, kam es vom alten Elf.

\enquote{Guten Morgen, Kreacher}, antwortete Harry. \enquote{Was ist los?}

\enquote{Sie haben heute einen wichtigen Termin, Sir.}

\enquote{Woher\abs?}

\enquote{Kreacher wäre ein schlechter Elf, wenn er die Termine seines Herrn nicht kennen würde. Kreacher hat sich erlaubt, Ihre Sachen selbst vorzubereiten, die Sie heute Nachmittag tragen werden. Kreacher hat außerdem noch ein spezielles Parfum hergerichtet.} Er zeigte auf Harrys Nachttisch. \enquote{Es wirkt nur bei Ihnen, Sir.} Kreacher verbeugte sich und sagte dann abschließend, bevor er verschwand: \enquote{Kreacher wird nun seinen Dienst in der Küche weiterführen.}

\trenn

Obgleich es draußen schneite, strahlte die Krankenstation in Hogwarts eine Wärme aus, die den Kranken dort helfen sollte, schneller zu genesen. Neben dem seit einigen Wochen in einem komatösen Zustand liegenden Professor Elber, lagen dort noch ein paar andere Schüler von Hogwarts. Ein paar Zauber gingen schief und so waren einigen Eselsohren, Hirschgeweihe oder Flügel gewachsen.

Madame Pomfrey hatte alle Hände voll zu tun. Die halbe Krankenstation war mal wieder voll von Schülern. Jedes Mal, wenn man ihr einen brachte, machte sie ein vorwurfsvolles Gesicht, stellte ein paar Fragen und begann dann mit der Arbeit. Sie wollte nicht wissen, wie das genau geschah, sondern nur, was das verursacht hatte. Sie stellte nie zu viele Fragen. Fragen die manchen Schüler in eine peinliche Lage gebracht hätte. In dieser Zeit geschah es auch, dass Professor Elber sich auf den Rücken drehte und plötzlich ein Keuchen von sich gab.

Viele Schüler und auch Madame Pomfrey schauten nun in seine Richtung. Er begann die Augen zu öffnen. Sie leuchteten grün. Es schien so, als ob er nicht bei Bewusstsein wäre. Sein Mund war ebenfalls leicht geöffnet und ein kleiner Lichtstrahl, der immer stärker wurde, verließ seinen Mund.

\enquote{Schnell}, rief Madame Pomfrey zu einem der Schüler, \enquote{holen Sie Dumbledore.}

Die angesprochene verließ die Krankenstation und kam schon wenige Sekunden danach zurück. Offenbar hatte sich Dumbledore entschlossen, die Krankenstation und ihre Insassen zu besuchen. Sein Blick fiel jetzt ebenfalls auf Professor Elber.

Der Lichtstrahl wurde jetzt nicht mehr stärker, dafür schlängelte sich ein zweiter um den ersten herum und verformte sich zu einer über Professor Elber frei schwebenden durchscheinenden Ebene. Langsam bildeten sich Personen und die sie umgebende Landschaft heraus. Es war Professor Elber, der Lord Voldemort gegenüber stand. Hinter Voldemort standen einige seiner Anhänger. Es gab einen heftigen Streit.

\enquote{Na gut}, meinte Voldemort, \enquote{wenn du nicht willst. Ich habe noch andere Mittel und Wege das zu bekommen, was ich will.} Er zückte seinen Zauberstab und sprach: \enquote{Imperius}.

Professor Elber schaffte es gerade noch, ebenfalls seinen Zauberstab zu ziehen und den Imperius-Fluch umzulenken. Er traf nun einen von Voldemorts Anhängern, der nun bewegungslos wie eine Marionette herumstand.

Voldemorts Augen begannen zu funkeln und sein Blick wurde zorniger und seine Stimme wurde ärgerlicher. \enquote{Avada Kedavra}, schrie Voldemort.

Die umstehenden Schüler erschraken und auch Professor Dumbledore zuckte ein wenig. Doch etwas Unerwartetes geschah. Der grüne Blitz aus Voldemorts Zauberstab wurde von dem Zauberstab Professor Elbers aufgefangen, umgeleitet und in Richtung Himmel geschickt.

Den umstehenden Personen fiel das Kinn herunter. Einzelne \accentuate{Boah!}-Rufe hallten durch die Krankenstation.

Etwas geschwächt durch den umgeleiteten Tötungsfluch sagte Professor Elber: \enquote{Der nächste kommt zu dir zurück.}

Das aber beeindruckte Voldemort sichtlich gar nicht. Er sprach abermals: \enquote{Avada Kedavra} und schleuderte einen weiteren Blitz auf Professor Elber. Dieser nahm auch diesen auf, schleuderte ihn aber Richtung Voldemort, welcher gerade noch ausweichen konnte und zur Seite hechtete. Am Boden liegend ließ Voldemort erneut einen Tötungsfluch auf Professor Elber los. Dieser streifte ihn leicht und lies Professor Elber zu Boden sinken. \enquote{Tja, wenn man sich mit mir anlegt, verliert man}, sprach Voldemort.

Professor Elber lag regungslos am Boden.

\enquote{Ist er tot?}, fragte ein Schüler.

Professor Dumbledore antwortete ihm. \enquote{Wenn er tot wäre, wäre er jetzt nicht hier, oder?}

\enquote{Stimmt}, gab er als Antwort zurück.

Voldemort lief auf Professor Elber zu und wollte noch einen Zauber nachsetzen, als Professor Elber begann, sich aufzulösen und zu verschwinden. Voldemort fluchte umher und war außer sich.

Die Szene löste sich langsam auf und begann ihre Form und Gestalt zu ändern. Man sah jetzt Hogwarts im Hintergrund. Professor Elber lag am Bahnsteig; immer noch regungslos. Langsam rührte er sich und stand auf. Mühsam schleppte er sich den Pfad hinauf nach Hogwarts, wo er schließlich gänzlich erschöpft auf den Stufen zum Eingangstor zusammenbrach. Die Projektion löste sich auf, Professor Elber schloss seine Augen und dreht sich abermals.

Stille herrschte im Raum und keiner wagte es etwas zu sagen, bis Madame Pomfrey plötzlich in die Hände klatschte und meinte: \enquote{So, Kinder, es wird Zeit, wieder in die Betten zu steigen, wir haben noch viel zu tun. Oder wollt ihr eure Hörner, Ohren oder Flügel etwa behalten?} Die Schüler begaben sich wieder in ihre Betten und Madame Pomfrey machte sich an ihre Heilung.

Professor Dumbledore murmelte nur \enquote{Ich hoffe, er wird wieder}, während die Schüler mit den Eselsohren nur laut zustimmten.

Währenddessen trieb es Harry aus einem inneren Antrieb heraus wieder in die Kammer. Er streifte umher und untersuchte die Gänge, um etwaige Geheimtüren zu erkennen. Hinter einer dieser als abgesetzte Steinbogen getarnten Tür fand er eine kleine Kammer mit einer Handvoll Büchern und einem kleinen Holzstuhl. Wahlweise nahm er eines der Bücher heraus und las. Es handelte von Basilisken. Keines der Bücher war sehr dick, sodass er alle auf einmal las. Sie handelten von Aufzucht, Pflege und Schutz vor Basilisken.

\fluestern{Interessant}, murmelte Harry. \fluestern{Man kann einen Basilisken auf einen Prägen, wenn man einen speziellen Trank braut und ihm eines der eigenen Haare zufügt. Dann muss man das bebrütete Ei hineinlegen. Fortan ist man vor den Blicken des Basilisken geschützt.}

Nachdenklich strich er über die letzte Seite, auf der der Trank abgebildet war. Er schloss das Buch und verließ die kleine Kammer, dessen Tür sich sofort schloss, nachdem er hinausgetreten war.

Bevor er die Kammer verließ, sah er noch einmal auf das Loch, aus dem der Basilisk damals auf ihn zukam. Ein innerer Drang überkam ihn und so kroch er durch die Röhre. In einer Kammer, die ein Nest aus Stroh enthielt, fand er ein Ei. Vorsichtig trat er mit gezücktem Zauberstab näher. Skeptisch betrachtete er das Ei.

\gedanke{Ob es wohl befruchtet ist?}, fragte sich Harry.

\stimme{Ja}, erklang es in seinen Gedanken. \stimme{Das ist ein Basilisken-Ei.}

Panisch trat Harry einige Schritte zurück.

\stimme{Keine Panik, Harry. Du hast einen Schutztrank gesehen. Dieser prägt die kleine Schlange auf dich. Du kannst ihr dann zwar noch nicht in die Augen sehen, aber wenn sie geschlüpft ist, kannst du ihr befehlen, die Augen zu schließen. Dann belegst du die Augen mit einem Zauber, damit der gefährliche Blick nicht mehr tötet. Es kribbelt bei dir nur noch und schmerzt in den Augen, sollte der Basilisk dich danach ansehen. Andere versteinert er noch. Erst wenn du ihm einen Trank gibst, verliert sich die Gefährlichkeit des tödlichen Blickes.}

Harry dachte nach. Er beherrschte mittlerweile einen Kopierzauber, der auf Entfernungen reagierte. So kopierte er den entsprechenden Trank, den er heute bei einer seiner Strafstunden brauen würde. Denn trotz seiner Okklumentikstunden, braute er ab und an einen Trank, damit sich seine Braukünste besserten. Er las das Rezept noch einmal durch und machte sich dann auf den Rückweg, um vor seinem Treffen mit Katharina noch ein paar Hausaufgaben zu erledigen. Er hatte noch zwei Stunden Zeit. Da ihm Kreacher nachher beim Anziehen und Zurechtmachen helfen würde, konnte er sich die Zeit nehmen.

\trenn

Als Harry gerade seine Hausaufgaben im Gemeinschaftsraum zusammen mit Ron machte, kam Lavender herein und setzte sich in die Nähe der beiden zu Parvati.

\enquote{Hast du schon gehört, was Professor Elber im Krankenflügel passiert ist?}

Harry drehte sich herum, um die Unterhaltung besser mit anhören zu können und auch Ron blickte plötzlich auf.

\enquote{Also}, fuhr sie fort, \enquote{ich lag gerade da um meine\abs nun ja ich war gerade in Behandlung}, fuhr sie fort, als sie Harry erblickte, \enquote{als Professor Elber seine Augen auf unnatürliche Art und Weise öffnete und ein grüner Lichtschein sich aus seinem Mund bahnte. Sie zeigte Professor Elber, der Du-weißt-schon-wem gegenüber stand.} Ein leichtes Zittern lag immer noch in ihrer Stimme. \enquote{Sie schienen sich zu streiten, als V-Voldemort den Tötungsfluch auf ihn warf.}

Plötzlich hatte sie die ganze Aufmerksamkeit des gesamten Gemeinschaftsraumes.

\gedanke{Den Tötungsfluch überlebt}, dachte Harry. \gedanke{Außer mir ist es Professor Elber nun auch gelungen.}

\enquote{Jedenfalls}, fuhr Lavender fort, \enquote{er hat ihn einfach umgelenkt und in den Himmel geschleudert. Dann hat es Du-weißt-schon-wer noch einmal versucht, aber Professor Elber warf ihn einfach zurück. Du-weißt-schon-wer konnte sich noch zur Seite werfen und warf einen weiteren auf Professor Elber.} Parvatis Augen weiteten sich und sie hielt ihre Hand vor den Mund. \enquote{Voldemort warf noch einmal einen Fluch auf ihn. Dieser streifte ihn allerdings und er brach zusammen. Er konnte gerade noch nach Hogsmeade apparieren und sich dann ins Schloss zurückschleppen.} Parvatis Blick fiel auf Harry, und Lavender drehte sich zu ihm herum.

\enquote{Woher weißt du das?}, fragte Harry sie.

\enquote{Ich habe es gesehen}, antwortete Lavender. \enquote{Es war wie eine Luftprojektion. Sie schwebte in der Luft und zeigte das Duell. Harry, ich habe noch nie von jemandem gehört, der den Tötungsfluch überlegt hat.} Und plötzlich fiel es ihr wieder ein. \enquote{Tut mir leid, Harry, ich meinte natürlich, von keinem weiteren.}

\enquote{Ist schon in Ordnung, Lavender}, antwortete Harry. \enquote{Aber wenn er den Tötungsfluch abgewehrt hatte, dann können wir das vielleicht auch lernen.}

\enquote{Daran habe ich noch gar nicht gedacht}, meinte Lavender. \enquote{Aber wieso hat Dumbledore so etwas nicht gemacht? Oder dir so was beigebracht?}

Harry konnte nur vermuten, antwortete aber: \enquote{Vielleicht kennt Dumbledore diese Art von Zauber nicht, oder Professor Elber wendete dafür schwarze Magie an. Schließlich unterrichtet er dieses Fach. Vielleicht sind seine Kenntnisse in schwarzer Magie derart umfangreich, dass er ihn abwehren konnte.} Plötzlich fiel ihm etwas ein. \enquote{Wir haben doch unten in der Kammer diese pulsierenden Blitze mit unseren Schilden abgewehrt. Vielleicht läuft das in genau diese Richtung und Professor Elber wollte uns nur noch nicht sagen, dass wir damit sogar den Tötungsfluch abwehren können.}

Schweigen herrschte im Gemeinschaftsraum.

\enquote{Wäre möglich}, meinte Parvati. \enquote{Aber ich will es nicht ausprobieren.}

\enquote{Ich werde mich jetzt umziehen und dann auf mein Valentinsdate warten}, sagte Lavender, stand auf und ging in ihr Zimmer.

\enquote{Gute Idee Lavender, das mache ich auch}, meinte Harry.

Einige im Raum mutmaßten bereits, dass \accentuate{die beiden} ein Date hatten.

Am Nachmittag stand Harry mit Kreachers vorbereitetem Parfum und frisch gewaschenem valentinstauglichem Dress im Haupteingang zum Schloss und wartete auf seine Überraschung. Wie von Katharina vorausgesehen, warteten viele Pärchen im Vorhof und starrten gespannt auf Harry. Nach etwa zwei Minuten kam, wie zuvor verabredet, Pansy und stellte sich kurz neben Harry.

Den anderen fiel ihre Kinnlade herunter. Sie raunte ihm in sein Ohr. \enquote{Sei froh, dass Katharina so eine gute Freundin ist.} Dann lief sie an ihm vorbei und traf sich mit Blaise Zabini und ging mit ihm den Schotterweg hinunter nach Hogsmeade.

Dieser Schock hat gesessen. Harry vernahm Gemurmel, doch Katharina kam gleich darauf und das Gemurmel verstummte wieder. Harry lächelte sie an.
Dann zog er einen kleinen Haarreif aus seinem Umhang und setzte ihn ihr auf. \enquote{Eine kleine Leihgabe, damit du auch entsprechend aussiehst}, sagte er und bot ihr seinen Arm wie bei einem Ball an. Diesen ergriff sie, indem sie ihren Arm ein hing und mit Harry loslief. Doch schon nach wenigen Metern, als sie an ihren Mitschülern vorbei, doch immer noch gut sichtbar waren, ließ Harry seinen Arm sinken, löste sich von Katharina und griff um ihren Arm herum. Dann verschränkte er seine Finger in ihren und lief den restlichen Weg bis nach Hogsmeade mit ihr Händchen haltend.

Harrys Lippen hingen immer wieder an Katharinas Ohr und immer wieder lachte sie. Mal, weil er eine Idee hatte, mal, weil seine Haare kitzelten, oder auch seine Lippen ihr Ohr sanft umspielten.

Luna war der Typ von Freundin, die die Bedürfnisse ihres jungen Freundes voll verstand. Ihr Verhältnis zueinander war durch ihre Art der Verbundenheit wohl etwas lockerer. Denn diese Art war einmalig. Auch sie hatte ein Date; mit Neville. Neville hatte ein Faible für Luna und genoss den Nachmittag mit ihr. Aber erst, nachdem ihm Luna und vor allem Harry mehrmals versichert hatten, dass es in Ordnung sei. Neville hatte sie mal erwischt, wie sie sich küssten. So weihten sie ihn, als einzigen Außenstehenden, ein; noch bevor sie es offiziell machten. Das, was Harry mit Luna verband, war mehr als nur Freundschaft. Es war eine Art Liebe, die man mit nichts vergleichen konnte. Doch Harry hatte das Gefühl, nicht bis zum Jahresende mit Luna zusammen zu sein. Trotzdem genoss er jeden Moment, den er mit ihr erleben durfte. Er verstand sie immer besser. Und sie ihn auch.

Aber heute war er mit Katharina da. Als er seinen Gedanken nachhing, überlegte er, wie er seine Mitschüler foppen konnte. In Hogsmeade angekommen, gingen beide erst einmal in den Honigtopf, um ihren Vorrat an Süßem aufzufüllen. Dann ging es die Einkaufsstraße entlang. Harry trug einen mit silbernen Runen bestickten nachtblauen Seidenumhang über seinem Hemd und seiner Stoffhose. Auf eine Krawatte oder Fliege verzichtete er. Katharina trug ein dunkelblaues Kleid, das mit seinem Umhang wunderbar harmonierte. Als sie ein Bekleidungsgeschäft betraten, um sich umzusehen, kam ein Verkäufer auf sie zu um sie zu bedienen und zu beraten. Als er Harrys Umhang sah, veränderte sich seine Miene.

Der Verkäufer sah das Paar ehrfürchtig an. \enquote{Ein ausgezeichnetes Stück, das Sie da tragen. Wissen Sie um die Symbolik der Runen?}, fragte der Verkäufer.

\enquote{Mein Elf hat sie mir erklärt}, antwortete Harry. \enquote{Diese hier}, er zeigte auf eine der Runen, \enquote{hält Neider ab. Diese hier}, er zeigte auf eine andere, \enquote{Stärkt Bindungen.}

\enquote{Wissen Sie auch um die tiefere Bedeutung, aus den alten Zeiten?}, fragte der Verkäufer nach.

Harry und Katharina verneinten.

\enquote{Sie stärkt die \accentuate{familiäre} Bindung \gst Früher wurden solche Umhänge zur Verlobung oder zur Hochzeit getragen, damit die Familien stärker aneinander gebunden wurden. Das hier ist ein sehr alter Umhang. Sie werden seit mehreren hundert Jahren nicht mehr hergestellt. Ich wusste gar nicht, dass die Familie Potter solche Sachen hat.}

\enquote{Der Umhang stammt nicht aus der Familie Potter}, erklärte Harry, was dem Verkäufer ein Stirnrunzeln einbrachte, \enquote{sondern aus der Familie der Black.}

Dem Verkäufer entglitten die Gesichtszüge. Sein Verhalten wurde, wenn es denn überhaupt noch möglich war, noch unterwürfiger. \enquote{Bitte verzeiht mir, Mister Potter.} Dann drehte er um und ging.

\enquote{Hatte der gerade Angst vor mir?}, fragte Harry Katharina.

\enquote{Sah ganz so aus}, antwortete sie. \enquote{Lass uns zu Madame Padifoot’s gehen.}

Harry nickte und gab Katharina einen Kuss auf die Wange. Diese errötete leicht und schmunzelte. Mit ihr hatte er ein Date, das Maßstäbe setzte. Die Kluft zwischen Gryffindor und Slytherin war heute etwas schmäler geworden.

Im netten Café angekommen fand sich schnell ein nettes Plätzchen, da ein Großteil der Paare den beiden gefolgt war, weil es so ungewöhnlich war. Nachdem beide eine Tasse Tee und ein Stück der Valentinstorte hatten, aßen und tranken beide, bevor Harry Katharinas Hände in seine nahm und mit ihren Fingern spielte und immer wieder küsste. Beide saßen an einem Tisch am Rand, aber nicht in einer Ecke, sodass beide eine gute Sicht auf die anderen Tische hatten. Sie funkelten sich an, als sie merkten, wie einige sich mit der Gabel und einem Stück Kuchen in die Wange oder das Kinn stachen. Sie mussten sich konzentrieren und die Augen des Gegenübers fixieren, damit sie nicht zu lachen anfingen. So sah es für Außenstehende aus, dass sie sich wirklich anschmachteten. Nachdem sie im Café fertig waren, schlenderten sie den Rest des Dorfes ab und machten sich danach auf den Rückweg zum Schloss.

Auf dem Rückweg tuschelten beide miteinander, wie sie ihre Show zu Ende führen wollten. Doch keinem der beiden fiel etwas ein, also entschlossen sie sich spontan zu sein.

Am Schloss angekommen legte Harry seine Hände um Katharina Taille, woraufhin sie ihre Hände um seinen Nacken schlug. Beide kamen sich näher und dann folgte der erste zarte Kuss, gefolgt von einem stürmischen. Dann gab es ein Poltern. Als sich beide wieder lösten und zum Schlossausgang sahen, lag eine Mitschülerin ohnmächtig auf dem Boden; vermutlich aus Schock, weil sich die beiden geküsst hatten. Katharina löste sich von Harry und lief auf ihre Mitschülerin zu, um sich um sie zu kümmern. Harry hingegen lief mit einem Lächeln auf den Lippen in das Schloss, um sich in den restlichen Stunden, die vor dem Abendessen lagen, noch für die morgigen Stunden vorzubereiten.

\trenn

Diesen Samstag ging es nach dem Mittagessen wieder in die Kammer. Heute würde ihnen Dumbledore etwas Neues beibringen. Die Schüler und ihr Schulleiter rutschten wie immer die Röhre hinunter, dann ging es durch das mittlerweile abgestützte Gewölbe in die Kammer. Die Tür stand bereits offen und Harry wartete mit ein paar Schülern auf den Rest, da sich die Tür immer wieder schloss; man konnte sie nicht dauerhaft aufhalten. Dumbledore kam mit ein paar Schülern herein, so langsam füllte sich der runde Raum.

Nachdem alle anwesend waren, begann Dumbledore und sagte: \enquote{Ich dachte mir, dass wir heute mal gelenkte Zauber durchnehmen. Nachdem ihr in diesen Kursen erweiterte Magie lernt, finde ich ist das passend. \gst Normalerweise}, sagte er und lief mit gezogenem Zauberstab im Raum umher, \enquote{wirken Zauber auf direktem Wege.} Er schwang seinen Zauberstab und ein leuchtend violetter Strahl verließ die Spitze seines Stabes und schlug auf der felsigen Wand ein. \enquote{Dies ist der Weg, den eure Zauber, wenn sie eure Stäbe verlassen, nehmen. Heute werden wir gelenkte Zauber durchnehmen, das heißt\abs} Er schwang seinen Stab und der Zauber vollzog eine Kurve, bis er hinter Dumbledore in den Felsen einschlug. \enquote{Ihr könnt den Zielort des Zaubers, nachdem er euren Stab verlassen hat, immer noch bestimmen. Das werden wir heute üben.} Dann stellte sich Dumbledore vor alle Schüler und begann erneut einen Zauber. Dieses Mal kam der Strahl viel langsamer aus seinem Stab und wieder beschrieb der Strahl eine Kurve. Dumbledore bewegte seinen Stab und es schien, dass der Strahl der Stabspitze folgte. Als alle aus dem Staunen heraus waren, hielt Dumbledore seinen Stab auf einer Position und lenkte den Strahl mit seinen Augen und seinem Kopf. \enquote{Es ist egal, ob sie ihren Stab nehmen, ihre Augen, oder sich ihr Ziel nur gedanklich vorstellen. \gst Wir fangen mit dem ersten an, da die anderen Sachen komplizierter sind.}

Harrys Magen wurde flau. Er musste aufpassen, dass er nicht zu viel von seinen Fähigkeiten preisgab. Er hielt sich zurück und wartete, wie sich die andern Schüler machten. Er hoffte, dass es bei ihm nicht durch Zufall auf Anhieb klappte. Er hing seinen Gedanken nach und bekam nicht ganz mit, was Dumbledore sagte. Er bekam nur mit, dass man sich darauf konzentrieren musste.

Nachdem Dumbledore fertig war, bat er seine Schüler, sich nebeneinander aufzustellen und mit dem Rücken zueinander zu stehen, damit sie sich nicht gegenseitig verletzten konnten. Dumbledore zählte von drei runter und die Schüler begannen ihre Zauber auszuführen. Viele \spruch{Ignitio solare}-Rufe klangen durch den Raum, mit denen die Schüler die leuchtend violette, aber sonst nichts wirkenden, Strahlen aus ihren Stäben lösten, um sie auf das gewünschte Ziel zu lenken. Bei ein paar wenigen konnte man eine kleine Korrektur um ein paar Zentimeter erkennen, obwohl es viel mehr sein sollte.

Harry sprach nun seinen Spruch und der Zauber kam genau dort an, wo er es beabsichtigte, nämlich ohne Abweichung genau dorthin, wo er gezielt hatte, als er den Zauber sprach und nicht dorthin, wo er danach mit seinem Stab zeigte. Während er den Zauber sprach, war er der Meinung, dass er die verstreichende Zeit langsamer empfand und so genügend Zeit hatte, den Zauber geradeaus zu lenken, obwohl er seinen Stab in einer Kurve beschrieb und der Strahl seiner Stabspitze zu folgen schien.

\enquote{Gut, gut}, sprach Dumbledore. \enquote{Das üben wir noch mal.} Erneut führte er den Zauber vor, damit die Schüler sich alles noch einmal einprägen konnten.

Draco Malfoy lächelte zu Harry herüber, als er seinen Zauber erfolgreich um mehrere Dezimeter abgelenkt hatte. Harry beherrschte sich jedoch und verkniff sich ein Duell mit ihm. Bei seinem nächsten Versuch lenkte er ihn auch weit ab, blieb aber deutlich unter dem von Draco. Innerlich freute er sich, da er es verstanden hatte und jederzeit seinen Zauber umlenkten konnte, wie er wollte. Zumindest war er der Meinung, dass es so sei.




\begin{kommentar}
Elber zeigt Harry und Dumbledore die Aufzüge. Aber wenn man bedenkt, dass er das Schloss erbaut hat, sollte man annehmen, dass er darüber Bescheid weiß.
\end{kommentar}

\begin{kommentar}
Nachdem Harry die Aufzüge entdeckt hat, muss er diese sofort benutzen, um Lavender und Parvati zu helfen. Im Aufzug drückt er das Symbol des roten Kreuzes und des roten Halbmondes. Die Symbole für medizinische Hilfe in der christlichen und muslimisch/arabischen Welt.
\end{kommentar}

\begin{kommentar}
Auf dem Rückweg von der Krankenstation sieht er Elber, wie er ein Amulett öffnet und eine blaue Kugel in seinen Körper eindringt. Er hat seinen Horkrux vernichtet und möchte seine Seele wiederherstellen.
\end{kommentar}

\begin{kommentar}
Kurz darauf kommt Elber schwer verletzt auf die Krankenstation. Dort zeigt sich einmal eine seiner wenigen Schwächen. Ab und an ist er sorglos oder zu unkonzentriert. Oder etwas zu arrogant und überheblich, was seine Fähigkeiten betrifft. Er überschätzt sich manchmal.
\end{kommentar}

\begin{kommentar}
Der erste Brief von Harrys Date enthält mehrere Anspielungen oder Vorgriffe auf die Geschichte. Pansy scheint ja eine ziemlich gute Freundin von Katharina zu sein. Später in der Geschichte fliegen ja alle Mädchen auf den jungen Zauberer. Hier kann man bei Pansy schon die ersten Anzeichen sehen.
\end{kommentar}

\begin{kommentar}
Der Name Katharina ist eine Anspielung auf Kathryn Janeway aus Star Trek Voyager. Dort nennt eine Figur den Captain Katharina.
\end{kommentar}

\begin{kommentar}
Pansy meint in einer Art Traum zu Harry, dass beide miteinander verwand seien. Erst im nächsten Teil erfährt Pansy, dass sie mit Gryffindor verwandt ist. Hier habe ich einen kleinen Hinweis darauf versteckt, obwohl man nicht darauf kommen kann. Erst, wenn man den zweiten Teil gelesen hat. Denn die Uniform kann im Traum auch nur so dagewesen sein.
\end{kommentar}

\begin{kommentar}
Ziemlich am Ende des Kapitels erfährt Dumbledore, dass Elfen Zauberstäbe haben dürfen und dass ein Gesetz von 1640 ungültig ist, da hierzu das vollständige Gamot hätte tagen müssen. Wenn man bedenkt, dass Elber schon ziemlich alt ist, kann man durchaus darauf kommen, dass er damals nicht dabei war, aber Mitglied des Gamots ist. Aber damals sahen das die anderen wohl etwas anders.
\end{kommentar}
