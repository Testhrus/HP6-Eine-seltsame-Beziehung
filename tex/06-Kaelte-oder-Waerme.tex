\chapter{Kälte oder Wärme}


Nachdem Harry wieder eine Nacht mit Luna verbracht hatte, wachte er am nächsten Tag auf. Es war wieder ein Sonntag, wie auch die vielen Male zuvor. Harry schaute zu Luna hinüber, streichelte langsam ihr Haar und legte sich wieder hin, als er merkte, dass sie noch schlief. Er erinnerte sich daran, dass er sich vorgenommen hatte, sich von ihr wecken zu lassen. \gedanke{Mal warten was passiert}, dachte er sich. Er schloss die Augen und entspannte. Nach einer Weile bemerkte er wie Luna erwachte. Unruhig drehte sie sich auf den Rücken, dachte er. Und er meinte, dass er sogar das schwach kratzende Geräusch des Sandes hörte, als sie sich die Augen rieb. Sie drehte sich zu Harry und fing unter der Bettdecke an sich umzudrehen. Harry musste sich beherrschen und tat so, als ob er schlafen würde. Er fühlte, wie Luna ihr rechtes Bein auf seine linke Seite legte, nur um kurz darauf direkt über ihm zu sitzen, die Arme an seiner Seite aufgestützt. Er fühlte ihre körperliche Nähe, obwohl er nicht das Gefühl hatte sie zu berühren. Langsam spürte er ihren warmen Atem seinem Gesicht näher kommen. Er musste sich anstrengen, nicht enttäuscht auszusehen, als er bemerkte, dass ihr Atem seine Nase wärmte. \gedanke{Verdammt}, dachte Harry. \gedanke{Will sie mich nur auf die Nase küssen? Hatte sie das Interesse an einem Kuss verloren?} Erleichtert stellte er jedoch fest, dass seine Nase langsam abkühlte und ein leichter warmer Luftstrom seinen Mund umspülte. Er gab einen befriedigenden Laut von sich. Fast so, als ob er jeden Moment mit einem guten Traum in Erinnerung aufwachen würde. Das war wohl der Moment, auf den Luna gewartet hatte. Er spürte ihre Lippen auf den seinen. Stärker, intensiver, fordernder als zuvor. Er öffnete seine Augen und Luna sah ihn an. Es war der gleiche Blick wie die beiden Male zuvor, als sie aufwachten und sich küssten.

Sie ließ von seinen Lippen ab und ein fröhliches: \enquote{Guten Morgen, Harry}, kam es ihm sanft entgegen.

Sanfter als er es zuvor erlebt hatte. Ihr Mund war nur wenige Zentimeter von seinem entfernt. Ihr Blick verriet ihm, sie wollte nun auch von ihm geküsst werden.

\enquote{Guten Morgen, Luna}, sagte Harry und erwiderte ihren Kuss. Er spürte wieder diese Zufriedenheit, diese Geborgenheit in sich. Es war anders, anders als bei Cho, als sie sich geküsst hatten, anders, als ihn Hermine in einer Phase überschäumenden Glückes Ende des vierten Schuljahres auf seinen Mund geküsst hatte. Dieser Kuss war anders. Nicht schwesterlich, sondern verlangend, begehrend, aber dennoch zart und wild entschlossen. Langsam löste er sich von ihr und dachte sich: \gedanke{Lass dich fallen Luna, lass dich einfach auf mich fallen. Ich möchte deine Wärme spüren, deinen Körper, einfach alles.}

\stimme{Ja Harry}, hörte er, als er in Lunas Gesicht blickte. Er hörte ihre Stimme klar und deutlich vor sich, als hätte sie selber gesprochen. Aber ihre Lippen bewegten sich nicht.

Jetzt wurde ihm klar, dass er ihre Gedanken gehört hatte. Sein Gesichtsausdruck nahm eine leicht erstaunte Form an und Luna ließ sich langsam nieder. Er spürte ihren Körper auf sich, ihre Wärme, ihre Nähe. Sie war mit nichts außer einem dünnen Nachthemd bekleidet. Ihre Backe lag dicht neben seiner. Ihre Hände begannen sanft seinen Hals an beiden Seiten zu umspielen. Harry genoss es, als er plötzlich merkte, dass er es scheinbar zu sehr genoss.

Jetzt hörte er wieder Lunas Stimme. \stimme{Mach dir keine Sorgen, mir geht es genauso, auch ich bin erregt.} Sie streichelte weiter seinen Hals und er konnte spüren, wie ihr Körper zu zittern begann, als auch er endlich aktiv wurde und seine Hände ihren Rücken berührten.

Er fing mit langsam kreisenden Bewegungen an ihrer Hüfte an und arbeitete sich langsam, fast quälend langsam, in der Mitte ihrer Wirbelsäule zu ihrem Hals entlang hoch. Befriedigend nahm er ihre Gedanken wahr.

\begin{abAchtzehn}

\stimme{Mach schneller Harry. Nicht so langsam. Ich halt's nicht mehr aus. Berühr’ meinen Hals.}

Innerlich lachend fühlte er, wie ihre Erregung in ihr hoch stieß. Ihr Verlangen nach seinen Bewegungen immer stärker wurde. Er dachte: \gedanke{Auch wenn ich nicht vorhabe mit dir zu schlafen, es ist ein wahnsinniges Gefühl.} Er biss sich beinahe auf die Zunge, als er diese Gedanken hatte, genau wissend und doch nicht hoffend, dass sie ihn hören würde. Als er aber ihre Gedanken empfing: \stimme{Das möchte ich auch nicht, noch nicht}, wurde er wieder ruhiger.

Mittlerweile war er an ihrem Hals angekommen und streichelte ihn ganz sanft. Er genoss jede ihrer Berührungen, wie auch sie jede der seinen genoss. Jetzt fing er an, sich in ihren langen schneeweißen Haaren zu vergraben. Er dachte daran ihr Gesicht zu umfassen und sie wieder zu küssen. Als er begann ihr Gesicht hochzuheben, kam sie ihm entgegen. Sie wollte es genauso wie er und beide versanken in einem langen Kuss. Er öffnete leicht seinen Mund und spürte bereits ihre Zunge, die anfing mit seinen Zähnen zu spielen.  Er wollte ewig mit ihr hier liegen, sich berührend, die Gedanken des anderen lesend und nichts weiter als sich zu streicheln, zu küssen und den anderen zu verwöhnen. Er hatte das Gefühl, eine Einheit mit ihr zu bilden. Er fing an, den Kuss kurz zu unterbrechen, nur um ihr Nachthemd auszuziehen, warf es aus dem Bett und drehte sich auf die Seite, bis er auf ihr lag. Seine Finger immer noch in ihren Haaren, begann er das Spiel mit der Zunge fortzufahren und ihr Gesicht mit vielen Küssen und dem leichten Spiel mit seiner Zunge zu verwöhnen. Er spürte sie und hörte ihre Gedanken, die ihm nur eines vermittelten \gst Mach weiter.

Als er an ihrem Hals angekommen war, löste er seinen Mund von ihrem und schaute ihr in die Augen. Er fand keine Spur mehr von ihrem sonst so verträumten Blick. Sie war voll konzentriert, entspannt und auf ihn fixiert. Er bewegte wieder seinen Mund in ihr Gesicht, aber nur um sie kurz zu küssen und dann wieder loszulassen. Er spürte wie sich ihre erregten Nippel gegen seine Brust drückten und genoss dieses Gefühl. Er wusste genau, was sie wollte, wie sie es wollte und wann sie es wollte. Er begab sich auf Entdeckungstour. Keinen Zentimeter ihres Körpers wollte er auslassen, dachte er sich. Sein Mund wanderte, ihre Haut ganz zart berührend, an ihrer Kehle entlang. Nur um dort kurz zu verweilen, einen Kuss zu hinterlassen, und sich dann weiter ihrem Körper zu widmen. Er umspielte ihre rechte Brust mit seinem Mund und seiner Zunge und hörte zum ersten Mal ein wohliges Raunen in seinen Ohren. Sie hatte es nicht nur gedacht. Mit seiner rechten Hand umspielte er ihre andere Brust. Sie genoss es sichtlich, denn als er einmal aufschaute, sah er ihre geschlossenen Augen. Er hörte sie wieder in seinen Gedanken.

\gedanke{Mach weiter, Harry. Anschauen kannst du mich später noch.}

Er machte weiter, bis er an ihrem Fußknöchel angekommen war. Für einen kurzen Moment hielt er inne. Setzte sich auf, sah an ihr herauf und war sprachlos. Sie lag nun vor ihm. Vollkommen nackt und schön. Sie öffnete ihre Augen und sagte zu ihm.

\enquote{Jetzt bist aber du dran. Ich habe mir für dich was Besonderes überlegt} wohl wissend, dass er ihre Gedanken lesen konnte.

Sie setzte sich auf und warf ihn mit leichtem Druck auf das Bett zurück. Sie kam wieder seinem Gesicht näher und küsste ihn, löste den Kuss und begann das gleiche Programm, das er ihr zu Teil hatte werden lassen. Denn jetzt hatte er nicht nur das Gefühl, ihre Gedanken lesen zu können, sondern auch ihre Gefühle zu teilen. Das war überwältigend. Er spürte ihre Haut auf seiner, ihren Mund seinen Körper entlang spielend, mit der Zunge seine Brustwarzen umspielen. Dabei durchdrang ihn das wohlige Gefühl wie sie es genoss. Das hatte er noch nie erlebt! Er hatte sich schon zuvor vorgestellt, wie es wäre, mit einem Mädchen intim zu werden. Aber das übertraf seine kühnsten Erwartungen. Er fühlte seine und Lunas Gefühle. Es war schwer sie auseinander zu halten und dabei so schön. Ihn überkam wieder das Gefühl mit ihr eine Einheit zu bilden. Zwei Wesen im selben Körper zu sein, oder ein Wesen in zwei Körpern zu sein. Er konnte sich fast vorstellen, wie sie ihn anblickte. Fast war es ihm so, also ob er auf sich hinab durch ihre Augen sehen konnte. Als Luna fertig war, setzte er sich auf und gab ihr einen langen Kuss. Er hatte das Gefühl, als würde ihn das Glück nie wieder verlassen.

\end{abAchtzehn}

\begin{safedivide}
\fskdivider
\end{safedivide}

Nebeneinander sitzend, voll von Schweiß und Speichel des anderen, lagen sie sich in den Armen. Er hatte seinen Kopf auf ihre Schulter gelegt und atmete wie nach einem Marathonlauf. Er wusste nicht, was er Hermine und Ron erzählen sollte, ober er ihnen überhaupt etwas erzählen sollte, denn es fiel ihm wieder Professor Elber ein, der ihn bei einem seiner morgendlichen Spaziergänge etwas über die Dementoren erzählt hatte. Er verwarf den Gedanken und war froh, dass Luna scheinbar zu erschöpft war, seine Gedanken zu lesen. Er stand auf und war auf dem Weg zur Tür.

\enquote{Was hast du vor}, hörte er Luna.

\enquote{Eine Dusche suchen}, gab er zur Antwort.

\gedanke{Warte, ich komme mit}, kam es ihm entgegen. Dieses Mal hatte sie es nur ge\-dacht und Harry grins\-te.

Ein paar Meter weiter fand er eine Aufschrift mit der Bezeichnung Dusch- und Baderäume. Er öffnete die Tür und sah sich um. Luna stand dicht gedrängt hinter ihm und lehnte ihren Kopf auf seine Schulter. Harry ging in eine der freien Kabinen und drehte sich um, nur um Luna die Hand hinzuhalten.

\gedanke{Nimm sie schon}, dachte er. \enquote{Komm Luna, duschen wir gemeinsam.}

Luna kam auf ihn zu und durch Harry lief ein wohliger Schauer. Sie nahm seine Hand und stieg mit ihm in die Dusche. Er öffnete die Hähne und ließ das Wasser über seinen Körper laufen. Luna drängte sich hinter ihn, fuhr mit ihrer Hand zwischen seinem Körper und seinem Arm durch und griff sich die Seife. Mit der anderen Hand tat sie dasselbe und rieb unter einem Wasserstrahl die Seife zwischen ihren Händen. Ihren Körper dicht an ihn gedrückt. Nachdem sich genügend Schaum gebildet hatte, legte sie die Seife zurück, nahm ihn einen Schritt mit nach Hinten, außer Reichweite des Wassers und begann seinen Vorderkörper einzuseifen. Er spürte ihre Brüste an seinem Rücken und ihm entstieg ein Seufzer. Nachdem sie seinen Vorderkörper und seinen Rücken gewaschen hatte, drehte sie sich um und signalisierte ihm, es ihr gleichzutun. Er wollte sich die Spannung etwas bewahren und fing mit ihrem Rücken an, da er sich nach dem Einseifen seiner Hände umdrehen musste. Er rieb in kleinen Kreisen ihren Rücken entlang, seifte sie von oben bis unten ein, und fuhr schließlich zwischen ihren Körper und ihre Arme. Er seifte ihre Vorderseite ein und umspielte zum Schluss ihre Brüste.

Ihr Kopf war auf seiner Schulter zu liegen gekommen und sie sagte neckisch zu Harry: \enquote{Hast dir das Beste wohl zum Schluss aufgehoben?}

\enquote{So weit sind wir noch lange nicht}, meinte Harry, als er einen Seufzer von ihr hörte.

Nachdem sie sich die anderen Körperteile selber eingeseift und von der Seife wieder befreit hatten, stieg Luna zuerst aus der Dusche, um sich ein Handtuch zu nehmen, es sich um den Körper zu wickeln und mit einem weiteren die Haare zu trocknen. Harry drehte in der Zwischenzeit das Wasser ab, stieg ebenfalls aus der Wanne und wickelte sich auch ein Handtuch um seine Taille. Er schaute sich um und entdeckte Spiegel, die über mehreren Waschbecken angebracht waren. Plötzlich fiel ihm auf, dass Luna verschwunden war; aber sie stand schon wieder im Türrahmen mit ihrem Zauberstab in der Hand. Harry stutzte. Sie stellte sich vor einen Spiegel, nahm ihr Haartuch ab und murmelte etwas, das Harry nicht verstand. Mit ihrem Zauberstab fuhr Luna nun über ihr Haar. Harry hörte ein leichtes Surren. Ihn im Spiegel beobachtend und einen etwas ratlosen Gesichtsausdruck zeigend, sagte ihm Luna, dass das ein Haar trocknender Zauberspruch sei. Als sie mit ihren Haaren fertig war, kam sie zu Harry, stellte sich hinter ihn und begann das Gleiche mit seinen Haaren. Im Nu waren sie trocken und Harry hatte sich vorgenommen, sie nach dem Zauberspruch zu fragen.

\zauber{Haraare desert}, hörte er plötzlich.

Sie musste wieder seine Gedanken aufgenommen haben. Trockenen Haares gingen beide wieder in ihr Zimmer und Harry fiel zum ersten Mal auf, dass an der Tür etwas stand. Er war sich sicher, dass dort nie ein Schild gehangen hatte. Auf dem Schild stand \accentuate{Privatraum von Luna Lovegood und Harry Potter.} Harry war sich nicht sicher, was er davon halten sollte und ging zum Schrank auf der gegenüberliegenden Seite des Raumes, um sich etwas zum Anziehen zu holen. Vor dem Schrank stehend drehte er sich noch einmal um. Er sah wie Luna ihr Handtuch abnahm und es aufs Bett warf.

\enquote{Du siehst bezaubernd aus Luna}, sagte er.

Ihre langen Haare lagen über ihren Schultern, ein paar lagen auf ihrem Oberkörper und fielen herab. Ihre Brüste sahen straff und doch zierlich aus. Sein Blick wanderte tiefer und er musste feststellen, jedes Haar an ihr war schneeweiß. Harry war sich der Tatsache bewusst, dass Luna ihn ebenfalls sehr genau beobachtete. Er konnte es fühlen, konnte es hören. Er drehte sich wieder um, als sich Luna auf den Weg zum Schrank machte und zog seine durch Dobby sorgsam zusammengelegte Kleidung heraus, um sich anzuziehen.

Luna stand neben ihm und bekleidete sich auch. Nachdem sich beide wieder angezogen hatten, hörte er wie Luna ein leises und unsicheres: \enquote{Sind wir jetzt ein Paar?}, abgab.

Harry antwortete: \enquote{Ich bin mir nicht sicher Luna, ich glaube schon. Aber noch ist es zu früh, dass anderen von uns zu erfahren. Warten wir noch etwas ab.}

Sie nickte und ging durch die Tür nach unten. Harry drehte sich wieder zum Schrank, schloss ihn und schaute sich ein letztes Mal, im Türrahmen stehend, um, nur um sich zu vergewissern nichts vergessen zu haben. Im Gemeinschaftsraum der Paare wieder angekommen, stand Luna vor dem Bücherregal und hatte eines der unbeschrifteten Bücher in der Hand.

Harry näherte sich und konnte lesen:

\begin{brief}
Dieser Gemeinschaftsraum der Paare wurde von uns als ein Zuflucht- und Rückzugsort in Hogwarts geschaffen. Verschiedene Schutzzauber verhindern die zufällige Entdeckung und das Aufspüren dieses Ortes. Leider hat einer der Zauber, den wir nicht mehr rückgängig machen konnten, einen kleinen Nebeneffekt. Wenn sich zwei seelenverwandte Personen hier einfinden, entwickeln sie eine so enge Beziehung zueinander, dass sie Gedanken und Gefühle des anderen lesen und deuten können; zu jeder Zeit, an jedem Ort. Dieses kann man durch Training abblocken. Außerdem kann es in \gst und sei die Wahrscheinlichkeit auch noch so klein \gst wenig Fällen dazu führen, dass man durch die Augen des anderen sehen kann, hören, was der andere hört, fühlen, was der andere fühlt. So als seien es die eigenen Empfindungen. Aber normalerweise haben die Paare nur eine intuitive Empfindung, wo sie ihren Partner finden und was er fühlt.
\end{brief}

\enquote{Das erklärt also, was mit uns passiert ist}, sagte Harry. \enquote{Steht noch mehr drin?}

Luna blätterte um, doch die nachfolgenden Seiten waren leer. \enquote{Vielleicht sind wir noch nicht so weit}, sagte Luna, als sie sich Harry zuwandte.

Sie stellte das Buch wieder in das Regal und die Beiden verschwanden, um in der Großen Halle zu frühstücken. Sie war noch leer, als die beiden ankamen. Beide beschlossen, sich an ihre jeweiligen Tische zu setzen und schon mal mit dem Frühstück zu beginnen. Harry hatte noch etwas Zeit und ließ sich, während er seinen Kürbissaft, den er dieses Mal mit Orangensaft mischte, etwas einfallen, was er Ron und Hermine sagen konnte. Schon halb mit dem Frühstück fertig, füllte sich die Große Halle langsam und auch Hermine trat mit Ginny ein. Sie setzten sich gegenüber von Harry.

Hermine fragte ihn: \enquote{Wo warst du letzte Nacht? Neville hat gesagt, du warst nicht in deinem Bett.}

\enquote{Ich war beim Lesen eingeschlafen.}

\enquote{Beim Lesen eingeschlafen? Du warst nicht einmal im Gemeinschaftsraum}, schnauzte ihn Hermine an.

Harry antwortete nur etwas von Strafarbeiten und Nachsitzen. Hermine und Ginny schüttelten, sich gegenseitig anschauend, nur die Köpfe und fingen mit ihrem Frühstück an.

Als er mit Ron und Hermine die Große Halle verlassen hatte, um zusammen Hausaufgaben zu machen, sagte Hermine plötzlich: \enquote{Sagt mal, ist euch aufgefallen, dass Luna Lovegood irgendwie anders aussieht?}

Harry wusste natürlich, was sie meinte, verkniff sich aber irgendwas zu sagen und antwortete nur lapidar. \enquote{Was meinst du?}

\enquote{Na, ihren Gesichtsausdruck. Sie schaut nicht mehr so aus, als ob sie den ganzen Tag träumen würde.}
Harry grinste nur.

\trenn

\enquote{Ein interessantes Buch haben sie da, Professor}, meinte Hermine zu Dumbledore, als er es auf dem Küchentisch des Grimmauldplatz 12 aufgeschlagen hatte.

\enquote{Das habe ich mir ausgeliehen}, sagte Dumbledore. Er blätterte durch das Inhaltsverzeichnis und schlug danach eine bestimmte Seite auf. Dann suchte er in den nachfolgenden Seiten einen bestimmten Zauberspruch. \enquote{Ah, hier haben wir ihn.} Er zog seinen Zauberstab und ging in den schmalen Flur vor das mit Stoffen verhangene Bild von Mrs Black. Dumbledore murmelte einige Zauber und das Bild fiel von der Wand.

Der Stoff viel herunter und sofort begann Miss Black zu fluchen und zu schreien. Bis sie merkte, dass ihr Bild am Boden stand. \enquote{Was habt ihr mit meinem Bild gemacht? Blutverräter. Abschaum}, brüllte sie. Sofort legte Harry wieder ein Tuch über das Bild, während er Dumbledore hinterherging und die Schreie und Flüche verstummten. Währenddessen blätterte Hermine weiter in dem Buch und sah sich diverse Zaubersprüche an.

Es war gerade Samstag und Harry und Hermine hatten die Erlaubnis, unter Aufsicht Sirius' altes Haus zu entrümpeln. Die Treffen des Ordens wurden verlagert, da keiner nach Sirius’ Tod mehr hier die wichtigen Themen besprechen wollte. Andererseits wusste niemand so genau, welche Auswirkungen es hatte, wenn die Person starb, welche den Zauber aussprach, aber nicht der Geheimniswahrer war. Doch trotz Dumbledores Bemühungen und der Zusage, dass es keine negativen Auswirkungen hätte, wollte keiner mehr an den Ort zurück, der das Zuhause eines ihrer gefallenen Kameraden war.

Harry hatte Kreachers Küchenschrank geöffnet und die gehamsterten Sachen auf den Küchentisch gelegt. Nachdem er den Abfall weggeworfen hatte (verschimmeltes Brot, verdreckte Küchentücher, welken Salat und andere unnütze Sachen), zauberte er eine Schachtel herbei, in die er die Reste legte, welche er Kreacher bringen würde. Dumbledore lief unterdessen durch das Haus und entfernte die Dauerklebezauber der Dinge, die man nicht mehr brauchte.

\enquote{Harry?}, sagte Dumbledore plötzlich. Harry sah zu ihm und er sprach weiter: \enquote{Nenn’ mich in Zukunft, wenn wir alleine sind ruhig Albus. \gst Aber eigentlich wollte ich dir etwas anderes sagen.} Er pausierte kurz. \enquote{Ich möchte, dass dir dein \VgddK-Lehrer Einzelunterricht gibt. Das heißt eigentlich \gst ich denke, dass er ihn dir sowieso gibt, auch ohne meine Erlaubnis. \gst Ich will dir damit eigentlich nur sagen, dass du dich darauf einstellen solltest, dass er dir private Stunden geben wird. \gst Er sieht es genauso wie ich, dass dir Gefahren drohen, für die ich keine Zeit habe dir Gegenmaßnahmen beizubringen\abs nicht mehr.}

\enquote{Professor}, sagte Harry. Dumbledore hob seine Hand und schaute ihn streng an. \enquote{Albus, was bedeutet das, was wird er mir beibringen?}, fragte Harry ganz überrascht.

\enquote{Ich schätze erweiterte Magie. Etwas, was man normalerweise nicht in der Schule lernt. Eventuell auch einen Einblick in die schwarzen Künste, damit du weißt, womit du es zu tun bekommst, da Voldemort ein großes Interesse an dir hat.}

\enquote{Da haben wir während des Unterrichtes schon etwas angefangen}, antwortete Harry vorsichtig.

Dumbledore sah ihn an, als hätte er zum ersten Mal einen Geist gesehen. \enquote{Er hat was?}, fragte er nach.

\enquote{Er weist uns gerade in die dunklen Künste ein}, sagte Hermine, die gerade in den Flur kam. \enquote{Nur leichte Sachen, damit wir wissen, warum und wie wir uns verteidigen müssen.} Dumbledore war sprachlos, aber auch leicht verlegen. Anscheinend hatte er es auf diese Weise noch nicht gesehen. \enquote{Wissen Sie, Professor}, erzählte Hermine weiter, \enquote{ich habe den Eindruck, dass er es als ganz normal empfindet, darüber zu sprechen.} Dann ging sie mit dem Bild von Mrs Black auf den Dachboden hinauf.

Dumbledore sah Harry nachdenklich an. \fluestern{Sollte ich mich in ihm getäuscht haben?}, fragte er sich selbst.

Um das Thema zu wechseln, sprach er: \enquote{Gilt das mit Albus, auch wenn Hermine oder Ron dabei sind, Albus?}, fragte er leicht grinsend.

Dumbledore antwortete: \enquote{Ja.} Dann begann sein Magen zu knurren. \enquote{Wird Zeit, dass wir was essen. Wenn Hermine wieder da ist, dann gehen wir in ein kleines Muggellokal. Ich lade euch ein. Komm, wir ziehen uns schon einmal unsere Jacken an.}

Beide gingen nach draußen und warteten mit ihren Jacken auf Hermine, die kurz darauf herunterkam und sie erstaunt ansah.

\enquote{Komm Hermine, Albus lädt uns zum Essen ein. Wir gehen in ein Muggellokal.}

\enquote{Albus?}, fragte sie ganz ungläubig.

\enquote{Solange wir alleine sind, oder maximal Minerva dabei ist, dann gilt das}, sagte Albus und reichte Hermine ihre Jacke.

Sie verließen das Haus und Harry sicherte es mit ein paar Zaubern, die er in einem Buch heute Morgen gefunden hatte. In diesem stand, wie man das Haus der Blacks sichern konnte und die Zauber, die auf ihm lagen, aktiviert wurden. Nur der Hausherr konnte diese Zauber aktivieren und sie für die Familienmitglieder freigeben.

\gedanke{Sirius musste davon nichts gewusst haben}, überlegte Harry.

Der Weg zum Lokal war kurz und für die Zeit etwas windig. Aber der Wind, der ihnen entgegenblies, war noch warm.

Im Lokal angekommen, es war eher eine gemütliche Wirtschaft, setzten sich Harry und Albus über Eck an die Wand hinter einen Tisch. Hermine setzte sich neben Harry, sodass alle drei den Schankraum gut im Blick hatten. Die Bedienung kam sofort und nahm die Getränke auf.

\enquote{Ihr seid meine Enkel, falls euch jemand fragt. Wir sind auf der Durchreise von Bath und wollen an die Ostseite, um Verwandte zu besuchen}, sagte Albus zu den Zweien.

Hermine und Harry nickten und sahen sich weiter um, während die Getränke gebracht wurden. Am Stammtisch fiel ihnen ein Mann auf, der scheinbar anzugeben schien.

\enquote{Ekelhaft, wie der angibt}, meinte Hermine. Harry nickte nur.

\enquote{Albus? Warum habe ich keinen vom\abs na ja, keinen im Haus gesehen?}

Albus druckste etwas herum. \enquote{Nach Sirius’ Tod wollten die anderen nicht mehr so gern ins Haus. Es war ihnen schon damals nicht ganz geheuer. Außerdem brauchen wir deine Erlaubnis als neuem Hausherrn.}

\enquote{Von mir aus gern}, antwortete Harry. \enquote{Ich bin eh die meiste Zeit über in Hogwarts und in den Ferien muss ich ja zu meinen Verwandten.} Dabei zog er eine Miene. \enquote{Was ist eigentlich mit meiner Tante los? Weißt du was? Ich bin der Meinung, dass sie die Zeit, in der ich bei ihr war, sich anders verhalten hat. Ich habe Begriffe von ihr gehört, von denen ich eigentlich sicher war, dass sie sie nicht kannte. Oder zumindest nicht in den Mund nehmen würde.}

Das Essen wurde gebracht und sie begannen den Fisch, das Rindfleisch und den Salat zu essen.

\enquote{Darauf kann ich dir leider keine zufriedenstellende Antwort geben}, sagte Albus.

Harry musste diese Antwort akzeptieren. Er wusste, dass Albus ihm die richtige Antwort nicht geben würde, sollte er sie wissen. Andererseits wollte er vielleicht auch nicht zugeben, dass er es nicht wusste. Es war recht kurzweilig. Nach dem Essen zahlte Albus und die beiden dankten ihm, als die Bedienung wieder gegangen war. Der Mann am Stammtisch gab immer noch an. Jetzt hörten die drei es genauer.

\enquote{Ich saufe noch immer jeden unter den Tisch, der mich herausfordert.}

Albus, der den Mann die ganze Zeit mit einer Abneigung angesehen hatte, kam zu dem Entschluss, dem Aufschneider eine Lektion zu erteilen.

\enquote{Wartet mal, oder trinkt noch einen Tee, das kann jetzt etwas dauern.} Er ging auf den Mann zu und sprach ihn an. \enquote{Nehmen Sie es auch mit mir auf?}, fragte er den Mann höflich.

\enquote{Ich trinke doch nicht gegen dich, Alterchen}, spottete der Mann.

\enquote{Ahh, große Klappe, aber wenn sie einer herausfordern will, dann ziehen sie den Schwanz ein?}

\enquote{Ich will dich nur vor Peinlichkeiten bewahren, Opa.}

\enquote{Und an Respekt mangelt es ihnen auch, wie ich sehe.} Dumbledore setzte sich und fragte weiter: \enquote{Was sind die Bedingungen des Wetttrinkens?}

Der Aufschneider wollte gerade seinen Mund wieder öffnen, doch einer seiner Kumpel unterbrach ihn. \enquote{Beide trinken gleichzeitig einen Schnaps nach dem anderen, bis einer keinen mehr trinken kann, oder will. Der Verlierer zahlt alle getrunkenen Schnäpse. Also seine eigenen und die des Gegners.}

\enquote{Das ist annehmbar. Ich akzeptiere.} Dumbledore schnippte mit den Fingern und sah dabei Richtung Theke.

Der Wirt nickte ihm zu und kam mit zwei Schnapsgläsern und zwei Schnapsflaschen auf einem Tablett zu den beiden Kontrahenten. Die Schnapsgläser stellte er vor Dumbledore und dem Mann ab, die Flaschen hingegen vor seinem Kumpel, der die Regeln erklärte, damit dieser sich vom ordnungsgemäßem Zustand des Schnapses überzeugen und die Gläser einschenken konnte.

Der Freund öffnete die Flasche und roch daran. Dann schenkte er beiden ein. Dumbledore nahm das Glas zwischen zwei Fingern und drehte es auf dem Tisch und eine viertel Umdrehung. Dann nahm er das Glas hoch und wartete, bis sein Gegenüber ebenfalls so weit war. Zeitgleich kippten sie sich die Schnäpse hinunter.

So ging das viele Gläser weiter. Dumbledore drehte jedes Mal sein Schnapsglas auf dem Tisch und trank es dann erst leer.

\enquote{Wie schafft er so viel Alkohol?}, fragte Hermine Harry, nach einem Schluck Tee. \enquote{Er müsste doch schon längst besoffen sein.}

\enquote{Schau dir doch mal an, was er mit dem Schnapsglas macht}, hauchte der in ihr Ohr. \enquote{Er dreht es, also wird er wohl irgendwas damit machen.}

Hermine spürte eine Gänsehaut an ihrem Nacken herunterlaufen. Harry entging dies nicht, also ging er einen kleinen Schritt weiter und küsste ihr Ohr und biss kurz darauf ganz sanft in ihr Ohr. Sofort ließ er wieder von ihr ab und nippte an seinem eigenen Tee.

Hermine war noch für eine Sekunde ganz benommen, dann fing sie sich wieder, schlug Harry leicht auf den Arm und meinte nur: \enquote{Harry.}

\enquote{Es hat doch funktioniert, oder?}, fragte Harry ganz scheinheilig.

\enquote{Was funktioniert?}, fragte Hermine nach.

\enquote{Es lenkt deine Gedanken von dem Wetttrinken und deinem Ekel davon ab}, grinste Harry.

Überraschend gab sie ihm einen Kuss auf die Wange und meinte dann: \enquote{Da hast du recht. Es hat geholfen, danke.}

Dann hörten sie ein Grölen und einen \geraeusch{Plumps}. Der Mann lag bewusstlos am Boden, während Dumbledore seinen letzten Schnaps austrank, dann aufstand und leicht zu wanken schien. Mit glasigem Blick verabschiedete er sich und verließ das Lokal. Harry warf ein paar Pfund auf den Tisch, nickte dem Wirt kurz zu und verließ mit Hermine das Lokal. Um die Ecke fing sich Dumbledore wieder und lief nüchtern und gerade zwischen Hermine und Harry zurück zum Grimmauldplatz.

\enquote{Netter Trick mit dem Schnaps-zu-Wasser. Aber wie?}

Dumbledore grinste. \enquote{Ein einfacher Zauber. \spruch{Decoholo}.}

\enquote{Den muss ich mir merken. Ist sicherlich hilfreich, falls man auf einer Party eingeladen ist.}

Hermine sah ihn nur entgeistert an. \enquote{Heißt das, dass sie\abs keinen Tropfen Alkohol getrunken haben, P\abs Albus?}

Albus grinste sie an: \enquote{Genau, Hermine. Nur Wasser.}

\enquote{Und ihr Wasserbauch?}

\enquote{Der war etwas schwerer. Das Wasser muss noch in der Speiseröhre verdunsten und als feiner, nicht sichtbarer Nebel durch Mund und Nase entweichen. Das hat nicht immer geklappt}, grinste er. \enquote{Die nächsten Stunden brauche ich keine Flüssigkeit mehr. Und jetzt muss ich erst mal, wenn wir zurück sind.}

Im Verlauf des weiteren Tages verbrachten sie die meiste Zeit damit, Räume zu putzen. Harry und Hermine machte es Spaß, die Räume aufzuräumen und mit Zaubern vor erneutem Einstauben zu sichern. Mithilfe des Buches war es leicht, die Doxys mitsamt den Eiern zu entfernen, die beißenden Teppiche stillzulegen und andere Artefakte zu entzaubern.

Am Ende des Tages hatten sie etwa ein Viertel der Räume von diversen lästigen Sachen befreit. Vom Staub wischen ganz abgesehen. Denn der Staub schien ebenso schnell wieder zurückzukommen, wie sie ihn entfernten. Doch es gelang ihnen schließlich mit speziellen Zaubern aus dem Buch. Dumbledore gab Hermine das Buch in die Hand. Dann verließen sie das Haus und stiegen auf die unterste Stufe der Treppe. Hermine und Harry fassten je einen Arm von Professor Dumbledore und er disapparierte mit ihnen.

Zurück auf Hogwarts überreichte Hermine Dumbledore das Buch und ging mit Harry und seiner Schachtel Richtung Gryf\-fin\-dor-Ge\-mein\-schafts\-raum. Den Inhalt würde er Kreacher morgen geben.

Hermine war bei der Hinreise erstaunt, wie es Dumbledore schaffte, durch die Schutzzauber des Schlosses zu kommen. Aber als er ihr sagte, dass er als Direktor die Möglichkeit hatte, sich einen speziellen Ort auszusuchen, von dem das klappte, und es nur ihm, sowie mit ihm verbundenen Begleitern, klappte, akzeptiert sie es. Bei nächster Gelegenheit würde sie in der Bibliothek nach diesem Phänomen suchen.

Nach dem Frühstück ging Harry in den Gemeinschaftsraum zurück, um seine Schachtel zu holen. Dann ging er hinunter Richtung Kerker und bog in den Gang ab, der in die Küche führte. Er stellte seine Schachtel ab und kitzelte die Birne, drehte danach den Knopf und öffnete die Tür, nahm seine Schachtel und betrat die Küche. Die Tür hinter ihm schloss sich.

Die Elfen räumten die Teller und das Besteck, sowie die Trinkbecher, weg, die auf den Tischen erschienen. Harry war der Meinung, dass die Elfen heute etwas langsamer wären als sonst. Er sah sich um und entdeckte Kreacher, der gerade Teller reinigte. Sie schwebten durch das Spülbecken und kamen sauber heraus. Danach flogen sie in die offenen Türen der Küchenschränke.

Harry lief auf ihn zu. \enquote{Kreacher?}

Der alte Elf drehte sich um. \enquote{Ja Meister}, sagte er mit öliger Stimme.

\enquote{Ich habe da etwas für dich.} Harry überreichte ihm die Sachen aus dem Küchenschrank, in dem er sonst schlief. \enquote{Das schlechte Essen und die dreckigen Sachen habe ich weggeworfen. Aber der Rest ist wohl deines}, sagte Harry.

Kreacher nahm die Schachtel an sich und sah sich jedes Stück genau an. Harry nahm an einer langen Bank Platz. Er machte sich keine Gedanken darüber, aber er saß an einem Tisch, der direkt unter dem Tisch der Slytherins stand. Kreacher verbeugte sich tief und lies ein leises \enquote{Danke Meister}, erklingen. Dann verschwand er.

\gedanke{Er muss wohl seine Schätze in Sicherheit bringen}, dachte Harry. Als Kreacher danach wieder kam, hatte er für Harry ein Glas Kürbissaft in der Hand. Harry nahm es dankend entgegen und trank es aus. Er wollte nicht unhöflich sein. Dann verabschiedete er sich von Kreacher, um sich seinen Hausaufgaben zu widmen.

\trenn

Als am Freitagabend Katie in den Gemeinschaftsraum kam, war sie ganz aufgelöst. Sie kam gerade von der Krankenstation, wo sie sich von Madame Pomfrey eine Verletzung an der Hand kurieren lassen wollte. Sie setzte sich schluchzend in einen Sessel und merkte nicht, dass dort bereits jemand saß.

\enquote{Au}, meckerte Ron, auf den sich Katie setzte. Sie schien ihn nicht zu beachten. Ron nahm Katie unter den Schultern hoch, um aufzustehen. Dann setzte er sich kurz auf eine Armlehne, um Katie herunterzulassen. Fast hätte er das Gleichgewicht verloren, wenn ihn Hermine nicht gehalten hätte. \enquote{Danke}, sagte er schnell. Er setzte sich neben Harry und betrachtete Katie, die bittere Tränen heulte. Ihre Hände hatte sie unter ihrer Robe versteckt.

\enquote{Was ist los Katie?}, fragte Hermine.

\enquote{Ich\abs} stammelte Katie. Doch ehe sie weiter machen konnte, um ihre Tränen aus dem Gesicht zu wischen, rutschten ihr, als sie ihre Hände nach oben nahm, ihre beiden Ärmel herunter und zeigten den Grund für Katies Heulattacke. Sie hatte nur noch ihre linke Hand. Die Rechte war ab und der Arm hörte kurz bevor das Handgelenk anfing auf. Sie hatte nur noch einen Stumpf.

Hastig schlug sie ihre Ärmel wieder hoch, damit es keiner sehen würde. Katie schluchzte noch mehr und saß kurz darauf auf Rons Schoss und begann ihn zu umarmen. Dieser riss die Augen ungläubig auf und wollte Katie schon abweisen, als Hermine seine Hände ergriff und um Katie legte. Sie flüsterte ihm leise etwas ins Ohr, worauf hin Ron sich doch noch entspannte und nun Katie festhielt.

Sie weinte minutenlang in Rons Schulter und machte seinen Umhang ganz nass. Harry rutschte näher an sie heran und legte jetzt ebenfalls einen Arm um sie.

Als Lavender die Treppe herunterkam und \enquote{Wird das ein Dreier?}, fragte, sahen sie alle mehr als böse an.

Hermine zog Lavender wütend nach oben und schimpfte unentwegt auf sie ein. \enquote{Du unsensibles Miststück. Du\abs} Doch mehr hörte man nicht mehr.

Als die beiden nach mehreren Minuten wieder herunterkamen, entschuldigte sich Lavender bei ihnen und umarmte Katie zusätzlich. Diese beruhigte sich und sah in Rons Augen. Sie nahm seine Hand in die ihre und gab ihm einen kurzen Kuss auf den Mund. \enquote{Danke}, hauchte sie ihm entgegen. Ron sah unsicher zu Hermine, die mit ihrem Fuß auf dem Boden wippte. Katie löste sich wieder von Ron und setzte sich in einen Stuhl ihnen gegenüber. Hermine nahm kurz auf Rons Schoß Platz und rutschte dann neben ihm auf den schmalen freien Platz zwischen ihm und Harry und sah mit gemischten Gefühlen zu Katie. Lavender hatte sich vor Katies Sessel niedergelassen und sah sie traurig an.

Dann zog sie einen Ärmel hoch und schüttelte den anderen zurück. \enquote{Madame Pomfrey konnte nichts mehr tun. Der Giftstachel war zu tief eingedrungen und meine Allergie gegen dieses Gift verschlimmerte es noch zusätzlich.}

Wieder flossen vereinzelte Tränen Katies Gesicht hinunter. Harry trocknete ihre Tränen mit einem Taschentuch. Dann gab er es Lavender, die näher an Katie saß.

Die nächsten Tage war die Stimmung durch Katies Verlust getrübt. Keiner hatte mehr Elan und war schlecht gelaunt.

\trenn

\enquote{Morgen werden wir uns dem Patronus-Zauber widmen. Schreiben Sie zunächst die ersten Absätze des Kapitels über Patroni ab.}

Die Klasse raschelte eifrig in ihren Taschen und nahm Buch und Pergament heraus. Lavender, die neben Katie saß, legte ihr Buch in die Mitte. Mit unsicherer Hand nahm Katie die Feder in ihrer linken Hand und versuchte den Abschnitt abzuschreiben.

Professor Elber ging durch die Reihen und beobachtete die Schüler. Als er an Katie vorbeiging, meinte er: \enquote{Seit wann schreiben sie mit der linken Hand?}

Harry konnte ihn kaum verstehen. Er stand sehr nah bei ihr und beugte sich zu ihr herunter. Harry konnte nur ein Rascheln hören.

\enquote{Deshalb Professor}, sagte sie. Sie musste ihm ihren Stumpf gezeigt haben.

\enquote{Bleiben Sie nach der Stunde noch hier. Ihre Freundin wird sicher ihre Tasche mitnehmen.} Dann ging er weiter durch die Reihen und setzte sich auf seinen Stuhl. Die Arme auf den Lehnen gelegt, wartete er, bis alle fertig waren.

\enquote{Welche Gestalt wird ihr Patronus wohl annehmen, wenn sie ihn heraufbeschwören? Falls er überhaupt eine Gestalt annimmt.}

Harry hob als Erstes seine Hand.

\enquote{Ja?}

\enquote{Ein Hirsch.}

\enquote{Sie nehmen an, dass\abs}

\enquote{Nein. Mein Patronus ist ein Hirsch.}

\enquote{Das werden wir dann sehen}, sagte er. \enquote{Weitere Meinungen?}

\enquote{Hund.}

\enquote{Otter.}

\enquote{Fuchs}, kam ihm entgegen.

\enquote{Gut, die Patroni werden wir dann alle morgen früh sehen. Kommen Sie bitte zu Hagrids Hütte. Wir werden morgen im Freien sein. Wir treffen uns dort und gehen dann zum Treffpunkt. Für heute ist die Stunde beendet. Gehen Sie jetzt Essen.}

Die Klasse packte zusammen. \enquote{Professor?}, fragte Hermine. \enquote{Was ist ihr Patronus?}

Professor Elber saß noch immer in seinem Stuhl. Nun verschränkte er seine Finger und legte die Daumen auf seine Lippen. Er sah jetzt genauso aus wie Dumbledore. \enquote{Lassen Sie sich überraschen.} Dann wandte er sich zu Katie. \enquote{Sie bleiben noch hier.}

Harry aß begeistert an einem Maiskolben herum, als Katie wieder in die Große Halle kam. \enquote{Und?}, fragte sie Lavender.

Katie ignorierte sie und legte ihre Ärmel auf den Tisch. Sie goss sich mit der linken Hand etwas Kürbissaft ein und verdünnte ihn mit Wasser. Dann nahm sie mit der Gabel einen Hühnchenschenkel und aß daran. Harry stellte gerade seinen Trinkbecher wieder ab und schluckte herunter, als Katie mit der anderen Hand nach ihrem Trinkkelch griff. Sie sah ihm in die Augen und trank dann daraus. Harry bekam große Augen. Zum Glück hatte er gerade heruntergeschluckt.

Harry sammelte sich kurz und biss beiläufig in seinen Maiskolben. \enquote{Was ist mit deiner Hand?}, fragte er so normal wie möglich.

Sie grinste ihn an. Dann sahen die anderen, dass Katies rechte Hand wieder vollkommen in Ordnung war Sie hatte nun wieder zwei Hände.

\enquote{Wie?}

\enquote{Was?}

\enquote{Woher?}, fragten die um sitzenden Katie plötzlich.

\enquote{Nachher, im Gemeinschaftsraum}, kam daraufhin von Katie.

Als sie die Große Halle verließen, hatte Katie wieder ihre Hände in den Ärmeln verborgen. Die gehässigen Sprüche einiger Slytherin überging sie. Sie würde nicht überreagieren und ihre zweite Hand herausholen, nur um es den Slytherin zu zeigen.

Im Gemeinschaftsraum angelangt, setzte sich Katie in einen Sessel. Tamara saß bereits auf einem Sofa, das gegenüber stand. Harry setzte sich auf den Boden und sah zu Katie auf. Er mochte sie seit sie zusammen Quidditch spielten. Dann fing Katie an zu erzählen.

\enquote{Nachdem ihr gegangen seid, nahm mich Professor Elber mit in den Krankenflügel. Er bat mich, mich auf einen Stuhl zu setzen, damit ich nachher meine Arme auf das Bett legen könnte. Dann ging er zu Madame Pomfrey in ihr Büro und kam kurz darauf mit einer Flasche Skele-Wachs und einer Salbe heraus. Madame Pomfrey folgte ihm. Dann legte er die beiden Sachen auf das Bett. Er bat mich, meine beiden Arme auf das Bett zu legen. Ich tat, was er mir sagte. Dann streifte er mir beide Ärmel hoch und besah sie sich. Ich fragte ihn, was er vorhabe. Er antwortete nur, dass er sich um mich kümmern würde und ich meine Hand zurückerhalten würde. Madame Pomfrey sah ihn ungläubig an und beschwerte sich, aber er unterbrach sie und bot ihr einen Stuhl an. Dann erklärte er ihr, dass es eigentlich nicht schwer sei, eine abgetrennte Hand wieder komplett wachsen zu lassen. Falls es sie interessieren würde, könne sie gern mitschreiben und ihn alles fragen.}

Alle, die Katie zuhörten, bekamen große Augen. Keiner wagte etwas zu fragen, bevor Katie nicht fertig mit ihrer Erzählung war. Harry fuhr leicht über ihre neue Hand. Sie zuckte, zog sie aber nicht zurück. Dann lauschten alle weiter gespannt.

\enquote{Dann gab er mir eine kleine Holzrolle und meinte, dass er jetzt leider, nachdem der Stumpf schon geheilt und nicht mehr offen war, vorne ganz knapp abschneiden müsse. Er würde zwar den Schmerz betäuben, aber es würde doch weh tun. Wenn er ihn ganz lindern würde, dann müsste ich mehrere Tage lang meine Hand in einer Art starre halten. Also stimmte ich zu und nahm die Holzrolle in den Mund. Madame Pomfrey sah mich mit feuchten Augen an. Dann sprach Professor Elber einen seltsamen Zauber. Ich glaube, er hieß: \enquote{Sectum sempra.} Meinen Stumpf schnitt er ganz vorne ab. Ich schrie in meinen Holzklotz, aber der Schmerz war nur von kurzer Dauer. Er nahm mir die Rolle wieder aus dem Mund und berührte mich dabei\abs Dann war mein Arm wieder offen und blutete. Die Blutung wurde sofort wieder gestoppt und ich musste einen Becher Skele-Wachs austrinken. Währenddessen sprach Professor Elber mit seinem Zauberstab auf meinen offenen Arm gerichtet Beschwörungsformeln. Ich habe noch nie zuvor gesehen, wie meine Knochen gewachsen sind. Schließlich waren alle vollständig. Dann sagte er mir, dass ich meine Hand nun ruhig halten müsse. Obwohl es kribbeln würde, müsse ich sie ganz still halten. Er strich mit seinem Zauberstab über meine Knochen und sagte Madame Pomfrey den passenden Spruch. Meine Sehnen bildeten sich an den Knochen entlang aus. Danach waren meine Muskeln dran. Und als letztes meine Hautschichten. Ich musste meine Hand noch eine Weile ruhig halten. Danach konnte ich sie bewegen. Er nahm meine neue Hand zwischen seine Finger und fuhr jeden einzelnen Finger ab. Jeden Zentimeter fuhr er mit seinen zarten Fingern ab. Als ich ihn fragte, warum er das machte, sagte er nur, dass er die Sensorik überprüfen müsse. Wenn eine Stelle taub sei, dann müsse ich ihm das sagen. Ich spüre jetzt noch die zarten Bewegungen und Berührungen auf meiner Haut. Dann konnte ich gehen.}

Katie fügte noch hinzu: \enquote{Als ich dann ging, hörte ich nur noch wie Madame Pomfrey fragte, warum das nicht bekannt sei. Er antwortete darauf hin nur: \enquote{Viele würden dazu schwarze Magie sagen, aber Magie hat keine Farbe.} Da fiel mir die erste Stunde ein, als du Hermine,} sie sah Hermine an \enquote{die beiden Pflanzen abbrennen musstest.}

\begin{rueckblick}
\enquote{Aber warum ist das schwarze Magie?}

\enquote{Wissen Sie, Poppy. Mit diesen Zaubern kann man die Kontrolle über die neu gewachsene Hand erlangen. Man muss beim Zaubern aufpassen, dass die Kontrolle beim Körper bleibt, bzw. ihm übertragen wird}, erklärte er ihr.
\end{rueckblick}

\enquote{War Madame Pomfrey sehr sauer?}, fragte Tamara.

\enquote{Eher entsetzt, dass man so etwas an einer Schülerin anwenden würde. Aber ich glaube, dass ihre Hingabe an diesen Beruf sie darüber hinwegsehen lässt und sie das in Zukunft auch anwenden wird.} Sie grinste. \enquote{Ich muss jetzt noch Hausaufgaben machen}, sagte Katie und ging nach oben, um ihre Tasche zu holen. Als sie wieder kam, setzte sie sich an einen Tisch und fing an.

Harry richtete sich nach einer Weile, in der er über seinen Lehrer nachdachte, fürs Bett her. Er legte die Hände hinter seinen Kopf und dachte nach. Er stellte sich die Szenen im Krankenflügel bildlich vor und in seinem Geist bildete sich der Ablauf. Er dachte noch an die Stunde, in der sie Patroni üben würden. \gedanke{Die DA dürfte damit kein Problem haben.} Dann schlief er müde ein.

\trenn

\enquote{Einen wunderschönen guten Morgen}, sprach Professor Elber die Klasse an. \enquote{Schön, dass sie es einrichten konnten, heute hier zu sein.} Die gesamte Klasse war anwesend. Gryffindor ebenso wie Slytherin. \enquote{Ich sagte ihnen bereits gestern, dass wir uns heute um den Patronus-Zauber kümmern werden. Sie kennen bereits den entsprechenden Zauber und die Bewegung des Zauberstabes. Sie haben sie ja gestern abgeschrieben. Nun folgen Sie mir bitte, wir gehen ein Stück Richtung Wald.} Er drehte sich um und die Klasse folgte ihm.

Hagrid, der gerade seine Hütte verlassen hatte, winkte Harry, Ron und Hermine zu. Die drei winkten zurück.

Kurz vor einem Waldstück blieb Professor Elber stehen. Es wurde bereits merklich kühler. Er zog seinen Zauberstab und sprach: \zauber{Expecto Patronum.} Eine dicke Nebelschwade breitete sich aus. Sie verdichtete sich zu einer kleinen Wolke und zog dann durch die Schüler hinweg.

Harry konnte vereinzeltes Kichern und Getuschel hören.

\enquote{Will uns einen Patronus beibringen\abs}

\enquote{Kann keinen gestaltlichen\abs}

Harry hatte das Gefühl, das da noch mehr kommen würde. Sobald er seinen Gedanken beendet hatte, wandelte sich der Nebel. Er bildete viele kleine Kügelchen aus. Die Kügelchen formten sich zu Bienen, Hummeln und Wespen, sowie Hornissen. Jetzt flogen viele kleine Patronus-Insekten um die Schüler herum und die aufsteigende Wärme, die der Nebel bereits brachte, verstärkte sich.

Sie folgten weiter ihrem Professor in den Wald. Dann sah Harry fünf Gestalten. Schwebende Figuren, die mit Stoffen behangen waren.

\enquote{Dementoren}, hörte er.

Er konnte es nicht fassen. Vor der Klasse schwebten tatsächlich fünf Dementoren.

Professor Elber drehte sich um und begann nun. \enquote{Diejenigen von ihnen, die gestern bereits ihren Patronus genannt hatten, treten zwei Schritte vor.}

Die gesamte DA trat jetzt aus der Menge hervor.

\enquote{Diejenigen unter ihnen, die schon einmal einen Patronus erzeugt haben, treten abermals einen Schritt vor.}

Wieder trat die gesamte DA einen Schritt vor.

Professor Elber zeigte sich erstaunt. \enquote{Ok, dann treten Sie bitte einzeln vor, aus dem schützenden Patronus-Feld heraus und stellt euch einem Dementor.}

Wie auf ein unsichtbares Zeichen arrangierten sich die Dementoren um und ein einzelner schwebte einige Meter vor, um den ersten Schüler zu erwarten.

\enquote{Na los Harry}, hörte er seine Mitschüler sagen.

Harry trat vor und ließ seinen Patronus erscheinen. \zauber{Expecto Patronum.} Der Patronus stellte sich dem Dementor in den Weg, der begeistert die Energie aufnahm. Als der Dementor sich wieder zurückzog, gab Professor Elber Harry ein Zeichen, der daraufhin seinen Patronus verschwinden ließ. Nun waren die anderen der DA dran. Alle absolvierten sie ihren Patronus mehr oder weniger überzeugend. Neville brachte ihn erst beim dritten Mal heraus. Und zum ersten mal sah Harry Nevilles Geier-Patronus.

Da bereits alle Gryffindors ihre Patroni zeigten und somit fertig waren, kamen jetzt die Slytherins an die Reihe. Harry grinste in sich hinein, als die Slytherins ihre Patroni aufzeigen sollten, doch er staunte, dass zumindest Blaise Zabini und Pansy Parkinson  ein Nebelschwaden-ähnliches Etwas zum Vorschein brachten, aber der Rest versagte kläglich beim ersten Mal.

\enquote{Pansy, Blaise, sie können sich zu den anderen stellen. Für heute haben sie ihr Soll mehr als erfüllt. Der Rest muss noch etwas üben.}

So verlief der Rest der Stunde. Die restlichen Slytherin versuchten sich an ihren Patroni und brachten zumindest am Ende der Stunde etwas zustande, was man als Nebel durchgehen lassen würde.

\enquote{Diejenigen unter ihnen, die bisher nichts, oder nur einen schwachen nebelartigen Patronus zustande gebracht haben, werden die nächste Zeit hier noch üben}, sagte Professor Elber.

Zehn Minuten vor Ende der Stunde meinte Professor Elber: \enquote{Die fünf Dementoren werden diesen Samstag zwischen Acht und Acht Uhr zum Üben bereitstehen. Also von morgens bis abends.} Er gab den Dementoren einen Fingerzeig und diese entfernten sich ein paar Meter. \enquote{Sie dürfen zurückgehen, damit sie für die nächste Stunde nicht zu spät kommen.}

\enquote{Professor?}, fragte Ron ihn, als ein großer Teil der Schüler schon gegangen war. \enquote{Werden wir auch lernen mit Patroni Nachrichten zu übermitteln?}

\enquote{Wenn sie möchten, dann machen wir das.}

Die restliche Klasse nickte.

\enquote{Also gut, aber nicht beim nächsten Mal. Wir werden das zu einem späteren Zeitpunkt durchnehmen. In diesem Fall werden sie allerdings alle diesen Samstag nochmals zum Üben kommen, mit echten Dementoren. Sie werden auf ihre Parallelklasse treffen. Also Hufflepuff und Ravenclaw. Sie sollten alle den Stand des gestaltlichen Patronus erreichen.}

Die Klasse verabschiedete sich und Professor Elber drehte sich um. Mit seinem Zauberstab zog er ein einzelnes Insekt an und löste den Rest auf. Das Insekt landete auf seinem Zauberstab und er flüsterte ihm etwas zu. Dann vervielfältigte es sich und je eines der Kreaturen flog auf die mundförmige Öffnung der Dementoren zu. Diese sogen sie auf und verschwanden danach im Inneren des Waldes.

Harry drehte sich wieder um und lief schnellen Schrittes seinen Freunden hinterher.

\enquote{Der wird mir so langsam unheimlich}, meinte Hermine, als Harry sie eingeholt hatte.

Harry und Ron nickten nur stumm.

\enquote{Ich meine, er ist ein guter Lehrer, soweit man das bisher sagen kann, aber trotzdem. Der steht da einfach mit Dementoren und referiert und lässt uns Patroni herauf beschwören. Und die tun auch noch genau das, was er ihnen sagt. Oder besser gesagt, das was er will. Denn er hat mit ihnen ja nicht wirklich geredet}, fuhr sie fort.

\enquote{Aber er hat mit ihnen kommuniziert. Als ihr schon weg wart, da hat er ein einzelnes Insekt auf seinen Zauberstab geholt und ihm etwas zugeflüstert. Dann hat es sich vermehrt und jeder der Dementoren hat eines aufgesogen. Dann sind sie davon geflogen}, meinte Harry.

\enquote{Guten Morgen}, kam es den Schülern von Professor Sprout entgegen. \enquote{Wir werden uns heute um fleischfressende Pflanzen kümmern, die dazu noch giftig für Menschen sind. Einige der Pflanzen, die sie die nächsten vier Unterrichtseinheiten kennenlernen, mögen sie vermutlich für intelligent halten. Sind sie aber nicht. Sie haben heute ausnahmsweise mit den Fünftklässlern Ravenclaw und nicht wie üblich mit den Slytherin zusammen diese Stunde.}

Harry stand neben Luna und lächelte sie an. \enquote{Müsst ihr diesen Samstag auch Patroni üben?}, flüsterte er Luna zu.

\enquote{Nein}, flüsterte sie zurück, während sie mit dem anderen Ohr Professor Sprout zuhörte.

\enquote{Was unterscheidet die drei Arten, Miss Lovegood?}, fragte Professor Sprout.

\enquote{Vier Arten, Professor}, antwortete Luna. \enquote{Sie sehen recht ähnlich aus, riechen aber grundverschieden. Leider ist es extrem schwer, an sie heranzukommen, da sie nach jedem greifen, der sich ihnen nähert.}

Professor Sprout hob eine Augenbraue. \enquote{Fünf Punkte für Ravenclaw. \gst Mister Potter, wie riechen die fünf Arten?}

\enquote{Wie Miss Lovegood soeben sagte, sind es vier Arten. Zwei riechen nach verwesendem Fleisch, um Fliegen anzuziehen. Sie kann man durch die Intensität unterscheiden. Eine Art riecht süßlich und hat zudem ultraviolette Lichtabstrahlung und die vierte Art riecht für uns Menschen gar nicht}, antwortete Harry.

\enquote{Sehr gut. Ebenfalls fünf Punkte für Gryffindor.} Harry war froh, dass er heute Morgen noch kurz in sein Buch geschaut hatte, als er nochmals auf Hermine warten musste. \gedanke{Professor Sprout dachte wohl, wir würden nicht aufpassen}, grinste er in sich hinein und lies Luna an seinen Gedanken teilhaben.

Den Rest der Stunde verbrachten sie damit, den Pflanzen auszuweichen und sie mit ihren Drachen"-haut-Hand"-schuhen umzutopfen. Luna war scheinbar die Einzige, die von ihrer Pflanze nur halbherzig angegriffen wurde. \enquote{Du musst mit ihnen reden}, sagte sie, als sie Harrys fragende Blicke spürte.




\begin{kommentar}
Als Elber Katies Hand wieder nachwachsen ließ, musste er das bei offener Wunde tun. Den Armstumpf hatte er dazu mit dem Sectumsempra-Spruch abgeschnitten, den Snape entwickelt hatte und von dem Harry im sechsten Band der Original-Reihe erfahren hatte. Ich fand es eine passende Gelegenheit, hier ein kleines Detail aus dem Original einzuflechten.
\end{kommentar}

\begin{kommentar}
Elber erzeugt als seinen Patronus viele kleine Insekten. Die Idee kam mir beim Lesen einer anderen Geschichte. Sie zeigt auch sehr schön, dass es nicht nur eine Figur sein muss und deutet schon an, was Elber später zu Harry sagen wird. »Übe mit deinem Patronus.« Er impliziert hierbei, dass man die Gestalt des Patronus doch ändern kann. Man muss nur genug wollen.
\end{kommentar}
