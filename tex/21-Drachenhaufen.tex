\chapter{Drachenhaufen}


Charlie schlich sich mit seinen Kollegen an die Grenzen des Todesser-Lagers heran. Im Schutze von Büschen und Bäumen warteten sie, bis Tabaluga und Frederick ihren ausgemachten Überraschungsangriff starteten. Sie wollten direkt in das Lager apparieren und somit den Schild zum Einsturz bringen. Dann kamen Charlie und seine Kollegen dran. Für die Todesser im Inneren des schützenden Feldes sah das so aus, sollten sie zusehen, dass sie sich an einen Drachen heranschlichen. Denn ein Drache hatte sich in die Nähe gelegt und war somit ein prima Alibi.

Zeitgenau apparierte Frederick mit Tabalugas Hilfe in das Innere der schützenden Magie-Kuppel, die mit einem starken Desillusionierungszauber, Verzerrungszaubern und Aufspür-Verhinderungszaubern ausgestattet war. Die beiden Todesser konnten gar nicht so schnell reagieren, wie der Drache ihnen Feuer entgegen spie, um seinen Begleiter zu schützen und die Todesser abzulenken. Frederick lies die Zauber brechen und die Wildhüter griffen von hinten an. Nach knappen zwanzig Sekunden war das Gefecht auch schon zu Ende. Gegen die Übermacht hatten sie keine Chance.

Als sie gefesselt am Boden saßen, ihre Zauberstäbe wurden ihnen abgenommen und ein Anti-Apparierfeld errichtet, fragte einer der Todesser \gst besser gesagt er schrie es fast heraus: \enquote{Wie habt ihr es durch die Abschirmung geschafft? Sie war perfekt.}

Charlie wollte schon antworten, doch Frederick war schneller. \enquote{Wissen Sie, es ist eine Sache einen Zauber zu wirken, wenn man ihn kennt. Aber eine ganz andere ist es, die Magie zu beherrschen und die Zauber unwirksam zu machen, weil man ihnen die Grundlage dessen entzieht, was sie antreibt.}

\enquote{Schöne Worte ohne Inhalt}, maulte er zurück und er zerrte an seinen Fesseln.

Frederick schnippt einmal mit seinen Fingern und der Mund des Todessers verschwand. Seine Stimmbänder brachten keinen Ton mehr heraus.

\enquote{Es ist leicht, sich zu beschweren, wenn man es nicht besser weiß, wenn einem die Fantasie fehlt.}

Dann machte er sich daran mit Tabaluga zum Zelt der Wildhüter zurückzugehen. Zuvor jedoch nahm er den schwarzen matt-schimmernden Brocken mit, legte ihn in das hölzerne Kästchen und nickte Tabaluga erleichtert zu.

\enquote{Was wird jetzt mit dem stimmlosen Todesser?}, fragte Randolf, einer der Wildhüter.

\enquote{Was soll mit ihm sein?}

\enquote{Wird er seine Stimme für immer los sein?}

\enquote{Nein, die Zauber von Askaban lösen die Blockade. Falls es für seine Verhandlung notwendig sein sollte, wird er dort auch sprechen können. Halt, wir sind ja in Rumänien. \gst Die im hiesigen Gefängnis wirkenden Zauber sollten das auch hinbekommen.}

Zurück im Zelt holte er unter den wachsamen Augen von Tabaluga den Stein wieder heraus. Er passte in eine Hand und sah unförmig aus.

\enquote{Ich brauche deine Hilfe}, sagte er. \enquote{Du musst mich berühren, mit einer Pranke auf nacktem Oberkörper. Je mehr Hautkontakt, desto besser.} Während dessen machte er sich oben herum frei. \enquote{Mit der andern Pranke, verletzt du meine Hand mit einer Kralle. Das Blut muss über den Stein fließen, während ein Zauber gesprochen wird. Das wird den Drachen hier helfen. Als Nächstes muss dein Blut über den Stein fließen, während du einen Zauber wirkst. Keine Angst, durch den Hautkontakt wird es dir möglich sein. Damit werden wir feststellen, ob noch andere Drachen betroffen sind. Sollte das sein, dann müssen wir einen stärkeren Zauber wirken. Anderenfalls werden wir den Stein zerstören.}

\enquote{Wird der Stein auf jeden Fall zerstört?}, fragte Tabaluga nach.

\enquote{Ja, auch wenn wir vorher noch einer anderen Kolonie helfen müssen, falls es notwendig wird.}

Tabaluga nickte und legte eine seiner Pranken auf den Rücken von Frederick. Die Innenseite seiner Pranke war ganz leicht pelzig und warm. Sofort spürten beide die Verbindung zueinander, die durch die Magie ausgelöst wurde, welche beide verband. Dann folgte der komplizierte Zauber. Zuerst wurde die Herde geheilt, dann eine Prüfung ausgesandt, um die restlichen Drachen zu finden, die noch betroffen sein sollten.

Ein einzelnes Echo kam zurück. Es kam von einem sehr schwachen Drachen. Dann erfolgte zwischen beiden eine Kommunikation, die keiner Worte bedarf. Sie verschwanden und tauchen in einer kalten Höhle auf; dort, wo der Drache lag, schon halb tot und nicht mehr in der Lage, ein Feuer zu speien, obwohl er es versuchte, um die Eindringliche zu verscheuchen.

Davon unbeeindruckt führten die beiden den Zauber durch und der Drache wurde von dem Fluch befreit. Erneut schickten sie eine Welle los. Dieses Mal kam kein Echo zurück. Deshalb zerstörten sie den Stein.

Vorsichtig ging Tabaluga auf den alten Drachen zu.

\enquote{Lasst mich allein. Ich sterbe doch schon. Lasst mir meine Ruhe und quält mich nicht länger.}

\enquote{Anscheinend ist der Drache nicht nur blind, sondern hat ein Defizit im Geruchssinn.}

\enquote{Wer seit ihr?}, fragte er jetzt.

\enquote{Tabaluga und Frederick}, antwortete er.

\enquote{Zauberergesocks}, gab er abwertend zurück.

\enquote{Ich verbitte mir diese Bezeichnung}, gab Tabaluga beleidigend zurück. Tabaluga sprach plötzlich. Irgendwie hatte er die Fähigkeit zu sprechen gewonnen, doch der Drache war sich der Tatsache noch nicht bewusst.

\enquote{Oh, ein Drache. Behandelt er dich gut?}, antwortete der alte Drache. Auch er sprach.

\enquote{Bisher ja, aber wir kennen uns erst seit gestern Mittag.}

\enquote{Und dann zwingt er dich schon dazu, ihn hier herzuführen, damit er mich quälen kann?}

\enquote{Nein.}

\enquote{Warte. Er muss schon länger von jemand gequält worden sein. Seine Haut hat mehrere Risse und seine Gesundheit ist sehr angeschlagen. Kaum Geruchssinn. Kein Augenlicht, Probleme mit den Lungen und den Nieren. Seine Milz arbeitet kaum noch. Seine Flugmuskeln sind sehr verkümmert.}

\enquote{Waren das die beiden Todesser? Haben Sie hier ihr erstes Opfer gefunden, um den Stein zu testen?}, fragte Tabaluga nach.

\enquote{Ja}, antwortete der Drache. \enquote{Zwei tote Männer. Zauberer. In Schwarz und mit Maske. Sie hatten den Drachenstein dabei.}

\enquote{Wie sieht er aus?}, fragte Tabaluga.

\enquote{Sah}, meinte Frederick.

\enquote{Etwa so groß, dass er in eine menschliche Hand passt. Schwarz und unförmig.}

Tabaluga sah Frederick an.

\enquote{Ich sagte doch: \inner{Sah.}}

\enquote{Das war der Drachenstein?}, fragte Tabaluga nach.

\enquote{Ja.}

\enquote{Der Schrecken aller Drachen. Ein Teufelszeug}, gab der alte Drache matt zurück.

Frederick nickte nur. \enquote{Leider}, gab er mit gesenktem Kopf zurück.

\enquote{Gibt es da mehr?} Erneutes nicken. \enquote{Was?}

\enquote{Das erzähle ich dir später. Ich hänge an meinem Leben. Jetzt werden wir erst einmal\abs Wie ist dein Name?}

\enquote{Tabaluga, aber das weißt du doch.}

\enquote{Ich meinte den anderen Drachen.}

\enquote{Nepomuk}, gab dieser zurück.

\enquote{Dann werden wir Nepomuk erst einmal helfen.}

\enquote{Wie?}

\enquote{Wir nehmen dich in ein Drachenreservat mit. Dort kümmern sich nette Zauberer, Tabaluga kann dir mehr davon erzählen, um dich\abs} Er berührte Tabaluga kurz, worauf hin dieser wusste, was und wie er es zu tun hatte. \enquote{Während die Wildhüter sich um deine körperliche Verfassung kümmern, Nahrung, Trinken und den Muskelaufbau, werde ich versuchen, dir das Augenlicht zurückzugeben.}

\enquote{Das geht?}, fragte Nepomuk mit ängstlichen und ungläubigem Unterton zurück.

\enquote{Ich versuche es. Ich weiß um die Wirkungen des Drachensteins. Daher bin ich guter Dinge.}

\enquote{Und mein Feuer?}

\enquote{Das weiß ich nicht. Es ist möglich, dass ich es nicht schaffe, dass es von allein wieder kommt, oder, dass es keine Möglichkeit gibt, es wiederzuerlangen.}

Nepomuk schnaufte einmal durch. \enquote{Ich habe keine Möglichkeit, eure Aussagen zu prüfen.}

\enquote{Wenn ich fies wäre, dann würde ich sagen, dass du eh keine Chance hast. Wenn ich es wollte, dann hättest du keine Chance gegen mich. Nicht in diesem Zustand, in dem du bist.}

\enquote{Gib mir ein paar Minuten zum Nachdenken.}

\enquote{In Ordnung}, gab Frederick zurück und wartete.

Dann sagte der Drache schließlich zu und zu dritt verschwanden sie aus der Höhle und tauchten vor dem Zelt auf. Frederick zog sich sein Hemd und seinen Umhang wieder an, als die Wildhüter zu ihrem Zelt zurückkamen.

\enquote{Was ist mit ihm denn los?}, fragten sie.

Nepomuk wurde aufgeregt und richtete sich auf. Er ging einige Schritte zurück.

\enquote{Ganz ruhig. Das gilt für alle. Bleibt stehen}, sagte Tabaluga.

\enquote{Du kannst sprechen?}, fragte Charlie.

\enquote{Ja. Seit ein paar Stunden.}

Langsam beruhigte sich Nepomuk.

\enquote{Er heißt Nepomuk und wurde vermutlich von den beiden Todessern misshandelt und gequält. Denen, die ihr geschnappt hattet. Geht vorsichtig und ruhig mit Nepomuk um. Er ist sehr schwach. Er braucht Nahrung und was zum Trinken. Seine Flugmuskeln sind verkümmert. Er hat kein Augenlicht und seine Geruchszellen funktionieren auch kaum.}

Charlie und seine Kollegen nickten und gingen behutsam mit dem Drachen um. Zitternd und ängstlich, aber dennoch geduldig, lies er alles mit sich machen. Immer bereit aufzuspringen und in Panik zu verschwinden. Währenddessen kümmerte sich Frederick um seine anderen Verletzungen. Sorgfältig begann er Augen und Nase zu untersuchen.

Zeitgleich knüpfte Tabaluga zu Nepomuk ein Band. Er erzählte ihm in Wort und Bild direkt in den Geist des Drachen von seinen Erlebnissen mit anderen Drachen und den Menschen. Nepomuk beruhigte sich zunehmend. Dann fing er an auf einem Auge hell und dunkel zu erkennen. Dann folgte das andere.

\enquote{Für jetzt ist es genug}, sagte Frederick. \enquote{Das muss erst einmal eine Stunde ruhen. Dann können wir weiter machen.} Dann kam die Nase dran.

\enquote{Ihr Menschen riecht noch immer sehr streng}, sagte Nepomuk. Frederick musste über diese Aussage lachen. \enquote{Was lachst du? Ich habe dich eben beleidigt.}

\enquote{Hast du nicht. Ich weiß, dass wir für euch streng und eigenartig riechen. Außerdem hast du mir bestätigt, dass mein Zauber gewirkt hat und dir einen Teil deiner Riechnerven wieder gegeben hat. Das war ein schweres Stück Arbeit.} Er schnaufte schwer. \enquote{Ich brauche einen Trank für Nepomuk. Seine Milz ist stark angegriffen. Leber und Nieren erholen sich von allein wieder.}

Einer der Wildhüter nickte und verschwand. Nepomuk konnte die Bewegung sehen. \enquote{Da läuft was}, sagte er.

\enquote{Einer meiner Kollegen}, antwortete Arthur, einer der Pfleger.

Nepomuk senkte seinen Kopf wieder und legte ihn auf das weiche Gras. Kurz darauf spürte er Arme auf seiner Schnauze.

Er hörte Tabaluga in seinem Geiste sagen: \stimme{Das muss ihn schwer erschöpft haben. Er ist auf deiner Nase eingeschlafen und liegt auf seinen Händen.}

Nepomuk brachte nur noch ein Lächeln zustande und döste danach etwas. Als der Trank für seine Milz ankam, öffnete er schläfrig sein Maul etwas, lies die Flüssigkeit hineinlaufen, schloss ihn wieder und schluckte den Trank. Dann schlief er weiter.

Als er wieder erwachte und seine Augen öffnete, sah er bereits Umrisse und konnte Konturen von Gestalten erkennen. Er konnte Mensch und Drache unterscheiden.

\enquote{Meine Augen sind besser geworden}, erklärte er.

\enquote{Es sind auch zwei Stunden vergangen}, sagte Tabaluga.

\enquote{Dann hast du eine weitere Behandlung vorgenommen?}, fragt er. In der Hoffnung, Frederick sei in der Nähe, da er die Konturen nicht wirklich auseinander halten konnte. Zumindest sahen alle Menschen für ihn gleich aus.

\enquote{Ja}, antwortete dieser und begann mit der nächsten.

Der nächste Trank wurde ihm gereicht und auch diesen schluckte er, ohne sich zu beschweren. Seine Augen begannen etwas zu tränen und leicht zu kribbeln. Er schloss sie.

\enquote{Es kribbelt leicht, nicht wahr?}, fragte Frederick nach. Nepomuk nickte mit seinem Kopf. \enquote{Das ist ein sehr gutes Zeichen.}

\enquote{Du bist auf mir eingeschlafen}, gab er amüsiert zurück.

\enquote{Du bist auch sehr weich}, antwortete Frederick.

Wieder glitt ein Lächeln über das Gesicht des Drachen. Nach einer weiteren Stunde, es war bereits Mittag, hatte er sein Augenlicht zurückerhalten. Glücklich fuhr er Frederick mit seiner Zunge über das Gesicht und wunderte sich, warum dieser nicht nass war. \gedanke{Automatischer Säuberungszauber}, dachte er. \gedanke{Er wird nass, aber das trocknet sehr schnell wieder.}

\enquote{Deine inneren Organe brauchen noch ein paar Tage. Deine Augen schonst du bitte noch die nächste Zeit. Einer der Wildhüter wird sie noch kontrollieren.}

\enquote{Und meine Nase?}

\enquote{Ich bin noch geschwächt. Ich glaube, heute nicht mehr. Eventuell auch morgen nicht.}

\enquote{Und wenn ich dir helfe?}, fragte Tabaluga nach.

Frederick sah ihn an. \enquote{Das kostet eine Menge Kraft. Du wirst dich die nächsten Tage dann nicht gegen Zauber zur Wehr setzen können.}

Tabaluga dachte nach und sah beide abwechselnd an. \enquote{Ich glaube, dann schaffe ich das nicht.}

\enquote{Das denke ich auch. Zumal ich nicht allein zurück apparieren kann. Ich brauche jemand, der mich begleitet. Ich muss unterrichten. Ich habe nicht mehr die Kraft, allein zu apparieren.}

\enquote{Du wirkst aber noch immer fit.}

\enquote{Es ist auch keine körperliche Schwäche. Es ist eine Schwäche der Magie.}

\enquote{Darf ich dann die paar Tage bei dir bleiben?}

\enquote{Nicht im Schloss. Aber bei unserem Wildhüter.}

\stimme{Wildhüter? Hogwarts? Arbeitet Hagrid immer noch dort?}, hörten beide eine Stimme von hinten.

Ein Drachenweibchen näherte sich ihnen. Sie war größer als Tabaluga, aber immer noch ein Teenager. Sie war rot und für einen Drachen hübsch. Zumindest aus Menschensicht.

\enquote{Ja}, antwortete Frederick schlicht.

\stimme{Darf ich mitkommen? Ich würde ihn gerne mal sehen.}

\gedanke{Das geht leider nicht. Ihr steht noch unter Quarantäne. Aber ich kann dir versprechen, dass du ihn bald sehen wirst. Ich schätze mal in spätestens zwei Jahren.}

Norberta, das Drachenweibchen, begann zu lächeln und ihre Augen leuchteten. \stimme{Schade, dass es jetzt nicht geht. Aber wir Drachen sind sehr geduldig.}

\gedanke{Geduldig? Ich habe da anderes gehört. Beim trimagischen Turnier waren deine Freunde anderer Meinung. Sie verteidigten ihre Eier sehr schnell. Und das, obwohl keine Brutzeit war.}

\stimme{Du scheinst viel über uns zu wissen.} Frederick nickte. \stimme{Aber es stimmt. Es wurden die wilderen von uns geschickt. Und sie wurden angewiesen, nicht zu geduldig zu sein}, sagte sie und grinste.

Frederick begann zu gähnen.

\enquote{Gehen wir?}, fragte Tabaluga.

\enquote{Gehen? Die Zeit haben wir nicht. Aber wir können apparieren.}

Tabaluga und Frederick berührten sich und apparierten. Leider nur ein paar Meter. Frederick fiel vor Müdigkeit fast um. Norberta kam auf beide zu und nahm Frederick vorsichtig hoch, um ihn auf Tabaluga abzusetzen. Doch auch der zweite Versuch klappte nicht. Frederick konnte sich kaum noch auf Tabaluga halten. Deswegen fixierte man ihn.

\stimme{Gibt es dort andere Drachen?}, fragte Norberta?

\gedanke{Nein.}

\stimme{Dann werde ich euch absetzen. Drachen können ihre Magie durch Zauberer nutzen.}

\gedanke{Das stimmt. Mir fehlt die Kraft. Leitet mich}, sagte Frederick matt und schlief schon wieder fast ein.

Norberta schüttelte ihn etwas, damit er wieder ein paar Sekunden wach blieb und sie verschwanden. Keine Sekunde später tauchte sie wieder auf. \stimme{Ich habe die beiden abgesetzt. Sie werden in zwei Tagen wieder kommen.}

\stimme{Dann kann ich endlich wieder richtig riechen.}

\stimme{Komm, ich zeige dir deinen Schlafplatz. Und später dann deine Herde, falls sie dich akzeptieren.}

Nepomuk nickte und folgte Norberta, dann begab er sich auf einem Strohbett zur Ruhe. Nepomuk war einer der wenigen Drachen, die umgänglich waren. Neben Tabaluga und Norberta und ein paar wenigen anderen, die friedlich waren, war der Rest eher wild. Zwar geduldig, aber dennoch wild.

Die drei tauchten außerhalb des Schulgeländes auf und Norberta verschwand kurz darauf gleich wieder.

Tabaluga blieb mit Frederick zurück und lief Richtung Schloss. Dieses erkannte die beiden und öffnete automatisch die Tore. Gelbe Leuchtpunkte erschienen auf dem Boden und zeigten dem jungen Drachen, wohin er musste. Vor dem Wandteppich wartete er, da dieser gerade nach oben wegrollte. Dann trat er mit seiner Fracht auf dem Rücken in den Bereich hinein. Die Punkte waren nun grün. Vor Fredericks Tür schaute der Drache auf das sich aus schimmernder Magie bildende Gesicht. Es stutzte, schaute den Drachen an und schien ihn zu durchleuchten. Dann nickte es und verschwand. Die Tür öffnete sich mit einem \geraeusch{Klick}.  Dann trat der Drache ein und ging auf die leuchtende Tür zu. Diese öffnete sich ebenfalls. Jetzt wurde Frederick von einer unsichtbaren Macht angehoben, bis auf die Unterhose entkleidet und schwebte in sein Zimmer. Dort hob sich die Decke an, er wurde abgelassen und die Decke legte sich über ihn. Tabaluga wusste nicht genau, was er jetzt tun sollte.

\stimme{Lege dich zu ihm. Leiste ihm in seinem Zimmer Gesellschaft}, hörte der Drache in seinem Geiste.

Tabaluga nickte, betrat den Raum und legte sich auf den weichen Teppich neben dem Bett, nachdem er ihn etwas von ihm weg gezogen hatte. Er legte sich so, dass er Frederick beobachten konnte, und schloss seine Augen um zu schlafen.

Morgens um vier Uhr erwachte Frederick. Er konnte im Dämmerlicht des frühen Morgens zwar wenig, aber dennoch genug sehen, um sich orientieren zu können. Doch er schien noch verwirrt. Er stand auf, zog seine Pantoffeln an und setze sich auf den Stuhl in seinem Zimmer. Die Hände zwischen den Beinen und umherblickend, sah er aus, als wäre es ihm unangenehm, in einem ihm fremden Zimmer zu sitzen. Leicht verstört sah er sich um und entdeckte den Drachen, der noch friedlich auf den Teppich schlief. Frederick legte seinen Kopf schief und schien nachzudenken, woher er ihn kennen könnte. Er sah ihm eine ganze Weile zu, bis um fünf Uhr Tabaluga erwachte.

Das erste, was der Drache sah, war ein leeres Bett. Er richtete seinen Kopf auf und blickte durch den Raum. Dann sah er Frederick. \enquote{Guten Morgen}, sagte der Drache.

\enquote{Guten Morgen, Tabaluga. Was ist gestern noch alles passiert? Ich weiß noch, dass Nepomuks Augen geheilt wurden, aber danach ist bei mir Feierabend.}

\enquote{Feierabend? Es ist Morgens und fünf Uhr zwei.}

Frederick schmunzelte. \enquote{Das ist eine Redewendung der Menschen. Sie bringen damit zum Ausdruck, dass danach nichts mehr kommt. Das heißt, ich kann mich an nichts mehr, was danach passierte, erinnern. Andererseits ist es ein Ausdruck dafür, dass man mit der täglichen Arbeit aufhört und die Freizeit beginnt.}

Tabaluga nickte. Wieder hatte er etwas gelernt.

\enquote{Frühstück?}, fragte Frederick ihn.

\enquote{Gerne.}

\enquote{Was essen Drachen denn so? Rind? Hase? Schwein? Gemüse?}

\enquote{Bevorzugt Fleisch. Aber auch Gemüse und Obst. \gst Wegen der Vitamine}, fügte er noch hinzu.

\enquote{Malcomin? Kommst du?}

Ein Elf erschien im Raum und staunte, als er den Drachen sah.

\enquote{Wir hätten gerne Frühstück. Für einen Menschen und einen Drachen. Ausgewogen bitte.} Der Elf nickte mit dem Kopf und verschwand. \enquote{Dann lass uns mal ins Wohn- und Speisezimmer gehen.} Frederick ging voraus und der Drache folgte ihm.

Als beide saßen, tauchte wieder der Elf auf und belegte den Tisch mit verschiedensten Früchten, Broten und Fleischstücken. Dann setzte er sich dazu und begann sich auf seinen Teller Früchte zu laden. Frederick tat ebenfalls Früchte auf seinen Teller. Ebenso eine Semmel, sowie Butter und Marmelade darauf, nachdem er ihn aufgeschnitten hatte. Tabaluga nahm sich einen Brocken Fleisch und schluckte die Brocken, die er abbiss ganz hinunter. Danach kamen noch einige Früchte dazu, sowie ein Apfel als Abschluss.

\enquote{Was machst du heute?}, fragte ihn Tabaluga.

\enquote{Den Vormittag habe ich immer frei, zumindest montags. Am Nachmittag habe ich die erste und die dritte Klasse in \VgddK. Wie wäre es, wenn wir zusammen zu Hagrid gehen? Hogwarts’ Wildhüter und Lehrer im Fach \fach{Pflege magischer Geschöpfe}. Vielleicht nutzt er die Gunst und die Schüler bekommen etwas über Drachen zu hören.}

\enquote{Er will mich als Anschauungsobjekt verwenden?}, fragte Tabaluga verwundert nach. \enquote{Warum nicht. Dann erfahre ich, wie die Menschen uns sehen und kann das meiner Herde weitergeben.}

\enquote{Pass aber auf. Hagrid ist etwas eigen, was den Umgang mit anderen Spezies betrifft.}

\enquote{Inwiefern eigen?}, fragte er und schob noch einen ganzen Hähnchenflügel nach.

\enquote{Na ja. Er wollte schon immer einen Drachen haben.}

\enquote{Norberta.}

Frederick nickte. \enquote{Erzähl ihm bitte noch nichts davon. Du wirst es später verstehen. Aber nun zurück zu Hagrid. Er sieht \gst verzeih mir \gst Drachen und andere Tiere nicht als gefährlich an. Aber seine Umgebung tut das. Also alle anderen Menschen.}

\enquote{Deswegen sind sie manchmal grob.}

\enquote{Ihr seid ja auch wilde Gesellen.}

\enquote{Ich nicht}, empört sich Tabaluga schmunzelnd.

\enquote{Dich würde ich auch jederzeit streicheln.}

\enquote{Vor der Klasse?}

\enquote{Mal sehen}, scherzte Frederick. \enquote{Soll ich dich noch etwas herumführen? Wir haben ja noch Zeit.}

\enquote{Gerne.}

Die beiden machten sich auf den Weg, während der Hauself den Frühstückstisch abräumte, nachdem er selber fertig war. Zuerst ging es in den Krankenflügel. Madame Pomfrey war bereits wach und war überrascht, die beiden zu sehen. Nach einer kurzen Untersuchung des Drachens durch die Krankenschwester ging es auch schon weiter. Sie musste diese seltene Gelegenheit nutzen, um etwas Erfahrung zu sammeln. Da aber Tabaluga den ganzen Tag auf dem Gelände verweilte, könne sie später noch einmal ein paar nicht invasive Untersuchungen durchnehmen, versicherte er.
%Invasiv
%Der Begriff invasiv wird in folgenden Zusammenhängen verwendet:
%1. In der medizinischen Diagnostik oder Therapeutik werden solche Methoden als invasiv (Vase) bezeichnet, die in den Körper, ein Korpergefäß eindringen, also z. B. eine Biopsie, Abstrich der Nasenschleimhaut. Eine Sonografie- oder Röntgenuntersuchung ist dagegen nicht invasiv.
%2. Beim Krebs spricht man von einem invasiven Tumor, wenn er in das umgebende Gewebe hineinwuchert.
%3. In der Softwareentwicklung spricht man von invasiven Programmiermodellen, wenn sie Auswirkungen auf viele Komponenten haben.
%4. In der Ökologie versteht man unter invasiven Organismen Neobiota, die sich durch eine hohe Vermehrungs- und Ausbreitungsrate auszeichnen. Siehe auch Biologische Invasion.

Als Nächstes ging es ins Lehrerzimmer. Nach einigen Minuten kam Professor Dumbledore herein und blieb wie angewurzelt stehen. Eine Weile schaute er den beiden zu. Frederick und Tabaluga standen am Fenster und sahen hinaus. Tabaluga stand wie ein Hund mit den Vorderpfoten auf dem Fenstersims.

\stimme{Wer ist das hinter uns?}

\gedanke{Mein Chef. Direktor Dumbledore. Der Schulleiter.}

\stimme{Du hast ihn gespürt. Nicht wahr?}

\gedanke{Du doch auch.} \enquote{Guten Morgen Albus}, sagte Frederick.

\enquote{Guten Morgen, Frederick. Wer ist Ihr Begleiter?}

\enquote{Tabaluga. Das war der Drache, der um Hilfe ersuchte.}

\enquote{Welche Art von Hilfe?}

\enquote{Seine Herde von einer Seuche zu befreien.}

\enquote{Hat er sie bekommen?}

\enquote{Ja.}

\enquote{Welche Seuche denn?}

\stimme{Drachenstein}, sagte Tabaluga.

\enquote{Wer?}

\stimme{Ich.}

\enquote{Der Drache.}

\enquote{Er kann sprechen?}, fragte Dumbledore.

\enquote{Es ist mehr eine gedankliche Kommunikation. Das müssten Sie doch wissen. Wie war das mit den zwölf Anwendungen für Drachenblut?}

\enquote{Da habe ich mich nur mit dem Blut beschäftigt und auch teilweise auf Arbeiten anderer aufgebaut. Ich habe nie mit einem Drachen gearbeitet.}

Tabaluga ging wieder auf seine vier Pfoten und meinte: \enquote{Zeigst du mir noch mehr?}

Dumbledore war ganz erstaunt darüber. \enquote{Du kannst sprechen?}

\enquote{Das hat mich auch schon gewundert}, sagte Elber. \enquote{Vermutlich liegt es an seinem besonderen Erbe.}

\enquote{Welche Art von Erbe?}, fragte Dumbledore und Tabaluga.

\enquote{Gehen wir ein Stück, dann erzähle ich es euch.} Er trat voran und verließ das Lehrerzimmer mit Dumbledore und Tabaluga, der zwischen ihnen lief. \enquote{Tabaluga hat mir erzählt, dass seine Mutter von einem Basilisken gebissen wurde. Deswegen auch der silberne breite Streifen auf seiner Brust. Das hat wahrscheinlich dazu geführt, dass er mehr Verständnis für die Magie entwickelt hat. Zudem dürfte ihm dies auch behilflich gewesen sein, unsere Sprache zu beherrschen.}

Beide staunten über diese Theorie. Erklärte sie doch sehr gut, was passiert war, aber nicht warum.

\enquote{Was ist mit dem Drachenstein?}, fragte Dumbledore nach. \enquote{Es gibt eine Legende, dass er einst von einem dunklen Zauberer erschaffen wurde. Damit sollten die Drachen im Zaum gehalten werden. Es war eine dunkle Zeit für die Drachen. Doch plötzlich verschwand der dunkle Tyrann und mit ihm der Drachenstein.}

Dumbledore und Tabaluga sahen zu Elber, doch dieser schien sie zu ignorieren.

\enquote{Was wissen Sie darüber?}, fragte Dumbledore nach.

\enquote{Ich möchte darüber nicht sprechen.}

Die Reaktionen der beiden Begleiter waren recht unterschiedlich.

Tabaluga dachte an die Gelegenheit, etwas darüber zu erfahren, wenn er wieder bei Nepomuk war, um ihm sein Riechorgan zu kurieren.

Dumbledore schaute Elber kritisch an und dachte erst einmal nach. Dann fragte er: \enquote{Etwas aus Ihrer Familiengeschichte?}

\enquote{So ähnlich}, gab er knapp zurück. Er wollte keine weiteren Nachfragen.

Sie verließen das Schloss und liefen hinunter zu Hagrids Hütte.

Dieser begrüßte die beiden Frühaufsteher. \enquote{Guten Morgen die Professoren.} Und dann, als er den Drachen erkannte: \enquote{Verdamm’ mich. Ein echter Drache. Wo habt’s denn den her? Das ist ja ein prächtiges Exemplar. Irgendwie erinnert er mich an Norbert.}

\gedanke{Sag jetzt bloß nichts Falsches}, ermahnte Frederick Tabaluga. \enquote{Er kam vorgestern Abend hier an und suchte nach Hilfe. Da es bei einigen Exemplaren noch etwas dauert, hat er mich begleitet. Ich dachte, Sie könnten heute eine kleine Stunde über Drachen halten, wenn schon mal einer da ist?}

\enquote{Wie zahm isser denn? Wegen reiten und so. Falls einer der Schüler mal.}

\stimme{Ich lasse keinen auf meinen Rücken steigen nur aus purem Vergnügen. Aber anfassen dürfen die mich und streicheln, wenn sie wollen.}

\enquote{Zahm genug zum Streicheln, Hagrid. Aber auf ihm reiten lassen, würde ich keinen. Da ich heute Vormittag freihabe, dachte ich mir, wir können uns noch etwas zusammen setzen, bevor die Schüler kommen. Tabaluga kann uns begleiten. Schließlich wird er heute kräftig dabei sein, nehme ich an.}

\enquote{Ich gehe dann mal eine Runde spazieren}, meinte Dumbledore und verabschiedete sich von den dreien.

Hagrid bat die zwei hinter die Hütte um auf einer großen Decke, auf der bereits Fang lag, sich zu unterhalten.

Als sich zur normalen Frühstückszeit die Halle füllte, fing Dumbledore mit einer kleinen Rede an, indem er mit einem Löffel gegen seinen Kelch schlug. Die Menge verstummte.

\enquote{Ich habe eine kleine Überraschung für euch. Zumindest für einen Teil von euch. Diejenigen unter euch, die heute bei Hagrid Unterricht haben, werden eine kleine Änderung des Schulplanes haben. Hogwarts hat das seltene Vergnügen, für heute einen jungen Drachen zu haben. Einen ungefährlichen Drachen. Hagrid wird euch heute etwas über ihn erzählen. Den anderen wünsche ich trotzdem einen schönen Unterricht. Auch wenn diese ihn sicherlich nicht haben werden.} Dann setzte er sich wieder.

\enquote{Drachen? Heute?}, fragte Ron entsetzt. \enquote{Nicht bei Hagrid.}

\enquote{Wovor hast du Angst Ron?}, fragte Harry.

\enquote{Angst? Eine Scheißangst. Harry das ist ein Drache. Du weißt doch noch, was während des trimagischen Turniers passiert ist.}

Die um die beiden herumsitzenden Gryffindors nickten und schauten nicht gerade begeistert drein.

\enquote{Warum habt ihr denn Angst?}, fragte Hermine.

\enquote{Hallo? Hagrid! Drache! Klingelt da etwas?}

\enquote{Also bei mir klingelt da nichts. Ich habe den Drachen als ungefährlich in Erinnerung.}

Das Klappern in der Umgebung wurde leiser.

\enquote{Als ungefährlich in Erinnerung?}, fragte Dean nach.

\enquote{Ja}, antwortete Harry. \enquote{Äh, ich denke es zumindest. Es war doch vor zwei Tagen einer da. Ich nehme einfach an, dass\abs}

\enquote{Du nimmst einfach an?}

Harry fiel erst jetzt auf, dass er einfach von dem Drachen ausgegangen war, der um Hilfe ersuchte. \enquote{Ups. Ja, ich nahm einfach an, dass es sich um diesen Drachen handelt. Der war ja zahm.}

\enquote{Dann träum’ mal weiter. Wenn Hagrid den Drachen ausgesucht hat, dann\abs}

\enquote{Moment}, unterbrach ihn Harry. \enquote{Jetzt nimmst du aber an.}

\enquote{Wie auch immer. Wir werden es bald sehen.}

Die Unterhaltung wurde durch ein klingendes Geräusch abgebrochen.

\enquote{Gerade haben mir einige Kollegen gesagt, dass sie heute keine Zeit haben zu unterrichten, da sie}, jetzt musste er schmunzelnd, \enquote{anderweitig beschäftigt sind.} Vereinzelt war jetzt ein Kichern zu hören. \enquote{Also wird es heute nur einen Unterricht geben. Hagrid wird euch alle heute unterrichten.} Es wurde ganz still in der Halle. Man konnte eine Stecknadel fallen hören. \enquote{Er wird euch etwas über Drachen erzählen. Auf diesem Gebiet ist er spitze, wenn ihr mir den Ausdruck gestattet.}

\enquote{Und wenn wir es nicht gestatten?}, fragte ein Schüler nach.

Dumbledore pausierte kurz. \enquote{Dann ist er sehr versiert und kann euch neben einem Drachenhüter sehr viel über diese Tiere erzählen. Ich schlage vor, wir essen zu Ende und werden dann gemeinsam\abs}, dabei sah er zu seinen Lehrerkollegen, \enquote{\aabs zu Hagrid gehen.} Dann setzte er sich wieder und aß weiter.

Während er aß, sinnierte Harry darüber, ob Snape freiwillig seine Stunden heute ausfallen ließ, oder ob er einfach überstimmt wurde. Er blickte kurz zum Lehrertisch und wartete einige Sekunden. Snapes Auge zuckte kurz, als er in seine Richtung sah. Der Rest seiner Miene war wie immer unergründlich. Er versuchte seine Legilimentik an ihm und übermittelte ihm die Frage, ob er es freiwillig getan habe. Als Antwort bekam er nur einen kleinen Jungen zu sehen, der auf eine Frage hin nickte.

\enquote{Was meinst du, Harry?}, fragte ihn Hermine.

Harry drehte langsam seinen Kopf. \enquote{Wozu?}

\enquote{Ob Snape seine Stunden freiwillig hergegeben hat!}

Harry überlegte, was er sagen konnte, oder wie er sich ausdrücken sollte. \enquote{Ich nehme an, er will auch einen Drachen sehen. Vor allem aber, wie Hagrid mit ihm fertig wird. Vielleicht hofft er auch, etwas abzubekommen. Drachen haben immerhin wirkungsvolle Substanzen, um sie in Zaubertränken zu verarbeiten.} Da er immer noch mit Snape verbunden war, bekam er als Antwort ein lachendes Gesicht eines kleinen Kindes. Unwillkürlich musste er lächeln. Dann verblasste es, als er aufblickte und das Gesicht seiner Mutter sah. Dann brach die Verbindung ab. Harrys Kopf zuckte kurz, aber er zwang sich, nicht mehr hinzusehen.

Dann, nach dem Essen, standen sie auf und gingen hinaus, um Hagrid aufzusuchen. In der Gruppe der Schüler ging es den üblichen Hang hinab. Doch als die Gruppe abbog, stutzte Harry \gst bis er den Grund dafür sah. In der Luft waren leuchtende Punkte, die sich zu bewegen schienen. Sie bildeten einen Pfeil. Darunter bildeten ebenfalls leuchtende Punkte einen Schriftzug. \accentuate{Drache} stand da. Sie folgten, wie die anderen, den Hinweisen und Wegpfeilern. Bunte Leuchtwürmchen überflogen sie und bauten sich weiter vorne wieder zu einem Schild auf. Harry musste grinsen.

\enquote{Die Idee stammt bestimmt nicht von Hagrid}, sagte er, als er knapp vor dem Leuchtwürmchen abbog.

Die Würmchen bildeten kurz das Wort, \accentuate{Stimmt!}, worauf hin Harry lachen musste.

\enquote{Warum lachst du}, fragte ihn Ron.

\enquote{Ich musste gerade an etwas denken. Etwas Lustiges.}

Dann kamen sie auch schon am Quidditch-Feld an und bestiegen die Ränge, da die Leuchtwürmchen ihnen wieder den Weg wiesen. Oben auf den Rängen angekommen stellte er fest, dass in der Mitte eine Bühne aufgebaut wurde, auf der bereits Hagrid und der Drache stand. Ganz am Rand konnte er Professor Elber sehen, der auf dem Boden der Bühne auf einem Kissen saß. Auf der anderen Seite saß Professor Hooch. Die beiden Professoren schienen zu ruhen. Von hier sah es aus, als ob beide ihre Augen zu hatten und dösten, oder meditierten.

Ginny saß neben ihm. Es war angenehm.

Nachdem sich alle hingesetzt hatten, begann Hagrid. \enquote{Doch so viele gekommen, ja? Na gut. Wie bereits alle wissen, haben Drachen sehr gute Chancen, einem Zauber zu entgehen, da dieser keine Wirkung auf sie hat. Dies verdanken sie der ihnen eigenen Magie, die in ihrer Haut lagert.}

Die beiden meditierenden Professoren hoben ihre Hände. Dazwischen bildete sich eine Kugel aus Magie. Eine blaue auf der Seite Professor Elbers und eine gelbe auf Seiten Professor Hoochs. Sie rasten auf den Drachen zu und verpufften wirkungslos. Dann saßen die beiden wieder stumm da.

So ging es die ganze Stunde durch. Immer wieder erzählte Hagrid etwas über einen Drachen und manchmal führten die beiden Professoren dies plastisch vor.

Dann, gegen Ende der Stunde, welche bereits vier ganze Stunden dauerte, fragte Hagrid in die Menge: \enquote{Wer unter euch stellt sich für eine kleine Mutprobe zur Verfügung? Diese beinhaltet, sich in die Mitte dieses Kreises zu stellen}, Hagrid zeigte darauf, \enquote{und sich von Tabaluga hier mit Feuer umfließen zu lassen. Wir machen das, wenn die Mittagspause beendet ist.}

Dann ging es an das Mittagessen. Die Elfen erschienen und brachten jedem Schüler ein belegtes Sandwich und einen Becher Kürbissaft, welcher so verzaubert war, dass man daraus trinken, es aber nicht verschütten konnte.

Als Harry wieder aufsah, nachdem er von einem Elfen sein Mittagsmahl entgegengenommen hatte, waren Professor Elber und Tabaluga verschwunden.




\begin{kommentar}
Nachdem der Drachenkolonie geholfen wurde und der Drachenstein zerstört worden war, sprechen Frederick und Tabaluga mit dem Drachen Nepomuk. Dieser Name stammt aus der Kindersendung 'Hallo Spencer'. Dort heißt der Erzähler so. Zwar gibt es auch einen Drachen, aber dessen Name ist Leopold oder auch kurz Poldi.
\end{kommentar}

\begin{kommentar}
Natürlich wurde der Drachenstein von Elber erschaffen. Daher wusste er auch, wie man ihn zerstört.
\end{kommentar}
