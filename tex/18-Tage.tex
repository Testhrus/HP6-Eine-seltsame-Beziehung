\chapter{Tage}


Harry stand an der letzten Stelle, an dem sich ein Punkt auf der Karte zeigte. Es war eine simple Wand. Doch etwas war an ihr anders. Das spürte er. Er bearbeitete die Stelle mit verschiedenen Zaubern und endlich gab die Wand nach und bildete eine Tür aus. Mit einem recht simplen \spruch{Alohomora} öffnete er die Tür und trat ein. Er stand in einem kleinen runden Raum mit Fackeln zu beiden Seiten, die begannen aufzuleuchten, als er den Raum betrat. Gegenüber von ihm war ein Torbogen und Treppen schienen nach unten zu führen. Sonst konnte er nichts feststellen. Er trat ein und ging die Treppen hinunter. Die Tür hinter ihm schloss sich automatisch. Am Ende der Treppen kam ein kleiner Gang von etwa vier Metern. Dann hörte er auf. Alles war in Stein gehalten. Harry untersuchte alles ganz genau. Jeden Zauber, den er kannte, verwendete er. Doch nichts half. Es schien ein Überbleibsel zu sein. Etwas, was man verschlossen hatte, als man das Schloss umbaute, vermutete er. Er ging wieder nach oben und versuchte durch die Tür zu kommen. Doch auch hier half nichts.

Nach einer halben Stunde gab er auf und setzte sich auf die Stufen. Er nahm den Kopf zwischen die Hände und resignierte.

Neben ihm tauchte Salazar auf. \enquote{Tut mir leid, dass du in dieser Situation bist. Wenn ich das früher bemerkt hätte, dann hätte ich dich gewarnt. Hier kommst du nicht durch eigene Kraft heraus.}

\enquote{Warum?}

\enquote{Das ist eine kleine Gemeinheit, für allzu neugierige Schüler. Der Direktor der Schule wird frühestens in vierundzwanzig Stunden etwas davon erfahren. Dann hat er eine Idee, wo du sein könntest. Nach weiteren vierundzwanzig Stunden wird es ihm bewusster, falls er nicht darauf reagiert und nach insgesamt zweiundsiebzig Stunden wird er magisch gezwungen, dich zu holen.}

\enquote{Das heißt, ich komme vor morgen hier nicht raus?}

\enquote{Tut mir leid, Harry. Aber wir können ja immer noch reden.} Dann verschwand sein Abbild.

\trenn

Etwas früher im Krankenflügel.

\enquote{Vor drei Tagen hat sie zum letzten Mal geschlafen. Wie lange können wir ihr das denn noch zumuten?}, fragte Professor Elber Madame Pomfrey.

\enquote{Ich verstehe deine Sorge, Frederick, aber ihre Lebenszeichen sind nach wie vor stabil. Sie scheint nicht in unmittelbarer Gefahr zu sein.}

\enquote{Sie kam mit einem unbekannten Gift in Kontakt, das vielleicht für uns nicht nachvollziehbare Wirkungen hat. Es könnte umschlagen und sie in Minuten töten.}

\enquote{Willst du mir etwa in meine Arbeit hereinreden? Willst du sie etwa machen?}

\enquote{Könnte sie etwas daran hindern, zurückgeholt zu werden?}, fragte Professor Sinistra.

\enquote{Nein, sie ist in gutem Zustand und hat die volle motorische Kontrolle.}

\enquote{Ist es sicher, dass diese Strapazen ihr helfen?}

\enquote{Ja. Ihre gesamte Biochemie erfährt eine Reihe von ungewöhnlichen Interaktionen. Es wurden Veränderungen in ihr ausgelöst, die auf ihre Magie wirken. Als sich das Gift in ihrem Körper löste, hat das einige interessante Veränderungen herbeigeführt.}

\enquote{Dann kann sie sich nicht zurückholen, wenn sie unter Drogen steht.}

\enquote{Dieses Gift ist vielleicht der Schlüssel zur Bekämpfung der Medusen.}

\enquote{Na schön}, gab er nach. \enquote{Ich werde aber bis zu ihrer Rückkehr nicht von deiner Seite weichen.}

\trenn

Katharinas Kopf brummte und langsam bewegte sie ihn.

Dann durchfuhr sie ein greller Lichtblitz, der durch ihre geschlossenen Augen jeden Nerv in ihrem Körper zu reizen schien. Kurz, aber heftig. Dann stand sie an einem Strand. Hinter ihr war eine Klippe zu sehen und vor ihr das blaue Meer, das durch den Wind eine Gischt bildete. Mit leichten Kopfschmerzen rieb sie sich übers Gesicht. Dann erst realisierte sie, wo sie war.

Sie hörte Geräusche und drehte sich in die Richtung, aus der sie kamen. \enquote{Was ist das?}, fragte sie ihre Führerin, die einige Meter von ihr entfernt und leicht hinter ihr stand. \enquote{Eine Halluzination?}

\enquote{Ich bin nur hier, um als Stimme zu dienen. Als Übersetzer für Ihre Vorfahren, Ihren Leidensgenossen, Ihrer ursprünglichen Fluchmutter, der Medusa.}

\enquote{Ich verstehe}, sagte Katharina begeistert und wandte sich wieder dem Meer zu. \enquote{Wenn hier meine Vorfahren sind, kann ich sie dann sehen?}

\enquote{Sie meinen, Sie wollen einen Beweis, dass wir existieren?}

\enquote{Ja, das wäre hilfreich.}

\enquote{Es ist bedeutungslos.}

\enquote{Ich will nicht respektlos sein. Ich habe einen jeden Abschnitt des Rituals durchlaufen, der mir aufgetragen wurde.}

\enquote{Alles, was Sie durchgemacht haben, ist bedeutungslos, das wurde Ihnen doch gesagt.}

\enquote{Ich weiß \gst Es ist nur mein Wunsch, es endlich abzuschließen, um Erlösung zu finden.}

\enquote{Tun Sie das, tragen Sie ihre Bitte vor. Niemand wird Sie davon abhalten.}

\enquote{Bitte, helfen Sie mir und denen, die ich versteinert habe. Sagen Sie mir, was ich tun muss.}

\enquote{Ihre Bitte ist inkonsequent, Sie haben alles in sich, was Sie brauchen und was für Ihre Rettung vonnöten ist.}

Katharina wurde mit Glück durchflutet. Sie begann zu lächeln. Wieder wurde ihre Umgebung hell und wieder wurden ihre Nerven stimuliert. Dann wurde ihr kurz schwarz vor Augen und sie sah, wie sich die Grube, in der sie lag, öffnete.

\enquote{Schön, dass Sie wieder da sind}, sagte Katharinas Führerin und reichte ihr die Hand.

\enquote{Wie lange?}, fragte Katharina

\enquote{Spielt das eine Rolle?}

\enquote{Ich würde es gerne wissen.}

\enquote{Neununddreißig Stunden \gst Sie müssen sich etwas schonen, ihr Körper ist schwach.}

\enquote{Ich schätze, die Duellübungen und mein Lauftraining sowie meine Fitnesseinheiten haben mich darauf nicht vorbereitet.}

\enquote{War es die Sache wert?}

\enquote{Ich denke schon. Es wurde mir gesagt, dass ich hätte, was ich brauche, um meine Freunde zu retten und mir mit meinem Fluch zu helfen.}

\enquote{Das muss die Wahrheit sein, Ihre Vorfahren würden Sie nicht belügen, nicht unter diesen Umständen.} Sie ging an den Rand der Kammer und holte Katharinas Kleidung, fein säuberlich zusammengelegt, und überreichte ihr das Bündel. \enquote{Wann immer Sie bereit sind.}

Katharina nahm ihre Kleidung an, betrachtete sie und sah dann ihre Führerin aus müden Augen heraus an. \enquote{Danke}, war das Einzige, was sie hervorbrachte.

Ihre Führerin verbeugte sich leicht und verließ den Raum, nachdem sie ihr ihren Zauberstab zurückgegeben hatte. Katharina zog sich mit einem glückseligen Lächeln an und ging, durch ihre Führerin begleitet, zurück zum Wartebereich, wo sie von Pomona freudig begrüßt wurde.


Zurück auf Hogwarts wurde sie von Madame Pomfrey untersucht. \enquote{Ihnen fehlt eine Nacht Tiefschlaf und eine anständige Mahlzeit. Ansonsten sind Sie in guter Verfassung.}

Professor Elber, aber vor allem Pomona, freuten sich.

\enquote{Das Ritual war sicher sehr anstrengend, aber wie ich vermutete ist das Gift der Schlüssel, um ihren Mitschülern zu helfen. Ich werde ihnen etwas Blut abnehmen und dann mit dem Alraunenextrakt mischen. Ihre Schlangen sollten sich die nächsten Tage zurückbilden}, schloss Madame Pomfrey.
%33:57
Sie nahm ihren Zauberstab und zog magisch etwas Blut in ein kleines Reagenzglas ab. Dann ging sie in ihr Büro und holte ein bereitstehendes Gefäß mit dem Alraunenextrakt und goss das Blut, nachdem sie einen Zauber darauf angewendet hatte, in den Becher. Wieder aus ihrem Büro zurück, ging sie hinter einen der Vorhänge, die um ein Bett gezogen waren, und war außer Sichtweite.

Professor Elber und Professor Sinistra verließen die Krankenstation und Pomona setzte sich zu Katharina. \enquote{So meine liebe, du hast es überstanden. Ich bin so stolz auf dich. Dann kannst du jetzt den blöden Turban abnehmen.}

\enquote{Aber mein Anblick\abs}

\enquote{Stört mich nicht im Geringsten}, unterbrach sie ihre Adoptivmutter. \enquote{Ich mag dich so, wie du bist, Katharina. Oder glaubst du, ich adoptiere nur jemanden mit makellosen Haaren?}

Katharina musste schmunzeln. Gedankenverloren schüttelte sie ihren Kopf. Dann begann sie ihren Turban abzuwickeln und ihre Schlangen zu offenbaren.

Zuerst passierte nichts, die Schlangen auf ihrem Kopf waren ruhig und bewegten sich nicht. Doch dann erwachten sie und Pomona versteinerte.

Ein Schrei entwich aus Katharinas Kehle, der von der Krankenschwester nur deshalb nicht gehört wurde, weil sie zeitgleich ebenfalls einen ausstieß. Nach viel zu langen Schocksekunden wickelte sich Katharina wieder ihren Turban um ihren Kopf und sah ihre Adoptivmutter mit Entsetzen im Gesicht an.

Die Türen zur Krankenstation wurden aufgestoßen und Professor Elber und Professor Sinistra kamen mit geschlossenen Augen herein.

\enquote{Können wir unsere Augen öffnen?}, fragten sie.

Katharina nickte, stutzte kurz und bejahte dann die Frage.

Mit vor Schock geweiteten Augen starrten die beiden sie und die steinerne Statue von Pomona Sprout an.

\enquote{Es hat nicht geklappt}, schluchzte Katharina.

\enquote{Ich brauche hier Hilfe}, kam ein ziemlich lauter Schrei hinter einem er Vorhänge hervor.

\enquote{Was ist, Poppy?}, fragte Professor Elber.

\enquote{Die sterben mir hier weg. Die Versteinerung ist nur teilweise aufgehoben.}

\enquote{Komm da raus, Poppy}, sagte Professor Elber und lief auf den Vorhang zu.

\stimme{Katharina, du kommst her und nimmst kurz deinen Turban ab, wenn wir hier wieder raus sind}, hörte sie in ihrem Geist.

Professor Elber zog den Vorhang beiseite und zog die sich sträubende Poppy mit Gewalt fort.

Katharina verschwand hinter dem Vorhang und kam nach wenigen Sekunden wieder hervor.

Professor Elber ließ Poppy los, worauf diese sofort auf den Vorhang zu stürmte und hinter ihm verschwand. Doch sie kam schnell wieder zurück.

\enquote{Versteinert}, sagte sie, worauf Katharina wieder zu weinen anfing.

\enquote{Es hat nicht geklappt}, schluchzte sie.

\enquote{Tut mir leid Miss Chapel, aber es scheint, dass alles, was sie durchlebt haben, bedeutungslos war} , sagte Madame Pomfrey.

Katharina schaute sie mit verweinten Augen an. Tränen liefen ihre Wangen hinunter.

\enquote{Gehen Sie auf ihr Zimmer und beruhigen Sie sich erst einmal.}

\enquote{Beruhigen? Ich habe gerade meine Adoptivmutter verloren, kaum, dass ich sie habe.}

\enquote{Sie haben sie nicht verloren. Sie ist momentan nur außer Reichweite. Das wird wieder. Gehen Sie jetzt, oder ich bringe Sie persönlich in Ihr Zimmer und decke sie zu.}

Das zeigte Wirkung, denn die Vorstellung, dass ihre Krankenschwester sie durch den Gemeinschaftsraum hindurch in ihr Zimmer bringen würde, war Ansporn genug. In ihrem Zimmer legte sie sich auf ihr Bett und weinte die restlichen Stunden bis zum Abendessen.

Obwohl sie, bevor sie sich hingelegt hatte, einen Apfelschnitzel gegessen hatte, der ihren größten Hunger vertrieben hatte, war ihr Hunger doch sehr groß. Sie hatte immerhin drei Tage lang nichts gegessen. Deshalb lud sie sich in der Großen Halle auf ihren Teller, was sie dachte zu essen, denn, auch wenn das Essen in der Tischmitte verschwand, wenn die Essenszeit zu Ende war, blieb das auf dem Teller liegen, sofern man noch aß.

\enquote{Und, wie ist es gelaufen?}, fragte sie Elisabeth, eine ihrer Schulkameradin.

Katharina sah sie eindringlich an. \enquote{Nichts ist in Ordnung. Meine Schulkameraden sind immer noch versteinert, Dumbledore auch. Und meine Adoptivmutter seit dem Nachmittag auch. Also kannst du dir denken, dass es mir nicht gut geht.}

\enquote{Kann ich dir helfen?}

\enquote{Ich weiß momentan nicht wo mir der Kopf steht.}

\enquote{Verständlich}, nickte ihre Mitschülerin, legte eine Hand auf ihrer Schulter ab und sagte: \enquote{Ich schreibe weiterhin für dich mit.}

\enquote{Danke}, sagte Katharina.

\trenn

Harry hörte ein Knarzen und stand auf. Er lief auf die offene Tür zu und trat hinaus.

\enquote{Dort haben Sie sich verkrochen, Mister Potter.}

\enquote{Tut mir leid, Professor McGonagall. Ich war einfach zu neugierig.}

\enquote{Ja, das waren Sie. Melden Sie sich bei Hagrid. Er wird Ihnen hoffentlich etwas davon nehmen können.}

\enquote{Ich hoffe doch, etwas Unangenehmes.}

\enquote{Wie bitte?}, fragte Professor McGonagall nach.

\enquote{Na ja, sonst verfehlt es doch seine Wirkung}, meinte Harry und machte sich auf den Weg zu Hagrid.

Nach dem Unterricht brauchte Katharina erst einmal eine Pause. Sie nahm sich nur ein paar Sandwichs aus der Großen Halle mit und ging dann Richtung Verbotener Wald. Mit diesem Misserfolg musste sie erst einmal zurechtkommen. Sie begegnete am Waldrand Harry, der ebenfalls seinen Gedanken hinterher hing. Dadurch etwas aufgemuntert, besserte sich ihre Laune etwas.

\enquote{Hallo Katharina. Bekomme ich eine ehrliche Antwort, wenn ich dich fragte, wie es dir geht?} Sie nickte nur. \enquote{Und, wie geht es dir?}

\enquote{Schlecht. Die Prüfung in Griechenland habe ich vermasselt. Ich habe das Ziel nicht erreicht. Jetzt ist auch noch Pomona\abs Professor Sprout versteinert.} Tränen begannen ihre Wangen hinunterzulaufen. \enquote{Jetzt habe ich auch noch meine Adoptivmutter verloren.}

Harry beruhigte sie, indem er auf sie zuging, ihre Tränen mit einem Taschentuch wegwischte und sie einfach nur in den Arm nahm. Langsam beruhigte sich ihr Zittern, das Harry sofort spürte, als er sie an seine Brust drückte. Nach einigen Minuten sah er aus seinem Augenwinkel, wie sich etwas bewegte. Er drehte seinen Kopf leicht und sah in das Gesicht einer kleinen Schlange, die sich sofort wieder zurückzog, als sie Harry erblickte.

Es dauerte eine Weile, in der er verkrampft dastand, bis sich die Erkenntnis durch sein Gehirn in seinen Verstand bewegt hat. Er hatte eine der Medusen-Schlangen auf Katharinas Kopf gesehen. Dann erkannte er, dass sich Katharina einen Turban um den Kopf gebunden hatte. Einen, wie sie sich Sikhs umbinden, oder wie Professor Quirrel ihn um hatte. Und mit einem Schlag wurde ihm klar, dass die Schlangen keine Auswirkung auf ihn hatten. Ihn irritierte nur, dass Katharina anscheinend Professor Sprout versteinerte, es aber auf ihn keine Auswirkung hatte. Seine Gedanken kreisten um sich, seine Magie, das was er gelernt hatte und, um Marcel \gst und um Salazars Amulett.

Er war sich unsicher, ob er Katharina davon erzählen sollte.

Sie löste sich von ihm, sah ihn dankbar an und gab ihm einen Kuss auf die Wange.

\enquote{Kommt’s mal mit}, hörten beide plötzlich aus dem Inneren des Waldes. Etwa zehn Schritte von ihnen entfernt, sahen beide Hagrid wie er innerhalb des Waldes stand und beide heranwinkte. Zuerst etwas unsicher, dann aber zunehmend sicherer ging Katharina zu Hagrid. Harry hatte keine Scheu und ging ohne Zögern hinein. Mit Hagrid konnte ihnen nichts passieren. Zumindest gab es keine Probleme was die Schulregeln betraf.

Ansonsten wusste keiner so genau, was sich im Inneren des Waldes befand. Stumm lief Hagrid voraus und kam nach wenigen Minuten an einen steinernen Torbogen.

\enquote{Da musste reingehen, Katharina}, sprach Hagrid.

\enquote{Was muss ich da tun?}

\enquote{Du gehst durch den Torbogen, den Pfad lang und bleibt ne Stunde innem Kreis. Dann kommste wieder raus.}

\enquote{Warum?}, fragte sie.

\enquote{Frag nich so viel}, sagte Hagrid und schob sie Richtung Torbogen, was einen deutlichen Schub auf Katharinas Rücken ausübte.

Kaum war sie durch den Torbogen getreten worden, wurde ihr merklich wärmer. Sie wurde ruhiger und akzeptierte, dass sie den Pfad entlang zu einem Kreis gehen sollte, um dort die nächste Stunde zu verbringen. Sie war aber durchaus in der Lage, sich dagegen zu wehren, denn es war mehr eine Erkenntnis, als ein Zwang, der sie dazu brachte. Dann schaute sie sich noch einmal während des Laufens kurz um und lief dann den Pfad entlang. Als sie in den Kreis trat, der eher wie eine Art leichte Vertiefung aus Moos aussah, inmitten einer erdigen Waldfläche, schaute sie sich erneut um.

Einige abgeschnittene Baumstümpfe dienten als Sitzgelegenheit und in der Mitte lag eine Art Hügel, der moosbewachsen war. Katharinas Gefühle übermannten sie und sie setzte sich auf einen der Baumstümpfe. Dann fing sie wieder einmal hemmungslos an zu weinen. Nach ca. einer viertel Stunde wurde sie durch eigenartige Geräusche in ihren Gedankengängen unterbrochen. Katharina kannte das Geräusch nicht, da sie nicht bei Muggeln aufgewachsen war. Harry hingegen würde es als schnelles vorspulen eines Kassettenrecorders während des Abspielens erkennen. Doch Katharina konnte damit nichts anfangen. Sie nahm ihre Hände von ihrem Gesicht und lauschte den Geräuschen. Als ihr Blick auf den Hügel fiel, konnte sie kleine weiße Erhebungen sehen, die sich bewegten.

Im ersten Moment dachte sie an Pilze, \gst aber Pilze bewegten sich nicht. Es mussten Wesen sein, deren Köpfe \gst oder Hüte \gst wie Pilze aussahen.

Sie wollte schon \accentuate{Hallo} sagen, entschloss sich dann aber zu einem: \enquote{Kommt ruhig raus und sagt mir, wer ihr seid. Ich bin Katharina Chapel, Schülerin auf Hogwarts.}

Langsam und zögerlich kam der Erste der beiden Wesen ein Stückchen höher, um sie anzusehen. Dann traute sich auch der Zweite, nachdem der Erste eine winkende Geste gemacht hatte. Der Anblick der beiden kleinen Wesen, die Katharina bis zum Knie gingen, verschlug ihr die Sprache. Ihr Kopf ging oben in einen pilzartigen weißen Hut über, die Hose, welche auch Teil ihres Körpers war, war ebenfalls weiß. Der Oberkörper sowie der Kopf waren hingegen blau, die Schnauze sah wie die einer Katze aus, aber die Schnurrbart-Haare fehlten.

Irgendwie erinnerten sie diese Wesen an eine Zeichnung, die sie einmal gesehen hatte. Eine ihrer Mitschülerin, eine muggelstämmige, hatte so eine Zeichnung eine Weile über ihrem Bett hängen. Stark stilisiert, aber dennoch gut erkennbar. Nur an den Namen erinnerte sie sich nicht mehr.

\enquote{Welcher Spezies gehört ihr an?}, fragte sie. \enquote{Ich bin ein Mensch.}

\enquote{Das stimmt nicht. Du bist eine der Medusen. Wir ernähren uns von den bösartigen Kreaturen, bevor sie zur Gefahr werden. Deshalb sind wir hier.}

\enquote{Ihr wollt mich essen?}, fragte sie panikartig.

\enquote{Nein, wir sind aus einem anderen Grund hier.}

Die beiden Wesen kamen näher. Einer der beiden hatte einen Wurfspieß, an dem eine Schnecke und eine kleine Rispe Beeren hing. Außerdem hatte er noch ein Bündel Seile geschultert.

\enquote{Wir sind hier, um auf dich aufzupassen und nur im Notfall uns von dir zu ernähren, falls du dich nicht unter Kontrolle bekommst.}

\enquote{Das habe ich schon versucht. Ich war in Griechenland und habe mich einem Ritual unterzogen. Aber es hat nicht geklappt. Meine Adoptivmutter wurde durch mich versteinert, nachdem ich meinen Kopfschutz abgenommen hatte.}

\enquote{Es sieht aus}, antwortete der mit der Schnecke, \enquote{als ob alles, was du durchgemacht hast, bedeutungslos war.}

Katharina erschrak. Genau das hatte ihre geistige Führerin in Griechenland auch gesagt. Sie meinte, was sie sagte.

\enquote{Was ist los, Katharina?}, fragte der andere kleine Kerl.

\enquote{Sie meinte, was sie sagte \gst meine Führerin \gst und Madame Pomfrey. Als ich dort war und meine Prüfung absolvierte \gst Ich muss da noch einmal hin.}

\enquote{Deine Stunde ist aber erst in fünf Minuten um.}

Katharina war wie in einem Rausch und musste sich Zwanghaft beruhigen. Dann fragte sie: \enquote{Wie heißt ihr eigentlich?}

\enquote{Ich bin Nimu und das ist mein Kollege Minu. Wir gehören dem Volk der Medusoner an. Wir nähren uns von Medusen, die gefährlich für andere Spezies sind. Aber bei dir haben wir keine große Sorge. Wir werden wieder gehen und uns anderen Medusen widmen.}

\enquote{Wann geht ihr? Ihr bleibt hoffentlich noch ein paar Tage, ich möchte noch mit euch reden, eventuell um Hilfe bitten, falls ich es nicht schaffen sollte\abs}

\enquote{Deine Zeit ist abgelaufen}, sagte der Jäger der beiden.

Katharina erschrak, was der andere deutlich sehen konnte.

\enquote{Die Stunde ist um, meint mein Kollege. Wenn du rechtzeitig wieder da bist, triffst du uns noch.}

Katharina nickte und verließ den Kreis, rannte den Pfad entlang und durch den Torbogen. Sie zog Harry mit sich, der gerade aufstand, als er sie sah.

\enquote{Entschuldigen Sie, Professor Hagrid, ich erklär’s ihnen später. Harry, komm mit.}

Harry wurde von ihr mitgezogen, murrte aber nicht. Sie nahm ihn direkt mit ins Schloss, zog ihn in die Kerker hinunter, durch den Gemeinschaftsraum, in dem es Sekunden nach ihrem Eintreten totenstill wurde, und hinauf in ihr Zimmer. Die Gespräche, die im Gemeinschaftsraum stattfanden, konnte Harry nur erahnen, er hatte aber keine Zeit zum Nachdenken.

In ihrem Zimmer angekommen, zeigte sie auf ihr Bett und gab Harry das Zeichen sich zu setzen. Währenddessen kramte sie in ihrem Koffer nach einem Gegenstand.

Er ließ seinen Blick durch das Zimmer schweifen. Er konnte nun einige Gemeinsamkeiten mit dem Gemeinschaftsraum der Paare feststellen. Die Betten hatten ebenso wie die der Gryffindors Vorhänge und sahen genauso aus. Allerdings waren die Farben der Vorhänge grün und nicht rot. Er ließ seinen Blick weiter schweifen und entdeckte Pansy, die ein Buch las. Sie schien nicht realisiert zu haben, dass jemand im Zimmer war, der dort nicht hingehörte. Harry überlegte lange, ob er sich einen Scherz erlauben sollte. Aber bevor er an die Ausführung denken konnte, war Katharina schon zurück mit einem Bündel Stoff, in das etwas eingewickelt war.

\enquote{Das ist ein Portschlüssel}, sagte sie, als sie etwas auswickelte. \enquote{Ich habe ihn behalten \gst er funktioniert noch. Er wird uns nach Griechenland bringen. Du kannst mir vermutlich als einziger helfen.}

\enquote{Wieso ich?}, fragte Harry.

\enquote{Du kannst aus irgendeinem Grund meinen Medusen widerstehen.}

\enquote{Wie kommst du darauf?}

\enquote{Ich habe es gespürt, als wir uns vor dem Wald umarmt haben.}

\enquote{Ich habe dich gehalten}, warf Harry ein.

\enquote{Wie auch immer. Auf jeden Fall kannst du mir helfen. Sie scheinen auf dich nicht die Wirkung zu haben, die sie auf andere haben.}

Harry nickte und wurde von Katharina an der Hand gehalten, bevor sie den Portschlüssel aktivierte, indem sie ihn anfasste. Beide wurden mit einem Ruck hinfort gezogen und fanden sich Sekunden später in Griechenland wieder.

Zielstrebig ging sie voraus und Harry ihr hinterher. Im Warteraum setze sich Harry hin und Katharina ging wieder hinein. Harry hing seinen Gedanken hinterher und ihm kam die Idee, dass die Medusen auf ihrem Kopf kleine Basilisken sein könnten. Durch seine Arbeit mit dem kleinen Marcel war er schon etwas abgehärtet. Harry überlegte, ob noch andere Theorien infrage kamen.

Währenddessen traf Katharina wieder auf ihre Prüferin und sagte: \enquote{Sie meinten, was Sie zu mir sagten. Alles, was ich durchgemacht habe, war bedeutungslos.}

Sie nickte. \enquote{Ja.}

\enquote{Ich tat alles, was Sie von mir verlangten. Und sie ließen mich glauben, den anderen und mir helfen zu können. Warum haben Sie mich dann durch dieses Ritual geführt?}

\enquote{Ich habe Sie nirgendwo hingeführt Katharina. Sie haben mich dorthin mitgenommen, wohin Sie gehen wollten. Das war Ihr Ritual. Sie haben sich diese Herausforderungen selbst gestellt.}

\enquote{Das ist richtig. Ich bin hier mit gewissen Erwartungen erschienen. Das heißt, Sie haben das alles nur gemacht, um meine Erwartungen zu erfüllen?}

\enquote{Alles andere hätte Sie nicht zufriedengestellt.}

Das musste Katharina akzeptieren. Sie hatte damals irgendwie auf ein Ritual gedrängt, obwohl ihr gesagt wurde, dass es bedeutungslos war.

\enquote{Ich bin nicht bereit aufzugeben. Falls es noch möglich ist den Anderen zu helfen, will ich es versuchen.}

\enquote{Sind Sie um ihre Magie zu suchen wieder gekommen?}

\enquote{Ich weiß nicht mehr, was ich suche.}

\enquote{Dann glaube ich, dass Sie bereit sind zu beginnen.}

Nachdem sie wieder gereinigt und umgezogen war, begann ihre Reise erneut und Katharina befand sich wieder in der Kammer mit den drei wartenden.

% ~35:40
\enquote{Seht doch mal, wer wieder hergekommen ist}, sagte der mürrische Mann.

Katharina sah nicht gerade begeistert aus.

\enquote{Also}, fuhr er fort. \enquote{Ihr kleines Abenteuer ist nicht ganz so ausgefallen, wie Sie es sich erhofft hatten. Sie haben sich völlig umsonst in Schwierigkeiten gebracht.}

\enquote{Keine Gewissensbisse}, fuhr der ruhige Mann fort. \enquote{Sie würden es nicht für möglich halten, was sich manche Leute auf der Suche nach der eigenen Magie alles angetan haben.}

\enquote{Ein echtes Ritual existiert nicht, oder?}

\enquote{Echt ist ein so relativer Begriff. Die meistern Herausforderungen des Lebens werden von uns heutzutage selbst geschaffen.}

\enquote{Und Sie sind besonders hart zu sich selbst}, bemerkte die Frau, \enquote{nicht wahr?}

\enquote{Ich war immer bestrebt, erfolgreich zu sein}, gab Katharina zur Antwort.

\enquote{Starrköpfig, würde ich sagen. Sie haben es nie in Betracht gezogen, gemeinsam mit uns zu warten, nicht wahr?}

\enquote{Ich bin nun hier und bitte Sie um Hilfe \gst Ich möchte verstehen, welchen Sinn es hat \gst in diesem Raum zu warten.}

\enquote{Aber reicht es nicht}, antwortete die Frau, \enquote{gesellschaftliche Beziehungen zu pflegen? Wir sind unterhaltsame Leute.}

\enquote{Das also wird von mir erwartet. Ich soll mit den Geistern meiner Vorfahren sprechen.}

\enquote{Oh, zuerst waren wir ein Test, und jetzt sind wir die Geister ihrer Vorfahren.}

\enquote{Sind Sie es denn?}

\enquote{Das würde Ihnen gefallen}, murrte der Mann, \enquote{nicht wahr? Wenn wir alle greifbar wären, wenn man uns anfassen könnte. Wenn Sie mit Ihrer Magie auf uns einwirken könnten.}

\enquote{Wenn man alles erklären kann, was bleibt dann noch übrig, das einen selbst ausmacht?}

\enquote{Ich weiß, dass ich einzigartig bin. So wie jedes Individuum. Es fällt mir aber schwer, die Magie als einen eigenständigen Organismus zu sehen, dessen vertrauen ich gewinnen soll.}

\enquote{Soviel zu Ihrer toleranten Einstellung gegenüber anderen.}

\enquote{Jetzt mal langsam. Ich habe mich immerhin erfolgreich gegen den Einfluss meiner reinblütigen und Rassen-wahnsinnigen Eltern und Großeltern wehren können. Ich bin vielleicht toleranter als so manch anderer Reinblüter.}

\enquote{Halten Sie sich für etwas Besseres?}

\enquote{Das kommt auf den Standpunkt des Vergleichsobjektes an}, konterte Katharina.

\enquote{Ah, wir werden wissenschaftlich. Sagen wir, gegenüber einem Muggel, einem Kobold und einem Hauselfen.}

\enquote{In welchem Bereich?}

\enquote{Bereich Magie.}

\enquote{Überlegen, überlegen, unterlegen}, antwortete sie.

\enquote{Schön, wenn Sie sich für etwas Besseres halten, dann gehen Sie wieder zurück an Ihre Schule und helfen Sie den anderen.}

\enquote{Aber es hat ja augenscheinlich nicht funktioniert, oder? Ihren Mitschülern, oder Ihrem Direktor geht es nicht besser. Im Gegenteil. Ihre Adoptivmutter ist jetzt auch versteinert.}

\enquote{Nein, leider nicht.}

\enquote{Warum nicht?}

\enquote{Unsere Krankenschwester versteht es ebenfalls nicht. Sie war sich so sicher.}

\enquote{Also ein unerklärlicher Vorgang. Eine mysteriöse Nicht-Wieder"-ge"-ne"-sung.}

\enquote{Sie\abs Wir\abs haben den Grund dafür noch nicht gefunden.}

\enquote{Aber Sie werden ihn natürlich finden. Wenn Sie nur genügend Zeit haben. Das glauben sie doch als reinblütige Magierin, nicht wahr?}, fragte der gutmütige Mann.

\enquote{Bitte ehrlich sein}, murrte der andere.

\enquote{Ja, das habe ich immer geglaubt. Mit Magie schafft man alles.}

\enquote{Selbst dann, wenn sie versagt}, sagte der ruhige Mann, \enquote{sind Sie immer noch voller Glauben in sie. Das kommt einem Glaubensbekenntnis gleich.}

\enquote{Bedingungsloses Vertrauen}, sagte die Frau, \enquote{das klingt vielversprechend.}

\enquote{Wenn sie sagen, dass Magie Ihnen nicht helfen kann, was hilft Ihnen dann?}

\enquote{Das wird Ihnen nicht gefallen}, gab der mürrische Mann zurück.

\enquote{Ich werde tun, was notwendig ist.}

\enquote{Alles?}

\enquote{Alles!}

\enquote{Töten Sie sie. Sie sind bereits so gut wie tot. Warum wollen sie es nicht zu Ende bringen. Geben Sie ihnen noch eine Ladung ihrer Medusen.}

\enquote{Oh, das würde wirken}, antwortete die Frau.

\enquote{Wie wird es wirken?}

\enquote{Da ist es schon wieder}, murrte er. \enquote{Sie fangen schon wieder an. Immer müssen Sie eine rationale Erklärung verlangen. Aber es gibt gar keine. Ihre Krankenschwester und Ihre Bücher sind eindeutig. Ihre Schlangen auf dem Kopf sind tödlich. Erst versteinern sie, dann bringen sie einen tatsächlich um.}

Katharina erschrak. Sie hatte ihre Mitschülerin erneut angesehen. Aber diese war nicht mehr versteinert. Zumindest nicht ganz. Sie hoffte von ganzem Herzen, dass sie es überlebt.

\enquote{Falls Sie diesen Fakten glauben}, antwortete die Frau.

\enquote{Lassen Sie all das von sich fallen}, sagte der ruhige Mann, \enquote{Sehen Sie ihre versteinerten Mitschüler, Ihren Direktor und Ihre Adoptivmutter in der Reihenfolge der Versteinerung an und vertrauen sie darauf, dass sie wieder gesunden.}

\enquote{Das Ritual war vollkommen bedeutungslos}, sagte Katharina, den Umstand erst jetzt komplett verinnerlicht. \enquote{Und ich habe nichts getan, um mich darauf vorzubereiten, kein Trank, kein Zauber. Wie kann ich es dann schaffen?}

\enquote{Wenn Sie glauben, dass es funktioniert, dann sind Sie bereit. Das ist alles, worum es hierbei geht.}

\enquote{Aber wenn Sie es mit dem geringsten Zweifel angehen, mit Zögern und Zaudern, dann werden die Versteinerten sterben}, murrte der andere. \enquote{\gst Also, was werden Sie nun tun, Katharina.}

\enquote{Sie wissen, ich werde nicht zusehen, wie meine Freunde, mein Direktor und meine Mutter\abs in diesem versteinerten Zustand vor sich hin vegetieren, wenn ich sie irgendwie retten kann. Ich möchte glauben, dass es mir möglich ist \gst Ich werde es versuchen.}

Katharina verließ die Kammer, zog sich um und traf auf Harry.

\enquote{Und, wie sieht es aus?}, fragte er.

\enquote{Ich muss glauben}, sagte sie. Auf Harrys fragenden Blick fügte sie hinzu: \enquote{Nur so kann die Magie in mir wirken, sich entfalten. Nur so, kann ich sie überzeugen, dass sie in einer Art und Weise wirken soll, sie zu retten.}

Harry verstand. Er sagte: \enquote{Weißt du noch, was uns Professor Elber erzählt hat?} Katharina sah ihn fragend an. \enquote{\inner{Die Magie ist mein Verbündeter, und ein mächtiger Verbündeter ist sie. Ihre Energie umgibt uns, verbindet uns mit allem. Erleuchtete Wesen sind wir, nicht diese rohe Materie. Sie müssen sie fühlen, die Magie die sie umgibt\abs}}

Und Katharina vervollständigte: \enquote{\inner{\aabs, hier, zwischen Ihnen, mir, dem Baum, den Felsen dort, allgegenwärtig ja, selbst zwischen dem See und dem Stein auf seinem Grund.} \gst Ich verstehe. Er hat uns schon damals den Schlüssel zu dem gegeben, was ich jetzt tun muss. Was ich gerade anfange zu verstehen. Was in meinen Verstand, in meinen Körper\abs in meine Magie sickert.}

Harry verstand nur die Hälfte, nickte aber.

% Aus Kapitel 13 Ortswechsel
%\enquote{Die Magie ist mein Verbündeter, und ein mächtiger Verbündeter ist sie. Ihre Energie umgibt uns, verbindet uns mit allem. Erleuchtete Wesen sind wir, nicht diese rohe Materie. Sie müssen sie fühlen die Magie die sie umgibt, hier, zwischen ihnen, mir, dem Baum, den Felsen dort, allgegenwärtig ja, selbst zwischen dem See und dem Stein auf seinem Grund.}

\trenn

Am Samstag darauf wartete bereits Professor Elber am Ende der Kammer, als die ganze Gruppe mit Professor Dumbledore eintrat.

\enquote{Ah Albus, schön, dass Sie hier sind. Wie haben sich meine Schüler denn so geschlagen?}

\enquote{Mehr als beeindruckend. Ich hatte beim ersten Mal keine Ahnung wie weit, die schon sind. Dann wollte ich Ihnen etwas zeigen und Harry hier hätte mich fast umgehauen.} Professor Elber zog beide Augenbrauen hoch und fing an zu lachen. \enquote{Ich hatte keine Ahnung, das Sie schon so viel können.}

\enquote{Das ist nur eine Sache der inneren Einstellung und der Fantasie}, meinte Professor Elber und lief Richtung einer Wand. \enquote{Vieles in der Magie hängt von der Fantasie und Vorstellungskraft des Zauberers oder der Hexe ab. Das wird im allgemeinen suggestive Magie genannt. Was steht heute an, Albus? Haben Sie sich was Bestimmtes vorgenommen?}, fragte Professor Elber nun lässig an die Wand gelehnt.

\enquote{Ich dachte}, sagte Professor Dumbledore, \enquote{dass wir einfach mit den Duellen weitermachen, die wir bereits seit zwei Wochen durchführen.}

Professor Elber schaute plötzlich interessiert und meinte nur. \enquote{Dann lasst mal sehen was\abs}

Doch Professor Dumbledore unterbracht ihn und meinte nur: \enquote{Ich weiß, Sie konnten sich nicht vorbereiten, aber wie wäre es mit einem kleinen Show-Duell?}

Professor Elber zog wieder beide Augenbrauen hoch und meinte nur: \enquote{Wir beide?}

\enquote{Ja}, antwortete Professor Dumbledore.

\enquote{Hm, und wer ist der Böse?}

\enquote{Ich dachte, das übernehmen Sie.}

\enquote{Recht gerne}, entgegnete Professor Elber. \enquote{Dann roste ich wenigstens nicht ein. Was darf ich alles nicht verwenden?}, fragte er nach.

\enquote{Wie darf ich das verstehen}, fragte Professor Dumbledore.

\enquote{Na ja, muss ich mich im gesetzlichen Rahmen bewegen?}

\enquote{Ja sicher}, sagte Professor Dumbledore.

\enquote{Schade}, sagte Professor Elber und grinste.

Harry fühlte sich in der Magengegend plötzlich etwas mulmig. \gedanke{Schade? \gst Welche Zauber kennt denn Professor Elber noch, wenn er sich eingeschränkt fühlt? \gst Oder spielt er nur mit Dumbledore, um ihm ein flaues Gefühl zu bescheren?}

Professor Elber ging in die Mitte der Kammer und zog seinen Zauberstab. Danach ging er in Position und wartete auf Professor Dumbledore.

Die Schüler stellten sich wie auf Zuruf in einem Kreis am Rande der Kammer auf. Professor Elber schlenkerte seinen Zauberstab und zog ein schützendes Kraftfeld um die Zuschauer. Dann warf er etwas, das wie ein Gedanke aussah an die Innenseite, die sich derzeit in leichtem Blau zeigte, nun rot färbte und dann verschwand.

Dann, ohne Vorwarnung, schlug Professor Elber zu und schleuderte einen Blitz auf Professor Dumbledore, welcher ihn mit einem Schildzauber abwehrte. Professor Elber fing unmerklich an die Augenbraue zu heben und leicht zu grinsen. Abermals schleuderte er einen Blitz auf Professor Dumbledore zu, der wieder versuchte ihn mit einem Schildzauber zu blocken. Doch dieses mal schlug der Blitz durch den Schild und der Zauberstab flog Professor Dumbledore aus der Hand und streifte ihn leicht am Handgelenk. Professor Elber stoppte sofort den Zauber und wartete, bis Professor Dumbledore seinen Zauberstab holte, ohne sich zu bewegen. Einige Oh's und Ah's durchzogen die Halle und widerhallten an den Wänden.

\enquote{Nicht aufhören, in einem echten Kampf habe ich auch keine Gelegenheit mir meinen Zauberstab zu holen}, ermahnte ihn Dumbledore.

\enquote{Ok}, gab Professor Elber lapidar zur Antwort und schleuderte sofort einen weiteren Blitz auf Dumbledore zu, der ihn dieses mal ablenkte und gegen das Kraftfeld lenkte, welches ihn sofort neutralisierte. Dumbledore ging nun zum Gegenangriff über und lenkte einen dicken blauen Strahl, der aus seinem Zauberstab schoss, auf Elber. Dieser reagierte sofort und schleuderte einen gelben zurück, sodass sich beide Zauber in der Mitte trafen.

Beide Kontrahenten liefen jetzt langsam sich umkreisend umeinander herum. Die Zauberstäbe immer noch verbunden.

Jetzt kamen Harry wieder die Erinnerungen auf, wie er und Lord Voldemort sich auf dem Friedhof duellierten.

\begin{rueckblick}
Er kauerte sich auf dem Friedhof hinter einem Grabstein, als Voldemort ihn einen Feigling nannte. Nein, das konnte er nicht auf sich sitzen lassen. Also stand er auf, um sich ihm zu stellen. Er musste nur schneller sein als er. Ihn entwaffnen und dann irgendwie fliehen. Also stand er auf und stellte sich ihm. Seinem Feind. Seiner Nemesis. Er erinnerte sich daran, wie Voldemort ihn alleine töten wollte und niemand eingreifen sollte. Dann sprach Voldemort den Tötungsfluch und Harry den Entwaffnungszauber. Die beiden Zauberstäbe wurden miteinander verbunden und ein Schild aus Licht baute sich um ihnen herum auf. Gesänge des Phönix’ erklangen und nach und nach erschienen Cedric, der alte Mann und seine Eltern in der Mitte, wo sich beide Zauber trafen. Seine Eltern sagten ihm, was er zu tun hatte und Cedric bat ihn, seinen toten Körper mitzunehmen. Als er den Zauber brach, sprang er zu Cedric, holte den Portschlüssel herbei und verschwand mit ihm.
\end{rueckblick}

Elber und Dumbledore liefen noch immer langsam im Kreis umeinander herum. Die Zauberstäbe immer noch verbunden. Dann, mit einer kurzen Bewegung seines Handgelenkes, zog Elber einen schmalen Strahl ab und lenkte ihn auf Dumbledore. Der Strahl traf ihn am Arm und ließ ihn zusammenzucken. Professor Elber versuchte es ein paar Mal hintereinander, doch die weiteren Male wurde er von Dumbledore aufgefangen oder mit der Hand abgelenkt. Das erzeugte Erstaunen in der umstehenden Gruppe. Elber brach den Zauber, indem er Dumbledores Strahl gegen das sie umgebende Kraftfeld lenkte. Dumbledore reagierte sofort und entwaffnete Professor Elber. Dieser konnte seinen Zauberstab nicht mehr halten und er flog in hohem Bogen in Dumbledores Hand. Professor Elber griff nach seinem Zauberstab, doch Dumbledore wehrte den Sog mit einem Schildzauber ab.

Elber stand nun ohne Zauberstab da. Harry ließ für einen Moment seinen Blick schweifen, um die Stimmung in der Kammer aufzunehmen. Er sah in bestürzte Gesichter und hörte Getuschel. \enquote{Jetzt ist er fällig. Er hat keine Waffe mehr.}

Elber hielt jetzt eine Handfläche nach oben und eine kleine Flamme erschien schwebend über seiner Hand. Sie wurde leicht größer und rötlicher. Dann schob er sie mit seiner anderen Hand weg, worauf sie herunterfiel und zu einer Schlange wurde, die nur aus Feuer zu bestehen schien. Die Schlange kam auf Dumbledore zu und wurde beständig größer. Dumbledore traf ein paar Schritte zurück, wohl um sich Zeit zu verschaffen. Er schleuderte ein paar Zauber entgegen, doch es half nichts. Die Schlange kam weiterhin auf ihn zu. Dann entsprang seinem Zauberstab ein dünner Wasserstrahl, mit dem er der Schlange den Kopf abtrennte.

Der Kopf fiel ab und löste sich auf. Der Rest der Schlange zuckte kurz zusammen, aber dann wuchsen an der Stelle, wo Dumbledore den Kopf abgeschlagen hatte zwei Köpfe heraus. Die Schlange kam immer näher.

Elber schob seine Hand nach vorne und kurz darauf wurde Dumbledore nach hinten gedrückt und fiel zu Boden. Die Schlange war nur noch wenige Zentimeter von ihm entfernt. Dann hüllte Dumbledore die Schlange vollkommen mit Wasser ein. Die Schlange erstarrte kurz, bewegte sich dann im Inneren der mittlerweile zu einer Kugel geformten Wassermasse und versuchte zu entkommen. Aber sie lebte immer noch. Dann trennte Dumbledore einen Teil des Wassers ab und schleuderte ihn auf Elber zu. Dieser war darüber so überrascht, dass er die Wassermasse gar nicht mehr abwehren konnte und so patschnass dastand. Die Schlange löste sich augenblicklich auf und Dumbledore ließ die Wasserkugel herunterfallen.

Elber griff nach seinem Zauberstab und er entwich Dumbledores Griff. Nun hatten beide Kontrahenten wieder ihre Zauberstäbe. \enquote{Machen wir weiter, oder darf ich mir erst etwas Trockenes anziehen?}, fragte Elber.

Dumbledore schnaufte \enquote{Beenden wir’s.} Dann schwang er seinen Zauberstab und das Kraftfeld löste sich auf.

Auf dem Rückweg konnte Harry Teile der Unterhaltung seines Direktors und seines Lehrers mit anhören. \enquote{Warum sind Sie zusammengebrochen, Frederick?}

\enquote{Das wissen Sie}, antwortete Professor Elber. \enquote{Voldemort hat mir eine Falle gestellt. Er hat jemanden benutzt, dem ich dahingehend vertraue. Der Person mache ich keinen Vorwurf. Nur Voldemort. Außerdem war ich geschwächt.}

\trenn

Harry stand in einem der Innenhöfe Hogwarts und wartete auf den Beginn der Stunde.

Katharina kam kurz vorbei und sagte ihm, dass es geklappt hat. \enquote{Alle sind wieder gesund. Madame Pomfrey wundert sich noch immer darüber, dass es geklappt hat.} Dann winkte sie ihm zu und verschwand zur nächsten Stunde.

Harry hatte es bereits vermutet, da Professor Dumbledore bereits wieder normal war.

Professor Elber kam mit einer Menge an Besen, die ihm schwebend folgten, um die Ecke. Sie schwebten hinter ihm und folgten den Bewegungen seines Zauberstabes. Er dirigierte sie an eine Wand und stellte sie dort schräg ab. Mit dem Schweif nach unten. \enquote{Bitte stellen Sie sich gegenüber eines Besens auf}, orderte Professor Elber seine Schüler an. Harry fragte sich wie seine anderen Mitschüler, was das mit dem Unterricht in \VgddK zu tun hat.

\enquote{Sie werden sich jetzt sicher fragen, weshalb ich Ihnen in \VgddK Besen hinstelle.} Fast alle Schüler nickten und Professor Elber grinste kaum merkbar. \enquote{Lassen Sie mich es so ausdrücken, Sie werden es am Ende der Stunde erkennen. Holen Sie nun alle ihre Besen zu sich.} Einige Schüler begannen bereits auf ihre Besen zuzulaufen. \enquote{Halt}, schrie, Professor Elber \enquote{ich sagte nichts davon, dass Sie auf Ihre Besen zulaufen sollen. Zurück auf Ihre Ausgangspositionen.} Die Schüler, welche bereits vorgelaufen waren, traten schüchtern zurück und gingen wieder auf ihre Plätze. \enquote{Nutzen Sie die Magie in Ihnen, um die Besen zu sich zu holen. Denken Sie, dass Ihnen der Besen entgegenkommt und strecken sie Ihre Hand aus, um ihn zu empfangen.}

Die Schüler fingen an zu üben und ihre Besen schwebten langsam aber zielsicher zu ihnen hin. Nach einigen Minuten kam Professor McGonagall auf Professor Elber zu und zog ihn etwas zur Seite. Harry konnte ihre Unterhaltung kaum hören.

\enquote{Was?}, meinte Professor Elber. \enquote{Ich kann doch nicht \gst}

Professor McGonagall unterbrach ihn. \enquote{Doch, Frederick.} Sie zog ihn noch ein Stück weiter weg. \enquote{Alle anderen sind Krank. Und die Leute im Ministerium \gst}

Doch Harry konnte nichts mehr hören, sie sprachen nun sehr leise und waren zudem in sicherer Entfernung. Er konnte aus Professor Elbers Gesicht nur leichten Missmut erkennen. Er wandte sich wieder seinem Besen zu und ließ ihn, nachdem er ihn in seiner Hand gehalten hatte, wieder zurück schweben. Er war so auf seine Aufgabe konzentriert, dass er gar nicht merkte, dass sich Professor Elber hinter ihn gestellt hatte und ihm interessiert zusah.

\enquote{Zu faul zum Laufen, Harry?}, fragte er ihn.

Harry erschrak. Sein Besen zitterte gefährlich in der Luft. Nur mit etwas Mühe konnte er ihn still halten. Er drehte sich um und meinte schuldbewusst, \enquote{Ja.}

\enquote{Na na na, kein Grund Schuld zu zeigen. Sie haben nur mitgedacht und mir den nächsten Schritt schon voraus genommen.} Jetzt sagte er zur Klasse gewandt: \enquote{Wenn Sie das Herholen Ihres Besens beherrschen, dann schicken Sie ihn wieder zurück an die Wand.}

Die meisten Schüler taten dies dann auch, ohne besonders zu reagieren.

\trenn

Aus einer Tür hörte Harry wie sich jemanden unterhielt. Vorsichtig kam er näher.

\enquote{Ahh! Es brennt, der Dunkle Lord ruft die Todesser.}

\enquote{Shh Draco. Komm und setz dich.}

Er hörte Schritte. Er kannte die Stimmen genau.

Dann hörte er einen Singsang, den er nicht deuten konnte. \zauber{Sema nibina, sema duljana. Sema nibina, sema duljana. Sema nibina, sema duljana.}

Harry versuchte durch ein Schlüsselloch etwas zu erhaschen, aber er fand keines.

\enquote{Besser, Draco?}, fragte die Stimme weiter.

\enquote{Es kribbelt nur noch leicht.}

\enquote{Gut \gst Warum hast du es dir überhaupt geben lassen?}

\enquote{Mein Vater. Ich hatte keine Wahl. Er hätte mich sonst umgebracht.}

\enquote{Möchtest du es denn?}

\enquote{Nein, aber ich bekomme es wohl nicht los.}

\enquote{Sicher?}

\enquote{Es wurde magisch eingebrannt, Frederick. Man kann es nicht entfernen.}

\enquote{Aber du möchtest es loshaben?}

\enquote{Sicher.}

\enquote{Du möchtest also im Sommer mit kurzer Kleidung herumlaufen, ohne dass du ständig das Mal schminken, oder sonst überdecken musst.}

\enquote{Ja.}

\enquote{Dann sag es. Schau auf das Mal und sage ihm, dass du es loshaben möchtest. \gst Dann werde ich versuchen, dir zu helfen. \gst Aber ich muss dir auch sagen, es brennt, wenn man es entfernt. Zwar nicht so stark, wie zu dem Zeitpunkt an dem man es bekam, aber es schmerzt doch. Es versucht sich zu wehren. Das ist mächtige schwarze Magie. Es strahlt eine Signatur aus, die man spüren kann.}

Dann herrschte eine Weile Stille. Harry stand neben der Tür und konnte die Unterhaltung klar und deutlich, aber dennoch durch die Tür gedämpft wahrnehmen. \gedanke{Warum? Die Türen sind doch aus dickem Holz, die Steinwände schlucken den Schall? Warum?}

\enquote{Dumbledore. Kann er es auch spüren?}

\enquote{Ich denke schon. Immer, wenn du in seiner Nähe bist.}

Draco dachte nach. \gedanke{Er hat nie etwas gesagt. Hat mich nicht darauf angesprochen. Warum nur?}, dachte er. \enquote{Wieso hat er mir deswegen keine Vorhaltungen gemacht?}, fragte er leise in den Raum hinein.

\enquote{Weil er immer an dich geglaubt hat und es immer noch tut. Er sieht deinen guten Kern. \gst Wollen wir beginnen?}

Dann herrschte eine Weile Stille. Harry stellte sich vor, wie Draco durchatmete und dann nickte.

\zauberextase{Erdrom Srom!}, hörte er plötzlich und erschrak, denn er hörte Draco schreien. Doch dieser beruhigte sich schließlich recht schnell. Harry verkrampfte, als ob er die Schmerzen selber spüren würde. Dann hörte er wieder diesen Sing sang. \zauber{Sema nibina, sema duljana. Sema nibina, sema duljana. Sema nibina, sema duljana.}

Er umfasste instinktiv sein Amulett und nahm nun verschwommen eine Szene wahr. Es war wie in einem Traum. Sein Lehrer und Draco saßen auf einer Couch und Elber hielt die Spitze seines Zauberstabes auf Dracos Arm. Das dunkle Mal verblasste.

Professor Elber hob seinen Kopf und es hatte den Anschein, dass er Harry direkt in die Augen sah. Danach schaute er wieder auf Dracos Arm. \zauber{Sema nibina, sema duljana.}

Harry ließ sein Amulett los und rannte lautlos um die nächste Ecke, um dann durch das Schloss zu rennen. Seine Schritte hallten an den Wänden wieder. Er wollte nur noch weg.

Etwas später unterhielt er sich gerade mit Luna auf dem Weg in die Große Halle, als ihm plötzlich schlecht wurde und sich sein Sichtfeld zunehmend schmälerte. Er fühlte sich so, als ob er jeden Moment zusammenbrechen würde. Dann wurde es schwarz um ihn und er sank auf dem Boden zusammen. Harry bekam es nicht mit, aber Luna brach ebenfalls zusammen und beide wurden auf die Krankenstation gebracht.

Als Harry das Bewusstsein wieder erlangte, standen Professor Dumbledore und Professor McGonagall um das Krankenbett neben ihm herum und an seinem Krankenbett standen Professor Flitwick und Madame Pomfrey.

\enquote{Und, Harry, wie geht es dir?}, fragte Professor Dumbledore die Person neben ihm.

\enquote{Mir geht es gut}, antwortete Harry. \enquote{Aber ich bin hier.}

Professor Dumbledore und Professor McGonagall drehten sich um und sahen ihn an. \enquote{Ah Miss Lovegood}, antwortete Professor McGonagall, \enquote{schön Sie wieder wach zu sehen.}

Harry Augen weiteten sich. \enquote{Aber ich bin Harry, Harry Potter. Erkennen sie mich nicht mehr?}

Plötzlich hörte er aus dem Nebenbett eine Stimme, eine Stimme, die ihm bekannt vorkam, die er aber nicht zuordnen konnte. \enquote{Das habe ich denen auch schon gesagt, aber mir glaubt natürlich nie einer was.}

Harry stutze, hob seinen Oberkörper und blickte nun der Person neben ihm ins Gesicht. Seine Kinnlade fiel herunter, als er in das fremde Gesicht sah. \enquote{Luna? Bist du das?}, fragte er mit einer Stimme die, wie ihm plötzlich auffiel, anders klang als sonst. Harry sah sich ins Gesicht. Er sah seinen Körper, aber er war nicht er selbst. Er blickte sich in der Krankenstation um und bemerkte einen kleinen Spiegel auf dem Krankentisch neben ihm. Er nahm ihn und schaute sich an. Sein Spiegelbild zeigte Luna.

\enquote{Ich habe denen schon gesagt, dass ich nicht Harry bin, aber sie sagten mir, ich sei verwirrt}, sagte Luna (in Harrys Körper).

Harry wusste nicht, wie ihm geschah, irgendwie hatte er den Körper mit Luna getauscht.

\enquote{Professor Dumbledore, anscheinend haben Luna und ich unsere Körper getauscht.}

Madame Pomfrey zuckte zusammen.

\enquote{Wie ist das möglich?}, fragte Luna.

Dumbledore antwortete, \enquote{Tja, Luna, ich habe keine Ahnung.} Dann zu Madame Pomfrey gewandt. \enquote{Können Sie sich einen Reim darauf machen Poppy?}

\enquote{Nein}, antwortete sie knapp. \enquote{Wie ist das passiert?}

\enquote{Ich weiß es nicht}, sagten Harry und Luna.

\enquote{Tja, bis wir herausgefunden haben, was mit Ihnen los ist, werden Sie sich daran wohl gewöhnen müssen}, sagte Madame Pomfrey.

Harry musste ein Lachen unterdrücken, da er bereits mehrmals in Luna war.

\enquote{Oh, daran haben wir uns schon gewöhnt}, meinte Luna. Harry drehte reflexartig seinen Kopf zu ihr (in seinem Körper) und auch die anderen Professoren und Madame Pomfrey ebenso. \enquote{Äh, ich meinte, daran werden wir uns schon gewöhnen}, schob Luna schnell hinterher, ebenfalls ein Lachen unterdrückend.

Madame Pomfrey hob eine Augenbraue. \enquote{Sie beide bleiben heute auf jeden Fall hier, da Sie eine leichte Gehirnerschütterung haben. Morgen früh können Sie dann gehen.}




\begin{kommentar}
Katharina wird von Hagrid in den Verbotenen Wald geschickt. Dort trifft sie auf Medusoner die aussehen wie - Schlümpfe.
\end{kommentar}

\begin{kommentar}
Nachdem Katharina zum zweiten Mal ihre Prüfung in Griechenland abgelegt hat, erinnert sie Harry daran, dass die Magie ihr Verbündeter ist. Und ein mächtiger Verbündeter ist sie.
Dieselben Worte hatte auch Yoda in Krieg der Sterne zu Luke gesagt. Eine weitere nette Anspielung.
\end{kommentar}

\begin{kommentar}
Etwas später sind Draco und Frederick Elber in einem kleinen Raum und Draco will das Dunkle Mal loswerden. Frederick benutzt einen Zauber. Schaut mal nach welchen und lest diesen dann rückwärts.
\end{kommentar}
