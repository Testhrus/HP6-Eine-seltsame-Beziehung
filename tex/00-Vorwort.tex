\chapter{Vorwort}

Diese Geschichte wurde im November 2024 überarbeitet.
Rechtschreibfehler wurden korrigiert und ein paar kleine Logikfehler ebenso.

Über mein Profil findet ihr auch einen Link zu PDFs dieser Geschichte.


\accentuate{Zum Inhalt:}

Die Geschichte spielt in Harrys sechstem Jahr auf Hogwarts und wird einen anderen Verlauf nehmen, als die Bücher. Daher die Warnung an alle: Lest nicht weiter, wenn ihr eine Geschichte sucht, die sich an den Büchern orientiert und nur ergänzend ist.

Harry hat zum ersten Mal einen schönen Geburtstag. Kurz darauf beginnt auch schon das neue Schuljahr, in dem er recht bald, dass er und Luna sich zueinander hingezogen fühlen. Als ob das nicht schon Probleme genug wären, fühlen sich plötzlich alle Mädchen in Hogwarts zu ihm hingezogen. Dieser unerklärliche Zustand sorgt für ziemlich viel Wirbel. Währenddessen scheint der neue VgddK Lehrer Geheimnisse zu haben. Man munkelt sogar, dass er ein Todesser da er mit den dunklen Künsten normal umgeht als wären sie nichts Besonderes. Doch trügt hier das Bauchgefühl unserer Schüler und Lehrer? Oder ist es wirklich so? In dem ganzen Rummel erkennt Harry, dass Magie vielleicht mehr ist, als plumpes Zauberstabgefuchtel. Schließlich erfährt er noch etwas über seine Vorfahren, was ihn ziemlich schockt.


\accentuate{Zur Geschichte:}

Diese hat 33 Kapitel. Sie ist über einen Zeitraum von fünf Jahren entstanden, in denen ich immer wieder mal pausiert habe. Sie wurde vorher noch nicht veröffentlicht. 2008 habe ich damit angefangen.

Geschichten über Harrys siebtes Schuljahr sowie die ersten Jahre danach sind bereits veröffentlicht worden. 
\accentuate{Das dunkle Ende}
\accentuate{Neue Herusforderungen}
Diese könnt ihr über mein Profil finden.

Vor allem lohnt es sich, auch mal die Reviews der anderen zu Lesen, da ich dort eventuell aufkommende Fragen beantworten werde. Es kann durchaus sein, dass ich Stellen habe, bei denen der Leser eine Wahl zwischen mehreren Lösungen hat. Sollte eine(r) dennoch nicht darauf kommen, werde ich die entsprechende Frage dazu im Review oder per PM beantworten. Es wird daher keinerlei Anmerkungen oder Danksagungen am Anfang, oder am Ende eines Kapitels geben.

Noch eins, bevor es losgeht. In der deutschen Ausgabe der Bücher ist öfter von Snape explodiert zu lesen, ein Spiel. Das englische Original sagt dazu exploding snap, also Spreng-Schnipp-Schnapp. Nur für den Fall, falls sich einer wundert, was das ist, denn es kommt in der Geschichte vor.

Und nun, viel Spaß beim Lesen.
