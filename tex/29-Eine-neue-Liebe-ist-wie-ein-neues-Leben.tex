\chapter{Eine neue Liebe ist wie ein neues Leben}


Diesen Abend und die darauf folgenden verbrachte er mit Lavender in einem abgelegenen Klassenzimmer. Die Hauselfen hatten ein gemütliches Bett und eine romantisch aussehende Umgebung hergerichtet. Harry hätte sich am liebsten nur ein Bett gewünscht und keine romantische Zusammenkunft. Er wollte nicht, wenn er an seine Zeit mit Lavender dachte, an diese Umgebung erinnert werden. Da es aber zwischen ihm und Lavender rein sexuell war, konnte er sich schließlich damit arrangieren.

Den anderen Schülern hatte Snape gesagt, dass sie Strafarbeiten zu leisten hätten. Bei Lavender konnte man das verstehen, so schusselig wie sie war. Und bei Harry hatte man das Gefühl, er müsste etwas ausgefressen haben und Snape behandelte ihn nur etwas härter als alle anderen.

Sie hatten bereits sechs Nächte miteinander verbracht, in denen sie jedes Mal dreimal miteinander geschlafen hatten. Und auch diesen Morgen war er alleine aufgewacht, sie war bereits gegangen. Er fühlte sich irgendwie leicht und unbeschwert. Eigentlich dachte er gar nicht mehr daran mit ihr schlafen zu wollen, aber er wollte die Woche voll bekommen und so wollte er diesen Abend auch wieder in das Klassenzimmer gehen. Außerdem wusste er nicht, ob der Anziehungseffekt noch anhielt, doch zuerst musste er noch mit Fleur und Gabrielle etwas klären.

Sie trafen sich in einem leeren, nicht gebrauchten Klassenzimmer. Fleur und Gabrielle saßen Harry gegenüber \gst ziemlich nahe. Fleur musste ein paar mal durchatmen, bevor sie sich vorbeugte und zu Harry flüsterte: \enquote{Danke 'Arry, dass du meine\abs} Sie öffnete ihre Bluse und zog sie aus. Auch ihre Schwester konnte nicht widerstehen und tat es ihrer großen Schwester gleich. Fleur stand auf und ging zum Fenster. Sie öffnete es und atmete leicht durch. \enquote{Es ist schwer 'Arry.} Er sah zu ihr. \enquote{Ich fühle mich zu dir 'ingezogen.}

Harry bemerkte nicht, dass sich Gabrielle währenddessen auf seinen Schoß gesetzt hatte.

Nur mit ihrem BH auf ihrem Oberkörper bekleidet, saß sie auf ihm und hatte beide Hände auf seiner Brust. Er war noch völlig bekleidet. \enquote{Ich\abs ich 'ätte da un'n ster'bn 'önnen, 'Arry}, sagte Gabrielle. Gedankenverloren spielte sie an seiner Schulrobe herum und öffnete seine Hemdknöpfe. \enquote{Du 'ast mein Le'bn gerett'.} Nun lagen ihre Hände auf seiner Brust. Mit ihren Fingerspitzen kraulte sie leicht seine Brust.

Harry schloss kurz seine Augen, da ihn eine Welle der Behaglichkeit durchlief.

\begin{abAchtzehn}

Fleur kam auf beide zu und stellte sich neben Harry. Ein Bein schwang sie um die Rückenlehne, um hinter ihm zu sitzen. Als Harry das bemerkte, rutschte er etwas vor und ließ sich nach hinten fallen, als sie saß. Er spürte ihre Brüste durch den dünnen BH auf seiner nackten Haut, da er inzwischen seine Robe und sein Hemd ausgezogen bekommen hatte. Er wusste selber nicht, warum er das zuließ.

Gabrielle kam seinem Gesicht näher, sodass sich ihre Nasen berührten. \enquote{Schwer unter Kontrolle\abs}, keuchte Gabrielle. \enquote{Will disch 'aben}, sagte sie mit zittriger Stimme.

Dann hörten die drei einen Knall und ihre Konzentration ließ plötzlich nach. 
\end{abAchtzehn}

%\begin{safedivide}
%\fskdivider
%\end{safedivide}

Dann geschah alles ganz schnell. Fleur verwandelte den Stuhl in einen bequemen Sessel, der nach unten nachgab und sich in eine dicke weiche Decke verwandelte. Fleur lag nun unten, Harry auf ihr und Gabrielle ganz oben. Sie fing an, ihn stürmisch zu küssen, als ihre Schwester sie harsch unterbrach. \enquote{Gabrielle, beherrsche disch.}

\begin{abAchtzehn}

Schuldbewusst zuckte sie zusammen, nickte und ließ sich seitlich von Harry heruntergleiten. Sie nahm ihn etwas mit, was Fleur die Gelegenheit gab, sich seitlich unter ihm herauszuziehen und um sich dann halb auf ihn zu legen. Nun lagen beide Mädchen halb auf ihm. Sie begannen synchron mit seinen Ohren und seinem Hals zu spielen. Sie fuhren mit ihren Zungen die Konturen seiner Ohren nach und bedeckten anschließend seinen Hals mit vielen Küssen.

\end{abAchtzehn}

%\begin{safedivide}
%\fskdivider
%\end{safedivide}

Harry ließ sich in diese Behandlung hineinfallen und verdrängte unbewusst alle anderen Gedanken. Mit geschlossenen Augen ließ er es sich gut gehen. Dann machte sich in seinem inneren ein Gefühl breit, das ihm sagte: \accentuate{Gib nach, dann hast du wieder alles unter Kontrolle.} Er gab diesem Gefühl nach und verdrängte den Rest seiner Gedanken aktiv mit Okklumentik. Immer mehr wurde ihm bewusst, dass er damit seine Gefühle komplett unter Kontrolle bringen, ja sie sogar unterdrücken konnte. Eine knappe Minute später hatte er es geschafft.

Fleur und ihre Schwester hörten abrupt mit ihren Liebkosungen auf, was Harry ein leises Seufzen entlockte. Er öffnet die Augen und hielt Fleur und Gabrielle zurück, indem er sie mit einem Arm um die Hüfte festhielt.

\enquote{Nicht! Sonst wird es peinlich. Wir haben uns lange genug so gesehen und müssen uns nicht schämen. Bleibt einfach liegen und sagt mir, was ihr loswerden wollt.}

Verlegen spielte Gabrielle mit seiner linken Brustwarze, was bei Harry einen leichten Kitzelreiz auslöste. Seine Gefühle hatte er jedoch immer noch unter Kontrolle.

\begin{abAchtzehn}
In seinem Geist hörte er Fleurs Stimme. \stimme{Wie sage ich es ihm, dass ich bis gerade eben mit ihm schlafen wollte, um ihm dafür zu danken, dass er meine Schwester gerettet hatte.} Sie blickte ihn nachdenklich an. \stimme{Eigentlich wollte ich ihm ein besonderes Geschenk machen, indem ich ihm etwas von meinem Veela-Erbe gebe, damit er sich durch Veelas nicht mehr so beeinflusst sieht, aber das war mir gerade eben noch viel zu wenig.}

Harry schaute sie überrascht an. \gedanke{Hast du mir gerade eben deine Gedanken übermittelt, Fleur?}, fragte er sich und konzentrierte sich auf beide Mädchen, da er aus seinem Augenwinkel heraus feststellte, dass ihn Gabrielle auch schon komisch ansah. Fleur wurde augenblicklich rot. Sie versuchte sich loszustrampeln, doch Harry hielt sie fest. Als sie schließlich aufgab, ließ sie sich fallen und vergrub ihren Kopf in seiner Schulter. Sanft strich er über ihr Haar. \gedanke{Scht, Fleur. Alles ist gut.} Und dann, nach einer kleinen Pause: \gedanke{Was meintest du mit: Etwas von meinem Veela-Erbe geben?}
\end{abAchtzehn}

%\begin{safedivide}
%\fskdivider
%\end{safedivide}

Dann hörte er Gabriellas Stimme in seinem Kopf. Klar und deutlich und vollkommen akzentfrei. \stimme{Weißt du, Harry}, er wandte seinen Kopf zu ihr, \stimme{Veelas können jemandem ein besonderes Geschenk machen. Sie geben einen kleinen Teil ihres Veela-Erbes an jemanden, den sie sehr mögen, um ihn vor ihrer \accentuate{Art} zu schützen. Je mehr Veelas das machen, desto besser wird der Schutz.}

Harry zog eine Augenbraue hoch, was Gabrielle veranlasst, leicht zu erröten. Jetzt hörte er wieder Fleur. Harrys Kopf drehte sich, um ihr ins Gesicht zu schauen.

Immer wieder fuhr sie dabei mit ihren Fingern über seine Lippen. \stimme{Es gibt eine spezielle Technik, die Veelas ab Geburt beherrschen. Sie nimmt die Essenz ihres Gegenübers auf und gibt gleichzeitig einen Teil davon an die zu schützende Person weiter.}

\gedanke{Was meinst du mit \accentuate{Essenz?}}

Fleur wurde wieder rot. \stimme{Wie teile ich ihm das mit, dass wir die Flüssigkeiten des anderen aufnehmen müssen, die sie bei einer Erregung bekommen. Am besten mit so viel Körperkontakt wie möglich.} \enquote{Harry}, druckste sie herum. Er zeigte keine Regung. \enquote{Lässt du dich von mir leiten? \gst Ich verspreche dir, es tut dir nicht weh.} Harry nickte nur. \enquote{Vertraust du mir?} Wieder nickte er.

Fleur stand auf und entkleidete sich komplett. Währenddessen öffnete Gabrielle Harrys Hose und zog sie herunter. Dann zog Fleur seine Unterhose herunter. Durch die Nähe des anderen und die sexuelle Spannung waren beide immer noch erregt. Fleur ging auf Harry zu und versuchte so viel Körperkontakt wie möglich herzustellen, indem sie ihn umarmte, was Harry ebenfalls machte, allerdings mit Fleur. Nachdem sie ein paar Minuten so da gestanden waren, zog Fleur mit einer Hand Harrys Vorhaut zurück und fuhr über seine Eichel. Danach leckte sie ihren feuchten Finger ab. Dann sah sie Harry bittend an. Dieser fuhr nach kurzem Zögern Fleur zwischen ihren Beinen und nahm einen Tropfen ihrer Flüssigkeit auf seinen Finger auf, den er danach ableckte. Sie standen nun wieder auf Distanz.

Dann ging ein blaues Leuchten von Fleur aus, wandelte sich in ein cyanfarbenes und begann auf Harry zuzuwandern. Auf Harry angekommen, wandelte es sich in ein goldenes Leuchten um und erhellte den Raum. Dann wurde das Leuchten von ihm aufgesogen und verschwand in Harrys Innerem.

Fleur nickte jetzt Gabrielle zu, was diese ebenfalls mit einem Nicken zurückgab. Harry sah unbehaglich zu Gabrielle. Sie stand auf und reichte Harry die Hand. Unsicher gab er sie ihr und zog sich leicht an ihrer Hand hoch.

\enquote{Ich möchte dir auch danken.} Panik machte sich in Harry breit. Gabrielle war noch extrem jung und Harry wusste nicht, wie es ablaufen sollte. Er sah einmal an ihr entlang hinunter, sie stand noch in Unterwäsche vor ihm. \enquote{Nicht so wie Fleur}, sagte sie, griff sich zwischen ihre Beine und zog ihren Finger, nachdem er nass genug war, wieder zurück und streckte ihn Harry entgegen. Dieser leckte ihn einmal ab. \enquote{Jetzt du 'Arry}, sagte sie und Harry nahm einen Finger und fuhr einmal über seine Eichel. Er hielt einen Tropfen seiner Flüssigkeit an seinem Finger und ließ ihn Gabrielle ablecken. Nun ging ein grünlicher Schimmer von ihr aus, der sich auf Harry zu bewegte und dann zu einem goldenen Ton wandelte. Dieses Mal jedoch dunkler.

Nachdem Harrys Leuchten nachgelassen hatte, fingen Fleur und Gabrielle an, sich anzuziehen.

Harry tat es ihnen gleich. \enquote{Was ist mit Ginny?}, fragte er plötzlich.

\enquote{Deine Freundin?}, fragte Fleur.

Harry war sich unsicher, was er sagen sollte, und nickte daher einfach nur.

Fleur hielt sich eine Hand vor den Mund und machte große Augen.

\enquote{Du musst sie inner'alb eines Tages k'üssen, 'Arry, sonst wird sie ausrasten und es bemerken, das ist ein großer Nachteil, der einzige.} Harrys sah nicht gerade begeistert aus.

Angezogen kam Gabrielle auf ihn zu und umarmte ihn. Sie legte ihren Kopf auf seiner Brust ab und meinte dann: \enquote{'Arry? Isch 'abe mittlerweile das Gefuhl, einen Bruder gefund' zu 'aben.}

Fleur kam ebenfalls auf Harry zu und umarmte ihn. \enquote{Mir geht es ebenso, 'Arry}, sagte sie. \enquote{Ich 'abe das Gefühl, einen kleinen Bruder zu 'aben.}

Harry dachte nach. \enquote{Ist das normal?}, fragte er nach einer Weile.

\enquote{Nein}, gab Fleur zurück. \enquote{Ich weiß nicht, was das auslöst.}

Dann hörte Harry eine Stimme: \stimme{Du hast das trimagische Turnier bestritten und beiden das Leben gerettet. Das hat eine unsichtbare Bindung geschaffen, die zusammen mit dem Veela-Erbe, das du erhalten hast, zu diesem Effekt geführt haben könnte.} Es war Salazar.

\enquote{Wir haben doch das trimagische Turnier zusammen bestritten}, sagte Harry. Fleur und Gabrielle nickte. \enquote{Ich habe euch beiden das Leben gerettet. Das könnte ein Band geschaffen haben, das mit dem Veela-Erbe, das ihr mir überlassen habt, zu diesem Effekt führt.}

\enquote{Ja, 'Arry. Du hast dort meine Schwester aus dem See gerettet und so\abs} Sie brach ab. \enquote{Wann 'ast du mir das Leben gerettet?}, fragte sie.

Harry sah sie an. Dann sagte er: \enquote{Als wir im Irrgarten waren. Du wurdest gerade unter die Hecke gezogen. Ich habe versucht, die Wurzeln von dir zu bekommen. Du warst betäubt. Also habe ich sie mit einem Zauber gekappt und rote Funken nach oben geschossen.}

Fleur bekam große Augen. \enquote{Und ich dachte erst, es wäre ein Sicherungszauber gewesen, bis mir Madame Pomfrey gesagt hat, dass ich wohl noch die Kraft gehabt hatte, dies selber zu tun. \gst Du warst das?}

Harry nickte, worauf ihm Fleur noch einen Kuss auf seine Stirn gab. \enquote{Danke, ich bin dir noch etwas schuldig.} Sie strich über ihr Haar und nahm eines davon heraus.

\enquote{Du bist mir nichts\abs}, doch Fleur legte einen Finger auf seine Lippen.

\enquote{Scht, 'Arry}, sagte sie, rollte das Haar zu einem kleinen Kreis auf, nahm seine Hand und legte das Haar in seine Handfläche. Dann drückte sie die Hand sanft zusammen und tippte mit ihrem Zauberstab einmal darauf.

Harrys Haar wurde kurz warm. Als er sie wieder öffnete, zeichnete sich eine feine, gerollte, rote Linie auf seiner Handfläche ab. Zunehmend verschwand die rote Färbung von seiner Hand.

\enquote{Ein kleines Dankeschön, 'Arry}, sagte sie.

\enquote{Was hast du mir gegeben?}, fragte er.

\enquote{Warte es ab. Du wirst es schon merken}, antwortete Fleur.

\enquote{Ich freue mich, einen großen Bruder auf Hogwarts zu 'aben}, sagte Gabrielle plötzlich und fügte in Gedanken hinzu: \stimme{wenn ich näckstes Jahr 'ier bin.}

Jetzt schaute Harry interessiert zu Gabrielle hinunter. Er ging leicht in die Hocke und hob sie hoch. Sofort schlang sie ihre Beine um seine Hüfte und legte ihre Arme um seinen Hals. So als wäre sie seine Schwester. Sie legte ihren Kopf in seine Halsbeuge und er hörte sie in seinem Geist.

\stimme{Danke, Harry. Das bedeutet mir sehr viel. Auch wenn es nur ein Jahr ist.}

\gedanke{Warum bist du überhaupt hier? Ist nicht Schule in Frankreich?}

\stimme{Wir haben ein paar Feiertage. Und da Fleur sich hier bei der Grundschule in Hogsmeade beworben hatte und heute ihr Vorstellungsgespräch hatte, wollte ich mit, damit ich mich bei dir bedanken kann.}

\enquote{Du willst nächstes Jahr mit Fleur hier in England bleiben?}

Fleur horchte auf. Dann sagte sie: \enquote{Das ist noch geheim Gabrielle.}

\enquote{Upps. \gst 'Arry wird nichts verrat'. Oder 'Arry?}

Harry lächelte sie an und schüttelte den Kopf. Dann gab er ihr einen Kuss auf die Stirn und sagte: \enquote{Nein, Schwesterchen.}

Das brachte Gabrielle dazu zu lachen und Harry einmal kurz, aber festzudrücken. Dann ließ sie von ihm ab und Harry ließ sie wieder nach unten.

Nachdem sie sich von ihm verabschiedet hatten, ging er in den Gemeinschaftsraum, wo Ginny in einem Sessel saß und mit ihren Mitschülerinnen sprach. Er ging zu ihr und sagte dann: \enquote{Ginny, kann ich dich kurz\abs ausleihen?}

Sie sah zu ihm und nickte, stand aber nicht auf, sondern wartete. Harry reichte ihr daher seine Hand, um sie hochzuziehen. Sie ergriff sie und Harry zog sie hoch. Erwartungsvoll sah sie ihn an. Harry musste schlucken. Dann legte er seine Arme um sie und zog sie zu sich heran. Bevor sie begriff was passierte, begann er sie zu küssen. Vor Schreck wich sie zurück, aber Harry zog nach. Doch nach einigen Sekunden schlang sie ihre Arme um ihn, küsste ihn zurück.

Als er wieder von ihr abließ, gab sie keuchend von sich: \enquote{Harry, was war das?}

Doch statt einer Antwort bekam sie einen weiteren Kuss. Dann sagte er: \enquote{Ich liebe dich.}

Sie bekam große Augen. Die Jubelrufe im Gemeinschaftsraum hörte sie nicht, da sie von seinen Taten und seiner Aussage überwältigt war. Er musste sie festhalten, da ihre Knie nachgaben.

\enquote{Ich muss leider heute noch meine Strafe absitzen, aber morgen haben wir alle Zeit der Welt.}

Durch diese Aussage aufgerüttelt, zog sie ihn in eine ruhige Ecke, drückte ihn in einen Sessel und setzte sich auf seinen Schoß.

\enquote{Ihr müsst miteinander schlafen?}, fragte sie leise in sein Ohr. Harry zuckte zusammen. \enquote{Es gibt Gerüchte, dass Lavender und du\abs na ja\abs sie hat den Zauber ausgelöst und ich habe etwas darüber gelesen. Dann dein Zustand. Und letzte Woche bist du ihr aus dem Weg gegangen. Dann habt ihr beide zeitgleich eure Strafe abzusitzen.} Harry antwortete ihr nicht. \enquote{Du kannst es mir sagen, Harry. Ich verkrafte das.}

\enquote{Ich will dich nicht anlügen, Ginny. Sagen wir so. Ab morgen bist du meine Freundin, da ich nicht vorhabe, dich zu betrügen.}

Ginny riss ihre Augen auf und wurde ganz rot. Sie wandte ihr Gesicht ab, was die anderen Gryffindors zu tuscheln oder zu Rufen veranlasste. \enquote{Küssen, küssen, küssen}, riefen sie.

Dadurch sich noch mehr schämend wendete sie ihr Gesicht abermals ab zu Harry, der die Chance ergriff und sie sofort küsste. \enquote{Hu, Harry}, schallte es wieder durch den Raum. Aber auch: \enquote{Ginny, schnapp ihn dir}, konnten sie hören. Wehmütig sah sie ihn an, nachdem sie den Kuss gelöst hatte.

\enquote{Amüsiere dich nicht zu sehr}, sagte sie.

\enquote{Keine Angst, meine liebe. Es ist nur noch heute Nacht, dann bin ich ganz dein.}

Ein strahlendes Lächeln kam über ihre Lippen und erreichte ihre Augen, die zu leuchten begannen.

\enquote{Was ist mit Pansy? Du hast öffentlich gesagt, dass du sie liebst.}

\enquote{Ja, das stimmt. Ich habe mich heute mit ihr getroffen und wir wollten\abs auf jeden Fall war plötzlich alles wieder weg. Ich hatte keine Gefühle mehr. Mitten im Kuss. Pansy schien es genauso zu gehen.}

\enquote{Nicht Parkinson?}

\enquote{Nein. Wir sollten mehr aufeinander zugehen. Damit meine ich die Häuser. Wir haben uns getrennt.}

\enquote{Dann bin ich deine Freundin?}

Harry überlegte kurz. \enquote{Wenn du es so siehst, ja.}

\enquote{Und wann wirst du mich verlassen?}

\enquote{Nie mehr, wenn ich ab Morgen\abs}

Ginny küsste ihn wieder. \enquote{Warum aber bei den anderen?}

\enquote{Na ja, bei Pansy dürfte doch mein Zustand mit Schuld dran sein. Sie schien nie wirklich etwas gegen mich zu haben. Nur durch Draco und unsere offene Feindschaft\abs} und leise, nur für Ginny fügte er hinzu: \enquote{Die sich zu etwas Neutralem gewandelt hat}, und dann wieder normal, \enquote{hat sie sich auf seine Seite gestellt, um die Haustreue zu wahren.}

\enquote{Und Luna?}

\enquote{Luna ist etwas Besonderes. Versteh mich nicht falsch. Ich hab sie immer noch gerne. Zwischen uns war das etwas, was man nicht beschreiben kann. Es ist eine Form der Liebe die\abs nicht romantisch, nicht schwesterlich und nicht familiär ist. Die Art unserer Verbindung ist viel mehr als das. Es scheint so, als ob wir bei Gefahr wie ein Wesen agieren könnten.}

\fluestern{Wie ein Wesen?} Ginny flüsterte jetzt nur noch.

Harry flüsterte zurück. \fluestern{Ja. Ich habe das bisher noch keinem gesagt, aber wir haben die Fähigkeit, uns in den anderen hineinzuversetzen.}

\fluestern{Eure Körpertausch-Aktion?}

\fluestern{Ja. Wir können uns außerdem in den Körper des anderen hineinversetzen, oder ihn beobachten. Erinnerst du dich an die Schachspiel-Runde mit Ron?} Ginny nickte. \fluestern{Da hat Luna zwei Runden mit ihm gespielt. Und eine habe ich mit Luna in meinem Geist gespielt \gst und gewonnen.} Dann küsste er sie wieder. \fluestern{Es gibt noch etwas, was du wissen solltest. Ich gehe ab und an mit Luna schwimmen.}

\fluestern{Da ist doch nichts dabei.}

\fluestern{Nackt.}

Ginny schaute ihn eine Weile lang an. \fluestern{Luna ist wohl die einzige Person neben Hermine, der ich das gestatten würde, mit dir nackt im See zu schwimmen.} Dann schaute sie ihn erschrocken an.

Harry küsste sie wieder. \fluestern{Danke}, war das Einzige, was er dazu sagte. Somit nahm er ihr die Möglichkeit, zurückzurudern und das zuletzt ausgesprochene zu relativieren.

\trenn

Lavender würde wieder auf ihn warten. Er öffnete die Tür, und da lag sie. Im selben dünnen Nachthemd, das er die vergangenen Tage schneller von ihr gerissen hatte, als er Quidditch sagen konnte. Doch heute war ihr Blick nicht so lasziv und scharf, heute war er anders. Harry sah darüber hinweg, denn er wollte ihr gegenüber nicht zugeben, dass sich etwas in ihm geändert hatte. Sie kletterte aus dem Bett und zog ihr Nachthemd aus. Harry zog seinen Morgenmantel aus. Er hatte nichts darunter. Sofort schoss das Blut durch seinen Körper und bahnte sich den Weg zu seinen Lenden.

Kurz darauf lag sie auch schon auf ihm und zog ihn zu einem langen Kuss heran. Sie ließ sich nur durchs Atmen und Reden unterbrechen. \enquote{Harry\abs du\abs hast\abs mir\abs gefehlt}, keuchte sie zwischen jedem Kuss hervor.

Noch nie hatte er ihre Lippen und Küsse als so intensiv empfunden wie heute. Vielleicht lag es auch nur daran, dass es heute ihre letzte gemeinsame Nacht sein würde. Oder es waren Nachwirkungen von Fleurs \accentuate{Behandlung}, dass sie so scharf auf ihn zu sein schien? 

\begin{abAchtzehn}
Sie richtete sich auf und ließ ihn in sich gleiten. Langsam bewegte sie ihr Becken auf und ab, lies es kreisen und wippte nach vorne und wieder zurück. Er spürte ihre Zuckungen und Muskelbewegungen und als sie sich nach vorne fallen ließ um seine Lippen zu empfangen, kam sie. Und Harry in ihr. Als sich beide beruhigt hatten und wieder zu Kräften kamen, drehte sie sich um und lag nun auf dem Rücken. Er steckte noch immer in ihr. Es brauchte eine Weile, bis er sie wieder ausfüllte, aber sie wollte ihn noch nicht verlieren. Langsam glitt er unter vielen Küssen nach unten. Zuerst um ihre Brüste mit seiner Zunge zu umspielen und mit seinem Mund daran zu saugen, dann hinunter zu ihrem Bauchnabel, bis er schließlich genau zwischen ihren Beinen seine Zunge spielen ließ.

Reflexartig schlug sie ihre Beine um ihn und vergrub ihre Hände in der Matratze. Sie wurde immer wilder und ekstatischer. Als sie endlich kam, rief sie nur seinen Namen: \enquote{Harry.}

Nachdem sie sich wieder entspannte und ihren Griff lockerte; ihre Beine öffneten sich wieder; schmeckte er ihren salzigen Saft auf seiner Zunge. Er krabbelte an ihr hoch, legte sich neben sie hin und schlief zusammen mit ihr ein.

Als er mitten in der Nacht erwachte, spürte er Lavender hinter seinem Rücken, einen Arm um ihn gelegt. Er spürte ihren Herzschlag und ihren gleichmäßigen Atem-Rhythmus an seinem Hals. \enquote{Lavender?}, fragte er vorsichtig. \enquote{Bist du wach?}

\enquote{Ja Harry}, antwortete sie.

\enquote{Weißt du, Lavender. Eigentlich war ich heute nicht mehr scharf auf dich, aber\abs}

\enquote{Du wolltest die Woche voll bekommen}, antwortete sie ihm und fiel ihm so ins Wort. Er musste schmunzeln. Sie beugte sich leicht über ihn und küsste seine Wange. \enquote{Geht mir genauso.}

\enquote{Aber, was mir sonst so durch den Kopf geht. Woher wissen wir, dass da nicht noch ein Restfunke ist, der wieder aufblüht?}

\enquote{Wie meinst du das?}

Harry drehte sich um und sah ihr in die Augen. In diesem schwachen Licht sah sie richtig schön aus. \enquote{Na ja, wenn wir die Sache jetzt beenden und es noch nicht vorbei ist, dann\abs werden wir wieder übereinander herfallen.}

Lavender schaute ihn nachdenklich an. \enquote{Was können wir dagegen tun?}

Harry hatte sich das ganze durch den Kopf gehen lassen. Er könnte Hilfe gebrauchen. \enquote{Versprich mir, dass du nicht schreist, wenn ich gleich einen Elfen rufe, um ihn zu fragen, ob er es feststellen kann?} Er konnte sich mittlerweile auf Kreacher verlassen, das wusste er nun. Lavender nickte nur stumm. \enquote{Kreacher? Kommst du? Kreacher, ich\abs wir brauchen dich.}

Es dauerte ein paar Sekunden, dann stand ein leicht verschlafener Elf mit müdem Gesichtsausdruck vor Harry und Lavender. Sie zog sich das Betttuch noch etwas weiter nach oben, um dem Elfen keinen allzu großen Einblick zu gewähren.

\enquote{Sir Harry hat gerufen?}, sagte der Elf und verbeugte sich.

\enquote{Ja Kreacher, tut mir leid dich geweckt zu haben, aber es ist wichtig.}

\enquote{Kreacher immer zu Diensten.}

\enquote{Kreacher, kannst du herausfinden, ob zwischen Lavender und mir noch ein Rest Anziehungskraft vorhanden ist? Ich nehme an, du hast von ihrem \gst unserem Unfall gehört.}

Lavender wurde rot hinter Harry. Das konnte er spüren. \gedanke{Unseren Unfall hatte er es genannt.}

\enquote{Sicher, Kreacher kann das. Sir und Madame müssen Kreacher nur die Hand geben.} Harry streckte Kreacher seine Hand hin. Lavender zögerte etwas, streckte dann aber doch ihre Hand ebenfalls dem Elfen hin. Dieser schaute die beiden lange an und meinte dann: \enquote{Es ist noch ein wenig vorhanden. Einmal sollte noch genügen.} Dann ließ er die Hände los und verbeugte sich.

\enquote{Danke Kreacher. Ich denke, einmal werden wir dich noch brauchen. Heute Nacht.} Der Elf verbeugte sich und verschwand. Harry drehte sich zu Lavender um und meinte dann: \enquote{Sieht so aus, als ob wir noch einmal müssten.}

Er wollte sie schon auf das Bett drücken, um auf sie zu klettern, als sie ihre Hand erhob und meinte. \enquote{Löffelchen-Stellung. Die hatten wir noch nie und Justin würde sie bestimmt auch gefallen.} Dann hielt sich Lavender vor Schreck die Hand vor den Mund. \enquote{Das\abs das wollte ich nicht.}

Doch Harry grinste nur. \enquote{Justin Finch-Fletchley?} Sie nickte stumm. \enquote{Er kann sich glücklich schätzen, dich zu haben.}

Erleichtert drehte sich Lavender um und schmiegte sich mit ihrem Rücken an Harry. Sie zog ein Bein an, um es ihm leichter zu machen, in sie einzudringen. Nachdem sie ein weiteres Mal dem Höhepunkt entgegensteuerten und anschließend Kreacher ihnen bestätigt hatte, dass nichts mehr zu finden sei, Harry sich bedanke und ihm eine gute Nacht wünschte, schliefen sie danach in dieser Position ein.

Am nächsten Morgen lag Lavender noch vor ihm, als er erwachte. Unbewusst fing er an, mit einer ihrer Brüste zu spielen.

\enquote{Harry?}, fragte sie plötzlich.

Er stoppte seine kreisenden Bewegungen und sagte: \enquote{Ja Lavender.}

\enquote{Lass uns ein letztes Mal miteinander schlafen.}

\enquote{Aber Lavender, wenn wir das machen, betrügen wir unsere Partner}, protestierte Harry.

Lavender fuhr mit einer Hand unter die Bettdecke und massierte sich zwischen ihren Beinen, worauf sie Wert darauf legte auch Harry zu liebkosen. \enquote{Ich sehe es nicht als Betrug an, Harry. Wenn wir jetzt aufstehen und uns anziehen würden und uns danach küssen, oder miteinander schlafen würden, dann wäre es Betrug. Jetzt ist es noch Skepsis gegenüber einem alten Elfen.}

Er spürte seine Erregung in ihm steigen. \fluestern{Lavender}, flüsterte er ihr ins Ohr. \fluestern{Was machst du nur mit mir?}

Er hörte ein leises Kichern. Sie ließ von ihm ab und setzte sich wieder auf ihn. Harry nahm noch einmal Lavenders Brüste in die Hand und küsste sie immer und immer wieder leidenschaftlich auf ihren Mund, sobald sie sich vorbeugte. Beide kamen nacheinander, Harry vor Lavender, und sackten zusammen. Harry fing sich wieder und nahm nun seine Finger zu Hilfe. Er fuhr mit einem Finger in ihre Scheide und drückte von außen mit dem Daumen auf ihr Lustknöpfchen. Er wusste nicht, woher er diese Gewissheit nahm, dass es ihr gefallen würde. Innen und außen rieb er jetzt mit seinen Fingern und verpasste Lavender so innerhalb kürzester Zeit einen zweiten und einen dritten Orgasmus. 
\end{abAchtzehn}

\begin{safedivide}
\fskdivider
\end{safedivide}

Danach schwanden ihr die Sinne. Als sie wieder erwachte, hatte Harry ihr bereits einen nassen Lappen auf die Stirn gelegt.

\enquote{Woher?}, stammelte sie.

Harry flüsterte in ihr Ohr: \enquote{Fleur \gst sie hat mir ein paar Tipps gegeben}, sagte er ausweichend.

Mit leuchtenden Augen gab sie ihm noch einen langen Abschiedskuss, bevor sich beide duschten, anzogen und schweigend in die Große Halle zum Frühstücken gingen. Sie würden weder darüber reden, noch dieses Ereignis wiederholen. Lavender war aber diejenige, die am meisten in diesen Nächten lernte, da Harry während seines Zustandes viele Erfahrungen sammeln konnte. Zwar unfreiwillig, aber dennoch\abs

\trenn

Nachdem er kräftig mit Agatha geübt hatte, war er bereit, vor Adriana zu treten und sie zu fragen, was es mit dem Buch über Dementoren zu tun habe. Er ging hoch Richtung Direktoren-Büro und stand dann vor dem Bild. Sie saß wie immer in einem Stuhl und schien zu schlafen. \enquote{Miss de Mimsy-Porpington?}, rief er dem Bild entgegen. Adriana öffnete ihre Augen und sah auf Harry herab, da sie leicht erhöht hing. Harry fing an, sie mithilfe der Gebärdensprache etwas zu fragen. Ganz erstaunt darüber, hob sie beide Augenbrauen und sah ihm zu, wie er sie etwas fragte.

Dann antwortete sie. \enquote{Das ist ganz einfach, Mister Potter\abs}

\enquote{Sie können reden?}, unterbrach er sie.

\enquote{Ja, sicher kann ich reden. Es ist nur mühselig, mit jedem zu reden. Immer wieder kamen Schüler und meinten, sich mit mir unterhalten zu müssen. Da ist mir die Idee mit der Gebärdensprache gekommen. Ich hatte sie gelernt, weil ich in meinem Freundeskreis jemanden hatte, der taub und stumm war. So schrecke ich fast alle ab. Nur Sie haben sich die Mühe gemacht und die Gebärden gelernt. Respekt. \gst Deshalb verdienen Sie es auch, dass ich mich mit Ihnen unterhalte.} Das verblüffte Harry. Er hätte nicht gedacht, dass eine Magierin zu solchen Tricks griff. Er hatte sich solche Mühe gegeben. Und jetzt brauchte er es nicht. Aber andererseits hätte er sich sonst gar nicht mit ihr unterhalten können. \enquote{Nennen Sie mich Adriana.}

\enquote{Dann nennen Sie mich aber Harry.} Adriana nickte und wartete auf Harrys Fragen. \enquote{Ich habe Ihr Buch \buchtitel{Vom Inferi zum Dementoren} gelesen. Haben Sie noch andere Informationen für mich, außer die, die in dem Buch stehen? Wissen Sie, Adriana, Dementoren haben ein gesteigertes Interesse an mir gezeigt.}

\enquote{Ach, das waren Sie?}

\enquote{Ja. Ich würde sie gerne bekämpfen.}

\enquote{Lernen Sie den Patronus-Zauber.}

\enquote{Den kann ich schon seit ein paar Jahren. Ich möchte sie aber nicht nur abwehren können. Ich möchte sie zerstören können. Mit einem Patronus soll das gehen. Aber wie?}

\enquote{Bauen Sie eine Verbindung zu Ihrem Patronus auf.} Sie hörte Geräusche. Eine Gruppe von Personen kam ihnen näher. \enquote{Lesen Sie in der Bibliothek darüber. Gehen Sie. Wenn Sie wieder eine Frage haben, dann wissen Sie ja, wo Sie mich finden.}

Harry nickte und bedankte sich. Dann lief er in die andere Richtung davon, weg von den Geräuschen. Darüber musste er erst einmal nachdenken und setzte sich in eine ruhige Ecke im Schloss. Er griff in seine Tasche und holte seinen Würfel heraus.

Er zog seinen Zauberstab und richtete ihn auf seinen Würfel. Er dachte nach, bevor er den Zauber sprach, der die Farbe vom Würfel entfernen sollte. Er konzentrierte sich und schwang seinen Zauberstab. Da sie das ganze Jahr über ungesagte Zauber bei Professor Flitwick anwandten, war dies eine prima Gelegenheit seine Fähigkeiten zu verbessern.

Die Farbe bröckelte ab und darunter kam der Würfel zum Vorschein. Es schien, dass im Inneren des Würfels ein kleines blaues Licht durch die diffuse Oberfläche leuchtete. An den Kanten des Würfels waren schmale, angelaufene Messingstreifen zu sehen, zusätzlich war auf jeder Seite ein auf der Spitze stehendes Quadrat ebenfalls aus Messingstreifen, und in deren Mitte ein Kreis, ebenfalls aus schmalen Messingstreifen.

Harry betrachtete gebannt den schimmernden Würfel. Er hatte die Farbschicht erfolgreich entfernt, aber noch immer keine Ahnung wie er ihn öffnen konnte. Er drehte ihn zwischen seine Hände hin und her, in der Hoffnung irgendwo einen Hinweis darauf zu erhalten, wie er sich öffnen ließe. Als ihm nichts einfiel, betrachtete er ihn näher.

Die auf der Spitze stehenden Quadrate waren aus zwei dünneren Streifen aufgebaut. Es schien so, als ob man die pyramidenförmigen Ecken des Würfels abnehmen konnte, doch sie bewegten sich nicht. Er legte ihn zurück auf seinen Nachttisch und krabbelte in sein Bett. Seine Zimmergenossen waren noch im Gemeinschaftsraum und unterhielten sich oder spielten. Doch Harry musste etwas alleine sein. Er brauchte nach der Aufregung heute Morgen und der letzten Woche etwas Ruhe. Er drehte sich um und schlief ein.

Harry konnte nach diesem Erlebnis am See, das schon einige Tage zurücklag, keinen klaren Gedanken mehr fassen.

\begin{rueckblick}
Chwalla hatte ihn in seinen Bann gezogen. Er konnte sich bei Flitwick heute nicht konzentrieren. Immer wieder wurde er von ihm ermahnt, was schließlich dazu führte, dass er für den Rest der Stunde als Übungspartner, oder besser gesagt als Versuchskaninchen, herhalten musste. Er musste sich verteidigen und Professor Flitwicks Zauber abwehren. Harry war schon am Ende der Stunde körperlich erschöpft und hatte noch Tränke bei Snape vor sich. Heute sollten sie den Trank der lebenden Toten zubereiten. Ein sehr komplizierter Trank. Harry hatte keinen Erfolg. Sein Trank war eher eine schwarze zähflüssige Masse, die so aussah und so roch, als bestünde sie aus Teer, daher musste er sich wieder Professor Snapes Spott anhören. Besonders Malfoy lachte über ihn. Und auch bei Professor Sprout war es nicht besser. Die fleischfressenden Pflanzen, um die sie sich heute wieder einmal kümmern mussten, bissen sich heute in Harry fest. Und nicht nur im Finger! Es schien so, als ob sie an ihm besonderen Gefallen fanden. Sie bissen ihn an allen möglichen Stellen, sodass er in den Krankenflügel gebracht wurde. Madame Pomfrey war darüber natürlich nicht gerade erfreut. Aber sie behandelte ihn, so wie sie alle Schüler behandelte, voller Hingabe an ihren Job. Und dieses Mal besonders gerne. Sie bemutterte ihn richtig!
\end{rueckblick}

\onelineback % Anderenfalls werden 2 Leerzeilen gesetzt
\trenn

Ron kam ganz aufgeregt in den Gemeinschaftsraum. \enquote{Harry, schon mal auf deinen Stundenplan geschaut?}

Harry schaute ihn erstaunt an. \enquote{Nein, wieso?} Ron zeigte ihm seinen Stundenplan. Das Fach \VgddK war rot gefärbt. Also drehte Harry den Stundenplan um und las.

\begin{brief}
Bringen Sie zu Ihrem nächsten Unterrichtstermin Badesachen mit. Wir treffen uns am See.
\end{brief}

Harry schaute Ron erstaunt an.

\enquote{Ron, Harry}, kam Hermine aufgeregt in den Gemeinschaftsraum. Sofort musste er grinsen. Er stellte sich Hermine gerade in Badesachen vor. Glücklicherweise war dieses Wochenende ein Hogsmeade-Wochenende.

\enquote{Gehen wir dir 'nen Badeanzug kaufen?} Ron konnte sich diese Frage nicht verkneifen.

Hermine schaute ihn böse an und stupste ihn in die Seite. \enquote{Nein Ron, aber hast du eine Badehose?} Ron schaute Hermine mit einem leichten Anflug von Panik an. \enquote{Dann gehen wir gleich nach dem Frühstück nach Hogsmeade und kaufen dir eine}, schlug Hermine vor und zog ihn Richtung Ausgang. Harry hatte bereits seit letztem Jahr eine Badehose, da er ab und an im See schwamm. Er traf auch immer mal wieder auf Luna, die aber immer nackt in den See ging. Seit er mit ihr zusammen war, ging auch er, sofern sie alleine waren, ohne seine Badehose in den See, um mit ihr zu schwimmen. Auch nach ihrer Trennung.

Er war gespannt, was ihn in dieser Stunde erwartete. Doch zunächst war er erst einmal hungrig und folgte Ron und Hermine in die Große Halle.

So einfach wie sich die drei das vorgestellt hatten, für Ron eine Badehose zu kaufen, war es nicht. Der Laden wimmelte nur so von Schülern, die noch Badesachen kaufen mussten. Schließlich fand sich für Ron eine schwarze Badehose mit gelben Streifen und einem grünen Kleeblatt darauf. Harry und Hermine tauschten vielsagende Blicke, als sie zur Kasse liefen. Als der Verkäufer den Preis nannte, stockte Ron. Harry öffnete seinen Geldbeutel und legte den Betrag auf den Tisch. \enquote{Alles Gute zum Geburtstag, Ron}, antworteten Harry und Hermine.

\enquote{Aber, der ist doch erst in zwei Wochen}, protestierte Ron.

\enquote{Dann lassen wir sie eben einpacken, und du kannst erst in zwei Wochen den Unterricht besuchen}, schloss Hermine pragmatisch.

Ron verzog seinen Mund, sagte dann aber glücklich: \enquote{Danke} und umarmte die beiden.

\trenn

Heute fingen sie bei Professor McGonagall mit der Transformation von Menschen an. Ein anspruchsvolles Thema. Harry und Neville taten sich in einem Team zusammen, als es darum ging, Körperteile seines Partners zu verwandeln. Sie mussten sich klar und deutlich einen Affen vorstellen und dem Partner einen Schwanz verpassen. Neville fing an und stellte sich ein Pin\-sel\-ohr\-äff\-chen vor. Er schwang seinen Zauberstab und aus Harrys Hintern wuchs ein Affenschwanz heraus. Harry hatte das Gefühl, ihn wirklich bewegen zu können. Interessiert betrachtete er ihn. Dann war Harry an der Reihe. Er stellte sich seinen Affen vor und wollte gerade seinen Zauberstab schwingen, als ihm wieder Chwalla einfiel. Er führte seine Bewegung aus, als er merkte, dass etwas schieflief. Neville bekam große Augen. Ihm begannen Kiemen zu wachsen. Er schnappte nach Luft. Plötzlich tauchte aus dem Nichts ein großer Eimer mit Wasser zwischen Harry und Neville auf. Er steckte sofort seinen Kopf ins Wasser, um atmen zu können. Harry lief es kalt den Rücken herunter.

\enquote{Woran haben Sie gedacht Mister Potter?}, fuhr ihn Professor McGonagall an. Völlig verstört, drehte er sich ihr zu und sah sie an. \enquote{Woran haben Sie gedacht Mister Potter}, fuhr ihn Professor McGonagall erneut und dieses Mal mit Nachdruck an.

\enquote{Äh, an\abs an\abs Wasserlebewesen. Solche wie im See leben.}

Sie schwang ihren Zauberstab auf Neville gerichtet und zog seinen Kopf aus dem Eimer heraus, welcher darauf hin verschwand. Nevilles Haare waren nass und Professor McGonagall reichte ihm ein herbeigezaubertes Handtuch. Nachdem er seine Haare abgetrocknet hatte, versuchte Harry es erneut. Er konzentrierte sich wieder und dieses Mal klappte es. Aus Nevilles Hintern wuchs nun ebenfalls ein Schwanz heraus. \enquote{Tut mir leid wegen vorhin, Neville. War keine Absicht. Ich bin die letzten Tage nur etwas abgelenkt und bekomme einen Gedanken nicht mehr aus dem Kopf.}

\enquote{Ich verstehe das, Harry. Bei mir ging schon viel mehr schief. Mir geht es gut.} Nachdem jetzt alle durch waren, ging es daran, die Schwänze wieder zu entfernen. Dieses Mal gab es keinen Zwischenfall, sodass Neville und Harry ihre Schwänze wieder verloren. Die hinteren.

Weiter ging es mit den Nasen, die jetzt in einen Rüssel verwandelt werden sollten.

Etwas später saßen die drei im Zaubergetränkekeller und warteten, dass Professor Snape sein Büro verließ und den Unterricht begann. Doch leider betrat Professor Elber den Keller und meinte: \enquote{Schlagt eure Bücher auf Seite 320 auf und mischt euren Trank an.} Er betrat Snapes Büro und kam kurz darauf mit einer Schachtel herein. \enquote{Die fehlenden Zutaten finden sich hier. Fangen Sie an.} Er stellte die Schachtel auf einen kleinen Tisch und setzte sich in den Stuhl hinter dem Schreibtisch, in dem normalerweise Snape saß.

\enquote{Professor, was ist mit Professor Snape?}, fragte Draco Malfoy.

\enquote{Der hat momentan keine Zeit, vielleicht kommt er später kurz vorbei. Ist aber eher unwahrscheinlich}, sagte er kurz angebunden.

Also schlug Harry sein Buch auf und fing mit Neville zusammen an, seinen Trank zu brauen. Ab und an sah er zu seinem Professor auf. Doch immer, wenn er aufsah, dann sah er in ein missmutiges Gesicht. Es schien, dass Professor Elber absolut keine Lust hatte, Professor Snape zu vertreten. Er stand auf und lief durch den Raum, um seinen Schützlingen bei eventuellen Problemen helfen zu können. Leider hellte das seine Stimmung nicht gerade auf.

Plötzlich gab es eine Explosion und schwarzer Rauch erfüllte augenblicklich den ganzen Raum. Er biss in den Augen und im Mund. Doch wenige Sekunden später war der Rauch verschwunden und Professor Elber hatte seinen Zauberstab in der Hand. \enquote{Unfähige Bande. Können Sie nicht besser aufpassen?}, brüllte Professor Elber durch den Raum. Keiner hatte mitbekommen, wie Professor McGonagall und Professor Snape im Türrahmen standen. \enquote{Hat euch Professor Snape nicht beigebracht, besser aufzupassen? Jetzt verstehe ich ihn! Da redet er mehr als fünf Jahre auf euch ein und noch immer passiert so etwas.} Er war richtig in Rage.

\enquote{Professor}, kam es von Lavender. \enquote{Liana liegt am Boden.} Professor Elber drehte sich zu ihr und sah den Kessel immer noch rauchen. Sofort rannte er zur verletzten Schülerin und fühlte ihren Puls. Schlagartig wurde er bleich und ließ seinen Zauberstab über ihr schweben. Danach sprach er einen Zauber, den Harry nicht verstehen konnte. Er konnte zusehen, wie sie innerhalb zweier Sekunden komplett zu Stein wurde. McGonagall und Snape waren zu geschockt über Elbers Reaktion, um etwas zu tun.

Professor Elber stand auf und griff Lavender an beide Schultern. \enquote{Was hattet ihr in diesen Trank getan?} Seine Stimme war zittrig, aber dennoch verärgert und lauter als normal.

\enquote{Wir\abs ich\abs haben\abs habe\abs} dann zeigte sie stumm auf das Kraut, welches neben dem Kessel lag. Professor Elber nahm mit einer Hand das Blatt und sah es sich an.

\enquote{Woher habt ihr das?}, fragte er jetzt zorniger.

\enquote{Frederick}, mischte sich jetzt Professor McGonagall ein. \enquote{Das arme Mädchen.}

Er drehte sich um. Professor Snape sah unterdessen nach der Schülerin.

\enquote{Das arme Mädchen hätte sich beinahe umgebracht, weil sie ein falsches Kraut verwendet hatte}, machte er weiter. Er ignorierte Professor McGonagall wieder und wandte sich wieder zu Lavender. \enquote{Woher habt ihr das?} Er wedelte mit dem Blatt vor ihrer Nase herum.

\enquote{Aus einem Laden aus der Winkelgasse.}

Professor Elber sah erstaunt auf. \enquote{Keiner. Ich wiederhole, keiner verwendet mir dieses Kraut noch. Werft eure Tränke weg. Zerstört sie, sofort. Ich gehe kein weiteres Risiko ein. Das hier ist ein gefährliches Kraut. Es ist dem Plautus-Blatt sehr ähnlich, hat aber eine verehrende Wirkung. Ich kann es nicht fassen, dass der Händler \gst ich hoffe, ihr habt es im üblichen Laden gekauft \gst euch das falsche gegeben hat.}

Lavender bekam große, leicht glasige Augen. \enquote{Ich werde mich später für mein Verhalten bei Ihnen entschuldigen, Miss. Aber jetzt sagen Sie mir bitte, woher genau Sie dieses Blatt haben!} Lavender erkläre nun, woher sie das Blatt hatte. \enquote{Wir müssen den Händler anschreiben. Er soll überprüfen, ob es noch weitere Verwechselungen gab. Die Stunde ist beendet. Bitte bringen Sie Ihre Mitschülerin in den Krankenflügel.}

Dann wandte er sich zu Professor Snape. \enquote{Severus. Ich brauche von Ihnen einen Trank um Liana zu helfen. Noch kann man sie zurückholen. Es bleiben aber nicht mehr viele Sekunden übrig. Deswegen habe ich sie in eine schützende Steinhülle gelegt. Aber der Trank ist kompliziert und ich bin nicht in der Lage, ihn zu brauen. Ich werde Ihnen das Rezept gleich zukommen lassen.}

Danach drehte er seinen Zauberstab in seiner Hand und sah im Raum umher. Als er wieder zu Professor Snape blickte, nickte dieser nur. Professor Elber nickte zurück.

Dann drehte er sich zu Lavender und nahm sie an ihrer Schulter. \enquote{Würden Sie kurz mit nach draußen kommen?} Sie nickte und verließ zusammen mit ihrem Professor den Raum.

Kurz darauf schwebte ein Buch in den Raum und blieb genau vor Professor Snape in der Luft stehen. Es drehte sich noch leicht und klappte danach auf einer bestimmten Seite auf. Professor Snape sah sich das Rezept an und verschwand dann in seinem Büro.

\enquote{Die Stunde ist beendet}, schaltete sich Professor McGonagall jetzt ein. Die Schüler wurden aus ihrem Schock-Zustand gerissen und packten ihre Taschen zusammen.

\enquote{Hast du einen Moment Zeit Harry?}, fragte Dumbledore, der gerade den Raum betrat.

\enquote{Ich habe Hermine versprochen, mit ihr Hausaufgaben zu machen. Das einzige Fach, bei dem sie Hilfe von mir möchte. Geht es noch in einer Stunde, oder soll es gleich sein?}

\enquote{So dringend ist das nicht, Harry. Mach erst deine Aufgaben.}


\trenn

\enquote{Harry, du wunderst dich sicherlich darüber, dass du wieder hier bist.}

\enquote{Ach weißt du Albus, ich\abs äh Professor\abs}

\enquote{Albus ist in Ordnung. Minerva weiß Bescheid.}

\enquote{Ich wundere mich mittlerweile über gar nichts mehr}, meinte McGonagall.

\enquote{Wie geht es mit Professor Elber voran?}, fragte Albus.

\enquote{Sehr gut. Manchmal weiß ich nicht genau, was er von uns will, wenn wir Unterricht haben. Aber auch wenn ich alleine mit ihm unterwegs bin. Aber ich frage mittlerweile gar nicht mehr nach. Er erklärt es mir, wenn wir einen gewissen Stand erreicht haben. Ich habe viel gelernt.}

\enquote{Gut, sehr gut. Ich möchte dich auf eine Reise mitnehmen. Ich habe den Ort eines von Voldemorts Horkruxen erfahren.}

\enquote{Wo?}, war alles, was Harry fragte.

\enquote{Eine Höhle an einer Klippe, mit Zugang zum Meer.}

\enquote{Kenne ich}, antwortete Harry.

Minerva fiel ihre sonst so sorgsam aufgesetzte Miene ab.

\enquote{Wie können Sie diesen Ort kennen, Potter?}, fragte sie fassungslos.

\enquote{Ich habe Träume. Realistische Träume. Ich habe mich dort gesehen, wie ich mit jemand in einem Boot über einen unterirdischen See fuhr. Dann sah ich in eine Schale hinein und entdeckte ein Medaillon. Ich gab meinem \accentuate{Gefährten} den Inhalt des mit einer Flüssigkeit gefüllten Beckens, und legte das Medaillon hinein.}

\enquote{Wie können Sie es erst sehen und dann hineinlegen?}, fragte Minerva.

\enquote{Es war ein Traum Min\abs Professor McGonagall. Eine Vision. Das Medaillon war beim ersten Mal schwach und verwaschen zu erkennen. \gst Dann fuhr ich alleine über den See zurück. \gst Es hat eine Weile gedauert, aber ich habe das Medaillon erkannt. Ich habe meinen Begleiter erkannt. Und ich habe das Medaillon hier im Schloss.}

Er griff in seine Tasche und zog es in ein Taschentuch gewickelt heraus. Dann legte er es auf den Tisch und öffnete das Taschentuch.

\enquote{Mister Potter. Wie leichtsinnig sind Sie eigentlich? Auf eigene Faust ein Versteckt von Du-weißt-schon-wem\abs}

\enquote{Voldemort}, unterbrach sie Harry frech.

\enquote{\aabs Aufzusuchen und auch noch einen Horkrux herauszuholen. Das war mehr als Leichtsinnig.}

Harry sah sie an und wartete, bis sie sich wieder einigermaßen beruhigt hatte.

\enquote{Ich hätte nicht gedacht, dass Sie zu der Sorte von Menschen gehören, die erst schimpfen und sich erst dann erklären lassen, was passiert ist.}

Das hatte gesessen. Professor McGonagall sah ihn vollkommen fassungslos an.

Um die Situation zu entschärfen, sah er sie weiterhin an. \enquote{Ich war nicht dort. Ich habe lediglich davon geträumt. Das Medaillon dort ist eine Fälschung, wie ich herausgefunden habe. Dieses hier auf dem Tisch ist das Echte.}

\enquote{Aber\abs woher\abs?}

\enquote{Kreacher, kommst du?}

Und der alte Elf erschien. \enquote{Sir Harry hat gerufen?}, krächzte er.

\enquote{Sei so gut und erzähle den beiden, was du mir über das Medaillon erzählt hast.} Harry hob das Medaillon noch einmal mit dem Taschentuch vom Tisch, um es Kreacher zu zeigen.

Kreacher nickte und erzählte von Meister Regulus. Wie dieser Kreacher befahl dem Dunklen Lord auf eine Mission zu folgen. Wie er über den See fuhr und von dem schrecklichen Trank trinken musste. Er erzählte, wie er meinte sterben zu müssen. Und dass er zurückkehrte. Ein Schluck frisches Wasser, zu Hause, und der Fluch war gebrochen.

Dumbledore und McGonagall hörten aufmerksam zu. Kreacher wollte sich schon wieder dafür bestrafen, dass er es nicht fertiggebracht hatte, das Medaillon zu zerstören, aber Harry hielt ihn davon ab.

\enquote{Das bringt nichts, Harry. Solange das Medaillon nicht zerstört ist, wird er das immer wieder tun. Selbst deine Anweisungen können dies nicht verhindern.}

\enquote{Dann zerstören wir es eben.}

Dumbledore nickte. Er stand auf und öffnete eine Vitrine. Dann nahm es das Schwert von Gryffindor heraus.

\enquote{Aber Albus, ein einfaches Schwert gegen einen Horkrux?} Seine Gedanken schwammen. \enquote{Ach ja, Kobold-gearbeitet. Es nimmt auf, was es stärkt. \gst Richtig, es ist mit Basiliskengift getränkt, mit jenem Gift, mit dem ich den ersten Horkrux zerstört habe.} Dann wurde ihm etwas bewusst. \enquote{Ich sollte es öffnen, oder?}

Dumbledore nickte nur. \enquote{Außerdem ist noch etwas wichtig.} Er nahm das Medaillon vom Tisch, lief einmal außen herum und legte es auf den Boden. Dann reichte er Kreacher das Schwert.

\enquote{Für Kreacher?}, fragte der Elf.

\enquote{Um das Medaillon zu zerstören, ja. Behalten, nein.}

Der Elf nickt und nahm das Schwert an sich.

\enquote{Wie läuft es ab?}, fragte Harry.

\enquote{Du befiehlst dem Medaillon, sich zu öffnen und dann sticht Kreacher zu, sobald es offen ist.}

Harry nickte. Dann sah er Kreacher an und fragte ihn: \enquote{Bereit?} Sein Elf nickte, also befahl Harry dem Medaillon sich zu öffnen: \parsel{Öffne dich.}

Lautlos öffnete sich das Medaillon und eine schwarze Wolke kehrte hervor. Eine neblige Gestalt manifestierte sich. Sie zeigte Walburga Black, Kreachers ehemalige Herrin. \enquote{Du wirst das nicht tun, Kreacher. Lass das bleiben. Das ist ein Befehl.}

Kreacher begann zu zittern. Bange Sekunden verstrichen. \enquote{Sir Harry ist jetzt mein Herr. Ihr habt mich immer gedemütigt und nicht gut behandelt.} Dann stach er zu und der Nebel verpuffte im Nichts.

Das Medaillon war durchbrochen und zerstört. Harry horchte in sich hinein, aber nichts passierte. Er konnte es immer spüren, wenn mit Voldemort etwas war. Plötzlich blitzten Orte auf, verschwommen, aber dennoch markant um sie zu erkennen, wenn man wusste, wo sich der Ort befindet.

Dumbledore entging das nicht. \enquote{Was ist los Harry?}, fragte er. Er nahm von Kreacher das Schwert entgegen und verstaute es wieder in der Vitrine.

\enquote{Alles in Ordnung, Kreacher?}, fragte Harry seinen Elf. \enquote{Sei ehrlich.}

\enquote{Etwas schlapp.}

\enquote{Dann ruhe dich etwa aus.}

Kreacher nahm dies als Einladung, noch etwas zu bleiben und legte sich auf eine herbeigezauberte Decke vor dem Kamin.

\enquote{Ich habe Orte gesehen, Albus}, antwortete Harry. \enquote{Orte, an denen Horkruxe versteckt sein könnten. Aber sie waren unscharf.}

\enquote{Arbeite weiter daran}, bat Dumbledore.

Harry dachte darüber noch eine Weile nach. Dann löste sich die Versammlung auf. Kreacher war mittlerweile eingeschlafen. Harry besah ihn lächelnd und sorgte mit einem vorsichtigen Schwebezauber dafür, dass die Decke samt Elf hinter ihm her schwebte. Er brachte ihn in die Küche, wo die Elfen gerade beisammen saßen und etwas aßen. Das Abendessen war bereits vorbei und die Küche war sauber. Als die Elfen ihn sahen, sprangen einige auf und holten ihm eine kalte Platte. Sie servierten ihm Reste des Abendessens, die nicht verwendet worden waren. Harry nahm die Speisen dankbar an, nachdem er Kreacher auf seinem Schlafplatz niedergelegt hatte. Während des Essens unterhielt er sich mit drei jungen Elfen, die er als Timmy, Tammy und Tommy kennenlernte. Sie erzählten ihm, dass sie lediglich zur Ausbildung hier seien. Die Familie, zu der sie gehörten, wollten sie nicht nennen. Harry respektierte dies und fragte nicht weiter nach. Nachdem er satt war und genug getrunken hatte, ging er zurück zum Gemeinschaftsraum der Gryffindors und machte sich für die Nacht fertig.

Heute würde Harry gut schlafen können. Er legte sich in sein Bett und schlief augenblicklich ein.




\begin{kommentar}
Der Würfel, den Harry öffnen soll, ist von Star Wars The Clone Wars übernommen. Ihr könnt ja mal nachschauen. Sucht einfach nach »Holocron«. Dann werdet ihr schon fündig.
\end{kommentar}
