\chapter{Prüfungen und Erkenntnisse}


Professor Elber sah sie nachdenklich an und setzte sich auf das Bett gegenüber, um das Mädchen anzuschauen.

\enquote{Wie haben Sie es bemerkt? Ich meine das mit Ihren Haaren und wieso ist eine Hälfte Ihres Gesichtes versteinert?}

\enquote{Bemerkt habe ich es, als ich aufgestanden bin und mich Marlene angesehen hatte und sofort versteinerte. Dasselbe ist mit Sandra passiert, als sie ins Zimmer kam. Pansy war glücklicherweise noch draußen. Ich wollte schon zu meinem Handspiegel greifen, entschied mich dann aber, die Augen zuzumachen und nur kurz eines zu öffnen. Das reichte schon, um eine Gesichtshälfte zu versteinern. Ich habe meine Haare sofort eingewickelt und bin zu Madame Pomfrey gekommen und habe ihr von den anderen erzählt. Jetzt liegen sie hier.}

\enquote{Können Sie mir die Schlangen auf Ihrem Kopf beschreiben?}, fragte er weiter.

\enquote{Ich habe sie nur kurz gesehen. Sie waren grau \gst grün. Es waren viele kleine Schlangen. Genauer weiß ich es nicht.}

\enquote{Darf ich in diesem Fall Ihre Erinnerungen daran sehen?}

\enquote{Falls es Ihnen hilft.} Sie nahm ihren Zauberstab und wollte ihn an ihre Schläfe setzen.

\enquote{Nein, das meinte ich nicht. Ich werde sie mir direkt in Ihrem Kopf ansehen.}

Leicht verunsichert meinte sie: \enquote{OK.}

\enquote{Keine Angst. Denken Sie nur an das Bild von heute Morgen. Ab dem Zeitpunkt, wo sie aufgestanden sind. Und falls sie sich schämen, ich werde auf diesem Wege die unwichtigen Sachen vergessen. In einem Denkarium würde ich sie nicht so leicht vergessen.}

Katharina nickte und entspannte sich. Sie stellte sich den Zeitpunkt des Aufstehens vor und spürte eine Präsenz in ihrem Kopf. Dann durchlebt sie die Szene ein zweites Mal. Zuerst entdeckte sie Marlene, die gerade in ihrem Bett aufrecht saß und sie anstarrte und dann versteinerte. Dann kam Sandra herein und versteinerte ebenfalls, nachdem sie Marlene sah und dann aufgeregt Katharina fragen wollte, was denn passiert sei. Dann griff Katharina nach einem Handspiegel und es wurde kurz dunkel. Schließlich öffnete sie ein einzelnes Auge und erblickte ihr Spiegelbild. Die Präsenz in ihrem Kopf schien das Bild anzuhalten und nun konnte Katharina sich selber für eine längere Zeit sehen. Dann wandelte sich das Spiegelbild und ihr Professor erschien im Spiegel.

\enquote{Ich glaube immer weniger, dass es sich um einen Scherz eines oder mehrerer Ihrer Mitschüler handelt.}

Das Spiegelbild wandelte sich wieder und die Realität kehrte zurück. Sie war wieder in der Krankenstation und saß Professor Elber gegenüber, der sie immer noch nachdenklich ansah.

Ihre stumme Frage konnte er von ihren Augen ablesen: \gedanke{Was meinen Sie mit: \inner{Ich glaube immer weniger an einen Streich ihrer Mitschüler}?}

\enquote{Wie viel wissen Sie über ihre Familie? Genauer über Ihre Ahnentafel}, wollte Professor Elber wissen.

\enquote{Weshalb, Professor?}

\enquote{Ich habe einen Verdacht. Es ist nämlich so, dass bei weiblichen Nachfahren der Medusa unter bestimmten Umständen dieser Fluch wieder aufflammt.}

Erschrocken sah sie ihn an. Madame Pomfrey hatte mittlerweile einen Arm um sie gelegt, um sie zu trösten. Dumbledore saß auf der anderen Seite und sah sie großväterlich an.

\enquote{Kommen Sie nachher zu mir. Dann trinken wir zusammen eine Tasse Tee. Dann können Sie mir etwas über ihren Kopfschmuck erzählen}, sagte Dumbledore.

Katharina nickte und schniefte.

\enquote{Ich werde mal in der Bibliothek lesen und schauen, was man dagegen tun kann. Ich nehme an, Sie wollen nicht immer mit Hut oder Handtuch auf dem Kopf herumlaufen.} Katharina schüttelte den Kopf. \enquote{Ein kleiner Lichtblick bleibt Ihnen auf jeden Fall. Wenn Ihre Gesichtshälfte geheilt ist und Sie sich noch ein paar mal im Spiegel mit einem Auge betrachten, gewöhnen Sie sich daran. Dann können Sie, wenn Sie alleine sind, ihren Kopfbewuchs betrachten.} Dann stand er auf und verließ den Krankenflügel Richtung Bibliothek.

\enquote{Ich werde Ihre Eltern benachrichtigen}, sagte Madame Pomfrey und drehte sich um, um einen Brief zu schreiben.

\enquote{Bitte nicht}, jammerte Katharina

\enquote{Aber Ihre Eltern haben ein Recht darauf}, antwortete Madame Pomfrey, nachdem sie sich umgedreht hatte. Danach setzte sie ihren Weg zu ihrem Büro fort.

Katharina fing an zu weinen, was Dumbledore veranlasste, sich neben sie zu setzen.

\enquote{Sie werden mich bestimmt verstoßen. Sie sind sehr altmodisch und haben mich entsprechend erzogen. Außerdem halten sie viel vom reinen Blut und haben einen Hass auf alles, was kein Zauberer oder Hexe ist. Wenn sie erfahren, was mit mir passiert ist, brauche ich nicht mehr nach Hause zu kommen.}

\enquote{Sie meinen, es ist so schlimm?}, fragte Dumbledore vorsichtig nach.

Katharina nickte und fuhr fort. \enquote{Ich habe bei meiner Auswahlzeremonie immer wieder innerlich gefleht, nach Slytherin zu kommen, da ich anderenfalls nichts hätte, wo ich sonst in den Sommerferien hingehen könnte.}

Die Tür ging wieder auf und Professor Elber durchquerte das Zimmer. \enquote{’Tschuldigung, hab’ was vergessen \gst Was ist mit Ihnen los?}, fragte er.

\enquote{Sie hat Angst, verstoßen zu werden.}

Professor Elber nickte und lief weiter.

Dumbledore blickte wieder zu Katharina und bemerkte nicht, dass sich durch das Kopfschütteln und Weinen ihr Handtuch lockerte. Als er aus einem Augenwinkel eine Bewegung sah und reflexartig dort hinsah, erblickte er eine Schlange und erstarrte zu Stein.

Katharina wurde bleich, als sie das sah und wickelte ihr Handtuch sofort wieder fester um ihren Kopf und versuchte die Schlangen unter ihrem Handtuch zu verstecken. In ihrem Nachtschränkchen suchte sie nach einer Sicherheitsnadel und sicherte damit ihr Handtuch um ihren Kopf. Danach versucht sie ihren Schulleiter anzusprechen. Doch diese steinerne Statue gab kein einziges Wort von sich. Außerdem fühlte er sich kalt wie Stein an.

\enquote{Das darf doch nicht wahr sein}, jammerte sie erneut.

Die Tür ging auf und Madame Pomfrey kam mit Professor Elber heraus. Als sie die steinerne Statue sahen, blieben sie stehen.

\enquote{Ich war das nicht\abs Ich meine, es war ein Unfall\abs mein Handtuch \gst} und wieder weinte Katharina. Die beiden Erwachsenen sahen sich nur kurz an, nickten einander zu und setzten sich links und rechts neben Katharina, nachdem sie die Statue von Professor Dumbledore auf ein Nachbarbett schweben gelassen hatten, und nahmen sie in den Arm.

Instinktiv drehte sich Katharina zu Madame Pomfrey und heulte hemmungslos auf die Schürze ihrer Krankenschwester. Doch nach wenigen Minuten hatte sie sich beruhigt und schlief einfach ein. Madame Pomfrey bettete sie vorsichtig und verabschiedete sich von beiden. Auch Professor Elber verließ die Krankenstation, um seinen ursprünglichen Weg zur Bibliothek fortzuführen.

Etwas später war Harry mit Elber alleine. Sie saßen auf einer Bank vor dem schwarzen See. Nur durch einen Kiesweg vom Ufer getrennt. Vor beiden lag ein runder, kugelförmiger Stein.

\enquote{Bereit?}, fragte Elber Harry.

\enquote{Ja.}

Professor Elber hob seine Hand über die Steinkugel und hatte kurz darauf die Kugel in der Hand. \enquote{Und jetzt Sie.}

Harry hob ebenfalls seine Hand über seine Kugel und sie schwebte in seine Hand. Allerdings etwas langsamer.

Professor Elber drehte seine Hand um, öffnete sie und zog sie nach unten weg. Die Kugel blieb in der Luft stehen. Harry machte es ihm nach. Er versuchte es zumindest, da die Kugel der Hand folgte und von dieser herunterrollte und zu Boden fiel.

\enquote{Nochmal}, sagte sein Lehrer. \enquote{Bis es klappt.}

Harry wunderte sich, was das Ganze sollte. Doch er hatte bisher immer erst später begriffen, warum er gerade diese Übungen machen musste. So auch bei dieser. Er würde schon noch erfahren, was es damit auf sich hatte. Das Schöne daran war, dass Harry einfach mal nachdenken konnte. Nicht wie im Unterricht. So fing er nach dem fünften fallen lassen der Kugel an nachzudenken, was er anders machen könnte. Er hielt die Kugel wieder in seiner Hand. Die letzten Male kam sie schneller in seine Hand. \gedanke{Würde bremsen helfen?}, dachte er sich. \gedanke{Wenn ich die Kugel einfach in der Luft bremse.} Er ließ seine Hand auf seinem Schoß, konzentrierte sich und ließ die Kugel wieder hochschnellen. Leider schoss sie in die Höhe, bis Harry sie nicht mehr sah. Professor Elber schob ihn etwas zur Seite und rutschte ebenfalls an den Rand der Bank. Harry kannte so etwas schon. Er wurde positioniert. Kurz darauf schlug die Kugel durch das mittlere Brett der Sitzfläche.

Harry wunderte sich nicht mehr darüber, dass sein Lehrer immer wusste, wo die Kugel oder andere Dinge ankamen.

\enquote{Woher wussten Sie, wo die Kugel runterkommt?}

\enquote{Wenn Sie die Kugel beherrschen, dann kennen Sie das Geheimnis}, sagte er schlicht. \enquote{Versuchen Sie es noch einmal.}

Harry nahm die Kugel wieder in die Hand. Er musste sie nicht mehr anheben. Er überlegte. Lange.

\stimme{Denke einfach, Harry.}

\enquote{Sal!}

\enquote{Was meinen Sie?}, fragte ihn sein Professor.

\enquote{Ach, ich habe nur laut gedacht.}

\enquote{Ach so.}

\gedanke{Und was soll ich denken?}

\stimme{Dass der Zauber funktionieren soll.}

\gedanke{Wie?}

\stimme{Denke einfach an den Zauber. Oder besser: Denke einfach an das, was passieren soll.} Gedankenversunken nickte Harry.

Er sah die Kugel erneut an und senkte danach seine Hand. Die Kugel gab etwas nach, blieb aber mit einer ständigen leichten Auf- und Abwärtsbewegung in der Luft schweben.

\gedanke{Einfach an das denken, was passieren soll}, ging ihm durch den Kopf.

\enquote{Gut, dann können Sie das jetzt üben, wenn Sie alleine sind. Sie wissen doch noch, was wir besprochen hatten?}

\enquote{Ja. Kein Wort zu niemandem. Selbst zu Ron und Hermine nicht.}

\enquote{Exakt.}

Harrys Blick war immer noch auf die Kugel gerichtet. Er wollte gerade fragen, wie er denn jetzt an das Geheimnis kommen konnte. Doch bevor er seine Frage stellen konnte, hatte er einen Verdacht. Er konzentrierte sich und ließ die Kugel nach oben schnellen. Im Geiste verfolgte er die Kugel und wusste nun, wo sie auftreffen würde. Er suchte die Stelle auf dem Kiesweg und wartete, bis die Kugel aufkam. Er hatte sich nicht geirrt. Leicht lächelnd sah er auf die Kugel und ließ sie wieder in der Luft schweben.

\enquote{Ich glaube, ich habe es verstanden.}

Professor Elber lies seine Kugel in die Luft schnellen und fragte dann: \enquote{Wo kommt sie auf?}

Harry dachte kurz nach, stand schnell auf und die Kugel schlug hinter ihm durch das hintere Brett der Bank.

Mit einem einfachen Zauber reparierte Elber die Bretter wieder und machte sich wortlos auf den Rückweg.

Harry wusste, dass die Stunde jetzt zu Ende war. Er grinste in sich hinein, dass er es wieder einmal geschafft hatte. Nur musste er im Unterricht aufpassen, dass er nicht zu viel zeigte. Er hatte teilweise mehr Zauber drauf als seine Mitschüler. Sein Privatunterricht hatte sich bezahlt gemacht. Aber er wusste immer noch nicht, worauf die Übungen abzielten. Trotzdem war er froh, nach einigen Terminen endlich den richtigen Dreh raus zu haben.

Einige Tage später, Madame Pomfrey verarztete gerade die Hand von Elber, fragte sie: \enquote{Was hast du wieder angestellt?}

\enquote{Ich war unvorsichtig und tat in der Bibliothek einen falschen Griff und \gst}

\enquote{Hast nicht aufgepasst}, vervollständigte Madame Pomfrey den Satz.

Die Tür ging auf und zwei elegant und leicht adelig-arrogant wirkende Personen betraten den Raum. \enquote{Wo ist sie?}, fragte der Mann in einem Ton, aus dem man anwidern heraushören konnte.

Hinter ihnen lief Professor Sprout und drückte sich in eine Ecke, um nicht aufzufallen.

Madame Pomfrey stand auf und begrüßte die beiden Personen. \enquote{Ich nehme an, Sie sind die Eltern von Miss Chapel.}

Die Gesichtszüge der beiden versteinerte sich. \enquote{Wir möchten sie sehen, um uns von ihrem Zustand zu überzeugen}, antwortete die Frau scharf.

\enquote{Sind Sie ihre Eltern? Denn sonst darf ich Sie nicht durchlassen.}

Die beiden sahen sich kurz an. Dann nickte die Frau Madame Pomfrey zu. \enquote{Wir sind ihre Eltern}, antwortete sie und schluckte.

Skeptisch führte sie Madame Pomfrey zum Bett mit dem Vorhang und zog ihn vorsichtig zurück. Dahinter lag Katharina mit ihrem Handtuch um den Kopf gewunden auf dem Bett und hatte ihre Augen zu.

Sie öffnete sie und sah ihre Eltern an. \enquote{Mama, Papa}, begrüßte sie die Beiden, doch diese sahen sie nur streng an.

\enquote{Stimmt es, was man uns schriftlich mitteilte?}

Katharina nickte nur. Sofort zog ihr Vater seinen Zauberstab und richtete ihn auf das Handtuch. Nach einem gemurmelten Zauber und einem kurzen intensiven rosa Schimmer um das Handtuch herum,  zog ihre Mutter ein Päckchen aus ihrem Umhang heraus, warf es auf das Bett und sagte: \enquote{Du bist nicht mehr unsere Tochter. Wage es ja nicht, nach Hause zu kommen. Zu deinen Großeltern brauchst du ebenfalls nicht zu kommen.}

Und ihr Vater fügte hinzu: \enquote{Du bist kein Mitglied unserer Familie mehr. Wir möchten kein solches\abs in unserer Familie haben.}

Dann drehten sich beide um und verließen ohne ein Wort den Krankenflügel, um danach das Schlossgelände zu verlassen.

Katharina liefen wieder die Tränen übers Gesicht. Madame Pomfrey war zu geschockt, um etwas zu unternehmen, so kam Professor Sprout aus ihrer Ecke und nahm sich Katharinas an.

\enquote{Weinen Sie nicht, Miss Chapel. Wissen Sie, ich habe keine Kinder und wollte schon immer welche. Wenn Sie wollen, dann nehme ich Sie bei mir auf. Ich habe Sie die letzten Jahre kennengelernt und möchte Sie gerne adoptieren.}

Katharinas Tränen versiegten. \enquote{Wirklich?}, fragte sie ganz entgeistert.

\enquote{Ja Katharina, wenn du möchtest, gerne, dann bin ich aber ab sofort Pomona für dich, denn ich nehme an, dass du mich nicht Mum nennen wirst.}

Katharina wollte gerade ja sagen, da unterbrach sie Madame Pomfrey: \enquote{Sie müssen wissen, dass wenn Sie jetzt ja sagen, die magische Bindung sofort ausgeführt werden wird.}

Katharina überlegte nicht eine knappe Minute und sagte dann schließlich: \enquote{Ja, gerne \gst Pomona} und wurde leicht rot.

Diese nahm sie in den Arm und beide begannen leicht zu leuchten.

Als das Leuchten nachließ, erhob sich Professor Sprout und meinte: \enquote{Ich muss jetzt in den Unterricht, aber nachher komme ich wieder vorbei und hole dich ab. Dann unterhalten wir uns bei mir. Da ich jetzt für dich verantwortlich bin, werden wir die Ferien miteinander verbringen. Also werden wir uns erst einmal richtig kennenlernen.}

Katharina nickte und sah ihrer neuen Adoptivmutter gedankenversunken nach. Dann wanderte ihr Blick Richtung Professor Elber und sah ihn an. Es schien, dass ihr Blick ins Leere lief und sie ihn gar nicht richtig wahr nahm. Dann hörte sie eine Stimme in ihrem Inneren. \stimme{Ich habe mich schlau gemacht. Ich kann dir nicht helfen.} Katharinas Mut sank wieder. Und sie dachte nach, wollte ihren Entschluss, sich adoptieren zu lassen, rückgängig machen. Sie wollte Professor \gst Pomona nicht damit belasten, nie ihre Haare sehen zu können. \stimme{Aber jemand anderes kann dir helfen.} Ihre Augen schienen jetzt die ihres Lehrers zu finden. \stimme{Du musst nach Griechenland reisen, in den Tempel der Medusa. Der dortige Orden kann dir helfen. Pomona kann dich bis zum Eingang des Tempels begleiten. Ab da musst du alleine durch, aber du wirst im Inneren des Tempels von einem Führer begleitet. Du wirst dort, wie die Mönche, ins Innerste des Tempels vordringen, um Hilfe zu erlangen.}

Jetzt fixierten Katharinas Augen die von Professor Elber und starrten ihn an. Dann hörte er in seinem Geist: \stimme{Ich hätte Professor \gst Pomona nicht mit dieser Aufgabe belasten sollen.}

\stimme{Unsinn. Sie macht das gerne. Denke daran, dass sie dich trotz deines Zustandes akzeptiert hat und immer noch aufnehmen will.}

\stimme{Da hast du vermutlich recht.} Sie merkte nicht, dass sie ihren Lehrer duzte. Als es ihr auffiel, wurde sie rot und versuchte eine Entschuldigung zu stammeln.

\stimme{Mach dir nichts draus, Katharina. Wir unterhalten uns gedanklich. Das ist die wohl intimste Art einer Unterhaltung zwischen zwei Menschen. Wenn wir uns hier also duzen, ist das mehr als nur ok. Ich persönlich sehe es als Voraussetzung. \gst Du hast die Art der Kommunikation aber sehr schnell gelernt. Oder hast du vorher schon geübt?}

\stimme{Ich hatte vorher keine Ahnung davon.} Sie verlor ihre Scheu. \stimme{Frederick.}

Er lächelte sie leicht an, und legte sich danach hin und ließ die Salbe unter seinem Verband, um sie wirken zu lassen, da er mit seinem Arm eh nichts tun konnte. Katharina legte sich ebenfalls hin und ließ ihre Gedanken schweifen.

\stimme{Denk etwas leiser, Katharina}, hörte sie in ihrem Geist und meinte ein leises Schmunzeln zu erkennen.

\enquote{Was haben Sie gehört?}, fragte sie nach.

\enquote{Nichts, ich wollte Sie nur Testen, damit ich weiß, wie lange Sie das schon machen. Außerdem war es ein netter Scherz, oder?}

Katharina musste lächeln und schloss ihre Augen. Auf merkwürdige Weise fühlte sie sich mit ihren beiden Lehrern und der Krankenschwester verbunden.

\enquote{Wir werden Sie überwachen, damit wir einen Hinweis erhalten, falls die Mönche und Schwestern Ihnen nicht helfen können. Sie bekommen einen Zauber auferlegt der es ihnen ermöglicht, wieder zurückzukehren. Einen internen Portschlüssel quasi.}

\trenn

Leicht schwankend tauchten sie aus dem Nichts mit einem Dreher auf und Pomona ließ den unansehnlichen Nachttopf fallen, um ihr eigenes Gleichgewicht zu finden. Katharina ging sofort in die Knie, um den Schwung abzufangen und drehte sich einmal um ihre eigene Achse. Pomona hatte nicht so viel Glück und musste sich mit einem Baum bremsen, auf den sie zu steuerte.

Ein paar Meter weiter weg wartete ein in ein weißes Gewand gehüllter Mann und empfing sie freundlich. \enquote{Wenn sie mir bitte folgen würden.} Er drehte sich um und ging in Richtung eines Waldes. Katharina und Pomona folgten ihm. Schweigend liefen sie mehrere Minuten hinter ihm her, bis es an einen Platz im Wald ging, an dem eine Luke im Boden etwas verdeckte. Der Mann öffnete die Falltür und stieg hinab. Die beiden Frauen folgten ihm. Unten angekommen ging es einen Gang entlang. Nichts als festgetretener Dreck war zu sehen; am Boden, an den Wänden und dem bogenförmigen Dach des Tunnels. Sie kamen durch eine Tür in eine Kammer mit vielen Stühlen und Bänken. \enquote{Die Begleitung wartet hier bitte, der Prüfling durchquert bitte diese Tür. Ihr Wegbegleiter erwartet Sie dort drin \gst viel Erfolg.} Dann ging der Mann wieder den Gang entlang und verließ die beiden.

\enquote{Hör zu, wenn du willst, dann belege ich dich mit ein paar Schutzzaubern. Sie überwachen deine Vitalfunktionen und wenn es kritisch wird, dann kann ich dich damit zurückholen.}

\enquote{Damit würde ich mich wirklich besser fühlen, nach alldem, was ich bisher über solche Art von Ritualen gelesen habe. Ich werde vermutlich Prüfungen bestehen müssen, die arg an die Substanz gehen. Körperlich, wie auch seelisch. Es wird mir sicher helfen, jetzt da ich meine Familie ver\aabs Ich meine, jetzt, da ich eine neue Familie gefunden habe.}

Pomona nickt und zog ihren Zauberstab. Sie überzog Katharina mit mehreren Schutzzaubern, belegte sie mit dem Rückholzauber und lief danach zu den Stühlen.

\enquote{Ich gehe dann mal. Eine Prüfung wartet auf mich}, sagte Katharina ganz entschlossen.

\enquote{Na dann}, sagte Pomona fröhlich, \enquote{du weißt, wie du zurückkommst?}, zog sich ein Buch heraus und setzte sich.

Katharina wurde zunehmend nervös und öffnete vorsichtig die Tür, um zu sehen, was dahinter wäre.

\trenn

Zeitgleich in Hogwarts lief Professor Elber sichtlich angespannt im Lehrerzimmer auf und ab.

\enquote{Nun setzt dich endlich, Frederick}, mahnte in Professor McGonagall.

\enquote{Ich kann nicht, ich hätte bei ihr bleiben sollen, hätte sie begleiten sollen}, antwortete er.

\enquote{Aber Pomona ist bei ihr. Sie ist jetzt für sie da und kümmert sich um sie.}

\enquote{Du verstehst das nicht, Minerva.} Er blieb stehen. \enquote{Meine Linie ist für ihren Zustand verantwortlich.}

\enquote{Wie meinst du das?}

\enquote{Meine Linie warf den Fluch auf die erste Medusa. Ich fühle mich für sie verantwortlich.}

Harry hingegen traf sich mit Professor Snape und Ron, Hermine und Ginny. Ihnen wollte er zeigen, was er über Salazar Slytherin herausgefunden hatte. Nachdem er sie hatte versprechen lassen, nichts zu sagen, nahm er sie mit Richtung Slytherin Gemeinschaftsraum. Vor dem Porträt von Slytherin angekommen, begrüßt dieser ihn jetzt etwas freundlicher.

\gedanke{Hat Sal also mit ihm gesprochen.}

\enquote{Hallo Harry, wen bringst du denn da mit?}

\enquote{Freunde\abs mehr oder weniger.}

Das Bild grinste ihn an. \enquote{Wollt ihr rein?}

Harry nickte, worauf das Bild von Slytherin zur Seite schwang.

\trenn

Katharina trat in eine große Höhle mit schweren steinernen Säulen, welche die Decke hielten. Symbole, die sie nicht kannte, waren in den Stein gemeißelt. Sie trat um eine der Säulen herum und entdeckte zwei verhüllte Gestalten von hinten, die wie Priester oder Gelehrte aussahen. Sie wollte sich ihnen gerade nähern, als sie von hinten Geräusche hörte. Sie drehte sich um und lief einige Schritte auf den Mann zu, der sich ihr zu nähern schien. Aber dieser beachtete sie nicht und lief an ihr vorbei in einen der Gänge, die von der großen Höhle abführten. Also drehte sie sich wieder herum und versuchte die beiden Gestalten anzusprechen, welche sie zuerst gesehen hatte, als sie die Höhle betreten hatte. Doch diese gingen bereits in einen anderen Gang.

Da hörte sie eine Stimme. Nein, es war eher ein Schnaufen und Keuchen. Sie näherte sich den Geräuschen und entdeckte eine junge Frau. Sie dürfte gerade erst die Volljährigkeit hinter sich gebracht haben. Diese versuchte an einem Seil einen Eimer voll Wasser aus einem Brunnen zu holen. Katharina vermutete, dass es sich um einen Eimer handeln musste, denn sie konnte nur das Seil sehen, welches über eine Umlenkrolle in das Innere des Brunnens führte. Der gemauerte Brunnen zeigte verschiedene Wesen. Alle gleich groß und alle auf derselben Ebene abgebildet. Auf der linken Seite konnte sie noch das Hinterteil eines Pferdes erkennen. \gedanke{Vermutlich ein Zentaur}, dachte sie. Daneben war ein Zauberer, der einen Muggel anblickte. Katharina vermutete dies, da eine der beiden Personen einen Zauberstab in der Hand hielt, die andere nicht. Daneben standen auf kleinen, unterschiedlich hohen Hockern ein Elf und ein Kobold. Ganz rechts auf dem runden Brunnen erkannte sie nur noch einen Flügel.

Sie wollte lieber nicht wissen, welchem Wesen dieser Flügel gehörte.

Sie kam näher und ging der jungen Frau zur Hand. Dann zog sie mit ihr am selben Seil, doch es bewegte sich nichts.

Schließlich gab sie auf uns zog ihren Zauberstab, richtete ihn in den Brunnen und sprach einen \spruch{Accio}. Was auch immer am Ende befestigt war, kam nun hoch.

Eine schwere Statue, die einen asiatischen Mönch zeigte, brach aus dem Brunnen hervor und zerstörte den hölzernen Oberbau, der das Seil und die Umlenkrolle hielt. Diese reparierte sie mit einem weiteren Zauber.

\enquote{Das ist aber ein schöner Zauberstab, den Sie da haben}, sagte die junge Frau, nahm ihn in ihre Hand und untersuchte ihn. \enquote{Weißdorn und Einhornhaar}, erkannte sie richtig.

Katharina nickte.

\enquote{Gut, wenn man einen hat}, sagte sie und steckte ihn ein.

\enquote{He, das ist meiner, geben Sie ihn zurück.}

\enquote{Nein, das werde ich nicht}, sagte die junge Frau und schaute sie abwartend an.

Katharina war verblüfft. Doch dann überkam sie die Erkenntnis. \enquote{Sie sind mein Führer.}

\enquote{Führer, Begleitperson, Hilfe auf dem Pfad. Wählen Sie eine Bezeichnung.}

\enquote{Warum haben Sie sich nicht schon früher zu erkennen gegeben?}

\enquote{Wollen wir?}, sagte sie, Katharina ignorierend.

Katharina nickte und die junge Frau ging voraus. Sie gingen einen schmalen Gang entlang, der in einer kleinen Kammer endete. Dort warteten bereits ein paar Frauen, die sich sofort Katharina näherten, als sie eintrat und begannen, sie zu entkleiden.

Katharina schreckte zurück und wollte den Raum schon verlassen, doch ihre Führerin beruhigte sie. \enquote{Es ist alles in Ordnung, Katharina.} Also gab diese nach, ging zurück und ließ sich vorbereiten.

Mit den Händen über den Brüsten und den Kopf nach hinten zu ihrer Führerin gerichtet, fragte sie: \enquote{Würden Sie irgendeinen Schwur brechen, wenn Sie mir sagen, was das Ritual beinhaltet?}

Währenddessen fingen die Helferinnen an, sie mit Schwämmen zu waschen und mit Symbolen zu bemalen.

\enquote{Warum glauben Sie, dass ich das weiß?}

\enquote{Haben Sie es nicht auch schon durchlaufen?}

\enquote{Keine Angst. Ich werde Ihnen helfen, Ihren Weg zu finden. \gst Sagen Sie mir, sind sie vollkommen bereit, diese Reise zu unternehmen?}

\enquote{Ja.}

\enquote{Sind Sie willens, all das zu durchleben, was Ihre Vorgänger seit Jahrhunderten durchlebt haben, um Hilfe zu finden?}

\enquote{Ja.}

Die Helferinnen begannen Katharina in neue Kleidung zu hüllen, die weit und bequem war.

\enquote{Damit Sie im Notfall abbrechen können, Ihre Schutz- und Verfolgungszauber, die Ihre Adoptivmutter auf Sie gelegt hat wirken können und Sie nicht sterben können, falls es schwierig werden sollte?}

Katharina schaute sie entgeistert und leicht verängstigt an.

\enquote{Wir haben hier die Möglichkeit, jeden Zauber zu entdecken.}

\enquote{Ich hoffe, das ist kein Problem?}

\enquote{Es spielt nicht die geringste Rolle. \gst Sie halten viel von der Magie, nicht wahr?}

\enquote{Sie leistet mir gute Dienste.}

\enquote{Das glaube ich gerne. \gst Kommen Sie mit.}

Sie liefen einen weiteren Gang entlang und die Führerin ließ Katharina den Vortritt, um eine weitere Kammer zu betreten. Kaum war sie über die Schwelle getreten, schloss sich die Schiebetür hinter ihr.

Im Inneren saßen drei Personen in der gleichen Kleidung. Sie hatte lediglich eine andere Farbe. Eine Frau und zwei Männer. Ein kurzer Blick genügte, um zu erkennen, dass der letzte der Männer, der neben der Frau auf der Bank saß, etwas mürrisch war.

Erschrocken von dem Geräusch der sich schließenden Tür, drehte sich Katharina um. Doch sie hatte nicht viel Zeit darüber nachzudenken, denn einer der Männer unterbrach ihren Gedankengang.

\enquote{Wer sind Sie?}, fragte er barsch.

\enquote{Ich bin Katharina Chapel}, sagte sie, als sie sich wieder umgedreht hatte. \enquote{Ist hier der Beginn des Rituals?} Sie sah sich weiter um und entdeckte eine weitere Tür.

\enquote{Oh, das Ritual}, sagte der andere Mann. Er hatte eine tiefe angenehme Stimme. \enquote{Ja.}

\enquote{Wir warten}, sagte die Frau. \enquote{Kommen Sie, gesellen Sie sich zu uns.}

Katharina ließ ihren Blick von der Tür ab und wandte sich den sitzenden Personen zu. \enquote{Dürfte ich erfahren, worauf Sie warten?}, fragte Katharina nach.

\enquote{Wir warten nur}, antwortete einer der Männer.

Katharina verarbeitete die Informationen und überlegte kurz. \enquote{Und wie lange warten Sie schon?}

\enquote{Wie lange sitzen wir schon hier?}, fragte der Mann seinen Nebensitzer.

\enquote{Was fragen Sie mich}, antwortete er aufbrausend. \enquote{Ich habe kein Zeitgefühl.}

\enquote{Es ist schon eine Weile}, sagte der Mann zu Katharina gewandt. \enquote{So viel weiß ich.}

\enquote{Oh wir warten hier, solange wir zurückzudenken vermögen}, antwortete die Frau.

Das war ein Schock für Katharina. \enquote{Soll das bedeuten, dass ich mein ganzes Leben lang warten muss, bis ich das Ritual durchlaufen kann?}

\enquote{Wir haben lediglich gesagt, dass wir hier warten}, merkte der ruhige Mann an.

\enquote{Ich möchte nur begreifen, wie es funktioniert. Die Mönche, die ich durch die Pforte gehen sah, waren jung. Sie können nicht lange gewartet haben, bis sie die Rituale durchliefen.}

\enquote{Damit hat sie recht}, erwiderte die Frau.

\enquote{Sie ist ein kluges Köpfchen.}

\enquote{Sie hält sich jedenfalls dafür}, sagte der andere Mann bissig.

\enquote{Das ist ein Test}, folgerte Katharina. \enquote{Ich soll meine Entschlossenheit beweisen.}

\enquote{Ein Test?}, fragte die Frau. \enquote{Sie glaubt, wir seien ein Test? Wovon spricht sie überhaupt?}

\enquote{Offenbar mag sie Tests. Ich nehme an, Tests sind wichtig für sie.}

\enquote{Hat schon mal jemand versucht diese Tür zu öffnen?}, fragte Katharina nach einer kleinen Pause.

\enquote{Wie oft sollen wir Ihnen noch sagen, dass wir hier lediglich warten}, raunzte einer der Männer.

\enquote{Meine Liebe, wieso setzen Sie sich nicht einfach zu uns und entspannen sich? Sie sind viel zu verkrampft.}

\enquote{Ich möchte nicht warten, ich habe zu Hause Freunde und Menschen, die mir etwas bedeuten. Ich kann nicht warten. Ich muss das Ritual durchlaufen.}

\enquote{Ich frage mich, ob sie immer vor Ungeduld platzt.}

\enquote{Oh, sie hat nur einen starken Willen. Sie möchte, dass es endlich vorwärtsgeht.}

Katharina wandte sich der Tür zu und klopfte, da sie sie nicht aufbrachte. Als die Tür aufging, staunte sie, da ihre junge Führerin sie hinter der Tür erwartete und sie begrüßte. \enquote{Ja?}

\enquote{Ich möchte nicht respektlos erscheinen, aber wenn ich hier nicht etwas tun muss, dann würde ich gerne mit dem Ritual weitermachen.}

\enquote{Wie Sie wünschen}, entgegnete die junge Frau und deutete mit ihrer Hand Katharina an, dass sie den Gang vorausgehen soll.

Sie gingen in einen Raum mit einer Empore, die rings herum von Stufen umsäumt war. In der Mitte war eine quaderförmige Vertiefung. Etwa so groß wie ein Mensch.

Katharina lief die Stufen hinauf und umlief die Empore. \enquote{Ich weiß nicht wie ich beginnen soll.}

\enquote{Sie glauben also, dass es nur darum geht zu tun, was man Ihnen sagt?}

\enquote{Nein, ich bin überzeugt davon, dass es meine Magie stärkt und den Fluch von mir nimmt. Ich habe viele Rituale studiert und viele haben Gemeinsamkeiten, aber dieses hier könnte komplett anders sein.}

\enquote{Das bezweifle ich nicht. Aber ist Ihnen bewusst, dass das hier alles bedeutungslos ist? \gst Wichtig ist einzig und alleine, dass Sie die Verbindung zur Magie finden.}

\enquote{Ich werde alles tun, um das zu erreichen. Aber ich bin nicht nur gekommen, um mich um meiner selbst willen von diesem Fluch zu befreien. Ich habe lieb gewonnene Menschen, die mir etwas bedeuten, versteinert. Ich bin für sie verantwortlich.}

\enquote{Ein ehrenwerter Grund. Ich hoffe, die Magie schenkt ihnen Gehör.}

Katharina konnte diesen Satz nicht nachvollziehen. \gedanke{Meine Magie gehört zu mir. Ich benutze sie. Wieso sollte sie mir nicht gehorchen?}

\enquote{Also, lassen Sie uns beginnen.} % 00:21:30

Die Führerin stieg die kleine Empore hoch und stellte sich gegenüber Katharina auf. \enquote{Also}, sprach sie. \enquote{Die erste Herausforderung \gst Stellen Sie sich so hin.} Sie stellte ihre Füße hüftbreit auseinander, schob den rechten Fuß nach vorne und ging leicht in die Knie. Katharina machte es ihr nach. Dann hob sie einen Stein auf, legte ihn in Katharinas Hand und zog die gefalteten Hände etwas zu sich. Der Stein ruhte nun in ihren übereinander liegenden Händen auf Brusthöhe.

\enquote{Nun \gst was sehen Sie.}

\enquote{Ich sehe \gst einen Stein}, antwortete Katharina.

\enquote{Und was weiter?}

Also sah Katharina den Stein intensiv an, atmete einmal kräftig durch und konzentrierte sich.

Unterdessen saß Pomona im Wartebereich und las an ihrem Buch weiter. Sie hatte inzwischen die Alarmzauber nach Hogwarts dupliziert, damit man auch dort zumindest etwas mitbekommen würde. Sie war gerade an einer spannenden Stelle, als einer der Alarmzauber anschlug und Sekunden später ein Pergament vor ihr erschien. Sie legte ihr Buch beiseite und las den Text.

Auf Hogwarts war Professor Elber aufgeregt, als ein Pergament vor Madame Pomfrey erschien. \enquote{Und, was ist mit ihr? Ist sie in Gefahr?}

\enquote{Nun mal langsam, Frederick. Lassen Sie mich erst mal lesen.} Gemütlich nahm sich Madame Pomfrey das Pergament vor. \enquote{Alles in Ordnung. Nur ein Anstieg der Milchsäure-Konzentration in den Streckmuskeln.}

\enquote{Also strengt sie etwas an!}, folgerte er. \enquote{Gefährliche Werte?}, fragte er nach.

\enquote{Nein, keinesfalls. Aber es könnte dem Fluch entgegenwirken, den sie in sich trägt.}

\enquote{Also nichts, was man als Ansatz für eine Heilung hernehmen könnte?}

\enquote{Nein.}

Zeitgleich hielt Katharina immer noch den Stein in ihren Händen. Schweiß rann von ihrer Stirn und die körperliche Erschöpfung stand ihr ins Gesicht geschrieben.

\enquote{Was sehen Sie denn jetzt?}, wurde sie gefragt.

\enquote{Ich sehe\abs immer noch\abs einen Stein.} Dann wurde sie ohnmächtig und viel auf eine weiche Unterlage.

Nachdem sie wieder wach wurde und einen neutral schmeckenden, aufbauenden Trank bekommen hatte, wurde eine weiße unbemalte Leinwand hereingebracht. Katharina musste sich setzen und anfangen zu malen. Am Boden standen Töpfchen mit Farben drin. Katharina griff hinein und bedeckte ihre Finger mit der blauen Farbe.

\enquote{Ich nehme an, Sie werden mir nicht verraten, was ich malen soll.}

\enquote{Das wäre zu einfach, nicht wahr? \gst Sie haben die Wahl. Malen Sie, wonach ihnen ist.}

\enquote{Eigentlich konnte ich nie malen. Meine Schwester war die Künstlerin in der Familie.}

\enquote{Und Sie die Wissenschaftlerin.}

\enquote{Ja das stimmt. Während andere Kinder spielten, löste ich Rätsel. Mathematische, sowie logische. Außerdem befasste ich mich mit unserer Familien-Geschichte, oder mit Zaubertränken.}

\enquote{Mathematik. Ich kann ihre Freude daran nachvollziehen. Ein Problem zu lösen, eine Antwort zu erhalten. Einen Trank zu brauen. Die Antwort ist falsch oder richtig. Ein Trank gut gebraut, oder Abfall. Es ist sehr absolut.}

\enquote{Ich fand es immer sehr befriedigend}, sagte Katharina und sah ihre Führerin an.

\enquote{Da bin ich mir sicher.}

Nach einer kleinen Pause, in der das Bild trocknete, wurde sie in eine Kammer geführt. Sie sollte ganz nach oben klettern und etwas herunterholen.

Katharina machte sich auf den Weg nach oben. Sie erklomm die erste kleine Ebene in drei Metern Höhe.

Langsam setzte sie einen Schritt über den anderen und kam stetig aber sicher voran. Doch je höher sie kletterte, desto unsicherer wurde sie. Immer schwerer kam sie voran und rutschte immer wieder mit Hand oder Fuß ab. Dann gab noch ein Felsvorsprung nach und rutschte ab, sodass sie nur noch an ihren Händen in der Wand hing und mit den Füßen sicheren Halt suchte. Doch sie schaffte es und setzte ihre Reise nach oben fort.

Doch oben war nichts. Nichts, außer einem schmalen Gang, der zu einer Treppe führte, welche Wendel-treppenartig nach unten führte. Unten angekommen musste sie sich gegen eine Mauer lehnen, die wie eine Geheimtür aussah. Sie war wieder in der Kammer und wurde von ihrer Führerin mit dem Stein in der Hand empfangen.

Katharina war ziemlich erschöpft, aber nahm es klaglos hin. Sie wollte sich schon Erleichterung verschaffen, indem sie sich kurz unter ihrem Turban kratzen wollte, aber sie erinnerte sich gerade noch rechtzeitig daran, dass unter ihrem Kopfschutz gefährliche Medusen waren und ließ davon ab. Sie wischte sich lediglich mit dem Ärmel den Schweiß von ihrer Stirn und stellte sich erneut hin, um den Stein zu halten.

Der Stein schimmerte im Licht und Katharinas Augen wurden immer schläfriger und schläfriger. Ihre Führerin betrachtete sie wartend und geduldig und sah zwischen Katharina und dem Stein immer wieder hin und her. Sie stand einfach nur da und wartete.

\enquote{Was haben Sie gesehen?}, fragte sie plötzlich.

\enquote{Ich weiß es nicht genau.}

\enquote{Beschreiben Sie es.}

\enquote{Das kann ich nicht.}

Währenddessen las Madame Pomfrey unermüdlich die immer neu erscheinenden Pergamente und gab Professor Elber gerade eine kleine Zusammenfassung ab, als Professor Sinistra den Raum betrat und sich anschloss.

\enquote{Was haben Sie?}, fragte Professor Elber Madame Pomfrey und war ganz zittrig. \enquote{Etwas nicht in Ordnung?}

\enquote{Es gab einen Anstieg bei ihrer Atmung und bei den neuralen Peptiden.}
% und den Adenosin-Triposphaden.}
%24:40

\enquote{Ist das schlimm?}, fragte er. Es nahm ihn sichtlich mit.

\enquote{Es weist auf schwere körperliche Anstrengung hin. Aber die neuralen Peptide sind sehr interessant. Die Sättigung ist auf einem höchst anormalen Niveau.}

\enquote{Ist das gut, oder schlecht?}

\enquote{Das weiß ich noch nicht, aber es könnte zu ihrer Heilung beitragen, oder sie zumindest erleichtern. \gst Es könnte den Fluch beeinflussen.}

\enquote{Also hilft es ihr?}

\enquote{Das weiß ich nicht und, hängen Sie nicht so an mir}, sagte sie erregt, da Professor Elber schon über ihrer Schulter hing und mitlas. \enquote{Gehen Sie. Ich melde mich, wenn es etwas zu berichten gibt.}

\enquote{Aber\abs}

\enquote{Nichts aber, raus hier.}

Folgsam trollte er sich und verließ die Krankenstation. \enquote{Ich sollte diese Medusen auf dem Kopf haben und nicht sie. Es waren meine\abs}

\enquote{\extase{Frederick}}, schrie Madame Pomfrey. \enquote{So etwas will ich nicht wieder von dir hören. Du hast daran keine Schuld. Denk nicht mal so etwas.}

Er nickte nur noch und verschwand um eine Ecke.

Katharina, die immer noch ihre Prüfungen meistern musste, bekam eine Tasse aufbauende Flüssigkeit. Dankbar nahm sie sie entgegen.

\enquote{Ich bin erschöpft}, brachte sie matt hervor. \enquote{Oh, danke sehr.}
%25:36

\enquote{Ihre Alarmzauber müssten inzwischen eine ganze Menge interessanter Informationen zu ihrer Adoptivmutter und nach Hogwarts geschickt haben.} Sie ging zu einer Wand und holte einen schmalen hohen Weidenkorb, der mit einem Tuch bedeckt war. Diesen stellte sie vor Katharina ab.

Bedrohliche Zisch- und leichte Rasselgeräusche kamen aus dem Inneren des Korbes.

\enquote{Was ist das?}, fragte Katharina, als sie ihren Becher leer getrunken hatte.

\enquote{Das ist ein Nesset. Sie sind in der Lage, Ihnen bei ihrer Suche zu helfen.}

\enquote{Dann bin ich bereit meinen Fluch zu verlieren?}

\enquote{Glauben Sie, dass Sie ihn dadurch verlieren?}

Sie legte ihren Becher beiseite. \enquote{Ja, das tue ich.}

\enquote{Sie haben recht \gst Kommen Sie, greifen Sie da hinein.}

Langsam kam sie dem Korb näher. Die rasselnden Geräusche wurden etwas lauter und auch das Zischen wurde etwas aggressiver, dachte sie sich. Ihre Hand näherte sich dem Tuch über dem Korb. Wieder zischte etwas kurz auf und sie zog ihre Hand etwas zurück.

\enquote{Wir hören sofort auf, wenn Sie wollen.}

\enquote{Nein, ich gebe nicht auf.} Mutig und mit geschlossenen Augen griff sie hinein. Der Nesset biss zu und Katharina zog ihre Hand schreiend heraus.

Sie rollte ihren Ärmel hoch und besah sich schwer atmend und mit Schmerz verzerrtem Gesicht ihre Bisswunde.

\enquote{Haben Sie keine Angst}, beruhigte sie ihre Führerin.

\enquote{Es verbrennt mich}, zitterte Katharina. \enquote{Oh, mein Brustkorb \gst Er wird immer enger.} Sie stöhnte und schrie, bis sie bewusstlos zusammenbrach.

Rückblenden-artig durchlief sie im Schnelltempo die Zeremonie. Das Entkleiden und bemalen mit rituellen Symbolen. Das Halten des Steins. Das Bemalen der Leinwand, und das Klettern an der Wand.

\enquote{Katharina, \gst Katharina}, hörte sie leise und weit entfernt. Doch mit der Zeit zunehmend klarer.

Als sie die Augen aufschlug, lag sie in der Vertiefung, auf dem Podest im ersten Raum.

\enquote{Ich sterbe}, brachte sie schwach hervor.

\enquote{Letzten Endes stirbt ein jeder.}

Dann schloss sich die Grube und es wurde dunkel.
%27:28

\trenn

Im Inneren von Salazars Räumen standen mittlerweile Harry, Ron, Hermine, Ginny und Professor Snape.

\enquote{Das hier sind Slytherins Räume. Ich habe sie vor ein paar Tagen entdeckt.} Harry setzte sich auf einen Sessel und wartete.

Seine Gäste wussten die ersten Sekunden nicht, was sie sagen sollten. Harry wartete geduldig ab.

\enquote{Das sind\abs die privaten Räume von Slytherin}, stotterte Ron.

\enquote{Das sagte ich bereits}, meinte Harry.

\enquote{Harry, diese Bücher hier\abs}, sagte Hermine ganz begeistert.

\enquote{Nur die linke Seite ist für dich interessant. Und da nur das oberste Drittel.}

\enquote{Wieso?}, fragte sie nach.

\enquote{Die anderen wirst du nicht lesen können.}

\enquote{Dann lerne ich die Sprache. Ich finde es schon heraus.}

\enquote{Du hast es bei meinem Schriftstück auch nicht geschafft. Also mach dir keine Hoffnungen.}

Ginny und Snape sahen sich während dessen still um und betrachteten den Raum.

\enquote{Da hinten ist ein Bad}, zeigte Harry auf die Tür. \enquote{Die Stufen führen in das Schlafzimmer, Toiletten und Gästezimmer.} Harry zeigte auf die Treppe.

\enquote{Und die andere Tür?}, fragte Professor Snape nach.

\enquote{Zu verschiedenen Räumlichkeiten. Unter anderem die Kammer, Gemeinschaftsräume und vermutlich auch die Räumlichkeiten der anderen Gründer.} Dabei sah er zum Bild hoch und sah Salazar leicht mit dem Kopf nicken.

\enquote{Wie sind Sie denn hier hereingekommen?}, fragte Snape.

\enquote{Durch das Bild}, antwortete Harry.

\enquote{Sie wissen, was ich meine}, gab er zurück und setzte sich in einen Stuhl neben Harry.

Die anderen drei machten es sich auf dem Sofa gemütlich. Kreacher erschien und brachte kleine Sandwichs und Kürbissaft, sowie Wasser. Dann verschwand er wieder.

Harry griff zu und biss ab. \enquote{Mein Amulett}, sagte er und zog es unter seinem Umhang hervor. \enquote{Irgendwie hat es mich geleitet.} Er wollte nicht sagen, dass ihm Salazar persönlich gesagt hat, wie er in seine Räumlichkeiten kommt.

\enquote{Waren Sie schon einmal hier?}

\enquote{Ja, eigentlich zweimal. Beim ersten Mal haben sie mich erwischt und ich musste wieder zurück. Etwas später habe ich es dann geschafft. Es wurde dann spät und ich habe hier geschlafen. Das Bett ist übrigens sehr bequem.}

\enquote{Das würde ich auch gerne mal probieren}, meinte Ginny und fing plötzlich an zu husten.

Harry lächelte sie leicht an. Ihm wurde warm ums Herz. Er konnte sich durchaus vorstellen, mit ihr hier zu schlafen. Eine Nacht zu verbringen, korrigierte er sich. \enquote{Ich kann dich gut verstehen, da ich ja bereits hier eine Nacht verbracht habe.}

\enquote{Sie haben hier geschlafen? Wann?}

\enquote{Vor ein paar Tagen.}

\parsel{Wer isst dass?}, hörte er plötzlich.

\parsel{Einer meiner Professoren und meine Freunde}, antwortete Harry der kleinen Schlange, die vor ihm ihren Kopf aus dem Bücherregal streckte.

Die drei auf dem Sofa erschraken und drehten sich leicht bleich im Gesicht um.

\enquote{Darf ich euch Marcel vorstellen? Ich kenne ihn ein paar Wochen.} Dabei schielte er zu Snape.

Dieser zog die Stirn kraus und fragte sich, wo er die kleine Schlange schon einmal gesehen hatte.

\enquote{Wollt ihr mal das Schlafzimmer sehen? Geht einfach hoch und schaut es euch an.}

Seine drei Freunde nickten und begaben sich nach oben.

Kaum waren sie weg, regte sich Snape. \enquote{Kenne ich die Schlange nicht irgendwo her?}

\enquote{Ja, nur ist das keine Schlange. Ich habe ihn mit einem Zauber belegt und ihm einen Trank gegeben.}

\enquote{Den, den sie einmal brauten, als ich noch etwas Zeit brauchte?}, folgerte Snape.

\enquote{Ja.}

\enquote{Dann ist das keine Schlange?}

\enquote{Richtig. Es ist ein Basilisk.}

\enquote{Warum haben Sie ihre Freunde weggeschickt?}

\enquote{Sie sollten es nicht erfahren. Es würde wohl keiner von ihnen verstehen.}

\enquote{Aber mir muten Sie das zu?}

\enquote{Sie haben ihn ja schon gesehen. Zwar nur in meinem Geist, aber dennoch. Sie haben ihn gesehen.}

\enquote{Ist das nicht gefährlich?}

\enquote{Nein}, kam vom Bild. \enquote{Es gibt viele Zauber, die mit Schlangen durchgeführt werden können. Einige davon, ausschließlich von mir stammende, behandeln Basilisken. Wie vielleicht bekannt sein dürfte, hatte ich einen. Ich habe viel mit einem gearbeitet und diese Zauber und Tränke an ihnen ausprobiert. Leider hatte ich dann diese Phase, wo ich einen ohne Behandlung im Schloss einquartierte. Bedauerlicherweise habe ich später vergessen, dass ich nur den Zauber für mich anwandte. Mir ist es nicht mehr aufgefallen, dass sie für andere gefährlich werden konnte}, sagte Salazar traurig.

\enquote{Wollen Sie damit sagen, dass es Ihnen leidtut, dass Sie den Basilisken hier im Schloss ließen?}

Diesen Satz bekamen auch Harrys Freunde mit, da sie von oben wieder herunterkamen.

\enquote{Ja, ich habe meine Meinung bezüglich des reinen Blutes geändert. Ich habe Jahre lang versucht, den angerichteten Schaden wieder zu beheben, stieß aber nur auf taube Ohren bei denen, die den Floh schon hatten. Ich habe versucht, den Schaden rückgängig zu machen. Erfolglos.}

\enquote{Wie? Sie sind gar nicht der, für den wir Sie halten?}, fragte Ron nach.

\enquote{Hast du gerade nicht zugehört?}, fragte Hermine.

\enquote{Doch, aber das klingt nur so unglaublich.}

\enquote{Gibt es irgendwo Beweise?}, fragte Snape nach.

\enquote{Nein}, antwortete Slytherin. \enquote{Ihr habt nur mein Wort. Und mein Tagebuch.}

\enquote{Das könnte auch gefälscht sein}, warf Hermine ein.

\enquote{Auf wessen Seite stehst du eigentlich}, fragte sie Ron bissig.

\enquote{Auf der Seite der Wahrheit. Wenn wir schon bei der Suche nach Beweisen sind, dann sollten wir auch stichhaltige haben. Tagebücher könnten gefälscht sein. Wir brauchen Beweise, die nicht aus Slytherins Besitz sind.}

\enquote{Hermine hat recht}, gab Ginny zu bedenken. \enquote{Wenn wir damit an die Öffentlichkeit wollen\abs}

\enquote{Nein}, warf Slytherin ein. \enquote{Bitte nicht. Ich habe schon vor langer Zeit damit abgeschlossen.}

\enquote{Aber ihr Haus hat einen besonders schlechten Ruf durch die Tatsache, dass sie, der Gründer dieses Hauses, ein Muggelhasser und Reinblutfanatiker waren und durch die Tatsache, dass besonders viele Schwarzmagier aus ihrem Haus kamen.}

\enquote{Dafür kann ich nichts.}

\enquote{Das hat aber dazu geführt, dass die Schüler Ihres Hauses ausgegrenzt wurden und immer noch werden.}

\enquote{Was meinen Sie?}

\enquote{Ich meine, dass Ihre Schüler von anderen gemieden werden. Ihr Haus wird praktisch ausgegrenzt. Das könnte Sie dazu veranlassen, sich der dunklen Seite zuzuwenden.}

\enquote{Dunkle Seite?}

\enquote{Verzeihung. Böses tun}, gab Hermine kleinlaut zurück.

\enquote{Ich habe Hunger}, sagte Ron plötzlich.

\enquote{Hunger? Du hast hier gerade etwas erfahren, was unser Weltbild über Slytherin über den Haufen geworfen hat und du denkst an Essen?} Hermine wollte etwas nach ihm werfen.

\enquote{Lass ihn einfach. Dann soll er zum Essen gehen. Wir können ja weiter reden}, meinte Harry und wandte sich dem Bücherregal zu. \enquote{Ist das Tagebuch hier? Und vor allem \gst darf ich es lesen?}

Salazar sah ihn prüfend an. Dann sagte er schließlich: \enquote{Ja und ja. Aber nicht jetzt. Komm später wieder. Und, lies es nur alleine.}

Harry nickte. Da Ron immer noch da saß, rief er nach Kreacher und ließ sich das Essen für fünf Personen bringen. Minuten später waren mehrere Elfen da und beluden den Tisch. Die fünf ließen es sich schmecken und gingen nach dem Abendessen zurück in ihre jeweiligen Räume.

Auf dem Weg zu Hagrid am nächsten Tag, fragten sie sich, was sie wohl heute dran nehmen würden. Dieses mal hatten sie keine Ahnung, denn Hagrid hielt dicht.

\enquote{Heute nehm’mer Skolks durch}, sagte der Halbriese. \enquote{Wer weiß was darüber?}

Harry, der nicht gerade dafür bekannt war, so etwas zu wissen, hob seine Hand.

\enquote{Du?}, fragte Hagrid ungläubig. \enquote{Dann lass mal hören.}

\enquote{Es ist nicht viel, aber Skolks haben zweierlei Umweltschutz. Im Sommer ein Gefieder und im Winter ein Fell. Sie ernähren sich von Blut und Schweiß, können aber auch, als einige der wenigen Spezies, Dementoren töten. In Bayern kennt man sie unter dem Namen Wolpertinger. Das ist auch ihre ursprüngliche Heimat. Dort haben sich noch ein paar kleine Kolonien gehalten. Sie werden von Hexen und Zauberern abgeschirmt und dort behütet. Bis heute sind sie im bayerischen und dem angrenzenden Österreich bekannt. Meist werden sie ausländischen Muggeln als ausgestopfte Attrappen verkauft. Skolks sind Mischwesen. Sie haben das Geweih eines Hirsches \gst aber etwas kleiner \gst Füße einer Ente, den Schnabel einer Katze und den Schwanz eines Bibers. So wird er zumindest überwiegend beschrieben. In der magischen Welt sind diese Wesen fast unbekannt.}

\enquote{Das nenns’de  nichts? Das hätt’ ich nich’ erwartet. Das gibt zehn Punkte für Gryffindor. Woher weiß’ denn das?}

\enquote{Ich bin über den Begriff gestolpert und neugierig geworden. Also habe ich in der Bibliothek nachgeschlagen.}

Hagrid nickte und lief zu einem kleinen Gatter. Als die Schüler darüber sehen konnten, sahen sie zum ersten Mal in ihrem Leben einen Skolk, oder wie ihn die Muggel auch nennen Wolpertinger.

\enquote{Eure Aufgabe is’ einfach. Stellt euch rein, lasst ein’ an euch lecken und zeichnet ihn. Wie, is’ mir egal. Aber beeilt euch, wir ham’ se nur heute. Morgen sin’ se weg.}

Also betrat die Klasse das umzäunte Gelände, setzte sich und holte ihre Zeichensachen hervor. Die Tiere näherten sich den Schülern und begannen ihnen den Schweiß von der Haut zu lecken. Einige bissen sogar kurz zu und leckten dann das Blut ab. Die Schüler begannen die seltsamen Geschöpfe zu zeichnen. Beim nächsten Male würde doch nur wieder stumpfe Theorie dran kommen. So genoss ein jeder das Tier zu streicheln, sobald er fertig war.




\begin{kommentar}
Katharina Chapel, die Schülerin mit den Schlangen auf dem Kopf (und die nach Kathryn Janeway benannt wurde), muss nach Griechenland und dort eine Prüfung ablegen. Diese wurde aus einer Folge von Star Trek Voyager übernommen, wo der Captain exakt dieselben Prüfungen machen musste. (Das Ritual 3x07)
\end{kommentar}

\begin{kommentar}
Katharina sagt während des Rituals folgenden Satz: »Meine Magie gehört zu mir. Ich benutze sie. Wieso sollte sie mir nicht gehorchen?«
Dieser unscheinbare Satz spannt einen Bogen bis zum Ende des nächsten Teils, wo Harry in die Mondbibliothek kommt.
\end{kommentar}
