\chapter{Eine Prüfung die ist (nicht) lustig}


Am nächsten Morgen, als Harry erwachte, war bereits die Hälfte der Klasse wach und wusch sich, die Jungs und Mädchen jeweils hinter spanischen Wänden. Harry wusch sich ebenfalls und zog danach seine Sachen wieder an. Die Tür öffnete sich und Professor Elber kam herein. \enquote{Alle fertig?}, fragte er fröhlich in die Runde. Er lief auf Professor McGonagall zu und unterhielt sich mit ihr. Harry konnte die Unterhaltung nicht hören, war sich aber sicher, dass Professor McGonagall nicht gerade vor Begeisterung sprühte. Sie sah im Gegenteil eher erschrocken aus. Er drehte sich wieder in den Raum und sprach: \enquote{Wir werden uns jetzt in den Prüfungsraum begeben. Dieser befindet sich im untersten Stockwerk.}

\gedanke{Die Gerichtssäle}, schoss es Harry durch den Kopf.

\enquote{Wir gehen nun alle geschlossen dort hin}, sagte Professor Elber. Er öffnete die Tür und ging voraus. Professor McGonagall machte den Abschluss. Wieder ging es durch unzählige Gänge und Treppen bis ganz hinunter in die Gerichtssäle. Dort angekommen, gab Professor Elber der Klasse wortlos zu verstehen, sie sollen sich doch bitte setzen. Danach verließ er den Raum. Professor McGonagall stellte sich jetzt vor die Klasse und erläuterte mit zitternder Stimme den Ablauf der Prüfung. \enquote{Ich werde vor ihnen zum Ziel apparieren. Sie kennen es bereits. Es befindet sich 100~km außerhalb des Ministeriums. Wir sind schon einmal dort gewesen. Die Prüferin wird Ihren Namen aufrufen und Sie treten vor in die Mitte. Danach apparieren Sie dorthin. Die Prüferin wird dann von mir mitgeteilt bekommen, ob Sie dort angekommen sind.} Sie zeigte mit einer Liste in die Höhe. \enquote{Danach ist der nächste dran. Sind alle dort angekommen, geht es den Weg zurück. Alles verstanden?}

Alle nickten. Die Tür ging auf und Professor Elber kam herein, dicht gefolgt von Narcissa Malfoy. Harry sprang sofort auf und zog seinen Zauberstab, doch Professor Elber war schneller und entwaffnete ihn. \enquote{Was denken Sie sich eigentlich dabei, Ihre Prüferin anzugreifen, Harry?}, fragte Professor Elber. Harry blieb das Gesicht stehen. \enquote{Was glauben Sie, wie viel Überzeugung es mich gekostet hat, Narcissa dazu zu überreden?}

\gedanke{Narcissa?}, dachte Harry. Sie stand immer noch dicht hinter Professor Elber, um sich zu schützen. \enquote{Aber Ihr Mann arbeitet für Voldemort.}

\enquote{Mrs Malfoy ist hier in ihrer Eigenschaft als zugelassene Prüferin des Ministeriums. Dass sie, oder jemand aus ihrer Familie, eventuell mit dem Feind im Bunde ist, steht hier nicht zur Diskussion.}

\enquote{Aber}, wand Harry ein.

\enquote{Kein Aber. Haben Sie sich jetzt wieder beruhigt?}

Narcissa Malfoy trat nun vor und sprach in einer liebevollen Art und weiße. \enquote{Ich werde euch nacheinander aufrufen und ihr werdet hier aus der Mitte des Raumes apparieren. Nachdem alle angekommen sind, werdet ihr euch auf den Rückweg machen.} Sie verlas jetzt den ersten Namen.

Der Aufgerufene stand auf und ging sichtlich unwohl in die Mitte des Raumes. Professor McGonagall disapparierte und die Prüfung begann. Der erste Prüfling disapparierte und die Prüferin sah auf ihre Liste. Es erschien ein kleiner Haken, worauf sie den nächsten Namen auf der Liste aufrief. Zunehmend wurde die Klasse weniger. Jetzt war Harry an der Reihe, er stand auf und ging hinab. Professor Elber gab ihm seinen Zauberstab wieder. Er steckte ihm ein, nahm seinen Platz ein und disapparierte. Er tauchte auf der anderen Seite wieder auf. Professor McGonagall sah ihn von allen Seiten an und machte danach einen Haken hinter seinem Namen auf der Liste. Es folgten noch fünf weitere Personen. Danach waren sie wieder komplett.

Es dauerte eine Weile bis Professor Elber erschien. \enquote{Soweit war der erste Teil für sie alle erfolgreich. Nun geht es den Weg wieder zurück. Du rufst die Personen auf, Minerva?} Sie nickte. Professor Elber verschwand wieder und Professor McGonagall rief kurz darauf die erste Person auf.

Währenddessen fand in Hogwarts eine Unterhaltung über den Verbleib von Minerva McGonagall statt.

\enquote{Wo ist eigentlich Minerva?}, fragte Professor Vector. \enquote{Ich habe sie seit heute Morgen nicht mehr gesehen.}

\enquote{Sie ist mit Frederick im Ministerium die Apparitionsprüfung abnehmen}, antwortete Dumbledore.

\enquote{Aber}, protestierte Professor Vector, \enquote{das Ministerium ist doch infiltriert worden.}

\enquote{Ich weiß}, antwortete Dumbledore, \enquote{aber das scheint Frederick nicht zu stören. Ihr kennt ihn ja, wenn er von etwas überzeugt ist, dann lässt er sich nur schwer davon abbringen. Ich wette, er hat seinen Spaß dabei ein paar aus dem Ministerium an der Nase herumzuführen.}

\enquote{Wenn das mal gut geht}, gab Professor Vector als Antwort zurück. \enquote{Sind die Schüler denn sicher?}

\enquote{Du kennst doch Frederick, er wird sie schon heil zurückbringen}, grinste Dumbledore.

Narcissa setzte gerade ihren letzten Haken auf ihrem Brett und sah danach auf die Gruppe. \enquote{Herzlichen Glückwunsch. Ihr habt alle die Prüfung bestanden.} Alle atmeten erleichtert auf. Sie wollte gerade den Raum verlassen, als ihr Professor Elber nachlief und noch etwas mit ihr abklärte. Sie nickte und ging dann voraus. \enquote{Draco? Sie können mit Ihrer Mutter schon vorausgehen.} Malfoy schaute etwas ungläubig, aber ging ihr hinterher. \enquote{Zeit zum Mittagessen. Alle mir nach.} Professor Elber lief Richtung Ausgang und die Gruppe folgte ihm.

Harry schoss es immer noch durch den Kopf. Er konnte es nicht fassen, dass Professor Elber Narcissa Malfoy zu kennen schien. Aber andererseits war da dieser seltsame Traum, in dem er seinen Lehrer gesehen hatte. Automatisch lief er der Gruppe hinterher. Erst als er in der Kantine stand, kam er langsam zur Ruhe. Professor Elber drehte sich zur Gruppe und sprach. \enquote{Setzt euch irgendwo hin, tippt mit eurem Zauberstab den Zettel auf der Stelle an, wo das Essen steht, welches ihr zu euch nehmen wollt und wartet.} Er setzte sich auf einen freien Platz und tippte auf den Zettel. Kurze Zeit später hatte Professor Elber sein Essen auf dem Tisch stehen. Sorgfältig und misstrauisch sah er immer wieder durch den Raum. Harry setzte sich ihm gegenüber, suchte sein Essen aus und sah dann gedankenverloren durch seinen Lehrer hindurch.

\enquote{Professor?}, fragte Harry.

\enquote{Ja.}

\enquote{Darf ich Sie etwas fragen?}

\enquote{Sie dürfen mich alles Fragen, Harry. Aber Sie werden nicht immer eine Antwort bekommen.}

\enquote{Woher kennen Sie Narcissa Malfoy?}

Professor Elber schmunzelte leicht. \enquote{Über Lucius}, war alles, was er sagte.

Die ganze Rückfahrt über dachte Harry über das nach, was er im Ministerium erlebt hatte. Er saß im offenen Abteil des Zuges und betrachtete gedankenverloren Professor Elber, wie er sich mit Malfoy unterhielt. Dann erinnerte er sich wieder an seinen Traum. Er spielte mit Lucius ja Schach. Man war das peinlich, wie er sich im Ministerium aufgeführt hatte. Etwa nach der halben Strecke zurück zum Schloss wurde der Zug plötzlich langsamer und hielt auf offener Strecke an. Harry zog es aus seinen Gedanken. Er ging an das bereits offene Fenster. Hermine hatte es geöffnet und sah nach draußen. Der Lokführer, sowie der Schaffner standen am Kopf der Zugmaschine und schienen sich über etwas aufzuregen. Der Schaffner kam zurück und stieg in Harrys Wagon ein. Er kam auf Professor McGonagall und Professor Elber zu und fragte die beiden: \enquote{Sie sind doch Professoren in Hogwarts! Würden Sie sich bitte mal die Schienen vor unserem Zug anschauen?} Die beiden standen auf und folgten dem Schaffner. Harry beobachtete wie die Drei an den Zugkopf gingen. Nach einiger Zeit kam der Schaffner zurück und verkündete: \enquote{Verehrte Fahrtgäste. Wir haben gerade ein kleines Problem, das vermutlich einige Zeit in Anspruch nehmen wird. Sie können sich außerhalb des Zuges die Beine vertreten. Aber gehen Sie nicht zu weit weg. Bleiben Sie in Hörweite.} Harry verließ den Wagon, um sich die Schienen anzusehen. Hermine begleitete ihn. Vorne angekommen sah er etwas, das ihn erstaunte.

Die Brücke, über welche sie sonst immer gefahren waren, war weg. Es sah so aus, als ob sie weggesprengt worden war.

\enquote{Was meinen Sie?}, fragte der Zugführer Professor Elber. \enquote{Schon zu einer Erkenntnis gelangt?}

\enquote{Die Brücke ist weg}, sagte er.

\enquote{Das sehe ich auch. Was sollen wir jetzt machen?}

\enquote{Den Zug sichern und nachdenken. \gst Wie viele Meter braucht es, damit der Zug seine Höchstgeschwindigkeit erreicht?}

\enquote{Etwa fünfhundertsechzig Meter.}

\enquote{Dann setzen Sie bitte um diese Distanz zurück und geben ein paar Meter dazu.}

\enquote{Was haben Sie vor?}, fragte der Lokführer skeptisch.

\enquote{Wir werden springen.}

\enquote{Springen?}

\enquote{Ja! Wir werden mit dem Zug}, er machte eine ausladende Bewegung mit seiner Hand, \enquote{über die defekte Brücke springen. Mit ein wenig Hilfe durch die Magie natürlich.} Dann drehte er sich zu Professor McGonagall um. \enquote{Darf ich dir assistieren?}, fragte er.

Plötzlich ploppte es hinter ihm und eine Person in schwarzem Umhang und mit einer Maske vor dem Gesicht tauchte auf. Reflexartig drehte sich Elber um und schockte die Gestalt und legte danach mehrere Zauber auf die nähere Umgebung.

\enquote{Wofür sind die denn?}, fragte McGonagall.

\enquote{Damit hier keiner überraschend apparieren kann. Sollte es einer versuchen, wird er neben ihn hier}, er zeigte auf die Gestalt, \enquote{umgeleitet und sofort geklammert.}

McGonagall nickte. \enquote{Dann wollen wir mal. Das habe ich schon lange nicht mehr gemacht}, meinte sie mit kindlicher Freude. Sie ging auf die Lok zu und bestieg das Führerhaus. \enquote{Ich wäre dann so weit}, sagte sie.

Elber nickte und machte mit magisch verstärkter Stimme eine Durchsage. \enquote{Liebe Fahrgäste, wir werden in wenigen Augenblicken wieder Fahrt aufnehmen. Bitte begeben Sie sich alle zurück in den Zug. \gst Wir werden zunächst etwas zurücksetzen und danach einen Sprung durchführen, da die Brücke vor uns leider durch Todesser zerstört wurde. \gst Bitte bewahren Sie Ruhe und denken Sie positiv. Denken Sie daran, dass alles gut gehen wird, das wird dem Zauber helfen zu wirken.} Dann nickte er dem Lokführer zu. \enquote{Setzen Sie bitte zurück. Ich werde während Sie fahren, die letzten Vorbereitungen treffen.} Dann stieg er auf das Dach und legte einige Zauber auf die Wagons.

Der Zug fuhr langsam rückwärts und hielt an, als Elber den letzten Wagon bearbeitet hatte. McGonagall stieg ein paar Stufen hoch, damit sie Elber Bescheid geben konnte. Sie hielt ihren Daumen nach oben, doch Elber fuhr mit seiner flachen Hand vor seinem Kehlkopf hin und her. Da während der kurzen Rückwärtsfahrt des Zuges nur noch zwei weitere Todesser aufgetaucht waren und es ansonsten still war, hatte er ein ganz mieses Gefühl. Er versuchte mit einem Zauber zu erkennen, was hinter der Biegung war, die gleich nach der Brücke war. Ein inneres Gefühl sagte ihm, dass da noch etwas sei, er wusste aber nicht genau, was.

Er ging wieder den Zug entlang nach vorne und blieb in der Mitte stehen. So hatte er eine bessere Sicht nach vorne. Er rief McGonagall zu, dass sie beginnen können. Sie sollte aber die Strecke im Auge behalten, da sie eventuell noch einen weiteren \accentuate{anderen} Sprung durchführen müssten.

Dann nickte Elber, setzte sich auf den Wagon und McGonagall gab dem Zugführer Bescheid loszufahren.

Der Zug nahm Fahrt auf und beschleunigte immer weiter, bis er mit voller Geschwindigkeit auf die Brücke zuraste. Als die Schienen ein Stück nach unten ragten und dann abrupt aufhörten, sank der Zug noch etwas und schwebte dann ruhig in dieser Höhe, bis er auf dem intakten Gleis zurückkehrte, wo sich die Lok und danach die Wagons wieder etwas anhoben. Als der letzte Wagon wieder auf den Schienen stand, fuhr die Lok um die Kurve. Doch es wurde nicht besser, denn die Schienen schienen mit den Holzschwellern verschmolzen zu sein.

McGonagall reagierte wie immer mit den Reflexen einer Katze und brachte den Zug zum Leuchten, was für Elber das Zeichen war, den Sprung einzuleiten. Der Zug verschwand mitsamt den Insassen in einem Wirbel und tauchte stehend etwa hundert Meter hinter dem Bahnhof in Hogsmeade auf, denn die Schienen waren bis zum Bahnhof mit den Schwellern verschmolzen.

Als McGonagall, Elber und der Lokführer ausgestiegen waren, öffnete der Schaffner die Türen der Wagons und half den Fahrgästen, die aussteigen wollten, aus dem Wagon heraus. Die Schüler stiegen ebenfalls aus und warteten, bis McGonagall oder Elber zu ihnen kam.

\enquote{Was meinen Sie?}, fragte der Zugführer Professor Elber. \enquote{Schon zu einer Erkenntnis gelangt?}

\enquote{Nicht direkt}, antwortete Professor Elber. \enquote{Ich vermute, da ist ein Zauber schiefgegangen. Man wollte den Zug anhalten, indem man die Schienen verschwinden ließ, falls wir das Hindernis mit der Brücke überspringen könnten. Oder so etwas in der Art.}

Professor McGonagall kam den Schülern entgegen. \enquote{Wir kehren jetzt alle zurück nach Hogwarts.} Sie führte die Schüler zu den Pferdelosen Kutschen. Harry musste jedes Mal schmunzeln, wenn er daran dachte, dass Thestrale die Kutschen zogen. Sie kamen gerade rechtzeitig zum Abendessen in die Große Halle. Alle staunten, als sie die Gruppe einmarschieren sah.

\enquote{Alle bestanden}, verkündete Professor McGonagall stolz. Harry dachte, er hörte eine gewisse Bitterkeit in ihrer Stimme und führte das auf Mrs Malfoy zurück.

Als Harry am anderen Morgen Richtung Hogsmeade ging, um sich die Schienen anzuschauen, bemerkte er, wie bereits Professor Elber auf dem Gleis stand und sie betrachtete.

\enquote{Guten Morgen Professor}.

\enquote{Guten Morgen Harry. Schon eine Idee?}, fragte ihn Professor Elber.

Harry war erstaunt. \enquote{Nein}, gab er zurück. Er wunderte sich darüber, dass ihn Professor Elber fragte.

\enquote{Guten Morgen Harry \gst Professor}, hörte er hinter sich.

\enquote{Guten Morgen Luna}, antworteten beide gleichzeitig.

\enquote{Mir war danach, hier herzukommen}, sagte sie.

\enquote{Sie beide haben sich wohl aus dem Schloss geschlichen?}

\enquote{Na ja}, antworteten beide unisono. \enquote{Ein bisschen.}

\enquote{Was meinen Sie?}, fragte sie Professor Elber. \enquote{Wie bekommen wir das wieder hin?}

\enquote{Das könnte ein schief gegangener Verwandlungszauber gewesen sein}, antwortete Luna.

\enquote{Interessante Idee. Und wie kommen Sie darauf?}

\enquote{Schlangen. Die Schienen erinnern mich an Schlangen. \gst Man hat wohl versucht, sie in Schlangen zu verwandeln.}

\enquote{Und was}, fragte sie Professor Elber, \enquote{gedenken Sie dagegen zu tun?}

\enquote{Entweder ein Rückführungszauber, oder den Zauber wiederholen. Damit würde ein unvollständig ausgeführter Zauber zu Ende geführt werden.}

\enquote{Guter Einfall, Luna. Aber welcher Zauber wurde angewandt?}, fragte sie Professor Elber.

Harry antwortete. \enquote{Wenn es einer von Voldemorts Leuten war, dann muss es ein Zauber gewesen sein, der zwei große Giftschlangen hervorbringen sollte. Dann würde ich einen Rückführungszauber anwenden.}

\enquote{Netter Einfall. Wer möchte es versuchen?}

Harry und Luna sahen sich nur an.

\enquote{Ich versuche es}, antwortete Luna. Sie nahm ihren Zauberstab und richtete ihn auf die Schienen. Professor Elber kletterte auf den Bahnsteig zurück und wartete auf Luna. \zauber{Revelatio Serpentino.} Es passierte eine Weile lang nichts. Doch dann begann das Metall sich zu sammeln und wieder eine Schiene zu formen. Der Vorgang wurde immer schneller. Bis er sich bei einem Meter Schienen pro Sekunde einpendelte.

\enquote{Gute Arbeit Luna. Wirklich gute Arbeit. Sie verstehen Ihr Handwerk.}

\enquote{Danke Professor}, antwortete sie und wurde leicht rot.

\enquote{Ich denke, ich werde Sie auch in erweiterter Magie unterrichten. Sie sind ein schönes Paar.}

Harry insistierte sofort. \enquote{Wir sind kein Paar mehr.}

\enquote{Das meinte ich auch nicht}, antwortete Professor Elber. \enquote{Ich meinte, bezogen auf die Magie. Zusammen können Sie viel bewegen. Sie scheinen sich gut zu ergänzen.}

Nach ihrer Rückkehr nach Hogwarts, meldeten sich Harry und Luna gewohnheitsmäßig auf der Krankenstation, um ihre Routineuntersuchung über sich ergehen zu lassen. Dieses Mal jedoch meinte Madame Pomfrey: \enquote{Mit Ihnen ist alles in Ordnung. Ich kann keine Interferenzen mehr feststellen. Es scheint so, als ob Sie von Ihrer Körpertauscherei geheilt seien.} Das musste er sofort Ginny erzählen. Harry fühlte sich zu ihr hingezogen, seit er seine Beziehung mit Luna beendet hatte. Er hatte bisher aber noch nicht den Mut gefunden, ihr seine Gefühle zu offenbaren.

\trenn

Es war bereits eine Woche vergangen, seit Harrys Aus- und Zusammenbruch in Dumbledores Büro. Er hatte seinen Rückzugsort nicht mehr besucht, sich aber mit Ron und Hermine über die Aufzüge unterhalten. Zusammen hatten sie schneller als Harry alleine herausgefunden, wohin die Meisten der Knöpfe gingen. Nur die, welche mit einem Passwort versehen waren, konnten sie nicht anfahren. So unter anderem auch Dumbledores Büro. Alle standen bei der letzten Runde im Aufzug und Ron drückte den Knopf, der zu Dumbledores Büro führte. Natürlich leuchtete die Fläche auf und erlosch danach. In der Nähe hörten sie Dumbledore und McGonagall sich über das Essen unterhalten. Es war Essenszeit.

\enquote{Gehen wir Essen}, sagte Ron und verließ den kleinen Raum. Hermine hakte sich bei ihm unter und zusammen verließen sie den Aufzug.

\enquote{Ich komme gleich nach}, sagte Harry, als ihn Hermine und Ron fragend ansahen.

Sie nickten und gingen vor.

Harrys Herz klopfte. Dumbledore ging zum Essen, also konnte er versuchen, ob er immer noch Zugang hatte. Er drückte den Knopf und legte seine Hand auf die leuchtende Fläche. Diese leuchtete grün auf und der Boden fing an zu vibrieren. Harry war auf dem Weg. Er verließ den Aufzug, als die Wand sich vor ihm öffnete und er Fawkes entdeckte. Er saß nicht auf seiner Stange neben Dumbledores Schreibtisch, sondern auf dem Geländer neben der Wand. Freudig ging Harry auf Fawkes zu und streichelte ihn. Nach ein paar Minuten drehte er sich um und sah die Wand an. Ihm wurde leicht mulmig, da er keinen Unterschied sah. Er hatte Angst, den normalen Weg zu gehen. \gedanke{Also gut Harry, du schaffst das.} Er setzte sich auf den Boden, schloss die Augen und konzentrierte sich. Nach wenigen Minuten sah er deutlich vor seinem inneren Auge einen Stein. Er öffnete seine Augen, stand auf und drückte den passenden Stein. Die Wand öffnete sich wieder und Harry betrat den Aufzug. Als er wieder in einem Seitengang war der zur Großen Halle führte, ging er direkt in den Saal zum Essen.

Freudig grinste er Ron und Hermine an. \enquote{Was?}, fragte ihn Ron mit vollem Mund.

\enquote{Ich weiß, wie man in Dumbledores Büro kommt.}

\enquote{Klar}, antwortete Hermine. \enquote{Durch die Tür.}

\enquote{Nein, mit dem Aufzug.}

Ron verschluckte sich fast an seinem Essen. \enquote{Was?}

\enquote{Ich habe einmal, als ich auf diesem Balkon war, Fawkes getroffen. Er hat mich dazu gebracht, diesen Knopf zu drücken. Und als nichts passierte, musste ich es erneut versuchen. Fawkes hat es geschafft, dass ich Zugang zu seinem Büro erhielt. Ich habe es gerade eben nochmal versucht. Fawkes hat mich erwartet und mich freudig begrüßt.} Dann lud er Essen auf seinen Teller.

\enquote{Du meinst, du kommst jederzeit in sein Büro?}, fragte Ron nach.

\enquote{Scheint so. Aber ich werde es nicht missbrauchen. Ich werde diesen Zugang nur verwenden, wenn es absolut nicht anders geht.}

Nach dem Essen wurde er draußen erwartet. Er lief neben seinem Professor her Richtung verbotener Wald. Dort angekommen blieb Professor Elber stehen. Harry musste noch etwa zehn Meter weiter laufen. Er bemerkte nicht, dass Professor Elber einige Schritte rückwärts lief.

\enquote{Was lerne ich heute?}

\enquote{Heute lernen Sie apparieren.}

\enquote{Hatte ich nicht vor Kurzem eine Prüfung?}

\enquote{Ja. Apparieren Sie auf mich zu.}

Harry tat wie ihm geheißen.

\enquote{Uff. Das war aber anstrengender als sonst.}

\enquote{Haben Sie die Fünf-Punkte-Regel angewandt?}

\enquote{Ja. Aber wieso war es dieses Mal etwas schwerer?}

\enquote{Ich habe einen leichten Widerstand erzeugt. Machen Sie weiter. Apparieren Sie zurück.} Harry drehte sich um und begann sich zu konzentrieren. \enquote{Haben Sie in der Zeit, in der Sie unterrichtet wurden, nichts gelernt?}, fragte Elber ihn plötzlich.

Harry drehte sich wieder um und sah ihn an. \gedanke{Mist}, dachte er. \gedanke{Apparieren, nicht umdrehen.} Harry nickte und konzentrierte sich kurz. Dann apparierte er wieder zurück an seinen Anfangspunkt.

Als er wieder am Ausgangspunkt angekommen war, sah er seinen Lehrer ganz kurz neblig. \gedanke{Das geht ganz schön auf die Augen}, dachte er sich.

\enquote{Apparieren Sie jetzt knapp hinter mich.}

\enquote{Was ist, wenn ich sie treffe?}

\enquote{Ich passe schon auf und werde entsprechend ausweichen}, sagte er und grinste ihn an. \enquote{Los.}

Harry tauchte knapp hinter seinem Lehrer auf.

\enquote{In welche Richtung schauen Sie?}

\enquote{Richtung Schloss.}

\enquote{Also von mir weg?}

\enquote{Ja.}

\enquote{Dann wieder zurück und mit dem Gesicht zu mir auftauchen.}

Harry apparierte wieder zurück. Von Mal zu Mal ging es leichter. Das merkte Harry. Er schnaufte ein paar mal durch und apparierte erneut. Jetzt sah er Elber auf den Rücken.

\enquote{Geklappt?}, fragte er knapp.

\enquote{Ja}, antwortete Harry.

\enquote{Also. Noch einmal zurück und dann ist kurz Pause.}

Harry nickte, was sein Lehrer aber nicht sah. Dann apparierte er wieder an seinen Ausgangspunkt. Als er sich umdrehte, verschwand gerade sein Lehrer.

\enquote{Hinter Ihnen}, hörte er.

Harry schreckte zusammen und drehte sich um. Elber hielt einen Korb in der Hand. Er reichte ihn Harry, nachdem er ein Sandwich herausgeholt hatte.

\enquote{Apparieren ist anstrengend}, erklärte er ihm.

\enquote{Das merke ich.}

Nach einer kurzen Pause ging es wieder weiter.

\enquote{Nehmen Sie diesen Beutel mit. Ich möchte sehen, ob es auch mit Gepäck klappt. Halten Sie ihn vor sich. So, als würden Sie einen Wäschekorb tragen.} Er ging wieder von Harry weg und drehte sich dann wieder um. Jetzt stand er jedoch weiter von Harry entfernt. \enquote{Los geht’s.}

Als Harry vor seinem Lehrer auftauchte, war dieser nicht mehr da.

\enquote{Hinter Ihnen}, hörte er aus einigen Metern Entfernung.

Überrascht drehte er sich um und sah seinen Lehrer an seinem Platz stehen.

\enquote{Wie haben Sie das gemacht?}, fragte er ihn.

\enquote{Alles eine Frage des Timings. Passen Sie auf und sehen Sie mir genau zu.} Professor Elber apparierte einen Meter nach links. \enquote{Ist Ihnen etwas aufgefallen?}

\enquote{Nein.}

\enquote{Wie lange hat das gedauert?}

\enquote{Etwa dreihundert Millisekunden}, schätzte Harry.

\enquote{Also langsam genug um selbst zu apparieren?}, fragte ihn Professor Elber.

\enquote{Wenn man darauf gefasst ist und darauf wartet, dann könnte das klappen}, antwortete Harry.

\enquote{Also versuchen wir es. Passen Sie auf. Wenn ich verschwinde und hinter Ihnen auftauchen werden, mit Blick Richtung verbotener Wald, also dieselbe Richtung, in die Sie jetzt schauen, dann werden Sie an meine Stelle apparieren und sich dabei drehen, damit Sie mich anschauen können. Verstanden?}

Harry nickte und Professor Elber apparierte. Fast hätte Harry seinen Einsatz verpasst und so war Professor Elber fast schon hinter ihm, als Harry zu apparieren begann.

\enquote{Etwas langsam, aber für den ersten Versuch gut. Jetzt hinter mich.}

Harry apparierte hinter seinen Lehrer.

\enquote{Gut, laufen wir ein Stück. Den Korb nehmen wir mit.} Elber hielt seine Hand ausgestreckt und der Korb tauchte mit dem Henkel in seiner Hand auf.

\enquote{Was war das?}, fragte Harry.

\enquote{Es nennt sich Distanz- oder Fremd-Apparieren. Man kann Sachen und Dinge, auch Menschen, Fremd-Apparieren. Es funktioniert auf Distanz. Stand in einem sehr interessanten Buch.}

Nach etwa hundert Metern sagte Professor Elber: \enquote{Mir ist das Laufen zu mühselig.} Er packte Harrys Hand. \enquote{Nehmen Sie mich mit?}

\enquote{Was meinen Sie? Soll ich Sie tragen?}

\enquote{Nein, Sie sind noch jung. Das Apparieren strengt Sie nicht so an wie mich.}

\enquote{Aber innerhalb des Schulgeländes kann man nicht apparieren.}

\enquote{Nicht?}

\enquote{Nein, das hat Hermine mir mehrmals gesagt. Das steht auch in \buchtitel{Hogwarts \gst Eine Geschichte.}}

\enquote{So. Tja}, grübelte sein Professor, immer noch Harrys Hand haltend. \enquote{Dann fragte ich mich, was wir die letzte Stunde gemacht haben!}

\enquote{Wir, das heißt ich, bin appariert. Aber außerhalb. Deshalb sind wir doch zum verbotenen Wald gegangen.}

\enquote{Sind Sie sicher? Schon Ihr erster Appariervorgang war von außerhalb des Geländes nach knapp innerhalb. Und im Verlauf der Stunde sind Sie immer weiter auf das Gelände vorgedrungen.}

Harry blieb stehen und zwang so seinen Lehrer es auch zu tun, da er ihn immer noch an der Hand festhielt.

\enquote{Können Sie nicht aus dem Laufen apparieren? \gst Ziel. Wille. Bedacht. Präzision. Bequemlichkeit?}, fragte er Harry.

\enquote{Sie wollen mich doch verarschen.}

\enquote{Trauen Sie mir das zu? Habe ich das jemals getan? Schätzen Sie mich für so fies ein? Harry, ich will Ihnen was beibringen, damit Sie sich gegen Voldemort und seine Todesser wehren können. Glauben Sie ernsthaft, ich will Sie damit hereinlegen? \gst Jetzt machen Sie schon. Ziel. Wille. Bedacht. Präzision. Bequemlichkeit.}

Harry resignierte. Erst glaubte er nicht daran, aber er wollte es zumindest versuchen, als er sich daran erinnerte, dass er es glauben musste. Ein Augenzwinkern später tauchten sie vor den Toren von Hogwarts und direkt hinter Dumbledore auf.

Dieser drehte sich erstaunt um, als er einen \geraeusch{Plopp} hinter sich hörte.

\enquote{Wo kommen Sie denn her?}

\enquote{Ich komme mit Harry. Ihn müssen Sie fragen, wo wir gestartet sind.}

\enquote{Wir kommen vom Schulgelände. Vom Rand des verbotenen Waldes sind wir etwa ein Viertel des Wegs bis hierhergelaufen und dann appariert.}

\enquote{Wie appariert? Ich dachte nicht, dass jemand anders\abs}, doch Dumbledore unterbrach sich.

Professor Elber grinste, was Harry jedoch nicht sah. \enquote{Ich gehe dann mal und trage den Korb in die Küche.} Damit verabschiede er sich und lies Harry mit Dumbledore alleine.

\enquote{Dich hat doch Professor Elber mitgenommen!?}, fragte Dumbledore nach.

\enquote{Nein}, antwortete Harry ehrlich. \enquote{Er bestand darauf, dass ich appariere. Um ehrlich zu sein, hätte ich nicht gedacht, dass es klappt. Aber die ganze letzte Stunde bin ich immer wieder durch das Feld appariert. Zwar nur von knapp außerhalb bis knapp innerhalb. Aber dennoch. Das hat im Nachhinein richtig Spaß gemacht, Albus.}

\enquote{Du erstaunst mich immer wieder.}

\enquote{Ich mich auch. Ich meine, ich bin einfach appariert, ohne mir was zu denken. Und als ich dann hier her apparieren sollte, dachte ich, er will mich veralbern. Dann erfuhr ich jedoch, dass ich die ganze Stunde über durch das Feld appariert bin.}

\enquote{Nimmst du mich eine Runde mit?}, fragte Dumbledore.

Harry griff wortlos zu und tauchte mitten in Hogsmeade vor dem Honigtopf auf.

\enquote{Woher wusstest du, dass ich Brausebonbons wollte?}

\enquote{Ich habe einfach gut geraten. Und offenbar richtig.}

Dumbledore betrat den Laden und holte sich einen neuen Vorrat an Bonbons. Dann ließ er sich von Harry zurückbringen.

\enquote{Hier, Harry.} Er reichte ihm eine kleine Tüte Bonbons. \enquote{Fürs Mitnehmen und wieder bringen.}

\enquote{Danke, Albus.}

Dann ging Dumbledore und Harry brachte seine \accentuate{Beute} grinsend in sein Zimmer. Dort dachte er über das nach, was er bereits gelernt hatte. Er konnte es nicht leugnen, er war mächtig geworden. Doch wieso war ihm das bisher nicht aufgefallen? Er spürte eine Art inneren Drang, nicht weiter darüber nachzudenken. Zuerst wollte er dagegen ankämpfen, aber eine andere Stimme sagte ihm, dass das nicht notwendig war. Ihn beschlich das Gefühl, dass die Magie selbst etwas dagegen hatte. Diese Ahnung lenkte ihn so ab, dass sein Widerstand bröckelte und er verstand, dass er niemandem etwas darüber erzählen konnte.

\trenn

Auf dem Weg zur Großen Halle bemerkte Harry eine Menge Schüler, die vor dem schwarzen Brett standen und sich heftig zu unterhalten schienen. Er kam näher und musste sich strecken, damit er auf den neuen Zettel dort sehen konnte.

\begin{brief}
Am Sonntag, dem 23.05, findet zum ersten Mal ein \enquote{Magie in Konzert} statt. Alle Schüler von Hogwarts werden dazu eingeladen und werden gebeten, festliche Zauberer- oder Hexenkleidung bzw. Schulroben zu tragen. Nähere Informationen am 23. zur Frühstückszeit.
\end{brief}

Harry wunderte sich. \enquote{Was soll das denn?}, fragte Ron.

\enquote{Ich nehme mal an, ein Konzert}, sagte Hermine.

\enquote{Du meinst, wegen des Namens?}, fragte Ron weiter.

\enquote{Ja, was sollte es auch sonst sein. Aber ich frage mich, welche Bands da spielen!}, meinte Hermine.

\enquote{Welche Bands?}, fragte Ron. \enquote{Was meinst du mit: \enquote{Welche Bands?} Das werden die \accentuate{Schwestern des Schicksals} sein, oder die \accentuate{Todesfeen}, so was in der Art.}

Hermine verschränkte ihre Arme \enquote{Und warum schreiben die das dann nicht hin?}, fragte sie. \enquote{Wenn laut deiner Meinung so wichtige und bekannte Bands da sind, warum schreiben sie es dann nicht hin?}

Harry machte sich auf den Weg in die Große Halle und setzte sich neben Ginny. Mit glänzenden Augen schaute sie ihn an. Er erwiderte ihre Blicke und drückte sanft unter dem Tisch ihre Hand, bevor er sie über den Tisch hob, um seinen Teller zu beladen. Es dauerte noch eine Weile, bis Ron und Hermine hereinkamen und sich auch an den Tisch setzten.

\enquote{Warum bist du gegangen?}, wollte Ron wissen.

\enquote{Ich wollte euren ersten offiziellen Streit nicht unterbrechen oder stören}, meinte Harry und zuckte mit den Schultern.

Ein paar Tage später, einen Tag vor dem großen Ereignis, stand Harry wie üblich auf, um seine Jogging-Sachen anzuziehen. Er verließ seinen Raum und wartete im Gemeinschaftsraum auf Ginny, die sich etwas verspätete.

Sie trafen sich mit Luna in der Eingangshalle vor dem großen Tor, öffneten es und begannen ihre übliche Runde zu joggen. Es war bereits seit geraumer Zeit wärmer und die Vögel zwitscherten ihnen begleitend zu. Sie joggten den schmalen Pfad hinunter, welcher zum Quidditch-Feld führte. Als sie näher kamen, bemerkten sie Musik, welche vom Quidditch-Feld zu kommen schien. Fragend schauten sich die drei an.

\enquote{Woher wohl die Musik kommt?}, fragte Luna.

\enquote{Ich nehme an, vom Quidditch-Feld}, sagte Ginny.

\enquote{Vom Quidditch-Feld?}, fragte Harry.

\enquote{Ja, das macht Sinn}, sagte Luna.

\enquote{Aber es kam noch nie Musik vom Quidditch-Feld}, meinte Harry.

Sie joggten ohne Unterbrechung weiter und kamen dem Quidditch-Feld immer näher. Normalerweise umrundeten sie es, oder liefen ein Stück den See entlang und drehten an der einsamen Buche um. Doch heute zog es sie zum Feld. Sie stiegen die Stufen hoch, um auf das Feld zu gelangen. Zuerst sahen sie nur ein paar Personen, die sich in der Mitte des Feldes unterhielten. Erstaunt blieben sie stehen. Harry erkannte Professor Elber, Professor Dumbledore und eine unbekannte Frau, die wie eine Geschäftsfrau mit Rock angezogen war. Sie unterhielten sich und die Frau schrieb sich Notizen auf ihren Block, den sie in ihren Händen hielt.

Professor Elber stand seitlich und blickte kurz herüber. Dann sagte er etwas zu Dumbledore und winkte die drei zu sich. Harry, Ginny und Luna kamen näher und schauten sich um. Überall standen dicke schwere Metallkoffer und Kabel. Stehlampen wie für Fußballübertragungen und Kameras waren über das gesamte Feld verteilt.

\enquote{Was ist das?}, fragten Luna und Ginny.

\enquote{Equipment für Fernsehübertragungen}, sagte Harry.

\enquote{Aber in der Nähe von Hogwarts\abs}, meinte Luna.

Doch Harry stieß ihr seinen Ellenbogen in die Seite und meinte: \enquote{Nicht hier. Die sehen mir nach Muggel aus.} Er zeigte auf einige Personen die wie Arbeiter bei einer Fernsehproduktion aussahen.

Sie kamen den dreien näher und die Frau fragte die beiden: \enquote{Schüler von Ihnen?}

\enquote{Ja}, antwortete Professor Elber. \enquote{Miss Lovegood, Miss Weasley und Mister Potter}.

Die Frau kam näher und schüttelte den dreien die Hände. \enquote{Miss Smith.} Sie zog drei Visitenkarten aus ihrer Tasche und überreichte sie den dreien. \accentuate{Fernsehveranstaltungen \gst Liveübertragungen \gst Gameshows}, stand auf den Kärtchen unter ihrem Namen.

\enquote{Fernsehveranstaltungen?}, fragte Harry.

Luna und Ginny schauten sich nur um und trauten sich nicht zu fragen.

\enquote{Ja}, begann Miss Smith. \enquote{Wir sind bei den Vorbereitungen für eine große Übertragung. Morgen findet hier das \enquote{Magie in Konzert} statt. Ich muss sagen, eine überwältigende Kulisse. Ein ideales Ambiente für ein Konzert.}

\enquote{Konzert?}, fragten alle drei und Luna und Ginny widmeten ihre Aufmerksamkeit wieder Miss Smith.

\enquote{Ja. Wussten Sie das nicht? Hier findet in ein paar Wochen\abs} und sie drehte sich mit erhobenen Händen im Kreis, als ob sie die Kulisse bewunderte, \enquote{ein großes Konzert statt.} Sie drehte sich zu Professor Elber und meinte: \enquote{Obwohl ich immer noch nicht weiß, wo sie so seltsame Gruppen wie die \enquote{Schwestern des Schicksals} herbekommen.}

Professor Elber schaute mit einem kaum merkbaren Lächeln zu Luna und Ginny. \enquote{Die habe ich auf einer meiner Reisen getroffen. Ich fand sie passend.}

\enquote{Welche Bands treten denn auf?}, fragte Harry.

\enquote{Wie? Welche Bands? Wissen Sie nicht\abs?}

\enquote{Wir hatten keine Ahnung}, sagte Ginny.

Professor Elber fügte hinzu: \enquote{Das soll eine Überraschung sein. Behalten Sie es bitte für sich. Dann dürfen Sie am Abend nach dem Konzert hinter die Bühne.}

Harry, Ginny und Luna nickten und schauten sich weiter um. Professor Elber widmete sich wieder Professor Dumbledore und Miss Smith. Harry, Ginny und Luna machten sich auf den Weg um ihren Morgensport fortzusetzen.

\enquote{Das hört sich gut an. Ein Konzert. Aber ich dachte immer, dass Magie sich nicht mit der Elektrizität der Muggel verträgt? Wie kann dann in der Nähe von Hogwarts ein derartiges Konzert stattfinden?}, fragte Luna.

Harry überlegte. Da  fiel ihm etwas ein. \enquote{Professor Elber muss einen Weg gefunden haben. Ich habe ihn vor einiger Zeit dabei beobachtet, wie er an einem Radio herumschraubte. Und es funktionierte. Ich sprach ihn darauf an und er meinte: \inner{Das ist nur der erste Schritt. Ich hoffe, dass ich da weiter komme. Da bin ich schon einige Jahre dabei, einen Weg zu finden unter bestimmten Umständen eine Art Symbiose zu erreichen.} Er hat es wohl erreicht, ein großes Areal so zu verzaubern, dass Elektrizität funktioniert.}

Sie joggten zurück, um sich zu duschen und um sich für das Frühstück umzuziehen. Gespannt warteten sie auf den Abend, an dem das Konzert stattfinden sollte.

Harry wechselte während er duschte mit Luna wieder in seinen Körper zurück. Glücklicherweise duschten die beiden in der Dusche ihrer jeweiligen Geschlechter, in dessen Körper sie gerade steckten. Er duschte also in der Jungendusche, wenn er in seinem Körper war und in der Mädchendusche, wenn er in Lunas Körper war. Sie tat es ebenso. So konnten beide außerdem kaschieren, wer gerade wer war und sich bei den anderen etwas umschauen. Beide hatten das beschlossen, nachdem sie einmal während einer ähnlichen Situation plötzlich ihre Körper getauscht hatten.

\trenn

\enquote{Kreacher}, schrie Harry, (er war wieder in Lunas Körper) als er sich in einem verlassenen Klassenzimmer eingesperrt und einen Schweigezauber auf die Tür gelegt hatte. Es dauerte eine Weile, dann erschien der alte Elf vor ihm. Sein Gesicht zum Boden gesenkt, um sich kurz zu verbeugen. \enquote{Kreacher?}

Der Elf sah zu ihm hoch und stutze. \enquote{Miss Luna, wie konnten Sie\abs}

Doch Harry hob seine Hand und unterbrach damit den Elfen. \enquote{Ich bin Harry. Ich habe auf irgendeine Art und Weise mit Luna den Körper getauscht. Das ist uns die letzten Tage mehrmals passiert. Ich habe dich bisher nur nicht gebraucht, wenn ich in ihrem Körper steckte.}

\enquote{Wie lange?}, fragte der Elf nach. Er sah nicht so aus, als ob er ihm glauben würde.

\enquote{Zwölf Tage etwa}, antwortete Harry.

Der Elf ging langsam und vorsichtig auf ihn zu. \enquote{Wenn mir Miss Luna ihre Hand geben würde?}, meinte er.

Harry reichte ihm seine Hand. Als der Elf seine Hand berührte, zuckte er zusammen und nahm seine Hand zurück. Er verbeugte sich abermals und sagte: \enquote{Verzeiht Kreacher, Sir Harry, dass ich euch nicht geglaubt habe.} Er wollte schon auf eine Wand zulaufen, um seinen Kopf dagegen zu schlagen, doch Harry hielt ihn auf.

\enquote{Kreacher, nein}, sagte er.

Der Elf drehte sich wieder um. \enquote{Wenn Sir Harry einverstanden ist, kann Kreacher mit ein paar anderen Hauselfen dafür sorgen, dass Sir Harry und Miss Luna wieder in ihren Körper zurückkommen.}

Harrys Augen weiteten sich. \enquote{Im Ernst? Geht das? Werden wir wieder Gefahr laufen zu wechseln?}

Kreacher antwortete: \enquote{Ja Sir Harry, Elfen sind dazu in der Lage. Aber nur Elfen, welche eine besondere Beziehung zum\abs} er überlegte kurz, was er sagen sollte, \enquote{getauschten haben, sind dazu in der Lage. Neben mir und Dobby sind noch drei andere Elfen dazu in der Lage. Sie finden Sie interessant. Das sollte genügen, um mit dem Prozess zu beginnen.}

\enquote{Gut, Kreacher, ich werde mit Luna in die Küche kommen. Ich muss erst schauen, wann sie Zeit hat.} \gedanke{Irgendetwas verheimlicht er mir. Aber was soll’s. Solange es mir nicht schadet.}

\enquote{Warten Sie aber nicht zu lange, Sir Harry, sonst könnte der Prozess nicht mehr umkehrbar sein, befürchtet Kreacher. Zumindest nicht für uns Elfen, für Zauberer oder Hexen schon. Dann aber mit Komplikationen. Eventuell großen Komplikationen.}

Harry nickte und winkte Kreacher zu, worauf dieser sich verbeugte und verschwand.

Harry musste wohl oder übel noch einige Stunden Unterricht über sich ergehen lassen, bevor er die Gelegenheit hatte Luna zu treffen. Hagrid hatte ihnen letztes mal erzählt, dass sie dieses Mal etwas besonderes durchnehmen wollten. Also machte er sich auf den Weg hinunter zu Hagrid. Er war einer der ersten, die dort standen. Leider auch Draco Malfoy mit seinen Kumpels. In der Öffentlichkeit waren sie immer bemüht nicht allzu freundlich aufeinander zu wirken, bis auf die Tatsache, dass er ein paar Mal Tamara zu Liebe eine bessere Miene ihm gegenüber an den Tag legte. Ihre Beziehung zueinander war zwar besser geworden, aber warm wurden sie sich deswegen trotzdem nicht. Sie ignorierten einander eher, als dass sie sich beschimpften.

Heute hatte ihnen Hagrid nur graue Theorie beigebracht. Er nahm etwas dran, was sie schon letztes Jahr einmal hatten, und sprach über dies und das. Spannende Tiere (für ihn spannend) und wie sie zu pflegen waren.

\enquote{Am liebsten würd' ich mit euch in'n Wald geh'n}, sagte er. \enquote{Zu den Zentaur'n.} Die Klasse reagierte geschockt. \enquote{Habt ihr 'n Glück. Dumbledore will's nicht. Sagt, das sei zu gefährlich.}

Die Klasse atmete erleichtert auf. Dann beendete Hagrid die Stunde und Harry lief neben Ron und Hermine zurück ins Schloss. Es war schön, ihnen zuzusehen, wie sie noch mit sich in der Öffentlichkeit kämpften.

\enquote{Brauchst du Hilfe bei Granger?}, fragte Draco spöttisch Ron. Harry drehte sich um und sah neben ihm Crabbe und Goyle, sowie Pansy Parkinson. Er wollte schon einen bissigen Kommentar abgeben, doch Ron war schneller.

\enquote{Wenn du, oder einer deiner beiden\abs Dienstboten andeuten wollt, dass\abs} doch Ron verstummte, da er Professor Snape aus seinem Augenwinkel sah. \enquote{Ich Hilfe brauche, dann sag mir wobei, und ich denke darüber nach.} Dann drehte er sich wieder um und zog Hermine sanft ins Schloss.

Harry sah zwischen Draco und Professor Snape hin und her. Er wusste, dass Draco so reagieren musste. Als Snape ihn ansah, trat er ebenfalls ins Schloss und verschwand in dessen Inneren. Er setze sich neben Luna und flüsterte ihr ins Ohr. \enquote{Ich habe heute mit Kreacher gesprochen.} Er belud sich seinen Teller mit einem saftigen Stück Fleisch und einer Portion in Butter geschmorten Maiskolben. Luna sah ihm zu. \enquote{Die sind in Butter geschmort}, sagte sie.

Die um sitzenden Schüler begannen zu kichern. \enquote{Ich habe, besser gesagt Kreacher, hat einen Weg gefunden, wie wir unseren normalen Zustand wieder herstellen können.}

\enquote{Ahm!}, gab sie als Antwort, da sie gerade den Mund voll hatte. Sie strich ihm über seine Wange und nickte ihm zu. Das tat sie immer, wenn ihr die Worte fehlten und sie ihm sagen wollte, sie hatte ihn verstanden.

Nach dem Essen gingen die beiden in die Küche, um Kreacher zu treffen. Dieser gab ihnen zu verstehen, dass sie hier am falschen Platz seien. Die fünf Elfen bildeten mit Harry und Luna einen Kreis und apparierten in ihr Zimmer im Gemeinschaftsraum der Paare.

Dort wies Kreacher die beiden an, sich gegenüberzustellen, die Hände gegeneinander und die Stirn aufeinander zu legen. Als sie sich in die richtige Position begeben hatten, stellten sich die Elfen so hin, dass sie jeden der beiden berührten. Harry und Luna wurde schwarz vor Augen und beide bekamen leichte Kopfschmerzen. Dann fühlte Harry sich, als würde er schweben, sein Geist löste sich von Lunas Körper und bewegte sich auf seinen zu. Dann war alles wieder normal. Die drei kleinen Elfen lösten ihre Hände von ihnen und verschwanden sofort. Kreacher und Dobby blieben noch eine kleine Weile. Kreacher hatte ihnen noch etwas mitzuteilen.

\enquote{Sir Harry, Miss Luna. Kreacher muss Ihnen noch etwas sagen. Sie sind durch die Reintegration geschwächt. Sie sollten die Nacht hier verbringen. Morgen früh sind Sie dann ausgeruht und alles ist wieder normal.} Die beiden Elfen verneigten sich leicht und verschwanden mit einem \geraeusch{Plopp.}

Nachdem sich beide ihre Pyjamas angezogen hatten, stiegen sie ins Bett und zogen die Decke über sich. Ihre Hände lagen über der Decke. \enquote{Es ist so, wie in unserer ersten Nacht hier, Harry.} Luna traf den Nagel auf den Kopf. \enquote{Meinst du}, fuhr sie fort, \enquote{wir haben unsere Körper nur deshalb getauscht, weil wir Sex hatten und unsere seelische Verbindung zueinander so stark ist? Meinst du, das hat zu Interferenzen geführt?}

Luna war sehr scharfsinnig. Schließlich war sie in Ravenclaw. Obwohl sie noch immer von vielen ihrer Mitschüler gemieden wurde, verstand Harry sie immer besser.

\enquote{Das könnte es sein}, antwortete Harry.

Die nächsten Wochen gingen ohne nennenswerte Tauschaktionen dahin. Es schien, als ob der Zauber der Elfen gewirkt hatte.

Als es einige Tage später zum Frühstück ging, war Harry schon gespannt, was passieren würde. Er war sich nicht sicher, was in letzter Zeit im Schloss vor sich ging. Seit einiger Zeit schienen alle Mädchen im Schloss ein eigenartiges Interesse an ihm zu haben. Sein Problem mit Luna war zwar gelöst, aber trotzdem konnte er sich darauf keinen Reim machen. Ihn hatten in letzter Zeit eine Menge Mädchen geküsst. Quer durch alle Häuser hindurch. Susan Bones genauso wie Elisabeth Marlow oder Lavender Brown. Auch Parvati und Padma Patil waren an ihm interessiert.

Er hatte ein Problem gegen ein anderes getauscht.




\begin{kommentar}
Kurz nachdem ich in einem Film einen Zug gesehen hatte, der von einer gesprengten Brücke fiel, hatte ich die Idee, die Rückfahrt vom Ministerium mit einer fehlenden Brücke zu schreiben. Da man immer wieder sieht, wie man in Filmen mit Autos eine große Distanz in der Luft überwinden kann, dachte ich mir, dass es schön wäre, dies mit einem Zug zu tun.
\end{kommentar}

\begin{kommentar}
Als Harry, Ginny und Luna auf dem Quidditch-Feld auftauchen, um zu sehen, woher die Musik kommt, treffen sie auf Miss Smith. Den Namen habe ich aufgrund der Aussprache so gewählt. Wegen des Doppel-s und dem th. Andererseits gibt es in Matrix einen Mr Smith. Eine schöne Doppeldeutigkeit, nicht?
\end{kommentar}
