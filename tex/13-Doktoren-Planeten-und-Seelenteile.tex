\chapter{Doktoren, Planeten und Seelenteile}

Im Inneren der Kabine angekommen meinte er: \enquote{Schon toll, was diese Muggel so alles erfunden haben. Immer wieder faszinierend, wenn man so davon hört, oder etwas darüber liest.}

Madame Pomfrey nickte nur. Philip nahm die Augenklappe ab und trat, von seiner Krankenschwester gestützt, heraus. Allerdings hatte es den Anschein, dass er seine Großmutter stützen würde. Sie meldeten sich an der Anmeldung zur Augenuntersuchung an und wurden gebeten, sich kurz im Wartezimmer zu setzen. Kaum saßen sie, wurden sie auch schon aufgerufen.

\enquote{Philip? Komm bitte mit deiner Großmutter mit. Zimmer drei.}

Philip stutzte kurz, reagiert dann aber schnell, da Madame Pomfrey nicht reagierte. \enquote{Komm schon, Granny. Wir sind dran.} Er zog an ihrer Jacke und sie reagierte.

Erst sah sie ihn entgeistert an. Dann schaltete es. \gedanke{Er improvisiert.} Sie lächelte ihn an und stand auf, um hinter ihm das Wartezimmer zu verlassen und danach das Zimmer drei aufzusuchen.

Auf dem Gang kam ihnen bereits der Herr entgegen und lief direkt hinter ihnen in das Zimmer. Er schloss die Tür und zeigte Philip seinen Platz. Dieser setzte sich und konzentrierte sich auf das Bild nur eines Auges, um dem Schwindelgefühl zu entgehen.

\enquote{Warst du schon einmal bei einem Optiker?} Philip schüttelte den Kopf. \enquote{Gut, dann erkläre ich dir, was dich heute erwartet. Ich werde nur deine Sehschärfe überprüfen.
Dann lege mal deinen Kopf auf diese Kunststoffwanne und drücke dein Kinn gegen die Begrenzung hier.}

Der Herr zeigte auf eine Apparatur, die zwei Sehlöcher hatte. Philip tat, wie ihm aufgetragen wurde, und legte sein Kinn auf die Stelle. Dann stellte der Optiker die Höhe ein, damit der junge Mann hindurchsehen konnte. Das linke Auge deckte der Optiker ab und schaltete einen Projektor ein, der Buchstabenreihen verschiedener Größe untereinander an die Wand warf.

\enquote{Lies bitte die dritte Zeile}, sagte der Optiker.

\enquote{A, F, H, E, G}, sagte Philip.

\enquote{Gut}, sagte er und verdeckte das andere Auge. Dann änderte er die Buchstaben. \enquote{Und nochmals die dritte Reihe.}

\enquote{B, G, R, O, P}, sagte Philip.

\enquote{Dann schauen wir jetzt mal, ob eine leichte Korrektur notwendig wäre. Ich werde jetzt nacheinander verschiedene Linsen vor das Auge setzen. Du sagst mir dann, ob es besser oder schlechter ist, Ok?}

\enquote{Ja.}

Der Optiker legte nacheinander Linsen vor die Augen und Philip sagte seine Meinung. \enquote{Schlechter, schlechter, schlechter, schlechter.}

\enquote{Wunderbar, deine Augen sind vollkommen in Ordnung.} Er klappte die Apparatur wieder weg und sah Philip an. \enquote{Keine Probleme. Null Dioptrien. Du brauchst keine Brille. Du und deine Großmutter könnt gehen.}

Philip stand auf und bedankte sich bei dem freundlichen Herrn. Dann gingen beide zurück zum Aufzug, und Philip legte seine Augenklappe wieder an. Sie verließen das Krankenhaus und kehrten zum Tropfenden Kessel zurück. Dort aßen sie erst einmal etwas und kehrten dann nach Hogwarts zurück.

\trenn

Es war wieder einmal Donnerstagmorgen, als Professor Elber gut gelaunt das Klassenzimmer betrat. \enquote{Es wird Zeit, dass wir anfangen zu üben. Dafür brauchen wir aber einen Raum. Ihr kennt euch hier aus und könnt mir doch sicherlich einen nennen.}

Stille herrschte im Raum. Harry summte in Gedanken das Lied vor sich hin und versuchte sich gerade an einem zweistimmigen Kanon.

\enquote{Wie wäre es mit der Großen Halle?}, fragte Lavender Brown.

\enquote{Dann bekommen wir Ärger mit den anderen, die wir kurz nach dem Frühstück herauswerfen müssen}, sprach Hermine. \enquote{Aber vor allem mit den Hauselfen. Die werden dann überfordert.}

Harry überlegte. Er musste einen Raum finden, der \gst plötzlich fiel ihm etwas ein. Aber würde es auch bei den anderen Anklang finden? \enquote{Wie wäre es?}, und Harry verließ plötzlich der Mut.

\enquote{Ja Harry, wie wäre was?}, fragte Professor Elber.

Harry druckste herum und meinte schließlich: \enquote{Wie wäre es mit der Kammer des Schreckens?} Die ganze Klasse erstarrte. \enquote{Schließlich wird sie nicht mehr gebraucht}, fügte Harry hinzu.

\enquote{Hm}, Professor Elber dachte nach. \enquote{Können Sie mir die Kammer mal auf die Tafel zeichnen?}, fragte er.

Harry stand auf und ging zur Tafel. Er malte eine grobe Skizze der Kammer auf und setzte sich danach wieder.

Professor Elber schaute sich die Zeichnung an, drehte sich um und meinte: \enquote{\accentuate{Kammer des Schreckens} \gst klingt irgendwie verboten und gefährlich. Wie lang ist die Kammer eigentlich? Und wie groß ist der Durchmesser des Hauptkreises?} Harry nannte Professor Elber die Daten, der sie auf die Tafel schreib, und erzählte ihm die Geschichte mit der Kammer und dem Basilisken, ließ aber Ginny außen vor, da noch immer keiner wusste, dass sie die Kammer geöffnet hatte.

\enquote{Hm. Das scheint ein interessanter Ort zu sein. Wo liegt die Kammer eigentlich?}

\enquote{Unter dem Schloss}, sagte Harry.

\enquote{Scheint mir ein guter Platz dafür zu sein}, meinte Professor Elber. \enquote{Und wie kommen wir dahin?}

Harry hatte diesen Punkt nicht bedacht. Er war sich nicht sicher, was er sagen sollte. Professor Elber lehnte sich wieder an sein Pult und schaute Harry an. Harry konnte schon Malfoys Blicke und gehässigen Sticheleien spüren, wenn er der ganzen Klasse sagte, wo der Zugang lag.

\enquote{Na ja}, begann Harry, \enquote{er ist im dritten Stock \gst im Mädchenklo.} Keine Reaktion kam aus der Klasse. Es war mucksmäuschen still. Harry wurde zunehmend unwohl.

\enquote{Woher}, begann Professor Elber.

Doch Hermine unterbrach ihn und meinte: \enquote{von mir.}

Professor Elber schaute sie an und zeigte ein leichtes Schmunzeln.

\enquote{Nun gut. Harry, Sie und Hermine kommen am Samstag nach dem Frühstück hierher und dann gehen wir gemeinsam dorthin und sondieren die Lage.} Er schaute wieder zur ganzen Klasse und fuhr fort. \enquote{Ich werde diese Kurse außerhalb der normalen Schulzeit halten. Alle Fünft-, Sechst- und Siebtklässler die dazu Lust haben können sich ab nächsten Montag in eine Liste vor der Großen Halle eintragen.} Er löste sich von seinem Pult und ging Richtung Tafel. Er schrieb die Worte \accentuate{Abwehr und Angriff} an die Tafel. \enquote{Was wisst ihr bereits aus dem Buch?} Er drehte sich wieder um und schaute durch die Reihen.

Hermine streckte wie immer die Hand in die Höhe, doch Professor Elber winkte ab. \enquote{Ich weiß Hermine, dass Sie das schon gelesen haben. Sie können ihre Hand herunternehmen.} Hermine senkte ihre Hand und lief leicht rosa an. \enquote{Sonst noch jemand?} Plötzlich streckte Draco Malfoy die Hand in die Höhe. \enquote{Ja Draco?}

\enquote{Krausets kann man sehr oft, aber nicht immer, an ihrem krausen Haupthaar erkennen. Ihnen kann man nicht ansehen, ob sie einen in die Irre führen wollen, oder einem beistehen. Oftmals ist es sogar so, dass sie plötzlich ihre Meinung ändern und ins genaue Gegenteil umschlagen. Viele von ihnen haben auch eine gespaltene Persönlichkeit, bei denen ein Teil einem Helfen, der andere aber einem Schaden will.}

\enquote{Sehr gut! Und wie kann man sich gegen sie zur Wehr setzen?}

\enquote{Mit \spruch{Anamorph, anatus, kressare}. Es bringt sie dazu, ehrlich zu bleiben. Auch wenn sie dadurch einem nicht helfen, falls sie gerade einen in die Irre führen wollen. Sie bleiben dann bei dieser Einstellung.}

\enquote{Wunderbar. Fünfzehn Punkte für Slytherin.}

Harry drehte sich nur um und schaute Draco leicht verärgert an. Dieser grinste Harry nur an.

\trenn

Nach dem Essen gingen Harry und Ron ihre Hausaufgabensachen holen, machten sich auf den Weg zu einem kleinen Innenhof innerhalb Hogwarts und setzten sich unter einen Baum. Ron nahm den Baum auf der anderen Seite des Weges, sodass sich beide anschauen konnten. Es war angenehm, obwohl es nicht besonders warm war. Die Sonne schien und erwärmte den Boden. Obwohl es Winter war, war es im Innenhof nicht besonders kalt. Irgendein Zauber verhinderte, dass sich das Klima mit dem außerhalb anglich. Professor Sprout hatte einige Pflanzen im Hof, die ständig gleiche Klimabedingungen benötigen. Harry hatte seine Hausaufgaben aufgeschlagen, als Luna sich ihm näherte und sich neben ihn setzte. Mittlerweile wusste die ganze Schule, dass die beiden zusammen waren. Seine Korrekturen für \fach{Geschichte von Hogwarts} hatte Harry bereits fertiggestellt und rechts neben sich gelegt. Harrys Blick wanderte über den Hof, als er Professor Dumbledore und Professor McGonagall nebeneinander herlaufen sah. Sie kamen an Harry vorbei und er fing ein paar Gesprächsfetzen auf.

\enquote{Ach Albus, ich glaube, ich hole mir bei Poppy ein Abführmittel. Ich kann schon seit Tagen nicht mehr}, sagte Professor McGonagall.

Doch Professor Dumbledore entgegnete ihr: \enquote{Nicht nötig, Minerva. Ich hab was Besseres}, und grinste.

Harry blieb der Atem stocken. Er schaute zu Ron und fing leicht zu grinsen an. Ron erging es genauso. Professor Dumbledore blieb stehen und griff in seine Tasche. Er holte zwei durchsichtige Plastiktütchen heraus und zeigt sie Professor McGonagall.

\enquote{Hier, Minerva. Die grauen sind bestens gegen Verstopfung geeignet. Oder vielleicht die schwarzen, die haben mich irgendwie mehr amüsiert}, sprach Professor Dumbledore. Als Professor McGonagall zu dem Tütchen mit den schwarzen griff, meinte Dumbledore nur: \enquote{Äh, ich glaube, Sie nehmen besser eines von den grauen. Die haben weniger pepp.} Sie hob eine Augenbraue und griff dann nach einem der grauen. \enquote{Und vergessen Sie nicht, sie erst kurz vor der Tür einzunehmen. Die wirken sehr schnell.}

Mit hochgezogenen Augenbrauen und einem ausdruckslosen Gesichtsausdruck ging Professor McGonagall, das Toffee fest verschlossen in der Hand, durch den Innenhof und verschwand in einem der Seitengänge. Professor Dumbledore drehte sich um und lächelte Harry an.

\enquote{Darf ich mich zu dir setzen?}, fragte Professor Dumbledore höflich. Harry nickte und Dumbledore nahm neben ihm auf dem Boden Platz und lehnte sich mit dem Rücken an den Baum. Er sah sich ein bisschen um und fragte Harry, nachdem er seinen Hausaufgabentitel gelesen hatte: \enquote{Darf ich mir die mal durchschauen?}

Harry nickte und Professor Dumbledore nahm sich die Rollen Pergament, die auf dem Boden lagen. Harry kümmerte sich inzwischen um seine anderen Hausaufgaben, als Professor Dumbledore ihn fragte: \enquote{Korrekturen für das Buch \buchtitel{Geschichte von Hogwarts}? Ich wusste gar nicht, dass das Professor Binns unterrichtet.}

\enquote{Das hatten wir auch nicht von Professor Binns auf}, antwortete Harry, \enquote{das war Professor Elber, er hatte Professor Binns vertreten. Und als er nicht wusste, was Professor Binns gerade unterrichtet, hat er uns gefragt was wir wissen wollen.}

\enquote{Und dann solltet ihr Fehler in der Geschichte Hogwarts finden?}, fragte Dumbledore.

\enquote{Nicht direkt}, antwortete Harry. \enquote{Er erzählte uns vielmehr etwas darüber, als wir ihn fragten, wie die Zauberei entstanden ist.}

Dumbledore drehte sich um und sah Harry in die Augen. \enquote{Ich glaube, ich sollte mich mal mit Professor Elber unterhalten. Mir scheint, er weiß mehr als er uns sagt. Aber was ist der Zweck dieser Hausaufgabe?}

Harry antwortete ihm. \enquote{Er hat uns aufgegeben, einen Korrekturzusatz zu schreiben, nachdem er uns einige Fehler im Buch aufgezeigt hatte, und die Besten gehen an den Autor für die nächste Ausgabe. Wir müssen aufgrund dieser Fehler die Einträge und Absätze im Buch korrigieren und dann abgeben.}

Dumbledore legte die Pergamentrollen wieder auf den Boden und drehte sich wieder zurück. \enquote{Ja, ich glaube, ich sollte mich mal mit ihm unterhalten.} Professor Dumbledore legte nun seine Hände auf seinen Bauch und entschloss sich, ein Nickerchen zu machen.

Etwas später hatte auch Harry seine Hausaufgaben beendet. Er räumte sein Tintenfass, seine Rollen an Pergament, die fertigen Hausaufgaben und seine Schreibunterlage in seine Tasche und verschloss sie. Luna legte ihren Kopf auf Harrys Schulter, lehnte sich an Harry und schloss die Augen. Harry fuhr ihr etwas durchs Haar, legte dann seinen Kopf gegen ihren und spürte ihr Haar an seiner Backe. Er schloss die Augen und schlief ein.

Harry fühlte sich eigenartig. Die Welt um ihn drehte sich und verlor sich in einem grauen Schleier. Nach einiger Zeit sah er wie sich seine Umgebung immer mehr herauskristallisierte und entdeckte Luna.

Sie drehte sich um und begrüßte Harry. \enquote{Hallo Harry.}

\enquote{Luna, was machst du hier?}

\enquote{Das hier ist mein Haus. Und da steht meine Mutter.}

Harry sah sich um und stand in der Küche. Die Decke und die Wände waren rußig und auf dem Küchentisch in der Mitte waren tiefe Brandflecken zu sehen.  Der Raum war kreisrund. Aus dem Tisch züngelten kleine Flammen. Harry vermutete, dass es magische Flammen sein mussten. Auf einem Stövchen darüber stand ein kleiner Kessel. Am Ende des Tisches war ein Kinderhocker mit einem kleinen Mädchen. Harry nahm an, dass es sich um Lunas Schwester handeln musste, da sie die gleiche Haarfarbe wie sie hatte. Um den Kinderhocker war eine Art Schildzauber gelegt worden. Wohl um zu verhindern, dass sich die kleine verletzte. Jetzt kam eine gut aussehende Frau um die Ecke und trug einige Kräuter auf dem Arm und legte sie auf den Tisch. Sie trug einen langen dunkelblauen Zauberumhang, der über ihre rot schimmernde Robe hing. Über dem Umhang lag ihr langes, schneeweißes Haar. Harry dachte an Luna. Sie hatte die Haare ihrer Mutter geerbt. Er trat etwas näher an Luna heran und nahm ihre Hand. Lunas Mutter sah auf, schwang ihren Zauberstab und holte damit ein paar weitere Kräuter aus einem Regal mit vielen Schubfächern heran. Ein Fach flog heran und blieb in der Luft schweben. Sie nahm wenige Zweige heraus und das Fach machte sich auf den Rückflug. Harry konnte ihr Gesicht sehen. \gedanke{Sie hat die Augen ihrer Mutter}, dachte Harry. Plötzlich brodelte und blubberte es. Harry fühlte wie Luna sich weiter an ihn schmiegte und seine Hand fester drückte. Die Funken im Kessel wurden immer stärker und Lunas Mutter griff wieder zu ihrem Zauberstab, mit einem leichten Ausdruck von Panik im Gesicht. Plötzlich explodierte der Kessel und warf Lunas Mutter zurück. Luna warf sich ihm nun ganz um den Hals und er spürte ihre Tränen auf seiner Haut. Die Welt fing wieder an sich zu drehen und Harry nahm Lunas Gesicht in seine Hände und schaute sie an.

\enquote{War das der Tag, an dem sie gestorben ist?}, fragte er.

\enquote{Ja}, sagte Luna mit Tränen im Gesicht.

\enquote{Dann träume ich gar nicht?}, fragte Harry.

\enquote{Nein}, sagte Luna, die sich die Tränen aus dem Gesicht wischte. \enquote{Früher hatte ich öfters diesen Traum. Ich sah, wie sie wieder eines ihrer Experimente machte. Aber seit wir zum ersten Mal nebeneinander gelegen und unsere Nacht verbracht hatten, waren sie seltener geworden. Und so habe ich das ganze auch noch nicht gesehen.}

Harry hielt sie wieder fest in seinen Armen, als die Umgebung um sie herum wieder klarer zu werden begann. Er schluckte, als er sah, wo sie sich befanden.

Luna löste sich von ihm und entdeckte ein Schlafzimmer mit einer hübschen jungen Frau in der Mitte, sie hatte lange, rötliche Haare und, wie sie feststellte, Harrys Augen. \enquote{Deine Mutter?}, fragte Luna.

\enquote{Ja}, entgegnete ihr Harry,  als ihm eine Träne die Wange herunterlief.

Harrys Mutter hielt ihren kleinen Sohn fest in ihren Armen, als sich Geräusche von draußen in Harrys Ohren wiederfanden. Harrys Mutter erschrak und lief hinter das Bett, um ihn abzusetzen. Von der Tür her kamen spratzelnde Geräusche. Es hörte sich so an, als ob sich zwei Zauberer duellierten. Es gab einen Aufschrei und durch den Türschlitz drang ein grüner Lichtschein. Jetzt war es stiller. Harry hörte nur die Fußtritte auf knarrendem Holz. Er musste schlucken. Er hörte eine Stimme und die Tür sprang auf. Eine Gestalt trat in das Zimmer. Sie war mit einem schwarzen Umhang verdeckt und hatte eine Kapuze auf. Harry konnte sein Gesicht nicht sehen, war sich aber sicher, er wusste, wer es war.

Seine Mutter schrie: \enquote{Nicht Harry, nicht Harry, bitte nicht Harry!}

\enquote{Geh zur Seite, du dummes Mädchen\abs geh weg jetzt\abs}

\enquote{Nicht Harry, bitte nicht, nimm mich, töte mich an seiner Stelle \gst}

Betäubender, wirbelnder, weißer Nebel füllte Harrys Kopf\abs was tat er da? Warum flog er? Er musste ihr helfen\abs sie würde sterben\abs sie wurde umgebracht\abs

Er fiel, fiel durch den eisigen Nebel.

\enquote{Nicht Harry! Bitte\abs verschone ihn\abs verschone ihn\abs}

Eine schrille Stimme lachte, die Frau schrie, und Harry schwanden die Sinne.

Er hörte nur noch: \enquote{Avada Kedavra} und ein grüner Blitz schoss aus der Spitze des Zauberstabes und auf seine Mutter zu. Sekunden später lag sie Tod auf dem Boden und Harry fiel vor lauter Trauer und innerlichem Schmerz auf die Knie und fing an zu schreien. Tränen quollen aus seinen Augen heraus. Luna nahm ihn in ihre Arme und streichelte seinen Kopf. Harry beruhigte sich wieder und sah, wie die vermummte Gestalt um das Bett herumlief und auf Klein-Harry zuging.

Harry stand auf, die Tränen noch immer in seinem Gesicht, um sich das ganze anzuschauen. Er wollte verstehen, was damals passiert war, wie er als einziger überleben konnte. Er sah dem Zauberer zu, von dem er sicher war, dass es sich um Voldemort handelte. Es war die Nacht, in der seine Eltern ums Leben gekommen waren. Die Nacht als Voldemort seine Macht verlor und für vierzehn Jahre verschwand. Der Zauberer hob wieder seinen Zauberstab und sprach wieder: \enquote{Avada Kedavra}. Jetzt sah Harry wie sich alles in Zeitlupe abspielte. Die Spitze von Voldemorts Zauberstab leuchtete auf und ein grüner Lichtstrahl schoss heraus. Er bewegte sich langsam auf Harry zu. Gerade als er wenige Zentimeter von ihm entfernt war, traf er auf eine Art Schild, der den größten Teil zurückwarf und Voldemort im Gesicht traf. Ein geringer Teil jedoch ging hindurch. Genau an der Stelle, wo Harry jetzt seine Narbe hatte. Er hörte, wie Voldemort aufschrie und sich die Hände vor das Gesicht hielt. Er begann sich aufzulösen bis nur noch eine kleine Wolke übrig blieb, die sich schnell durch die Tür aus dem Zimmer heraus bewegte und dabei ein leises Schreien und Weinen hinterließ. Ein winziger Teil bewegte sich unbemerkt auf Harry zu und verschwand in dessen Narbe.

Die Welt drehte sich wieder und Harry wachte auf. Er stieß einen Schrei aus. Schweiß lief von seinem Gesicht. Luna nahm ihn in ihre Arme und auch Hermine eilte zu ihm. Ron blickte von seinen Hausaufgaben hoch und Dumbledore erschrak.

\enquote{Was, wie, wo}, entfuhr es Dumbledore, der anscheinend gerade träumte, als Harry ihn aus seinem Schlaf riss. Er drehte sich um und sah Harry. \enquote{Was ist los, Harry? Hattest du wieder einen Traum?}, fragte ihn Dumbledore.

\enquote{Ja. Wir haben \gst ich habe \gst hatte einen Alptraum.} Harry schnaufte. \enquote{Ich war dabei, Professor, dabei, und sah wie Voldemort mich umbringen wollte. Als kleines Kind. Es war so, als wäre ich in einer Erinnerung gewesen. Wie vor zwei Jahren, als ich in ihrem Denkarium die Verurteilung von Barty Crouch Jr. sah.}

Dumbledore sah ihn an. \enquote{Und? Viel dir etwas auf?}

Harry überlegte kurz und sagte dann: \enquote{Ja. Ich sah, wie der Fluch auf mich zukam. Doch wenige Zentimeter bevor er aufschlug, wurde er durch einen seltsamen Schild abgehalten und zurückgeworfen. Nur ein bisschen ging durch.} Harry zeigte auf seine Narbe. \enquote{Voldemort löste sich auf und nur eine kleine Wolke blieb zurück. Sie floh durch die Tür hinaus. Dann bin ich aufgewacht.}

Dumbledore sah Harry an und rieb sich sein Kinn.

\enquote{Was hat das zu bedeuten, Professor?}, fragte Harry.

\enquote{Ich weiß es nicht, Harry. Es ist uns allen immer noch nicht genau klar, wie dich deine Mutter schützen konnte. Warum genau der Fluch dich nur gezeichnet, aber nicht getötet hatte. Die Liebe deiner Mutter hat dich wahrscheinlich geschützt. Ihr Opfer, um dich zu retten. Das ist alte Magie, dessen genaue Wirkung nicht mehr bekannt ist. Zu viel ging verloren.} Er schnaufte und sprach weiter. \enquote{Ich weiß, das ist jetzt nicht einfach, aber könntest du versuchen, falls du noch einmal diesen Traum hast, herauszufinden, wer Voldemorts Zauberstab mitgenommen hat, oder wohin er verschwand?}

\enquote{Aber Professor}, sagte Harry, \enquote{Ich dachte, man kann nur die unmittelbare Umgebung sehen.}

\enquote{Harry, so viele Sachen zwischen dir und Voldemort liegen für uns noch im Nebel. Wer weiß, ob die Verbindung zu Voldemort dir nicht erlaubt, ihm zu folgen und zu sehen was er sieht.}

Harry nickte, immer noch schwer schnaufend.

\enquote{Was ist denn hier los?}, fragte Professor McGonagall, die sich ihnen näherte.

Harry sah auf und bemerkte erst jetzt, dass wohl alle, die noch im Innenhof waren, um ihn herum standen.

\enquote{Harry hatte einen Alptraum, Minerva}, sagte Dumbledore. \enquote{Aber sagen Sie mal, haben Sie ihre Geschäfte nun erledigt?}

Professor McGonagall schnürte ihre Lippen zusammen und hatte einen undefinierten Gesichtsausdruck. \enquote{Hmpf. Nie wieder esse ich eines von Ihren verrückten Toffees.}

\enquote{Oh, das sind nicht meine verrückten Toffees. Die stammen von den Weas\-ley-Zwillingen.}

\enquote{Was?}

\enquote{Ja, sie boten mir eines an, als ich im Vorbeigehen meine \gst ja, Problemchen vor mich hin murmelte. Aber glauben Sie mir, Minerva. Wenn Sie das schon schlimm fanden, dann sollten Sie keines von den schwarzen nehmen. Die haben nämlich einen lustigen Nebeneffekt.} Dumbledore grinste, als Professor McGonagall mit einem weiter undefinierten Gesichtsausdruck den Innenhof verließ. Dumbledore drehte sich wieder zu Harry und zwinkerte ihm zu. Er griff in seine Hosentasche und schaute auf seiner Uhr nach.

\enquote{Wird langsam Zeit fürs Abendessen}, meinte Dumbledore und stand auf.

\gedanke{Was für ein eigenartiger Traum}, dachte Harry. \gedanke{Oder war es doch mehr?} Harry sah zu Luna, die in Richtung Ginny sah, die vor ihm stand. Er blickte zu ihr und bemerkte einen eigenartigen Gesichtsausdruck, der ihm auf eine seltsame Art und Weise zu sagen schien, dass da mehr war als nur die Sorge um einen guten Freund und Schulkameraden.

Er hörte Luna wieder in seinen Gedanken. \gedanke{Ich versuche mal herauszufinden, wie sie fühlt. Lass mich nur machen.} Luna stand auf und meinte zu Ron, Ginny und Hermine: \enquote{Kommt, lasst uns essen gehen.}

Ron packte seine Sachen zusammen und stand auf. Zu fünft gingen sie Richtung Große Halle. Ron und Hermine vor ihnen und Harry lief neben Luna und Ginny. Sie berührten sich nicht. Als Ginny in die Große Halle hinein lief, ging Luna ihr hinterher. Harry entschied sich, sich ihnen gegenüber hinzusetzen, und lief daher auf die andere Seite der Tafel. Nachdem das Essen auf den langen Tafeln der Häuser erschienen war, begannen alle zu essen. Harry sah auf und entdeckte, dass einige der Ravenclaws die Köpfe zusammensteckten und zu seinem Tisch herüberschauten. Einige drehten sich kurz um, nur um danach die Köpfe wieder zueinander zu stecken. \gedanke{Vermutlich unterhalten die sich wegen Luna.} Denn Luna, eine Ravenclaw, saß am Tisch der Gryffindors. Harry sah zum Tisch der Lehrer und bemerkte wie Dumbledore zu ihnen heruntersah. Er bemerkte Luna und sah dann zurück zu Harry. Er zwinkerte ihm zu und Harry begann zu schmunzeln. Er war sich nicht sicher, was Dumbledore wusste oder vermutete, aber er wusste, dass er ihn nicht fragen würde.

Ron und Harry waren bald mit dem Essen fertig und Ron zog sich in die Bibliothek zurück. Ginny und Luna unterhielten sich scheinbar gut und so ließ ihnen Ron einen kurzen Zettel zurück, auf dem stand, wohin er gegangen war. Harry wollte mit, wurde ab er durch den fast kopflosen Nick abgelenkt.

\enquote{Sir Nicolas, kann ich Sie mal was fragen?}

Der Geist drehte sich um. \enquote{Oh, Harry. Gerne doch, um was geht es?}

\enquote{Ich frage mich, ob sie eine Adriana de Mimsy Porpington kennen?}

\enquote{So hieß meine Schwester, aber warum fragen Sie?}

\enquote{Ich habe ein Buch von einer Adriana de Mimsy Porpington gelesen. Und da Sie, Sir Nicolas de Mimsy Porpington, diesen nicht sehr häufigen Namen ebenfalls tragen, habe ich mich gefragt, ob es jemand aus Ihrer Verwandtschaft ist.}

\enquote{Ja, das kann durchaus sein. Meine Schwester hat viele Bücher über Tiere geschrieben. Welches war es denn?}

\enquote{Aufzucht und Pflege von Dementoren}, antwortete Harry.

Nick staunte. \enquote{Nein, so ein Buch kenne ich nicht. Ich habe mich eigentlich immer mit ihr über ihre Bücher unterhalten. Aber ein derartiges Buch kenne ich nicht. \gst Die letzten Jahre haben wir uns selten gesehen und als ich sie mal besucht hatte, hatte sie zwar ein eigenartiges Haustier, aber als Dementor würde ich es nicht bezeichnen.}

Harry zog die Augenbrauen zusammen. Er setzte seine Tasche ab, holte Pergament, eine Unterlage und einen Stift hervor und begann zu zeichnen. Als er fertig war, zeigte er es Sir Nicolas. \enquote{Sah das Tier so aus, Sir Nicolas?}

Nick sah sich das Bild lange an und versuchte sich zu erinnern. \enquote{Ja, ich glaube, es hatte eine große Ähnlichkeit mit dem Tier, das sie damals besaß. Aber fragen Sie sie doch persönlich. Irgendwo im Schloss hängt ein Bild von ihr. Lassen Sie mich überlegen. \gst Ah ja, in der Nähe des Direktoren-Büros.}

\enquote{Das ist Ihre Schwester?}, fragte er. Er hatte sie schon ein paar mal gesehen. Sie hing genau gegenüber eines Torbogens, der die Aufzüge in Hogwarts markierte. \enquote{Sie haben mir sehr geholfen, Sir de Mimsy Porpington}, bedankte sich Harry artig und ging mit der Gewissheit, Nick in Zukunft jede Frage stellen zu können. Denn so wollte er am liebsten genannt werden.

\enquote{Sie können mich in Zukunft Sir Nicolas nennen}, rief ihm der Geist hinterher.

Harry drehte sich noch einmal um und winkte zur Bestätigung, bevor er nun in die Bibliothek ging.

\trenn

Harry sah, wie Madame Pomfrey neben einem künstlichen Körper kniete, den Hintern auf ihren Füßen. Heute würden sie Reanimationstechniken lernen.

\enquote{Kommen Sie ruhig näher und sehen Sie genau zu. Wir nehmen heute Reanimationstechniken durch. Sowohl nach Muggelart, als auch auf magische Art und Weise. \gst Sprechen Sie die Person erst einmal an. Sollte sie bewusstlos sein, fühlen Sie den Puls der Person. Entweder an den Handgelenken}, sie zeigte ihnen die genaue Stelle, \enquote{oder an den Fußgelenken}, abermals zeigte sie ihren Schülern die genaue Stelle, \enquote{oder am Hals. Entweder links oder rechts, aber niemals an beiden Stellen gleichzeitig.}

Sie nahm den Kopf der Puppe und überstreckte ihn. Danach tat sie genau, was sie den Schülern beschrieb. \enquote{Überstrecken Sie den Kopf des Bewusstlosen und öffnen Sie den Mund, um ihn nach Erbrochenem oder anderen Gegenständen zu untersuchen, die er eventuell in seinem Mund hat. Sollte etwas in seinem Mundraum sein, so drücken Sie Daumen und Zeigefinger in die Backen um zu verhindern, dass er seinen Kiefer zuklappt, während Sie in seinem Mundraum einen oder mehrere Finger haben. Wenn Sie sichergestellt haben, dass der Mundraum frei ist, schließen sie den Mund und Pusten Sie ihm über die Nase Luft ein.}

\enquote{Dies machen Sie zehnmal. Danach gehen Sie her und suchen einen Punkt zwei Finger breit unter dem letzten Rippenbogen und führen eine Herz-Druckmassage durch. Aber nur, wenn der Patient keinen Puls hat. Haben Sie keine Angst, wenn Sie ihm dabei ein paar Rippen brechen. Die heilen wieder. Sie wiederholen diese Technik ständig; also Atemspende und Herz-Druckmassage, solange bis ein Muggelarzt kommt, sollten sie sich in der Muggelwelt bewegen und einige Leute um sie herum stehen. Verteilen Sie Aufgaben an die Leute. Sprechen Sie sie direkt an und sagen Sie ihnen, was sie tun sollen: Einen Arzt rufen, Schatten spenden, und so weiter.}

Die Klasse führte die Anweisungen in kleinen Gruppen nun aus. Jeder musste einmal das ganze Programm durchziehen. Dann fuhr Madame Pomfrey fort.

\enquote{Und jetzt machen wir das Ganze auf magische Art und Weise. Schauen Sie mir wieder zu. Wir prüfen auf dieselbe Art, ob der Patient bei Bewusstsein ist.} Sie schüttelte ihn leicht an seinen Schultern und zwickte ihn danach in den Arm. \enquote{Dann fühlen Sie seinen Puls und schauen sich den Mundbereich an. Jetzt machen Sie die Herz-Druckmassage mit dem Zauberstab.}

\enquote{Richten Sie Ihren Zauberstab auf das Herz des Bewusstlosen und sagen Sie deutlich: \spruch{cor-pressus aliptes}. Beachten Sie aber, dass je nach Konzentration die Stärke des Druckes auf das Herz stärker oder schwächer ist. Sie brauchen etwas Übung, bis Sie die richtige Stärke finden. Zusätzlich müssen Sie einen zweiten Zauber ausführen. Sie müssen sich auf den Ersten zusätzlich Konzentrieren, dass er nicht abbricht. Im Verlaufe des weiteren Kurses werden wir mehr als nur einen Zauber gleichzeitig ausführen. Der zweite Zauber ist der für die automatische Luftzufuhr. Sagen Sie wieder deutlich: \spruch{Aer pulmo mutare}}

Nachdem sie wiederum alles vorgeführt hatte, durften die anderen ran.

\trenn

Am nächsten Samstag, kurz nachdem Professor Elber sein Frühstück beendet hatte, verließ er die Große Halle.

Harry und Hermine bemerkten nichts davon, als plötzlich Ron fragte: \enquote{Wo ist denn Professor Elber abgeblieben?}

Harry erschrak und schaute sofort zum Lehrertisch, genauso Hermine. Ohne zu Ende zu essen, stand er auf und lief kauend nach draußen.

Hermine konnte ihm nur noch ein \enquote{Warte} hinterherrufen, doch knapp außerhalb der Großen Halle blieb er plötzlich stehen. Hermine, die aufholte, konnte gerade noch bremsen, bevor sie auf ihn aufschlug. \enquote{Was ist, warum bleibst du plötzlich stehen?}, fragte sie.

\enquote{Professor Elber! Warten Sie schon lange?}

\enquote{Nicht der Rede Wert}, antwortete Professor Elber. \enquote{Bereit?}

\enquote{Ja}, antworteten Harry und Hermine fast gleichzeitig.

Professor Elber machte eine Geste, um Harry und Hermine den Vortritt zu lassen. Sie gingen die Treppen hoch, die sich ständig veränderten, und gelangten irgendwann in den dritten Stock. Ein paar zehn Meter vor dem Eingang zum Mädchenklo fing Hermine an zu Professor Elber zu sprechen.

\enquote{Äh. Professor? Da gibt es etwas, was Sie wissen sollten.}

\enquote{Ja?}

\enquote{Auf dem Mädchenklo gibt es einen Geist. Die Maulende Myrte.}

\enquote{Und das heißt jetzt genau was?}, fragte Professor Elber.

\enquote{Sie ist ein wenig schwierig}, antwortete Hermine.

An der Tür angekommen, öffnete Hermine sie und trat zuerst ein. Dann folgte ihr Harry und zuletzt Professor Elber.

\enquote{Uhhhh, Harry. Schön, dich mal wieder zu sehen} kam es Harry von Myrte entgegen.

Professor Elber zeigte sich wenig interessiert an Myrte und ging um die Waschbeckenanordnung herum. Myrte schwebte von oben auf Harry herab und kam kurz vor ihm in der Luft schwebend zum Stillstand.

\enquote{Hast du mich vermisst?}, fragte sie.

\enquote{Myrte}, sagte Hermine leicht verärgert. \enquote{Wir wollen in die Kammer.}

\enquote{Uhhh, kann ich mit? Sag ja, Harry}, bettelte Myrte.

\enquote{Das ist eine gute Idee}, kam es von Professor Elber. Harry und Hermine drehten sich um und sahen Professor Elber fragend und verdutzt an. \enquote{Ich finde, es ist von Vorteil}, sagte Professor Elber und kam auf Myrte zu. Diese sah ihn nur verwirrt an. \enquote{Einen Geist zum Üben zu haben ist geradezu ideal. Ein Ziel, das nicht so leicht zu treffen ist.}

\enquote{Uaaaah}, entfuhr es Myrte. \enquote{Kommt, lass uns auf Myrte zielen und sie versuchen zu treffen. Was für ein Spaß.}

\enquote{Geister können sich prima wehren und haben eine größere Resistenz gegenüber Zaubern und Flüchen, wenn sie es richtig machen}, antwortete Professor Elber. \enquote{Sie sind schnell und können einen Zauberer ganz schön ins Schwitzen bringen. Und außerdem haben sie sicherlich auch jede Menge Spaß daran! Und etwas Abwechslung.} Professor Elber sah Myrte interessiert an und fuhr dann fort. \enquote{Ich kann Ihnen ein paar Tricks zeigen, wie Sie es den Sterblichen so richtig zeigen können. Die werden Mühe haben, Ihren Attacken und Angriffen auszuweichen.}

Myrtes Augen fingen an ein Leuchten zu zeigen, obwohl sie durchlässig war. \enquote{Das hört sich interessant an.} Sie drehte sich zu Harry. \enquote{Ich glaube, wir werden eine Menge Spaß haben.}

\enquote{So. Wie kommen wir jetzt in die Kammer?}, fragte Professor Elber. Harry bewegte sich auf die Waschbecken zu und ging auf den Wasserhahn mit den eingegossenen Schlagen an der Seite zu. Dann fing er an Parsel zu sprechen und der Eingang zur Kammer öffnete sich.

\enquote{Jetzt müssen wir springen}, sagte Harry und hüpfte hinunter. Hermine folgte ihm etwas unsicher kurz darauf.

Professor Elber sah Myrte nur leicht verwundert an und fragte sie: \enquote{Wollen Sie mitkommen?}

\enquote{Nein, nein}, kam es ihm entgegen. \enquote{Ich warte noch ein wenig und muss es erst den anderen Geistern erzählen.}

Professor Elber nickte und hüpfte den beiden hinterher. Unten angekommen führte Harry Hermine und Professor Elber den Gang entlang, an den heruntergefallenen Steinen und an dem leeren Platz, an dem die Haut des Basilisken lag, vorbei, zur großen Türe mit den Schlangen. Wieder sprach er etwas auf Parsel, worauf sich die Tür zu öffnen begann.

\enquote{Bleibt die Tür denn offen, oder geht die immer wieder zu?}, fragte Professor Elber.

\enquote{Ich weiß es nicht}, antwortete Harry und ging in die Kammer hinein. \gedanke{Beim ersten Mal habe ich nicht aufgepasst und vor ein paar Tagen habe ich sie geschlossen}, dachte Harry.

Hermine und Professor Elber folgten ihm. Sie gingen den langen Gang entlang und kamen in einem großen Raum an, worauf Professor Elber sich umschaute. Er strich mit seinen Händen die Steinmauer entlang und sah sich sehr genau um. Als er alle Gänge einmal abgelaufen war, ging er wieder auf Harry und Hermine zu. Dann sprach er: \enquote{Nett hier. Ein prima Übungsraum.} Er zog seinen Zauberstab und richtete ihn auf Harry. \enquote{Entwaffnen Sie mich mal}, sagte er zu Harry.

\enquote{Was?}, fragte Harry nach. \enquote{Na entwaffnen. Sie nehmen ihren Zauberstab, richten ihn auf mich und sagen dann: \spruch{Expelliarmus}.} Harry stand da, holte seinen Zauberstab heraus und sah mit einem leicht mulmigen Gefühl Hermine an. \enquote{Na los. Oder haben sie Angst mich anzugreifen?}

Harry drehte sich um und ohne ein Wort zu sagen, sprach er: \zauber{Expelliarmus}, worauf sein eigener Zauberstab aus seiner Hand Richtung Professor Elber flog, welcher ihn geistesgegenwärtig fing.

\enquote{Ah, ja}, kam es von Professor Elber. \enquote{Hab ich vermutet.}

Er gab Harry seinen Zauberstab wieder zurück und ging an der Wand der großen runden Halle entlang. Er suchte eine kleine Spalte oder Ritze im Mauerwerk und steckte, nachdem er sie gefunden hatte, seinen Zauberstab mit der Rückseite voraus hinein. Dann stellte er sich an seine Seite und fuhr dann mit seiner Hand an seinem Schaft entlang und ging ein paar Schritte zurück, worauf eine Explosion aus der Wand kam und ein großes Loch in sie hinein riss. Der Zauberstab flog herunter und blieb auf den Schutthaufen am Boden liegen.

\enquote{Wie mir scheint, kann man hier noch keine Zauberstäbe verwenden.}

\enquote{Ja aber}, fragte Harry, \enquote{wie sollen wir dann hier\abs} doch Harry kam nicht weiter, denn Professor Elber hatte bereits seine Hand über seinen Zauberstab ausgestreckt und rief: \enquote{Auf.} Der Zauberstab erhob sich und war Sekundenbruchteile später in seiner Hand, welche er schloss.

\enquote{Wie haben Sie das denn gemacht? Ich dachte hier kann man keine Zauber verwenden?}, fragte Hermine, denn Harry stand immer noch da, als wolle er seinen Satz beenden.

\enquote{Ich habe lediglich gesagt, dass hier noch keine Zauberstäbe verwendet werden können. Ich habe nichts von zauberstabfreier Magie gesagt.}

\enquote{Zauberstabfreie Magie?}, entfuhr es aus Harry. \enquote{Was ist das denn?}

\enquote{Das, was sie soeben gesehen haben. Manche sagen auch stabfreie Magie dazu.} Professor Elber öffnete seine Hand mit der Handinnenfläche nach oben und ließ seinen Zauberstab in ein paar Zentimeter Höhe schweben. Plötzlich begann er sich leicht zu drehen. \enquote{Das ist zauberstabfreie Magie. Wenn sie wollen, können wir das auch an den Samstagen lernen, an denen wir hier üben.}

Beide nickten und stimmten zu. \enquote{Aber wo üben wir dann mit Zauberstab?}, fragte Hermine.

\enquote{Auch hier}, antwortete Professor Elber. \enquote{Aber zuvor müssen wir schauen, dass wir diese Schutz\-zau\-ber loswerden. Ihr könnt euch ja mal was überlegen. Ich habe zwar schon eine Idee, wie wir sie weg\-be\-kom\-men, aber alleine schaffe ich das nicht. Und außerdem, drei Gehirne denken mehr als eines. Vielleicht fällt euch etwas ein, was mir entgangen ist.}

Harry wunderte sich und fragte sich, was Professor Elber sonst noch so drauf habe. \enquote{Jetzt räumen wir erst einmal diesen Schutt hier weg und dann schaue ich mal in der Bibliothek vorbei. Ihr könnt jetzt gehen, wenn ihr wollt. Ich habe vorerst genug gesehen. Und wenn das Problem mit dem stabfreien Zauber hier gelöst ist, können wir anfangen zu trainieren.}

Professor Elber sah auf den Schutthaufen hinunter und bewegte seine Hände über ihm. Er zog eine kleine Schleife und zog seine Hände dann nach oben, worauf der Haufen Schutt anfing sich ebenfalls nach oben zu bewegen. Mit einer eleganten Bewegung beförderte er die Steine und die kleinen Brösel in das Loch an der Wand. Es schien so, als ob die Steine und der Staub seinen Bewegungen folgten und die exakt selbe Stelle in der Mauer einnahmen, die sie vorher innehatten. Als Professor Elber fertig war, sah man keine Spur mehr von einem Ausbruch in der Wand.

Er lief voraus und Hermine und Harry folgten ihm. Sie verließen die Kammer.

Gegen Abend ging Harry durchs Schloss. Salazar hatte ihm gesagt, wo seine Privatgemächer wären. Bald war Sperrstunde, aber Harry hatte noch genug Zeit. Notfalls würde er in Salazars altem Bett schlafen. Je näher er den Kerkern und somit seinem Ziel kam, desto bekannter kam ihm der Weg vor. Er war im selben Gang, wie in seinem zweiten Jahr mit Ron, als sie sich in Crabbe und Goyle verwandelt hatten, um Draco auszuspionieren. Er ging gerade am Zugang zum Gemeinschaftsraum der Slytherin vorbei und blieb nach einigen Metern vor einem Bild von Salazar Slytherin stehen.

Er schien zu schlafen, doch machte er seine Augen auf, sobald Harry vor ihm stehen blieb. Misslaunig sah er ihn an.

\enquote{Wer sind Sie denn? Sie sind der Erste, der sich für mich interessiert.}

\enquote{Ich bin's, Harry}, sagte Harry gut gelaunt.

\enquote{Wie kommen Sie dazu, mich derart vertraut anzureden? Wir kennen uns doch gar nicht.}

\enquote{Aber wir haben uns doch schon öfter unterhalten.}

\enquote{Wir? Unterhalten? Sie haben wohl einen Schlickschlupf zum Frühstück gegessen!?}

Jemand trat im Schatten an Harry und das Bild heran, hielt sich jedoch im Schatten und beobachtete nur.

\gedanke{Schlickschlupf?}, fragte sich Harry. \enquote{Schlickschlupf? Was ist denn ein Schlickschlupf?}

\enquote{Sie kennen keine Schlickschlupfe? Das sind kleine fliegende Wesen. Unsichtbar und sie schwirren einem durch den Kopf, um einen ganz wuschig zu machen. Nicht im erotischen Sinne. Sie sorgen für Irrsinn und übertriebene Heiterkeit, sowie zu viel an Humor. Je nach Charakter des Befallenen.}

\enquote{Nein, ich habe keine Schickschlupfe.}

Dann stutzte Harry. Er senkte leicht seinen Kopf, danach seinen Blick und dachte nach. Dann sah er Salazar Slytherin wieder an. Dieser blickte ihn fragend an.

\enquote{Wie kann es sein, dass ich spüre, wenn mich jemand beobachtet}, fragte er Slytherin.

\enquote{Sie haben die notwendige Magie in sich}, antwortete ihm das Bild.

\enquote{Kann das jeder?}

\enquote{Ja. Jedes Objekt, jedes Lebewesen ist miteinander durch die Magie verbunden. Wenn jemand einem näher kommt, dann gibt es Interferenzen. Diese Interferenzen kann man spüren, wenn man seinen Zugang zur ihr gefunden hat.}

\enquote{Apropos Zugang, ich bin hier, weil ich\abs}

\enquote{Längst im Gemeinschaftsraum sein sollten, Mister Potter. Sie haben noch fünf Minuten Zeit um ihn zu erreichen, dann gibt es zehn Punkte Abzug}, sagte Severus Snape, der aus dem Dunklen an Harry herantrat.

\enquote{Professor Snape?}, sagte Harry erstaunt. Er wusste, dass er es nicht schaffen würde. Er grübelte nach. Ihm war so, als ob die Schulregeln in diesem Punkt nicht sehr genau waren. \enquote{Wie meinen sie das, Zeit ihn zu erreichen?}

\enquote{Zum Gemeinschaftsraum der Gryffindors geht es da lang} und er zeigte in die Richtung aus dem Kerkern heraus.

\enquote{Fünf Minuten? Das schaffe ich nicht. Dann kann ich ja noch etwas mit Sal\aabs Mr. Slytherin plaudern. Punktabzug bekomme ich ja wohl in jedem Fall.}

\enquote{Treiben Sie es nicht zu bunt, Mister Potter. Sonst bekommen Sie noch mehr Punkte abgezogen.}

Harry hatte eine Idee, wie er sich die fünf Minuten raus schinden konnte. Er wusste, dass neben ihm ein Gemeinschaftsraum war. Und er hatte die Möglichkeit da reinzukommen. Kurz vorher würde er die Tür öffnen und sich knapp hinter den Eingang stellen. Dann dürfte ihm Snape keine Punkte mehr abziehen.

Er drehte sich zu Mr. Slytherin und fragte ihn: \enquote{Mr. Slytherin, in dem Rezept über die \accentuate{verbesserten Sinne}, wie kamen Sie auf die Idee, eine Basilisken-Haut zu nehmen, anstelle der einer Schlange, zumal in allen Rezepten vor der falschen Schlangenhaut gewarnt wird?}

Damit hatte er das Interesse der beiden Männer gewonnen.

\enquote{Daher haben Sie also die Rezepte}, meinte Professor Snape.

\enquote{Wie kommen Sie an meine Rezepte?}, fragte Slytherin skeptisch nach.

Harry antwortete nur: \enquote{Ich habe meine Quellen.} Dann drehte er sich um und ging vor den Eingang zum Gemeinschaftsraum der Slytherins. Er hatte noch wenige Sekunden Zeit. Er flüsterte das Passwort: \parsel{Öffne dich.} Die Tür öffnete sich und Harry trat über die Schwelle. Dann drehte er sich um und sah auf seine Uhr.

Dann passierte es \geraeusch{Dong. Dong. Dong. Dong. Dong. Dong. Dong. Dong. Dong. Dong.} Die Uhr schlug zehnmal. Alle Schüler mussten in den Gemeinschaftsräumen sein.

\enquote{Was tun Sie da, Potter?}, fragte Snape gereizt.

\enquote{Ich bin in einem Gemeinschaftsraum und das rechtzeitig. Laut unserer Schulordnung muss ich nur in einem Gemeinschaftsraum sein, wenn es zehn Uhr schlägt. Und das bin ich.}

Snape kniff die Augen zusammen. \enquote{Kommen Sie raus, ich bringe Sie in Ihren Gemeinschaftsraum.}

Harry hob einen Fuß und wollte ihn schon aufsetzen. Doch er zog ihn zurück und sagte: \enquote{Wenn ich von Ihnen aufgefordert werde, dann bekomme ich keine\abs} Er trat heraus und die Tür schloss sich. \enquote{\aabs Strafe.}

Snape verkniff sich einen bissigen Kommentar und ging neben Harry her, um ihn zum Gryffindorturm zu bringen.

\enquote{Was mich interessiert, woher haben Sie die Rezepte? Dass sie von Slytherin sind, weiß ich jetzt. Aber woher haben Sie sie?}

\enquote{Die Rezepte habe ich aus einem Buch von Slytherin. Ich fand es sehr interessant.}

\enquote{Das würde ich gerne mal sehen.}

\enquote{Das würde Ihnen nichts nützen. Die Rezepte können Sie eh nicht lesen. Sie sind alle in Parsel geschrieben.} Sie liefen weiter nebeneinander her und erreichten die letzten Stufen zum Zugang zum Gryffindorturm. \enquote{Ich habe Sie extra abgeschrieben. Sonst hätte es keinen Sinn gehabt. Parsel ist nicht leicht lesbar. Ich habe es mal Hermine gezeigt. Nach einer Woche meinte sie, jemand hätte mich damit hereingelegt. Wissen Sie, ich habe ihr erzählt, dass ich es von einem Händler hätte. Es würde zu einem wertvollen Gegenstand führen. Sie fand es nicht heraus. Also habe ich mich schamhaft und reuig gezeigt. Ich will nicht, dass mich jeder schon wieder für etwas hält, was ich nicht bin. Sonst wäre es wieder wie im zweiten Jahr, als jeder geglaubt hat, ich sei Slytherins Erbe.} Im Stillen gab er sich damit recht. Er war Slytherins Erbe. \enquote{Und als mich jeder schief angesehen hatte, als ich die Schlange von Justin abhielt\abs}

\enquote{Das haben Sie also gemacht.} Snape blieb stehen.

Sie brauchten noch ein paar Meter, um zum Bild der fetten Dame zu kommen. So störten Sie sie nicht.

\enquote{Ja. Ich habe der Schlange gesagt, dass sie auf mich hören soll. Sie soll Justin nicht angreifen. Sie soll den Jungen nicht angreifen. Sie gehöre mir und soll auf mich hören. Und als die Schlange von ihm abließ, haben Sie sie vernichtet. Mann, war ich damals sauer. Nicht nur auf Sie. Sondern auf all die anderen, die mich isolierten, als hätte ich eine ansteckende Krankheit. Ich wollte das Gefühl nicht noch einmal erleben. Allerdings hatte ich im vierten Jahr mit demselben Problem zu Kämpfen. Wieder glaubte mir keiner.}

\enquote{Der Fall mit dem Feuerkelch?}

\enquote{Ja.}

Dann schwiegen sie eine Weile.

\enquote{Und das Zweite; wie konnten Sie die Tür zum Gemeinschaftsraum der Slytherin öffnen?}

Harry lächelte. \enquote{\inner{Öffne dich.} Mehr habe ich dem Zugang nicht gesagt. Allerdings auf Parsel. Es ist Sal\aabs Slytherins Haus. Logisch, dass er einen Zugang für sich geschaffen hat, der nicht an ein ständig wechselndes Passwort gebunden ist. Ich versuchte einfach mein Glück, da ich es sonst nicht geschafft hätte, in einem Gemeinschaftsraum zu sein., denn der der Ravenclaws ist noch etwas weiter weg und den der Hufflepuff kenne ich noch nicht einmal.} \gedanke{Aber das wird demnächst noch anstehen, wenn ich durch den speziellen Zugang über die Röhren zur Kammer dorthin gehe und auf der Karte nachsehe. \gst Die Karte. Ich habe nie auf der Karte nachgesehen, wo die Gemeinschaftsräume liegen.}

Snape nickte und trat einen Schritt zurück. \enquote{Gehen Sie nun schlafen. Morgen haben Sie die nächsten Stunden bei mir.}

Harry nickte und teilte der fetten Dame das Passwort mit. Leicht schläfrig schwang sie beiseite und ließ ihn ein.

Innen wurde er bereits erwartet und gleich von Ron gelöchert. \enquote{Harry, wo kommst du denn her? Was, wenn dich ein Lehrer erwischt hätte. Snape zum Beispiel.}

\enquote{Och, wir hatten eine nette Unterhaltung auf dem Weg hierher.}

\enquote{Was? Snape hat dich erwischt? Wie viel Punkte hat er dir abgezogen?}, mischte sich Hermine ein.

\enquote{Keine. Ich habe ihn dieses Mal\abs sagen wir\abs überrascht, so hat er mir nichts abgezogen.}

\enquote{Nichts abgezogen?}, fragte Hermine nach?

\enquote{Nein, er wollte, da ich nicht rechtzeitig in einem Gemeinschaftsraum sein würde.} Er setzte sich neben Ginny und berührte ihre Hand, als er sie neben sich legte und erzählte weiter. Er bemerkte es nicht einmal. \enquote{Ich war in den Kerkern. Dort, wo Ron und ich im zweiten Jahr waren. Beim Gemeinschaftsraum der Slytherins. Ich habe ein Bild von Salazar Slytherin gefunden. Nicht sehr gesprächig. Dort hat mich Snape getroffen.}

\enquote{Erwischt hat er dich}, warf Ron ein.

\enquote{Er hat gemeint, er müsse mir Punkte abziehen, wenn ich nicht rechtzeitig im Gemeinschaftsraum bin. Ich habe ihn dann hingehalten und bin rechtzeitig im Gemeinschaftsraum der Slytherins gestanden. Dann hat die Turmuhr zehn Uhr geschlagen und ich war laut Schulregeln nicht außerhalb der Gemeinschaftsräume. Es steht nirgends, dass man in seinem eigenen sein muss.} Harry grinste.

Ron und Hermine sahen ihn mit großen Augen an.

\enquote{Wie hast du es geschafft, dort hineinzukommen?}, fragte Hermine.

\parsel{Öffne dich.}

\enquote{Wie bitte? Sprich Deutsch}, sagte Ron.

\enquote{\inner{Öffne dich.} Ich habe der Tür einfach gesagt, dass sie sich öffnen soll.}

\enquote{Wie bist du darauf gekommen?}, fragte Ginny.

Harry drehte sich zu ihr. Er bemerkte seine Hand an ihrer, nahm sie aber nicht weg. Sie lag weiter neben ihrer. \enquote{Ich habe spekuliert. Slytherin war ein Parselmund. Es erschien mir logisch, dass er sich einen Zugang schaffte, der nicht an das aktuell gültige Passwort gebunden ist.} Dann gähnte er.

Er legte seinen Kopf gegen die Lehne des Sofas und dann seinen kleinen Finger über den Ginnys. Gedanken versunken spielte er mit seinem kleinen Finger an ihrem. Er strich an der Innenseite ihres Fingers auf und ab. Dabei schaute er an die Stelle, die vor wenigen Wochen noch den Zugang zum Gästebereich darstellte.

Ginny schien es nichts auszumachen, dass Harry mit ihrem Finger spielte. Obwohl er nicht wusste, wie sie reagieren würde, oder ob sie einen Freund hatte, versuchte er sein Glück. Den andern blieb diese Spielerei verborgen.

\trenn

\enquote{Heute beschäftigen wir uns mit Helene}, sprach Firenze, als seine Hufe auf dem bewaldeten Klassenzimmerboden kaum Geräusche hinterließen.

Parvati war ganz begeistert, den stattlichen Zentauren zu sehen, fragte sich aber, warum er eine Figur der griechischen Geschichte in Wahrsagen dran brachte. \gedanke{Vielleicht war sie jemand, die wahrsagte}, dachte sie sich. Beeindruckt sah sie zu ihm auf.

Heute saß die Klasse auf einer Waldlichtung; die Sterne über ihnen. Firenze zog mit ausgestrecktem Arm einen Halbkreis vor sich und die umstehenden Bäume begannen zu schrumpfen. Dann griff er in den Himmel und zog mit leicht geschlossener Faust seinen Arm herunter. Die Punkte am Himmel veränderten sich und kamen näher. Am Horizont konnte man eine rote Kugel erkennen, die gerade unterging. Dann kam eine näher und wurde zunehmend größer. Die Kugel hatte viele schmale Ringe, die in einer Ebene um ihren Zentralplaneten herum kreisten. Dann stand der Planet groß am Himmel und einer seiner Monde war deutlich sichtbar.

\enquote{Helene}, fuhr Firenze fort, \enquote{ist einer der vielen Saturnmonde. Neben Epimetheus, Pandora, Ymir, Suttungr, Skathi, Tarqeq und vielen anderen, insgesamt 62 Monde, von denen die Muggel gerade einmal 18 entdeckt haben, werden wir nur die eben genannten dieses Jahr durchnehmen. Wir werden uns ansehen und anschaulich machen, wie die verschiedenen Monde unseres Planetensystems, nicht nur die des Saturn, sondern auch anderer Planeten unsere wahrsagerischen Fähigkeiten beeinflussen können.}

Er lief durch die Gruppe.

% Ausschnitt aus Wikipedia: http://de.wikipedia.org/wiki/Helene_(Mond)  (18.08.2012)
\enquote{Helene umkreist Saturn in einem mittleren Abstand von ziemlich genau 377.420~km in 65 Stunden und 41 Minuten. Die Bahn weist eine Exzentrizität von 0,0071 auf und ist 0,21° gegenüber der Äquatorebene des Saturn geneigt. \gst Sie ist einer von zwei kleinen Monden auf der Bahn des großen Monds Dione. Helene läuft Dione in einem Winkelabstand von 60° im führenden Lagrangepunkt L4 voraus. Im folgenden Lagrangepunkt L5 läuft der Mond Polydeuces Dione im Winkelabstand von 60° hinterher. \gst Bevor sie ihren offiziellen Namen erhielt, wurde Helene von den Muggeln üblicherweise als \accentuate{Dione B} bezeichnet.}
%Bevor sie ihren offiziellen Namen erhielt, wurde Helene üblicherweise als „Dione B“ bezeichnet.}

Die Klasse starrte Firenze an, als sei der ein Astronom, der von seinem Lieblingsthema \accentuate{Saturn} sprach und nicht über Wahrsagen. Für einen kurzen Moment schienen sie vergessen zu haben, dass sie einem Zentauren zu Hufen saßen. Er knickte mit den Vorderhufen ein und saß kurz darauf ganz auf dem Boden auf, die Klasse im Halbkreis vor sich.

\enquote{Um uns die Genauigkeit unserer Vorhersagen ins Bewusstsein zu rufen, müssen wir uns die Bahnen der einzelnen Planeten, Monde und Kometen in unserem Sonnensystem bewusst sein, da wir sie in unsere Berechnungen einfließen lassen müssen. \gst Früher war das eine zeitraubende Angelegenheit, bis die Muggel die Bahnen der Planeten anfingen zu berechnen und Tabellen erstellt hatten, die für sehr lange Zeiträume die Positionen aufzeigten. Seit dieser Zeit ist es einfacher, die Konstellationen abzulesen. Da die Muggel aber noch nicht alle Monde unserer Planeten entdeckt haben, mussten wir die fehlenden Tabellen und Werte selbst berechnen. Wir wurden aber mit entsprechenden Formeln beliefert und errechneten uns die Tabellen selber. Den Rest haben wir von eingeweihten Muggeln und Squibs erhalten.}

\enquote{Professor?}, fragte Parvati nachdenklich nach.

\enquote{Ja, Miss Patil}, sagte Firenze und drehte sich zu ihr.

\enquote{Was ist der Lagrangepunkt?}
%http://de.wikipedia.org/wiki/Lagrangepunkt (18.08.2012)
%Die Lagrange-Punkte oder Librations-Punkte sind die nach Joseph-Louis Lagrange benannten Gleichgewichtspunkte des eingeschränkten Dreikörperproblems der Himmelsmechanik. An diesen Punkten im Weltraum heben die Gravitationskräfte zweier benachbarter Himmelskörper und die Zentrifugalkraft der Bewegung einander auf, sodass jeder der drei Körper kräftefrei ist und bezüglich der anderen beiden Körper immer denselben Ort einnimmt.

\enquote{Am Lagrangepunkt im Weltraum heben sich die Gravitationskräfte zweier benachbarter Himmelskörper und die Zentrifugalkraft der Bewegung gegenseitig auf.}

Ratlose Gesichter beherrschten die Szene.

\enquote{Stellen Sie sich unsere Sonne und die Erde vor. Die Erde kreist um die Sonne und beide ziehen einander an. Legen Sie jetzt einen Körper zwischen die beiden, dann wird er mit der Zeit zu einem der beiden Körper hingezogen. Es gibt aber einen bestimmten Punkt zwischen den beiden, an dem die Anziehungskraft der beiden Körper gleich groß wirkt. Also wird ein Körper, der dort liegt, von beiden gleichermaßen angezogen und bleibt auf diesem Abstand praktisch stehen, in stetig gleicher Entfernung.}

Langsam sickerte die Erkenntnis von den Ohren in die Gehirne seiner Schüler. Er wartete geduldig ab, bis es jeder begriffen hatte. Firenze konnte sehr gut an ihren Reaktionen lesen, ob dies der Fall war.

Wieder einmal bewegte er seine Hand und vor seinen Schülern erschienen Blätter mit Tabellen, die zu aktuellen Daten und Tageszeiten Werte enthielten, die die Wahrscheinlichkeit angaben, mit denen eine korrekte Vorhersage zutraf.

Harrys Augen wuselten über das Papier und er sah keinen Wert über 90\%.

\gedanke{Wie Firenze sagte, Wahrsagen ist keine exakte magische Wissenschaft}, dachte sich Harry.

Für den Rest der Stunde mussten sie sich hinlegen und ihrem Geist freien Lauf lassen. Dies lief Harrys Ok"-klu"-men"-tik-Übungen entgegen. Firenze musste dies bewusst sein, da er ihn kaum dran nahm und ausfragte. Nur wenn es um konkrete Benotung ging. Harry war es recht und so konnte er vor sich hin dösen und seine Gedanken sortieren. Doch heute hatte er eine zweifelhafte Vision, über die er sich die nächsten Tage den Kopf zerbrechen würde.

Als ihn jemand anstupste, sah er nur zwei Hufe über sich und Firenze der ihn anblickte. \enquote{Die Stunde ist zu Ende, Mister Potter.}

Harry rappelte sich auf und sagte: \enquote{Danke, Professor Firenze. Passen Sie die nächsten Tage besonders auf sich auf.} Dann etwas leiser und nur zu ihm gewandt. \enquote{Ich hatte einen Traum \gst eine Vision \gst jemand will Ihnen Schaden. Ich konnte ihn \gst sie, nicht richtig sehen. Aber anders als bei einem Traum habe ich dabei ein sehr ungutes Gefühl gehabt. Es war doch eine Vision.} Den letzten Satz sagte er mehr zu sich selbst, um sich zu beruhigen. Dann wandte er sich ab.

\enquote{Professor?}, fragte Lavender, die nun auf ihn zukam.

\enquote{Ja}, antwortete Firenze.

\enquote{Wie haben Sie das mit den Bäumen und dem Planeten gemacht?}

\enquote{So}, antwortete er und fuhr erneut mit seiner Hand vor sich her. Die Bäume wuchsen wieder.

\enquote{Das meinte ich nicht}, antwortete sie leicht unsicher. \enquote{Ich meine, mir ist nicht bekannt, dass Zentauren große magische Fähigkeiten haben.}

\enquote{Ah, belesen die junge Dame ist}, neckte Firenze sie. \enquote{Der Raum wurde auf meine Bedürfnisse zugeschnitten. Er reagiert auf meine Gedanken und Gesten}, antwortete er. \enquote{Er hat alles und kann alles, was ich brauche.}

Sie nickte verstehend und verabschiedete sich, während Harry um die Ecke bog, als er durch die Tür das Klassenzimmer verließ.

Auch an diesem Abend sortierte Harry seine Gedanken und dachte nur an das Lied von Salazar, während des Essens und zwischen den Stunden, sowie kurz vor dem Schlafen gehen. Morgen musste er wieder an seinem Trank \accentuate{üben}, was Okklumentik-Stunden bei Snape hieß, da die Tränke während des Unterrichts besser wurden, wenn er sich nicht ablenken ließ.

\begin{traum}
Er schlief ein und Träume durchzogen seinen Geist. Er stand auf einem Felsen vor einer Steilküste. Das Wasser brandete an die Klippe und wiegte sich im Takt des Windes. Harry sah noch Richtung Meer und drehte sich dann um. Er sah nun gegen die Klippen und entdeckte eine Höhle. Das Wasser war plötzlich ganz ruhig. Harry schien über das Wasser auf die Höhle zu zu schweben, obwohl er seine Füße bewegte. Nachdem er in der Höhle war, ging es langsamer voran. Der Vorwärtsdrang verschwand und Harry hatte das Gefühl, normal laufen zu müssen. Nach knappen zwanzig Metern erreichte er das trockene Ufer. Sein Blick ging durch die Höhle und blieb an einer Wand hängen. Er trat auf die Wand zu und sie verschwand. Dann ging er nach Innen. Er sah einen See, dessen Oberfläche spiegelglatt war und sich nicht zu bewegen schien. Es war dunkel in der Höhle. Trotzdem konnte Harry alles erkennen. Er sah eine kleine Insel mitten im See. Sein Blick schien die Insel heranzuholen. Er sah ein kleines Podest. Als er sich umsah, stand er bereits auf der Insel. Hinter sich fand er ein Boot, mit dem er auf die Insel gekommen war. Das war zumindest seine Meinung. Er ging an die Säule aus Bergkristall heran und sah hinein. Dort war eine Vertiefung mit einer Flüssigkeit. Unter der Flüssigkeit im Stein war ein Medaillon. Er griff hinein und zog es heraus. Dann versuchte er es zu öffnen. Er fand einen Zettel darin. Die Worte konnte er nicht genau lesen, aber er erkannte, dass sie eine Fälschung sein musste.

\enquote{Eines meiner Verstecke}, hörte er plötzlich eine Stimme neben sich.

Er kannte die Stimme und es überraschte ihn nicht im geringste. Es sah neben sich und die Gestalt sah auf den Kristall. Dann sah er Harry an.

\enquote{Warum?}

\enquote{Um nicht zu sterben}, antwortete er.

Harry nickte. \enquote{Gibt es noch weitere?}

Voldemort nickte.

\enquote{Wie viele?}, fragte Harry.
\end{traum}

Er drehte sich um und schlief weiter. Jetzt träumte er von Ginny.




\begin{kommentar}
Später macht Harry in einem Innenhof unter einem Baum seine Hausaufgaben. Er hat einen Alptraum und erwacht. Kurz darauf kommt McGonagall und beschwert sich bei Dumbledore über seine Toffees. Es sind Abführpralinen. Die anderen sind welche, bei denen man lachen muss, während sich der Darm entleert. Kann sein, dass es nicht so genau rüberkommt in der Geschichte.
\end{kommentar}

\begin{kommentar}
Einmal sagt Firenze: »Ja, belesen die junge Dame ist.« Eine kleine Hommage an Yoda aus Krieg der Sterne.
\end{kommentar}
