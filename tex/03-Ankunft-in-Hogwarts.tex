\chapter{Ankunft in Hogwarts}


Als Harry den Saal betrat, fiel ihm auf, dass Professor McGonagall auf dem Stuhl des Schulleiters saß und Professor Snape fehlte. Auch Ron und Hermine fiel das auf und die drei fragten sich, was wohl mit Dumbledore passiert sei. \enquote{Wir werden es wohl bald erfahren}, sagte Hermine.

Der Stuhl mit dem alten Hut stand bereits da. Es dauerte nicht lang und die große Flügeltür ging auf. Die Erstklässler betraten den Raum und alle schauten ihnen zu. Zu Harrys erstaunen begleitete Snape die jungen Schüler. Während dessen lief Hagrid außen um die Tische herum, um zu seinem Platz am Lehrertisch zu gelangen. Am Ende angekommen zog Snape ein Pergament aus seiner Tasche und sprach: \enquote{Ich werde Ihre Namen aufrufen und sie werden den Hut auf diesem Stuhl dort aufheben, sich auf den Stuhl setzen und danach werden sie sich den Hut aufsetzen. Er wird sie in ihre Häuser einteilen. Doch zuvor hören wir noch das Lied des Hutes.} Und der Hut hob von seinem Stuhl ab und begann schwebend zu singen.

\begin{lied}
Vor langer Zeit, was keiner weiß,\\
Hogwarts gegründet zu einer Zeit,\\
nach Krieg und Schrecken in jener Nacht\\
dies Schloss hier war ausgedacht.\\
Von den vier größten Zauberern damals sie waren,\\
Godric Gryffindor, Rowena Ravenclaw,\\
Helga Hufflepuff, und Salazar Slytherin.\\
Im Traume entdeckten sie dieses Schloss,\\
das ihnen als Schule dienen sollte.\\
Der Hausherr ganz zügig überließ es ihnen,\\
damit sie der Welt Vermächtnis erhalten.\\
Drum haltet zusammen, wie damals jetzt auch,\\
das Böse nur immer wartet darauf,\\
dass schwach wir werden und einsam dazu,\\
bekämpfen wir es ja klar nur zu.
\end{lied}

Der Hut sank wieder auf den Stuhl und Snape begann die Schüler aufzurufen.

\enquote{Was heißt hier: \inner{Der Hausherr ganz zügig überließ es ihnen,\abs}} fragte Hermine.

\enquote{Pst}, sagte Ron, \enquote{die Auswahlzeremonie.}

Allman Phillip, ein kleiner Junge, den Harry nicht genau sehen konnte, aber er vermutete, dass er auf seinem Gesicht neben seinem einzigen Auge auch noch Narben erkennen konnte, wurde zu den Hufflepuffs geschickt. Snape rief die anderen Nacheinander auf und sie wurden vom Hut in ihre Häuser eingeteilt und durch tosenden Applaus der jeweiligen Schüler auf ihren Platz begleitet. Harry ließ seinen Blick durch die noch verbleibenden Schüler schweifen und sein Blick blieb wieder bei einem kleinen blonden Mädchen hängen. Er erinnerte sich daran, dass er sie schon auf dem Bahnsteig gesehen hatte.

\begin{rueckblick}
Harry und Hermine passierten gerade die Absperrung und bogen ab. Fast wären sie in ein kleines Mädchen gelaufen.

\enquote{Oh Entschuldigung}, sagte sie. Dann erkannte sie ihn. \enquote{Du bist Harry Potter}, bemerkte sie. \enquote{Ich heiße Tamara. Mein Bruder hat mir von dir erzählt.} Dann drehte sie sich um und lief in den Zug.

Verdutzt sah er Hermine an. Er fragte sich, zu wem sie wohl gehören mochte.
\end{rueckblick}

Dann rief Professor Snape \enquote{Malfoy, Tamara.}

Harry drehte sich um und sah zu Malfoy hinüber. Er hatte also eine Schwester. \gedanke{Das könnte interessant werden}, dachte er sich. Professor Snape setzte ihr den Hut auf, nachdem sie sich gesetzt hatte. Der Hut bewegte sich und eine Weile passierte nichts. Harry fühlte sich an seine Auswahl erinnert.

\begin{rueckblick}
Ihm wurde der Hut aufgesetzt und er hörte den Hut in seinem Geiste. In ihm war viel Talent und der Drang, sich zu beweisen. Aber auch Mut steckte ihn ihm. Doch er wollte nicht nach Slytherin und wünschte sich nicht dorthin zu kommen. Also schickte ihn der Hut nach Gryffindor.
\end{rueckblick}

Der Hut brauchte immer noch um eine Entscheidung zu treffen und Malfoys Schwester sah sich in der Großen Halle um. Ihr Blick fiel auf ihren Bruder, der sie anlächelte, danach über die anderen Tische zu Harry. Sie lächelte ihn an. Schließlich war der Hut mit seiner Entscheidung fertig: \enquote{Wenn du dir sicher bist: \extase{Gryffindor!}} Der gesamte Tisch jubelte ihr zu und hieß sie herzlich willkommen. Harry sah wieder zu Malfoy hinüber und sah ihn lächeln. Er freute sich für seine Schwester. Warum, konnte er sich nicht erklären, aber das Lächeln stand ihm. Ron war leider alles andere als begeistert ihn so zu sehen. In all dem Jubel konnte er Ron nicht richtig verstehen, aber das war Harry nur recht. Sie hatten eine Malfoy, die in Gryffindor war. Vielleicht die Erste; vielleicht auch nach langer Zeit wieder eine. 

Pestrow, Cornelia schickte der Hut nach Ravenclaw und als vorletztes rief Snape Slystorp, Harold auf.

Harry kannte seine größere Schwester Anna, sie war im siebten Jahr und ihre Familie war seit mehreren Generationen in Slytherin. Er würde bestimmt auch dorthin kommen, dachte Harry. Aber zu seinem Erstaunen, und scheinbar zum Erstaunen vieler, kam er nach Gryffindor. Harry blickte schnell zu seiner Schwester, die ein entsetztes Gesicht machte. Harry musste grinsen. Es hatte für ihn den Anschein, dass Harold ganz froh darüber war. Snape rief noch Zabini Ludger auf und dieser wurde vom Hut nach Slytherin geschickt.

Nachdem alle Schüler ihren Häusern zugeordnet waren und Snape den Stuhl samt Hut nun mitgenommen hatte, stand Professor McGonagall auf und sprach zu den Schülern.

\enquote{Ich freue mich, dieses Jahr viele neue Schüler begrüßen zu dürfen und möchte so viele junge und enthusiastische Neuankömmlinge herzlich willkommen heißen. Der Schulleiter Professor Dumbledore hat heute leider keine Zeit dem Fest und der Auswahlzeremonie beizuwohnen, da er mit anderen wichtigen Dingen beschäftigt ist. Ihren neuen Lehrer in \VgddK werden sie morgen beim Frühstück kennenlernen. Seien sie also pünktlich. Den Neuankömmlingen möchte ich sagen, dass ihr Professor Dumbledore morgen früh sehen werdet und er euch noch einmal persönlich begrüßen möchte.}

Sie machte eine kurze Pause und fuhr dann fort. \enquote{Mister Filch, unser Hausmeister}, sie zeigt auf ihn und die neuen drehten sich um, \enquote{bat mich, ihnen mitzuteilen, dass das Ausüben von Zaubern oder von Flüchen jedweder Art auf den Korridoren untersagt ist und dass der verbotene Wald aus gutem Grund verbotener Wald heißt, weil es nicht erlaubt ist diesen zu betreten. Möge das Fest nun beginnen}, sie holte ihren Zauberstab heraus und machte eine schwingende Bewegung. Sofort erscheinen auf den langen Haustischen die unterschiedlichsten Speisen und Getränke.

\enquote{Ob sie ihren Zauberstab dazu wirklich braucht?}, fragte Ron Hermine.

\enquote{Weiß nicht}, antwortete sie, \enquote{vielleicht wollte sie es auch nur etwas effektvoller aussehen lassen.}

\enquote{Aber was wohl Dumbledore gerade macht? Sucht er einen neuen Vgddk-Lehrer und hofft, dass er morgen gleich mitkommt?}, fragte Harry.

Nicht weiter darüber nachdenkend, begannen sie zu essen. Ron und Hermine begleiteten nach dem Essen die Neuankömmlinge in ihre Quartiere und standen noch eine Weile ihren Fragen zur Verfügung, Ron den Buben und Hermine den Mädchen. Ebenso die anderen Vertrauensschüler, die die kleine Gruppe begleitete.

\trenn

In dieser Nacht hatte Harry einen seltsamen Traum.

\begin{traum}
Er stand in einem gewölbeartigen Raum. Um ihn herum lagen Goldmünzen und Bestecke, sowie Teller und andere Gegenstände aus Gold auf Tischen und auf dem Boden verteilt. Harry sah sich im Raum um. Auf einem der dunklen, fast schon schwarzen Regale, die an der Wand standen, entdeckte er einen goldenen Trinkpokal. Auf ihm war ein Dachs abgebildet. Er sah sich weiter in dem Raum um, doch immer wieder zog es seinen Blick auf den Pokal auf dem Regal zurück.

Hinter sich sah er eine Tür. Er überlegte eine Weile, während er immer wieder auf den Pokal sah. Dann traf ihn eine Erkenntnis. Er stand wohl in einem Verlies in Gringotts. Aber wie kam er hierher? Gerade als er sich noch über diese Frage den Kopf zerbrach, hörte er mechanische Geräusche. Panikartig suchte er nach einem Versteck. Doch nichts kam in Betracht. Die Tür wurde geöffnet und Bellatrix Lestrange betrat das Verlies.

Harry war dabei seinen Zauberstab zu ziehen, doch er hatte ihn nicht dabei. Die irre Hexe trat direkt auf ihn zu und blieb vor ihm stehen. Schweiß lief von seiner Stirn. Doch Bellatrix sah sich nur um, nahm einige Münzen und schob sie in ihren Umhang. Harry stand wie paralysiert da. Es lief ihm kalt den Rücken hinunter. \gedanke{Hat sie mich gesehen und jagt mir kurz bevor sie geht einen Fluch hinterher? Oder hat sie mich nicht bemerkt? Nein, sie muss mich gesehen haben}, dachte er sich.

Dann traf ihn der sprichwörtliche Blitz der Erkenntnis. Er war am Träumen. \gedanke{Das würde auch erklären, dass sie mich nicht gesehen hatte und dass ich meinen Zauberstab nicht dabei habe}, dachte er sich.

Als Bellatrix ihr Verlies wieder verließ, folgte er ihr. Doch sein Blick zog es ein letztes Mal zu diesem Pokal mit dem Dachs. Seinen Blick rückwärts gerichtet, verließ er das Gewölbe. Sein Blick fiel auf die Nummer des Verlieses: 431. Unbewusst registrierte er die Zahl. Dann wechselte die Szene auf eine Wiese und die Erkenntnis, dass er träumte, verlor sich in den über ihm schwebenden Wolken.
\end{traum}

Harry wachte auf und stellte fest, dass er alleine im Schlafsaal war. Er hörte ein leises: \enquote{Guten Morgen Harry.} Es war Myrte.

\enquote{Was machst du hier Myrte?}, fragte er, ehe er ein: \enquote{Guten Morgen übrigens}, hinterher gab, da er nicht unhöflich sein wollte und da ihm Myrte das sicherlich wieder vorwerfen würde.

\enquote{Ich wollte dich mal wieder besuchen. Du warst schon lange nicht mehr bei mir, Harry}, sagte sie.

Harry schwang die Füße aus seinem Bett und sah Myrte nun direkt an.

Interessiert betrachtete sie sein Basilisken-Amulett. \enquote{Oh, woher hast du denn das?}, fragte sie.

\enquote{Das war ein Geschenk von Ginny}, antwortete Harry.

Myrte schwebte näher. \enquote{Leuchtet es leicht, wenn du es anfasst?}, fragte sie ihn.

Harry staunte und sagte dann: \enquote{Ich weiß es nicht.} Dann umfasst er das Amulett und hielt es kurz darauf in der offenen Hand, da er kaum etwas sehen konnte, wenn es in seiner Hand eingeschlossen war. Und tatsächlich, die Augen leuchteten auf.

Myrte ließ einen glücklichen Seufzer los. \enquote{Das ist schön}, antwortete sie. Sie kam Harry näher und drückte ihm einen Kuss auf seine Wange.

Harry spürte eine leichte, aber kurze Kälte, ehe sich Myrte wieder löste, ihm zuwinkte und durch die Wand verschwand.

Ron und Hermine saßen bereits am Frühstückstisch, als Harry hereinkam und setzte sich. Als die Erstklässler alle zur gleichen Zeit hereinkamen und sich setzten, stand Dumbledore auf und begann mit seiner Rede.

\enquote{Es freut mich, dass auch dieses Jahr wieder viele Erstklässler dabei sind, in die Geheimnisse der Zauberei eingewiesen zu werden. Ich nehme an, Professor McGonagall hat ihnen bereits erzählt, was es mit dem verbotenen Wald auf sich hat und dass das Zaubern auf den Gängen untersagt ist. Ich möchte\abs}

Doch Dumbledore kam nicht mehr dazu den Rest zu sagen, denn ein lauter Knall war durch die ganze Halle zu hören. Eine Schrecksekunde später kam ein blau schimmernder Vorhang zur linken Seite der Großen Halle herein und bahnte sich seinen Weg durch die Selbige. Innerhalb einer Sekunde war alles vorbei. Harry fröstelte es ein bisschen. Dann konnte er einen weiteren Knall hören, und noch einen und noch einen. Plötzlich fiel ein Etwas durch die Decke und plumpste auf den Boden. \gedanke{Es sah aus wie ein Geist}, dachte Harry. Nach ein paar Sekunden begann er im Boden zu versinken. Harry schaute zu Professor Dumbledore und fragte sich, was denn nun passiert sei. Lautes Gemurmel erfüllte plötzlich die Halle. Keiner wusste genau was passiert war. Harry sah, wie Professor Dumbledore Professor McGonagall etwas ins Ohr flüsterte, um den Lehrertisch herum lief und zwischen den Reihen die Große Halle verließ. Er hatte es sehr eilig und beachtete niemanden.

Harry schaute zurück zu Ron und Hermine und dachte bereits an Fred und George. \enquote{Meinst du Fred und George haben etwas damit zu tun?}, fragte er Ron.

Ron antwortete: \enquote{Ich glaube nicht, die sind nicht so fahrlässig und würden es riskieren jemanden zu verletzen, oder einen Geist durch das halbe Schloss schleudern. Und außerdem sind sie nicht mehr da.}

\enquote{Was immer es auch war}, sagte Hermine, \enquote{die Explosion muss von außerhalb des Schlosses gekommen sein.}

Harry und Ron sahen sie nur ungläubig an.

\enquote{Wie kommst du darauf?}, fragte Ron.

\enquote{Na ja, schau dir doch mal an von wo die blaue Welle kam.} Hermine zeigte auf die Wand. \enquote{Und dahinter ist nicht mehr viel. Also muss die Explosion von außerhalb des Schlosses gekommen sein.}

Ron schaute sie mit leichtem Entsetzten im Gesicht an.

Nach dem Essen machten sich die drei wieder auf den Weg zu ihren Schlafsälen, wo ihre Stundenpläne in der Zwischenzeit auf ihre Betten gelegt worden waren. Harry freute sich, als er montags in der ersten Stunde \fach{Pflege magischer Geschöpfe} bei Hagrid hatte. \gedanke{Mal sehen was Hagrid wohl wieder neues dran bringt}, dachte Harry. Er schnappte sich sein Schulbuch und machte sich auf den Weg zu Hagrids Hütte. Dieses Jahr hatte Hagrid wieder ein normales Buch und kein beißendes wie vor drei Jahren. Damals hatte Hagrid unter anderem auch Hippogreife durchgenommen. \gedanke{Wie es jetzt wohl Seidenschnabel geht?}, fragte er sich.

Vor Hagrids Hütte angekommen führte er sie in Richtung eines Geheges, das Leer zu sein scheint. Vor dem Gehege angekommen sagte Hagrid: \enquote{Heute fangen wir mit Clestinern an. Clestiner sind schwer zu sehen. Es sei denn man singt ihnen etwas vor. Dann können sie nicht widerstehen und zeigen sich uns.} Harry betrachtete das Gehege und meinte etwas zu sehen. Etwas, das wie fahle, kaum sichtbare Umrisse aussah. Etwas, das ihn an einen Elch erinnerte. Hagrid öffnete das Gehege und deutete seinen Schülern an, hereinzukommen. Nachdem alle drinnen waren, schloss er das Gatter wieder und holte eine kleine Flöte aus einer seiner Taschen.

\enquote{So, jetzt passt mal auf}, sagte er und begann zu spielen. Viele Schüler hielten sich die Ohren zu, denn Hagrid konnte nicht besonders gut auf der Flöte spielen. Jetzt zeigten sich, zwar noch durchlässig, aber man konnte sie schon sehen, die Clestiner. Geschöpfe, die ein dunkelgrünes und an manchen Stellen Moos bewachsenes Fell hatten. Ihre Schaufeln waren weiß und die Hufe glänzten in einem Purpur schimmernden rot. Hagrid hörte sein Spielen auf der Flöte auf, worüber Harry mehr als nur froh war.

\enquote{Ich weiß, das war nicht besonders gut, aber so habt'er jedenfalls eine Vorstellung, von dem, was euch erwartet. Jeder nimmt sich ein' Beutel aus dem Kasten hinter mir. Dann verteilt euch im Gehege und setzt euch auf'n Boden, öffnet den Beutel und legt den Inhalt vor euch auf'n Boden. Wenner merkt, dass euch ein Clestiner den Inhalt aufisst, singt 'r ihm etwas vor, damit 'r 'n sehen könnt. Wenn 'r 'n vollständig seht, gebt ihm schnell einen Namen. Akzeptiert er ihn, wird er sich euch nähern und ihr müsst 'n streicheln. Ab da könnt 'r 'n dann immer sehen. Natürlich nur euren eigenen.}

Die Schüler drängten sich nun um den Kasten hinter Hagrid und jeder nahm sich einen Beutel heraus. Dann verteilten sie sich im Gehege und setzten sich wie geheißen auf den Boden. Es hörte sich eigenartig an, als alle ein Liedchen in die unsichtbare Morgenluft hinein trällerten.

Harry wollte seinem Clestiner den Namen \accentuate{Mike} geben als er merkte, dass es sich um ein Weibchen handelte. So entschied er sich für \accentuate{Rosalie}. Anscheinend gefiel ihr der Name und sie kam näher, um ihn abzuschnuppern. Harry streichelte ihr grünes Fell und das Weibchen begann sein Gesicht zu lecken.

\enquote{Wer fertig ist, steht auf und versucht sich auf sein' Clestiner zu setzen.}

Harry hatte etwas Mühe, da Clestiner doch recht große Tiere waren und man auf sie nicht so leicht wie auf ein Pony steigen konnte. Rosalie hatte anscheinend Mitleid mit ihm, als es ihm auch nach seinem fünften Versuch nicht gelang, auf ihren Rücken zu steigen. Sie senkte ihren Vorderkörper, indem sie die Vorderbeine einknickte, und gab Harry so die Möglichkeit aufzusteigen. Hagrid grinste, als alle ihre Tiere bestiegen hatten.

\gedanke{Es muss für Hagrid ein ungewöhnlicher Anblick sein}, dachte Harry.

\enquote{So, wenn alle fertig sind, können wir ja weiter machen.} Er schlug die Hände gegeneinander, was einige Clestiner dazu brachte zusammen zu zucken. Einige Schüler mussten sich festhalten, um nicht herunterzufallen. \enquote{Wenn euch eure Schulkameraden die Namen ihrer Clestiner verraten, ihr sie ruft und die dann zu euch kommen, könnt ihr auch die eurer Schulkameraden sehen. Aber das macht ihr am besten nach Gutdünken aus. Die nächsten vier Wochen werden wir uns um die Aufzucht und die Pflege eurer Tiere kümmern. Ihr werdet alles darüber Lernen, wie sie euch helfen können. Wie sie Verletzungen heilen können, wie sie euch in einem Kampf einen Überraschungsmoment verschaffen können und wie ihr euch mit ihnen unterhalten könnt. Nach den vier Wochen gibt es eine kleine Prüfung und den Rest des Schuljahres müsst ihr sie nur ab und an besuchen, damit sie an euch gewöhnt bleiben. Wer will, kann sie am Ende des Schuljahres mit nach Hause nehmen.}

Harry musste bei dem Gedanken grinsen, denn er stellte sich gerade Dudleys, oder Onkel Vernons Gesichter vor, wie Rosalie durch den Garten lief, Spuren hinterließ, sie aber nichts sehen konnten. Aber sie würden bestimmt ihn im Verdacht haben und ihm deshalb wieder Ärger verursachen. Der Rest der Stunde verlief recht angenehm und jeder kümmerte sich um seinen Clestiner.

\gedanke{Dieses Mal scheint Hagrid kein gefährliches Tier mit im Unterricht zu haben. Zumindest noch nicht.}

Die Stunde bei Professor McGonagall bestand heute aus Wiederholungen und dem Besprechen des diesjährigen Schulstoffes. Es war also recht anspruchslos. Zumindest dachte Harry das.

Ebenso erging es der Klasse heute in Kräuterkunde. Professor Sprout ließ sie wieder Alraunen umtopfen, da es diese bitter nötig hatten und die Vorgängerklasse \gst das zweite Schuljahr \gst es nicht schaffte. Danach wechselten sie noch das Gewächshaus und besahen sich die Pflanzen, die sie ab dem nächsten Mal durchnehmen wollten. Es waren rankenartige Fleischfresser. Zwar standen auf der Speisekarte der Pflanzen maximal Mäuse, aber vor solchen Pflanzen musste man sich trotzdem in Acht nehmen, da sie sonst zubissen.

Die erste richtige Herausforderung kam aber jetzt auf der Krankenstation.

Dieses Jahr hatte Harry montags Unterricht bei Madame Pomfrey, welche sich mal wieder beim Schulleiter Professor Dumbledore beschwert hatte und Erste-Hilfe-Kurse für die Schüler für sinnvoll befand. Professor Dumbledore gab dem nach und so hatten sie am Montag immer Heilkunde.

Heute nahm sie Madame Pomfrey richtig ran. Sie mussten heute den richtigen Umgang mit dem Zauberstab zu Diagnosezwecken lernen. Etwas, was jedem Heiler und jeder Krankenschwester beigebracht wird. Die Kunst, den Stab richtig zu führen zu erlernen, dauerte meist ein paar Stunden, hält aber \gst auch bei seltener Anwendung \gst jahrelang an und bedarf nur gelegentlicher Auffrischung.

Die Klasse schnaufte, da Madame Pomfrey bereits zu jedem mehrmals kam, um die Haltung des Zauberstabes zu korrigieren.

\enquote{Warum brauchen wir das eigentlich? Bei unseren anderen Zaubern achten wir auch nicht auf die richtige Handhaltung.}

\enquote{Da hängt auch nicht das Leben ihres Patienten davon ab. Die Handhaltung ist für diese präzisen Zauber sehr wichtig. Sie beeinflusst die Magie, die sie wirken. Deshalb bin ich so streng in diesem Punkt. Eine falsche Handhaltung und die Diagnose verändert sich.} Sie suchte sich einen Sitzplatz und erzählte dann weiter. \enquote{In früheren Zeiten wurde so manche Fehldiagnose gestellt, die die Leute auch in den Selbstmord trieb. Erst als Heiler Antyos und seine Frau aus Versehen den gleichen Patienten untersuchten und verschiedene Diagnosen erstellten, begannen sie zu forschen. Sie beobachteten einander, als sie ihre Patienten untersuchten, und stellten Differenzen in der Art der Zauberstabhaltung fest. Sie experimentierten mit verschiedenen Haltungen und entwickelten die Grundlagen der modernen Diagnostik.}

\gedanke{Deshalb triezt sie uns so}, dachte Harry.

\trenn

Es war eine ruhige Nacht, ein Uhr vierundzwanzig und der Wind blies nur langsam und ab und an. Eine dunkle, ganz in schwarz gekleidete Gestalt trat aus dem Zwielicht heraus auf den Weg. Langsam bewegte sich die menschliche Gestalt auf das verschlossene Tor mit den geflügelten Ebern zu, die auf den Steinsäulen links und rechts das Tor flankierten. Die Gestalt blieb davor stehen und sah durch das schmiedeeiserne Flügeltor auf den Weg, der zum Schloss führt. Dann sah sie zu den Ebern hoch und betrachtete beide eine Weile. Schließlich zog sie ihren Zauberstab und fuhr den schmalen Spalt zwischen den beiden Torflügeln entlang. Das Tor öffnete sich geräuschlos und gab den Weg frei. Als die Gestalt hindurchgeschritten war, erhoben sich die Eber auf ihre Hinterbeine und schlugen mit den Flügeln. Die Gestalt winkte ab und hinter ihr schloss sich das Tor wieder und sie lief den Weg zum Schloss entlang hoch.

Das wenige Licht der Nacht brach durch die Bäume und Äste und fiel auf den Weg. Der Umhang der Gestalt leuchtete leicht bräunlich im fahlen Licht, welches auf den Gehweg schien. Kurz nachdem das Schloss wieder in Sichtweite gekommen war, blieb die Gestalt stehen und fuhr sich mit einer Handschuh-verpackten Hand übers Gesicht. An Stelle der Augen waren nun kaum zu sehende, schwache, grünliche Lichter zu sehen. Aufmerksam beobachtete die Gestalt das Schloss und es schien fast so, als ob sie durch die Wände in das Innere schauen würde.

Dann hörte das Leuchten der Augen auf und die Gestalt lief den Weg weiter zum Schloss. Vor der großen Flügeltür blieb sie wieder stehen und besah sich das Holz der Tür. Wieder griff die Gestalt in ihren Umhang und zog erneut den Zauberstab heraus. In einer waagerechten Bewegung fuhr sie vor sich auf dem Holz entlang. Die Maserung wurde undeutlicher und leichte Wellenbewegungen zeichneten sich ab, wie bei einer Wasseroberfläche über die leicht der Wind strich. Die Gestalt schritt hindurch und die Oberfläche zeichnete den Umriss der Person nach. Dann wurde die Maserung wieder deutlicher und das Holz war nicht mehr durchdringbar. Zwei dunkle Glockenschläge erklangen synchron, aber die zweite Glocke schlug um einen Halbton höher als die Erste und so konnte man doch deutlich den Glockenschlag, der zur Ankunft des seltsamen Gastes erklang, von dem, der zur halben Stunde schlug, unterscheiden.

Im Inneren des Schlosses vollzog die Gestalt dieselbe Bewegung wie im Schatten der Bäume, um das Schloss zu beobachten. Mit erneut grün leuchtenden Augen sah sich die Gestalt sorgfältig um, ohne jedoch den Platz knapp hinter der Tür zu verlassen. Als die Augen wieder mit dem Dunkel verschmolzen, nahm sie ihren Zauberstab und ging ein paar Schritte am Holzportal entlang und suchte mit einer Hand ein kleines Loch in der Wand neben dem Scharnier der Tür. Sie steckte ihren Zauberstab bis zum Griff in das Loch, worauf hin der Griff leicht heller wurde aber nicht selbst zu Leuchten begann. Das Licht im Raum wurde etwas heller und man sah nun den Umhang in dunklem Braun mit grauen Schlieren darin, welcher perfekt mit jeder dunklen Umgebung verschmolz. Doch selbst in der leicht erhellten Umgebung war zu wenig Licht, um ganz zu erkennen, welche Gestalt dort stand. Sie drückte ihre Hand an die Wand, behielt aber immer den restlichen Körper und den Blick in die Hallenmitte gerichtet.

Mit leiser und krächzender Stimme begann sie einen Monolog, in dem das Schloss durch Töne antwortete.

\enquote{Hogwarts. Ich werde in Zukunft öfters hier im Schloss auftauchen. Bitte mach bis auf Weiteres keine große Sache daraus. Behandle mich wie jeden Gast hier in deinen Mauern. Keine Extrabehandlungen \gst Hast du mich verstanden?} Ein leiser hoher Glockenschlag erklang wie zur Bestätigung. \enquote{Ich danke dir}, sagte die Gestalt und verneigte sich. Sie nahm ihre Hand von der Wand, worauf die Beleuchtung des Raumes zurückging. Danach zog sie ihren Zauberstab aus der Wand und berührte damit, ohne sich umzudrehen, die Tür hinter sich.

Abermals zerfloss die Struktur des Holzes und die Gestalt trat rückwärts aus dem Schloss heraus und sofort begann sich die Struktur wieder zu verfestigen. Nach einem Kreuzschritt und einer geschmeidigen Drehung schritt die Gestalt den Weg zum Tor hinunter. Das gusseiserne Tor öffnete sich automatisch und die geflügelten Eber gaben nur ein kurzes Grunzen vor sich. Ein kleines und leises Kichern ging von der Gestalt aus, bevor sie während des Laufens lautlos verschwand. Eine einzelne Windböe verwischte die restlichen Spuren im Kies und den durch Dreck, Sand und Laub bedeckten Stellen des Weges.

\trenn

Am nächsten Tag ging jeder der Gryffindors nach der Stunde im Gewächshaus 6 in Richtung Krankenstation. Sie verabschiedeten sich von den Hufflepuffs, mit denen sie immer Kräuterkunde hatten und gingen zur nächsten Stunde.

In der Krankenstation angekommen wartete bereits Madame Pomfrey und die Ravenclaws auf sie. Sie deutete ihnen an, sich in einem Halbkreis aufzustellen. \enquote{Das Wichtigste, das sie lernen werden, ist, verschiedene Verletzungen zu erkennen und herauszufinden, ob der Verletzte transportiert werden kann. Warten Sie hier kurz.} Sie drehte sich um und verschwand in ihrem Büro. Kurz darauf kam sie mit einer Puppe zurück, die so groß wie ein ausgewachsener Mensch war. \enquote{An ihm werden sie in den folgenden paar Wochen lernen, zu erkennen, welche Art Verletzung die betroffene Person hat. Ab dem nächsten Mal werden ihnen mehrere Avatare zur Verfügung stehen.} Sie nahm ihren Zauberstab aus ihrer Schürze und fuhr dem Avatar auf und ab. \enquote{Mister Longbottom, würden Sie bitte ihren Zauberstab nehmen, ihn über den Avatar halten und dann folgendes sagen: \zauber{Scindescenti}}

% ital. 	scintille incandescenti
Neville nahm seinen Zauberstab und sprach: \zauber{Scindescenti}, worauf Funken aus der Spitze seines Stabes sprühten und den Avatar bedeckten.

Neville erschrak kurz, doch Madame Pomfrey fing an weiterzuerzählen. \enquote{Sie machen das sehr gut, Mister Longbottom. Das klappt hervorragend. \gst An der Art, Form und Farbe der Funken können sie erkennen, um welche Art von Verletzung oder Krankheit es sich handelt. \gst Betrachten Sie bitte die Funken. Hier handelt es sich um gelbliche Kugelfunken mit einem leichten bläulichen Einstich. \gst Sie können jetzt aufhören.} Neville nahm seinen Zauberstab zurück und stellte sich wieder in die Reihe. \enquote{Schlagen Sie bitte in ihren Büchern nach und suchen sie nach den entsprechenden Funken. Jeder von ihnen schreibt mir dann seine Meinung dazu auf einen kleinen Zettel \gst ein paar Sätze genügen. Gebt ihn dann bei mir ab. Die Zeit läuft.}

\enquote{Wie, jetzt?}, fragten ein paar Schüler.

\enquote{Na klar jetzt, der Verletzte wartet nicht, bis sie Zeit haben, sich um ihn zu kümmern und sich wohlgenährt an die Arbeit machen.} Sie räumte den Avatar zur Seite und setzte sich entspannt auf ein leeres Bett in der Nähe. Auf der Krankenstation herrschte reges Treiben. Überall saßen oder standen Schüler und blätterten in ihren Büchern.

Madame Pomfrey wartete geduldig auf die Annahmen und Vermutungen. Bereits nach wenigen Minuten kamen die ersten Schüler und gaben ihre Zettel ab. Madame Pomfrey nahm sie entgegen und las sie aufmerksam durch. \enquote{Interessante Bemerkungen, das muss ich schon sagen.} Ihre Augen weiteten sich etwas. \enquote{Und bei einigen ist der Patient bereits dem Tod nahe.}

Sie stand wieder auf und meinte dann: \enquote{Wir werden uns jetzt erst mal um den richtigen Umgang mit dem Zauberstab kümmern. Mir scheint, sie brauchen eine Auffrischung. Nehmen sie ihn mal alle hervor.} Sie wartete, bis alle so weit waren und bewegte ihren Zauberstab in der Luft. \enquote{Scindescenti}. Viele weiße sternförmige Funken sprühten ohne Unterlass aus der Spitze ihres Zauberstabes. \enquote{Als Erstes lassen Sie bitte alle diese netten Analysefunken aus ihren Zauberstäben heraussprudeln. Da es nichts gibt, worauf sie sich niederlassen könnten, das einer Heilung bedarf, bleiben die Funken weiß und ihre Form wird sich auch nicht verändern.} Jeder schwang nun seinen Zauberstab und schon sprudelten die ersten Funken heraus.

\enquote{Sehr schön. Sie können dann, wenn ihre Funken weiß und sternförmig sind, ihren Zauberstab gegen ihren eigenen Körper richten. Dort dürften sie, wenn sie gesund sind, keinerlei Änderung in Form oder Farbe feststellen.}

Harrys Funken sprühten aus seinem Zauberstab heraus und er beobachtete sie, wie sie Richtung Boden fielen. Kurz davor begannen sie sich aufzulösen. Dann richtete er seinen Zauberstab gegen sich selbst und schaute, ob sich die Funken veränderten.

Als die Funken seinen Körper berührten leuchteten sie nur kurz auf und verschwanden dann. Es hatte nicht den Anschein, dass Harry krank war. Er fühlte sich zudem kerngesund. Harry schaute sich um. Es waren nur vereinzelt leichte Färbungen bei anderen zu erkennen. \gedanke{Nichts Schlimmes}, dachte er sich. Madame Pomfrey löste noch die Krankheit des Dummys auf, es war eine mittelschwere Bronchitis. Nachdem die Stunde beendet war, ging er wie gewöhnlich zum Abendessen in die Große Halle.

Es war Zeit zum Abendessen und als Harry die Große Halle betrat, stand Dumbledore geduldig hinter seinem Podest, auf dem er immer Ansprachen hielt. Als alle zum Abendessen gekommen waren und die Türen der Großen Halle geschlossen wurden, begann Dumbledore. \enquote{Nachdem ich heute Morgen unterbrochen worden bin, machen wir eben jetzt weiter. Viele von Ihnen wissen noch, was letztes Jahr passiert ist, also brauche ich mich nicht zu wiederholen. Umso erfreuter bin ich, dass ich jetzt einige Preise vergeben darf. Ich bitte alle Mitglieder des ehemaligen Inquisitionskommandos zu mir vor. Sie haben sich alle den \accentuate{Salazar Slytherin}-Orden für Hinterlist und Spionage verdient.} Eine Unruhe durchdrang nun die Große Halle und nacheinander standen die Inquisitionsmitglieder auf und gingen nach vorne. \enquote{Die Mädchen bitte links von mir, die Jungen bitte Rechts.}

Professor McGonagall stand auf und lief um den Tisch herum zu den Jungs. Als alle standen, nahm Professor Dumbledore von Professor McGonagall ein kleines Holzkästchen entgegen und begann, den Mädchen die einzelnen Orden anzuheften und ihnen zu gratulieren.

\enquote{Bleiben Sie noch kurz stehen}, sagte Dumbledore zu den stolzen Empfängern der Orden. Er schritt zurück zu seinem Podest und Professor McGonagall verschwand währenddessen in einen kleinen Nebenraum, um kurz darauf mit noch ein paar Holzkästchen wiederzukommen. Sie heftete auch den Jungs die Orden an. \enquote{Dieser Schulorden macht ihnen alle Ehre. Sie dürfen sich setzen.} Die Ordensträger gingen jetzt auf ihre alten Plätze, doch Dumbledore stand immer noch hinter dem Podest. Professor McGonagall stand an seiner Seite mit ihrem Holzkästchen in der Hand. Als sich die Slytherins gesetzt hatten, fing Dumbledore wieder an.

\enquote{Ich möchte nun folgende Personen zu mir bitten: Hannah Abbott, Katie Bell, Lavender Brown, Susan Bones, Terry Boot, Cho Chang, Michael Corner, Colin Creevey, Dennis Creevey, Justin Finch-Fletchley, Hermine Granger, Anthony Goldstein, Lee Jordan, Neville Longbottom, Luna Lovegood, Ernie Macmillan, Parvati Patil, Padma Patil, Harry Potter, Zacharias Smith, Alicia Spinnet, Dean Thomas, Fred Weasley, George Weasley, Ginny Weasley und Ronald Weasley.}

Harry wunderte sich, was denn jetzt noch kommen würde. Er stand wie die anderen auf und ging nach vorne. \enquote{Alles Mitglieder der DA} hörte er Hermine sagen.

Harry war erstaunt. \gedanke{Werden wir auch einen Schulpreis erhalten?}, fragte er sich.

Vorne angekommen, stand jetzt eine bunt gemischte Reihe. Dieses Mal hatte  Dumbledore die Schüler nicht nach Geschlecht sortiert. \enquote{Diese Schüler hier}, erklärte Dumbledore, \enquote{haben mehr getan, als jeder andere, letztes Schuljahr. Sie haben sich gegen Verleumdung und Verhetzung gestellt und sich sogar selber ausgebildet, als sie in einem bestimmten Fach nur die Theorie erlernen sollten.} Dumbledore machte eine kurze Pause, damit auch allen klar war, wen er damit meinte.

\gedanke{Umbridge}, kam es Harry in den Sinn.

\enquote{Das Kollegium ist der Meinung, dass dieses Verhalten belohnt werden soll. Ich möchte nun folgende Personen bitten einen Schritt vor zu treten: Hannah Abbott, Katie Bell, Lavender Brown, Susan Bones, Terry Boot, Cho Chang, Michael Corner, Colin Creevey, Dennis Creevey, Justin Finch-Fletchley, Anthony Goldstein, Angelina Johnson, Lee Jordan, Ernie Macmillan, Parvati Patil, Padma Patil, Zacharias Smith, Alicia Spinnet und Dean Thomas.}

\enquote{Sie alle erhalten den Orden der Schule zweiter Klasse.} Ein leises Raunen ging durch die Große Halle. Währenddessen ging Professor McGonagall um die Schüler herum und heftete ihnen jeweils einen Orden an. Die Holzkästchen neben ihr schwebend und das Oberste geöffnet, entnahm sie jeweils einen Orden. Nachdem alle ihre Orden erhalten hatten, machte Dumbledore weiter. \enquote{Den Orden der Schule erster Klasse erhalten folgende Schüler: Hermine Granger, Neville Longbottom, Luna Lovegood, Harry Potter, Ginny Weasley und Ronald Weasley.} Wieder heftete Professor McGonagall die Orden an, dieses Mal aus der unteren Schachtel, die ihre Position nach oben wechselte.

\enquote{Als Letztes möchte ich noch zwei Urkunden vergeben. Eine für die grandiose Idee, diese Gruppe zu gründen und für Verwendung eines Protheus-Zaubers. Hermine Granger, bitte kommen Sie zu mir.} Er zog ein Pergament aus seiner Tasche und überreichte es Hermine. \enquote{Und eine weitere an Harry Potter, für die Leistung ihren Mitschülern die Anwendung der Zauber beizubringen. Vor allem aber den Patronus-Zauber, den sie vielen ihrer Mitschüler beibrachten, obwohl er schwer ist. Viele erwachsene Hexen und Zauberer können das nicht.} Harry ging zu Dumbledore. Dieser überreichte ihm die Urkunde und Harry kehrte dann auf seinen Platz zurück. \enquote{Nachdem nun die Verleihungen beendet sind, wünsche ich allen ein angenehmes Abendessen. \gst Noch eines: Die Schüler, welche die Schule bereits verlassen haben, bekommen die Orden natürlich nachgeliefert.} Er klatschte wieder in die Hände und augenblicklich erschienen auf den Tischen die Speisen und Getränke.

Harry belud seinen Teller und betrachtete, während er aß, ausgiebig seinen Stundenplan. \gedanke{Warum ist mir das nicht schon heute Morgen aufgefallen?} \enquote{Hermine, ich habe auf meinem Plan kein \fach{Verteidigung gegen die dunklen Künste} mehr}.

Hermine schreckte hoch. \enquote{Das kann nicht sein.} Sie zog ihren Stundenplan aus ihrer Tasche und überflog ihn. \enquote{Bei mir fehlt es auch.} In diesem Moment lief Professor McGonagall am Tisch vorbei. \enquote{Professor}, fragte Hermine. \enquote{Warum fehlt bei Harry und mir \fach{Verteidigung gegen die dunklen Künste}? Wir haben dieses Fach doch belegt. Ist das ein Versehen?}

\enquote{Oh nein, Miss Granger. Sie haben dieses Fach jetzt noch nicht. Erst in ein paar Tagen. Es ist etwas dazwischen gekommen.}

Hermines Augen weiteten sich. \enquote{In ein paar Tagen? Und was machen wir so lange?}

Professor McGonagall zog die Schultern hoch. \enquote{Ich bin mir sicher, dass \accentuate{ihnen} etwas einfällt.}

Hermine starrte auf ihren Stundenplan, während Professor McGonagall sich entfernte.

\trenn

Harry stand mal wieder im Wald. Er betrachtete die Thestrale \gst wie schon so oft \gst wie auch in den Tagen an den Wochenenden zuvor. Seine Gedanken glitten dahin und vorsichtig, als ob sie ihn nicht erschrecken wollte, näherte sich ihm ein Thestralweibchen um ihn abzuschnuppern. Thestrale waren eigenartige Tiere, sie konnten nur von Personen gesehen werden, die dem Tod ins Auge geblickt hatten. Harry schmunzelte bei dem Gedanken wie die anderen letztes Jahr sich wohlfühlen mussten, als sie auf einem für sie unsichtbaren Tier nach London flogen. Denn außer für Luna und ihn waren sie für fast alle anderen Schüler nicht sichtbar. Harry fragte sich, ob es eine Auszeichnung sei. Mann musste dem Tod ins Auge schauen, sehen, wie jemand stirbt. Ein grässlicher Gedanke. Doch andererseits wird man mit Thestralen belohnt, sanftmütige Kreaturen, die wohl nur deshalb versteckt waren, weil die meisten sich vor ihnen fürchten. Das Thestralweibchen schnupperte noch immer an ihm und auch die anderen schienen sich für ihn zu interessieren, denn langsam kamen einige von ihnen näher. Harry spürte, dass sich von hinten etwas näherte, das kein Thestral zu sein schien.

\enquote{Hallo Luna}, sagte er, wohl wissend, dass nur sie es sein konnte.

\enquote{Hallo Harry}, kam es ihm verträumt entgegen.

Sie kam näher und stellte sich neben ihn. Ihre Hand berührte leicht die seine. Harry war ganz in Gedanken und bemerkte es zuerst nicht. Es war ein lauer Samstagmorgen, die Sonne schien und erhellte das Waldstück. Es war ein toller Anblick, wie die Lichtstrahlen sich um die Bäume und Büsche herum ihren Weg auf den Boden suchten, um ihn langsam zu erwärmen und die golden glänzenden Lichtstrahlen das Moos und die Blätter auf dem Waldboden ausleuchteten. Nach ein paar Minuten waren sie von einigen Thestralen umkreist, die die beiden beobachteten und beschnupperten. So langsam bemerkte Harry Lunas Hand und sah sie an.

\enquote{Weißt du, Luna}, sagte Harry, \enquote{ich mag dich.}

\enquote{Ich mag dich auch, Harry}, antwortete Luna, und ihre Hand berührte noch immer die seine.

Plötzlich drehte sie sich zu Harry, legte ihre Arme um seine Taille und klammerte sich an ihn, mit ihrem Kopf auf seiner Schulter. Harry konnte nicht anders: Er hielt sie fest. Einige Minuten standen sie so da, von Thestralen umringt, und hielten sich fest. Mit einer unruhigen Stimme begann sie zu erzählen.

\enquote{Ich hab das bisher noch nie jemandem erzählt, Harry. Aber bei diesem Versuch, bei dem meine Mutter starb, habe ich auch meine sechs Monate alte Schwester verloren.} Tränen kullerten ihre Augen herunter. Er hielt sie weiter fest. \enquote{Es schmerzt, immer noch.} Nach einer Weile wischte sie sich ihre Tränen weg. \enquote{Weißt du, Harry \gst ich träume noch heute davon. Es geht mir immer noch nach. Immer noch, nach all den Jahren.} Dann löste sie sich etwas von ihm und ließ ihn los. Beide standen sich nun gegenüber. Noch immer waren ihre Augen feucht und Harry nahm ein Taschentuch und trocknete sie. \enquote{Danke Harry}, sprach Luna.

\enquote{Keine Ursache}, antwortete Harry.

Gerade als er sich wieder den Thestralen zuwenden wollte, nahm Luna seine Hand und zog ihn vorsichtig zu sich. Sie legte ihren Kopf auf seine Schulter und betrachtete die Thestrale. Nach einigen Minuten gab sie ihm einen schamhaft flüchtigen Kuss auf die Backe.

\enquote{Ich gehe jetzt besser zurück}, sagte sie in ihrer verträumten Art. \enquote{Kommst du mit?}

Harry stand da und war vollkommen durcheinander. \gedanke{Will Luna mit mir anbandeln?}, fragte er sich. Doch er verwarf den Gedanken wieder. Einer der Thestrale stupste ihn leicht mit der Schnauze an, in der Hoffnung, das, was in Harrys Tasche war, zu bekommen. Denn Harry hatte sich angewöhnt, immer, wenn er zu den Thestralen ging, ihnen etwas mitzubringen. Er ging in die Hocke und öffnete seine Tasche, die auf dem Boden stand. Er nahm ein kleines papierumschlagenes Paket heraus und wickelte ein paar Brocken rohes Fleisch aus, sowie etwas Gemüse aus seiner Tasche. Er verteilte es auf dem Boden \gst das heißt, er warf es etwa zwei bis drei Meter entfernt von sich auf den Boden, da er die fressenden Thestrale nicht in seiner unmittelbaren Umgebung haben wollte. Das letzte Stückchen behielt er jedoch in der Hand und reichte es dem Thestralweibchen, dass noch immer vor ihm stand. Sie nahm es ihm mit dem Maul aus der Hand und schlang es hinunter. Nachdem sie damit fertig war, leckte sie ihm die nach Fleisch schmeckenden Hände ab. Es war ein ungewöhnliches Gefühl. Er hatte zwar schon mehrere Zungen verschiedener Tiere gespürt, aber keine war vergleichbar mit denen der Thestrale. Es war so, als ob ein pelziger Belag die Zunge umhüllte, obwohl man nichts davon erkennen konnte. Dann drehte er sich um und folgte Luna, die während des gesamten Weges zurück Harrys Hand hielt. \gedanke{Luna will tatsächlich mit mir anbandeln}, dachte Harry und schmunzelte. Noch vor einem Jahr hätte er es nicht für möglich gehalten; als er sie zum ersten Mal in den Schulkutschen traf, die, wie er inzwischen weiß, von Thestralen gezogen werden. Sie hatte damals eine Zeitschrift in der Hand, die sie verkehrt herum las, erinnerte sich Harry.




\begin{kommentar}
Im Lied des Hutes gleich zu Beginn des Kapitels ist ein erster Hinweis darauf, dass das Schloss bereits besteht, als die vier Schulgründer es zur Schule umbauen. Nur sagt der Hut in dem Punkt mit dem Traum nicht ganz die Wahrheit. Aber er ist nur ein Hut und weiß es halt nicht besser.
\end{kommentar}

\begin{kommentar}
Später hat Harry einen seltsamen Traum. Er steht in Bellatrix’ Verlies in Gringotts. Dort entdeckt er den Pokal von Hufflepuff. Ein weiterer Hinweis auf die Magie des Amuletts, die bereits wirkt. Nur weiß Harry da noch nicht, dass der Pokal ein Horkrux ist.
\end{kommentar}

\begin{kommentar}
Als Dumbledore seine erste Rede hält, kam eine blaue Welle durch das Schloss. Leider weiß ich nicht mehr, was ich mir als Ursache dabei gedacht habe. Es war etwas mit Frederick Elber, aber den ursprünglichen Grund habe ich nach der langen Zeit vergessen. Bedauerlicherweise.
\end{kommentar}

\begin{kommentar}
Da aber Elber erst einige Tage später zu unterrichten beginnen kann, muss er wohl bei einem Versuch verletzt worden sein und als Geist (oder ähnliches) durch die Decke gefallen sein.
\end{kommentar}

\begin{kommentar}
In einer windigen Nacht darauf schleicht sich eine Gestalt durch das Schlosstor und in das Schloss. Sie teilt dem Schloss mit, dass es ganz normal reagieren soll. Es ist Frederick Elber, der nicht will, dass man seine Beziehung zum Schloss erfährt, denn er hatte es damals gebaut.
\end{kommentar}

\begin{kommentar}
Kurz danach, nach einem Trenner, kommt ein Satz, der später als kleine Anspielung nochmal auftaucht: >Harry stand mal wieder im Wald.<
\end{kommentar}

\begin{kommentar}
Nach seiner ersten Nacht mit Luna im Gemeinschaftsraum der Paare trifft Harry zum ersten Mal auf Frederick Elber. Dieser erklärt ihm, was es mit dem Patronus-Zauber auf sich hat und dass er manche Gefühle und Erinnerungen für sich behalten sollte.
\end{kommentar}
