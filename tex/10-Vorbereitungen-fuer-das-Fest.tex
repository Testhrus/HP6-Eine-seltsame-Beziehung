\chapter{Vorbereitungen für das Fest}


Wieder einmal war Samstag und Harry hatte frei. Seine Hausaufgaben waren größtenteils erledigt und das Quid"-ditch-Match gegen Ravenclaw hatten sie knapp verloren, mit nur zehn Punkten Rückstand. Er war nicht schnell genug gewesen, um den Schnatz zu greifen, denn Sekunden vorher hatte Cho ihn gegriffen.

\begin{rueckblick}
\enquote{Und weitere Zehn Punkte für Ravenclaw. Was ist heute nur mit den Gryffindors los}, verkündete Stan Lindsay. \enquote{Dreiiissig für Ravenclaw NULL für Gryffindor. Scheinbar haben sie heute keinen guten Tag erwischt.} Harry besah sich das Spiel von oben und warf Cho immer wieder ein mattes Lächeln zu. Doch plötzlich kam eine Wendung in das Spiel. Knappe zwanzig Minuten später lag Gryffindor mit vierzig Punkten im Vorsprung. Es stand jetzt Dreißig zu Siebzig. \enquote{Na also. Es scheint, als seien die Gryffindors endlich aufgewacht}, brüllte Stan, nachdem sie punktgleich mit Ravenclaw waren. Doch als sie 140 Punkte Vorsprung hatten, passierte es. Cho hielt den Schnatz triumphierend in den Händen. Das war ihr nur ein einziges Mal gelungen, seit sie gegen Harry spielte.
\end{rueckblick}

Damals hatten sie das Spiel trotzdem verloren.

Er war heute nicht gut in Form. Er hatte sich den Schnatz vor der Nase vom Ravenclaw-Sucher wegschnappen lassen. Cho war in ihrem letzten Jahr und strahlte nur so, als sie Harry den Schnatz vor seiner Nase wegschnappte. Gryffindors Quidditch-Mannschaft war deswegen zwar nicht besonders niedergeschlagen, da sie mehr Hauspunkte als Ravenclaw hatten, aber sie waren auch nicht gerade gut gelaunt da sie Harry, wenn auch indirekt, die Schuld gaben. Neben einigem unzufriedenem Murmeln kam kein böses Wort über die Lippen seiner Mannschaftskameraden. Sie befanden sich im Gruppenraum der Gryf\-fin\-dor-Quidditch-Mann\-schaft, um das aktuelle Spiel Revue passieren zu lassen. Harry fiel wieder das Spiel ein, das er bei Arabella gespielt hatte. Ob sie es ihm wohl ausleihen würde? Das könnte seiner Mannschaft helfen, die nächsten Züge zu planen.

\gedanke{Ich glaube, ich werde Arabella mal einen Brief schreiben}, dachte sich Harry, als er von seinen Team-Kollegen aus seiner Träumerei wieder zurück in die Realität geholt wurde.

\enquote{Harry, bist du bei der Sache?}, fragte ihn Katie. \enquote{Du wirktest etwas abwesend.}

\enquote{Ach nichts}, meint er. \enquote{Ich hatte nur so eine Idee wie wir unser Spiel verbessern können. Aber darüber muss ich erst mit jemanden Reden, ob ich es auch bekomme.}

\enquote{Was?}, fragten seine Teamkollegen.

\enquote{Ich sage es euch, wenn ich es habe, falls ich es bekomme. Ich möchte euch nicht unnötig enttäuschen. Nicht nach dem heutigen Spiel.}

Nach der Abschlussbesprechung war es auch schon wieder Zeit für das Mittagessen, und so ging Harry, nachdem er im Gruppenraum der Quidditch-Mannschaft geduscht und sich umgezogen hatte, mit dem Rest der Mannschaft zurück zum Schloss. Als er in die Große Halle zum Essen ging, standen Professor Dumbledore und Professor McGonagall da und unterhielten sich mit Professor Elber.

\enquote{Dann ist es abgemacht, sie kümmern sich dieses Jahr darum}, sagte Professor Dumbledore und ging mit Professor McGonagall in die Große Halle. Professor Elber folgte ihnen und setzte sich auf seinen Platz. Harry wunderte sich, was sie wohl zu bereden gehabt hatten. Nach dem Mittagessen schaute Harry in der Bibliothek vorbei, da er noch seine Hausaufgabe bei Professor Sprout machen musste. Harry konnte nicht erkennen, was Professor Elber da las, aber als er sich näherte, um ein anderes Buch aus dem Regal zu holen, schlug Professor Elber sein Buch zu und wartete bis Harry wieder wegging. Harry nahm sein Buch und setzte sich an einen Tisch auf der anderen Seite, an dem schon Ginny und Parvati saßen. Er flüsterte ihnen zu. \enquote{Professor Elber hat wohl irgendetwas vor. Er schlug sein Buch zu, als ich ein Buch aus dem Regal in seiner Nähe geholt habe.} Beide wunderten sich, sagten aber nichts.

Am nächsten Tag sah Harry wie Professor Elber mit diesem Buch durch die Gänge lief und sich im Schloss umschaute. Gerade so, als ob er etwas suchen würde. Er wollte schon nachfragen, als er von Hermine aufgehalten wurde die ihn fragte: \enquote{Hast du schon ein Weihnachtsgeschenk für Ginny?}

Harry verneinte und fragte Hermine: \enquote{Hast du irgendeine Idee? Bleiben sie und Ron überhaupt über Weihnachten?}

\enquote{Ja}, antwortete Hermine. \enquote{Ich bleibe auch. Ich habe mich schon eingetragen. Und du? Fährst du Heim?}

\enquote{Natürlich nicht. Glaubst du, ich will Weihnachten bei den Dursleys verbringen?}

\enquote{Na ja, ich frage nur, weil du dich nicht eingetragen hast.} Harry erschrak und rannte sofort zur Großen Halle, wo die Anschläge hingen. Er trug sich gerade ein, als Hermine um die Ecke kam. Sie war noch vollkommen außer Puste und meinte: \enquote{So eilig musst du es auch nicht haben.}

\enquote{Du kennst meinen Onkel und meine Tante nicht. Wenn ich mich nicht rechtzeitig eintrage, dann fahre ich zurück. Und glaube mir, die wären nicht begeistert.}

Hermine grinste und legte eine Hand auf seine Schulter.

Kurz darauf machte er sich auf den Weg zu Professor Elbers Büro, denn er wollte wissen, was es mit seinem Tarnumhang auf sich hat. Er klopfte, doch es tat sich nichts. Er wollte sich gerade umdrehen und gehen, als die Tür aufging. Entgegen besserem Wissen trat er vorsichtig ein und sah sich um. Er stand noch so, dass die Tür nicht zufallen konnte und lies seinen Blick schweifen. Im Raum waren wenig persönliche Sachen zu finden. Auf einem Sideboard stand ein Bild von Elfen.

\gedanke{Wahrscheinlich seine Elfen, oder die seiner Eltern, die ihn mit aufgezogen haben}, dachte er sich. \gedanke{Habe ich das Bild das letzte Mal übersehen oder steht es erst jetzt da?}

Sein Blick wanderte weiter durch den Raum und blieb über dem Schreibtisch hängen. Dort hingen noch immer die vier Bilder der Gründer von Hogwarts. Rowena Ravenclaw, Salazar Slytherin, Godric Gryffindor und Helga Hufflepuff. Auf der anderen Seite des Raumes, kurz bevor die Tür anfing, sah er ein Bild einer jungen hübschen Frau.

\gedanke{Seine Schwester? Tante? Frau?}, fragte sich Harry.

Plötzlich spürte er eine Präsenz hinter sich. Er war ruhig genug, nicht zu erschrecken. \gedanke{Snape oder Filch können es nicht sein, die hätten schon längst los gebrüllt. Ron und Hermine hätten gefragt, was ich hier mache. \gst Dumbledore? Elber?} Harry dachte nach.

Er versuchte sich zu konzentrieren und hinter sich zu blicken. Als er anfing das verschwommene Bild klarer wahrzunehmen, dämmerte es ihm, dass es nur Elber sein konnte. Ein anderer hätte sich schon längst bemerkbar gemacht. Er öffnete seine Augen und drehte sich während der Ansprache um.

\enquote{Guten Tag, Professor Elber.}

Doch hinter ihm war niemand.

\enquote{Hinter Ihnen}, hörte er.

Erschrocken drehte er sich erneut um. \enquote{Professor? Wie kommen Sie\abs?}

\enquote{Setzen Sie sich.}

Harry setzte sich auf den ihm angebotenen Stuhl und wartete.

\enquote{Wissen Sie, das möchte ich mit einigen von ihnen durchgehen, um am Ende des Schuljahres eine kleine Demonstration zu vorzuführen. Netter Effekt, oder? Man zeigt sein Abbild an anderer Stelle. Das verwirrt den Gegner.} Dann verschwand das Bild und Harry hörte nun wieder hinter sich eine Stimme. \enquote{Sie haben mich richtig erkannt. Ich stehe aber immer noch hinter Ihnen.}

Harry drehte sich auf dem Stuhl um und sah zur Tür. Dort stand Professor Elber und betrat gerade den Raum, um sich auf seinen Platz hinter dem Schreibtisch zu setzen.

\enquote{Sie wollen was von mir?}, fragte er nach.

\enquote{Ja Professor. Sie sagten mir\abs wollten mir etwas über meinen Tarnumhang erzählen.}

\enquote{Haben Sie ihn dabei?}

\enquote{Nein, leider nicht. Aber ich kann ihn holen.}

\enquote{Sie wissen, wie er sich anfühlt?}

\enquote{Ja, sehr genau sogar.} \gedanke{Ups.}

\enquote{Gut. Im linken Sideboard, von Ihnen aus gesehen, in der obersten Schublade ist ein anderer. Schauen Sie ihn sich an, fühlen Sie ihn und legen Sie ihn dann wieder zurück.}

Harry nickt und stand auf. Er öffnete die Schublade und sah den Tarnumhang. Er nahm ihn heraus und fuhr über den Stoff. Dieser fühlte sich anders an.

\enquote{Darunter liegt ein Buch. Holen Sie es heraus, bevor Sie den Umhang hineinlegen und lesen Sie eine der Geschichten.}

Harry legte nach einer Weile den Umhang in die Schublade und zog das Buch darunter hervor. Dann schloss er die Schublade und setzte sich wieder auf den Stuhl. Professor Elber korrigierte unterdessen Schulaufgaben der Drittklässler. Harry erkannte es an dem Irrwicht, über den sie eine Abhandlung schreiben mussten. Einige Zettel lagen wild verteilt auf dem Tisch. Auf einigen war nur wenig Rot zu sehen. Bei den meisten waren nur ein paar Anmerkungen zu finden und zwei hatten sogar mehr rot als schwarz auf ihrem Pergament, vermutete Harry. Er war respektvoll genug nicht zu lesen, was darauf stand.

Er schlug gerade das Buch auf, als es selbstständig weiter blätterte und eine bestimmte Seite aufschlug.

\enquote{Das werden Sie auch noch lernen}, sagte sein Professor zwischen zwei Sätzen, die er schrieb, ohne aufzusehen.

Harry begann zu lesen.

\begin{buch}
Die Geschichte der drei Brüder

Es waren einmal drei Brüder, die in der Abenddämmerung spazieren gingen. Sie kamen an einen Fluss, der\abs
\end{buch}

Als er die Geschichte fertig hatte, schlug sich das Buch automatisch zu und flog zurück an seinen Platz. Harry dachte darüber nach.

\enquote{Was halten Sie davon?}, fragte sein Professor nach einer Weile.

\enquote{Sie wollen mir damit etwas sagen. Aber ich weiß noch nicht genau was. Dieser Umhang des Todes\abs}

\enquote{Vom Tod}, korrigierte ihn sein Lehrer.

\enquote{\aabs vom Tod. Meinen Sie, das könnte mein Umhang sein?}

\enquote{Die Idee ist gut, aber die Richtung ist falsch. Was ist mit den anderen Sachen?}

\enquote{Den Stab hätte man sicher schon gefunden. Ich meine, einen unbesiegbaren Zauberstab?}

\enquote{Warum hätte man ihn finden sollen?}

\enquote{Na ja, wenn sich jemand duelliert und ständig gewinnt?}

\enquote{Und wenn sich die Person nicht duelliert? Wie findet man ihn dann? Wenn die Person schlau genug ist\abs}, er sah auf und Harry an, \enquote{\aabs und nicht durch die Welt reist und überall herumerzählt: \inner{Ich habe den unbesiegbaren Zauberstab, komm und fordere mich heraus.} Was ist dann?} Das hatte Harry nicht bedacht. \enquote{Was ist mit dem Stein?}

\enquote{Der Stein, der lebende zurückholt? Der Stein der Auferstehung?}

\enquote{Holt er sie zurück?  Sind die Geister in Hogwarts alle durch den Stein zurückgeholt worden? Und was ist mit Geistern anders wo auf der Welt?}

Harry dachte lange nach. Professor Elber hatte ihm Rätsel aufgegeben. Dabei wollte er doch nur etwas über seinen Umhang erfahren. Er bemerkte nicht, wie Professor Elber etwas auf zwei Zettel schrieb und diese in die Luft hielt. Dann ließ er sie los, dass sie dort schwebten und verließ sein Büro. Minuten später tauchte Harry aus seiner Denkstarre wieder auf und sah sich irritiert um, da er seinen Professor nicht mehr fand. Dann entdeckte er die beiden Zettel. Es war eine Ausgangserlaubnis, die ihn frei von Ärger halten sollte, falls er auf seinem Rückweg zum Gemeinschaftsraum erwischt werden sollte. Der andere war eine Anweisung an Madame Pince, dass sie Harry ein bestimmtes Buch aushändigen möge. Den Titel konnte er nicht lesen. Er schien kodiert zu sein.

Er nahm beide Zettel, steckte sie in seine Taschen und macht sich auf den Weg Richtung Gemeinschaftsraum. Unterwegs traf er Professor Sprout, der er den Zettel zeigte und die ihn den restlichen Weg begleitete. Sie unterhielten sich über alles Mögliche und Harry redete sich einfach das, was ihn bedrückte, von der Seele.

\enquote{Ja, manchmal ist er schon komisch}, pflichtete ihm Professor Sprout bei. \enquote{Er ist halt doch ein komischer Kauz. Aber nicht immer. Neulich hat er doch im Lehrerzimmer gesagt, als ihn jemand gefragt hat, was er so in seiner Freizeit mache: \inner{Wissen sie, manchmal mache ich einfach, dass Luft schlecht riecht. Es ist ungeheuer entspannend. Zwar muss ich in seltenen Fällen danach lüften, aber das macht nichts. Es gibt ja noch andere Räume.} Ein anderes Mal hat er einfach seine Hobbys aufgezählt, oder anerkennend genickt, wenn jemand über seines erzählt hat und gemeint: \enquote{Ja, das mache ich ab und an. Es ist richtig befriedigend.} Er meinte damit Gartenarbeiten. \gst Oh, wir sind schon da. Dann guten Schlaf Mister Potter.}

\enquote{Ihnen auch Professor Sprout.} Harry gab ihr die Hand und schüttelte sie. Professor Sprout hatte körperliche Nähe gern. Die Hände schütteln, einen Knuff auf die Schulter oder eine Umarmung. Das ließ sie wieder aufleben.

Harry drehte sich einmal um, um sich zu vergewissern, dass sie allein seien. \enquote{Schon schön, das Schloss bei Nacht.}

\enquote{Ja, das ist es.}

Dann umarmte er sie, sagte der Dame das Passwort und stieg durch das Loch in der Wand. Jetzt musste er sich noch weiter gedulden, bis er endlich etwas über seinen Umhang erfuhr.

\trenn

Keiner der beiden hörte die jungen Schritte über den steinernen Boden der Krankenstation laufen.

\enquote{Frederick, weshalb ich Sie habe kommen lassen \gst Sie kennen Philip Allman? Hufflepuff! Narbiges Gesicht und nur ein Auge.}

\enquote{Ja Poppy, kenne ich}, antwortete Elber.

\enquote{Ich habe mir gestern Abend etwas durch den Kopf gehen lassen. Sie haben doch Katie geholfen ihre Hand wiederzubekommen\abs}

Elber hob die Hand und unterbrach sie. \enquote{Ich weiß, worauf Sie hinaus wollen. Ich habe deshalb nichts gesagt, weil \gst na ja, schwarze Magie \gst und mir ist es zu schwierig.}

Plötzlich klopfte es an der Tür.

\enquote{Herein}, sagte Madame Pomfrey.

Die Tür öffnete sich und Philip Allman trat ein.

\enquote{Madame Pomfrey? Ich habe eine\abs Oh entschuldigen Sie, ich habe nicht mit Besuch gerechnet. Ich gehe wieder.}

\enquote{Bleiben Sie ruhig hier und setzen Sie sich.} Madame Pomfrey bot Philip einen Platz an.

\enquote{Danke}, antworte der Junge und setzte sich auf einen Stuhl.

\enquote{Ich bat gerade Professor Elber um Rat, wegen ihres Unfalls, aber er sagte mir gerade, dass da nichts\abs}

\enquote{Poppy, das habe ich nicht.}

\enquote{Aber, Sie haben doch gesagt, dass es\abs}

\enquote{Mir ist es zu schwierig, da ich nicht die notwendigen medizinischen Kenntnisse habe. Die Narben wären nicht das Problem. Aber die Augenhöhle auszubilden. Und wenn das mit Magie nicht funktioniert, dann muss das ein plastischer Chirurg der Muggel machen. Ganz zu schweigen von der Tatsache, dass man auf jeden Fall ein Auge nachwachsen lassen muss. Und zwar in einer Nährlösung mit Unterstützung durch Magie, da die Muggel das nicht können. Aber ein Auge durch Magie nachwachsen zu lassen und es richtig mit dem Gehirn, bzw. einem noch vorhandenen Sehnerv zu verbinden und mit dem Gehirn zu verschalten, übersteigt mein Können bei weitem. \gst Manchmal habe ich das Gefühl, ihr traut mir alles zu. Gut, meine Möglichkeiten mögen vielfältiger sein, als die jedes anderen, aber auch Magie hat ihre Grenzen. Und ohne Unterstützung durch Muggel ist da nichts zu machen.}

\enquote{Dann ist Hilfe möglich?}, fragte Philip zaghaft nach.

Professor Elber kratzte sich kurz am Kinn und antwortete dann: \enquote{Sagen wir mal so. Madame Pomfrey muss erst einmal den Rest ihrer Augenhöhle untersuchen. Ich kann ihr die passenden Bücher geben. Ich selber trau mir das nicht zu. Zu viel könnte bei meinem Eingreifen so nah am Gehirn schiefgehen \gst wenn das erledigt ist, dann kann man die Augenhöhle der intakten angleichen \gst Das wird nicht angenehm; es wird beißen und jucken und sie dürfen nicht kratzen. Das dauert ein paar Tage, bis das Wachstum der Zellen angeregt und in die richtigen Bahnen gelenkt wurde. \gst Dann wird sie Madame Pomfrey zu einem Augenspezialisten der Muggel \gst es gibt ein paar Squibs \gst bringen, der ihnen eine Gewebeprobe entnehmen wird, damit man daraus in einer Nährlösung ein neues Auge mit magischer Unterstützung heranzüchten kann. Beim Augenarzt wird ihnen dann ein Verband auf das Auge gelegt werden und sie werden ihn etwa drei Tage lang auf dem Auge behalten müssen.}

\enquote{Soviel zum Thema: \enquote{Ich habe davon keine Ahnung.}}

\enquote{Ich habe mich etwas eingelesen und mit einem Arzt gesprochen, das gebe ich zu, aber nach deiner Reaktion bei Katie, war ich noch mehr entmutigt, dir etwas in der Richtung vorzuschlagen.}

\enquote{Ich sehe jetzt nichts, was da mit schwarzer Magie zu tun hat.}

\enquote{Das kann ich dir sagen, Poppy. Das Auge, oder ein anderes Körperteil, mithilfe von Magie heranzuzüchten, hat große Ähnlichkeiten mit der Erschaffung eines Inferius. Wer immer das tut, hat die Kontrolle über das Teil und somit dann über ein Auge Philips hier. Durch speziellen Einsatz von Zaubern und der festen Absicht, die Kontrolle in die richtigen Bahnen zu lenken, nämlich die Kontrolle des Eigentümers des Auges, verliert das ganze seine schwarze Seite. Aber das sehen nicht alle so.} Dann drehte er sich zu Philip und sagte: \enquote{Überlegen Sie es sich. Besprechen Sie das Ganze mit ihren Eltern. Schreiben Sie ihnen und erklären es ihnen, oder noch besser, wenn der Schulleiter zustimmt, besuchen Sie sie über das Wochenende. Und Sie, Poppy, denken bitte darüber nach, denn ich werde Ihnen nur assistieren können.}

Beide schluckten schwer nach diesen Ausführungen und hingen schweigend und sich gegenseitig anblickend ihren Gedanken hinterher.

\enquote{Würden sie es machen?}, fragte Philip vorsichtig.

Madame Pomfrey sah ihn immer noch an und nickte schließlich. Glücklich erhob sich der Junge und trat nachdenklich aus dem Raum.

\enquote{Was glauben Sie, wird er es machen}, fragte Madame Pomfrey.

\enquote{Ich weiß es nicht. Ich habe von seiner Familie gehört, dass sie nicht gerade begeistert war, dass er nach Hufflepuff gekommen ist. Und soweit ich weiß, steht kaum jemand den schwarzen Künsten offen gegenüber. Ich habe Tendenzen zu Rassismus in seiner Familie über Gerüchte gehört. Und einige Zweige streben nach reinem Blut. Ich weiß nicht, was sie machen werden. Ich hoffe, seine Eltern sind stark genug. Sollten sie ablehnen, dann können wir immer noch einige Vorbereitungen treffen, die ich als medizinisch notwendig deklarieren würde. Das müssen sie natürlich verantworten und unterstützen. Und wenn er volljährig ist, oder ein entsprechendes Reifezeugnis abgelegt hat, dann kann er seine Eltern überstimmen, was ihn aber seine Familie kosten könnte. Kurzum, ich weiß nicht was passiert, habe keine Ahnung und warte einfach nur ab. Ein Gespräch mit den Eltern wird trotz allem kommen, schätze ich.}

Keiner der beiden merkte, dass der Junge noch draußen vor der Tür stand und sich den Dialog anhörte, bevor er leise aus dem Krankenflügel verschwand, um in sein Zimmer zu gehen und erst einmal zu weinen. Ob aus Trauer oder vor Glück konnte er selbst nicht mehr genau sagen, als er am anderen Tag erwachte. Er schrieb seinen Eltern einen langen Brief und erklärte ihnen alles, was er noch in Erinnerung hatte, und wartete auf deren Antwort.

\trenn

Nachdem Harry wieder einmal seine Hausaufgaben gemacht hatte, nahm er sich die Zeit, Arabella einen Brief zu schreiben.

\begin{brief}
Liebe Arabella,

wie du sicher weißt, bin ich immer noch in der Quidditch-Mannschaft. Zurzeit haben wir aber große Probleme mit unserem Training und auch unsere Taktik ist zur Zeit der, der anderen unterlegen. Ich möchte dich daher bitten mir dein Mini-Quidditch-Spiel auszuleihen, damit wir besser trainieren können.
\end{brief}

Jetzt fiel ihm noch ein, dass er Arabella etwas schenken könnte, da in ein paar Tagen die Weihnachtszeit begann. Er überlegte eine Weile, bis er wieder seine Feder in sein Tintenfass stecken wollte, um seinen Brief abzuschließen, da ihm nichts mehr einfiel. Er war kurz vor seinem Tintenfass, als er plötzlich innehielt und sein Tintenfass anstarrte. \gedanke{Ich habe ja noch meine Spezialtinte, die beim Schreiben und nach dem Trocknen ständig die Farbe ändert.} Er stand auf und ging nach oben, um das Tintenfässchen zu suchen. Nachdem er es gefunden hatte, ging er wieder nach unten und setzte sich an seinen Platz im Gemeinschaftsraum, um seinen Brief weiterzuschreiben.

\begin{brief}
Ich habe dir außerdem noch etwas eingepackt, das du aber erst an Weihnachten öffnen darfst.
\signumspace
Liebe Grüße\\
Harry
\end{brief}

Nachdem er seinen Brief fertig hatte, legte er ihn zum Trocknen beiseite und packte währenddessen sein Geschenk an Arabella ein. Danach rollte er seinen Brief zusammen und band ihn an das Päckchen. Er verließ den Gemeinschaftssaal und machte sich auf den Weg zur Eulerei. Kurz darauf kam er wieder zurück und lief in sein Zimmer. Er hatte den für Hedwig obligatorischen Eulenkeks vergessen. Oben in der Eulerei traf er auf Donan.

\enquote{Hi Donan.}

\enquote{Hi Harry}, antwortete Donan. \enquote{Hast du schon bemerkt, dass zwischen Hermine und Ron aus deinem Haus etwas läuft?}

Harry schaute ihn nur an. \enquote{Bist du dir sicher? Es deutet zwar einiges darauf hin, aber falls da was laufen sollte, verstecken sie es ziemlich gut. Ich versuche die nächsten Tage mal was herauszubekommen.}

\enquote{Gut}, antwortete Donan und verließ die Eulerei.

\enquote{Hallo Hedwig}, begrüßte Harry seine Eule, die er in den letzten Wochen weniger besuchte, da er wenig Zeit hatte. \enquote{Ich habe ein Päckchen für Arabella, bring es ihr bitte.} Er griff in seine Tasche und gab Hedwig den Eulenkeks. Danach streichelte er ihr Gefieder und wünschte ihr einen guten Flug. \enquote{Du kannst dich bei Arabella eine Weile ausruhen und aufwärmen, falls du es möchtest. Es ist zurzeit sehr kalt draußen. Ich werde wohl die nächsten Tage keine Briefe versenden.}

Harry verließ die Eulerei und machte sich auf den Weg zurück in das Schloss. Als er so seinen Blick schweifen ließ, sah er Professor Elber, wie er sich mit Hagrid neben einer Reihe gefällter Bäume unterhielt. \gedanke{Aha, es geht also los mit der Schmückerei}, dachte Harry. \gedanke{Wahrscheinlich kümmert er sich um die weihnachtliche Gestaltung.}

Dann machte er sich auf zur Bibliothek. Dort angekommen suchte er erst einmal Madame Pince, da sie nicht an ihrem Platz war. In einem der hinteren Gänge wurde er fündig.

\enquote{Madame Pince? Ich habe hier einen Anforderungsschein.}

\enquote{Lassen Sie mal sehen.} Sie nahm den Zettel an sich, schaute kurz und fuhr dann mit ihrem Zauberstab darüber. Stöhnend meinte sie: \enquote{Was der immer für ausgefallene Wünsche für seine Schüler hat. \gst Folgen Sie mir, Mister Potter.} Sie lief durch die Gänge Richtung verbotener Abteilung. Kurz vorher bog sie ab. Sie sah die Regale entlang und zog dann ein Buch heraus. \enquote{Dieses hier?}, fragte sie. Als Harry nichts sagte, schlug sie die Augendeckel nieder und atmete einmal ein und wieder aus. \enquote{Sie wissen nicht, was Sie lesen sollen, habe ich recht?}, fragte sie mit geschlossenen Augen.

\enquote{Ja, ich habe schon geschaut, konnte den Text aber nicht entziffern.}

Sie öffnete wieder ihre Augen. \enquote{Und das, wo sie gerade unterrichtet werden, hinter Tarnungen zu sehen.} Dabei hob sie eine Augenbraue. Dann fing sie an, zu ihrem Platz zu laufen, um das Buch auf Harry einzutragen.

\enquote{Woher wissen sie davon?}

\enquote{Wir haben uns unterhalten. Er suchte gestern Abend noch ein Buch. Dabei sind wir ins Gespräch gekommen.}

\enquote{Sie wussten also, dass ich komme und das Buch hole?}

\enquote{Ich ahnte es, als er mir sagte, dass Sie etwas über ihre Vergangenheit erfahren wollen.}

\enquote{Über meine Vergangenheit? Moment mal. Ich will etwas über meinen\abs} Harry stockte. \enquote{Ja, was eigentlich?}

In der Zwischenzeit hatte sie die Ausleihkarte ausgefüllt und gab nun Harry das Buch in die Hand.

\enquote{Danke}, sagte Harry und verließ die Bibliothek.

Zurück in seinem Zimmer überfiel ihn eine Müdigkeit, die ihn zwang ins Bett zu gehen. \gedanke{Scheinbar muss ich mich noch schonen. Das hat mich gestern schon geschlaucht.} Er krabbelte ins Bett und schloss die Augen.

Als er wegdämmerte, zogen sich seine Vorhänge wie von selbst zu und er glitt ins Reich der Träume.

\begin{traum}
Zusammen mit seinen Brüdern Ron und Neville ging er in der Abenddämmerung spazieren. Noch immer trauerten sie um ihre Eltern, die sie nie kennengelernt hatten. Nach einer Weile kamen sie an eine Flussströmung. Um durchzuwaten oder zu schwimmen, war er zu tief und zu schnell. Also zauberten sie sich eine Brücke aus Stämmen, Ästen und Zweigen hervor. Nacheinander gingen sie über die Brücke.

Kaum hatten sie zwei Drittel der Strecke geschafft, als eine Gestalt vor ihnen erschien und stellte sich als der Tod vor. Da sie es geschafft hatten, ihm zu entkommen, denn für gewöhnlich ertranken die Personen im Fluss und er konnte sie sich einverleiben, stellte er jedem von ihnen einen Preis in Aussicht.

Die drei liefen noch über die Brücke auf den sicheren Grund und konnten dem Tod ihre Wünsche vorschlagen. Neville wünschte sich einen Zauberstab, der von keinem andern geschlagen werden konnte. So wollte er sich am Tod seiner Eltern rächen. Also formte der Tod einen Zauberstab aus einem nahe stehenden Wacholderbusch.

Dann kam Ron an die Reihe. Er wünschte sich nur seine Eltern zurück, also verlangte er vom Tod die Möglichkeit, Menschen zurückholen zu können. Der Tod nahm einen Stein vom Flussufer und gab ihn ihm.

Dann kam Harry dran. Er wünschte sich die Möglichkeit, sich vor allem und jedem verbergen zu können. Also gab ihm der Tod ein Stück seines eigenen Umhanges.

Glücklich, dem Tod etwas abgeluchst zu haben, gingen sie weiter.
\end{traum}

Harry schlug seine Augen auf und sah an den Baldachin seines Bettes. Die Vorhänge gingen wieder auf, was Harry aber nicht registrierte. Er wunderte sich über seinen eigenen verzogenen Traum und schüttelte den Kopf. Wie konnte er nur so einen Mist träumen, wunderte er sich. Wieder zog die Geschichte durch seinen Kopf. Die drei Gegenstände \gst Heiligtümer des Todes \gst sah er nun klar vor seinem geistigen Auge. Ein eckiger Stein, seinen Umhang und einen Zauberstab mit größer werdenden kugelförmigen Ausbildungen.

\gedanke{Den Stab habe ich doch schon einmal gesehen}, durchfuhr es Harry. Er überlegte fieberhaft, wo. Immer wieder sah er eine Hand, die den Stab hielt. Um sich abzulenken, durchstöberte er seinen Koffer und nahm seinen Tarnumhang in die Hand. \gedanke{Ja, das ist er. Aber ist er wirklich vom Tod?} Er setzte sich wieder auf sein Bett und legte seine Brille auf seinem Nachtschränkchen ab. Dann fuhr er mit beiden Händen über sein Gesicht. Als er seine Brille wieder aufsetzen wollte, bemerkte er das Buch, das er gerade eben aus der Bibliothek geholt hatte.

Er nahm es mit nach unten und begann es zu lesen. Schon nach wenigen Minuten hatte er zwei Mitleser. Ginny und Hermine schauten ihm zu und fragten ihn, was er denn lese.

\enquote{Etwas über meine Vergangenheit, schätze ich. Ich bin noch nicht sicher. Ich wollte eigentlich etwas anderes erfahren.}

\enquote{Was?}, fragte Ginny nach.

\enquote{Etwas über\abs} Durfte er darüber überhaupt sprechen? Er dachte kurz nach. \enquote{\aabs meinen Tarnumhang}, machte er leise weiter.

\enquote{Den hast du doch von deinem Vater geerbt.}

\enquote{Ja schon, aber woher hat er ihn? Er sieht ziemlich alt aus. Und ehrlich gesagt denke ich, dass er es ist.} Er widmete sich wieder seinem Buch.

Teilweise war es eine Biografie der Familie Peverell, teilweise eine Analyse ihres Schaffens. Es waren sehr begabte Magier und Erfinder. Einer von ihnen war ein Zauberstabmacher. Ein anderer war Auror und musste sich immer wieder Tarnen. Ein dritter war ein Träumer. Er redete immer wieder davon, mit den Geistern der Vergangenheit sprechen zu können.

\gedanke{Das war es doch}, dachte er sich. \gedanke{Die drei Brüder aus dem Märchen könnten die Peverells sein. Und diese haben die Gegenstände erschaffen. \gst Wenn der Tarnumhang immer innerhalb der Familie weitergegeben wurde, dann wäre ich ein Nachfahre\abs} Harry lehnte sich zurück. Er blätterte nun durch das Buch und las gelegentlich wieder einen Abschnitt. Er schwankte zwischen Sicherheit und Unsicherheit. Er müsste wohl noch einmal, oder besser gesagt endlich einmal, mit Professor Elber sprechen.

\enquote{Ich muss noch einmal kurz weg.} Er stand auf und ging zum Porträt. Er blieb stehen und drehte sich erneut um. \enquote{Wenn ich wieder komme, dann werde ich hoffentlich schlauer sein und euch davon erzählen können.} Dann ging er endgültig. Er suchte das Klassenzimmer für Verteidigung auf, doch es war verschlossen. \enquote{Na toll, jetzt ist er nicht da und abgeschlossen hat er auch noch.}

Dann dachte er nach. \gedanke{Lehrerflügel. Es käme auf einen Versuch darauf an.} Harry machte sich also auf den Weg zum Lehrerflügel, in der Hoffnung fündig zu werden.

Als er um die Ecke bog, sah er gerade noch, wie Professor Dumbledore auf den Teppich drückte, der den Zugang versperrte. \enquote{Nimmst du mich mit? Ich muss Professor Elber sprechen. Es ist wichtig.}

Dumbledore sah zu Harry und nickte kurz. Dann bot er ihm den Vortritt an und trat hinter ihm in den Durchgang. Der Teppich versperrte wieder den Weg und beide gingen nebeneinander zur richtigen Tür. Es ging über ein paar Abzweigungen, bis sie dort waren.

\enquote{Was möchtest du denn von ihm, wenn ich fragen darf?}

Harry erinnerte sich wieder, was man ihm mal gesagt hatte, und antwortete deshalb: \enquote{Fragen kannst du mich alles\abs}

\enquote{Nur erhältst du nicht immer eine Antwort}, ergänzte Albus.

Harry grinste. \enquote{Ich möchte ihn etwas über meinen Tarnumhang und meine Vorfahren fragen. Ich habe vor ein paar Tagen geübt, durch Tarnungen zu sehen. Besser gesagt, sie zu erkennen.}

\enquote{Was hat das mit deinem Tarnumhang zu tun?}

\enquote{Das erste Hindernis waren einfach Buchszweige und eine Pappe. Dann kam ein Desillusionierungszauber. Und die letzten beiden waren ein Tarnumhang und dann meiner. Der letzte hat mich umgehauen. Madame Pomfrey gab mir einen \accentuate{Schlaflosen}.}

\enquote{Ah, deshalb warst du auf der Krankenstation.}

Harry sah Dumbledore an und meinte dann: \enquote{Immer bestens informiert, richtig?}

Dumbledore nickte nur und grinste nun ebenfalls. \enquote{Ich erfahre von Poppy nur, wer und wann bei ihr auf der Krankenstation liegt. Aber nicht warum. Das ist meist geheim. Außer sie braucht Hilfe.} Dann waren sie angekommen. Dumbledore zeigte auf die Tür und verabschiedete sich. Er lief den Gang wieder zurück. \enquote{Viel Spaß.}

Harry wollte gerade klopfen, als die Tür aufging und Professor Elber herauskam. Er rannte Harry fast um. Gerade noch konnte er den strauchelnden Harry auffangen und von größerem Schaden bewahren. Seine Bücher, die er in der Hand hielt, blieben neben ihm schweben, nachdem er sie los gelassen hatte, um Harry zu fangen.

\enquote{Was machen Sie denn hier?}

\enquote{Ich wollte Sie sprechen. Es geht um unser Gespräch von kürzlich}, sagte er, da eine andere Lehrerin gerade vorbeilief.

\enquote{Ah Aurora. Hier sind Ihre Bücher. Ich wollte gerade zu Ihnen kommen und sie vorbeibringen.}

\enquote{Hi Frederick, ich habe leider keine Zeit. Ich muss weg. Aber die Bücher dürften doch ihren Weg allein finden, oder?}, sagte sie und zwinkerte ihm zu.

\enquote{Wenn sie dürfen, dann finden sie ihren Weg bis ins Regal. Ansonsten nur bis vor die Tür.}

\enquote{Was muss ich tun?}

\enquote{Es ihnen erlauben. Sagen Sie es ihnen einfach, dann klappt das schon.} Er schob die Bücher in ihre Richtung.

\enquote{Ihr könnt zu mir und euch ins Regal stellen}, sagte sie und die Bücher flogen von dannen. Man hörte, nachdem die Bücher um die Ecke geflogen waren, eine Tür auf und wieder zu gehen. Aurora lief weiter und verabschiedete sich im Laufen.

\enquote{Kommen Sie rein Harry}, sagte Professor Elber.

Harry folgte ihm und stand nun in Professor Elbers Wohnung. Es war ein kleines Zimmer mit zwei Türen, die an den Stirnseiten des Raumes angebracht waren. Zwischen den Türen stand ein Regal mit Büchern. Links war ein Fenster durch das Licht hereinkam. Ein verzaubertes Fenster. Denn dahinter war eine Wand. Auf der rechten Seite waren nur wieder die vier Bilder der Gründer von Hogwarts zu sehen.

\enquote{Bin gleich wieder da}, sagte Professor Elber und verschwand durch eine der Türen.

Harry sah sich im Raum um und danach auf die Bilder. Er lächelte Salazar zu, doch dieser verzog keine Miene. Auch die anderen Bilder zeigten keine Reaktion.

Professor Elber betrat wieder den Raum (Die Spülung hatte Harry nur unbewusst wahr genommen) und trat seitlich neben Harry. \enquote{Die geben nicht jedem eine Antwort.} Harry sah ihn fragend an. \enquote{Die bewegen sich nicht}, sagte er nach einigen Sekunden.

\enquote{Also Muggelbilder?}

\enquote{Kann man so sagen.}

\enquote{Warum haben Sie eigentlich\abs?} Harry brach ab, da sich sein Professor auf die im Raum stehende Sitzgruppe zu bewegte. Harry folgte ihm und setzte sich auch. \enquote{\aabs in Ihrem Büro und hier Bilder der Gründer?} Als er es aussprach, merkte er, dass er wohl übers Ziel hinausgeschossen war und wurde rot.

\enquote{Wissens-hungrig und schamhaft, keine gute Kombination.} Er sah ihn an und dachte nach. \enquote{Sie bedeuten mir recht viel}, war das Einzige, was er dazu sagte. \enquote{Weswegen wollten Sie mich sprechen?}

\enquote{Weswegen? \gst Mein Tarnumhang. Und über die Familie Peverell.}

\enquote{Haben Ihnen die Hinweise, die ich Ihnen gegeben habe, nicht gereicht?}

Damit hatte Harry nicht gerechnet. Aber am meisten wunderte Harry, dass sein Lehrer scheinbar nicht nur über seinen Tarnumhang Bescheid wusste, sondern auch darüber, wer seine Vorfahren waren. Das war ihm unheimlich.

\enquote{Woher weiß der so viel?}, murmelte er halblaut vor sich hin und wurde direkt danach rot, als er merkte, dass er dies laut ausgesprochen hatte. Jedoch zeigte sein Lehrer keine Reaktion. \enquote{Ehrlich gesagt, hatte ich auf einen Beweis gehofft, dass ich wirklich mit einem der Peverells verwandt bin.}

\enquote{Wollen Sie einen Stammbaum haben?}

\enquote{Wäre hilfreich.}

\enquote{Damit kann ich leider nicht aufwarten. Aber ich kann Ihnen vielleicht etwas anderes geben. Wissen Sie um die besonderen Eigenschaften dieses Umhanges?}

\enquote{Ja, er schützt den Träger vor Entdeckungen.}

\enquote{Das machen andere Umhänge auch.}

\enquote{Aber meiner ist schon sehr alt. Er gehörte meinem Vater. Er muss also älter als ich sein. Ich habe mich schlau gemacht. Andere Tarnumhänge lassen mit der Zeit nach, bekommen Risse und sind nicht so fluchsicher. Bei meinem Umhang habe ich das noch nie festgestellt. \gst Er ist einfach perfekt.}

\enquote{Und das reicht Ihnen nicht?}

\enquote{Ich möchte nur wissen, ob ich von einem der Peverell-Brüder abstamme.}

\enquote{Der Umhang hat aber noch eine andere Eigenschaft.} Er stand auf und holte ein schmales Buch mit nur wenigen Seiten und überreichte es Harry. \enquote{Lesen Sie das. Darin steht genau, was sie tun müssen. Der Umhang hat die Eigenschaft, seinen wahren Herrn zu erkennen. Also die eigene Blutlinie. Wenn Sie von einem der Peverell-Brüder abstammen, dann wird der Umhang das anzeigen. Lesen Sie das Buch.}

Harry schaute auf den Umschlag. Er war rot. Keine Beschriftung.

\enquote{Heute haben wir Samstag. Bringen Sie es am Mittwoch zum Unterricht mit und geben Sie mir es dort. \gst Wollen Sie sonst noch etwas wissen?}

Harry schüttelte den Kopf, bedankte sich und stand auf. Dann verließ er den Raum und stand draußen auf dem Gang.

\enquote{Was suchen Sie denn hier Potter?}, hörte er Snapes Stimme.

\enquote{Ich war gerade bei Professor Elber, Professor.} Snape sah ihn nur an. \enquote{Und Sie?}

\enquote{Wo ich war, geht Sie nichts an, aber ich gehe jetzt in mein Büro. Unterricht vorbereiten.}

\enquote{Nehmen Sie mich bis zum Ausgang des Flügels mit?}, fragte Harry höflich, da er nicht sicher war, ob er den Weg allein finden würde.

\enquote{Wenn Sie nicht trödeln}, sagte Snape und lief los.

Harry war ihm dich auf den Fersen.

\trenn

Als Harry in die Große Halle kam, fragte er sich, warum die Hauselfen noch nicht anfingen, sie weihnachtlich zu schmücken. Immerhin war es bereits Dezember und die Gemeinschaftsräume wurden gerade geschmückt. Nur die Große Halle sah so aus wie immer. Professor Elber hatte noch immer dieses eigenartige Buch in der Hand und saß an seinem Platz in der Großen Halle, um zu Essen. Als er fertig war, blätterte er darin herum, ohne dass irgendjemand sehen konnte, was er da las.

Er verließ die Große Halle und bog ab. Harry wunderte sich noch immer, was Professor Elber wohl vorhatte. \gedanke{Hatte er die Aufgabe die Halle zu schmücken? Warum hatte er noch nicht damit angefangen?} Harry schwirrten die Fragen nur so durch seinen Kopf. Aber keiner wusste, was Elber wohl tun würde, sollte er damit beauftragt worden sein.

Dann hörte er eine merkwürdige Unterhaltung.

\enquote{Haben Sie Zeit, Hagrid?}

\enquote{Ja, Professor.}

\enquote{Gut, haben Sie Lust mit mir auf den Dachboden zu gehen und Weihnachtsschmuck zu holen?}

\enquote{Wird der nicht immer herbeigezaubert?}

\enquote{Ich weiß nicht, welchen Schmuck Sie meinen, aber den, den ich meine, kann man nicht so einfach herbeizaubern. Wenn Sie ihn sehen, dann wissen Sie, was ich meine.}

\enquote{Kann ich euch zwei begleiten?}

\enquote{Aurora? Schön dich zu sehen. Gern doch.}

\enquote{Was heißt hier: \enquote{Schön dich zu sehen?} Wir sehen uns doch fast jeden Tag, du Charmeur.}

\enquote{Und jeden Tag wirst du schöner.}

\enquote{Jetzt hör aber auf, ich werde ja gleich rot.}

\enquote{Ähm, läuft da was zwischen euch?}, fragte Hagrid.

\enquote{Ehrlich? Nein. Nur ein bisschen erotisches anzügliches Geplänkel unter sich gut verstehenden allein lebenden und unverheirateten Kollegen. Es gibt dem Leben etwas mehr Würze.}

\enquote{Ich hätte es nicht besser ausdrücken können, Aurora Schätzchen.}

\enquote{Jetzt hör aber auf, Frederick! Du machst mich ja ganz verlegen. Ich wette, das machst du mit anderen Frauen auch.}

\enquote{Ich würde mich nie an unsere stellvertretende\abs!}

\enquote{Die meinte ich ja auch nicht. Wie wäre es mit Septima?}

\enquote{Nicht im Entferntesten so wie mit dir. Sie ist nicht so locker drauf wie du. Sie kann man nicht so einfach mal in den Arm nehmen und trösten, oder an der Hand haltend um den See laufen, wenn kein Schüler zusieht.}

Harry musste grinsen. Das Bild würde er zu gern mal sehen.

\enquote{Also, gehen wir jetzt auf den Dachboden und schauen uns an, was wir dieses Jahr brauchen, oder nicht?}

\enquote{Gut, Hase. Gehen wir.}

\enquote{Ihr seid nicht zusammen?}, fragte Hagrid ungläubig.

\enquote{Ich glaube, egal was ich sage, falls ich dich küssen würde, glaubt der uns das nie, dass wir nicht zusammen sind. \gst Hagrid, wir verstehen uns sehr gut, aber wir sind kein Paar. Ich bin schon anderweitig\abs gebunden}, fügte er nach einer kleinen Pause hinzu.

Dann ging die Gruppe den Gang entlang und eine große geschwungene Treppe hinauf in das oberste Geschoss.

Am Tag vor dem 24. Dezember saß Harry wieder beim Abendessen, aber noch immer war die Große Halle leer. Aber keinen der Lehrer schien das zu stören. Zurück im Gemeinschaftsraum las Harry noch einmal seine Hausaufgaben durch, um sicherzugehen, dass er auch alles hatte und sie in Ordnung waren. Dann ging er zu Bett.

Mitten in der Nacht wachte Harry auf und machte sich auf den Weg zum Klo. Auf dem Rückweg hörte er komische Geräusche aus dem Gemeinschaftsraum und ging vorsichtig die Treppe herunter. Er sah Professor Elber, wie er eigenartige Zeichen auf eine Steinmauer im Gemeinschaftsraum malte und dann eigenartige Worte sprach. Die Zeichen verschwanden und Professor Elber wirkte zufrieden. Er drehte sich um und verließ den Gemeinschaftsraum. Harry ging vorsichtig zur Wand, sah aber nichts. Nach einigen Minuten, die er die Wand anstarrte, ging er wieder ins Bett. \gedanke{Morgen werde ich Ron und Hermine darüber Bescheid geben.} Harry schlief wieder ein und erwachte am nächsten Morgen.

Er zog sich an und ging in den Gemeinschaftsraum. Ron war auch schon da und beide warteten auf Hermine. Als sie endlich auftauchte, ging es ab zum Frühstück. Auf dem Weg dorthin erzählte er von seinen nächtlichen Beobachtungen, doch auch seine beiden Freunde konnten sich keinen Reim darauf machen. Vor der Großen Halle war bereits eine Ansammlung an Personen aller Häuser. Einige versuchten die Türen der Großen Halle zu öffnen, aber nichts rührte sich. Auch der sonst zuverlässige Alohomora-Zauber funktionierte heute nicht. Professor Dumbledore kam gerade um die Ecke, als einer sagte.

\enquote{Wir kommen nicht in die Große Halle Professor.}

\enquote{Schon mal geklopft?}, fragte Dumbledore.

Er klopfte und aus dem Inneren der Halle kam ein: \enquote{Moment noch, in zwei Minuten könnt ihr alle rein.}  Endlich dann, als die zwei Minuten um waren, öffnete sich die Flügeltür zur Großen Halle und Professor Elber stand am anderen Ende und breitete die Arme aus. \enquote{Fröhliche Weihnachten euch allen}, sagte er.

Die Große Halle sah wieder einmal wunderbar aus. In jeder Ecke stand ein geschmückter Weihnachtsbaum mit Figuren dran, die sich dauernd änderten. In einem Moment waren es Elefanten, die silbern das Licht brachen, im nächsten Moment wurden daraus Kamele und im nächsten Moment, wieder andere Figuren. Über jedem Platz, an dem ein Gedeck stand, war ein kleiner Weihnachtsbaum mit vielen Kerzen, feierlich geschmückt. Schnee rieselte von der Decke und erreichte den Boden. Die Wände waren mit Zweigen bestückt und wunderbar dekoriert. Mistelzweige hingen schwebend im Raum und auch über der großen Flügeltür waren Mistelzweige, die aber noch keiner bemerkte.

\enquote{Bitte setzt euch auf die Plätze, an denen euer Name steht}, sprach Dumbledore, \enquote{und keine Scheu, ihr könnt schon hereinkommen.}

Nach und nach betraten die Schüler den Raum. Die Farbe des Schmucks wandelte sich in rot-goldene Töne, grün-silberne, bronze-blaue, oder gelb-schwarze. Zuerst konnte man kein Muster erkennen, doch eine Vermutung ließ Harry die Schüler zählen und nach Häusern einordnen. Dann kam er dahinter. \gedanke{Klever gemacht}, dachte er sich.

Nachdem alle ihre Plätze eingenommen hatten, schlossen sich die Türen der Großen Halle und Dumbledore schnippte mit den Fingern. Ein Hauself erschien und Dumbledore gab ihm ein Zeichen. Der Elf verschwand sofort wieder und Dumbledore begann mit seiner Rede. \enquote{Es dauert leider noch ein paar Minuten, bis ihr zu frühstücken anfangen könnt, da ich noch eine Überraschung für euch habe.}

Harry schaute nur hin und her und fing die fragenden Blicke seiner Mitschüler auf. Die Große Halle sah leer aus, fand er. Wenige waren über die Weihnachtsferien in Hogwarts geblieben. Nur ungefähr fünfzig Schüler waren in der Großen Halle und jeder saß vor seinem Teller.

Nach einigen Minuten, in denen jeder sich umgesehen hatte, schaute Professor Elber auf seine Uhr und meinte: \enquote{Gleich ist es so weit.} Er zählte herunter. \enquote{Fünf - Vier - Drei - Zwei - Eins.} Er klatschte in die Hände und die Türen der Großen Halle gingen mit lautem Knarren auf.

Alle Blicke wanderten zur Tür, in der viele Personen standen. Freudenschreie fielen in der Halle und reflektierten an den steinernen Wänden. Denn im Türrahmen standen die Eltern, Großeltern oder nahe Verwandte der Schüler. Einige Schüler liefen ihnen entgegen und nahmen sie in ihre Arme. Harry machte nur große Augen.

\enquote{Ich freue mich, dass alles geklappt hat}, sagte Professor Elber wieder. \enquote{Bitte nehmen Sie ihre Plätze ein.} Platzkärtchen erschienen und zeigten den Gästen wo sie zu sitzen hatten. Nachdem alle ihre Plätze gefunden und Platz genommen hatten, klatschte Professor Dumbledore in die Hände und das Frühstück erschien auf den gemütlichen, großen Tischen. Alle begannen zu frühstücken und sich zu unterhalten. Mr. und Mrs. Weasley waren da und auch Hermines Eltern saßen da und frühstückten. Überhaupt waren viele Eltern oder Großeltern da. Nur Harry hatte niemanden, denn sein Onkel und seine Tante würden niemals einen Fuß in Hogwarts setzen. Aber um das zu vergessen, saß Mrs. Weasley neben ihm und lenkte ihn so ab.

Einige waren früh mit Essen fertig und führten ihre Besucher die Große Halle hinaus und durch das Schloss. Harry hielt sich nach seinem Frühstück an Hermine und ihre Eltern, da er sie nur einmal kurz gesehen hatte, als sie in \fab waren. Sie verließen zu viert die Große Halle. Harry hätte es wissen müssen, denn als Erstes ging es Richtung Bibliothek. \enquote{Hermines Lieblingsplatz}, sagte Harry zu ihren Eltern.

Sie gab ihm einen Knuff in die Seite und ihre Eltern lachten. In der Bibliothek angekommen, staunten Hermines Eltern, denn so eine große Bibliothek hatten sie noch nie gesehen. Sie führte sie etwas herum und machte sich dann auf den Weg zum Gemeinschaftsraum.
Als sie die Bibliothek verließen, kam ihnen Professor Elber entgegen und fragte: \enquote{Harry, haben Sie mal kurz Zeit? Dauert nur zwei Minuten.} Er nahm ihn zur Seite und fragte ihn dann: \enquote{Ich habe bereits mit den anderen gesprochen, und mit Ihnen wären sie vollzählig. Ich möchte unseren Gästen ein kleines Quidditch-Match bieten und über die Ferien sind mit Ihnen gerade vierzehn Quidditch-Spieler da. Kommen Sie bitte nach dem Mittagessen zum Quidditch-Feld, sie müssen sich zu zwei Demo-Teams zusammenstellen, damit unsere Gäste Morgen etwas zu sehen bekommen.}

Harry bekam große Augen. \enquote{Sie meinen, die bleiben über Nacht?}

\enquote{Ja}, antwortete Professor Elber.

\enquote{Ja, aber wo sollen die schlafen?}

\enquote{Können Sie sich das nicht denken? Sie haben mich doch heute Nacht beobachtet.}

Harrys Augen weiteten sich und er erinnerte sich wieder daran, wie Professor Elber im Gemeinschaftsraum stand und etwas auf die Wand malte. \enquote{Dann haben Sie Gästequartiere in den Häusern geschaffen?}, fragte Harry.

\enquote{Ja}, antwortete Professor Elber.

Harry grinste und fragte \enquote{Wissen die anderen schon, dass ihre Eltern über Nacht bleiben?}

\enquote{Nein, und das sollen sie bis heute Nacht auch nicht erfahren.}

Später dann, als es Zeit für das Abendessen war, gingen Harry, Hermine und ihre Eltern in die Große Halle. Die Tafel war bereits gedeckt und es gab keine Platzkärtchen mehr. Die großen Tische waren verschwunden und es gab viele kleinere Tische, denn es waren nicht viele Personen während der Ferien da. Nachdem alle gegessen hatten, stand Dumbledore auf und fing wieder an.

\enquote{Ich hatte heute Morgen bereits gesagt, dass es noch eine Überraschung geben wird. Unsere Gäste verlassen uns erst am zweiten Weihnachtsfeiertag nach dem Mittagessen.}

Die Augen der Schüler begannen zu leuchten und einige Münder öffneten sich. Viele schauten ihre Eltern an, die sie angrinsten. Harry bemerkte Professor McGonagall, die Professor Elber an seinem Mantel zog. Dieser setzte sich und unterhielt sich mit Professor McGonagall. Fast schien es ihm, als ob sie verärgert sei. Aber nach ein paar Worten besserte sich ihre Miene und Harry hatte den Eindruck, dass Professor McGonagall zu schmunzeln begann. Nach dem Essen begaben sich einige nach draußen, andere hingegen gingen direkt mit ihren Eltern, oder nahen Verwandten, in ihre Gemeinschaftsräume. Nach einigen geselligen Unterhaltungen der Schüler mit den Gästen, gingen viele von ihnen ins Bett. Harry unterhielt sich noch mit Hermine; besser gesagt Hermine erzählte die ganze Zeit und Harry hörte ihr zu. Nach einiger Zeit wurde Harry Müde und er ging ins Bett. Hermine stand ebenfalls auf und gab ihm einen Gute-Nacht-Kuss. Harry rieb sich erstaunt seine Backe und ging die Treppen hoch ins Bett.

Am Tag darauf während des Mittagessens nahm Dumbledore einen Löffel in die Hand und klopfte gegen seinen Trinkkelch. \enquote{Ich möchte alle Schüler bitten, nach dem Mittagessen mit ihren Gästen auf das Quidditch-Feld zu gehen, da dort eine kleine Demonstration stattfinden wird.}

Harry grinste, als Hermine ihn fragend ansah. \enquote{Du wusstest das?}, fragte sie ihn.

Harry grinste. \enquote{Nur seit gestern.}

Er stand auf und ging Richtung Quidditch-Feld, da er sich noch umzuziehen hatte. Es dauerte eine Weile bis alle eingetroffen waren, also ging das Team um Harry noch einmal ihren Plan durch. Als ein Pfiff ertönte, gingen sie an den Rand des Feldes, bestiegen ihre Besen und stiegen hoch in die Lüfte. Es begann leicht zu schneien, als das erste Tor geschossen wurde, weil Harry den Ball durch die Ringe ließ. Heute hatte er die Position als Hüter. Ihm kam dies gelegen, da er bei Arabella schon einmal die Position des Hüters innehatte und daher auf etwas Erfahrung zurückgreifen konnte. Harry erspähte während des Spieles ein paar Mal den Schnatz, aber da er dieses mal nicht der Sucher war, durfte er ihn nicht fangen. Er versuchte verzweifelt, seinen Team-Sucher darauf aufmerksam zu machen. Aber es war unmöglich, da beide Sucher in seiner Nähe waren. Derart abgelenkt, ließ er einen weiteren Ball durch einen der Ringe. Beide Sucher drehten sich nun um und entdeckten den Schnatz. Jetzt entbrannte ein Kampf um den kleinen goldenen Ball. \gedanke{Ich hätte ihn schon längst gehabt}, dachte Harry. \gedanke{Aber Cho ist auch nicht schlecht.} Kurz darauf hatte sie ihn auch schon und das Spiel war beendet. Am Ende gewann Harrys Team mit 160 zu 20 und das Spiel war zu Ende.

Auf dem Weg zurück zum Schloss meinte Hermines Vater: \enquote{Das war ein tolles Spiel. Und als Torwart\abs}

\enquote{Hüter}, unterbrach ihn Hermine.

\enquote{Ja, als Hüter warst du sehr gut. Soweit ich das beurteilen kann.}

\enquote{Danke Mister Granger. Aber normalerweise bin ich der Sucher in unserem Team und fange den goldenen Schnatz. Aber wir sind über die Ferien so wenig, sodass ich heute eine andere Position eingenommen hatte.}

\enquote{Ah ja}, meinte Hermines Vater.




\begin{kommentar}
Harry will etwas über seine Vergangenheit und seine Vorfahren wissen und befragt deshalb Elber. Dort entdeckt er wieder die vier Gründer in Bildern an der Wand. Der andere Ort ist Elbers Büro. Ein weiterer schöner Hinweis darauf, dass diese vier seine Kinder sind.
\end{kommentar}

\begin{kommentar}
Als Harry Elber dann wieder verlässt, trifft er auf Snape. Dieser nimmt ihn als kleine Geste des guten Willens mit. Das Verhältnis der beiden scheint so langsam besser zu werden.
\end{kommentar}

\begin{kommentar}
Kurz darauf unterhält sich Elber mit Hagrid und Auror. Dort sagt Elber, dass er anderweitig vergeben ist. Erst im nächsten Teil kommt heraus, dass er mit der Zwillingspsyche von Bellatrix Lestrange verbandelt ist. Aber hier sind bereit die ersten Andeutungen zu erkennen.
\end{kommentar}
