\chapter{Leben und leben lassen}


Harry saß in dem grünen Ledersessel mit dem schwarzen Holz in Salazars Räumen und sah zu den verzauberten Fenstern hinaus. Vor ein paar Tagen erst hatte er entdeckt, wie man die Aussicht ändert und es einmal versucht. Doch nach wenigen Minuten kehrte er wieder zur alten Ansicht zurück, da er sich den Ort sehr genau vorstellen musste. Und das einzige, was ihm dabei einfiel, war das Haus seiner einzigen Verwandten. Er überlegte, was er Dobby sagen würde. \gedanke{Traupaar-Führer}, ging ihm wieder durch den Kopf.

Er wusste jetzt, dass seine Rolle der eines Priesters entsprach und er die Trauung vornehmen sollte. Dobby hatte ein paar Reden beigefügt, damit Harry eine Ahnung davon hatte, aber eigentlich musste man keine vorgegebene Regeln beachten. Das einzige, was ihm nichts sagte, war der Teil mit der magischen Bindung.

\enquote{Dobby}, rief er. \enquote{Es geht um deine Trauung.}

Es dauerte ein paar Sekunden, dann tauchte er mit seiner Braut Winky auf.

\enquote{Sir Harry hat gerufen?}, sagte Dobby.

\enquote{Ja Dobby. Hallo Winky}, begrüßte er die beiden Elfen. \enquote{Kommt, setzt euch.} Dobby und Winky sahen sich an, dann setzten sie sich schließlich auf den Boden. \enquote{Eigentlich meinte ich auf den Sessel. Aber egal.} Harry setzte sich ebenfalls auf den Fußboden und fragte Dobby über die genauen Umstände aus.

\enquote{Soweit ich verstanden habe, soll ich euch trauen.} Die beiden Elfen nickten. \enquote{Und ihr beide wollt, dass ich das mache.} Dabei sah er Winky an; diese nickte ein paar Mal in schneller Abfolge. \enquote{Bis dahin habe ich es verstanden und durch deine beigefügten Reden habe ich auch eine Ahnung, was ich sagen möchte.} Damit hatte er schon mal implizit zugestimmt, wenn man die offenen Punkte beiseite ließ.

\enquote{Was mich aber irritiert, beziehungsweise was ich nicht ganz verstehe, ist der Teil mit der magischen Bindung.}

Winky begann zu erzählen. \enquote{Die magische Bindung dient dazu, dass sich die Ehepartner trotz der Schutzzauber, die die Elfen zusätzlich über die Häuser ihrer Meister legen, damit fremde Elfen nicht in die Häuser apparieren können, sehen und besuchen können. Sonst müssten wir uns immer gegenseitig anmelden. Aber durch die Bindung können wir zusammen wohnen, auch wenn unsere Meister räumlich getrennt sein sollten. Die Schutzzauber erlauben das.}

\enquote{Und geht das automatisch, oder muss ich da etwas tun?}

\enquote{Ihr müsst einen Zauber sprechen, Sir Harry}, antwortete Dobby. \enquote{Dobby wird ihn euch mitteilen, wenn Ihr zustimmen solltet. Ihr dürft ihn aber niemandem verraten.}

Harry nickte. \enquote{Aber warum soll ich das übernehmen?}

Dobby sah Harry mit gesenkten Ohren an. Dann wandte er einen Blick Richtung Boden. Er war nicht sein Meister, deshalb konnte er ihn auch anlügen. Er dachte nach, was er sagen sollte. Winky stieß ihn von der Seite an und nickte kaum merkbar.

\enquote{Dobby und Winky hoffen, später einmal von Sir Harry aufgenommen zu werden}, sagte der kleine Elf schließlich ziemlich leise und zurückhaltend.

Das musste Harry erst einmal verdauen. Er verstand den Zusammenhang noch immer nicht. \enquote{Ist das der Grund?}

\enquote{Nicht ganz}, sagte Winky. \enquote{Durch den Zauber, den Sie, Sir Harry, sprechen, sind Sie in kleinem Maße Teil des Zaubers. Es ist ein intuitives Gespür, falls es einem von uns schlecht gehen sollte.}

Und Dobby ergänzte: \enquote{Die Aufgabe des Traupaar-Führers geht über die normale Trauung hinaus. Er begleitet die beiden zu Trauenden zum Zeremonialplatz. Will heißen: Er appariert mit ihnen hinter die Gäste und läuft dann hinter dem Brautpaar nach vorne, umrundet sie und führt die Zeremonie. Dann bringt er sie zu den Feierlichkeiten, sollten sie an einem anderen Ort sein.}

Harry dachte noch eine kleine Weile nach. Im Prinzip hatte er eh schon zugestimmt. \enquote{Also gut. Ich mache es. Wann findet die Trauung statt?}

\enquote{Dobby und Winky dachten, Mitte des nächsten Jahres. An einem Sonntag-Nachmittag hier in Hogwarts. Die Erlaubnis des Direktors haben Dobby und Winky schon.}

Jetzt musste Harry schmunzeln. \enquote{Wer ist noch dabei?}, fragte er nach.

\enquote{Alle Elfen von Hogwarts; ein paar Freunde, andere Elfen; Sie, Sir Harry und der Schulleiter. Sonst niemand.}

Dann hatte Harry doch noch eine Frage. \enquote{Wird die Zeremonie im Stehen durchgeführt, oder ist das egal?}

\enquote{Der Traupaar-Führer leitet die Zeremonie}, sagte Winky ganz verständnislos.

Das war für Harry das Zeichen, dass er sie gestalten konnte, wie er wollte. Er würde, wie in japanischen Filmen, auf seinen Beinen auf dem Boden sitzen. Ebenfalls die Elfen. Die Gäste bekamen Kissen. Er musste also die Halle schmücken. Aber bis dahin war noch eine Menge Zeit. Es konnte sich noch viel ändern.

\trenn

\enquote{Schnell, rufen Sie Ihren VgddK-Lehrer}, hörte Harry die Stimme von Professor McGonagall. Kurz danach hörte er schnelle Schritte auf sich zukommen und um die Ecke bog ein kleines Mädchen, das ihn fast umrannte. Er kannte sie zwar vom Sehen, hatte aber keine Ahnung wie sie hieß. Nach ein paar Schritten kam er um die Ecke und sah, wie Professor McGonagall verzweifelt versuchte, ein sehr lebhaftes Feuer zu löschen. Es schien sich zu verändern und umherzuwandern. Aber das war nicht alles, so etwas hielt Harry zwar für ungewöhnlich, aber er hatte sich angewöhnt über nichts mehr zu staunen, was in Hogwarts passierte.

Erst als er den Grund für Professor McGonagall mehr als verzweifelte Versuche sah, entglitten Harrys Gesichtszüge. Hinter dem Feuer in einen kleinen Winkel zurückgedrängt waren zwei Schüler und zitterten und schrien. McGonagall konnte die zwei schließlich beruhigen, sodass sie nur noch zitterten, aber nicht mehr schrien. Das Feuer schien den Zaubern auszuweichen. Es schien so, als ob das Feuer leben würde und immer, wenn es ihr zu gelingen schien, das Feuer einzudämmen, flammte es kurz darauf wieder zu seiner vollen Größe wieder auf.

Da sonst nur jüngere Schüler da waren, beschleunigte Harry seine Schritte, zog seinen Zauberstab und fragte Professor McGonagall: \enquote{Brauchen Sie Hilfe Professor? Kann ich irgendetwas tun?}

Professor McGonagall drehte ihren Kopf. \enquote{Ja Mister Potter. Versuchen Sie das Feuer in Schach zu halten, oder zu löschen.} Harry nickte. \enquote{Haben Sie schon einmal mit Dämonenfeuer zu tun gehabt?} Harry schüttelte den Kopf. \enquote{Versuchen Sie einfach es in Schach zu halten, bis Professor Elber kommt. Hoffentlich weiß er, wie man es löscht, denn es entzieht sich meinen Versuchen. Es scheint irgendwie anders zu sein.}

Zusammen mit seiner Verwandlungslehrerin versuchte er das Feuer in Schach zu halten. Es konnte sich nur noch in Grenzen bewegen, aber die beiden Schüler konnten noch immer nicht entkommen. Zumindest hörten die beiden auf zu zittern und begannen sich dann zusammengekauert auf den Boden zu setzen.

Wasser war zwecklos, erkannte Harry, nachdem er einen Schwall darüber fließen gelassen hatte. Es ging einfach durch das Feuer hindurch und benässte den Boden. Dann versuchte er es mit Wind wegzublasen. Zuerst vertrieb er das Feuer, doch als einer der Schüler aufstand um durch den schmalen Spalt zu entwischen, bewegte sich das Feuer wieder auf seine Ausgangsposition zurück. Das einzig Gute daran war, dass das Feuer sich nicht vermehrte oder sonst irgendwie ausbreitete.

Dann erinnerte er sich, was Professor Elber über das lebendige Feuer gesagt hatte. Er dachte kurz nach und steckte seinen Zauberstab ein. Mit geschlossenen Augen und ruhigen Gedanken hielt er das Feuer nur mit seinem Geist in Schach; an Ort und Stelle. Außer bei seinen privaten Stunden und ein paar wenigen in Professor Elbers Unterricht zauberte er immer mit Zauberstab. Deshalb entfiel ihm immer, dass er es auch ohne beherrschte.

Jetzt hörte er wieder Schritte hinter sich. Er drehte sich kurz um und sah die Kleine mit Professor Elber hinter sich herlaufen. \enquote{Minerva! Was?} Dann stockte er. \enquote{Lebendiges Feuer? Das können die Siebtklässler doch löschen? Weshalb bin ich hier Minerva? Du kannst das doch löschen.}

\enquote{Eben nicht}, gab sie sauer zurück und drehte sich um. \enquote{Es widersetzt sich sämtlichen bekannten Versuchen.}

Harry hatte das Feuer gut unter Kontrolle. Nur löschen konnte er es nicht. Er spürte förmlich die Blicke seines Lehrers hinter sich.

\enquote{Lässt du Harry jetzt ganz alleine dagegen ankämpfen?}, fragte Elber Professor McGonagall.

\enquote{Nein, du sollst dich darum kümmern.}

Harry hörte Schritte. Dann das Rascheln eines Umhanges und spürte danach eine Hand auf seiner Schulter. \enquote{Sie können aufhören}, hörte er. Dankbar ließ er gedanklich los, lies seine Hände sinken und sah Professor Elber zu.

Professor Elber schlenkerte mit seinem Zauberstab, um das Feuer zu testen. \enquote{Hmmm. \gst Du hast nicht übertrieben, Minerva. Eigenartiges Feuer.}

\enquote{Sagte ich doch.}

Er blieb stehen und hielt das Feuer in Schach, während er nachdenklich sein Kinn rieb. Er versuchte noch ein paar Zauber, doch Harry hatte den Eindruck, dass er nicht besonders entspannt war und locker wie bisher darüber hinwegging, das Problem löste und sich wieder seinen Aufgaben widmete. Harry bemerkte nur, wie er seinen Zauberstab einsteckte. Das Feuer baute sich danach zu seiner vollen Größe wieder auf und loderte breiter als vorher und etwa bis zu Harrys Bauchnabel.

\enquote{Frederick, was tust du?}

Er gab ihr keine Antwort. Harry stand entgeistert neben ihm und beobachtete ihn. Er schloss seine Augen und begann eine Beschwörungsformel zu sprechen. Harry hatte den Eindruck, dass die Luft um ihn herum zu flimmern anfing. Gerade so, als würde er in der Wüste stehen und an den Horizont blicken, oder im heißen Sommer die Asphalt-Straße beobachten.

Professor Elbers Haare wurden weiß und er öffnete seine Augen wieder. Dann blickte er kurz zu Harry und trat auf das Feuer zu. Genau an den Stellen, an denen er sich bewegte, verzog sich das Feuer, hatte Harry den Eindruck. Als er durch das Feuer hindurch war, ging er in die Hocke und sprach mit den beiden Schülern. \enquote{Ich nehme euch jetzt an meiner Seite hoch. Legt eure Beine um meinen Bauch und schmiegt euch so fest ihr könnt an mich. \gst Euer Leben hängt davon ab. Lasst nicht los, solange ich es euch nicht sage. Verstanden?}

Die beiden nickten und standen auf. Professor Elber legte je einen Arm um sie und zog sie hoch. Sofort schlangen sie ihre Beine um ihn und schmiegten sich fest an ihn. Ihre Köpfe vergruben sie in seinen Haaren, sodass sie nichts sahen.

\gedanke{Sie müssen furchtbare Angst haben}, dachte Harry.

Langsam aber sicher, Schritt um Schritt, trat Professor Elber durch das Feuer, welches jetzt stärker loderte. Es versuchte, die beiden zu verbrennen, kam aber nicht an sie heran. Etwas schien das Feuer abzuhalten. Als er das Feuer um einige Meter hinter sich gelassen hatte, ging er wieder in die Hocke. \enquote{Ihr könnt jetzt loslassen}, sagte er.

Die beiden lösten sich von ihm und umarmten ihn noch einmal kurz. \enquote{Danke}, sagten beide fast gleichzeitig.

Er sah zu ein paar anderen Mitschülern. \enquote{Ihr bringt die beiden direkt in den Krankenflügel.} Die angesprochenen nickten und nahmen sie mit.

Professor Elber stand wieder auf und zog seinen Zauberstab. Er fuhr an sich hinunter und das eigenartige flimmern verschwand. Dann wandte er sich wieder dem Feuer zu. Er schwang seinen Zauberstab über seinen Kopf und beschwor ebenfalls ein Feuer hervor und ließ es auf das andere zufließen. Beide Feuer vereinten sich und führten einen atemberaubenden Tanz auf. Die Flammen versuchten sich gegenseitig zu unterdrücken und zu übertrumpfen und die anderen Flammen zu verdrängen und auszulöschen.

Dann schwang er wieder seinen Zauberstab und das Feuer klatschte mit Gewalt an die dahinter liegende Wand, an der zuvor noch die beiden Zweitklässler gesessen hatten. Jetzt war das Feuer verschwunden, aber die Wand war mit mehreren Millimetern Ruß überzogen.

Nachdenklich stand Harrys Lehrer da und sah ihn an, sprach aber zu McGonagall: \enquote{Minerva? Tust du mir einen Gefallen?}

\enquote{Welchen?}

\enquote{Treibe alle Schüler in der Großen Halle zusammen. Ich möchte eine Ansprache halten.} Er drehte sich zu Professor McGonagall um. \enquote{Das war kein normales lebendiges Feuer. Und ich möchte wissen, ob irgendein Schüler dafür verantwortlich ist.}

\enquote{Zusammentreiben?}, fragte Professor McGonagall aufgeregt.

\enquote{Dann eben zusammen rufen.}

\enquote{Ok, Frederick.}

Alle Schüler, mit Ausnahme der beiden im Krankenflügel, standen nun in der Großen Halle. Die Tische und Bänke schwebten an der Decke der Halle, sodass genug Platz war. Professor Elber stand mit einem grimmigen Gesichtsausdruck auf der Empore und schritt auf und ab. Dann drehte er sich und begann.

\enquote{Es dürften noch nicht alle hier darüber Bescheid wissen, was heute im fünften Stock vor dem Ge\-schichts\-kun\-de-Klas\-sen\-zim\-mer stattfand.} Dann lief er wieder auf und ab. Professor McGonagall machte ein betretenes Gesicht und beobachtete Professor Elber genau. Alle anderen Lehrer waren ebenfalls da.

Er drehte sich wieder zu den Schülern und nahm seinen Zauberstab heraus. Dann ging er in die Mitte der Empore und beschwor Flammen hervor, welche er über seiner Hand schweben ließ.

\enquote{Das hier ist ein ganz normales Feuer. Es nährt sich von brennbaren Materialien, oder von der Menge an Magie, die sie ihm zuteilen, damit es nicht erlischt.} Er schloss seine Hand und das Feuer verschwand wieder.

Er schwang wieder seinen Zauberstab. \enquote{Das hier ist lebendiges Feuer. Oder auch Dämonenfeuer.} Eine kleine Flamme schwebte wieder über seiner Hand. Aber man konnte ihr ihre Gefährlichkeit nicht ansehen. Er drehte seine Hand und das Feuer fiel zu Boden. Sofort breitete es sich auf die dreifache Größe aus und begann zu wachsen. Mit seinem Auge auf dem Feuer begann er zu erzählen. \enquote{Ich hatte Ihnen schon einmal davon erzählt. Aber dieses Mal verwende ich den Ausdruck Teufelsfeuer. \gst Dieses Feuer war heute im fünften Stock zu sehen. Es schloss zwei Schüler ein. Zweitklässler. Die sind jetzt im Krankenflügel.} Dann wurde er deutlich zorniger. \enquote{Jeder von Ihnen wird sich einer kleinen Überprüfung unterziehen müssen. Keiner verlässt mir die Große Halle, bevor nicht der Letzte in diesem Raum seinen Zauberstab hat kontrollieren lassen.} Leichte Panik durchflutete den Raum. \enquote{Diejenigen, die Ihre Zauberstäbe nicht bei sich haben, werden sich konzentrieren und laut, aber deutlich, Ihnen befehlen, hierherzukommen.}

Das Feuer breitete sich unterdessen immer weiter unter den wachsamen Augen Professor Elbers aus.

Jetzt sagte er zornig: \enquote{Sollte ich feststellen, dass irgendein Schüler dieses Feuer}, und er fügte leiser hinzu, \enquote{oder irgendein Lehrer}, und nun wieder in normaler Lautstärke, \enquote{wird derjenige, diejenige, oder diejenigen von mir etwas bekommen. Und zwar eine Reise nach Hause. Ich denke, Ihr Schulleiter wird Sie dafür hinauswerfen.}

Das Gemurmel, welches mittlerweile die Halle erfüllte, wurde lauter und leichte Panik stieg auf.

\enquote{Kommen Sie einzeln hervor und treten Sie durch das Feuer. Es wird Ihnen nichts tun.} Schlagartig breitete sich das Feuer auf die gesamte Breite der Großen Halle aus. Man musste hindurch, um sich der Überprüfung zu unterziehen. \enquote{Strecken Sie mir Ihren Zauberstab mit dem Griff von sich weg entgegen. Ich werde ihn kurz berühren und Ihnen dann sagen, ob Sie gehen können.}

Keiner der Anwesenden bewegte sich. Also machte Harry den Anfang, da er wusste, ihm würde nichts geschehen. Selbstbewusst trat er vor und holte seinen Zauberstab aus seinem Umhang. Er blieb kurz vor dem Feuer stehen und drehte seinen Zauberstab in der Hand um. Nun hielt er die Spitze in der Hand. Das Teufelsfeuer bildete eine schmale Gasse und er hatte das Gefühl, dass das Feuer zu ihm züngelte.

Er atmete noch einmal kurz durch und schritt ohne zu zögern durch die Flammen. Sie fühlten sich warm und weich an. Keine Spur von übermäßiger Hitze oder Angriffsgedanken. Er streckte seinen Zauberstab seinem Lehrer entgegen. Dieser berührte ihn einmal sachte und tippte ihn danach mit seinem Zauberstab an. Kleine grüne Funken kamen aus seiner Spitze hervor. Professor Elber nickte und bat Harry an die Seite.

Bei jedem Schüler vollzog er dieselbe Prozedur. Bei einigen Slytherins war Harry gespannt, ob sie den Test bestehen würden! Doch Malfoy zeigte keine Spur von Unsicherheit, als er auf seinen Lehrer zuging. Aus seinem Zauberstab stoben ebenfalls grüne Funken. Grinsend ging er an Harry vorbei und stellte sich neben Pansy und hinter Maria.

Einmal stutzte Professor Elber, als er Crabbes Zauberstab überprüfte. Er nahm ihn aus seiner Hand und betrachtete ihn nochmals. Dann sah er ihm in die Augen und meinte, da noch ein knappes Dutzend Schüler übrig waren: \enquote{Bleiben Sie hier neben mir stehen. Ich behalte ihn noch kurz bei mir.} Harry konnte in Crabbes Gesicht Panik erkennen. Nachdem die restlichen Schüler und auch Lehrer überprüft worden waren und auch Professor Dumbledores Zauberstab überprüft wurde, widmete Professor Elber sich wieder Crabbe.

Noch einmal untersuchte er Crabbes Zauberstab und anschließend Crabbe selber. Er flüsterte einen komplizierten Zauber und Crabbes Gesicht veränderte sich. Kleine Wolken kamen aus seinen Ohren und verbanden sich über seinem Kopf. Eine unheimliche Stimme drang daraus hervor. \enquote{Imperio. Du wirst ein Dämonenfeuer nächste Woche im Schloss legen. Pass aber auf, dass niemand dabei zu Schaden kommt. Übe jetzt einmal.} Dann verschwand die Rauchwolke.

\enquote{Mister Crabbe}, fing Professor Elber an. \enquote{Ich muss Sie bitten, Professor Snape zu folgen.} Professor Elber blickte zu Snape und meinte: \enquote{Nehmen Sie ihn nicht zu hart ran. Ich habe das Gefühl, dass er erst nächste Woche das Feuer legen sollte. Für das heute war jemand anderes verantwortlich.}

\enquote{Aber wer?}, schaltete sich Dumbledore ein. \enquote{Wer ist denn übrig? Filch, Madame Pomfrey, die beiden im Krankenflügel und Sie. Sonst war es jemand, der ins Schloss eingebrochen ist.}

Professor Elber gab seinen Zauberstab an Dumbledore, der ihn überprüfte. Es dauerte eine halbe Minute, bis er ihn zurückerhielt und allen bestätigte, dass auch er es nicht gewesen war.

\enquote{Wenn keiner etwas dagegen hat}, machte Professor Elber weiter, \enquote{dann werde ich die letzten Vier überprüfen.}

\enquote{Mister Filch können Sie auslassen}, meinte Professor Flitwick.

\enquote{Sicher? Ich meine zu wissen, dass man unter gewissen Umständen unter dem Imperius auch \gst Nicht-Magiern \gst eine gewisse Menge an Magie geben kann, um den Auftrag auszuführen.}

Dann ging er auf Argus Filch zu und untersuchte ihn. \enquote{Ok. Dann werde ich mal die Letzten drei im Schloss angehen.}

Als er zwanzig Minuten später wieder kam, standen immer noch alle in der Halle herum. Das Feuer hatte sich inzwischen ausgebreitet und die Große Halle mächtig angewärmt.

\enquote{Upps.} Er schlenkerte seinen Zauberstab und das Feuer wurde weniger, bis es schließlich verschwand. \enquote{Negativ bei Poppy}, grummelte er. \enquote{Aber die beiden Schüler waren positiv. Eindeutige Anzeichen vom Imperius und anderen bewusstseinsverändernden Mitteln und Zaubern.}

Die Tische und Bänke kamen langsam von der Decke und Professor Elber setze sich auf eine der Bänke.

\enquote{Ich verstehe nicht, wie so etwas möglich ist!}, sagte Professor McGonagall.

\enquote{Man hat die Schüler wohl während ihrer Ferien unter den Imperius gestellt, um hier Chaos zu verursachen. Leider, oder sollte ich sagen zum Glück, haben sich die beiden nur selber verletzt.}

\enquote{Nein}, intervenierte Dumbledore. \enquote{Was sie meinte war: Wie kommen Sie zu der Ansicht, dass sie unter dem Imperius standen \gst stehen?}

\enquote{Es gibt ein paar Zauber, um das herauszufinden. Und nein, ich glaube kaum, dass ein Auror diese anwenden würde. Es kostet eine Menge Kraft und wird im Allgemeinen der dunklen Seite zugeordnet. Es ist selber eine leichte Form des Imperius, aber legal. Ich teste damit den Willen und ob er gelenkt wird.} Nachdenklich sahen die anderen Lehrer ihn an. \enquote{Ich muss mich erst einmal ausruhen. Meine Möglichkeiten waren begrenzt und auch das Dämonenfeuer zu löschen hat mich sehr beansprucht, die ganze Aktion hat mich sehr mitgenommen.}

Dann stand er auf und verließ mit schweren Schritten die Große Halle.

\enquote{Warum hast du ihn gewähren lassen?}, fragte McGonagall Dumbledore leise.

\enquote{Weil er mehr über Hogwarts und Magie weiß, als wir beide zusammen.} Dann ging auch Dumbledore.

Das brachte Harry zum Nachdenken. Er schlenderte durch das Schloss, begleitet von Ginny. Und wieder einmal spürte er eine Lampe, die nicht wie die anderen war. Er entfernte sie und ersetzte sie durch eine dauerhaft funktionierende. Ginny half ihm dabei, da er ihr den Zauber schon vor Wochen gezeigt hatte. Zusammen hatten sie bereits mehr als sechzig Lampen repariert.

In Salazars Räumen angekommen, kuschelten sich beide aneinander. Das letzte Schuljahr lief vor Harrys geistigem Auge vorbei. Bei der Unterhaltung mit Dumbledore über die Bilder von Hogwarts blickte er automatisch zu Salazar.

\stimme{Ein interessanter Einblick, den du uns da gewährst}, tönte es in seinem Geist. \stimme{Ich finde es nur Schade, dass man euch heutzutage nicht mehr das lehren kann, was wir damals gelehrt haben.}

\gedanke{Wie meinst du das?}

\stimme{Ich meine die alte Magie. Man hat über die Jahrhunderte scheinbar viel vergessen. Zumindest du bist in diese Richtung unterwiesen worden.}

\gedanke{Du meinst unseren \VgddK-Lehrer!}, folgerte Harry.

\stimme{Nenn ihn ruhig beim Namen \gst Frederick. Er hat dir die Sachen beigebracht, die man früher gelehrt hat.}

\gedanke{Du meinst die alte Magie?}

\stimme{Ja.}

\gedanke{Woher kennt er die?}

\stimme{Auf die Frage muss ich dir eine Antwort schuldig bleiben. Kümmere dich lieber um deine Freundin.}

Harry nickte und griff Ginny unter ihr Kinn, um sie vom Lesen abzuhalten. Seit sie hier waren, hatte sie ununterbrochen gelesen. Zuerst sanft, dann etwas wilder begann er sie zu küssen, bis sie schließlich eng umschlungen auf dem Sofa lagen. Harry unten und Ginny auf ihm.

\enquote{Ich liebe dich, Harry}, flüsterte Ginny.

\enquote{Ich liebe dich auch, Ginny}, antwortete Harry. \enquote{Wie viele Kinder wolltest du später mal?}

\enquote{Drei mindestens. Du weißt, ich komme aus einer kinderreichen Familie. Ich gebe mich mit einem nicht zufrieden.}

Harry grinste sie an und küsste sie dafür.

\trenn

Es war der Tag des letzten Quidditch-Spieles. Der Tag, an dem sie gegen Hufflepuff spielen würden. Dieses Jahr hatte Hufflepuff eine gute Saison hinter sich gebracht und mussten nur noch Gryffindor schlagen, damit sie den Quidditch-Pokal gewinnen würden. Harry war auf seinem Besen in der Luft und suchte das Spielfeld sowie den Himmel ab. Er zog kleine Kreise, um in Bewegung zu bleiben. In der gegnerischen Hälfte des Spiels blieb er irgendwann stehen und blickte wieder über das Feld.

\begin{rueckblick}
\enquote{Ist dir auch aufgefallen, dass Myrte in letzter Zeit fröhlicher wirkt?}, fragte Hermine ihn.

\enquote{Nein}, antwortete Harry, \enquote{nimm mich halt nächstes Mal mit, wenn du dort auf die Toilette gehst.}

Hermine streckte ihm die Zunge raus und verzog ihr Gesicht.

Er wusste genau, wovon sie sprach, aber er konnte ihr doch nicht sagen, dass er mit Myrte unglaublichen Sex gehabt hatte. Sie würde ihm entweder nicht glauben, oder aber würde ihn für krank halten, ihn sogar der Nekrophilie beschuldigen.
\end{rueckblick}

%Der Begriff Nekrophilie bezeichnet eine Sexualpräferenz, die auf Leichen gerichtet ist. Nekrophilie ist im ICD-10-Verzeichnis der psychischen Störungen unter „Sonstige Störungen der Sexualpräferenz“ (F65.8) als Paraphilie klassifiziert.

Dann sah er plötzlich den Schnatz und innerhalb weniger Augenblicke durchzuckten ihn verschiedene Gedanken. \gedanke{Wenn ich jetzt den Schnatz fange, dann haben wir sechsmal hintereinander den Quidditch-Pokal gewonnen. Sechsmal. Jedes Mal, seit ich im Team spiele. Hufflepuff hatte seit mehr als dreißig Jahren keinen Pokal mehr gewonnen. Wenn aber Lisa jetzt den Schnatz fangen würde, \abs}

Er folgte dem Schnatz mit seinen Augen. Die der gegnerischen Sucherin waren auf seine gerichtet. \gedanke{Los, komm schon Lisa, dreh dich um und fang ihn}, durchzog es Harry. \gedanke{Ich werde ihn jetzt noch nicht nehmen.} Seine Augen folgten unablässig dem Schnatz. Die Hufflepuff-Sucherin folgte für einen kurzen Moment seinem Blick, um ihm danach wieder in die Augen zu sehen. Dann registrierte sie, dass sie den Schnatz erblickt hatte. Blitzschnell drehte sie sich um und raste auf ihn zu. \gedanke{Na also, jetzt nur nichts anmerken lassen.} Harry gab Gas und flog ihr hinterher. Nun jagten beide den Schnatz.

\enquote{Und Harry Potter und Lisa Dervall jagen den Schnatz. Wollen wir hoffen, das der richtige Sucher ihn fängt}, trällerte Lee Jordan über das Spielfeld. Professor McGonagall warf ihm einen bösen Blick zu. \enquote{Ich habe keine Namen genannt}, sagte er daraufhin ebenso laut. \enquote{Jetzt hängt alles davon ab, wer den Schnatz fängt und somit den Quidditch-Pokal gewinnt. Hufflepuff \gst \extase{oder} \gst Gryffindor.} Das Gryffindor schrie er mit mehr Begeisterung heraus.

\gedanke{Kein Wunder, ist ja auch sein Haus}, dachte Harry. Beide Sucher waren jetzt auf gleicher Höhe. Vierzig Meter über dem Boden. Beide waren gleich auf. Kaum hatte Lisa den Schnatz in der Hand, wurde sie auch schon von einem Klatscher getroffen und fiel vom Besen Richtung Boden. Ohne lange zu überlegen, stürzte Harry ihr hinterher. Er bekam nicht mehr bewusst mit, wie Lee verkündete, dass Hufflepuff das Spiel gewann und somit auch den Quidditch-Pokal.

Er bekam sie an ihrer Quidditch-Robe zu packen und hielt wenige Zentimeter über dem Boden mit ihr in der Luft an. \enquote{Lisa, stütze dich am Boden ab, ich kann dich nicht mehr lange\abs} Lisa tat, was Harry ihr sagte. Kaum berührte sie den Boden, ließ Harry sie auch los und stieg von seinem Besen. Er ließ ihn senkrecht in der Luft schweben und kniete neben Lisa hin.

\enquote{Danke, Harry}, sagte sie glücklich und hielt den Schnatz in ihrer Hand. Sie öffnete sie leicht damit Harry ihn auch sehen konnte.

\enquote{Herzlichen Glückwunsch zum Sieg}, sagte Harry und strich ihr mit seiner Hand über den Rücken.

\enquote{Komm nach dem Spiel nochmal auf die Tribüne, ich möchte dir noch was sagen}, sagte Lisa.

Harry nickte. Mehr konnten sie nicht mehr besprechen, da der Trubel zu laut und die auf sie stürmenden Leute zu nah waren.

Sie hoben Lisa als ihre Gewinnerin in die Luft und trugen sie vom Spielfeld. Harry konnte nur innerlich lachen. Ihm war es egal, dass sie dieses Jahr den Pokal nicht gewannen. Sie hatten die letzten fünf Jahre McGonagall den Pokal für ihr Büro beschert. \gedanke{Würde sie es verstehen, wenn sie es herausfinden würde?}, fragte sich Harry. Er schob den Gedanken beiseite und wartete erst einmal, ob seine Kameraden überhaupt etwas davon mitbekamen.

\enquote{Harry, du brauchst dich nicht von uns zu trennen. Wir machen dir keinen Vorwurf, dass du den Schnatz nicht rechtzeitig geschnappt hast. Hufflepuff hatte dieses Mal einfach mehr Glück und sie haben den Pokal genauso verdient, wie jede andere Mannschaft}, sagte Ginny, die im Umkleideraum stand und ihre Jäger-Sachen verstaute. \enquote{Eigentlich ist es sogar gut. Das gibt den Hufflepuffs mal wieder richtig Mut. Vielleicht werden die Spiele gegen sie dann etwas herausfordernder. Bisher waren sie doch recht langweilig, meine ich.}

\enquote{Ich weiß}, antwortete Harry, \enquote{ich brauche nur kurz etwas Zeit für mich alleine. Es hat nichts mit dem Schnatz zu tun.} Er verließ die Umkleiden und setzte sich auf die Ränge des Spielfeldes. Er wartete auf Lisa.

Mittlerweile hatte sich das Feld geleert und nur noch Harry saß auf den Rängen. Er wartete, ließ sich verschiedenen Dinge durch den Kopf gehen und begann schließlich mit seinen Okklumentik-Übungen. \gedanke{Lisa wird schon noch kommen}, sagte sich Harry immer wieder. \gedanke{Hufflepuffs sind zuverlässig.} Nach zwanzig Minuten tauchte sie schließlich auf und setzte sich sehr nah neben Harry. Gerade so, dass sie ihn nicht berührte, er aber immer wieder ihren Atem spürte, wenn sie ihn ansah.

\enquote{Danke, Harry}, sagte sie sanft.

\enquote{Das hätte jeder gemacht}, antwortete Harry.

\enquote{Nein, das glaube ich nicht.}

\enquote{Doch, jeder andere hätte dich auch aufgefangen, als du auf den Boden zugerast bist.}

Lisa schaute ihn an. Er spürte ihren Atem auf seiner Backe. Sie rückte ein Stück näher an ihn heran. Nun berührte sie ihn wirklich.

\enquote{Ich meine nicht, dass du mich aufgefangen hast.} Erstaunt sah Harry jetzt Lisa an, ihr direkt in die Augen. Sie hatte eine schwarze Iris mit einer Spur gelb, wie Harry es noch nie gesehen hatte. \enquote{Ich meinte damit, dass du uns den Pokal geschenkt hast.}

Harry spielte den erstaunten. \enquote{Ich habe euch nicht\abs Du hast mich doch gesehen. Ich bin dir hinterher und war gleich auf mit dir.}

\enquote{Tu nicht so, Harry. Ich kann dich verstehen. Du hast den Schnatz schon lange vor mir gesehen. Du hättest einfach nur auf ihn zu fliegen müssen, hättest ihn weit vor mir haben können. Außerdem ist dein Besen schneller als meiner. Du hättest mich bei der Jagd überholen können. Aber du tatest es nicht.}

\enquote{Ich war in Gedanken. Ja, ich habe den Schnatz gesehen, war aber in meinen Gedanken vertieft}, sagte Harry entschuldigend.

\enquote{Aber du hättest mich überholen können und hast es trotzdem nicht getan.} Harry sah wieder geradeaus und sagte nichts mehr. \enquote{Harry?}

\enquote{Hm!}

\enquote{Darf ich dir dafür, als kleines Dankeschön, einen Kuss geben?}

Er sah Lisa wieder an, betrachtete ihr gewelltes kurzes dunkelbraunes Haar, welches sie zu einem Pferdeschwanz zusammengebunden hatte.

\enquote{Du hast mich doch schon geküsst. Erinnerst du dich nicht mehr daran?}, fragte Harry ausweichend.

\enquote{Das schon, aber da stand ich unter fremdem Einfluss. Da hattest du eine Wirkung an dir, die ja alle anderen Mädchen auch zu spüren bekamen. Und ich möchte wenigstens einmal einen Kuss von dir, ohne dass ich dazu\abs gezwungen werde.}

Harry betrachtete nun ihre wenigen Sommersprossen auf ihrer Nase und unter ihren Augen. Schließlich öffnete er leicht seinen Mund und legte seinen Kopf schief. Langsam kam sie ihm näher. Wenige Millimeter vor seinem Mund, er konnte ihre Lippen beim Sprechen fast schon spüren, sagte sie: \enquote{Danke.}

Dann küsste sie ihn kurz aber intensiv. Harrys ganzer Körper empfing nun Lisas Kuss. Dann sah sie ihn wieder an.

\enquote{Ich werde keinem davon erzählen}, sagte sie. \enquote{Nicht davon, dass du uns den Sieg einfach gemacht hast, und auch nicht von unserem Kuss.}

\enquote{Dein Kuss}, korrigierte sie Harry.

Lisa lächelte und ging. Harry saß noch eine Weile und betrachtete die Umgebung, als er Schritte neben sich hörte.

\enquote{Hi, Harry.}

\enquote{Hi, Ginny, bist du gerade gekommen?}

\enquote{Nein, ich war schon vor Lisa da und habe euch dann beobachtet.}

Harry sah jetzt mit Schrecken in Ginnys Gesicht, doch diese rutschte näher an ihn heran. \enquote{Ich weiß es}, sagte sie. Harry wurde ganz bleich im Gesicht. \enquote{Ich weiß, dass du Hufflepuff den Sieg fast geschenkt hast und dass Lisa sich bei dir\abs bedankt hatte.} Ihre letzten Worte, dachte Harry, waren nicht allzu freundlich. Um dem Ganzen die Spannung zu nehmen, küsste er seine Freundin. Diese stieß ihn jedoch nicht weg, wie Harry zuerst vermutete, sondern ließ ihn einfach.

\enquote{Interessanter Lippenbalsam, dass Lisa da verwendet}, sagte Ginny, als sie sich von Harry löste. Dann sagte sie in einem strengen Ton zu Harry: \enquote{Dass mir so etwas nicht noch einmal passiert, Harry James Potter.}

Harry schluckte. \enquote{Ja Ma'am}, sagte er schuldbewusst, im Klaren darüber, dass er von seiner Freundin erwischt worden war und senkte seinen Kopf.

\enquote{Ich hab dich viel zu gern, Harry}, sagte sie und zog ihn zu einem langen Kuss zu sich ran, \enquote{als dass ich dir dafür lange böse sein könnte. Aber mach so etwas ja nicht noch einmal}, ermahnte sie ihn wieder. \enquote{Ich kann ihre Beweggründe verstehen und auch, dass du sie gewähren hast lassen. Aber\abs}

Und Harry verstand. Sie schmusten noch eine Zeit lang, bis es schließlich beide zurück ins Schloss zog. Dort angekommen sah er Professor Elber, der sich mit Sirin unterhielt.

\enquote{Und, wie gefiel Ihnen Ihr Wohlfühltag?}, fragte er.

\enquote{Gut. Vor allem der Egalisierungs-Zauber hat es mir angetan. Bringen Sie uns den auch mal bei?}

\enquote{Was bezeichnen Sie als Egalisierungs-Zauber?}

\enquote{Na ja, den Zauber, der die Peinlichkeit und die Scham minderte. Wir waren dort alle ohne etwas an und hatten weder Scham noch etwas anderes. Das ist mir erst hinterher in den Sinn gekommen. Das ist ein toller Zauber.}

\enquote{Ach, diesen meinen Sie. Ja, der ist schon praktisch. Er liegt schon lange auf dem Raum. Jetzt, da Sie vier Kenntnisse davon haben, können Sie jederzeit dort hinein. Es wird Ihnen immer wieder möglich sein, dort zu baden, zu duschen, oder\abs andere Sachen zu machen.} Beim letzten Absatz grinste er sie leicht an.

Sirin wurde leicht rot.

\enquote{Was glauben Sie, was ich in diesem Raum früher für Mädchen\abs}, wieder grinste er leicht. Sirin wurde knallrot bei diesem Gedanken. Jetzt lachte er aus voller Kehle. \enquote{Sie müssen aufpassen, dass Sie nicht zu früh in eine peinliche Situation kommen. Vierundzwanzig Stunden sollten sonst schon dazwischen liegen. Denn sonst wird es umso peinlicher. Beim ersten Mal braucht es bei manchen noch etwas länger.} Er strich ihr sanft über die Wange. \enquote{Aber je peinlicher die Situation, desto schneller vergeht sie und desto weniger wird es die nächsten Male. Es tritt ein Gewöhnungseffekt ein.}

Da hatte Harry einen Gedanken, den er beizeiten mit Ginny ausleben wollte.

\enquote{Aber wie kommen Sie darauf, dass ich ihn kennen würde?}

\enquote{Was meinen Sie?}

\enquote{Den Zauber. Sie fragten mich, ob ich Ihnen den beibringen würde.}

\enquote{Na ja}, sagte Sirin, \enquote{ich schätze Sie so ein, dass Sie solche Sachen wissen.}

\enquote{Haben Sie Zeit?}, fragte er.

\enquote{Ja, warum?}

\enquote{Dann kommen Sie mit. Dumbledore wollte den Zauber auch lernen. Zwar aus einem anderen Grund, aber es interessiert ihn. Wissen Sie, es ist leichter einen Zauber von jemandem zu lernen, als aus einem Buch.}

Damit verschwanden beide in einem der Gänge des Schlosses.

Am nächsten Tag waren Prüfungen. Heute hatte die gesamte Schule Prüfung im Fach \VgddK.

\gedanke{Seine letzte Tätigkeit als Lehrer}, ging Harry durch den Kopf.

Die gesamte Schule versammelte sich auf einer großen Freifläche und starrte auf eine große milchige Halbkuppel, die auf einer Backsteinmauer mit einer Höhe von einem halben Meter stand. In der Mitte war ein Durchgang aus Holz zu sehen, dessen Mitte komplett dunkel war. Es war fast so, als ob da nichts wäre. Es war nicht einfach schwarz, da war einfach nichts, so hatte man den Eindruck.

\enquote{Ich mache es kurz}, sagte der Prüfer, der die Prüfung abnahm. \enquote{In diese Kuppel gehen Sie mit etwa drei Sekunden Abstand. Keine Sorge, Sie passen alle da rein. Im Inneren werden Ihnen verschiedene Aufgaben gestellt werden, die Sie mehr oder weniger erfolgreich absolvieren werden. Mit der Zeit wird es immer schwieriger, bis irgendwann der Punkt erreicht ist, an dem Sie aufgeben müssen. Es wird Ihr Fortschritt innerhalb der Kuppel bewertet, sowie Ihre Lösungen, beziehungsweise Lösungsansätze.} Er pausierte kurz. \enquote{Dann fangen wir mit der siebten Jahrgangsstufe an. Bitte treten Sie vor, drei Schritte Abstand und laufen Sie langsam in das Innere der Kuppel. Sie werden einander bis zum Ausgang nicht mehr begegnen.}

Die Siebtklässler nahmen ihre Stellung ein und schritten nacheinander durch den Bogen in das Innere. Als Nächstes war Harrys Jahrgangsstufe dran.

Mit den Dementoren am Eingang war er gleich fertig. Er schickte ihnen einen Patronus entgegen. Dann musste er durch ein Gewässer tauchen, da ihm ein unbekanntes Feld den direkten Weg mit dem Boot verwehrte. Harry überlegte kurz und belegte seine Kleidung mit einem wasserabweisenden Zauber und tauchte. Doch er hatte nicht mit Wasserlebewesen gerechnet, die ihn unter Wasser behalten wollten und nach unten zogen. Es dauerte eine Weile, bis er wieder wusste, wie er sich dagegen wehren konnte. Einige konnte er durch ein paar Brandflecken mit seinem Zauberstab auf deren Haut abwehren, anderen versuchte er die Finger zu brechen, damit sie von ihm abließen.

So kämpfte er sich durch die schwerer werdenden Aufgaben.

Dann wurde es zunehmend kälter. Harry zauberte sich eine warme Jacke herbei und zog sie an, doch sie spendete ihre Wärme nicht lange. Auch Wärmezauber schienen bei weitem nicht so effektiv zu sein. \gedanke{Wenn ich doch nur dieses ewige Feuer richtig könnte}, ging ihm durch den Kopf. Dann kam ihm etwas in den Sinn. \gedanke{Lebendiges Feuer. Das ist doch die gleiche Art der Magie. Deshalb haben die Siebtklässler das durchgenommen. Es wird Teil ihrer Prüfung sein; durch die Kälte mit diesem Feuer zu kommen.} Harry grinste in sich hinein. Leicht fröstelnd ließ er die kalte Passage hinter sich und widmete sich der nächsten Aufgabe. Er musste durch einen Dschungel. Er hatte während des Schuljahres viele Wesen durchgenommen, die im Dschungel wohnten, davon viele magische Schlangenarten, und auch diese komischen Wesen, die wie Äste aussahen.

Und kürzlich hatten sie erst diese blauen kleinen Wesen durchgenommen: Medusoner wurden sie genannt. Sie waren normalerweise harmlos, doch wenn man ihr Revier ohne Einladung betrat, konnten sie ungemütlich werden und angreifen. Harry versuchte sich zu erinnern, woran man deren Reviergrenzen erkannte. Als es ihm wieder einfiel, bemerkte er, dass er vor wenigen Schritten an einem Hinweis vorbeigegangen war.

Er blieb stehen und lief die wenigen Schritte rückwärts und betrachtete den Hinweis, der die Form eines kunstvoll aus Lianen gewundenen Pfahls hatte. Harry betrachtete den Pfahl eine Weile und überlegte, in welche Richtung er am besten gehen sollte. Beide Wege der Abzweigung, die nur zehn Meter zurücklag, konnten ein Umweg von mehreren Kilometern sein. \gedanke{Einer der beiden ist bestimmt doppelt so lang wie der andere}, dachte Harry.

Dann hörte er etwas rascheln. Er drehte seinen Kopf in die Richtung, aus der das Geräusch kam und fragte in Richtung des Busches: \enquote{Sind sie ein Medusoner? Wenn ja, hätte ich gern die Erlaubnis, das Gebiet ihres Stammes zu durchqueren.}

Ein blauer Kopf mit angewachsenem Lamellenpilz-artigem Hut schaute hervor.

Harry sah das Wesen und stellte sich vor. \enquote{Mein Name ist Harry Potter und ich erbitte die Erlaubnis, dieses Gebiet zu durchqueren.}

\enquote{Was bekommen wir dafür?}

\enquote{Was hättet ihr denn gern?}

\enquote{Basiliskenschuppen.}

\enquote{Habe ich nicht bei mir. Wie wäre es mit etwas zu essen?}

\enquote{Ja, das geht auch}, sagte der Medusoner, als wäre es nichts Besonderes so etwas Wertvolles gegen etwas zu Essen einzutauschen.

\enquote{Was hätten Sie denn gern?}

\enquote{Schnecken und Früchte.}

Harry zauberte einen großen Früchtekorb herbei, den er mit einem Zauber wenige Millimeter über dem Boden schweben ließ.

Ein zweiter Kopf kam aus dem Gebüsch heraus. Die beiden blauen Wesen schauten sich an und nickten einander zu. Einer kam auf den Früchtekorb zu und nahm ihn mit. Der andere kam auf Harry zu und verlangte, dass er ihn während der Durchquerung des Stammesgebietes zu tragen habe, damit keiner angreifen würde.

Harry akzeptierte die Bedingung und nahm das blaue Wesen auf die Schulter.

Während der Durchquerung entwickelte sich eine kleine Unterhaltung zwischen den beiden. Er fragte ihn zu seiner Prüfung aus. Doch mehr als ein: \enquote{Wir melden das Verhalten der Prüflinge. Alles andere liegt bei Ihrem Prüfer}, bekam er nicht zu hören.

Nachdem Harry das Stammesgebiet durchquert hatte, setzte er seinen kleinen Begleiter ab, verabschiedete sich von ihm und setzte seine Prüfung fort. Dann kam er an ein Labyrinth. Er fühlte sich an das trimagische Turnier erinnert. Inzwischen kannte er mehr Zauber, so auch einen Zauber, den er in einem Buch während seiner Strafarbeit in der Bibliothek gefunden hatte. Er wandte ihn an und wartete ein paar Minuten, bis sich der Zauber entwickelt hatte. Dann sah er an jeder Abzweigung und Kreuzung unterschiedlich dicke Linien. Harry folgte der jeweils dicksten Linie und fand so den Weg aus dem Labyrinth. Der Zauber war ein sogenannter Schleimpilzzauber. Diese Pilze konnten sehr schnell und effektiv den besten Weg zu einer Nahrungsquelle finden. Der Zauber funktionierte ähnlich. Er fand den schnellsten Weg zum Ausgang.

Am Ende des Labyrinthes stand er vor einer Felswand. Er sah sich um, entdeckte aber keinen Weg, der drumherum führte, er musste also klettern. Doch eine Kletterausrüstung konnte er sich nicht herbeizaubern. Auch jeder andere Zauber versagte. Also musste er von Hand klettern und machte sich an den Aufstieg. Doch kurz vor dem Ziel rutschte er ab und fiel.

Er landete weich. Als er aufstand, war kein Labyrinth mehr vorhanden. Nur eine weite Wiese mit einem Torbogen. Als er näher kam, erklang eine Stimme. \stimme{Prüfung abgeschlossen. Sie erfahren Ihr Ergebnis draußen.}

Harry trat durch den Torbogen und fand sich außerhalb der Kuppel wieder. Einige seiner Mitschüler, die nach ihm hineingegangen waren, sah er bereits. Sie mussten schon vorher ihre Grenzen erreicht haben.

Harry hatte ein Ohnegleichen in diesem Fach und war vollauf mit sich zufrieden.

\stimme{Warum hast du mich bei dem Feuer nicht gefragt, wie der Zauber geht, Harry?}, fragte ihn Salazar in seinem Geist.

\gedanke{Das wäre den anderen gegenüber nicht fair gewesen.}

\stimme{Nicht fair? Junge, das ist ein Teil von dir. Dieses Wissen hast du bereits. Hättest du dich etwas konzentriert, dann wäre es dir eingefallen.}

\gedanke{Warum sagst du mir das erst jetzt?}

\stimme{Das weißt du schon länger. Das habe ich dir schon früher gesagt.}

\gedanke{Tut mir leid, habe ich wohl vergessen.}

In einem unbeobachteten Moment auf dem Rückweg zum Schloss ließ er, nachdem er angestrengt nachgedacht hatte, eine kleine Flamme auf seiner Handfläche erscheinen, sie dann unter seinem Ärmel nach oben gleiten und dann ganz zart über seinen Brustkorb und den Rücken verteilen. Er ließ sich etwas wärmen, bis der Zauber endete. Er hielt nicht lang, aber er schaffte es doch.

\trenn

Harry hatte seine letzte Prüfung hinter sich gebracht und lief gerade in die Große Halle, um zu Abend zu Essen. Mitten im Raum schwebte eine große Version des komischen Würfels, den sie zum Öffnen bekommen hatten. Dahinter der gleiche Würfel, nur ohne Farbe. Und wiederum dahinter die geöffnete Version. Die pyramidalen Ecken mit den dreieckigen Grundflächen hatten sich um 60 Grad gedreht und das Innere der Kreise an den sechs Seiten war entfernt worden. Im Inneren schwebte eine kleine Kugel, die das bläuliche Licht ausstrahlte. Die Große Halle war bereits zu drei Vierteln gefüllt. Harry sah sich um. Ihm fiel nur ein Würfel auf, der in der Luft schwebte und normale Größe hatte. Er war geöffnet. Harry musste ein paar Schritte weitergehen, um zu sehen, über wem der Würfel schwebte.

\gedanke{Bringen Sie am Abend des letzten Prüfungstages Ihren Würfel mit. Egal ob Sie ihn öffnen konnten, oder nicht}, hatte Professor Elber sie angewiesen.

Von seiner neuen Position aus konnte Harry erkennen, wer unter dem Würfel saß. Es war Luna. Sofort ärgerte er sich, dass ihm Luna nichts darüber erzählt hatte. Doch im nächsten Moment war ihm klar, dass das doch Betrug gewesen wäre. Er griff in seine Tasche und besah sich seinen Würfel.

\gedanke{Konzentrieren Sie sich auf den Würfel, bis Sie ihn öffnen können.}

\gedanke{Warum bin ich nicht früher darauf gekommen}, dachte Harry. Plötzlich war alles klar. Das ganze Jahr über hatten sie darauf hin gearbeitet. Sie übten Zauberstabslose Magie. Sie ließen zwar nur ihre Zauberstäbe und Besen auf sich zukommen, aber das war ein Schritt auf dem Weg. Und vor allem, was Professor Elber sagte. \gedanke{Konzentrieren Sie sich auf den Würfel, bis Sie ihn öffnen können.} Es war so einfach. Er warf den Würfel vor sich in die Luft und konzentrierte sich darauf, dass er schweben blieb. Der Würfel fiel noch einige Zentimeter runter, bis ihn Harry vor seinen Augen stabilisieren konnte. Dann schloss er seine Augen und stellte sich sehr genau vor, wie er den Würfel öffnen würde. Er öffnete die Augen und sah, wie die pyramidalen Ecken des Würfels abhoben, das Innere der Kreise langsam verschwand und die Ecken sich langsam zu drehen begannen. Dann kamen die Ecken zurück und rasteten hörbar ein.

Mit zufriedenem Gesichtsausdruck setzte er sich zum Abendessen hin. Der Würfel folgte ihm in sicherem Abstand über seinem Kopf. Nun schwebten zwei kleine Würfel in der Großen Halle. Die großen Würfel drehten sich immer noch langsam unter der Wolken-behangenen verzauberten Decke. Er fing Dumbledores Blick auf, dessen Würfel verschlossen, aber von Farbe befreit vor ihm lag. Harry musste sich beherrschen, seinen Kürbissaft nicht über dem Tisch zu verteilen. Er hätte erwartet, dass zumindest Dumbledore von allen Lehrern es schaffen würde. Denn inzwischen lagen vor sämtlichen Lehrern ebenfalls solche Würfel.

Doch Harry war nicht vorbereitet, nicht auf das, was gleich passieren würde, denn er verteilte seinen Kürbissaft dennoch über den Tisch. Zum Glück saß ihm niemand gegenüber, als Professor Elber die Halle betrat, lief er bis zum oberen Ende der Tische. Er blieb stehen und schaute auf Dumbledores Würfel. Dumbledore fing seinen Blick auf und fuhr danach mit seiner Hand vor dem Würfel vorbei. Dieser hob leicht vom Tisch ab und die Ecken entfernten sich, das Innere der Kreise verschwand und die Ecken rasteten gedreht wieder ein. Danach schwebte der Würfel wenige Zentimeter über dem Tisch.

Harry hustete, als er sah, wie beiläufig Dumbledore den Würfel öffnete. Harry wusste nicht mehr, was er während seines Hustenanfalls herausbrachte, aber kurz darauf erschien Kreacher. Er sah Harry an und danach die Sauerei, die er hinterlassen hatte. Er verschwand, um kurz darauf wiederzukommen; mit einem Putzlappen in der Hand. In Windeseile wischte er den verschütteten Kürbissaft auf und erneuerte die angespuckten Speisen und Getränke. Danach verschwand er wieder. Harry war froh, dass er sich mit Kreacher nun besser verstand. Seine Hetzreden waren nicht mehr zu hören. Er war ihm gegenüber jetzt vollständig loyal geworden. Auch nannte er ihn seit geraumer Zeit nicht mehr Meister, sondern nur noch \accentuate{Sir Harry.}

Harry sah sich immer wieder um, doch außer seinem Würfel und dem von Luna, war keiner geöffnet. Abgesehen von dem Würfel Dumbledores. Die Tore der Großen Halle schlossen sich, nachdem der letzte Schüler die Halle betreten hatten und signalisierte somit, dass jetzt alle Anwesend seien. Nachdem alle gegessen und getrunken hatten, stand Dumbledore auf und trat hinter sein Pult. Die Flügel der Eule breiteten sich aus und so als ob jemand einen Schweigezauber auf den Saal gelegt hätte, verstummte das Gemurmel.

\enquote{Ein weiteres Jahr haben wir jetzt hinter uns gelassen. Ein Jahr voller Aufregung und ein Jahr des Lernens. Obwohl die Bedrohung dort draußen beständig gewachsen ist, haben wir uns bislang gut geschlagen. Ihr seht hier über euch an der Decke drei Würfel. Diese Würfel mussten einige von euch bis zum Jahresende öffnen. Aber ich denke, dass dieses Rätselfieber sich bereits ausgebreitet und auch uns Lehrer erwischt hat. Ich muss zugeben, es war nicht leicht, dahinterzukommen, wie diese Würfel zu öffnen sind.} Er ließ seinen Blick durch die Halle schweifen. \enquote{Wie ich sehe, haben es nur zwei Schüler geschafft, ihre Würfel zu öffnen.} Er drehte sich um und blickte zum Lehrertisch, um sich danach wieder den Schülern zuzuwenden. \enquote{Und kein Lehrer.} Er grinste leicht. \enquote{Ich möchte nun Professor Elber bitten!}

Professor Elber stand auf und während er auf das Pult zulief, trat Dumbledore zurück. \enquote{Danke, Albus}, sagte Professor Elber und stellte sich vor das Pult mit der Eule.

\enquote{Wie Professor Dumbledore bereits sagte, sind es nur zwei Schüler, die begriffen hatten, was ich damit meinte, als ich euch sagte: \enquote{Konzentrieren Sie sich auf den Würfel, bis Sie ihn öffnen können.} Dies gilt noch immer. Aber um Ihnen den ersten Schritt zu erleichtern: Entfernen Sie einfach mal die Farbschicht von Ihrem Würfel.}

Jetzt erfüllte allgemeines Rascheln die Große Halle und viele kleine Wolken aus Schwarz verpufften im Nichts.

\enquote{Konzentrieren Sie sich auf den Würfel, bis Sie ihn öffnen können}, sagte Professor Elber erneut. Er zog einen Würfel aus seiner Tasche. Die Farbe war bereits entfernt. Er vergrößerte ihn mit seinem Zauberstab und lies ihn vor sich schweben.

\enquote{Stellen Sie sich vor Ihrem geistigen Auge den Würfel vor. Schauen Sie ihn sich genau an. Dann schließen Sie die Augen und stellen sich vor, dass der Würfel sich öffnen würde.} Er schloss die Augen und der Würfel begann sich zu öffnen. Dann öffnete er wieder seine Augen und meinte: \enquote{Bitte, versuchen Sie es alle. Sollte es nicht klappen, haben Sie die Ferien über Zeit zum Üben. Den Würfel können Sie behalten. Sie können ihn jederzeit öffnen und schließen. Sogar kleine Dinge lassen sich in ihm verstecken, wenn sie die Lichtkugel entfernen.} Er drehte sich wieder um, lief zurück und setzte sich.

Harry war erstaunt, wie um ihn herum viele Würfel langsam zu schweben begannen, teilweise zusammenstießen, aber sich dennoch öffneten. Es schafften nicht alle. Kaum einer aus den ersten drei Schuljahren hatte seinen Würfel öffnen können. Er lächelte Tamara an, die erschöpft, aber glücklich aussah, als ihr Würfel vor ihr schwebte. \enquote{Eine echte Malfoy}, sagte er zu ihr.

Sie schaute an Harry vorbei zu ihrem Bruder und meinte: \enquote{Draco hat es auch geschafft.}

\enquote{Die unteren Jahrgangsstufen könnten damit eventuell Probleme haben. Lassen Sie sich nicht entmutigen. Versuchen Sie es beständig}, sagte Professor Dumbledore und ermutigte die Schüler.

\trenn

\enquote{Warum sind wir hier, Harry?}, fragte Ginny, als sie wieder diesen Gang mit den angestaubten Rüstungen entlang liefen.

Harry hatte mittlerweile geübt, den \spruch{Protego}-Zauber ungesagt und ohne Zauberstab auszuführen. Er antwortete nicht gleich und sagte nach einigen Metern: \enquote{Warte es ab, Ginny. Hast du deine Badesachen?}

Sie bejahte und lief neben ihm her. Sie nahm seine Hand in ihre und beide liefen an den Rüstungen vorbei. Die wenigen Elfen, die sie trafen, verbeugten sich kurz und arbeiteten dann weiter. Sie verschwanden nicht mehr nach Harrys Rede.

\gedanke{Das könnte nächstes Jahr interessant werden}, dachte sich Harry, \gedanke{wenn die Elfen im Gemeinschaftsraum sind, wenn ich alleine noch arbeite, oder sonst irgendwas tue.}

Sie waren mittlerweile bei dem Porträt angekommen. Harry sah hinauf zur blauen Blume und dachte: \gedanke{Protego}. Sie hörten ein leises \geraeusch{Klick}-Gearäusch und der Bilderrahmen sprang wenige Millimeter vor.

Harry öffnete den Rahmen und ließ Ginny den Vortritt. Sie lief voraus und Harry folgte ihr. Als sie um die Ecke sah, blickte sie sich erst einmal vor Erstaunen um. Dann drehte sie sich Richtung Ausgang, öffnete einen der Spinde und zog ihre Kleidung aus; auch ihren Badeanzug. Dann lief sie in das Becken und setzte sich so hin, dass sie jeder sehen konnte, der auf das Becken zuging.

Harry hatte sich auf diese Situation vorbereitet und zog sich ebenfalls aus. Dann trat er um die Ecke, lief auf das Becken zu und verschwand ebenfalls bis zum Bauchnabel im Wasser. Er setzte sich Ginny gegenüber auf den Sitz unterhalb der Wasseroberfläche und sah Ginny an.

Beide betrachteten sich nun von Kopf bis Fuß. \gedanke{Selbst wenn sie ihren Badeanzug anbehalten hätte, würde es mir nichts ausmachen.}

Ginny bewegte sich nicht, da sie Harrys Körper betrachten wollte. Harry tat es ihr gleich.

Dann schwamm sie zu ihm herüber, setzte sich auf seine Knie und legte ihre Stirn gegen seine. \enquote{Warum sind wir hier?}, fragte sie.

\enquote{Ich muss mit dir reden, Ginny.}

\enquote{So?}, fragte sie ihn.

\enquote{Du hast angefangen. \gst Ja, so.}

Sie lächelte leicht. \enquote{Was sagte deine Freundin dazu?}

\enquote{Ich hab mich doch von Pan\aabs Ähh\abs Es scheint ihr nichts auszumachen. Zumindest hat sie sich nicht negativ geäußert.}

\enquote{Hast du sie gefragt?}, kam jetzt mit etwas Nachdruck.

\enquote{Kann ich kurz machen. \gst Ginny, hast du was dagegen?}

Statt einer Antwort küsste sie ihn.

\enquote{Tut mir leid, aber ich habe keine eindeutige Antwort erhalten. Sie hat mir nonverbal mit einem Kuss geantwortet. Reicht dir das?}

Ginny nickte leicht. \enquote{Wirst du dich von ihr trennen?}, fragte sie nun.

\enquote{Weißt du noch, was ich dir in den Ferien gesagt habe?}

Ginny ging das Bild auf Harrys Geburtstagsfeier durch den Kopf. Ihre Augen wurden leicht feucht.

Harry erschrak und küsste ihre feuchten Augen. \enquote{Weißt du, das hat sich grundlegend geändert. Ich empfinde mittlerweile mehr für dich. \gst Viel mehr.}

Jetzt glitt ein Lächeln über Ginnys Gesicht. \enquote{Das spüre ich}, antwortete sie ihm.

Erst jetzt bemerkte Harry seine Erektion, die gegen Ginnys Pobacken und die äußeren Schamlippen drückten. Er lächelte mit einem verführerischen Gesichtsausdruck zurück und meinte: \enquote{Du bist aber auch schon ganz feucht.}

Sie patschte ihm mit der flachen Hand auf die Brust, was einen Wasserschwall auf beide Gesichter spritzte. Harry zog sie näher zu sich heran und küsste sie ausgiebig. Gern ließ sich Ginny diese Behandlung gefallen, spürte den intensiven Druck zwischen ihren Beinen, seine Brust auf ihrer, seine Hände an ihrem Rücken und seine Lippen auf den ihren, ihrem Kinn, dem Hals und der Stirn.

\enquote{Ich bin noch nicht bereit dafür, Harry}, sagte Ginny.

Harry verstand und sagte: \enquote{Ich weiß, deshalb habe ich dich nicht hierher geholt. Sondern für das, was wir gemacht haben.} Er überlegte kurz. \enquote{Ich habe mir vielleicht etwas weniger vorgestellt, aber ich richte mich ganz nach dir.}

\enquote{Ich habe Angst, Harry. Du bist schon\abs hast schon mit so vielen\abs}

\enquote{Ginny, ich wünschte, ich könnte diese Erfahrungen verdrängen\abs Aber ich glaube kaum, dass sich das\abs}

Sie küsste ihn wieder. \enquote{Du bist so lieb.}

\gedanke{Außer ich verwende Okklumentik. Aber wie lange kann ich dieses Wissen unterdrücken? Und wenn ich die Konzentration verliere, dann bricht es über mich herein und ich könnte mit Ginny etwas machen, was ich hinterher bereue}, ging ihm durch den Kopf. \enquote{Aber ich werde dir, wenn du so weit bist, jeden Wunsch erfüllen, dass unser gemeinsames erstes Mal für dich unvergesslich wird.}

\enquote{Woher willst du\abs}

\enquote{Ich werde dir keine Namen nennen, Ginny.}

Sie schmollte und schaute ihn mit Dackelaugen an.

\enquote{Es waren auch Slytherins dabei. Mehr sage ich nicht}, lenkte Harry ab.

\enquote{Lehnst du dich zurück, Harry?}, forderte sie von ihm.

Harry rutschte mit seinem Becken etwas nach vorne, damit er schräg mit durchhängendem Rücken auf dem Steinsitz saß. Ginny küsste ihn noch einmal. Dann richtete sie ihren Oberkörper auf, stieg links und rechts von ihm auf die kleine Mauer rings um das Becken und stieg über ihn hinweg. Harry blieb die Luft bei diesem Anblick weg. Er saß mehrere Sekunden starr da und sog das Bild in sich auf, das er eben gesehen hatte. Es brannte sich für immer in sein Gedächtnis ein.

Dann stand er auf und folgte Ginny. Nachdem sich beide angezogen hatten, gingen sie Hand in Hand und glücklich zurück zum Gemeinschaftsraum.

\trenn

Die Prüfungszeit war vorbei und es waren nur noch wenige Tage, bis der Hogwarts-Express die Schüler nach Hause bringen konnte. Alle saßen gemeinsam in der Großen Halle und waren beim Essen. Nach einer Weile stand Professor McGonagall auf und lief hinter dem Tisch herum und zog dann mit Professor Snape, der neben Professor Elber saß, diesen hoch. Dumbledore schwang seinen Zauberstab und in der Mitte erschien ein kleiner Stuhl mit dem sprechenden Hut. Die drei waren auf dem Weg zu ihm, als ihn Professor Elber entdeckte und begann sich zu wehren. \enquote{Nein, nein} und stemmte sich gegen sie um zu verhindern, dass sie ihn weiter Richtung Stuhl zogen. Doch es half nichts. Unaufhaltsam zerrten Snape und McGonagall an ihm, bis er auf dem Stuhl saß. Ergeben legte er seine Hände gegeneinander und klemmte sie mit gesenktem Kopf zwischen seine Beine. Harry war der Meinung, in seinen Augen Tränenflüssigkeit zu sehen. Dann hob McGonagall den Hut vom Boden auf und Professor Elber wollte schon wieder aufstehen, als er von Professor Snape wieder heruntergedrückt wurde. Dann setze sie Elber den Hut auf.

Sofort spürte Harry eine Fröhlichkeit in sich aufsteigen und eine innere Zufriedenheit machte sich in ihm breit. Doch den anderen ging es scheinbar genauso. Ginny nahm direkt seine Hand in ihre und warf ihm einen kurzen aber intensiven Blick zu.

Dem Hut entwich nur ein \enquote{Oh}, als er den Kopf unter sich spürte. Professor Elber nahm die Hände vors Gesicht und senkte es noch weiter.

Stille.

Der Hut sagte eine Weile nichts. Langsam konnte Harry Flüssigkeit zwischen den Händen seines Professors feststellen. Nach und nach tropfte sie an ihm herunter auf den Boden. Dann sagte der sprechende Hit nur noch ein Wort: \enquote{Magistri.} Die Halle wurde noch stiller.

Professor Elber sprang vom Stuhl auf und verließ mit schnellen Schritten und Tränen überströmtem Gesicht die Große Halle. Währenddessen warf er den Hut von seinem Kopf, direkt auf den Boden. Als er die Klinke der großen Holztür in der Hand hielt, drehte er sich noch einmal kurz um und schrie in die Große Halle. \enquote{Ich hoffe, ihr seid jetzt zufrieden.} Dann öffnete er die Tür und schlug sie beim hinausgehen mit voller Wucht zu.

Schlagartig sank die Stimmung in der Großen Halle.

\enquote{Da haben wir wohl übertrieben}, murmelte Professor McGonagall.

Harry fühlte sich plötzlich schlecht. Er sah sich in der Halle um und merkte, dass keiner der Anwesenden mehr sich seinem Essen widmete. McGonagall ging mit gesenktem Blick zu ihrem Platz zurück und setzte sich ebenso wie Snape. Nur mit dem Unterschied, dass man dessen Gefühle nicht lesen konnte. Keiner der Lehrer aß mehr, oder sah durch die Halle. Keiner rührte mehr sein Essen an.  Alle sahen nur mit gesenktem Kopf auf den Teller vor sich.

Die Stimmung war auf einem Tiefpunkt, wie ihn Harry noch nie erlebt hatte.

Durch Harrys Kopf klangen noch einmal die Worte des sprechenden Hutes. \accentuate{Oh.} Und im selben Abstand wie der Hut brauchte: \accentuate{Magistri.}

Schließlich stand er auf und verließ die Große Halle, als er noch einmal kurz seinen Blick schweifen ließ. Dracos Blick traf seinen, und er stand auf um nun ebenfalls die Halle zu verlassen. Sein Blick blieb an Luna hängen, die als Einzige versuchte, etwas zu essen. Langsam, aber dennoch unsicher, schob sie einen Bissen nach dem anderen in sich hinein. Harry lächelte leicht in sich hinein und verließ neben Draco die Halle. Schweigend gingen sie nebeneinander her, bis sich ihre Wege trennten. Sie blieben kurz stehen und nickten einander zu, bevor jeder seinen Weg ging.

Harry hatte sich bereits für die Nacht zurechtgemacht und sah durch das Fenster in die dunkle Nacht. Er hörte Donner grollen und dachte: \gedanke{Klasse, das wird eine tolle Nacht.}

Die Tür öffnete sich und ein Blitz erhellte die Nacht. Er schlug in der Nähe des Schlosses ein. Harry zuckte kurz zusammen. Ron und Dean, sowie Seamus und Neville kamen herein und bereiteten sich ebenfalls für die Nacht vor. Als die vier wieder hereinkamen, lag Harry schon im Bett. Die Hände hinter seinem Kopf gelegt, starrte er an die Decke. Das Gewitter war in der Zwischenzeit etwas lauter geworden. Harry hatte das Gefühl, mit jemandem reden zu wollen. \enquote{Gute Nacht}, sagte er zu seinen Freunden und schloss mit einer gelangweilten Geste die Vorhänge zu. \enquote{Wow, Harry}, kam von der anderen Seite durch den schweren Stoff. \enquote{Das muss ich auch mal versuchen.}

Harry grinste in sich hinein. Er nahm kurz sein Amulett in die Hand und schloss für einen kurzen Moment die Augen. Als er sie wieder geöffnet hatte, schwebte Salazar vor ihm. Er verwickelte ihn in eine gedachte Diskussion.

\gedanke{Ich brauche deinen Rat, Salazar.}

\stimme{Wobei, Harry?}

\gedanke{Es geht um einen meiner Lehrer und sein komisches Verhalten beim Essen.}

\stimme{Hat er sich beklagt, dass es ihm nicht schmeckt?}, gluckste Salazar.

\gedanke{Nein. Professor McGonagall und Professor Snape haben ihn zu einem Stuhl gezogen und ihm den sprechenden Hut aufgesetzt.}

Salazar zuckte kaum merkbar zusammen. \stimme{Was ist passiert?}, fragte er so unbeeindruckt, wie er konnte.

\gedanke{Zuerst hat der Hut nur \accentuate{Oh} gesagt. Und dann, nach einer Weile: \accentuate{Magistri}.} Wieder zuckte Salazar unmerklich zusammen. \gedanke{Was hältst du davon?}, fragte ihn Harry nun.

Salazar überlegte kurz. \stimme{Nun}, sprach er, \stimme{ich denke, der sprechende Hut hat etwas Besonderes in ihm erkannt.}

Ein lauter Blitz schlug jetzt in der Wand vor Harry ein er fuhr hoch. Mit dem Gesicht mitten in Salazar lauschte er in die dunkle Nacht hinein. Als ihm wieder einfiel, wo er seinen Kopf hatte, legte er sich wieder hastig hin.

\gedanke{Was denkst du?}, fragte Harry weiter.

\stimme{Er dürfte wohl etwas Besonderes sein. Der Hut wollte ihm wohl sagen, dass er sich sein Haus aussuchen konnte.}

\gedanke{Er ist aber mit Tränen in den Augen sofort rausgerannt, als der Hut \accentuate{Magistri} zu ihm sagte.}

\stimme{Er war wohl zu geschockt}, sagte Salazar und sah dabei aber nicht überzeugend aus.

\gedanke{Ich denke eher, er hatte eine Ahnung. Denn als er da saß, nahm er direkt seine Hände zwischen seine Beine und senkte seinen Kopf. Und als der Hut dann \accentuate{Oh} sagte, schlug er seine Hände vors Gesicht und fing an zu weinen. \gst Ist so etwas schon einmal passiert?}

\gedanke{Soweit ich weiß, einmal. Warte kurz.} Salazar verschwand und kam eine Minute später wieder. Genau, als wieder ein Blitz ins Schloss einschlug. Harry zuckte wieder zusammen. Dann hörte er ein Rascheln.

\stimme{Dies ist erst das zweite Mal}, schloss Salazar und verschwand.

Die Vorhänge um Harrys Bett wurden aufgezogen und Neville stand davor. Er sah ihn an und meinte dann: \enquote{Ach, du kannst also auch nicht schlafen?}

\enquote{Nein}, gab Harry matt zurück und stand auf. Zusammen gingen sie zu den Anderen und dann zusammen in den Gemeinschaftsraum. Dort war es brechend voll, weil keiner der Gryffindors schlafen konnte. Und wieder ging ein greller Blitz vor dem Turm der Gryffindors nieder und schlug in die Schlossmauern ein. Das Donnern und Grollen wurde bedrohlicher. \gedanke{An Schlaf ist wohl heute nicht mehr zu denken.} Ginny drückte sich durch die Masse, um sich an Harry zu kuscheln. Er nahm sie in seine Arme und gab ihr einen flüchtigen Kuss. Doch sie zog ihn zu sich und so standen sie da, bis sich Ron neben ihnen räusperte. Unwillig löste sich Ginny von Harry und legte ihren Kopf an seine Schulter.

\enquote{Ron?}, fragte Harry.

\enquote{Ja?}

\enquote{Holst du mit Hermine zusammen McGonagall?}

\enquote{Warum?}

\enquote{Weil du und Hermine Vertrauensschüler seid und keiner von uns schlafen kann. Und wenn wir versuchen, hier einen Schutzzauber auf die Wände zu legen, damit es nicht mehr so laut ist, dann kann es sein, dass er entweder nicht richtig ist, oder zu sonst irgendwelchen Problemen führen kann.}

\enquote{Ah!} Ron sah dabei nicht sehr intelligent aus.

Ron suchte in dem Gewirr Hermine und begab sich dann durch das Porträt-Loch nach draußen. Nach einer gefühlten Stunde, aber laut seiner Uhr nur einer viertel Stunde, kamen sie zu dritt wieder. \enquote{Meine Güte, hier auch?}, fragte Professor McGonagall ganz entsetzt. \enquote{Das ganze Schloss bebt schon.}

\enquote{Wie meinen Sie das, Professor?}, fragte Martina.

\enquote{In den anderen Gemeinschaftsräumen ist es laut den andern Hauslehrern auch so. Und nicht nur dort. Auf das ganze Schloss prasseln Blitze ein.}

Stille.

Erst jetzt nahm Harry die leicht dunklere Farbe der Mauersteine wahr, die sich seit dem Vorfall in der Großen Halle verändert hatte. Er ließ seinen Blick durch den Raum schweifen und blieb an der Stelle hängen, an der Professor Elber für die Eltern an Weihnachten einen zusätzlichen Raum erschaffen hatte. Er drückte sich durch vereinzelte Schüler und setzte sich auf eine Treppenstufe.

Er ließ sein Schuljahr noch einmal vor seinem geistigen Auge Revue passieren und dachte an die vielen Extra-Stunden, die er genossen hatte und die ihm auf seinem Weg weiter helfen sollten. Seine Hand suchte ein Stück Mauer, das er berührte. Langsam keimte in ihm ein Verdacht. Er schloss seine Augen und lehrte seinen Geist von allen störenden Gedanken. McGonagalls Worte hatte er schon vor geraumer Zeit vollkommen ausgeblendet. Er sah nun die Blitze vor seinem geistigen Auge und zwang sie, weniger zu werden. Er konzentrierte sich auf deren Intensität, um sie zu vermindern. Leider konnte er dem Donner keinen Respekt abtrotzen. Aber die Blitze wurden weniger, verloren etwas an Intensität und schlugen jetzt nicht mehr ins Schloss ein. Danach wurde ihm schwarz vor Augen.

Er erwachte erst wieder, als er etwas an seinen Lippen fühlte. Er legte seine Hände um den Nacken der Person und zog sie zu sich. Bedauerlicherweise hörte er kurz darauf ein entrüstetes \enquote{Harry.}

Er kannte die Stimme genau. \gedanke{Das war Ginny. Aber wieso war sie so weit weg? Und wieso kann sie reden, wenn ich sie küsse?} Er löste den Kuss und öffnete seine Augen. Er blickte in zwei wunderbare Augen, konnte aber wegen seiner schlechten Augen nicht viel sehen, ließ die Person los und suchte nach seiner Brille. Als er sie aufgesetzt hatte, lief ihm ein kalter Schauer über den Rücken.

Er wollte sich gerade entschuldigen, aber er entschied sich im letzten Moment um und so entfuhr ihm ein einfaches: \enquote{Danke.}

Die angesprochene presste ihre Lippen kurz aufeinander, um den leichten Schimmer von Harrys Speichel zu entfernen, bevor er sie anlächelte und dann zu Ginny sah und ihr seine Arme entgegenstreckte. Als sie in seinen Armen lag, fasste Harry doch noch ein Herz und meinte: \enquote{Entschuldigung, Madame Pomfrey. Ich dachte, Sie wären Ginny.} Wortlos drehte sie sich um und ging in ihr Büro.

\enquote{Die hast du jetzt aber geschafft}, flüsterte sie in sein Ohr.

\enquote{Ich dachte, sie wär du. Sie hat sich nach dir angefühlt.}

\enquote{Muss ich ihretwegen jetzt eifersüchtig sein?}

Harry antwortete ihr nicht, sondern gab ihr einen fordernden Kuss. \enquote{Darf ich schon gehen? Was meinst du?}

\enquote{Auf jeden Fall. Zieh dich an.}

Sie wollte sich gerade umdrehen, als sie Harry festhielt.

\enquote{Wenn wir weiter zusammen sein wollen, dann sollten wir langsam anfangen, uns zu vertrauen.} Er schlug die Decke zurück und zog sein Nachthemd aus. Nur in seiner Unterhose stand er vor ihr und begann, sich seine Sachen anzuziehen. Als er fertig war, gab er Ginny wieder einen Kuss und meinte: \enquote{Ich rede kurz mit Madame Pomfrey. Wartest du hier so lange?}

Ginny nickte und Harry verschwand in Madame Pomfreys Büro. Als er wieder kam, verließen sie Hand in Hand den Krankenflügel.

Noch immer donnerte und blitzte es draußen. Und das sollte die nächsten Tage noch anhalten.

\trenn

Zu seinem vorletzten Termin saß Harry in einem bequemen Stuhl und sah an die Decke der privaten Räumlichkeiten Professor Elbers in Hogwarts.

\enquote{Dieses Jahr haben Sie viel gelernt, Harry. Ich habe Ihnen so viel beigebracht, wie ich konnte. Es gibt aber noch etwas, was Sie wissen müssen. \gst Ich weiß nicht, ob es Ihnen schon klar geworden ist, oder ob Sie es nur vermuten. Vielleicht haben Sie auch nur das unterbewusste Wissen? \gst Vielleicht verstehen Sie das auch nicht sofort in seiner vollen Tragweite, was ich Ihnen jetzt sagen werde. Vielleicht hören Sie nur die Worte und verstehen den oberflächlichen Sinn dahinter. Ich will aber, dass Sie den tiefen Sinn dahinter verstehen. \gst Magie muss gepflegt werden. \gst Sie beherrschen doch den Patronus-Zauber. \gst Wie oft haben Sie ihn in den letzten Jahren erscheinen lassen? \gst Meinen Sie nicht, dass es für Ihre Beziehung untereinander besser wäre, ihn öfter zu rufen? Er muss nicht unbedingt eine Aufgabe haben. Rufen Sie ihn einfach. Spielen Sie mit ihm. \gst Das stärkt Ihre Bindung zueinander. Mit einer guten Bindung haben Sie viel mehr Möglichkeiten. \gst Stärken Sie sie.}

Harry war wie vom Donner gerührt. Vor seinem geistigen Auge sah er ein paar Monate zurückblickend Professor Elber auf einer Wiese mit seinem Patronus herumalbern. Damals hatte er nur seinen Kopf geschüttelt und es nicht verstanden, zumal er kurz darauf im Krankenflügel einige Zeit verbringen musste. Doch so langsam ergab es einen Sinn. Man konnte mit einem Patronus Dementoren vernichten, hatte er gelesen. \gedanke{Dazu muss man eine enge Bindung mit seinem Patronus eingehen. Wie er wohl reagieren wird, wenn ich ihn jetzt erschaffe?} Kurz darauf stand ein silbern-blau leuchtender Hirsch im Zimmer. Und wiederum ein paar Sekunden später ein Schwarm geflügelter Insekten. Diese formierten sich und bildeten ebenfalls einen Hirsch. Nun sprangen beide Tiere im Zimmer umher und in die angrenzenden Räume und spielten miteinander.

\enquote{Fragen Sie mich nicht, warum die einen Hirsch gebildet haben. Von mir haben sie das nicht. Und wenn ich ehrlich bin, habe ich das nie so richtig verstanden}, sagte sein Lehrer und schloss seine Augen. \enquote{Bauen Sie eine Verbindung zu Ihrem Patronus auf.}

Harry versuchte es, wurde aber eine viertel Stunde später durch gleichmäßige Atemgeräusche und ab und an einen schnarchenden Laut gestört. Er musste schmunzeln, schaffte es aber doch, zumindest die Präsenz seines Patroni und den seines Lehrers zu spüren. Ob letzteres etwas über seine Verbindung zu ihm aussagte, wusste er nicht. Er wusste aber, dass zwei Menschen, die sich nahe stehen, gleiche Patroni hatten.

Nachdem er seinen Patronus verschwinden hatte lassen, weckte er seinen Professor. Dann fragte Harry ihn, ob er etwas über die Mondbibliothek wisse. Doch leider half ihm seine Antwort nur bedingt weiter.

\enquote{Ja, die gibt es. Aber ich werde Ihnen noch nichts darüber erzählen. Für Sie ist es noch zu früh.}




\begin{kommentar}
Die Prüfung in VgddK beginnt. Diese findet in einer großen Kuppel statt, deren Eingang ein schwarzes 'Nichts' ist. Genauso, wie das 'Nichts' in der unendlichen Geschichte.
\end{kommentar}
