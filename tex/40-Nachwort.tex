\chapter{Nachtrag}


So, meine lieben Reviewer und nicht Reviewer, damit wären wir am Ende des sechsten Schuljahres angekommen. Zusammen mit Harry und seinen Freunden haben wir das Schuljahr erlebt und können entspannt in die Ferien gehen.

Natürlich habe ich mich bei meiner Geschichte auch von Ideen anderer beeinflussen lassen. Diejenigen Ideen, die ich bewusst angenommen habe, möchte ich hier einmal aufzählen. Dabei kann es auch vorkommen, dass der eine Autor oder die andere Autorin selber eine ihrer Ideen in dieser Geschichte findet, aber nicht erwähnt wird. Tja, dann habe ich es entweder vergessen (unwahrscheinlich), oder eure Geschichte nicht gelesen. Im Nachfolgenden nun eine kleine Auflistung der Sachen, die ich aus anderen Geschichten habe. Natürlich ist nur der Grundgedanke übernommen worden. Wenn ihr die Geschichten lest, werdet ihr feststellen, dass sie in einem anderen Zusammenhang stehen, als bei mir.

\accentuate{Die übersinnliche Schlange}: Harrys Basiliskenamulett, sowie die Tatsache, dass er mit Jogging beginnt.

\accentuate{Und dann kam sie}: Draco Malfoy hat eine Schwester

Was ist denn eine Geschichte, ohne ein paar Querverweise. Ich bin gespannt, ob ihr die entsprechenden Stellen auch findet.

Die Schlümpfe, Herr der Ringe, Star Trek (Voyager), Farscape, Musical von Peter Maffay

Und nun noch ein paar Links zu einigen Stücken aus dem Konzert:
\skipspace

>> Medusoner (Lena-Stardust):

http://www.youtube.com/watch?v=EPE2hATCwwE

http://www.youtube.com/watch?v=syp5Lst-viU (Offizielles Video)

http://www.youtube.com/watch?v=E6xGvH9R5ZQ
\skipspace

>> Trautonium:

http://www.youtube.com/watch?v=WZ02S8kfLfU

http://www.youtube.com/watch?v=DBwWpOrqwxY << ab etwa 1:12 bis 1:58 kommt ein guter Part

(Kleiner) Hintergrund zum Trautonium:

http://www.youtube.com/watch?v=Nv9ZQtjlCh4

http://www.youtube.com/watch?v=1mtySbBiGxM

http://www.youtube.com/watch?v=sUIE-jeiC64 << Vertonung von Hitchcocks 'Die Vögel'

http://www.youtube.com/watch?v=xWuoUQy3EO4 << fertige Tonabmischung
\skipspace

>> Thunderstruck (auf dem Dudelsack):

http://www.youtube.com/watch?v=oO1bGlyHDNU

http://www.youtube.com/watch?v=K-Op1Mng4oY
