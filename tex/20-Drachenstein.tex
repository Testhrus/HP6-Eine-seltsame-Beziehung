\chapter{Drachenstein}


Harry dachte über das gerade eben gelesene Kapitel im Buch über Dementoren nach. Sie bedürfen keiner Pflege. Sie gedeihen da, wo wenig glückliche Gedanken sind. Also in Askaban. Allerdings konnte man sie in jungen Jahren unter vielen Mühen auf einen prägen. Sie folgten einem dann. Doch das interessanteste war, dass sich Dementoren auch von schlechten Erinnerungen nähren konnten. Sie waren zwar nicht so effektiv wie die guten, aber sie sättigten doch.

Er erzählte es gerade Ron und Hermine, als er unterbrochen wurde. Ein Fauchen und ein Zittern durchfuhr das Schloss. Alle, die noch beim Essen saßen, sprangen auf und wollten bereits nachsehen, oder fluchtartig in ihre Gemeinschaftsräume rennen.

\enquote{Ruhe}, rief Dumbledore, stand auf und trat durch die Menge hindurch.

Die Schüler und Lehrer folgten ihm. Einige waren bereits an der Eingangstüre angelangt und sahen den Quell der Störung. Viele Schüler gingen ängstlich zurück oder drückten sich an die Wand. So auch Harry, der gerade noch etwas frische Luft schnappen wollte, als das Tier landete. Es war ein grüner Drache. Smaragd-Grün. Dann wartete er, bis sich die Schüler beruhigt hatten und nicht mehr so ängstlich waren.

Dumbledore kam durch die Tür und blieb stehen. Skeptisch sah er zu dem Drachen, der da stand. Es war ein kleiner Drachen. Etwa drei Meter maß er mit Schwanz und sein Stockmaß betrug gerade mal einen Meter fünfzig. Wenn er seinen Kopf ganz streckte, dann hatte er zwei Meter zwanzig. Langsam besah sich der Drache die Reihen der Schüler. Unsicher und vorsichtig zog Dumbledore seinen Zauberstab, was der Drache sofort registrierte und leicht zu fauchen begann. Er fixierte Dumbledore und wartete ab. Harry sah sich ebenfalls um. Einer stand mit einer Seelenruhe da, als würde ihn das gar nicht interessieren, und aß mit einem Löffel aus einer Müslischale. Auch das registrierte der Drache. Harry sah noch einmal auf den Drachen und entdeckte, dass sein Hals silbern schimmerte.

Ein Bild formte sich in seinem Kopf und ohne es bewusst zu wollen, sagte er leise: \enquote{Tabaluga} und sah dabei den Drachen an.

Dieser blickte zurück. \stimme{Du kennst also meinen Namen. Sehr gut. Dann kannst du auch meine Botschaft vermitteln.}

\enquote{Du hast eine Botschaft?}, fragte Harry.

Dumbledore hörte ihn und wendete seinen Blick nun Harry zu, den Drachen aber immer noch aus dem Augenwinkel heraus beobachtend.

\stimme{Denke mit mir. Sprich nicht. Hör zu und erzähle es den anderen.} Harry nickte. \stimme{Ich komme aus Rumänien. Charlie Weasley schickt mich. Ich soll hier Hilfe ersuchen. Es geht eine Seuche um, die uns Drachen befällt. In unserem Reservat sind alle erkrankt außer mir. Keiner weiß warum. Deshalb hat man mich los geschickt.}

Harry nickte und erzählte Dumbledore, was ihm der Drache erzählt hat. Dumbledore ließ seinen Zauberstab sinken, hielt ihn aber immer noch in der Hand.

\enquote{Eine Seuche unter den Drachen?}, sagte Dumbledore und dachte nach.

Währenddessen fühlte sich Harry merkwürdig zu dem Drachen hingezogen. Er verließ das Schloss und ging in den Außenbereich und auf den Drachen zu. Er blieb vor ihm stehen. Nach einigen Schritten blieb auch Draco Malfoy neben ihm stehen, der dem Ruf des Drachen ebenfalls zu folgen schien.

\stimme{Ich spüre eine eigenartige Verbindung zwischen euch. Zuerst Hass und Neid, Misstrauen und Ablehnung. Doch dies scheint sich gewandelt zu haben. Ich spüre den Keim einer Freundschaft, der nur langsam wächst. Ihr feindet euch nicht mehr an. Das ist gut.}

\enquote{Du kannst so etwas spüren?}, fragte Draco.

\stimme{Ja, wir sind Wesen der Magie, empfänglich für ihre Schwingungen. Aber auch für dich gilt, denke mit mir, nicht, rede mit mir.}

Harry lächelte kurz. Hatte ihn der Drache erst selbst kurz zuvor ermahnt.

\enquote{Mir ist keine derartige Seuche bekannt}, sagte Dumbledore. Der sah den Drachen nun wieder an und bemerkte erst jetzt, dass er Harry und Draco neben ihm stehen sah. \enquote{Was macht ihr denn neben dem Drachen?}, fragte er nach.

\enquote{Wir unterhalten uns mit ihm.}

\enquote{Aber ihr habt doch kein Wort mit ihm gewechselt.}

\enquote{Es gibt auch noch andere Arten der Kommunikation}, sagte Draco Malfoy.

Harry nickte nur.

Dann sah der Drache Elber an. \stimme{Du kennst die Ursache?}

Dieser nickte nur. \stimme{Ich bin mir aber sicher, dass dies nicht der Auslöser ist. Es müsste schon jemand in das Heiligtum eingebrochen sein, die Fallen und Hindernisse überwunden haben und das Teil befreit haben. Dann muss er oder sie es unbemerkt nach Rumänien geschafft und unter euch ausgesetzt haben. Das hätte mächtige Spuren hinterlassen.}

Der Drache sah ihn skeptisch an. Elber legte seinen Löffel in seine Müslischale, hob sie in die Luft und zog seinen Arm zurück, als ob er sie nur auf einer hohen Kante abgestellt hatte. Die Schale blieb dort stehen. Dann kam er auf die drei zu.

\gedanke{Ich werde das aber trotzdem prüfen. Wenn du mich begleiten möchtest?}

\stimme{Sehr gerne sogar.}

\enquote{Ich bin weg, Albus. Ich begleite den Drachen auf eine kleine Expedition.}

Dumbledore, der die letzten Sätze des Drachen und seines Lehrerkollegen mitbekommen hatte, sowie die ganze Schule, nickte nur und sah nachdenklich drein.

\enquote{Passt auf euch auf, denn aufhalten werde ich dich wohl nicht können?}

\enquote{Nein}, gab dieser zurück.

\stimme{Apparieren wir zusammen?}, fragte er den Drachen. Dieser nickte und ließ sich berühren. Dann waren sie auch schon verschwunden.

\enquote{Und nun?}, fragte Harry.

\enquote{Ich gehe ins Bett}, sagte Draco und verabschiedete sich Wortlos.

Harry tat es ihm gleich und ging ebenfalls ins Schlossinnere. Die Menschenmenge löste sich auf und zurück, blieb ein einsamer Dumbledore, der seinen Zauberstab einsteckte und in die Dämmerung sah.

\trenn

Der Drache tauchte mit seinem menschlichen Begleiter auf und beide orientierten sich erst einmal. Sie waren an einer Steilküste auf einem Vorsprung von gerade einmal vier Metern aufgetaucht. Hinter ihnen das Meer, welches zwanzig Meter unter ihnen war, und vor ihnen eine aufgebrochene große Steintüre, die an manchen Stellen immer noch so aussah, als seien nur deren Konturen in den Stein gehauen worden.

Beide sahen in den Gang dahinter und Elber zog seinen Zauberstab. Langsam ging er Schritt um Schritt auf den Eingang zu. Wachsam bei jedem Schritt. Der Drache dicht neben ihm. Plötzlich trat der Drache vor ihn und schirmte ihn gegen einen ankommenden Zauber ab.

Erschrocken und auch dankbar sah er zu seinem Begleiter. \enquote{Danke.} Er schloss kurz die Augen und schüttelte seinen Kopf. \gedanke{Danke.}

\stimme{Gerne geschehen}, sagte der Drache. \stimme{Ich nehme dich auf den Rücken, damit dir nichts passiert.}

Elber nickte und stieg vorsichtig auf den Rücken des Drachens. Dieser wollte gerade losgehen, als er in der Anfangsbewegung innehielt und ihn ansprach.

\stimme{Du musst noch viel lernen, obwohl ich eine Menge Magie in dir spüre. Eine ungewöhnliche Menge an Magie.} Der Drache konzentrierte sich und sagte dann: \stimme{Ich spüre nicht nur Gutes in dir. Auch das Böse hat Besitz von dir ergriffen.}

\gedanke{Das ist wahr. Ich war nicht immer so ausgeglichen und um das Gleichgewicht bemüht. Es gab Zeiten in meinem Leben, da war ich ganz anders. Da habe ich schreckliche Dinge getan.}

Der Drache nickte zum Zeichen, dass er verstanden hatte, und schritt voran in den dunklen Gang. Elber warf ein paar Lichtkugeln aus seinem Zauberstab den Gang entlang, um ihn zu erleuchten. Nach einigen Abzweigungen, durch die er den Drachen führte, und mehreren hundert Metern an Gängen, ließ er ihn anhalten.

\gedanke{Stopp. Ich spüre etwas.}

\stimme{Was? Ich habe dieses Gefühl nicht.}

\gedanke{Etwas Seltsames wartet auf uns in diesem Gang.}

\stimme{Ich spüre immer noch nichts}, sagte der Drache, nachdem er sich konzentriert den Gang angesehen hatte.

\gedanke{Ihr könnt doch über Fähigkeiten anderer verfügen.}

\stimme{Ja.}

\gedanke{Dann benutze meine mit}, sagte Elber und öffnete seinen Geist speziell für den Drachen, damit dieser seine Fähigkeiten erweitern konnte.

Den Drachen schüttelte es kurz, als er mit seinem \accentuate{Reiter} verbunden war.

\stimme{Jetzt verstehe ich, was du meintest. Und die Gefahr spüre ich auch. Dank deiner zusätzlichen Fähigkeiten, können wir gefahrlos durchgehen. Du musst nur Körperkontakt mit mir halten. Ein einfaches Aufsitzen mit Kleidung dazwischen funktioniert nicht.}

Elber klammerte sich an den Schuppen des Drachen fest, der jetzt durch den Gang schritt. Die aufkommende Kälte machte ihm nichts aus. Wie ein Leichentuch versuchte etwas die beiden Wesen zu umhüllen, doch es konnte keinen Angriffspunkt erkennen.

Schließlich hatten sie den Gang hinter sich gelassen und sie kamen in eine zentrale Kammer. Dort schien alles normal zu sein. Auf einem Podest in der Mitte stand ein kleines Holzkästchen. Es war verschlossen, stellte Elber fest, nachdem er abgestiegen war und versucht hatte den Deckel zu heben. Gedanklich war er noch immer mit dem Drachen verbunden.

\gedanke{Weißt du}, sagte er, als er verschiedene Zauber auf das Kästchen warf um es zu öffnen. \gedanke{Ich wünschte, ich könnte mein Wissen und meine Erfahrung weitergeben. Ich bin dabei einen jungen Zauberer zu unterweisen. Er versteht so langsam die Zusammenhänge der Magie. Viel Zeit bleibt mir aber nicht mehr.}

\stimme{Dann gib ihm dein Wissen einfach weiter.}

\gedanke{Das sagte ich doch bereits, dass ich ihn unterrichte.}

\stimme{Das meinte ich nicht}, sagte der Drache. Er nahm seine Pranke hoch und hielt eine Zehe an die Stirn seines Begleiters.

Dieser zuckte kurz zusammen und sah danach entgeistert Tabaluga an. \gedanke{Ich hatte keine Ahnung, dass das überhaupt funktioniert. Und dazu noch so einfach.}

\stimme{Du musst noch viel lernen}, sagte der Drache und legte seine Zehe wieder an die Stirn.

Nach etwa zehn Minuten war er fertig. Er nahm seine Pranke zurück und sah ihn an.

\gedanke{Danke}, sagte Elber, den Drachen ehrfürchtig anblickend.

\stimme{Auch ich habe zu danken. Ich habe durch die Verbindung zu dir viel erfahren. Wir beide haben dadurch profitiert. Ich weiß jetzt auch, warum ich verschont blieb.}

Elber nickte. \gedanke{Basiliskengift?}, fragte er nach.

\stimme{Richtig. Meine Mutter war mit mir schwanger, erzählte man mir. Sie hatte sich mit einem Basilisken gestritten, der sie dann biss. Nach meiner Geburt starb sie. Wenige Wochen danach. Ich hatte sie kaum kennengelernt. Aber sie hatte mir eine Menge Wissen vermittelt.}

\gedanke{Deswegen auch deine ungewöhnliche Farbe.}

Der Drache schaute ihn an. \stimme{Daran habe ich noch gar nicht gedacht.}

Elber öffnete das Kästchen und beide sahen hinein. Es war leer.

\stimme{Und nun?}

\gedanke{Ins Reservat. Ich würde mich dort gerne umsehen.}

\stimme{Das dauert aber, bis wir dort angekommen sind.}

\gedanke{Wieso? Apparieren geht schneller als fliegen.}

\stimme{Aber auf die\abs} Weiter kam der Drache nicht mehr. \stimme{\aabs Entfernung geht das doch\abs} Sie tauchten mitten im Zelt mit den Arbeitern auf, die gerade zu Abend aßen. \stimme{\aabs gar nicht}, beendete Tabaluga seinen Satz.

Sofort wirbelten die Männer und die Frau herum und richteten ihre Zauberstäbe auf die beiden vermeidlichen Eindringlinge. Als sie Tabaluga erkannten, nahmen sie ihre Stäbe herunter, steckten sie ein und aßen weiter. Charlie Weasley kam auf ihn zu und wurde freudig begrüßt.

\enquote{Hallo kleiner}, sagte Charlie. \enquote{Hast du meine Nachricht überbracht?}

\stimme{Ja, ich habe jemanden mitgebracht, der uns helfen wird.}

Charlie schüttelte die Hand von Elber und stellte sich vor. \enquote{Charlie Weasley. Alle sagen aber nur Charlie.}

\enquote{Dann nennen Sie mich Frederick.}

\enquote{Und wie noch?}

\enquote{Elber.}

\enquote{Dann sind sie der anfangs mutmaßliche Todesser?}

\enquote{Das hat Ihnen Ihr Bruder geschrieben, stimmt’s?}

\enquote{Ups. Ja}, gab er kleinlaut zu.

\enquote{Sie sagten ja, anfangs und mutmaßlich, dann passt das schon. Es scheint, dass er seine Meinung geändert hat.}

Charlie nickte nur. \enquote{Was machen wir jetzt?}

\enquote{Jetzt möchte ich als Erstes einen ihrer Drachen sehen, damit ich weiß, ob mein Verdacht stimmt.}

\stimme{Hat es was mit diesem Holzkästchen zu tun?}

\enquote{Ich hoffe nicht, aber ich befürchte es. Das, was darin war, ist verschwunden.} Charlie Weasley macht ein besorgtes Gesicht. \enquote{Keine Angst, wenn es das ist, dann ist eine Heilung nicht nur möglich, sondern auch einfach und relativ unkompliziert.}

\enquote{Aber was befürchten sie dann?}

\enquote{Dass andere Drachen-Kolonien auch betroffen sind. Wir müssen das, was entwendet wurde, wieder bekommen.}

\enquote{Vernichten wir es dann?}

\enquote{Das geht leider nicht. Zumindest habe ich keine Ahnung, wie ich es anstellen soll.}

\enquote{Dann werden sich andere darum kümmern.}

\enquote{Die werden das Teil nicht mal analysieren können. Es ist mit einer speziellen Art von Magie ausgestattet. \gst Reden wir draußen weiter.}

Sie verließen das geräumige Zelt und traten in die warme und dunkle Sommerluft hinaus und gingen einen ausgetretenen Pfad entlang.

\enquote{Das klingt alles so, als ob sie das Teil, was auch immer es ist, sehr genau kennen.}

\enquote{Ich habe mich damit beschäftigt. Es studiert.}

\enquote{Was ist es?}

\enquote{Es ist etwas um Drachen umzubringen. Es verströmt eine Art von Magie, die Drachen krank macht. Ihre Zellen werden verändert. Da es durch die Magie selbst geschützt ist, kann man es fast nicht zerstören.}

Charlie dachte nach. \enquote{Sie sagten fast.}

\enquote{Es kann unter Umständen gelingen, es zu zerstören.}

\enquote{Wie?}

\enquote{Man wendet etwas gegen das Objekt, das ebenfalls Zugriff auf die Magie hat. Ich meine damit keinen Zauber, oder Fluch. Ich spreche von der reinen Magie.}

\enquote{Das verstehe ich nicht. Erst sagen Sie: \enquote{Man kann es nicht zerstören.} Dann sagen Sie: \enquote{Unter Umständen.} Und jetzt plötzlich: \enquote{Man muss nur.} Was nun?}

\enquote{Ich weiß, dass und wie man es zerstören kann. Aber ich traue mich nicht, diese Methode anzuwenden. Sie ist gefährlich.}

\enquote{Dann werde ich es machen.}

Der Drache begleitet die beiden und lief zwischen beiden hin und her. Er beobachtete sie und lauschte ihren Geistern, damit er verstand, was sie zu sagen hatten, denn die Sprache der Menschen beherrschte er nicht.

\enquote{Sie werden sterben. Ich werde vielleicht für eine Weile außer Gefecht gesetzt. Oder mithilfe von Tabaluga sogar unbeschadet davon kommen.}

Charlie schaute ihn erstaunt an.

\stimme{Wir sind in der Mitte des Reservates}, informierte Tabaluga sie.

\gedanke{Dann werden wir uns hier niederlassen und die Gedanken schweifen lassen.}

\enquote{Sie können mit Tabaluga kommunizieren?}

\enquote{Ja.}

\enquote{Ich dachte nicht, dass das jemand kann, der nicht in direktem Kontakt mit Drachen steht.}

\enquote{Lassen Sie uns später darüber reden. \gst Schlafen Sie gerne draußen?}

\enquote{Ja.}

\enquote{Dann werden wir uns an\abs} Er sah Tabaluga an. \enquote{\aabs dich ankuscheln.}

\stimme{Warum?}

\gedanke{Drachen haben eine Menge an Magie in sich. Sie können sie zwar nicht aktiv nutzen, aber deshalb können sie sich sehr gut dagegen wehren. Kaum ein Zauber kommt durch sie durch. Wenn wir mit Körperkontakt entspannen, dann wird sich diese Magie mischen und wir können die Verursacher aufspüren, sofern sie noch hier sind. Sie verstehen eine Menge von Magie.}

Tabaluga nickte und legte sich auf die Seite. Frederick und Charlie legten sich mit ihren Köpfen auf seinen Bauch. Dann schlossen sie die Augen und ließen ihren Geist treiben. Jeder hatte das Gefühl über dem eigenen Körper zu schweben. Die Distanz vergrößerte sich. Mehrere hundert Meter über dem Reservat drehten sich die Geister um und überblickten das Reservat. Langsam bildeten sich Magie-Inseln auf dem Boden. Eine stach daraus deutlich hervor. Von ihr gingen in zyklischen Abständen Wellen aus, die den Luftraum zu überwachen schienen. Vorsichtig und immer nur dann, wenn keine Welle kam, bewegte sich die geistige Energie auf die große Insel zu.

Als sie in Reichweite waren, sahen und hörten sie zwei Todesser.

\enquote{Wir warten noch zwei Tage, dann gehen wir ins nächste Reservat.}

\enquote{Meinst du, das ist sicher? Wir sind schon einige Tage hier. Ich bin mir nicht sicher, ob der Zauber hält.}

\enquote{Den Zauber hat mir der Dunkle Lord persönlich beigebracht und getestet. Er wird halten. Keiner kann uns entdecken.}

\enquote{Na schön. Dann schalten wir das Gerät aber aus.}

\enquote{Von mir aus. Die Drachen schaffen es eh nicht mehr.}

Dann fanden sie sich wieder in ihren eigenen Körpern.

Charlie schlug die Augen auf und meinte: \enquote{Dann mal los, wenn wir sie noch erwischen wollen.}

\stimme{Du hast doch gehört, dass sie noch zwei Tage hier bleiben wollen. Also keine Hektik.}

Doch Charlie gab sich nicht ganz geschlagen. \enquote{Und die Drachen. Dann sollten wir denen helfen.}

\stimme{Wenn du dich wieder beruhigt hast, dann kannst du uns helfen.}

\gedanke{Wobei?}

\stimme{Den Drachen zu helfen?}

\gedanke{Wie?}

\stimme{Leg dich hin, Charlie. Du wirst es merken.}

Charlie war verwirrt. Doch dann legte er sich hin und dieser schwebende Zustand stellte sich wieder ein. Zusammen mit Tabaluga und Frederick reiste er gedanklich zu jedem Drachen und half ihm mithilfe seiner Magie. Zumindest hatte er das Gefühl. Er glaubte, dass er mit jedem Drachen eine Verbindung einging, eine Verbindung die einer Gedankenverschmelzung gleich kam. Doch nach und nach verloren sich die erlangten Erkenntnisse und Gedanken, die die Verbindung mit sich brachte. Die Essenz der Drachen blieb zurück.

Infolgedessen, wurde Charlie ruhiger. Sein Verhältnis zu den Drachen wurde besser, doch das würde er erst in ein paar Wochen langsam feststellen.

Der Vorgang, die Drachen zu heilen, dauerte die ganze Nacht. Doch als sie bei Morgendämmerung fertig waren, hatte keiner der drei Ermüdungserscheinungen.

Die anderen Drachenbändiger kamen und tuschelten, als sie die drei liegen sahen.

\enquote{Wir sind wach}, antwortete Charlie. \enquote{Ich weiß, wo die Todesser sind.}

Frederick setzt sich auf und auch Tabaluga sah ihnen interessiert zu.

Nach einer Weile fragte Charlie: \enquote{Wollen Sie sich nicht daran beteiligen, Frederick?}

\enquote{Nein. Ich warte ab, höre zu, komme mit und kümmere mich ausschließlich um das Teil. Es darf keinen Schaden nehmen. Es muss richtig zerstört werden. Wenn es angeschlagen ist, kann dies unvorhergesehene Auswirkungen haben.}

Charlie nickte und beredete mit seinen Kollegen die weitere Vorgehensweise.

\trenn

\enquote{Ich dachte immer, von Hogwarts aus kann man nicht apparieren}, sagte Hermine vorwurfsvoll.

\enquote{Kann man auch nicht. Aber Drachen sind mächtige Geschöpfe.}

\enquote{Aber Drachen können doch nicht zaubern.}

\enquote{Das nicht, aber Drachen können sich durch die Magie schützen. Warum wohl, kann man Drachen fast nichts anhaben.}

\enquote{Wie?}

\enquote{Drachen haben, wie auch immer, Zugriff auf die Magie. Sie können sich damit hauptsächlich schützen.}

Jetzt zeigten mehr Interesse an der Unterhaltung zwischen Hermine und Dumbledore. Draco und Harry hielten an und lauschten.

Als Dumbledore dies merkte, winkte er allen zu und ging Richtung Große Halle. \enquote{Machen wir eine Unterrichtsstunde daraus.}

In der Großen Halle angekommen, lies er mit seinem Zauberstab die Bänke verschwinden und viele weiche Kissen und Decken in der Halle erscheinen. Er selber setzte sich an den Kopf des Arrangements auf ein weiches Kissen und begann zu erzählen, nachdem die Schüler ihre Plätze eingenommen hatten.

\enquote{Wie sicher alle wissen, haben Drachen einen guten Schutz gegen viele Zauber. Es wird ihnen nachgesagt, dass ihre Haut die Zauber abwehrt. In gewisser Art und Weise ist das richtig, denn in ihrer Haut ist besonders viel Magie versammelt, die sie vor Angriffen zu schützen versucht. Das ist bei den Drachen genetisch veranlagt und lässt sich nicht ändern. Leider hat das den Nachteil, dass die Drachen dadurch schwierig werden, kurzum, sie werden aggressiver. Selber können sie nicht aktiv über ihre Magie verfügen. Sie haben aber die Möglichkeit, sofern es sich um zahme Drachen handelt, was selten vorkommt, dass sie ihre Magie mit einem Zauberer verbinden. Dann bildet sich eine Einheit. Beide profitieren davon, während der Verbindung. Diese muss aber eine körperliche Verbindung sein.}

Jetzt kamen die ersten Fragen.

\enquote{Aber ich dachte, dass Drachen von sich aus böse sind.}

\enquote{Nein, aber missmutig und skeptisch gegenüber anderen Arten.}

\enquote{Wie kann ich das verstehen?}

\enquote{Wenn die Drachen nicht diese Fähigkeit, beziehungsweise Magie, in ihrer Haut hätten, dann wären sie friedlicher. Zwar immer noch aggressiv und misstrauisch, aber etwas freundlicher.}

\enquote{Sie meinen also, dass die Magie die Drachen beschützt?}

\enquote{In gewisser Weise ja.}

\enquote{Was werden die zwei wohl machen?}, fragte ein anderer Schüler.

\enquote{Wie meinen Sie das?}

\enquote{Es sah so aus, als ob sich die beiden unterhielten, bevor sie verschwanden.}

\enquote{Ja, das hat mich auch verwundert}, sagte Dumbledore nachdenklich.

\enquote{Ich finde ihn ja manchmal etwas eigenartig}, sagte ein weiterer Schüler.

\enquote{Inwiefern?}

\enquote{Er redet über Zaubersprüche und Flüche, egal welcher Couleur, so, als würde er sich mit jemandem über das Wetter unterhalten. Er macht keinen Unterschied, ob es sich um schwarze oder weiße Magie handelt.}

\enquote{Hast du nicht aufgepasst? Es gibt k\aabs}, konterte ein anderer Schüler.

\enquote{Ich weiß, ich weiß, das war nur ein Beispiel. Drück es besser aus. Auf jeden Fall spricht er über diese Zauber als seinen sie ganz normal. Er unterscheidet nicht. Man könnte wirklich meinen, er arbeitet für\abs die andere Seite.}

\enquote{Ja}, gab Dumbledore zu. \enquote{Das könnte man glauben. Jetzt aber nicht mehr, oder? Ist noch irgendjemand der Meinung, dass es sich um einen Todesser handelt? Oder um einen Unterstützer von Voldemort?}

Es war totenstill in der Großen Halle. Keiner sagte mehr etwas. Stumm schüttelten sie ihre Köpfe.

\enquote{Dann meine ich, dass es an der Zeit wird, ins Bett zu gehen.}

Die Versammlung löste sich auf und ging in ihre Zimmer. Auch Harry legte sich nach der abendlichen Toilette in sein Bett. Während des gesamten Vortrages kam ihm immer wieder der Drache in den Sinn.

\gedanke{Er hatte die Farben Slytherins. Grün und Silber. Aber Drachen sind nicht Silber. Basiliskengift schimmert silbern. Dementorenblut ist, glaube ich silbern. Quecksilber ist \gst Quatsch. Quecksilber. \gst Am wahrscheinlichsten ist Basiliskengift.} Er schloss seine Augen und dachte weiter. \enquote{Das wäre schön, jetzt zu fliegen, wie ein Drache, oder ein Vogel.}

Harry versank in einen leichten Dämmerschlaf.

\begin{traum}
Er stand auf einer Klippe, doch er war nicht er selber. Trotzdem erkannte er seinen Körper. Dann nahm er ein paar Schritte Anlauf und stürzte von der Klippe. Seine Flügel breiteten sich sofort aus und er flog. Er war ein dunkler Rabe. Er flog über Wälder und Äcker. Über Wiesen und Hügel. Dann sah er Futter. Er bewegte sich auf ein Säugetier zu und spie Feuer. Er war ein Drache. Die Verwandlung bekam er nicht bewusst mit. Er wusste nur, dass er ein Drache war und unter sich einen Wildhund entdeckt hatte, der keine Chance hatte; genüsslich verspeiste er ihn.
\end{traum}

Harry bekam nicht mit, wie sich während seines Traumes seine Haut zu verwandeln begann. Sie bildeten Schuppen aus und verschwanden wieder, nachdem der Traum beendet war.

Am nächsten Morgen wunderte er sich darüber, dass es ihm nicht aufgefallen war, dass er sich plötzlich von einem Raben in einen Drachen verwandelt hatte. Beim Raben war die Frage der Farbe noch einfach. Raben sind schwarz. Aber Drachen können viele Farben haben. Da er sich nicht sehen konnte, hatte er keine Ahnung.

\enquote{Ron, Hermine. Guten Morgen, Ginny. Hört mal her. Ich hatte heute Nacht einen komischen Traum.} Er erzählte den dreien auf dem Weg zum Frühstück von seinem Traum.

\enquote{Meinst du, das hat was zu bedeuten?}

\enquote{Klar, aber was?}

\enquote{Du sehnst dich nach Freiheit, schätze ich}, sagte Hermine.

\enquote{Vielleicht willst du auch durchs Feuer gehen}, meinte Ginny.

\enquote{Sehr witzig}, gab er zurück.

\enquote{Schau doch in der Bibliothek nach einem Buch über Traumdeutung}, schlug Ron vor.

\enquote{Weißt du, wie viel man dafür beachten muss? Ich habe mal eine Sendung darüber ansehen müssen. Das ist mir zu aufwändig. Beizeiten werde ich es schon erfahren.}
%Zitat: Der Mächtige wird erst mächtig, wenn er seine Macht gebraucht. (Ernst R. Hauschka, deutscher Aphoristiker, geb. 1926) (Oktober 2007)




\begin{kommentar}
Auf dem Hof vor dem Großen Tor taucht plötzlich ein Drache auf. Sein Name ist Tabaluga. Dieser Name setzte sich schon aus den Buchstaben zusammen, die Harry bei seiner Übung hinter Tarnungen zu sehen, entdecken musste und denen auf dem Boden davor.
\end{kommentar}
