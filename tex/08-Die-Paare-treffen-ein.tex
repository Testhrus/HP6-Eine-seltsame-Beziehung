\chapter{Die Paare treffen ein}


Nach einer Stunde kam Professor Elber in die Bibliothek und setzte sich zu Harry. Nach einem kurzen Blick auf Harrys Hausaufgaben stand er wieder auf und verschwand zwischen den Regalen. Knappe zwei Minuten später kam er wieder und setzte sich mit drei Büchern wieder neben Harry. Er blätterte im ersten und suchte etwas. Dann legte er das Buch vorsichtig vor Harry ab. Während er im nächsten blätterte, sah Harry auf die Seiten und erkannte, dass ihn das ein gutes Stück weiter brachte. Das zweite Buch legte Professor Elber vor sich ab und suchte auch im dritten die richtige Seite heraus. Dann stand er auf und ging.

Er drehte sich noch einmal kurz und meinte: \enquote{Ein kleines Dankeschön dafür, dass sie das Buch gefunden und mir geholfen haben. \gst Und noch ein Hinweis. Beobachten Sie die Bücher, wenn sie sie aufräumen lassen.} Dann verschwand er.

Mitten in die Arbeit vertieft kamen Ron und Hermine, setzten sich neben ihn und begannen ebenfalls mit ihren Hausaufgaben. Mittlerweile war er recht gut darin Lunas Gedanken zu blockieren, die er sonst immer aufgefangen hatte, außer, sie wollte ihm etwas mitteilen, das hörte er dann klar und deutlich. Nachdem es fast Zeit fürs Abendessen wurde, entschieden sich die drei ihre Sachen im Gemeinschaftsraum zu lagern und danach in die Große Halle zu gehen. Harry erzählte ihnen vom Grauzauber und den Büchern, die Elber ihm herausgesucht hatte. Während des Essens erzählte Harry Ron und Hermine was er neues über die Karte herausgefunden hatte, natürlich ohne etwas über den Gemeinschaftsraum der Paare zu erzählen, oder wie genau er die Punkte gefunden hatte. Er deutete auf seine Tasche und versprach, ihnen einen Klon der Liste anzulegen, damit jeder, der Zeit hatte, sich auf die Suche und die Erkundung der Stellen machen konnte.

Plötzlich kam ein Vogel hereingeflogen, ein kleiner grüner Wellensittich. Professor McGonagall lief ihm hinterher. \enquote{Fiorin, Fiorin, komm her. Komm zu Mama.}

Als der Vogel an Harry vorbeiflog, versuchte er ihn zu fangen. Doch er griff daneben.

Nach langem Versuchen schaffte es schließlich Luna den kleinen Vogel zu fangen. Sie gab ihn Professor McGonagall. Nachdem sie den Wellensittich vorsichtig in den Käfig zurückgesetzt hatte, sprach Luna zu ihr: \enquote{Jetzt weiß ich, dass sie gut zu Vögeln sind, Professor.}

Harry drehte seinen Kopf zu Luna und sah sie entsetzt an.

\enquote{Miss Lovegood, denken sie an ihre Ausdrucksweise} maßregelte Professor McGonagall sie.

Harry hatte das Gefühl, ihr helfen zu müssen. Er sagte zu Professor McGonagall: \enquote{Wer zweideutig denkt, hat eindeutig mehr vom Leben.}

Professor McGonagall dreht sich zu Harry. Er hatte das Gefühl, dass nun sie es war, die rot wurde.

Sie nahm den Vogel im Käfig mit und verließ die Große Halle. Erst jetzt bemerkte Harry, dass alle umstehenden Schülerinnen und Schüler sie anstarrten und krampfhaft ein Grinsen unterdrückten. Es dauerte es ein paar Minuten bis die Große Halle vollständig gefüllt war, da Dumbledore angekündigt hatte, eine Erklärung abzugeben. Jeder Schüler und jede Lehrkraft war jetzt anwesend. Selbst Madame Pomfrey.

Professor Elber baute aus drei Gabeln, die er sich von Nachbartellern holte, eine kleine Skulptur, indem er die Zinken der Gabeln gegeneinander stellte und so versuchte ein Dreieck auf dem Teller zu legen. Die Seiten der Gabeln mit den Spitzen zeigten nach oben. Als das Essen verschwand, stand ein paar Sekunden danach ein Elf neben ihm und sah ihn abwartend an. Er lächelte ihm zu und sprach mit ihm. Der Elf nickte und wartete.

Dumbledore stand auf und begann gegen sein Glas zu klopfen. Alles verstummte. \enquote{Wie ich gestern bereits angekündigt habe, wird Professor Elber heute noch einmal versuchen, euch zu erklären, warum er euch das beibringt, was viele als dunkle Magie ansehen. Ich möchte heute noch einmal betonen, dass Professor Elber nichts mit Voldemort\abs} ein Raunen und Zucken lief wieder einmal durch die Halle. \enquote{\aabs zu tun hat. Er ist kein Todesser und auch kein Sympathisant. Er wird jetzt noch einmal versuchen euch zu erklären, warum er es für wichtig hält, dies zu unterrichten.} Dann setzte er sich wieder.

Die Tische begannen sich zu bewegen und nach oben an die Decke zu schweben. Dann folgten die Bänke. Die Schüler schreckten hoch und standen in der leeren Halle.

\enquote{Bitte bilden Sie alle einen großen Halbkreis.}

Die Schüler folgten und machten einen großen Halbkreis in mehreren Reihen. Eine Abwärts-Geste machte ihnen  klar, dass sie sich setzen sollen. Unter jeder Schülerin und jedem Schüler bildete sich ein Kissen, bevor er den Boden berührte und mit seinem Hintern auf den harten Boden aufkam. In der Mitte erschienen viele kleine Kissen auf denen Elfen erschienen.

In den hinteren Reihen kam es zu Gemurmel und Tumulten.

\enquote{Die werden später gebraucht, damit sie nicht verletzt werden. Sie werden sie schützen}, sprach Professor Elber, stand auf und lief in die Mitte der kleinen Empore. Er setzte sich auf den Boden und stellte die Füße auf den steinernen Boden darunter. Die Hände auf seinen Oberschenkeln atmete er einmal kurz durch und fing dann an zu erzählen.

\enquote{Ich war verletzt, als ich hörte, sie haben vor mir Angst. Es war nicht meine Absicht sie zu ängstigen. Es war vielmehr meine Absicht, ihnen klarzumachen, dass es notwendig ist, den Todessern etwas entgegensetzen zu können. Und das besteht nun mal darin, deren Methoden zu kennen. Sie müssen wissen, was sie erwartet. Wenn man die Angriffsmethoden und -taktiken seiner Gegner kennt, kann man effektiver zuschlagen.}

Er sah wieder auf und wartete kurz.

\enquote{Ich habe Ihnen in ihrer ersten Stunde erzählt, dass es keine dunkle oder helle, schwarze oder weiße Magie gibt. Ich habe ihnen erzählt, dass Magie keine Farbe hat. Haben sie das so weit verstanden?}

Keine Antwort.

\enquote{Dann lassen sie es mich wiederholen. Ich nehme an, jeder kennt einen Hammer.} Nicken quer durch alle Häuser und Geschlechter, sowie Jahrgangsstufen. \enquote{Und sie wissen, was man mit einem Hammer alles machen kann? Mit einem Hammer kann man Nägel in die Wand einschlagen. Aber man kann auch den Hammer jemanden auf den Kopf hauen und ihn somit umbringen. Bedeutet das jetzt, dass ein Hammer böse oder dunkel ist?}

Die Schüler schüttelten die Köpfe.

\enquote{Eben. Und genau so verhält es sich mit der Magie. Die Magie ist nur ein Werkzeug. Man verwendet sie. Soweit noch alles klar?}

Wieder kam ein Nicken von den Schülern.

\enquote{Den siebten Klassen habe ich es schon gezeigt und bin auch ins Detail gegangen.} Er erzeugte eine kleine Flamme auf seiner Hand. \enquote{Diese Flamme hier wird von vielen Dämonen- oder auch Teufelsfeuer genannt. Aber warum? Weil viele, die es kennen es für zerstörerische Zwecke einsetzen. Dabei kann man mit diesem Feuer viel mehr machen.}

Er ließ es fallen und sofort schlängelte es sich in die Mitte der Halle und bildete einen Ring inmitten des Halbkreises.

\enquote{Diesem lebendigen Feuer kann man seinen Verwendungszweck mitgeben. Wie sie hier alle sehen, greift es niemanden an, sondern brennt nur vor sich hin und wärmt. \gst Wer von ihnen kennt denn das ewige Feuer? Diese netten blauen Flammen, die man überall hin in einem Glas tragen kann?}

Ein paar Hände gingen in die Höhe.

\enquote{Würden sie diese Flammen der dunklen Magie zuordnen?}

Diejenigen, die gezeigt hatten, schüttelten ihre Köpfe.

\enquote{Warum nicht? Dieses Flammen unterscheiden sich vom lebendigen Feuer unwesentlich. Es ist dieselbe Art von Magie, die dahinter steckt. Ein paar kleine Änderungen und die Farbe der Flammen wird blau und es breitet sich nicht automatisch aus. Aber prinzipiell ist das dasselbe.}

Die Augen der Schüler und Lehrer wurden größer.

\enquote{Lassen sie mich noch etwas sagen. Lebendiges Feuer kann für viele Zwecke eingesetzt werden. Nehmen Sie zum Beispiel einen Waldbrand. Einen großen Waldbrand wie in Amerika. Diesen müsste man mit hunderten Zauberern bekämpfen und in Schach halten. Wenn diese aber das lebendige Feuer verwenden, dann reichen etwa ein Dutzend. Man gibt dem Feuer mit, dass es eine Brandschneise in den Wald brennen soll. Das Feuer schlägt nicht auf benachbarte Bäume über und verlöscht nach getaner Arbeit.}

Er stand auf und lief in die Mitte des Feuers. Dieses bildete nun einen Ring um ihn, nachdem es ihn eingelassen hatte. Dann fing eine kleine Schlangengestalt an aus dem Feuer heraus zu kommen; immer noch mit dem Feuer verbunden schlängelte sie sich an seinem Bein entlang hoch und legte dann ihren Kopf auf seine Schulter.

\enquote{Sehen Sie, keine Gefahr.}

Das Feuer löste sich auf. Jetzt traten die Elfen in Aktion. Sie bauten vor sich eine Energiekuppel auf, in der nur noch Professor Elber und der Elf stand, der schon die ganze Zeit in seiner Nähe war.

\enquote{Um ihnen zu zeigen, dass man sich gegen \accentuate{dunkle} Magie sehr wohl wehren kann, durch Abwehr oder auch Gegenangriffe, haben Henry und ich eine kleine Demonstration vorbereitet. Die Elfen vor ihnen werden sie vor den Zaubern schützen, die sicher kommen werden.}

Dann stellten sich beide voreinander auf, nickten einander zu und griffen an. Die nächste viertel Stunde kam Zauber um Zauber, der den Gegner verwirren sollte, ihm zusätzliche Gliedmaßen wachsen lassen, oder ihn entstellen sollte. Doch jeder der Zauber wurde abgewehrt. Dann griff der Elf in seine Trickkiste und warf durch eine Bodenverschiebung Professor Elber um. Dieser ließ sich fallen und fiel weich auf den Boden. Der Elf sprang in die Luft und wollte von oben auf ihn drauf fallen. Doch Professor Elber baute einen Schild auf, worauf der Elf abprallte und in die Luft geschleudert wurde. Er blieb förmlich an der Decke stehen und sah nach unten. Dann legte er sich hin und fing im Liegen an, Zauber nach unten zu werfen. Professor Elber warf Zauber nach oben. Als der Elf wieder den Boden berührte, brachen beide ihre Zauber ab und die Elfen lösten die Energiebarriere und verschwanden. Der Elf und Professor Elber verbeugten sich voreinander und der Elf verschwand auch.

Professor Elber setzte sich wieder auf seinen Platz an der Empore und fragte: \enquote{Sind Fragen da?}

\enquote{Wie sieht es mit den Unverzeihlichen aus?}, fragte ein Schüler.

\enquote{Bevor ich ihnen darauf antworte\abs Wie sehen sie das?}

\enquote{Na ja, ich würde sie der dunklen Magie\abs Sie wissen wie ich das meine}, sagte er.

\enquote{Sie würden also sagen: \enquote{Dunkle Magie.}}

\enquote{Ja}, antwortete der Schüler.

\enquote{Nehmen wir an, diese drei Flüche seien erlaubt, also legal.} Das verwunderte Kopfschütteln, oder das Gemurre einiger ignorierte er. \enquote{Nehmen wir weiterhin an, dass eine Person, die ihnen sehr nahe steht \gst zum Beispiel ein guter Freund, aber kein Partner \gst gerade aus einer Beziehung kommt und mit den Nerven vollkommen fertig ist.} Die Schüler nickten zu Bestätigung. \enquote{Nehmen wir weiterhin an, dieser Freund steht auf einer Klippe. Über ihm der Himmel, unter ihm der Boden und einen Schritt weiter ein Abgrund von zwanzig Metern Tiefe, der in einer stürmischen Brandung mit Felsen endet, sodass er bei einem Sprung aufprallt und stirbt. Er sagt ihnen, dass sie keinen Schritt weiter gehen sollen, dann würde er springen. Sie können sich ihm also nicht nähern. \gst Wie würden sie ihn retten? \gst Denken sie daran, was ich gerade noch am Anfang gesagt habe.}

Dann war es eine Weile still. Eine Schülerin hob die Hand und Professor Elber nickte ihr zu.

\enquote{Ich würde es mit einem Schwebezauber versuchen.}

\enquote{Dann ergreift er Gegenmaßnahmen. Er belegt sie mit einem Kitzelfluch und stürzt ab. Sie rennen lachend an den Rand und sehen vielleicht gerade noch, wie ihr Freund stirbt. Und sie lachen, da der Fluch noch wirkt.}

Betretene Gesichter.

\enquote{Andere Vorschläge?}

Wieder wurde ein Schüler dran genommen. \enquote{Ich würde dann den Imperius-Fluch nehmen, da wir ja annehmen er sei in unserem Beispiel legal. Der Fluch kann zwar mit einem starken Willen überwunden werden, aber wenn einer emotional dermaßen aufgelöst ist, wird es ein leichtes sein, ihn zu halten. Dann hole ich ihn zu mir, beruhige ihn und nehme den Fluch von ihm.}

\enquote{Guter Einfall. Wie sehen sie den Fluch jetzt?}

\enquote{Er kann auch nützlich sein. \gst Ich glaube, ich verstehe. Mit einem Messer zum Beispiel kann man sein Essen zubereiten, Seile zerschneiden, aber auch jemanden umbringen. Es kommt darauf an, was man damit macht.}

\enquote{Warum, denken sie, sind Messer dann nicht verboten? Man kann damit schließlich jemanden töten?}

Wieder dauerte es eine Weile, bis sich jemand meldete.

\enquote{Ich nehme an, dass es weniger \accentuate{missbräuchliche} Anwendungen für Messer gibt, als für den Imperius-Fluch.}

\enquote{Gut gesagt. Wissen sie, was mich bei unserer Gesetzgebung im Falle dieses Fluches stört? \gst Die Tatsache, dass man per se einen Aufenthalt in Askaban gewinnt und nicht per Fall unterschieden wird. \gst Nächster Fluch, der Tötungsfluch.}

Dieses Mal dauerte es etwas länger, bis sich jemand meldete. Professor Elber nickte und der Schüler begann.

\enquote{Die einzige sinnvolle Anwendung, die ich mir vorstellen könnte, wäre jemanden wie \accentuate{Sie-wissen-schon-wen} seinem Schicksal zuzuführen.}

\enquote{Also würden sie den Zauber nicht ganz der \accentuate{dunklen} Seite zusprechen?}

\enquote{Nein.}

\enquote{Gut. Was ist mit dem letzten Zauber? Dem Folterzauber?} Nach einer Weile löste Professor Elber auf. \enquote{Dieser Zauber ist wohl eine der wenigen Ausnahmen, die ich tatsächlich der \gst wie sie so schön sagen \gst \accentuate{dunklen Seite} zuordne.}

\enquote{Was sind die anderen Zauber, die sie dazu zählen?}

\enquote{Die werde ich ihnen nicht nennen. Dagegen ist der Cruciatus-Fluch ein erholsames Bad. Wobei nicht alle mit körperlichen Schmerzen zu tun haben.} Nach einer ganzen Weile fragte er: \enquote{Noch Fragen?}

Einhelliges Kopfschütteln folgte. Die Tische und Bänke kamen langsam von der Decke und nahmen ihre Plätze ein.

Die Türe öffnete sich und Professor Elber stand auf. \enquote{Es tut mir leid}, sagte er, \enquote{dass ich sie geängstigt habe. Das war nicht meine Absicht. Ich werde mir etwas überlegen, um sie dafür zu entschädigen.} Dann ging er Richtung Ausgang.

Dumbledore kam auf ihn zu und fing ihn ab. \enquote{Auf ein Wort, Frederick, in meinem Büro.}

In Dumbledores Büro angekommen setzen sich beide auf ein Sofa hinter Dumbledores Bürotisch. Es war ein kleines Separee.

\enquote{Warum haben sie den Siebtklässler das Dämonenfeuer beigebracht?}

\enquote{Lebendiges Feuer}, korrigierte ihn Elber.

\enquote{Meinetwegen, aber warum haben sie es den Schülern beigebracht?}

\enquote{Weil es prüfungsrelevant sein wird.}

\enquote{Wie?}

\enquote{Ich stelle mir das so vor. Es gibt draußen auf der Wiese eine große Kuppel. Milchig und mit einem Tor als Eingang. Dahinter ist es dunkel. Der Raum wird magisch erweitert. Jeder Schüler geht mit zehn Sekunden Verzögerung rein. Dort bekommt er verschiedene Aufgaben, welche zunehmend schwerer werden. Irgendwann steht er vor einem Problem, das er nicht lösen kann. Dann kommt der Ausgang und die Punkte werden analysiert. Wenn er den Schulstoff beherrscht, dann gibt es ein Annehmbar, hat er darüber hinaus eigene Ideen, gibt es ein \note{Erwartungen übertroffen} und wenn man den ganzen Parcours fast ohne Fehler geschafft hat, dann gibt es ein Ohnegleichen. Je schlechter man allerdings gegenüber dem Schulstoff wird, das heißt, je weniger man weiß, desto niedriger wird die Note.}

\enquote{Aber warum das Dämo\aabs lebendige Feuer?}

\enquote{Jeder, der es so weit schafft, kommt in eine Zone, die sehr kalt ist. Kein Zauber wirkt. Nur das lebendige Feuer wird ihn oder sie wärmen. Deshalb habe ich es dran genommen. Die Schüler müssen lernen, Zauber nicht nach Gut und Böse einzuteilen, sondern nach Notwendigkeit und Anwendungszweck. Wenn sie das Feuer einsetzen, um sich selber zu wärmen, oder die Kälte zu bekämpfen, ist es ein positiver Einsatz.}

\enquote{Schön und gut, soweit bin ich d'accord\footnote{So gut wie einverstanden, Ok, abgemacht}, aber wie wollen Sie sie überwachen? Mit einem Zauber? Diese kann man auch austricksen. Und die Schüler sind sehr gewieft.}

\enquote{Hauselfen, dachte ich mir. Ich habe schon bei den Elfen in der Küche angefragt, ob sie Lust darauf hätten. Das war vor drei Tagen. Wenn ich eine positive Antwort erhalte, dann wäre ich zu Ihnen gekommen und wollte das Ganze mit Ihnen durchgehen. Aber wenn Sie wollen, dann können wir das auch jetzt machen.}

Während des Gesprächs, warte Ron auf Harry im Gemeinschaftsraum, da dieser noch sein ausgeliehenes Buch zurückgab.

\enquote{Gibt es sonst noch irgendwo Bücher, welche die Mondbibliothek betreffen?}, fragte er Madame Pince.

\enquote{Leider nein, Mister Potter. Aber ich kann gerne einmal nachschauen.} Sie schwang ihren Zauberstab und kurz darauf erschien ein Pergament in der Luft. Sie schaute was darauf stand und zeigte das Pergament schließlich Harry. \enquote{Sonst gibt es nichts.}

Auf dem Pergament waren lediglich zwei Bücher aufgelistet. Eines davon hatte sich Harry bereits ausgeliehen und zurückgebracht. Das andere hatte er in der Bibliothek gelesen.

\enquote{Danke Madame Pince.} Dann ging er.

Nach dem Essen und zurück im Gemeinschaftsraum fragte Harry nach einer Partie Schach. Ron willigte ein, weil er wusste, dass er Harry mit Sicherheit schlagen würde. Doch Harry wusste, was er zu tun hatte. Er schloss kurz die Augen, nahm mit Luna Verbindung auf und spürte ihre Anwesenheit. Er spürte wie sie auf einem Stuhl im Gemeinschaftsraum der Ravenclaws saß und durch seine Augen und Ohren wahrnahm, was Harry sah und hörte. Ron begann das Spiel, und Harry entspannte sich, sodass er das Gefühl hatte, ein Fremder in seinem Körper zu sein. Er sah zu, wie er seine Hand hob und die einzelnen Züge fuhr. Nachdem er, oder besser Luna, Ron zweimal geschlagen hatte, wollte er sein Glück selber versuchen und glitt wieder in seinen Körper zurück. Luna war noch immer präsent, um ihn zu beobachten oder Tipps zu geben. Harry gewann knapp, nachdem er sich von Luna ein paar Züge abgekupfert hatte. Ron war sprachlos.

\enquote{Wie hast du das geschafft?}, fragte Ron.

\enquote{Übung}, entgegnete ihm Harry, \enquote{einfach nur Übung.}

\enquote{Ja, aber wie, wer?}

\enquote{Glaubst du ernsthaft, dass ich dir meine Quellen oder Übungspartner verrate?}

Ron grinste. \enquote{Jetzt weiß ich, warum du dich immer mal wieder nachts herumgetrieben hast und nicht im Bett warst. Du hast heimlich geübt.}

\enquote{Vielleicht}, antwortete ihm Harry.

Ron grinste und meinte dann nur: \enquote{Du hast dich verbessert. Dann muss ich jetzt wohl aufpassen, wenn ich mit dir spiele.}

\trenn

Der Samstag verlief angenehm ruhig und wurde nur durch die Hausaufgaben hinausgezögert. Kurz nach dem Abendessen verschwand Harry aus der Großen Halle und Ron vermutete, er würde wieder zum Schach üben gehen. Kurz darauf verschwand auch Luna, was aber nicht sonderlich auffiel. Einige Zeit später machten sich, für die anderen unbemerkt, die einzelnen Pärchen zum dritten Stock im Westflügel auf. Harry und Luna standen bereit und warteten auf die anderen. Als Dean und Amanda auftauchten, bekam Harry einen Kloß im Hals, da er immer noch nicht wusste was Dobby ihnen geschrieben hatte. Er hatte zwar einen weiteren Brief von Dobby erhalten, in dem er ihm die Anzahl der Pärchen mitteilte, aber nicht was sie wussten.

\enquote{Oh Harry}, sagte Dean erstaunt, als er Harry sah. \enquote{Hast du auch einen Brief bekommen. Und}, dann erst erblickte er Luna und sein Gesicht blieb stehen. \enquote{Du und Luna? \gst Seit wann seid ihr?}

\enquote{Ein paar Wochen schon}, entgegnete ihm Luna.

Langsam füllte sich der Gang und als Harry feststellte, dass fast alle Paare da waren, merkte er, dass wohl keiner wusste, dass er auf sie wartete. Langsam wurden die Paare unruhig, also ergriff Harry das Wort.

\enquote{Ich freue mich, dass ihr alle hier seit, obwohl ich nicht weiß, was in euren Briefen stand, möchte ich euch begrüßen und}, begann Harry als Dean ihn unterbrach.

\enquote{Du weißt nicht, was da drinnen stand? Du hast doch auch einen bekommen.}

\enquote{Na ja, Luna und ich wissen schon etwas länger, dass\abs}

\enquote{Was?}, sagte Dean erstaunt, \enquote{warum bist du dann hier?}

\enquote{Luna und ich werden euch einweisen.}

Den anderen fiel die Kinnlade herunter.

\enquote{Na ja}, machte Dean weiter, \enquote{es stand nur drin, dass, wenn wir einen Platz für uns Pärchen haben wollen, wir uns heute hier melden sollen. Ich wusste nicht, dass du\abs}

Doch Harry antwortete: \enquote{Und ich dachte schon, als du mich danach angeschaut hattest, du wüsstest, dass ich das sein werde.}

\enquote{Nein}, sagte Dean.

Plötzlich blieb Harrys Gesicht stehen, als er Draco Malfoy um die Ecke schauen sah.

\enquote{Malfoy?}, rief er. Die anderen drehten sich herum und schluckten ebenfalls; so als hätten sie etwas Verbotenes gemacht. Doch Malfoy kam um die Ecke dicht gefolgt von Maria Mayquen. \enquote{Was machst du hier?}, fragte Harry.

\enquote{Was machst du hier, Potter?}, antwortete Malfoy spöttisch.

\enquote{Ich unterhalte mich mit meinen Kollegen.}

\enquote{Alles Pärchen, wie?}

\enquote{Du bist aber auch nicht allein hier, wie? Hast wohl auch einen grauen Brief bekommen.}

In diesem Moment viel Dracos Kinnlade runter. Er wollte gerade anfangen wieder etwas zu sagen, als ihn Harry unterbrach.

\enquote{Alles, was ihr hier und jetzt erfahren werdet, dürft ihr niemandem erzählen}, sagte Harry.

Die ganze Meute sagte fast gleichzeitig: \enquote{Verstanden.}

\enquote{Und wieso kommst du auf die Idee, ich würde meinen Mund halten?}, fragte Malfoy.

\enquote{Nun, ich denke, dass du genauso froh darüber bist einen Zufluchtsort vor Mister Filch und den anderen Lehrern zu haben, dass du garantiert hier niemanden verpfeifen wirst.}

Luna drehte sich herum und sprach gleichzeitig mit Harry, der sich immer noch Malfoy zu wandte: \zauber{Aqua Neros.} Das Porträt öffnete sich und Luna verschwand. Harry hielt seine Hand hin, als wollte er die ganzen Pärchen hinein delegieren. Er ging als Letztes durch das Loch hinter dem Porträt, worauf es sich schloss. Dann kämpfte er sich durch die doch jetzt große Menge an Personen und stellte sich neben Luna. Gerade so als hätten sie es eingeübt, ergänzten sich ihre Sätze passend zu dem des anderen.

\enquote{Wir freuen uns, euch hier und heute zu sehen}, sagte Harry.

\enquote{Es ist selbstverständlich, dass hiervon keiner außer den hier anwesenden wissen darf.}

\enquote{Was ist mit den Lehrern? Wissen die hiervon?}, fragte ein Siebtklässler aus Hufflepuff, den Harry nur anhand der Uniform einordnen konnte.

\enquote{Nein}, entgegnete Luna. \enquote{Nur die Hauselfen und wir wissen von diesem Ort.}

\enquote{Die Hauselfen}, fuhr Harry fort, \enquote{werden ab heute jeden Tag hier sein und wenn nötig die entsprechenden Aufräumarbeiten vornehmen.}

\enquote{Hier hinter uns findet ihr eure Schlafzimmer.}

\enquote{Alles Doppelbetten.}

\enquote{Aber passt auf, der erste Raum\abs}

\enquote{\aabs gehört schon uns.}

\enquote{Da steht unser Name drauf.}

\enquote{Wie habt ihr von diesem Ort erfahren?}, fragte Amanda, Deans Freundin.

\enquote{Durch Dobby}, antwortete Harry.

\enquote{Der ehemalige Hauself der Malfoys?}, fragte Dean nach.

\enquote{Ja}, antwortete Harry. \enquote{Der hat uns in unserer dritten Woche diesen Ort gezeigt.}

\enquote{Und da kommst du erst jetzt?}, fragte Draco.

\enquote{Dobby hat uns gesagt, es sei noch zu früh für die anderen. Und ich musste ihm recht geben.} Er wurde leicht rot als Luna frei heraussagte: \enquote{Wir mussten ihn erst ausprobieren und verbrachten einige Nächte hier.}

Den anderen blieb das Gesicht stehen. Einige ließen ihre Kinnlade herunterfallen.

Als Dean seine Fassung wiedergefunden hatte, fragte er Harry \enquote{Ihr beide habt miteinander?}

\enquote{Nein}, antworteten Harry und Luna. Und Luna fügte hinzu: \enquote{Aber nahe dran. Glaubst du, wir haben keine Kontrolle über unsere Körper?}

Harry wurde rot. Doch seine Röte verblasste gleich wieder. \enquote{Schaut euch hier ruhig etwas um}, fügte Luna hinzu. Sie nahm Harrys Arm und zog ihn zu dem Schachtisch hinüber, drückte ihn auf den Stuhl und setzte sich auf seinen Schoß. Danach legte sie ihren Arm um seinen Hals und ihren Kopf auf seiner Schulter ab. Dean und Amanda beobachteten sie dabei. Dean konnte sich ein Grinsen nicht verkneifen, schaute zu Amanda um küsste sie.

Währenddessen machten sich Ron und Hermine im Gemeinschaftsraum Sorgen um Harry.

\begin{rueckblick}
Gerade als Luna und Harry wieder im Gemeinschaftsraum der Paare waren und es sich in einem Stuhl gemütlich gemacht hatten, schwebte ein Buch hinter dem Wandteppich hervor, direkt auf Luna und Harry zu. Es blieb wenige Zentimeter vor ihnen stehen und begann sich zu öffnen. Es blätterte auf eine bestimmte Seite und begann zu leuchten, um auf sich aufmerksam zu machen. Harry und Luna ließen voneinander ab und Harry las vor:

\enquote{Jedes Paar, egal ob es gerade diesen Raum hier benutzt, oder früher einmal benutzt hat, kann keinem Fremden gegenüber etwas über diese Räume erzählen. Zum Teil gibt es Schutzzauber, welche die Erinnerung blockieren oder verändern, zum anderen können auch schwere Schmerzen auftreten. Dies hängt vom Charakter des Individuums ab, das versucht diese Räumlichkeiten zu verraten.} Harry grinste erleichtert. Jetzt war er sich absolut sicher, dass ihn Malfoy nicht verraten würde.
\end{rueckblick}

Am anderen Morgen stand Harry auf und war gerade auf dem Weg zur Dusche, als ihn Luna am Handgelenk fasste. Sie stand auf, küsste ihn und meinte: \enquote{Willst du etwa ohne mich duschen?}

Harry lächelte sie an und sagte: \enquote{Ich wollte dich nicht aufwecken.} Er löste ihren Griff, nahm sie bei der Hand und führte sie zu den Duschen. Nachdem Harry die Tür geöffnet hatte und ein paar Schritte mit Luna hineingegangen war, bemerkte er Dean und Amanda in einer Duschkabine. Die beiden erschraken, als sie Luna und Harry sahen, lächelten aber dann, als sie merkten, dass sie genauso nackt waren wie Harry und Luna.

Dean und Amanda widmeten sich wieder sich selber und Luna zog Harry in eine weitere Duschkabine. Gerade als sie ihn eingeseift hatte, kam Dean ums Eck und meinte: \enquote{Hier ist es ja traumhaft.}

\enquote{Deswegen seid ihr ja auch hier}, antwortete Luna.

Dean fing an zu lachen und auch Amanda, die hinzugekommen war, lächelte.

\enquote{Lasst uns abtrocknen, anziehen und dann frühstücken gehen}, schlug Harry vor.

\enquote{Ok}, antworteten die anderen.

\enquote{Wir sollten aber nicht immer nachts hier sein. Sonst fällt das nur auf.}

Nach einem gemütlichen Sonntag und ein paar Runden Schach gegen Hermine, ging Harry ins Bett.

Am Montag darauf hatte Harry wieder Unterricht bei Hagrid. Als Harry Richtung Hagrids Hütte lief, bemerkte er, dass ein großes Areal mit Schnee bedeckt war. Es war klar abgegrenzt, sodass Harry vermutete, es sei herbeigezaubert worden. Außerdem meinte Harry, dass sich da etwas bewegen würde, denn es waren viele Berge voll von Schnee zu sehen. Jeder hatte eine andere Größe und Form.

\enquote{Heut nehm mer Schneewalker durch}, sprach Hagrid.

\enquote{Schneewalker sin ganz friedliche Geschöpfe und ihre, naja, des was hinten rauskommt, is auch ganz nützlich. Wir werdn uns heut etwas um sie kümmern, sie füttern und warten, bis’ mit dem Verdauen fertig sind. Dann sammelt des Ganze auf und bringt’s in’n Krankenflügel.}

Sämtliche Schüler verzogen das Gesicht bei dem Gedanken, schon mal etwas von Schnee\-wal\-ker-Ex\-kre\-men\-ten bekommen zu haben, als sie im Krankenflügel gewesen waren. Sie gingen mit Hagrid zwischen die Schneehaufen. Als sie mitten im Schnee standen, drehten sie sich um und sahen ein paar von den eigenartig aussehenden Tiere. Etwas, das wie ein Hirsch mit Flügeln und kleinem Geweih aussah, näherte sich ihnen. Es war ganz weiß und näherte sich Malfoy von hinten. Als er sich umdrehte, erschrak er. Das Schneewalker-Männchen kam einen Schritt näher und schnupperte an ihm.

Malfoy zitterte leicht, doch er blieb stehen.

Plötzlich wirbelte Harry herum, da er eine Zunge an seinem Haaransatz spürte. Vor ihm stand ein weißes Etwas. Dann erinnerte er sich, dass sie Schneewalker durchnahmen. Es war ihm für kurze Zeit entfallen. Jetzt spürte er allerdings dieselbe Zunge im Gesicht. Der Schneewalker hatte anscheinend an ihm gefallen gefunden. Oder er hatte Salzmangel und leckte ihn deshalb ab. Harry besah sich das Tier genauer und merkte, dass es ein Männchen war. Er strich ihm über das Fell und war erstaunt darüber, wie weich es war. Doch trotz alle Bemühungen durch reden und füttern sowie streicheln, hatte Harry heute kein Glück. Das Männchen, das er betreute, machte keinen Haufen, den Hagrid aufsammeln konnte. Aber trotzdem musste er sich das Männchen und seine Fellzeichnung einprägen, damit er ihn in der nächsten Stunde wieder finden konnte. Da das Tier aber weiß war, war das mit der Fellzeichnung gar nicht so einfach. Man musste sehr genau hinsehen, damit man eine leichte Schattierung im Fell wahrnehmen konnte.

\trenn

Donnerstagnachmittag hatte Harry wieder Geschichte der Zauberei bei Professor Binns. Es hatte bereits geläutet und alle bereiteten sich darauf vor, dass Professor Binns durch die Tafel schwebte, doch nichts passierte. Plötzlich öffnete sich die Tür zum Klassenzimmer und Professor Elber kam herein.

\enquote{Einen schönen Nachmittag, Klasse}, sagte Professor Elber, als er durch die Reihen zum Lehrerpult durchlief.

Etwas irritiert kam ein \enquote{Guten Nachmittag Professor Elber} zurück.

\enquote{Professor Binns ist leider für etwa eine Woche weg. Er sprach etwas von einem vierhundertjährigen Geburtstag und sieht sich daher außer Stande zu unterrichten.}

Ein Murmeln ging durch die Klasse. \enquote{Also}, sprach Professor Elber weiter, \enquote{da ich nicht weiß, was Professor Binns bisher durchgenommen hat und noch unterrichten möchte, er hatte es nämlich ziemlich eilig, dachte ich, machen wir eine Fragestunde. Was wollt ihr wissen?} Stille herrschte im Klassenzimmer und ein paar Schüler weckten ihre bereits schlafenden Nachbarn. Professor Elber grinste als er dies sah. \enquote{Professor Binns scheint wohl eine einschläfernde Wirkung zu haben?}

Hermine war die erste, die wieder die Hand hob, und Professor Elber fragte. \enquote{Professor, wann kamen eigentlich die ersten Magier und Hexen auf die Welt?}

Professor Elber schaute sie ernst an und trat dann vor den Reihen auf und ab, seinen Blick auf den Boden gesenkt. Dann blieb er stehen und schaute Hermine an. Ihr Gesichtsausdruck zeigte einen Hauch von Panik. Harry hatte den Eindruck, dass sie bereute, die Frage gestellt zu haben.

\enquote{Nun, vor über sechstausend Jahren gab es noch keine Magier. Eines Tages dann wurde eine Person geboren, die schon sehr früh feststellte, dass sie über besondere Fähigkeiten verfügte. Diese könnte man als den ersten Magier bezeichnen.} Er machte eine Pause und lief wieder umher. \enquote{Irgendwann begann er zu heiraten und Nachwuchs zu zeugen. Seine Kinder erbten seine Fähigkeiten und er begann sie auszubilden. Damals dauerte es wesentlich länger, da die Entwicklung der Zauberei noch nicht so weit fortgeschritten war. Er merkte, dass seine Frau und seine Kinder alterten, spürte selber aber nichts dergleichen. Er entschloss sich sein aussehen dem seiner Frau anzupassen und als es an der Zeit war zu verschwinden, ging er. Damals war es nichts Besonderes, wenn jemand von einer großen Reise nicht zurückkam. Er verjüngte wieder sein Aussehen und entschloss sich, die Population der Magier und Hexen zu vergrößern. Ob er damals an Heirat untereinander dachte, ist nicht bekannt, aber man geht allgemein davon aus. Irgendwann wurde es ihm zu mühselig, immer wieder neue Magier und Hexen heranzubilden, und er entschloss sich, die Grundlagen der heutigen Magie zu legen. Dann rief er alle lebenden Zauberer und Hexen zusammen.} Er wanderte wieder durch die Reihen und lehnte sich an das Lehrerpult, verschränkte die Arme und erzählte weiter.

\enquote{Er brachte ihnen die neue Form der Magie bei und erteilte ihnen den Auftrag, diese weiterzugeben.} Er drehte sich um und ging zur Tafel. \enquote{So wie wir heute zaubern} und er malte vier Blöcke untereinander, \enquote{bauen wir auf der obersten Schicht auf. Wir benutzen unseren Zauberstab, um Magie auszuführen. Dies erleichtert uns das Zaubern.} Er füllte den oberen Kasten mit grün aus und fuhr fort. \enquote{Nur bei einigen Sachen können wir keinen Zauberstab verwenden. Denken wir mal an die Verwandlung eines Animagus. Oder habt ihr schon mal einen Animagus gesehen, der sich mit einem Zauberstab antippte?}

Allgemeines Kopfschütteln. Er füllte den zweiten Block in Blau aus. \enquote{Über die anderen Blöcke ist heute so gut wie nichts bekannt. Also fragt lieber nicht.} Professor Elber stellte sich wieder vor das Lehrerpult und lehnte sich an.

\enquote{Aber wieder zurück zu unserem Magier. Irgendwann entschloss er sich, sich zur Ruhe zu setzen und nicht mehr zu unterrichten. Er baute ein Schloss und verbrachte viel Zeit darin. Dann kam die Zeit der großen Kriege, an denen er vielfach beteiligt war, und am Ende hatte er genug und überließ sein Schloss den vier größten Magiern ihrer Zeit und verschwand. Keiner weiß wohin. Er stand natürlich auf der anderen Seite.}

Er stieß sich vom Pult ab und fragte erneut: \enquote{Gibt es sonst noch etwas, was ihr wissen wollt?}

\enquote{Ja}, fragte Draco Malfoy. \enquote{Wer waren die vier?}

\enquote{Ich glaube, ihr kennt sie}, sagte Professor Elber vergnügt. \enquote{Es waren die Gründungsmitglieder von Hogwarts.}

\enquote{Was?}, klang es durchs Klassenzimmer.

\enquote{Das kann unmöglich sein Professor Elber}, sprach Hermine. \enquote{Im Buch Geschichte von Hogwarts\abs}

\enquote{Halt}, unterbrach sie Professor Elber. \enquote{Das wurde nie im Buch erwähnt. Er wollte nicht, dass sein Name mit diesem Gemäuer in Verbindung gebracht wurde. Sonst wärt ihr alle jetzt nicht hier.}

\enquote{Wie hieß er denn?}, fragte Ron.

\enquote{Sein Name war Friedward Alejious Elberon.} Ein Raunen ging durch die Klasse. \enquote{Er war Zeitweise unter dem Namen \enquote{Schlächter von Nervot} bekannt und wo wir schon beim Thema sind, gibt es noch einen Fehler in der Geschichte Hogwarts}, fuhr Professor Elber fort. \enquote{Betrachten Sie doch einmal das Wappen.}

Er drehte sich zur Tafel und mit dem Schwung seines Zauberstabes erschien das Wappen auf der Tafel. Er drehte sich zur anderen Seite und zauberte vier weitere Tafeln herbei und zog jedes einzelne der vier Wappenteile auf die vier Tafeln, um sie dort vergrößert anzuzeigen. Links das der Slytherins, rechts davon das der Hufflepuffs, daneben das der Ravenclaws und ganz rechts das von Gryffindor.

\enquote{Nun}, fuhr er fort, \enquote{irgendjemand eine Idee?} Er lief wieder durch die Reihen. Totenstille. Kein Laut war zu hören.

Hermine machte nur große Augen. \gedanke{Ein Fehler in der Geschichte Hogwarts? Wieso habe ich das nicht bemerkt?}

Als Professor Elber am Ende des Raumes angelangt war, lehnte er sich gegen die Wand und zeigt mit seinem Zauberstab auf die Tafel mit dem Wappen von Slytherin. Das Logo begann sich herauszulösen und schwebte als dreidimensionales Gebilde in der Luft.

\enquote{Irgendjemand eine Ahnung?}, wiederholte Professor Elber.

Hermine streckte die Hand und sprach: \enquote{Sollte die Schlange vielleicht ein Basilisk sein?}

\enquote{Wie kommen sie darauf, Miss Granger?}

\enquote{Nun ja, sie haben gerade dieses Logo ausgewählt. Also dachte ich, es müsse etwas mit einer Schlange zu tun haben. Und das ähnlichste einer Schlange ist eben ein Basilisk.}

\enquote{Interessanter Ansatz}, sagte Professor Elber. \enquote{Aber ich habe einfach irgendwo angefangen. \gst Nein, das ist kein Basilisk.} Einige Minuten verstrichen, während der alle ratlos waren und ihre Köpfe anstrengten.

Er schwang wieder seinen Zauberstab und das Logo kehrte zurück zu seiner Tafel. Das nächste Logo kam von der Tafel und schwebte in den Raum hinein.

\enquote{Wie sieht es hier aus?}

Das Hufflepuff-Logo schwebte im Raum und zeigt einen Dachs. Betretenes Schweigen herrschte. Nach einer Weile schwang Professor Elber wieder seinen Zauberstab und der schwebende Dachs kehrte zurück und der Adler von Ravenclaw schwebte im Raum. Immer noch schwiegen die Schüler. Professor Elber schwang ein letztes Mal seinen Zauberstab und die schwebende Figur kehrte wieder zurück auf die Tafel und der Löwe der Gryffindors begann sich von der Tafel zu lösen und im Raum zu schweben. Wieder sagte keiner etwas.

Professor Elber tippte ungeduldig mit seinem Zauberstab umher und fing wieder an, durch den Raum zu laufen. Er drehte sich um und zeigte auf die ersten drei Tafeln, sodass nun alle Figuren im Raum schwebten. \enquote{Nun, was ist? Irgendjemand \gst eine Ahnung.} Immer noch schwiegen alle. Professor Elber sprach jetzt leicht sauer: \enquote{Also, wenn ihr nicht mitarbeitet, dann kann das hier nichts werden. Ich glaube, ich muss Professor Binns mal erzählen, wie in seiner Klasse die Arbeitsmoral ist \gst wenn ihr keine Ahnung habt, dann ratet wenigstens. Schaut euch die Figuren an. Gibt es irgendwelche Tiere, die ähnlich aussehen?}

Die Klasse zuckte zusammen. Keiner von ihnen hatte Professor Elber derartig zornig gesehen. Immer war er bisher freundlich gewesen und noch nie hatte er einen Ton drauf, der einen zusammen zucken ließ. Außer vielleicht dem berechtigten Aufbrausen über ihre bisherigen VgddK-Lehrer.

Jetzt meldete sich ein Hufflepuff. \enquote{Es könnte vielleicht ein Greif sein? Ein goldener Greif. Eine Mischung zwischen einem Greifen und einem Löwen.}

Professor Elbers Miene zeigte jetzt wieder einen fröhlicheren Ausdruck. \enquote{Wie kommen sie darauf?}

\enquote{Na ja, ein goldener Greif sieht einem Löwen ähnlich. Und das Haus heißt schließ\-lich auch Gryf\-fin\-dor.}

Professor Elber lief wieder durch die Klasse und meinte: \enquote{Zehn Punkte für Hufflepuff. Das war eine kluge Schlussfolgerung. Eigentlich sollte ich die zehn Punkte Gryffindor abziehen, weil keiner von ihnen darauf gekommen ist.} Am Pult vorne angekommen drehte er sich wieder um. \enquote{Es ist in der Tat so, dass hier der Name Gryffindor nicht nur für den Namen des Hausgründers steht, sondern auch der Greif das Wappentier ist. Genauer gesagt ein goldener Greif, so wie man die Löwen-ähnlichen Wesen auch nennt.}

\trenn

Als Harry am nächsten Freitag zum Frühstücken kam und die Treppen zur Großen Halle hinunterging, lagen dort Fred und George, sich die Hände vor den Bauch haltend und lachend, auf dem Boden. Ihr Lachen schallte durch die ganze Große Halle. Er war mit Ron und Hermine zum Frühstücken unterwegs, und die drei blieben vor Fred und George stehen.

\enquote{Was macht ihr da?}, fragte Ron die beiden.

Fred und George standen auf und fingen an zu erzählen.

\enquote{Also}, erzählte George, \enquote{wir wollten gerade frühstücken, da lief uns Dumbledore über den Weg.}

\enquote{Er murmelte etwas von Verstopfung und Bauchschmerzen}, fügte Fred hinzu.

\enquote{Also haben wir ihm eines unserer Toffees angeboten.}

\enquote{Unsere neueste Kreation, Montezumas-Kicher-Toffees.}

\enquote{Was ist das denn?}, fragte Harry.

Und George antwortete ihm: \enquote{Eine erweiterte Form der Montezumas Rache Toffees.}

Harry, Ron und Hermine blieb das Gesicht stehen.

\enquote{Ihr habt doch nicht etwa unserem Schulleiter ein Abführmittel gegeben?}, fuhr Hermine entsetzt dazwischen.

\enquote{Oh doch}, antwortete George, der sich immer noch den Bauch vor lauter Lachen hielt.

Hermine schüttelte nur den Kopf, während Harry und Ron zu grinsen begannen. Sie zog die beiden in die Große Halle und begann zu frühstücken. Bereits nach einer viertel Stunde kam Dumbledore in die Große Halle und lief an Harry vorbei, welcher nur ein paar Worte aufschnappte.

\enquote{Tolle Dinger, diese Toffees. So viel Spaß hatte ich schon lange nicht mehr. Ich muss die Weasley-Zwillinge unbedingt fragen, ob sie mir noch mehr davon haben}, murmelte Dumbledore im Vorbeigehen und als er Harry sah: \enquote{Nur für den Fall der Fälle.}

Man konnte Harry ansehen, dass er genau wusste, was Dumbledore meinte. Und Dumbledore schien auch aufzufallen, dass Fred und George es den dreien erzählt hatten. Fred und George hatten zwar die Schule verlassen, kamen aber immer mal wieder zu Besuch. Sie vertrieben ihre Waren und blieben meist zum Mittagessen. Natürlich wurden sie hinausgeworfen, wenn sie entdeckt wurden, was aber nicht immer passierte. So waren sie einmal im Monat für eine Stunde da.

Diese Woche hatte Harry am Nachmittag, nach Snape, wieder Kräuterkunde bei Professor Sprout. Sie mussten wieder ein paar Pflanzen umtopfen und aus den Blättern einen Sud brauen, der etwaige offene Fleischwunden vor Infektionen schützen konnte. Der Nachmittag verlief ruhig, und Harry konnte am Abend den Großteil seiner Hausaufgaben erledigen. Am Samstagvormittag war Quidditch-Training angesetzt und Harry war erschöpft und hungrig, als er die Große Halle passierte, um zu Mittag zu Essen.

\trenn

Harry war mit Luna auf dem Weg zum dritten Stock, als er kurz vor dem Ziel auf Professor Elber stieß, der ein Bild an der Wand betrachtete.

\enquote{Professor Elber, was machen sie denn hier?}, frage Harry.

\enquote{Ah, Harry, Luna, ich schaue mir ein paar Bilder an. Ich bin auf der Suche nach etwas}, antwortete Professor Elber.

\enquote{Auf der Suche nach etwas Bestimmtem?}

\enquote{Sagen wir eher nach etwas, was da sein muss, von dem ich aber nicht weiß, wo es sich genau befindet, oder wie es genau aussieht.}

\enquote{Können wir ihnen helfen?}, fragte Luna.

Harry bekam plötzlich Herzklopfen, als er das hörte und Luna merkte auch, dass sie das wohl besser nicht gesagt hätte. \gedanke{Wärst du bloß still gewesen}, dachte Harry und hörte von Luna ein: \gedanke{Upps, Entschuldigung.}

\enquote{Sagen euch die Namen Sardak Slyhoot und Selvine Vertap etwas?} Harry bekam große Augen und auch Luna erweckte den Eindruck, die Namen schon einmal gehört zu haben. Professor Elbers Mund zeigte ein leichtes Schmunzeln. \enquote{Wisst ihr, die beiden haben mich mal, es ist jetzt schon einige Zeit her, besucht. Sie waren auf der Suche nach einem ruhigen Plätzchen und ich habe ihnen ein paar Sprüche verraten, die es ihnen ermöglichen sollten, ihr Ziel zu erreichen. Hier ungefähr}, er machte eine ausladende Geste durch den ganzen Flur, \enquote{sollte eine Art Kammer sein, in der die beiden ungestört sein konnten.} Harrys und Lunas Gesicht färbte sich leicht rot. \enquote{Ich nehme an, ihr wisst, von was ich spreche.} Luna und Harry blieben standhaft und verzogen keine Miene. Plötzlich runzelte Professor Elber leicht die Stirn und sein Kopf schwenkte zum Porträt, das an der Wand hing. Er fing an leicht zu grinsen, schloss die Augen und meinte dann kurz darauf: \enquote{Dann will ich euch mal nicht weiter stören. Es scheint, als ob dieser Ort benutzt wird. Ich hätte ihn mir gerne einmal angeschaut. Aber das ist jetzt wohl leider nicht mehr möglich.} Er drehte seinen Kopf wieder zu Harry und Luna und ging Richtung Treppenaufgang, woraufhin er kurz im Gewirr der sich ständig bewegenden Treppen verschwand.

Harry und Luna schauten sich nur fragend an und waren verwirrt. Eine Stille erfüllte plötzlich ihre Köpfe und beide erschraken, als sich ihnen ein Pärchen näherte, das kurz zuvor aus dem Loch hinter dem sich öffnenden Porträt stieg. Beide atmeten erleichtert auf, als sie die beiden sahen und begannen, sich auf das immer noch offen stehen Porträt zu zubewegen. Innen angekommen setzten sich beide und starrten sich mit ausdruckslosen Gesichtern an. Ein Hufflepuff kam vorbei und setzte sich auf den freien Platz neben Harry.

\enquote{Hi Harry}, sagte er fröhlich gelaunt.

\enquote{Hi Donan}, kam es aus Harry heraus und das, obwohl er nicht mal seinen Namen wusste.

\enquote{Woher kennst du meinen Namen?}, fragte Donan.

Harry war erst jetzt wieder voll bei Sinnen und sagte: \enquote{Luna hat ihn mir gesagt.} Das war zum Teil richtig, denn Luna kannte ihn. Aber sie dachte nur seinen Namen.

\enquote{Du schaust bedrückt aus Harry.}

\enquote{Weißt du, es ist Folgendes\abs} und Harry und Luna erzählten von ihrem Zusammentreffen mit Professor Elber.

\enquote{Der wird mir so langsam unheimlich, dieser Professor Elber}, meinte Donan, \enquote{Aber, wenn er wie du sagtest, die beiden getroffen hat und die Angaben in dem Buch stimmen, welches dort hinter dem Wandteppich steht, dann müsste er ja über 128 Jahre alt sein. Und wenn er ihnen die Tricks verraten hat, dann muss er bestimmt schon die Zauberschule abgeschlossen und einige Jahre praktische Erfahrungen gesammelt haben. Das würde dann bedeuten er wäre über 150 Jahre, so grob geschätzt. Und dafür sieht er mir doch noch zu jung aus.}

Harry dachte nach.

\enquote{Also, entweder lügt er uns an, oder er ist wirklich so alt und hat sich nur verplappert. Denn ich glaube kaum, dass er es jemanden auf die Nase binden wollte.}

\enquote{Bei dem, was er uns schon während des Unterrichts erzählt und auch gezeigt hat, glaube ich schon, dass er einige Jahre auf dem Buckel hat}, meinte Harry. \enquote{Aber so alt? Wie alt werden denn Zauberer?}, fragte er nach.

\enquote{Schon so um die 150~Jahre. Die sehen dann aber so aus wie Dumbledore, oder noch faltiger}, sagte Donan.

\enquote{Harry?}, sagte Luna plötzlich, \enquote{hat er nicht etwas über die ersten Magier erzählt?}

\enquote{Ja Luna}, antwortete Harry; unsicher, worauf sie hinaus wollte.

\enquote{Wie wäre es, wenn er dieser Magier ist?}

\enquote{Welcher Magier?}, fragte Donan.

\enquote{Der erste}, antwortete Luna. \enquote{Über siebentausend Jahre alt.}

Das kam den beiden dann doch etwas abwegig vor.

\trenn

Als Harry eines Morgens wieder mit Luna das Schloss durchstreifte, traf er auf Ron und Hermine, die sich mit Professor Elber unterhielten. Er näherte sich ihnen, um ihre Unterhaltung mit anzuhören.

\enquote{Guten Morgen, Harry, Luna}, kam es ihnen entgegen.

Er bemerkte nicht, dass Professor Elber auf seine Brust schaute.

Dann sprach Professor Elber ihn an. \enquote{Ein interessantes Amulett haben sie da.} Harry schaute zuerst seinen Professor an und danach auf seine Brust. \enquote{Ein Erbstück?}, fragte Professor Elber.

\enquote{Nein}, antwortete Harry. \enquote{Ein Geburtstagsgeschenk.}

Professor Elber hob erstaunt seine Augenbrauen und öffnete seine Augen. \enquote{Die Person, die ihnen das geschenkt hat, muss sie sehr, sehr gernhaben.}

\enquote{Wieso?}, fragte Harry. \enquote{Wissen Sie was für ein Amulett das ist?}, fragte er Professor Elber.

Ginny bog gerade um die Ecke und sagte: \enquote{Ich habe ihm das geschenkt.}

\enquote{Wo haben sie das eigentlich her?}, fragte Professor Elber.

\enquote{Von Borgin und Burkes.}

Elbers Augen weiteten sich. \enquote{Wie die da wieder herangekommen sind?}, fragte sich Professor Elber nun.

\enquote{Was hat es mit dem Amulett auf sich?}, fragte Hermine.

\enquote{Es gehörte einst \gst nun ja, gehören ist vielleicht übertrieben, er hat es erschaffen.}

\enquote{Wer?}, bohrte Hermine nach.

\enquote{Salazar Slytherin.} Die fünf keuchten. \enquote{Das Amulett stammt von ihm. Er hat es ursprünglich als Hochzeitsgeschenk erschaffen.} Noch immer schauten die fünf erstaunt. \enquote{Ach Harry, sehen sie was, wenn sie das Amulett in die Hand nehmen und ihre Augen schließen?}

Harry wurde zunehmend unwohl. \enquote{Ja}, antwortete er zögerlich.

\enquote{Dann sind sie in direkter Linie ein Nachfahre von Slytherin. Nur einer seiner Nachfahren ist in der Lage, etwas durch das Amulett zu sehen.}

\enquote{Aber}, machte Harry weiter. \enquote{Als ich in der Kammer Godric Gryffindors Schwert aus dem sprechenden Hut zog, meinte Professor Dumbledore, ich sei ein wahrer Gryffindor.} Jetzt keuchte Professor Elber. Unsicher drehte er seinen Kopf und suchte nach einer Sitzmöglichkeit. Er ging zu einer kleinen Nische in der Wand und setzte sich. Er atmete schwer. Mit einem durchdringenden Blick schaute er Harry an. Harry kannte diesen Blick sonst nur von Dumbledore.

Harry wurde wieder zunehmend unwohl. \enquote{Was hat das zu bedeuten, Professor Elber?}, fragte Harry.

Er schaute ihn lange und intensiv an, bevor er zu erzählen begann. \enquote{Sie sind einerseits ein Ur-Ur-Ur-Enkel von Salazar Slytherin, andererseits von Godric Gryffindor, scheint mir.}

Harry staunte. \enquote{Wollte mich deshalb der Hut nach Slytherin stecken? Weil ich ein Nachfahre von Slytherin bin?}, fragte Harry nach.

\enquote{Das kann gut sein. Aber sie sind auch ein Nachfahre von Gryffindor und nach dem sie in seinem Haus sind, müssen sie den Hut wohl überzeugt haben, sie hier reinzustecken.}

\enquote{Aber warum wollte dann der Hut mich nach Slytherin stecken?}

\enquote{Voldemort.}

\enquote{Voldemort?}

\enquote{Ich habe von Dumbledore gehört, dass er einige seiner Kräfte versehentlich auf sie übertragen hat. Sie sind ein Parselmund. Der Hut hat das erkannt und die Seite Slytherins ist in Ihnen scheinbar etwas stärker ausgeprägt als die Seite Gryffindors.}

\enquote{Nein}, schrie Harry.

\enquote{Sachte, sachte. Salazar war kein schlechter Mensch oder Zauberer. Er hatte seine Ideologie der Rassenreinheit erst nach der Schulgründung entwickelt. Der Hut war zu diesem Zeitpunkt schon erschaffen. Er entscheidet nach den ursprünglichen Eigenschaften von ihm. Vielleicht schlummerten einige seiner Thesen schon in ihm, als er dem Hut etwas von sich überließ. Vielleicht sind deshalb in den letzten tausend Jahren wenig Muggelstämmige oder Halbblüter in Slytherin.}

Harry stand auf. Voller Zorn darüber, dass er ein Nachfahre von Salazar Slytherin sein sollte, riss er sein Amulett vom Hals und warf es in hohem Bogen von sich.

\enquote{Nein!}, rief Professor Elber. Reflexartig sprang er hoch und streckte seine Hand nach dem Amulett aus. Wenige Millimeter vom Boden entfernt hielt es an und schwebte in der Luft. Es dreht sich noch. Doch langsam kam es zum Stillstand. Professor Elber zog seine Hand kurz zurück, worauf das Amulett in seine Hand sprang. Er umschloss das Amulett und sah Harry vorwurfsvoll an. \enquote{Wenn sie es nicht mehr wollen, dann nehme ich das gerne. Aber zerstören Sie es nicht einfach, Harry.}

\enquote{Von mir aus können sie es haben, Professor. Ich bin nicht scharf darauf.} Zornig verließ er die Gruppe und begab sich zum Speisesaal. Seine weiteren Worte konnte keiner mehr vernehmen, da er sich bereits außer Hörweite befand.

\enquote{Wie?}, fragte Hermine nach. \enquote{Sie meinen, es gab in Slytherin keine reinrassigen Zauberer oder Hexen?}

\enquote{Ja, wieso erstaunt sie das?}




\begin{kommentar}
Ein kleiner Vogel kommt während des Abendessens hereingeflogen und flattert in der Großen Halle herum. Er gehört McGonagall. Als er wieder eingefangen wurde, sagt Luna: »Jetzt weiß ich, dass Sie gut zu Vögeln sind, Professor.« Leider kann man dem geschriebenen Satz nicht so leicht die Doppeldeutigkeit entnehmen, die er in gesprochener Form haben sollte. Aber ich hoffe doch mal, dass ihn viele trotzdem entdeckt haben. (Tipp: Vögeln kann man auch kleinschreiben. <vögeln>)
\end{kommentar}

\begin{kommentar}
Als Elber das lebendige Feuer demonstriert und über die unverzeihlichen Flüche referiert, sagt er am Ende, dass er den Schülern eine Entschädigung zu deren Verängstigung anbieten möchte. Da hat er schon Das Magie in Konzert im Hinterkopf, das später stattfinden wird.
\end{kommentar}

\begin{kommentar}
Die erste Vertretungsstunde die Elber in Geschichte der Zauberei gibt, ist zugleich auch ein Hinweis, dass er der erste Magier ist, der auf der Welt entstanden ist und auch, dass er Hogwarts gebaut hatte, um es später an die vier Schulgründer zu übergeben.
\end{kommentar}
