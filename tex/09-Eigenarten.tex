\chapter{Eigenarten}


\enquote{Hhhr. Das wirft alles über den Kopf, was wir von Slytherin glauben zu wissen. Wieso sollte es uns also nicht erregen?}

Professor Elber schaute Hermine nun direkt an und meinte: \enquote{Was halten Sie von einer Geschichtsstunde über Hogwarts? Professor Binns wird sicherlich die eine oder andere Stunde mir überlassen und sich vielleicht auch dazu bereit erklären, mir in \VgddK zu helfen. Ihr braucht nämlich noch Übung, um Geister und Gespenster abzuwehren.}

\trenn

In der Nacht wachte Harry plötzlich auf, da er auf die Toilette musste. Er verließ seinen Schlafsaal und erledigte sein Geschäft. Auf dem Rückweg zu seinem Bett hörte er eine Stimme aus dem Gemeinschaftsraum. Er schlich leise in sein Zimmer und holte seinen Zauberstab, bevor er sich auf den Weg nach unten machte. Im Gemeinschaftsraum angekommen sah er eine durchsichtige Gestalt. Doch es war keiner der üblichen Hausgeister. Nicht einmal ein Geist, den er kannte.

\enquote{Hallo}, sagte Harry zögerlich.

Der Geist drehte sich um und begrüßte Harry. \enquote{Hallo Harry.}

Harry stutzte. \enquote{Wer sind Sie?}, fragte er.

\enquote{Später Harry. Setz’ dich erst einmal}, bot ihm der Geist an und platzierte sich in einem Sessel, sodass Harry sich gegenüber setzen konnte. \enquote{Zuerst einmal musst du das Amulett wieder zurückholen. Es ist im Kampf gegen Voldemort sehr wichtig.}

Harry staunte. Ein Geist, der Voldemorts Namen sagte. \enquote{Wer sind Sie?}, fragte Harry erneut.

Doch der Geist schien ihn nicht zu hören. Oder er wollte es nicht. \enquote{Du musst mein Amulett wieder an dich nehmen, Harry.}

Harry durchzuckte ein Blitz. Ihm wurde plötzlich heiß und kalt. Dann brach es aus ihm heraus. \enquote{Slytherin. Sie sind Salazar Slytherin.}

\enquote{Gut bemerkt, mein Junge. Und zudem dein Urahn. Aber zurück zum Amulett.}

\enquote{Aber}, brach es aus Harry heraus, \enquote{ich weigere mich das zu akzeptieren. Sie können nicht mein Urahn sein.}

Doch Slytherin protestierte. \enquote{Erstens, es ist so. Zweitens, es ist so, und drittens, es ist so. \gst Dich stört wahrscheinlich das weit verbreitete Gerücht, dass ich nur reinblütige Zauberer mag und etwas gegen Halbblüter oder Muggelgeborene habe. Aber dem ist nicht so. Zugegeben\abs} und so erzählte Slytherin Harry seine Geschichte. \enquote{Ich habe damals die Theorie vertreten, dass sich die magischen Künste durch die Heirat von Magiern untereinander stärken sollten. Von vielen wurde sie damals begeistert aufgenommen. Aber nach ca. einem Jahrzehnt oder zwei habe ich erkannt, dass diese genetische Inzucht untereinander zu vielen Problemen führt. Zudem brachte sie nicht das erwartete Ergebnis. Also gab ich meine Theorie auf und versuchte sie zu bereinigen. Doch meine Versuche schlugen fehl. Zu stark war inzwischen die Idee der Rassenreinheit geworden. \gst Ich schäme mich noch heute dafür.}

\enquote{Sie schämen sich dafür?}, fragte Harry.

\enquote{Na na, nicht so förmlich, Harry.}

\enquote{Also gut, Salazar.}

\enquote{Schon besser.}

\enquote{Aber warum?}, wollte Harry schon wieder fragen, doch Salazar unterbrach ihn.

\enquote{Nicht jetzt. Das kannst du alles nachlesen. Ich möchte mit dir über dein Amulett sprechen. \gst Doch zuerst, sag mir was du siehst, wenn du es fest in der Hand hältst.}

\enquote{Ich sehe meine Freundin}, antwortete Harry.

\enquote{Ah, sehr gut. Du musst wissen, dass du mit diesem Amulett aber auch andere Personen sehen kannst. Zu jedem Zeitpunkt, an jedem Ort. \gst Ich habe erfahren, dass du eine Verbindung mit jemand anderen hast. Dich und Tom Riddle verbindet etwas.}

\enquote{Ja}, antwortete Harry leicht überrascht.

\enquote{Aber mit dem Amulett kannst du eine besondere Verbindung zu ihm aufnehmen. Du kannst ihn beobachten, ohne dass er etwas davon mitbekommt. Du kannst ihm etwas zuflüstern, ohne dass du in der Nähe bist. Du kannst ihm Bilder in den Kopf setzen.}

Harry fiel sein Kinn herunter. \fluestern{Voldemort Bilder in den Kopf setzen}, murmelte er.

\enquote{Ganz genau. Du musst es dir wieder holen. Gleich morgen früh. Je länger du es trägst, desto besser kannst du Verbindung mit ihm aufnehmen.} Langsam begann Salazar zu verblassen.

\enquote{Meine Zeit ist bald um Harry. Wir sehen uns morgen Abend wieder hier. Gleiche Zeit. Ich muss dir noch was über das Amulett\gst} Doch weiter kam er nicht. Er war verschwunden.

\trenn

\enquote{Herein!}

Harry öffnete die Tür und ging auf Professor Elbers Schreibtisch zu. Sein Büro war anders als die Büros der anderen Lehrer. Seines enthielt kaum einen Gegenstand. Neben dem reich verzierten Schreibtisch aus edlem Holz war dort nur ein kleines Sideboard mit vielen Schubladen, ebenfalls aus Holz. Ansonsten war sein Büro komplett leer, wenn man von den vier Bildern der Hausgründer von Hogwarts einmal absah.

\enquote{Professor?}, fragte Harry.

\enquote{Ja, was gibt es, Harry? Setzen Sie sich doch.}

Harry setzte sich. \enquote{Es ist wegen gestern. Ich habe da vermutlich überreagiert, als ich das Amulett wegwarf. \gst Ich hätte es gerne zurück.}

Professor Elber sah ihm lange in die Augen. \enquote{Und was genau führte zu Ihrem Sinneswandel, wenn ich fragen darf?}

Harry druckste leicht herum und meinte schließlich: \enquote{Die Verwandtschaft zu Salazar. Ich habe eingesehen, dass er doch nicht so war, wie wir alle vermuteten. Das war nicht leicht.}

\enquote{Deshalb sind sie heute Morgen in der Bibliothek gewesen?}

\enquote{Ja}, log Harry.

Professor Elber griff in seine Tasche und nahm das Amulett heraus. Er hatte den Verschluss bereits repariert und legte das Amulett auf seinen Tisch. \enquote{Lassen Sie mich eines klarstellen. Sie baten mich, Ihnen mein Amulett zu schenken, denn als sie es wegwarfen und nichts mehr damit zu tun haben wollten, war es herrenlos. Also habe ich mich seiner angenommen. Momentan gehört es noch mir.} Damit gab er Harry zu verstehen, dass es wieder in sein Eigentum übergehen würde, sollte er es nehmen. Dann dürfte er es aber nicht mehr so einfach wegwerfen.

Harry betrachtete es kurz und nahm es dann in seine Hand. Es lag nun in seiner Innenhand und die Kette lag um seinen Zeigefinger herum und hing zu Boden. Er schloss seine Hand und wollte gerade zu Professor Elber schauen, als er sich plötzlich drehte und in einem dunklen Saal wieder fand.

Harry sah einen langen, Kerzen-erleuchteten Tisch an dessen Kopfende Voldemort saß. Links und rechts am Tisch saßen Todesser. Einige davon kannte Harry. Malfoy, Lestrange, Rowls, Dolohov. Harry bekam große Augen, als Voldemort ihn sah.

\enquote{Harry Potter}, säuselte er.

Alle Todesser drehten sich um. Harry wurde zunehmend unwohl. Voldemort zückte seinen Zauberstab und Harry griff instinktiv in seine Tasche. Doch er hatte keinen dabei.

\enquote{Wo soll Harry Potter sein?}, fragte Bellatrix Lestrange.

\enquote{Ja}, meinte ein Todesser, den Harry nicht kannte.

\enquote{Er steht genau dort, drei Meter vom mittleren Fenster weg}, antwortete Voldemort.

Harry bemerkte, wie sich die Todesser ratlos ansahen. \enquote{Ich weiß nicht, woher du plötzlich kamst Harry Potter, aber diesen Raum verlässt du nicht lebend. \gst \spruch{Avada Kedavra}}, brüllte er.

Ein grüner Blitz verließ seinen Zauberstab, durchdrang Harry und schlug hinter ihm eine Vase und ein kleines Holztischchen entzwei. Verängstigt und ungläubig wegen dessen, was gerade eben passiert war, betastete er seinen Brustkorb, doch da war nichts. Er war noch am Leben und es ging ihm gut. Plötzlich drehte es Harry wieder und die Umgebung wandelte sich.

Er saß wieder in Professor Elbers Büro. Genau in derselben Position, wie er den Raum gedanklich verlassen hatte und hob seinen Kopf, um seine Augen zu sehen. Dann sah er sich irritiert um, als er realisierte, wieder im Büro zu sein. Seine Hand öffnete sich leicht. Das Amulett lag nun wieder lose in seiner Hand.

Professor Elber schaute ihn mit leicht schrägem Kopf an. \enquote{Alles in Ordnung?}, fragte er.

\enquote{Ja}, antwortete Harry. Er bedankte sich, stand auf und ging. Im Türrahmen drehte er sich noch um und sah zu den Porträts. Sein Blick blieb bei Salazar Slytherin stehen. Er blinzelte ihm aus seinem Bild kurz zu. Harry schmunzelte leicht und verließ das Büro. Ihn wunderte nur, dass seine Narbe nicht schmerzte. Voldemort musste doch wütend sein.

Die Tür zum Büro flog zu und Harry hörte eine gedämpfte Stimme. \enquote{Er hat es sich gerade geholt, Salazar.}

Harry fiel noch etwas ein, er drehte auf der Treppe um und klopfte abermals an die Tür. \enquote{Herein!}, vernahm er von drinnen.

\enquote{Was kann ich für sie tun?}, fragte Professor Elber Harry ohne aufzusehen.

\enquote{Ich hätte gerne eine Erlaubnis für die verbotene Abteilung}, sagte Harry.

Professor Elber hob seinen Kopf, deutete ihm an sich zu setzen, nahm seinen Zauberstab und holte ein Formblatt aus dem Nichts herbei.

An der Überschrift konnte er eine Berechtigung für die Bibliothek erkennen. Professor Elber wartete, bis Harry weitermachte, damit er das Formular ausfüllen konnte.

\enquote{Es ist ein Buch über Dementoren}, sagte Harry.

\enquote{Fünfzehn}, gab Professor Elber zur Antwort.

\enquote{Wie bitte?}, fragte Harry nach.

\enquote{Fünfzehn \gst Bücher, die auf diese Beschreibung passen. Eines mehr oder weniger.}

Harry begriff. Er versuchte sich an das Inhaltsverzeichnis oder den Titel zu erinnern. \enquote{Da war was über Abwehr von Dementoren drin.}

\enquote{Neun.}

\enquote{Und über Aufzucht und Vermehrung.}

\enquote{Zwei.}

\enquote{Es hatte einen braunen Einband}, sagte Harry schließlich. Professor Elber wollte schon zur Feder greifen, als er es sich anders überlegte und stattdessen das Formblatt zerriss, worauf es sich in Luft auflöste. Er stand auf und lief um den Tisch herum. Harry bekam es mit der Angst zu tun. \enquote{Entschuldigung Professor}, stammelte er und stand hastig auf, um den Raum zu verlassen. Harry drehte sich um wollte gerade das Büro verlassen, als er hinter sich eine Stimme hörte.

\enquote{Habe ich gesagt, dass sie gehen sollen?} Harry lief es eiskalt den Rücken runter. Wenn er die Stimme nicht erkannt hätte, hätte er schwören können, dass es Severus Snape war, der das sagte. Derselbe Tonfall, dieselbe Betonung. Mulmig blieb er stehen. Professor Elber lief an ihm vorbei und stieg die Treppe zum Klassenzimmer hinunter. \enquote{Kommen Sie mit.}

Harry folgte ihm. Sie standen nun vor einem der vielen Schränke an der Seite des Klassenzimmers. Alle hatten eine gläserne Front mit Holzrahmen. Professor Elber holte einen kleinen Schlüsselbund aus der Hosentasche und öffnete das Schloss, um an die Bücher im Schrank zu gelangen. Er zog eines der Bücher ein paar Zentimeter heraus und schob es nach rechts. Es sah so aus, als ob die Bücher der Reihen ihm folgen würden. Von Links erschienen neue Bücher und Rechts verschwanden sie im Holz. Dann bemerkte Professor Elber seinen Fehler und drückte das Buch in die andere Richtung. In gewissem Abstand tauchte ein neues Buch auf, das ebenfalls ein paar Zentimeter herausgezogen war. Als er das richtige Buch gefunden hatte, zog er es ganz heraus. \enquote{Ist es das hier?}, fragte er Harry und zeigte es ihm. Harry nickte. Professor Elber nahm das Buch mit zum Pult und legte es darauf. Dann schwang er seinen Zauberstab darüber und ein Blatt Pergament erschien aus dem Nichts. Er legte seinen Zauberstab darüber und zog ihn waagerecht von oben nach unten. Ein weiteres Formblatt erschien. Der Titel des Buches war bereits ausgefüllt. Professor Elber winkte Harry zu sich und meinte dann \enquote{Unterschreiben sie bitte hier.}

Harry schritt um den Tisch und sah sich das Blatt an. Dort stand in etwa:

\accentuate{Der Unterzeichnete bestätigt, sich folgendes Buch ausgeliehen zu haben.}

Dann kam der Titel des Buches. Harry kontrollierte ihn, indem er den Zettel anhob und verglich. Dann las er weiter. Das heutige Datum war unter \accentuate{Ausleihdatum} bereits eingetragen und das Feld \accentuate{Rückgabedatum} war leer. Harry unterschrieb neben dem Ausleihdatum und reichte Professor Elber das Pergament. Dieser nahm es entgegen und legte es in den Schrank an die Stelle des Buches. Danach verschloss er den Schrank wieder.

Harry bedankte sich, schob das Buch in seine Tasche und wollte gerade den Raum verlassen. \enquote{Spätestens am Jahresende hätte ich das Buch gerne wieder.} Harry drehte sich kurz um, nickte und verschwand.

\trenn

Auf dem Weg zur Großen Halle traf er auf Luna, die ihn fragte: \enquote{Was wolltest du bei Voldemort?}

Harry stutzte. \enquote{Woher?}

\enquote{Unsere Verbindung. Ich habe es gesehen. Die ganze Runde an Todessern.}

Harry war erstaunt. \enquote{Davon habe ich gar nichts mitbekommen}, meinte Harry.

\enquote{Du warst auch ziemlich beschäftigt}, meinte Luna. \enquote{Ich wäre fast beim Gehen gestolpert. Daran müssen wir üben. Wir müssen lernen, uns gegeneinander abzuschotten.}

Harry verfolgte seine Vision noch das ganze Essen lang. Er erzählte Hermine und Ron, sowie Ginny davon. Diese konnten mit der Art der Vision gar nichts anfangen.

Nach dem Essen zog er sich um und nahm seinen Besen, um auf dem Feld mit seiner Mannschaft zu trainieren. Er suchte das Feld nach dem Schnatz ab und ließ seinen Blick schweifen. Er suchte im Himmel, auf dem Boden und auf den Rängen. Auf einem Turm der Slytherin sah er Dumbledore, der ihnen scheinbar beim Training zusah. Er wunderte sich, warum Dumbledore ihnen zusehen sollte. Aber andererseits kam er auch kurz zu einem Training der Slytherins und der Ravenclaws, sowie der Hufflepuffs, fiel ihm wieder ein.

\gedanke{Er schaut wohl bei allen Gruppen einmal vorbei}, ging Harry durch den Kopf.

Er sah sich weiterhin um und entdeckte den Schnatz am Boden. Sofort sauste er nach unten, um den Schnatz zu fangen. Während seines Fluges durchzog ihn ein Gedanke. \gedanke{Kann ich meinen Besen anweisen, mir hinterherzufliegen und mich zu überholen? Ohne Gewicht sollte er doch schneller sein, da er weniger Masse zum Beschleunigen und Bremsen hat. Unten könnte er mich leicht auffangen.}

Dann musste er seinen Besen bremsen und nach oben ziehen, da der Schnatz zur Seite schwenkte. Er schaffte es nicht ganz und stieg mit einer weniger eleganten Rolle von seinem Besen herab. Doch er verletzte sich nicht. Er stand etwas bedröppelt am Boden und nahm seinen Besen auf. Der Schnatz flog vor seiner Nase vorbei und er musste nur zugreifen. Doch der Schnatz war schnell. Also stieg er wieder auf seinen Besen und jagte ihm hinterher, bis er ihn hatte. Dabei musste er ein paar Klatschern ausweichen, die seine Kameraden ihm immer wieder in den Weg zu schlagen versuchten.

Mit dem Schnatz in der Hand flog er zu den Rängen und blieb vor Dumbledore in der Luft stehen. Er nutzte die Gelegenheit, nachdem er den Schnatz wieder fliegen gelassen hatte, Dumbledore zu fragen, ob diese Aktion mit seinem Besen möglich wäre. Dabei beobachtete er immer wieder den Himmel und den Boden, sowie das gesamte Spielfeld. Er blickte Dumbledore kaum an. Doch auch dieser schaute mehr der Mannschaft zu, als dass er Harry anblickte.

\enquote{Das käme auf einen Versuch an, Harry. Du brächtest auf jeden Fall bei deinen ersten Versuchen jemand am Boden, der deine Aktionen überwacht und deinen Sturz bremst. Es wäre während eines Spiels natürlich eine tolle Aktion, \gst falls es sich anbieten würde.}

Harry nickte in den Raum hinein und bekam die nächsten Worte von Dumbledore nicht mehr mit, da er den Schnatz erspähte und ihm zu folgen versuchte. Doch dieses Mal war der Schnatz schneller. Also kehrte er nach einer viertel Stunde wieder zu Dumbledore zurück und stellte sich in dessen Nähe.

\enquote{Ich war fertig, Harry.}

Harry nickte und wartete, ob er den Schnatz sehen würde. Doch vor Trainingsende kam er nicht mehr in Harrys Blickfeld, obwohl er Kreuz und Quer über das Spielfeld schwebte.

Kurz bevor er an diesem Tag ins Bett stieg, stellte er sich seinen Wecker. Er durfte sein Treffen mit seinem Vorfahren nicht verpassen. Als er geweckt wurde, verließ er sein Bett und schlich sich in den Gemeinschaftsraum. Weil er dort niemanden sah, setzte er sich wieder in denselben Sessel, wie am Abend zuvor und wartete. Harry erschrak, als Salazar Slytherin plötzlich wieder auftauchte.

\enquote{Hallo Harry}, sagte Salazar Slytherin.

\enquote{Hallo Salazar}, gab Harry zurück.

\enquote{Wir haben auch heute nicht viel Zeit}, fuhr Salazar fort. \enquote{Ich wollte dir noch etwas über dein Amulett sagen.}

\enquote{Gleich}, antwortete Harry. \enquote{Ich würde vorher gerne noch wissen, inwieweit ich mit dir und Godric Gryffindor verwandt bin.}

\enquote{Ah ja}, antwortete er. \enquote{Mein Ururur-Enkel hatte die Ururur-Enkelin von Godric Gryffindor geheiratet und du bist der letzte aus dieser Linie.}

\enquote{Und Voldemort?}, fragte Harry.

\enquote{Der ist in direkter Linie mit mir verwandt. Aber fast ausschließlich über die weibliche Linie. Ich schäme mich ein wenig für ihn. \gst Aber nun zurück zum Amulett. Hast du es?}

Harry zog es unter seinem T-Shirt hervor, um es Salazar zu zeigen. \enquote{Ich trage es wieder, ständig.}

\enquote{Sehr gut}, meinte Salazar. \enquote{Du musst wissen, dass dich das Tragen des Amuletts stärkt. Ein Teil meiner Kraft wird somit dir übertragen. Du musst Voldemort aufhalten. \gst Hast du bisher schon mal etwas Eigenartiges mit deinem Amulett erlebt?}, fragte Salazar.

\enquote{Ja}, antwortete Harry. \enquote{Ich hatte mir gerade das Amulett zurückgeholt, als ich mich plötzlich in einer Halle mit einem großen Tisch befand. Voldemort und seine Todesser waren da. Aber nur er konnte mich sehen. Er versuchte sogar, mir den Todesfluch auf den Hals zu hetzen.}

Salazar antwortete: \enquote{Dann hast du bereits eine der Funktionen des Amulettes kennengelernt. Du kannst dich Voldemort zeigen, wann immer du willst, und dich mit ihm unterhalten. \gst Wenn du dich auf ihn konzentrierst, sobald du das Amulett hältst. Oder du kannst ihm Gedanken oder Bilder in seinen Geist und seinen Träumen platzieren.}

\enquote{So wie Legilimentik?}, fragte Harry nach.

\enquote{So in etwa}, antwortete Salazar. Harry grinste etwas. Der Gedanke, Voldemort zu ärgern, gefiel ihm.

\enquote{Bedenke aber, dass er es merken wird. Er ist ein guter ich meine großer Zauberer. Zwar böse, aber er hat viel Macht. Sein Zorn auf dich wird umso größer, je mehr du ihn ärgerst. Also übertreibe es nicht. Oder noch besser. Fange gar nicht damit an.} Salazar fing wieder an zu verblassen.

\enquote{Noch eine Frage, Salazar. Sind die Malfoys auch mit dir verwandt?}

Salazar überlegte eine Weile. \enquote{Nein \gst halte dein Amulett fest und denke an mich. Dann können wir uns unterhalten. Aber nur einmal am Tag können wir uns sehen und dann nur wenige Minuten}, antwortete er und verschwand.

Harry grinste. Das musste er Malfoy unter die Nase halten. Irgendwann. Er stand auf und ging wieder ins Bett. Seine Narbe kribbelte leicht, also begann er wieder mit seinen Okklumentik-Übungen, die er selbst wieder aufgenommen hatte. Er leerte seinen Geist, während er im Bett lag und beruhigte ihn. Dann schlief er mit dem Gedanken daran, in einem schwarzen Raum in einem Bett zu liegen und zu schlafen, ein.

\trenn

\enquote{Würdet ihr mir nachher helfen?}, fragte er Hermine und Ron.

\enquote{Wobei?}, fragte Hermine.

\enquote{Bei einem Experiment.}

\enquote{Welchem?}, fragte Ron nach.

\enquote{Nach dem Essen sage ich es euch. Es ist nicht schwer. Ihr sollt mich nur überwachen und müsst schnell reagieren. Nichts Gefährliches.}

Nach dem Essen schickte er seine zwei Freunde zum Quidditchfeld, während er noch seinen Besen holte. Unten angekommen erklärte er Ron und Hermine was er vorhatte. Er erzähle von seinem Gespräch mit Dumbledore und dass es auf einen Versuch ankäme. Ron war natürlich sofort begeistert, nur Hermine war skeptisch. Aber alleine Ron wollte er sich nicht anvertrauen, da seine Zauberkünste zwar besser wurden, aber er sein Leben einem Zauber von Ron nicht anvertrauen würde, wenn er eine Wahl hätte. Er würde es Ron nie sagen, aber diese Angst hatte er.

Er nahm Hermine etwas beiseite und wagte einen unkonventionellen Versuch sie zu überzeugen. \enquote{Ich werde diesen Versuch machen. Und ich vertraue darauf, dass du über deinen Schatten springst und meinen Arsch rettest, falls es Ron nicht schaffen sollte. Du kennst ihn genauso gut wie ich. Also Hase, spring über deinen Schatten.}

Dann ließ er ihren Arm los und ging zurück zu Ron. Falls sie errötete, merkte er es nicht, da er von ihr weg lief. \enquote{Ich glaube, ich habe sie überzeugt. Bist du bereit?} Ron nickte. \enquote{Arresto Momentum, wenn ich fallen sollte.} Wieder nickte Ron.

Dann stieg er auf seinen Besen, sah noch einmal zu Hermine und stieg nach einen knappen und zögernden Nicken nach oben.

Während des Steigfluges dachte er sich: \gedanke{Ich hätte Hermine nicht Hase nennen sollen. Entweder versteht sie es falsch, oder sie ist sauer auf mich. Ich mag sie zwar, mein Schwesterchen\abs} bei diesem Gedanken musste er lachen. \gedanke{Aber mit seiner Schwester führt man keine Beziehung.}

Oben fiel ihm ein, dass er gar nicht daran dachte, wie er seinen Besen dazu bringen sollte, ihn aufzufangen.

\stimme{Konzentriere dich. Und vertraue auf deine Magie}, hörte er in seinem Kopf.

Er konnte die Stimme nicht zuordnen, aber sie klang vertraut und beruhigend. Also versuche es Harry. Er konzentriere sich und spürte seinen Besen unter sich. Aber nicht nur so, als wenn er ihn berührt. Sondern er spürte ihn in seinem Geiste. Es war eine Art der Kommunikation, die keine Worte benötigt. Worte waren nicht nötig, ja sogar überflüssig. Nicht möglich.

Er teilte seinem Besen mit, dass er abspringen würde und er ihn rechtzeitig auffangen sollte. Doch er konnte es nicht beschreiben, selbst wenn er es müsste. Es war eine Kommunikation, die unbewusst passierte und auf reinem Vertrauen und auf Intuition basierte. Er sprang ins Leere und der Besen folgte ihm. Er spürte, wie er langsamer wurde und er seinen Besen unter sich wieder spürte. Harry hatte das Gefühl, dass Rons Zauber ihn zu früh bremste, denn er hatte noch Zeit, bevor er unten aufschlug.

\enquote{Ron, hast du einen Zauber gewirkt?}, fragte er,

\enquote{Ja, wieso? War das falsch?}, fragte Ron nach.

\enquote{Nein, nein. Nur etwas voreilig. Du hättest noch Zeit gehabt. Das machen wir nochmal. Ich brauche schließlich Sicherheit. Gleiches Szenario wie vorher, nur greifst du später, wenn überhaupt, ein.}

Ron nickte begeistert. Harry stieg erneut auf und bereitete sich vor. Dann sprang er erneut ab. Dieses Mal spürte er keine Bremsung. Allerdings kam er hart auf seinem Besen auf und fiel pustend vom Besen auf den Boden, als er unten ankam. Der Besen hatte ihn hart zwischen den Beinen getroffen und seine Hoden gequetscht. Er brauchte einige Minuten, bis er einen erneuten Versuch wagen wollte, doch Hermine überzeugte ihn, zu Madame Pomfrey zu gehen und sich untersuchen zu lassen.

Es war Harry zwar peinlich, doch er fügte sich, da er sonst nicht zur Ruhe kam. Er machte sich auf den Weg zur Krankenstation. Schweren Ganges ging er durch das Schloss in den Krankenflügel. Sein Besen begleitete ihn den ganzen Weg. Stehend schwebte er neben ihm her, bis sich ihre Wege trennten und er Ron und Hermine folgte. Im Gemeinschaftsraum flog er selbstständig in sein Zimmer und blieb in der Ecke stehen, in der er immer stand.

Doch Harry bekam von all dem nichts mit. Er war zu sehr mit sich selbst beschäftigt.

\enquote{Madame Pomfrey? Könnten sie mich untersuchen?}

Sie war alleine auf der Krankenstation und kam zu ihm heran. \enquote{Was haben sie denn?} Harry wurde rot. \enquote{Oh. Schwere Geschütze.} Sie zog ihren Zauberstab und umrahmte mit Schutzwänden das Bett. Dann legte sie einen Stillezauber auf das Gebiet, um sich vor neugierigen Ohren zu schützen.

\enquote{Ich habe ein Problem mit meinen Hoden. Ich habe sie mir\abs Sagen wir so. Ich habe ein kleines Experiment gemacht und bin quasi auf meinen Besen gefallen. Es war ziemlich hart. Der Schmerz ist zwar kaum noch da, aber sicher ist sicher.} Jetzt fiel ihm eine Last von der Seele. Jetzt, da es raus war.

\enquote{Dann legen sie sich mal hin}, forderte ihn Madame Pomfrey auf. Harry folgte ihrer Aufforderung und sie begann ihn zu untersuchen. Sie fuhr mit dem Zauberstab über seinen Intimbereich und zog eine Augenbraue in die Höhe. \enquote{Ich gebe ihnen eine Salbe mit. Drei Tage lang zweimal täglich eincremen. Den Penis, die Hoden und auch zwischen den Pobacken.} Harry wartete, bis sie wieder kam und ihm die Salbe gab. \enquote{Nehmen Sie nächstes Mal einen Schutz, oder belegen sie ihre Geschlechtsorgane mit einem Zauber}, riet sie ihm.

Harry nickte, bedankte sich bei ihr und verließ den Krankenflügel. Abends cremte er sich ein und ging dann zu Bett.

\trenn

Madame Pomfrey betrat ihre Räume, ging zielstrebig in das Badezimmer und begann sich auszuziehen. Nackt ließ sie sich Badewasser in die Wanne einlaufen und betrachtete sich danach im Spiegel. All die Jahre, die sie schon hinter sich hatte, machten sich in dem freundlichen Gesicht bemerkbar. Das magisch eingelaufene Wasser war bereits hochgestiegen, sodass sie in die Wanne trat und unter wohligem stöhnen sich in das warme Wasser gleiten ließ. Ihren Kopf lehnte sie an die Kante der Wanne und ließ ihre Haare in das Wasser gleiten. Der anstrengende Tag hinter ihr und das warme Wasser um sich ließ sie ihre Gedanken gleiten und ihre Muskeln entspannen. Ihr ging der Narben-geplagte Erstklässler durch den Kopf, der sich Anfang des Schuljahres bei ihr einfinden musste, damit sie ihn untersuchen konnte, um eventuelle Probleme besser abschätzen zu können.

\begin{rueckblick}
\enquote{Hallo, ich bin Madame Pomfrey und Sie sind sicherlich Mister Allman.}

\enquote{Ja Madame.}

\enquote{Sie wissen, weshalb Sie hier sind?}

\enquote{Ja Madame.}

\enquote{Dann setzen sie sich mal auf das Bett hier.}

Der Junge gehorchte und stieg auf das Bett. Madame Pomfrey legte die Krankenakte beiseite und begann mit ihrem Zauberstab den Jungen zu untersuchen. Nach Abschluss der Untersuchungen erschien ein Pergament mit den Ergebnissen neben der Akte. Madame Pomfrey las sich das Ergebnis durch und ordnete das Pergament ein. Dann wandte sie sich wieder ihrem Patienten zu.

\enquote{Das ist vor einem Jahr passiert, richtig?} Der Junge nickte. \enquote{Wie, wenn ich fragen darf?}

\enquote{Ich sprach gerade mit meinem Vater über den Kamin. Ich war gerade fertig und zog meinen Kopf heraus, als die Flammen plötzlich aufloderten und mir mein Gesicht verbrannten. Kurz darauf kam mein Bruder durch den Kamin, sah mich und brachte mich sofort in das Heilerhaus Sankt Mungo. Aufgrund dieses Unfalls hat mein Bruder sofort eine Sicherung für das Kaminnetzwerk erwirkt \gst Dieser arbeitet in der Flohnetzwerk-Überwachung, oder so \gst Jedenfalls kann so etwas nicht mehr passieren. Es war scheinbar das erste Mal, dass es einen Unfall in dieser Art gegeben hat. Es gab zwar eine Entschädigung, aber mein Auge bekomme ich dadurch nicht wieder.}
\end{rueckblick}


In einem leicht dämmernden Zustand drangen nun andere Gedanken in sie ein.

\begin{rueckblick}
\enquote{Das, was sie so abwertend als schwarze Magie bezeichnen, ist nur eine Einordnung von Menschen, die vorwiegend negative Erfahrungen mit dieser Art der Zauberei gemacht haben. Wie ich schon meinen Schülern begreiflich gemacht habe \gst Magie hat keine Farbe. Ich ließ eine Schülerin zwei Pflanzen mit Feuer niederbrennen. Eine ist daran zugrunde begangen. Die andere verbrannte, aber ihr Samen fiel herab, bohrte sich in die Erde und brachte eine wunderbare Blume hervor. Es war also zweimal derselbe Zauber. Nur einmal sahen wir ihn als weiße Magie an, das andere Mal jedoch als schwarze. Es ist also die Intention hinter einer Tat, die gut oder böse ist.}
\end{rueckblick}

Das war das intensive Gespräch, nachdem Katies Hand wieder gewachsen war.

Langsam verbanden sich die beiden Ereignisse in ihrem Kopf zu einem gemeinsamen Gedanken. \gedanke{Wäre es möglich Mister Allman mit dem Einsatz von schwarzer Magie zu helfen?} Dieser Gedanke nahm in ihrem Kopf immer stärkeren Raum ein. Sie öffnete ihre Augen und starrte für eine Weile an die Decke. Dann griff sie zu ihrem Zauberstab und öffnete wortlos die Tür zu ihrem Badezimmer. Außer Sichtweite, aber dennoch in Hörweite, hing ein Bild in ihrem Wohnzimmer, mit dem sie ihre Gedanken teilen wollte und um eine zweite Meinung bat. \enquote{Alain?}, fragte sie in den Raum hinein \gst Stille \gst \enquote{Alain?}, rief sie nun kräftiger.

\enquote{Ja Poppy, ich bin wach. Was gibt es?}

\enquote{Ich brauche eine zweite Meinung.}

\enquote{Zu einem Fall eines Patienten?}

\enquote{Man kann es so nennen. Es ist aber vielmehr ein Gedanke zu einer möglichen Heilung. Ich weiß nicht mal, ob es möglich ist.}

\enquote{Schieß los meine Liebe.}

\enquote{Du erinnerst dich noch an den Schüler, von dem ich dir erzählt habe\abs?}

\enquote{Du hast mir von vielen Schülern erzählt, Poppy.}

\enquote{Lass mich ausreden. Der Erstklässler dieses Jahr. Der mit dem einzelnen Auge und den vielen Narben im Gesicht.}

Es dauerte eine Weile, aber dann kam die Antwort. \enquote{Ja.}

\enquote{Ich konnte ihm nicht helfen. Er muss weiterhin mit einem Auge leben und die Narben im Gesicht tragen.}

\enquote{Ja, den habe ich gesehen, als ich mir eine Birne aus dem Küchengemälde geholt habe.}

Madame Pomfrey musste schmunzeln.

\enquote{Schmunzelst du etwa?}

\enquote{Nein}, kam nicht sehr ernst. \enquote{Wie kommst du darauf?}

\enquote{Das sehe ich bis hierher.}

Ein belustigtes Schnaufen erklang aus dem Bad heraus.

\enquote{Und du erinnerst dich an den Fall \accentuate{abgetrennte Hand}?}

\enquote{Ja.}

\enquote{Könnte man dem Jungen nicht genauso helfen?}

\enquote{Ihm den Kopf abschneiden und nachwachsen zu lassen?}

\enquote{Nein, die Narben durch \accentuate{schwarze Magie} zu entfernen und ein Auge nachwachsen zu lassen\abs Warte mal, wenn wir eine Augenhöhle formen könnten, dann könnte er doch ein magisches Auge bekommen.}

\enquote{So ein Monstrum? Meinst du nicht, dass das für einen jungen Mann hinderlich sein kann?}

\enquote{Nein, denn es gibt mittlerweile kleinere Ausführungen, die nicht größer als ein normales Auge sind. Optisch also keinerlei Beeinträchtigungen.}

\enquote{Hmmm}, erklang aus dem Wohnzimmer. \enquote{Das könnte gehen. Aber ist es auch ungefährlich?}

\enquote{Darüber muss ich mit meinem Kollegen reden. Vielleicht hat er ein paar Hinweise oder Anregungen. Aber ich möchte es auf jeden Fall selber machen.}

\enquote{Du willst dich mit deinem Kollegen beraten, der im Verdacht steht die dunklen Künste zu praktizieren und mit Du-weißt-schon-wem zu sympathisieren?}

Madame Pomfrey stieg aus der Wanne, zog sich einen Bademantel über und stürmte in ihr Wohnzimmer. \enquote{Rede nicht so über meinen Kollegen. Er ist unheimlich nett, zuvorkommend und weiß sehr viel. Man kann sich mit ihm über fast alles unterhalten. Er hat mir sogar ein paar seltene Zutaten aus seinem Garten zu meinem Namenstag geschenkt. Eine Geste, die heute gar nicht mehr zelebriert wird.} Sie war so in Rage, dass sie gar nicht bemerkte, wie ihr Bademantel sich etwas lockerte und sich ihre Brüste bis kurz vor ihre Brustwarten entblößten. Dem älteren Herrn im Bild entging das nicht und so warf er immer wieder einen sehnsüchtigen Blick auf den Bademantel, in der Hoffnung mehr zu sehen. Doch diese Hoffnung blieb unerfüllt, da Madame Pomfrey schon vor Jahren ihren Bademantel entsprechend verzaubert hatte, dass nichts zu sehen war.

\enquote{Ich habe nie behauptet, dass er\abs Ich habe die andern darüber reden hören\abs Ich\abs ach egal. Frag ihn, wenn er dir helfen kann.}

Poppy nickte und ging zurück ins Badezimmer, um sich ihre Schlafsachen anzuziehen. Zurück im Wohnzimmer nahm sie wieder einmal das kleine Schraubglas vom Tisch und setzte sich in einen Stuhl. Mit geschürzten Lippen besah sie sich die seltene Pflanze in dem Glas, bevor sie es abstellte und nach der Abendtoilette zu Bett ging.

\trenn
\onelineback % Anderenfalls werden 2 Leerzeilen gesetzt

\begin{rueckblick}
\enquote{Ich habe gehört, dass sie einen Tarnumhang haben.}

\enquote{Ja}, antwortete Harry.

\enquote{Dann bringen sie ihn zu ihrer nächsten Stunde mit, dann werden sie lernen, wie sie Objekte die durch ihren oder einen Tarnumhang, durch Desillusionierung, oder durch Tarnungstechniken der Muggel verdeckt sind, entdecken können.}
\end{rueckblick}

Harry war auf dem Weg zu seinem heutigen Training mit Professor Elber, als er über seine Karte nachdachte. \gedanke{Wieso zeigt die Karte Punkte an? Die Rumtreiber kannten die doch nicht. Sind die neu hinzugekommen? Und wieso sind mir die nicht früher aufgefallen?}

Harry trat gerade auf das Quidditchfeld, wo eine Sichtschutzwand aufgebaut war. Auf dem Boden davor waren vier Buchstaben angebracht. \accentuate{L, U, G, A.} Was hinter dem Sichtschutz war, konnte Harry nicht sehen. Sein Lehrer verlangte Harrys Tarnumhang, den Harry nur widerwillig hergab. Dann verschwand sein Lehrer kurz hinter der Wand und kam dann wieder hervor.

Er schob die Wand ein Stück beiseite und gab die erste Aufgabe frei. Harry sah einige Buchswedel, die vor einer Pappwand standen. Was dahinter war, konnte er nicht sehen.

\enquote{Was sehen sie?}, fragte ihn sein Lehrer.

\enquote{Ein paar Buchswedel, die vor einer Pappwand stehen.}

\enquote{Was noch?}

\enquote{Nichts.}

\enquote{Was ist hinter der Wand?}

\enquote{Das kann ich nicht sehen.}

\enquote{Dann setzen sie sich und konzentrieren sie sich.}

Harry sah sich um und entdeckte nur den Sand. Er entschied sich ein Kissen herzuzaubern, damit er bequemer sitzen konnte. Dann setzte er sich auf das Kissen und starrte auf die Zweige. Er überlegte, wie er zum Ziel kam. Immer wieder musste er selber darauf kommen. Manchmal gab es Hilfestellungen, mal musste er selber die Lösung finden und manchmal bekam er eine detaillierte Anleitung. Sein Lehrer setzte sich neben ihn auf ein Kissen, das erschien, sobald er dem Boden nahe genug war. Auch er sah nach vorne und schloss die Augen. Harry tat es ihm gleich.

\enquote{Gut}, sagte Elber. \enquote{Versuchen Sie ihre Gedanken auszublenden.}

Harry wandte die Okklumentik-Kenntnisse an, die er bisher erworben hatte, und versuchte, seinen Geist zu leeren, was ihm nicht ganz gelang. Immer wieder drängten sich ihm Gedanken in den Sinn.

Dann passierte etwas, was Harry nicht erwartet hätte. Er begann Farben und Formen zu sehen. Erst verschwommen, dann immer klarer. Nach und nach begann er die einzelnen Zweige vor seinem inneren Auge zu sehen. Dann kam die Pappwand in sein Sichtfeld. Statisch stand sie für mehrere Minuten vor ihm, bis ihn die Konzentration verließ und er wieder in die Wirklichkeit zurückkehrte.

\enquote{Für den ersten Versuch nicht schlecht}, kommentierte Professor Elber.

\enquote{Wie können sie wissen, wie weit ich gekommen bin und ob ich überhaupt etwas gesehen habe?}

\enquote{Wenn sich jemand Zugang zur Magie verschafft, um Objekte magisch zu untersuchen, dann verändert das das magische Feld um einen herum. Da ich neben ihnen sitze, habe ich das bemerkt. Ich weiß nicht genau, wie weit sie gekommen sind, aber ich habe bemerkt, dass sie es zumindest versuchen. Das lernen sie mit der Zeit automatisch, wenn sie diese Technik erlernt haben.}

Harry war erstaunt. Eigentlich hatte er so viel in der Schule und von Dumbledore gesehen und gelernt, dass er sich vornahm, sich nicht mehr zu wundern, aber das hier brachte ihn doch zum Staunen. Dann begann er zu begreifen, dass es weitaus mehr gab, als man sich vorstellen konnte; oder einfacher gesagt, die Magie wirkt so, wie man es sich vorstellt. Die Vorstellung alleine wirkt einen Zauber. Mann muss nur an das glauben, was man bewirken will.

\gedanke{Dumbledore hat es mir schon einmal gesagt. Ich bin ein Magier mit hohen suggestiven Kräften. Wenn ich glaube, dass mir ein Zauber gelingt, dann wird es das auch. Wenn aber ein Zauber gegen meine Natur ist, dann wird er mir auch nicht gelingen, egal wie sehr ich mich bemühe.}

Jetzt hatte er es begriffen. Er schloss wieder seine Augen und begann erneut. Langsam bildeten sich wieder die Formen und die Farben. Dann kam, als alles scharf war, die Pappwand. Harry überlegte, was er nun tun musste.

\enquote{Laufen Sie}, hörte er aus weiter ferne.

Gedanklich lief er auf die Äste und die Wand zu. Knapp hinter der Pappwand sah er unscharf einen Holzbalken. Doch so sehr er sich auch anstrengte, er konnte den Balken nicht scharf bekommen. Er lief wieder zurück und dachte nach. Er drohte schon wieder die Konzentration zu verlieren. Daher beeilte er sich. Er stellte sich vor, wie die Äste verschwinden. Nach wenigen Sekunden begannen diese zu verschwinden. Danach war die Pappwand dran. Dahinter war ein Holzgestell in Form eines \accentuate{T}-s zu sehen. Harry kehrte zurück, bevor seine Konzentration vollständig nachließ.

\enquote{Ein Holzgestell in Form eines \accentuate{T}}, sagte Harry.

\enquote{Gut.}

Mit einem Handstreich schob er die Sichtschutzwand um ein Schulungsobjekt weiter. Doch da war nichts. Harry konnte nichts mehr sehen.

\enquote{Jetzt wird es schwerer. Aber die Möglichkeiten, die sie haben, sind um ein Viertel gesunken. Jetzt stehen nur noch drei mögliche Schutzmaßnahmen zur Auswahl.}

Harry nickte. Er atmete ein paar mal durch, da er sich sammeln musste.

\enquote{Hunger? \gst Durst?}, fragte sein Lehrer.

\enquote{Nein danke. Momentan nicht. Aber ich müsste mal.}

\enquote{Dann los.}

Harry stand auf und ging schnell sein Geschäft erledigen. Der Sichtschutz folgte ihm um die Objekte herum, sodass er keine Möglichkeit hatte, vorab etwas zu sehen. Auf seinem Rückweg dachte er bereits nach, wie er seine Aufgabe lösen konnte. Er konnte nicht so einfach das Objekt davor verschwinden lassen. Oder doch?

Konnte er einfach durch einen Zauber durchsehen? War es überhaupt ein Zauber?

\gedanke{Ja klar. Es wird immer schwerer. Desillusionierungszauber sind schwer und können bei komplizierten Objekten mit schnellen Bewegungen Verzerrungen hervorrufen. Doch mein Tarnum\aabs Was sagte er? \inner{\aabs Objekte die durch ihren oder einen Tarnumhang\abs} richtig. Er hat einen Tarnumhang. Und dann kommt meiner. Ron hat mir doch damals gesagt, dass sie selten sind. Und alle, die er kennt, seien nicht so gut wie meiner. Die meisten verschleißen nach Jahren. Aber meiner gehörte meinem Dad. Er muss also sehr alt sein. Also muss ich gegen einen Zauber angehen. Ok.}

Er setzte sich wieder.

\enquote{Und? Hatten sie eine Erleuchtung?}

\enquote{Ja}, sagte Harry und grinste.

\enquote{Lassen sie mich teilhaben?}

\enquote{Ich glaube, ich weiß die Art, wie das Objekt verborgen wurde.}

\enquote{Und wie kommen sie darauf?}

\enquote{Ich nutze die Hinweise, die sie mir gegeben haben.}

\enquote{Welche wären das?}

\enquote{Sie sagten mir, welche Hindernisse ich überwinden muss. Außerdem beginnen wir immer mit dem leichten, und kommen dann zu den schweren. Die Muggeltarnung habe ich überwunden. Also wird dies hier\abs} Harry zeigte auf das nicht sichtbare Objekt vor ihm. \enquote{\aabs der Desillusionierungszauber sein. Dann kommen die beiden Tarnumhänge.}

\enquote{Gut, dann machen sie.}

Durch diesen Satz kurzzeitig verunsichert, dachte Harry kurz nach und versuchte den Zauber zu überwinden. Er schloss wieder seine Augen und dachte nach. Dieses Mal musste er etwas erscheinen lassen. Er versuchte nicht, ein Objekt vor seinem geistigen Auge zu fixieren, sondern er versuchte einer magischen Signatur zu folgen. Dies fiel ihm mitten in seiner Konzentration ein. Sein Professor sagte ihm, dass sich das magische Feld ändert, wenn man Zauber wirkt. So versuchte Harry magische Signaturen zu finden. Er näherte sich geistig dem Zielobjekt, bis er etwas spürte. Langsam kribbelte es, dann begann sich das Feld aufzulösen und Harry sah ein Metallgestell in Form eines \accentuate{A}-s.

Wieder in der Realität zurück, sah er, dass er den Zauber aufgelöst hatte. Sein Professor sah ihn ungläubig an. Nach einigen Sekunden fing er sich wieder und fing an zu meckern. \enquote{Sie sollten hinter den Zauber schauen und ihn nicht auflösen.}

\enquote{Entschuldigung}, gab Harry kleinlaut zurück. \enquote{Ich habe das Objekt dahinter aber schon vorher erkannt, bevor ich die Augen geöffnet hatte.}

Professor Elber hob eine Augenbraue und sah ihm direkt in die Augen. Nach einigen Momenten meinte er: \enquote{Na gut. Das will ich Ihnen mal vorerst glauben.} Er nahm seinen Zauberstab heraus und legte wieder den Zauber darüber. \enquote{Dieses Mal aber ohne den Zauber zu lösen. Verstanden?}

Harry nickte und begann erneut.

Als wieder das Kribbeln begann, zog sich Harry zurück, bis es nachließ. Erneut versuchte er den Zauber zu brechen, doch mitten drin unterbrach er sich. Dann fing er an, einer neuen Idee nachzugehen. Er versuchte, um den Zauber herum zu sehen. Doch dies hatte auch nicht den gewünschten Erfolg. Zumindest hatte er jetzt die Ausmaße des Feldes erkannt. Schließlich stellte er sich, aufgrund mangelnder anderer Ideen, vor, wie er durch das Feld schauen könnte. Schemenhaft kristallisierte sich ein Metallgestell heraus, das ein \accentuate{A} darstellte.

Als Harry wieder in der Realität war, sah er nichts. Der Zauber schien noch Bestand zu haben.

\enquote{Knapp. Der Zauber hat etwas gewackelt, was ein eventueller Insasse bemerkt hätte}, merkte sein Lehrer an. \enquote{Aber für Ihren ersten Versuch, sehr gut. Sie scheinen für diese Art der Magie ein Händchen zu haben. Respekt. Ich hätte weniger erwartet.}

Durch dieses Lob angespornt, fühlte sich Harry bekräftigt auch die anderen beiden Aufgaben zu lösen. Und wieder wurde ein Testobjekt freigelegt. Jetzt musste er das verborgene Etwas hinter dem \accentuate{G} erkennen.

\enquote{Aha, der normale Tarnumhang}, rutschte es Harry heraus.

Sein Lehrer lächelte. \enquote{Sie scheinen sich bei ihrer Pinkelpause ja richtig Gedanken über die Art des Schutzes und der Reihenfolge gemacht zu haben. Wie kommen sie darauf, dass das ein normaler Tarnumhang sein soll? Und was ist der Unterschied zum nächsten?}

\enquote{Mein Tarnumhang kommt als letzter. Das hier ist ein normaler, wie er überall zu haben ist. Na ja, relativ überall. Sie sind schon so recht selten.}

\enquote{Und der Unterschied?}

\enquote{Laut meinen Informationen halten normale Tarnumhänge nicht sehr lange. Außerdem lassen ihre Eigenschaften mit der Zeit nach. Meiner dagegen hat schon meinem Vater gehört. \gst Wieso erzähle ich Ihnen das eigentlich?}

\enquote{Weil ich ihnen etwas beibringen will.}

\enquote{Richtig}, sagte Harry gedankenverloren. \enquote{Woher wissen Sie \gst wollen Sie wissen, dass mein Tarnumhang anders ist?}

\enquote{Ich hatte eine sehr interessante Unterhaltung mit ihrem Direktor. Außerdem\abs}

\enquote{Was?}

\enquote{Behalten Sie das bitte für sich, ja?}

\enquote{Ok.}

\enquote{Ich habe eine eigenartige Signatur gespürt, als ich nahe an ihrem Zimmer vorbeiging. Es war außerhalb. Also habe ich mich auf die Suche gemacht. Ich habe etwas in ihrem Koffer gespürt. Da es privat war, verließ ich das Zimmer wieder. Erst danach hatte ich diese Unterhaltung mit Dumbledore. Dann wurde mir klar, was ich gespürt hatte. Ich hatte den Tarnumhang gespürt. Dann reifte die Idee, Ihnen dies beizubringen. Ich wollte einerseits Gewissheit, andererseits muss ich Ihnen viel beibringen. Ich bin nicht lange hier, wissen Sie?}

\enquote{Ja, aber was hat es mit meinem Umhang auf sich? Was ist an ihm besonders?}

\enquote{Lösen Sie erst einmal die beiden Aufgaben. Dann reden wir darüber. Eventuell erst in ein paar Tagen.}

Harry nickte schweren Herzens und machte sich an die Arbeit. Durch den Tarnumhang kam er recht schnell. Es war leichter, als beim Desillusionierungszauber. Dieses Mal bildeten Bambusstangen ein \accentuate{B}. Dann verschwand die Wand und das letzte Objekt verlangte nach einer Lösung. Harry brauchte mehrere Minuten, bis er sich wieder konzentrieren konnte. Danach hatte er Schwierigkeiten, überhaupt etwas zu spüren. Er schlich gedanklich den Bereich ab, in dem er das Objekt vermutete, doch da war nichts, was er erfassen konnte. Er brach ab.

\enquote{Ist da wirklich was?}, fragte er entkräftet nach.

\enquote{Wollen Sie ihn fühlen?}, fragte Professor Elber nach. Harry schaute ihn an und überlegte. \enquote{Bin ich nicht vertrauenswürdig genug?}, fragte Elber halb belustigt, halb emotionslos nach.

\enquote{Das ist es nicht. Oder doch? Ich weiß es nicht. Ich bin entmutigt. Dadurch, dass ich nichts gefunden habe. Da ist nichts.}

\enquote{Bohren Sie tiefer. Spüren Sie Signaturen auf. Sie haben den Vorteil, dass Sie wissen, dass da etwas ist. Dies müssen Sie nutzen.}

\gedanke{Tiefer bohren, das sagt der so einfach.}

\stimme{Was hast du unter deinem Umhang immer gedacht?}, hörte er in seinem Kopf.

\gedanke{Dass mich niemand entdeckt.}

\stimme{Wieso sollte es jetzt anders sein? Das Objekt darunter will auch nicht entdeckt werden. Zumindest schützt der Umhang das Objekt.}

\gedanke{Aber, was ist der Unterschied zu anderen Umhängen?}

\stimme{Andere Umhänge wurden nur mit Zaubern belegt und/oder benutzen Haare eines Demiguise.}

Das half Harry nur bedingt weiter. Aber er hatte einen Anhaltspunkt. Er schloss einfach nur die Augen und lies seine Gedanken schweben. Und wieder sickerten so langsam Ideen und Erkenntnisse in seinen Verstand. \gedanke{Das Objekt will nicht gefunden werden. Der Umhang schützt es aktiv. Ich muss es wie eine Person behandeln. Aber wie will ich mit einem Umhang ein Gespräch beginnen?} Die letzte Frage ließ er einfach frei, als ob er sie in die Welt hinaus denken wollte.

\stimme{Da gibt es reichlich Möglichkeiten}, hörte er.

Intuitiv begann er, die Quelle zu orten. \gedanke{Zumindest habe ich jetzt einen Ort}, dachte er. \gedanke{Wieso hast du mir geantwortet?} Keine Antwort. \gedanke{Hallo? Was interessiert dich?}

\stimme{Tut mir leid Harry. Das war ich. Ich konnte nicht anders.}

\gedanke{Salazar? Du hast mich\abs? Hmpf.}

Er war keinen Schritt weiter. Salazar hatte ihn veralbert.

\gedanke{Das Objekt will nicht gefunden werden. Der Umhang schützt es aktiv}, ging ihm wieder durch den Kopf.

Da er keine Kopfschmerzen bekam, wenn er es nur gedanklich vollzog, schwebte Harry in schneller Folge immer wieder durch den gedachten Mittelpunkt in verschiedenen Richtungen. Langsam begann er einen leichten Widerstand zu spüren. Erstaunt öffnete er seine Augen und sah ganz schwach den Buchstaben \accentuate{A} aus Luftschlangen gebildet. Dann waren seine geistigen Kräfte erschöpft. Er sackte zusammen.

\enquote{Kreacher, ihr Herr braucht sie}, sagte Professor Elber.

Kreacher erschien und blickte sich um. Professor Elber nahm gerade Harrys Tarnumhang vom Gestell und reichte es Harrys Hauself. \enquote{Bringen Sie den Umhang bitte in das Zimmer Ihres Herrn. Ich nehme an, Sie wissen, wo er hingehört.} Kreacher nickte und sah danach zu Harry. \enquote{Ich werde ihn in den Krankenflügel bringen.}

\enquote{Wenn Kreacher darf, er kann es schneller.}

\enquote{Wenn ich mit darf?}

Kreacher nickte und nahm beide bei der Hand, sobald Elber in Reichweite war. Kurz darauf waren sie auf der Krankenstation angekommen und Kreacher entschuldigte sich und verschwand.

\enquote{Poppy, Kundschaft.}

Madame Pomfrey kam aus ihrem Büro und schaute Harry an, der gerade auf ein Bett geschwebt wurde. \enquote{Was ist mit ihm?}

\enquote{Geistig erschöpft. Entweder war ich zu nachlässig und hätte ihn eher aufhalten sollen, oder er hat sich zu sehr verausgabt. Suchen Sie sich etwas aus.}

\enquote{Wie soll ich das verstehen?}

\enquote{Wir sind irgendwie beide an seinem Zustand schuld.}

Dann erwachte Harry für einen kurzen Augenblick. Er sah Professor Elber, lächelte kurz und meinte dann: \enquote{Es war ein \accentuate{A} aus Luftschlangen.} Dann dämmerte er wieder ein.

Für die nächsten drei Stunden schlief er ruhig durch. Madame Pomfrey gab ihm während des kurzen wachen Momentes noch einen Schlummertrunk und ließ ihn dann schlafen.

In einem fremden Bett erwachte er. Er schlug die Augen auf und sah an die Decke. Dann setzte er sich auf und verließ das Bett. In seinem Inneren spürte er die Gewissheit, dass er noch immer im Krankenflügel in Hogwarts lag und schlief. Also hatte er wieder eine Art Vision. Er öffnete die einzige Tür im Raum und trat auf den Gang hinaus. Die Tür fiel hinter ihm ins Schloss und verschwand, denn als sich Harry herumdrehte, war sie nicht mehr vorhanden. Er ging die Treppe in das Erdgeschoss hinunter, da er Stimmen von dort vernahm. Er trat in eine Küche ein. Zwei Gestalten saßen an einem Tisch. Harry konnte keine Gesichter ausmachen. Es war eigenartig. Eine der Gestalten hatte vor sich eine Karte liegen. Harry trat näher heran, wurde aber ignoriert, oder nicht wahr genommen.

\enquote{Was machst du da?}, fragte eine weibliche Stimme. Sie klang verzerrt, sodass er sie nicht zuordnen konnte.

\enquote{Ich trage Sachen nach. Auf dieser Karte sind zwar viele Sachen verzeichnet, aber bei weitem nicht alle. Es fehlen noch einige.}

Harry sah nun genauer hin und entdeckte, dass sie der Karte des Rumtreibers zum Verwechseln ähnlich sah.

\enquote{Warum zeichnest du nur Punkte ein?}, fragte die weibliche Stimme erneut.

\enquote{Damit man sich auf die Suche machen muss, um etwas mehr über das Schloss zu erfahren}, sagte die männliche Stimme.

Auch diese konnte Harry nicht zuordnen. Wie die Stimme der Frau klang auch sie verzerrt.




\begin{kommentar}
Harry war mit seinem Tarnumhang auf dem Weg zum Quidditchfeld, wo er lernen sollte, durch Tarntechniken hindurch Objekte zu erkennen. Auf dem Boden sind bereits Buchstaben und weitere verbergen sich hinter den geschützten Objekten. Zusammen ergeben sie das Wort Tabaluga. Ein Drache mit selbem Namen wird später noch vorkommen. Den Namen gibt es wirklich. Es ist ein kleiner Drache, der vom Sänger Peter Maffay erdacht wurde. Es gibt mehrere Musicals über Tabaluga.
\end{kommentar}
