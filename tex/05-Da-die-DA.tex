\chapter{Da die DA}


\enquote{Sagt mal, habt ihr in diesem Fach überhaupt etwas getan?}, stellte Professor Elber die Frage an die Klasse. \enquote{Im ersten Jahr habt ihr nur ca. 70\% dessen erreicht, was im Lehrplan steht. Im zweiten Jahr habt ihr fast nichts gemacht. Das Klassenbuch hier liest sich so, als ob ihr keinerlei Unterstützung von eurer Lehrkraft bekommen habt.} Sein Blick streifte durch die Klasse \gst betretenes Schweigen \gst und er fuhr fort. \enquote{Wenigstens habt ihr im dritten Jahr was gelernt. Da habt ihr euer Soll sogar überboten. Im vierten Jahr habt ihr einen schönen falschen Professor gehabt, aber wenigstens hat er euch einen Einblick in \accentuate{die richtigen Bereiche} gegeben. Aber den Vogel hat ja wohl eure Lehrerin im fünften Schuljahr abgeschossen. \enquote{Theoretische Verteidigung} \gst blabla Mumpitz. Draußen geistert Voldemort herum und sie macht nur Theorie. Pah.} Wieder streifte sein Blick durch die Klasse und einige Schüler zuckten, als er Voldemorts Namen nannte.

\enquote{Eines kann ich euch jetzt schon mal sagen. Ich bin stinksauer. Da opfert man seine Zeit, tut Dumbledore einen Gefallen, gibt sein ruhiges Leben auf und nimmt sich Zeit, junge Leute in die Geheimnisse der gehobenen Magie einzuweihen, und dann stellt man fest, dass euch die Grundlagen fehlen. Ich glaube, ich muss mein ganzes Konzept auf den Kopf stellen. Ich muss mich nachher mal mit eurem Direktor unterhalten und ihm in den Hintern\abs Wir werden wohl mehr Praxis machen als Theorie. Die ersten Wochen werdet ihr wohl viel in eurer Freizeit lesen müssen, denn ihr habt einiges nachzuholen. Ich schätze mal so 14 Tage lang mindestens eine Stunde jeden Tag.} Betrübt blickte er auf sein Klassenbuch, nahm die Feder aus dem bereitstehenden Tintenfass und machte einen Eintrag. Nachdem er fertig war, schlug er das Buch zu und steckte die Feder in das Tintenfass zurück. Er stand auf und lief Richtung Tür. \enquote{Ihr könnt hier weiterlesen oder in euren Gemeinschaftsräumen. Ich gehe mal zu Dumbledore.} Mit diesen Worten verließ er das Klassenzimmer und ließ die Schüler mit einem Staunen im Gesicht zurück.

Es war spät, als Harry im Gemeinschaftsraum saß und an seinen Hausaufgaben arbeitete. Eine Viertklässlerin war die letzte, die sich von ihm verabschiedete. \enquote{Gute Nacht Harry}, sagte sie.

\enquote{Gute Nacht Emilie}, gab Harry zurück.

Als er mit seinen Hausaufgaben fertig war, nahm er sich noch das Fachbuch und setzte sich auf eine Couch. Er wollte noch etwas nachlesen, bevor er ins Bett ging. Doch nach wenigen Minuten schlug er sein Buch wieder zu, nahm seine Sachen mit und wollte gerade ins Bett gehen, als er Schritt von den Mädchenschlafsälen hörte. Gespannt wartete er, wer so spät abends noch herunterkam. Es war Tamara, Malfoys kleine Schwester.

\enquote{Harry?}, fragte sie. \enquote{Kann ich kurz mit dir reden?}

Harry legte seine Sachen wieder auf dem Tisch ab, setzte sich in ein Sofa, schlug mit der Hand neben ihm ein paar Mal auf und sagte dann: \enquote{Was hast du auf dem Herzen?} Fast hätte er noch \accentuate{Kleine} gesagt.

Sie setzte sich neben ihn, schlüpfte aus ihren Pantoffeln und drehte sich so, dass ihr Kopf auf einer Armlehne lag und ihre Beine angewinkelt nach oben zeigen. Ihre Füße berührten Harry nicht. Dann saß sie schweigend da.

\enquote{Draco hat mir erzählt\abs}, dann stockte sie wieder eine Weile. \enquote{Draco hat mir erzählt, dass du es ihm nicht gerade leicht machst.} Sie erzählte es in einer Ruhe, dass Harry nicht wagte, sich darüber zu beschweren. Also hörte er ihr weiter zu. \enquote{Trotzdem bewundert er dich. Irgendwie.}

Jetzt verschlug es Harry endgültig die Sprache. \gedanke{Malfoy bewundert mich?}

\enquote{Er hat es mir nie direkt gesagt, aber\abs} Wieder pausierte sie kurz. \enquote{Es klang immer wieder durch, wenn er über dich sprach. Ich glaube, er hätte nie den Mut dazu gehabt, das zu tun, was du getan hast.}

Dann stand sie auf und ging nach oben. Ohne sich umzudrehen, sagte sie noch: \enquote{Gute Nacht, Harry.}

Harry blieb verwirrt eine Zeit lang sitzen. \gedanke{Warum hat sie mir das erzählt? Ist sie Schlaf gewandelt und kann sich morgen nicht mehr daran erinnern?} Dann stand er auf, nahm seine Sachen vom Tisch und ging nach oben. Nachdenklich schlief er ein.

Am nächsten Morgen hatte er wieder Vgddk.

Nach der Stunde sprach ihn Dean an. \enquote{Du Harry, machst du wieder die DA? Nachdem Umbridge nicht mehr da ist, können wir sie doch wieder aufleben lassen. Zudem können wir dann gleich für den Unterricht hier üben.}

\enquote{Ich weiß nicht, Dean. Das muss ich mir erst einmal durch den Kopf gehen lassen. Vielleicht läuft das unserem Unterricht hier auch entgegen. Es wäre also kontraproduktiv.}

\enquote{Überlege aber nicht zu lange. Wir warten auf deine Antwort.}

\enquote{Wer ist wir?}, fragte Harry skeptisch nach.

\enquote{Gut zwei Drittel der alten Truppe. Die anderen konnte ich noch nicht fragen.}

\enquote{Wie, du hast alle gefragt?}

\enquote{Nicht alle. Aber ja, ich wollte erst mal nachfragen, ob noch Bedarf besteht, bevor ich dich damit überfalle.}

Harry sah zu seinem Lehrer, der interessiert dem Gespräch folgte.

\enquote{Mal sehen}, sagte Harry, als er Dean wieder ansah.

\enquote{Haben Sie noch eine Frage, Harry?}, fragte ihn Professor Elber, als schon alle gegangen waren und Harry alleine im Raum stand und seinen Gedanken hinterher hing.

\enquote{Wie?}

\enquote{Ob sie noch eine Frage haben, wollte ich wissen!}

\enquote{Nein. Oder doch. Können wir auch Lerngruppen machen, um den Stoff aufzuholen und auch praktisch üben?}

\enquote{Sehr gerne. \gst Das, was ich gerade über eine DA mitbekommen habe\abs Deswegen hat mich Dumbledore schon vorgewarnt. Er konnte mir nicht viel sagen, außer, dass sie einer kleinen Gruppe Zaubersprüche aus diesem Fach beigebracht hatten.} Harry nickte. \enquote{Falls sie Hilfe zu einem Zauber brauchen, wir Lehrer stehen zur Verfügung. Suchen Sie einfach einen von uns auf und fragen nach. Ich habe nichts dagegen. Führen Sie die Gruppe nur fort.}

Harry war erleichtert und sagte: \enquote{Danke.}

\enquote{Wofür? Für die Genehmigung, dass sie üben dürfen? Das hätten sie auch ohne mich gemacht.} Bevor Harry den Raum verließ, meinte er noch: \enquote{Wir zwei werden demnächst anfangen zu üben. Ich werde ihnen Sachen beibringen, die sie in der Schule nicht lernen werde, die ihnen aber hilfreich sein werden. Unter anderem Tarnen und Aufspüren. Dann Angriffs- und Verteidigungszauber. Dann kommen noch Element- und Gedankenzauber dazu. Wenn sie diese Gruppe leiten werden, dann werde ich auch Sachen einbauen, die sie den anderen beibringen können. Ansonsten gilt: \inner{Kein Wort zu niemandem darüber, was sie lernen.} Wir werden diesen oder nächsten Samstag anfangen.}

Harry nickte. Er war froh, etwas zu haben, um die DA weiterhin ausbilden zu können.

\trenn

\enquote{Willkommen zurück}, begrüßte sie  Professor Sprout im Gewächshaus. \enquote{Heute werden wir mit einer neuen Pflanze arbeiten und dabei etwas über sie lernen. Ich habe neue Pflanzen aus Japan erhalten, über die noch wenig bekannt ist. Wir werden uns die nächsten Wochen mit diesen Pflanzen auseinandersetzen. Auf den Tischen vor ihnen sehen sie die Exemplare. Sie werden sich zu viert um die Pflanzen kümmern. Die Liste hängt am Eingang aus.}

Die Schüler gingen zu den Listen und schauten, auf welchen Platz sie gehen mussten. Harry wurde mit Susan Bones und Alice McBrite aus Hufflepuff, sowie mit Neville zusammengebracht. Alice wollte gerade die Pflanze anfassen, um sie zu untersuchen, doch Neville hielt ihren Arm zurück.

\enquote{Langsam Alice. Wir wissen nichts über diese Pflanze. Es könnte eine giftige sein, oder eine fleischfressende, die sich getarnt hat.} Dann zog er seine Drachenhandschuhe an und griff nach der Pflanze. Als er sie berührte, fing sie an ein giftiges Sekret über die Handschuhe laufen zu lassen und kleine Tentakel versuchten den Handschuh zu packen. Dies verhinderte Neville aber, indem er seine Hand rechtzeitig zurückzog. \enquote{Seht ihr?}

\enquote{Und warum hast du keinen Gesichtsschutz?}, fragte Susan nach.

\enquote{Wie kommst du da drauf? Ich lege generell einen Zauber über mein Gesicht, bevor ich eines der Gewächshäuser betrete. Nur für den Fall der Fälle.}

Die beiden Hufflepuffs und Harry waren erstaunt darüber, wie professionell Neville mit der neuen Pflanze umging. Er führte praktisch die kleine Gruppe an und sagte ihnen, wie man vorgehen solle. Harry war richtig beeindruckt. Nach einer viertel Stunde kam Professor Sprout zu ihnen und stand nun hinter Neville. Keiner der vier bemerkte sie, da sie in ihre Arbeit vertieft waren. Neville war in seinem Element.

\enquote{Ich muss ihnen gratulieren, Mister Longbottom. Beeindruckend, wie sie ihren Mitschülern das wichtigste über die Vorgehensweise bei unbekannten Pflanzen beibringen. Sie haben sicher gesehen, dass es manch anderer Gruppe nicht gut erging. Sie mussten zu Poppy\abs Verzeihung, Madame Pomfrey. Da haben sie Glück gehabt, dass sie mit Mister Longbottom zusammen gekommen sind.}

Neville wurde rot.

\enquote{Sie haben uns doch mit ihm in eine Gruppe gesteckt}, meinte Susan.

\enquote{Nein, nein. Die Liste ist per Zufall entstanden. Ich habe eine Namensliste genommen und dann einen Zauber gesprochen, der die Nummern zufällig verteilt hat. Ich bin unschuldig.} Dann ging sie zu einer anderen Gruppe.

Harry sah ihr hinterher. Sie ging zu einer Gruppe, in der Hermine stand. Einer aus ihrer Gruppe verdrehte die Augen. Dann sah er zu Harry, der ihn schmunzelnd ansah. Er formte das Wort \accentuate{Redeschwall}, worauf sein gegenüber stumm nickte und sich wieder der Pflanze widmete.

Am Ende der Stunde auf dem Rückweg zum Schloss hatte er einen Einfall. Er nahm seine verzauberte Münze aus der Tasche und setzte ein neues Datum. Es war der kommende Samstag. \gedanke{Mal sehen, wie viele kommen werden}, dachte er noch, als er die Münze wieder in seine Tasche schob.

Gleich nach dem Essen zogen Ron und Hermine Harry beiseite.

\enquote{Was hast du vor, Harry?}, fragte Ron. \enquote{Willst du die DA wieder aufleben lassen?}

\enquote{Wie kommst du\abs} Ron hielt ihm eine Münze hin. \enquote{Ja. Dean hat mich gefragt. Ich möchte vorfühlen, wer kommt, ob Bedarf besteht und wie der aktuelle Stand ist.}

\enquote{Glaubst du, das brauchen wir jetzt noch?}, wollte Hermine wissen.

\enquote{Laut Dean schon. Er hat schon einige gefragt und ist selber schon gefragt worden.}

\enquote{Und was willst du ihnen dann beibringen?}

Harry überlegte kurz. \enquote{Zuerst machen wir den Patronus nochmal. Dann eine Wiederholung der bisherigen Sachen. Die Übungen für \VgddK und dann noch einige andere Sachen. Da lasst euch überraschen. \gst Kommt jetzt, gehen wir in unseren Gemeinschaftsraum. Ich möchte diesen lästigen Aufsatz endlich hinter mich bekommen.}

Ron und Hermine schauten sich an, nickten kurz und nahmen Harry zwischen sich mit.

Ein paar Tage später war es so weit. Erstaunlicherweise hatte sich der Raum der Wünsche selbst repariert. Oder waren es die Elfen gewesen? Keiner wusste es so genau. Harry wartete mit Ron und Hermine bereits, als die ersten Schüler eintrafen. Zehn Minuten später waren alle der alten DA da. Selbst Marietta Edgecombe konnte er ausmachen. Sie schien irgendwie abseits zu stehen.

\enquote{Es freut mich, dass alle kommen konnten. Ich habe mir gedacht, wir machen da weiter, wo wir das letzte Mal aufgehört haben. Wir üben erst einmal den Patronus.} Harry erschuf seinen und der Hirsch lief langsam durch den Raum, während er weiter erzählte. \enquote{Denkt an ein sehr schönes und intensives Erlebnis. Dann sprecht ihr: \spruch{Expecto Patronum} und ruft euren Beschützer herbei. Ob als Figur oder als Nebelschwade. Hauptsache er schützt euch.}

Während die Gruppen zu üben begannen, ging Harry auf Marietta zu, nahm sie zur Seite und setzte sich neben sie auf den Boden.

\enquote{Magst du mir sagen, was dich dazu gebracht hat?}, fragte er. Sanft aber traurig. Seine Wut und sein Zorn waren in der langen Zeit seit ihrem Verrat verraucht.

Noch immer hatte sie ihre Haare über ihre Stirn gezogen um das Wort \accentuate{Petze}, das sie zierte zu verbergen. \enquote{Damals, als Umbridge noch hier war\abs}, dabei schüttelte sie sich leicht, \enquote{\aabs Meine Mutter sagte damals zu mir, dass ich aufpassen solle, mit wem ich mich zeigen soll. Da sah ich es als meine Pflicht an\abs} Erste Tränen liefen ihre Wangen hinab. Harry holte ein Taschentuch heraus und gab es ihr. \enquote{Als meine Mum davon erfahren hat \gst Umbridge hat es ihr am selben Tag noch gesagt, was ich für eine vorbildliche Tochter wäre \gst gab es nach meiner Heimkehr erst einmal eine kräftige Standpauke. Dann hat sie mich in Ruhe gelassen, damit ich nachdenken konnte. \gst Erst langsam, dann immer schneller wurde mir klar, dass mich meine Mutter gewarnt hatte, offen zu dir zu stehen. Ich sollte dich warnen. Ich hatte sie komplett falsch verstanden.}

Harry legte einen Arm um sie und zog sie ein Stück zu sich heran. Dann strich er ihr die Haare aus der Stirn. \enquote{So siehst du schon viel besser aus.}

\enquote{Bitte nicht. Ich will nicht, dass man das sieht.}

\enquote{Du willst nicht, dass man deine glatte Stirn sieht?}

\enquote{Glatte Stirn? Wo lebst du denn\abs} Sie stutzte, da sie über ihre Stirn fuhr, um ihre Haare wieder runterzuziehen. \enquote{Sie ist glatt!}, staunte sie. \enquote{Was hast du gemacht?}

\enquote{Ich? Gar nichts. Wahrscheinlich wurde der Zauber gebrochen.}

\enquote{Aber durch was?}

\enquote{Dein Eingeständnis, deine Reue und meine Bereitschaft, dir zu Verzeihen. \gst Weißt du, Dumbledore hat mir mal erzählt, dass es Arten von Magie gibt, die durch Taten oder Gefühle ausgelöst werden. Dazu braucht es keinen Zauberspruch.}

Mittlerweile hatten die anderen aufgehört zu üben und hörten ihnen zu. Jeder bekam mit, wie sich Marietta an die Stirn fasste. Jetzt gehörte sie wohl für alle wieder ganz zur Gruppe und wurde nicht nur geduldet. Herzlich wurde sie in der Gruppe aufgenommen. Für heute war die Stunde beendet. Harry übte noch eine viertel Stunde mit ihr alleine.

Die nächsten zwei Wochen vergingen wie im Flug und es war der Abend, an dem er sich wieder mit Luna treffen wollte. Harry saß gerade in der Großen Halle gegenüber Ron.

\enquote{Eulen-Pudding}, hörte Harry nur und begann daraufhin Ron danach zu fragen.

\enquote{Was ist das?}, fragte Harry.

\enquote{Was?}, fragte Ron, \enquote{Eulen-Pudding?}

\enquote{Ja genau.}

\enquote{Nun, das ist eine Art Pudding, der, wenn du ihn isst, deinen Hals in den einer Eule verwandelt. Das heißt du kannst deinen Hals um fast 360 Grad drehen, ohne dich umzudrehen.}

\enquote{Du hättest mal McGonagalls Gesicht sehen sollen, als sie uns von hinten antippte und wir nur unseren Kopf drehten}, warf Seamus ein. Harry prustete fast sein Essen heraus, das er gerade wieder in sich hinein stopfte, und konnte sich nur schwer beherrschen.

\enquote{Das heißt ihr hab ihn an euch ausprobiert und an McGonagall getestet?}, fragte Harry immer noch entsetzt und dabei leicht grinsend.

\enquote{Ja}, meinte Seamus. \enquote{Hat Spaß gemacht. Aber was heißt getestet. Fred und George haben die gemacht. Das war ein Probeexemplar.}

Harry schluckte seinen Bissen herunter, schaute auf seine Uhr und verließ die Große Halle. \enquote{Wo gehst du hin, Harry?}, fragte Ron.

\enquote{Noch was in der Bibliothek nachschlagen}, log Harry, genau wissend, dass er sich gleich mit Luna treffen wollte. Er verließ die Große Halle und machte sich auf den Weg zum Gemeinschaftsraum der Paare. Auf dem Weg dorthin traf Harry auf Peeves. Jener kleine Taugenichts-Geist, der nur Unsinn im Kopf hatte. Er bemerkte Harry zwar, grinste ihn frech an, ließ ihn aber in Ruhe. \gedanke{Scheinbar hat er etwas Größeres vor}, dachte Harry und lief weiter in Richtung Westflügel.

\enquote{Was haben sie da in der Hand, Frederick?}, fragte Professor Flitwick, der mit Professor Elber und Professor Sprout unterwegs war.

\enquote{Einen Plan des Schlosses}, sagte Professor Elber.

Harry horchte auf und folgte ihnen in sicherer Entfernung.

\enquote{Wie meinen sie das?}, fragte Professor Sprout.

\enquote{Das ist ein Bauplan. Alle Räume, Gänge und Innenhöfe, sowie Türme und andere Anbauten sind darin verzeichnet. Ich habe darauf einen Hof gesehen. Er wird als Rosenhof bezeichnet. Jetzt möchte ich ihn mir mal ansehen, da an der Stelle wo der Durchgang sein sollte, ein Teppich hängt.}

\enquote{Und warum sollen wir dann mitkommen?}, fragte Professor Flitwick.

\enquote{Weil ich sie vermutlich brauchen werde. Die Pflanzen im Garten werden überwuchert sein. Deswegen sie, Pomona. Und vermutlich sind sie auch aggressiv. Es sind laut diversen Aufzeichnungen \accentuate{Rosa clavicula aggressiva}\footnote{Aggressive Rankrosen}. Deswegen auch sie, Filius.}

\enquote{Rechnen Sie mit Problemen, Frederick?}, fragte der kleine Zauberkünstler.

\enquote{Könnte sein.}

Sie kamen vor der besagten Stelle zum Stehen. Harry stand um die Ecke etwa zehn Meter entfernt und beobachtete die Gruppe. Professor Elber zog seinen Zauberstab, sah noch einmal auf dem Plan nach und tippt dann den Teppich an verschiedenen Stellen an. Der Teppich rollte sich nach oben und Professor Elber stutzte. Er sah erneut auf die Karte und tippte danach auf einen Stein neben der hinter dem Teppich erscheinenden Tür. Der Teppich rollte sich wieder ab und Professor Elber tippte den Teppich an anderer Stelle an und er verschwand. Er klappte die Karte zusammen und steckte sie ein.

Er trat zur Seite und deutete seiner Begleitung an, es ebenfalls zu tun. Die beiden drückten sich ebenso wie Professor Elber an die Wand und dieser tippte mit seinem Zauberstab die Tür an. Diese verschwand und sofort fielen dornige Ranken auf den Gang und schlichen am Boden entlang. Auswüchse, aussehend wie eine Venus-Fliegenfalle suchten Opfer und kamen den drei Zauberern immer näher.

\enquote{Deswegen habe ich sie mitgenommen}, sagte Professor Elber.

Professor Sprout schwang ihren Zauberstab und versuchte die Ranken zurückzudrängen. Zu dritt kamen sie langsam voran und drängten die Pflanzen durch die Tür.

\enquote{Wir müssen sie knapp über dem Boden kappen und danach entwurzeln}, sagte Professor Elber.

\enquote{Nein}, antwortete Professor Sprout. \enquote{Es reicht sie zu kappen. Ich habe einen Zauber, der sie vor dem Nachwachsen schützt. Diese Pflanzen sind zu wertvoll, um sie zu verlieren.}

Die beiden Zauberer nickten und verschwanden nach Professor Sprout im Hof.

Harry trat hinter der Ecke hervor und schaute nach einigen Metern neugierig in den Hof. Die letzten Sträucher wurden gerade gekappt und danach begann die Verbrennung der riesigen Mengen an Schnittgut.

Harry ging weiter Richtung Westflügel, während die drei Professoren sich nach getaner Arbeit auf eine Bank setzten, die Frau in die Mitte, und sich begannen zu unterhalten.

\enquote{Woher haben sie den Plan eigentlich?}, fragte Professor Flitwick neugierig.

\enquote{Der ist schon lange in unserer Familie weitergegeben worden.} Wie vom Donner gerührt sahen ihn die beiden an. \enquote{Ich kann ihnen nicht sagen, woher er stammt, oder wie alt er ist. Als ich von Dumbledore eingestellt wurde, habe ich ihn mitgebracht. So konnte ich mich zumindest auf die Örtlichkeit anpassen.}

Oben angekommen lief Harry Richtung Porträt, blieb vor ihm stehen und sagte: \zauber{Aqua Neros!} Doch nichts geschah. Harry versucht es abermals. \enquote{Aqua Neros!}, doch noch immer geschah nichts. Plötzlich hörte er Schritte. Er drehte sich um und sah, dass Luna auf ihn zukam.

\enquote{Hallo Harry}, sagte sie mit dem gleichen, verträumten Gesichtsausdruck wie immer. \enquote{Was ist los?}

\enquote{Ich komme nicht rein}, sagte Harry.

\zauber{Aqua Neros!}, sagte Luna, als sie sich dem Porträt zuwandte. Doch es öffnete sich nicht. Bange Sekunden verstrichen, als Luna plötzlich sagte, \enquote{Ich glaube, wir sollten es gemeinsam versuchen. Das ist doch der Gemeinschaftsraum der Paare.}

Harry nickte und fast zeitgleich sagten beide: \zauber{Aqua Neros!}

Das Porträt öffnete sich und beide gingen hinein. Da es noch etwas früh für das zu Bett gehen war, schaute sich Harry genauer um. Er nahm in einem, wie Luna sagte, Schaukelstuhl der Ravenclaws Platz und betrachtete die Decke. Dort hingen die vier Hausfahnen von Gryffindor, Ravenclaw, Hufflepuff und Slytherin. Ihm kam es vor, als ob die Decke in diesem Raum etwas höher war. Er ließ seinen Blick schweifen, und stellte fest, dass der Rauchabzug des Kamins ihm gegenüber mit dem Hauswappen von Hogwarts besetzt war. An beiden hervorstehenden Säulen, die vom Rauchabzug bis an den Boden gingen, waren die Wappentiere der vier Häuser in Stein gemeißelt. Harry überlegte, warum hier alle vier Häuser vertreten waren, wo doch nur die beiden Gründer dieser Räumlichkeiten darin unterwegs gewesen waren. Vielleicht lag es daran, dass sie vorhatten, nachdem sie gegangen waren anderen Pärchen hiervon zu erzählen, aber keine Zeit mehr dazu hatten oder es aus anderen Gründen nicht mehr konnten.

Der ganze Raum strahle eine Gemütlichkeit aus, die Harry gefiel. Also schloss er die Augen und schaukelte ein wenig. Währenddessen schaute sich Luna an der Wand hinter ihm um und entdeckte hinter einem Vorhang das Ende eines Regals. Sie schob den Vorhang beiseite und fand ein Paar Bücher. Nur eines davon hatte auf dem Rücken etwas stehen. Auf dem smaragd-roten Buchrücken war in schwarzen Lettern zu lesen: \enquote{Geschichte des Gemeinschaftsraumes der Paare von 1875\agst}

\enquote{Schau mal, Harry}, sagte Luna und zog das Buch heraus. Harry öffnete die Augen und drehte sich zu Luna um. Sie zeigte ihm das Buch, setzte sich in einen Stuhl neben ihm und blätterte darin. Es waren kaum Seiten beschriftet, aber als Harry den letzten Eintrag sah, versteinerte seine Mimik. \accentuate{September, 1996\\ Harry Potter, Luna Lovegood haben hier genächtigt.}

\enquote{Das scheint eine Art Tagebuch über die Aufenthalte hier zu sein}, sagte Luna. Sie schlug das Buch zu und ging an den Tisch, an dem sie das letzte Mal Schach gespielt hatten. Sie winkte ihn herüber und zauberte die Figuren wieder her. Eigentlich hatte Harry keine Lust zu spielen, aber wenn er Ron mal wieder schlagen wollte, brauchte er ein wenig Training. Also stand er auf und lief zu Luna, um mit ihr zu spielen.

Nach einigen viel zu kurzen Runden, in denen Luna immer wieder gewonnen hatte, häufiger als er, machten sie sich auf den Weg zu den Schlafräumen, denn hier gab es mehr als nur einen. Er zog seinen Zauberstab aus der Innentasche seiner Robe und ließ die Figuren verschwinden. Anscheinend hatten die beiden Erbauer vorausgedacht und hier so manche Zeit verbracht. Harry wunderte sich immer noch, wie die beiden an die ganzen Sachen hier gekommen waren. Er machte sich auf den Weg in den Schlafraum wo Luna ihn schon erwartete. Sie war bereits umgezogen und unter die Decke geschlüpft. Harry zog sich ebenfalls um, d.h. er zog sich bis auf seine Unterhose aus und schlüpfte ebenfalls ins Bett. Beide drehten ihre Gesichter einander zu und schauten sich an. Da lag sie wieder, mit ihren weißen Haaren, ihren blauen Augen und einem Gesichtsausdruck, der jedem sagte, sie sei nicht bei der Sache, wenn er sie anschaute. Aber Luna hatte einen sehr scharfen Verstand, wie Harry letztes Jahr schon bemerkt hatte. Sie hatte kaum Freunde, weil alle sie für \gst \accentuate{wie hatte sie sich ausgedrückt} \gst speziell hielten. Luna griff nach Harrys Hand und zog sie etwas zu sich. Da es im Zimmer warm war, hatten beide die Decke nur bis zu den Hüften hochgezogen und die Hände einander haltend über dem Stoff liegen. Langsam schloss Luna die Augen und Harry wünschte ihr eine gute Nacht. Luna lächelte und bald schliefen sie ein und träumten. Unbemerkt für beide begann sich in der Nacht die Decke zu bewegen. Sie bewegte sich erst Richtung Beine, um kurz, nachdem die immer noch gefassten Hände auf das Leinentuch gefallen waren, die Richtung zu ändern. Die Decke begann an beiden entlang hoch zu wandern, um sie vor dem Auskühlen zu schützen.

Am nächsten Morgen lagen beide nebeneinander auf der Seite. Harrys Oberkörper schmiegte sich an Lunas Rücken und die gefassten Hände der Nacht zuvor waren geöffnet. Harry öffnete verschlafen die Augen und roch an Lunas Haar. Ihm stieg ein Duft die Nase hoch wie er ihn an ihr noch nie wahrgenommen hatte. \gedanke{Um ernst zu bleiben}, dachte Harry, \gedanke{du hast noch nie an ihrem Haar gerochen, oder an irgendeinem anderen Körperteil.} Er setzte sich etwas auf und gab ihr einen Kuss auf die Backe, sodass sie erwachte.

Sie drehte sich auf den Rücken und rieb sich die Augen. Da war er wieder, dieser eigenartige Blick in ihrem Gesicht. Nicht verträumt, nicht abwesend, sondern kristallklar. Er gab ihr einen Gute-Morgen-Kuss auf den Mund den sie erwiderte. Dieses Mal dauerte er etwas länger als beim letzten Mal, was sie aber nicht zu stören schien. Harry war wieder glücklich und nahm sich vor, nächstes Mal von ihr geweckt zu werden. Sie standen beide wieder auf und zogen sich ihre Schülerroben an.

Heute war Sonntag und Harry musste noch Hausaufgaben machen. Professor Sprout wollte etwas über Giftpilze wissen, und wie man das Gift aus ihnen herausbekommt, um einen wirksamen Heiltrank zu brauen. Beide liefen noch Händchen haltend den Gemeinschaftsraum entlang, bis sie zum Portraitloch kamen. Hinaus schauend entdeckten sie Filch, den Hausmeister, der mit einem Mob durch die Gänge patrouillierte. Als er um die Ecke gebogen war, warteten beide noch ein paar Sekunden um sich dann zurück in ihre Gemeinschaftsräume zu schleichen. Als Harry seinen Gemeinschaftsraum betrat, war noch keiner da. Also beschloss er erst einmal zu duschen. Er schlich sich langsam nach oben und kurze Zeit später lief das klare Nass an ihm herunter.

Mit einem Handtuch um die Hüften verließ er die Dusche wieder und zog sich in seinem Zimmer um. Er ging wieder nach unten und setzte sich im noch immer leeren Gemeinschaftsraum in einen Sessel mit Armlehnen. Das Feuer im Kamin war aus und Harry betrachtete die Decke. Nach ein paar Minuten anstarren sah er sich in seinem Raum um und versuchte Gemeinsamkeiten zu erkennen; zwischen diesem und dem anderen Raum. Er versuchte sich auch an den Gemeinschaftsraum der Slytherins zu erinnern, in dem er in seinem zweiten Schuljahr einmal gewesen war. Er erinnerte sich an ein paar schwarze Sessel und ein schwarzes Sofa. Ein Kamin mit Schlangen die ihn einfassten und einen Erker, der in der Mitte einen Tisch und außen eine an die Wand gebaute Bank hatte. Harry hörte wie eine Tür aufging, war aber mit seinen Gedanken so beschäftigt, dass er es gar nicht realisierte, wie sich der Raum langsam mit Leben erfüllte. Er dachte immer noch an die Nacht mit Luna, den Kuss am nächsten Morgen und versuchte sich die Gemeinschaftsräume der Ravenclaws und der Hufflepuffs vorzustellen.

Harry realisierte erst wieder wo er war, als ihn Ron mit einem Knuff gegen den Arm aus seinen Träumen riss.

\enquote{Morgen Harry}, sagte er.

\enquote{Morgen Ron.}

\enquote{Gut geschlafen?}

\enquote{Ja.}

\enquote{Komm, lass uns Frühstücken. Hermine braucht noch ein bisschen und kommt dann mit Ginny nach.}

\enquote{Ja, geht in Ordnung.}

Noch leicht verträumt folgte er Ron durch das Porträt und ging Richtung Große Halle. Als Ron Dean sah, entschuldigte er sich bei Harry und meinte. \enquote{Ich geh' mal kurz zu Dean. Den muss ich noch was fragen}, und sprang davon. Kurz darauf bog Luna um eine Ecke und lief das restliche Stück neben Harry her. Harry war noch immer in Gedanken über die Einrichtung im Gemeinschaftsraum der Paare versunken und Luna hatte ihren üblichen Ausdruck in den Augen.

Als ein Gryffindor-Fünftklässler die beiden Seite an Seite sah, konnte er sich nicht beherrschen. Er stupste Ron an, zeigte auf die beiden und sprach dann: \enquote{Das schönste Liebespaar, das Hogwarts seit mehr als hundert Jahren gesehen hatte.}

Dies sagte er so laut, dass alle, die hinter den dreien standen, sich abrupt umdrehten und zu kichern anfingen. Fast zeitgleich kam Colin, der Haus- und Hof-Fotograf der Gryffindors und der restlichen Schule, um ein Foto von den beiden zu machen. Erst jetzt kam Harry wieder zu klarem Verstand und verspürte einen Drang, seinem Mitschüler eines auszuwischen. Er zwinkerte Colin zu, der sofort wieder seine Kamera vor sein Auge hob, und drehte sich zu Luna um, die ihn nun anlächelte und ihn in den Arm nahm. Beide sahen zu Colin und lächelten ihn an. Der machte ein Foto von den beiden. Harry löste sich von Luna und ging zu Colin, um ihm etwas ins Ohr zu flüstern. Dieser nickte und verschwand, da er scheinbar mit dem Frühstück schon fertig war. Harry stellte sich jetzt zu Dean und Ron und meinte nur: \enquote{Kommt, lasst uns frühstücken} und fügte mit einem leicht schnippischen Grinsen zu seinem Mitschüler, \enquote{Klatschpresse von Hogwarts}, hinzu. Luna war schon an ihrem Platz und setze sich hin, um zu frühstücken, während Harry neben Ron in die Große Halle eintrat und auf einen freien Platz zu lief. Er und Ron setzen sich und Lavender und Sally saßen ihnen gegenüber.

\enquote{Was hast du zu Colin gesagt?}, wollte Dean wissen. \enquote{Hast ihm gedroht ihn zu verhexen, wenn er die Bilder nicht vernichtet? So wie der gerannt ist} und drehte sich grinsend Sally und Lavender zu.

\enquote{Das werdet ihr schon sehen}, antwortete Harry, in dessen Kopf sich ein Plan zu formen begann. Ein Bild hatte er vor, im Tagebuch im Gemeinschaftsraum der Paare zu verewigen, und das andere mit einem netten Kommentar leicht modifiziert im Gryffindor-Gemeinschaftsraum aufzuhängen.

Dann machte Professor Dumbledore eine Ankündigung: \enquote{Wie ich soeben mitgeteilt bekommen habe, haben wir einen weiteren Innenhof zur Verfügung. Einen sogenannten \accentuate{Rosenhof}. Es dauert noch ein paar Tage, bis er in altem Glanz erstrahlt, da er Jahrhunderte lang verschlossen war und entsprechend verwahrlost aussah. In diesem Innenhof werden viele Bänke zum Entspannen aufgestellt und viele duftende Rosen werden den Hof zieren. \gst Aber! Gehen Sie nicht zu nah ran. Diese Rosengewächse sind sehr selten und entstammen der Gattung \accentuate{Rosa clavicula aggressiva}. Diese Rosen-Art ist eine Kreuzung zwischen normalen Rosen und der Venus-Fliegenfalle. Halten sie also gebührenden Abstand. \gst Noch ein Hinweis: Spezielle Schutzzauber verhindern Schlimmeres. Sie können sich an diesen Rosen durchaus verletzen, aber es werden keine lebensbedrohlichen Verletzungen sein. Sie werden höchstens gezwickt werden oder bekommen einen Ausschlag, den sie dann bei Madame Pomfrey kurieren müssen. Gehen Sie also nicht zu nah ran.}

\trenn

Diesen Donnerstag hatte Harry wieder Pflanzenkunde bei Professor Sprout und er musste seinen Aufsatz abgeben. Sie mischten einen wirksamen Dünger zusammen, welcher die Pflanzen schneller wachsen lassen sollte. Zu ihrem Erstaunen hatte Harrys Wachstumstrank eine besonders schnell wachsende Wirkung, wofür er mit zwanzig Punkten belohnt wurde und Professor Sprout ihn nach der Stunde fragte, wie er das fertiggebracht habe. Er meinte, dass er zum Umrühren seinen Zauberstab verwendete und einen kleinen Wachstumszauber gesprochen habe. Dann schaute er noch in der Bibliothek vorbei, bevor er in dem Gemeinschaftsraum ging.

Er setzte sich neben Malfoys Schwester Tamara. \enquote{Hi Tamara. Ich hatte keine Zeit mehr dich zu fragen, was du meintest, als du sagtest: \enquote{Ich glaube, er hätte nie den Mut dazu gehabt, das zu tun, was du getan hast.} Du hast deinen Bruder gemeint.}

Tamara schaute ihn verständnislos an. \enquote{Daran kann ich mich nicht erinnern.}

\enquote{Ach komm schon, ich hatte meine Hausaufgaben gemacht und wollte noch etwas lesen, dann bin ich aufgestanden, als du in den Gemeinschaftsraum gekommen bist und dich mit mir unterhalten wolltest.}

Tamara sah ihn erstaunt an. In ihr Gesicht schienen sich einerseits Erkenntnis, andererseits aber auch Unkenntnis zu spiegeln.

\enquote{Oh}, sagte sie schließlich. \enquote{Weißt du}, und sie senkte ihren Blick, \enquote{ich Schlafwandle manchmal, Harry.} Harry zog eine Augenbraue hoch. \enquote{Und am nächsten Tag weiß ich nicht mehr, was ich danach getan habe.}

\enquote{Geh' zu Madame Pomfrey und lass dir was gegen Schlafwandeln geben.}

\enquote{Äh}, antwortete Tamara. \enquote{Wo lang?}

\enquote{Nicht wo lang. Zu Madame Pomfrey unserer Krankenhexe.}

\enquote{Oh}, entgegnete Tamara.

\enquote{Warte kurz.} Harry schrieb seinen Satz zu Ende, klappte seine Bücher zu und schob sie von der Tischkante. Sie fielen leicht ab und bewegten sich dann auf ihre Plätze zu. Dann stand er auf und sagte: \enquote{Gehen wir.}

Freudig nickte sie und packte schnell ihre Tasche. Das mit den Büchern versuchte sie auch gleich. Sie schob sie von der Tischkante, doch ihre Bücher fielen herunter. Harry konnte sich ein Schmunzeln nicht verkneifen.

Tamara schaute ihn mit Erstaunen an. \enquote{Da hast du mich aber schön hereingelegt}, antwortete sie ihm.

\enquote{Nicht wirklich}, meinte Harry. Er hob die Bücher auf und legte sie auf den Tisch. \enquote{Überleg’ mal wie ich die Bücher vom Tisch herunter geschoben habe und wie du die Bücher geschoben hast.}

Er konnte an ihrem Gesichtsausdruck erkennen, dass sie angestrengt nachdachte. Dann schüttelte sie den Kopf und meinte: \enquote{Ich sehe keinen Unterschied.}

Harry lächelte. \enquote{Pass auf.} Er schob das erste Buch mit der flachen Hand vom Tisch, worauf es herunterfiel. \enquote{Gesehen?} Tamara nickte nur. Dann nahm er ein zweites Buch und schob es mit dem Handrücken vom Tisch. Es fiel leicht ab und begann danach seinen Platz zu suchen. \enquote{Gesehen?} Tamara nickte. \enquote{Nun versuchst du es mal.}

Tamara schob jetzt auch mit dem Handrücken die Bücher vom Tisch. Sie fielen etwas weiter dem Boden entgegen, fingen sich aber schließlich und suchten sich ebenfalls den Weg zu den Regalplätzen. Dann hob sie das heruntergefallene Buch auf und fragte Harry: \enquote{Und was machen wir mit dem? Nochmal auf den Tisch legen und es herunterschubsen?}

Harry grinste sie an. \enquote{Heb’ es in die Luft, halte es an der Unterseite und lass es los.} Tamara schaute ihn leicht skeptisch an, tat aber, was Harry ihr sagte. Sie nahm das Buch an der Unterseite, hob es in die Luft und ließ es los. Langsam und sanft schwebte es an seinen Platz im Regal.

Dann sagte er: \enquote{Komm} und verließ die Bibliothek. Er glaubte noch aus den Augenwinkeln heraus einen blonden Jungen zu erkennen. Auf dem Wege zum Krankenflügel nahm Tamara seine Hand und lief neben ihm her. Er hatte das Gefühl, dass er verfolgt wurde, wollte sich aber nicht ständig umdrehen. Eine innere Stimme sagte ihm, dass es Draco Malfoy sein musste. Er lachte innerlich, obwohl er nicht wusste, wie er darauf reagieren würde.

Die Türen der Krankenstation öffneten sich und Harry betrat mit Tamara den Krankenflügel. Madame Pomfrey kam ihnen entgegen und fragte die beiden, was sie denn benötigen würden. Harry sah zu Tamara, welche seinen Blick auffing. Wortlos gab er ihr durch eine Kopfbewegung zu verstehen, dass es an ihr lag Madame Pomfrey zu sagen, was sie wollte.

\enquote{Ich brauche etwas gegen mein Schlafwandeln, Miss}, sagte sie.

Madame Pomfrey nickte und meinte: \enquote{Setzen sie sich. Das dauert höchstens zehn Minuten.} Sie verschwand in ihr Büro.

\enquote{Kann ich dich alleine lassen?}, fragte Harry.

Tamara musterte ihn. So, als wolle sie ihm sagen, sie sei bereits groß genug. \enquote{Ich komme zurecht. Danke Harry}, sagte sie.

Harry nickte und verließ die Krankenstation. In der Tür traf er auf Draco Malfoy.

\enquote{Malfoy.}

\enquote{Potter.}

Er hörte nur noch wie Malfoy zu seiner Schwester sagte: \enquote{Und kleine\abs äh Tamara, wie hast du dich bislang eingewöhnt?}

Im Gemeinschaftsraum angekommen, wartete Colin auf Harry. Er überreichte ihm einen Umschlag und nickte Harry zu. Der nahm ihn an sich und erblickte William, den Viertklässler, der Harry und Luna diesen peinlichen Moment beschert hatte, in der anderen Ecke des Raumes. Harry grinste William an. Es war ein fieses Grinsen. Dann suchte er sich eine freie Stelle im Gemeinschaftsraum und drückte den Umschlag samt Inhalt gegen die Wand. Er nahm seinen Zauberstab heraus und murmelte einen Zauberspruch. Der Umschlag blieb an der Wand hängen und Harry entfernte sich, seinen Zauberstab einsteckend. Kurz nachdem Harry die ersten Stufen zu den Schlafräumen hochgegangen war, rannte William an die Stelle, an der Harry den Umschlag an die Wand gepinnt hatte. In der Zwischenzeit löste dieser sich auf und Harry drehte sich um, um das Schauspiel zu bewundern. Vor dem Bild angekommen, sah William Luna und Harry Händchen haltend und atmete erleichtert auf. Dann jedoch passierte es. Luna verschwand und an ihrer Stelle tauchte Hermine auf. Nach ein paar Sekunden verschwand Harry und an seiner Stelle tauchte Ron auf. Jeweils die Hand des anderen haltend. Hermine verschwand und es tauchte Marlice auf, Williams Freundin. Seinen Blick konnte Harry nicht sehen, aber er nahm an, dass es nicht unbedingt erfreulich für ihn sein musste. Schließlich verschwand Ron und an seiner Stelle erschien William. Jetzt war Harry gespannt auf die nächste Paarung und bereitete sich schon einmal darauf vor, zu verschwinden, denn jetzt taucht an Marlices Stelle wieder Luna auf, die mit William Händchen haltend dastand. In diesem Moment kam Marlice durch das Porträt und für William unbemerkt schlich sie sich an ihn heran. Ihn eigentlich überraschen wollend, hörte er nur noch einen Schrei. Er drehte sich um und Harry verschwand in seinem Schlafraum.

Diese Standpauke wollte er wirklich nicht miterleben. \gedanke{Colin hatte gute Arbeit geleistet}, dachte Harry. \gedanke{Nicht nur mit dem Bild, sondern auch mit dem Timing, als er Marlice zum Gemeinschaftsraum lotste.} Harry hatte seine Rache und wollte erst einmal duschen. Nachdem er sich umgezogen hatte, ging er wieder in den Gemeinschaftsraum. William und Marlice waren verschwunden.

Ron grinste ihn an und Dean tätschelte ihm auf die Schulter. \enquote{Das hätte ich dir nicht zugetraut}, sagte er.

Harry grinste zurück und lud Ron zu einem Spiel Schach ein. Nach ein paar verlorenen Runden legte er seinen Kopf in die Rückenlehne des Stuhles und schloss die Augen. Langsam und sachte formte sich ein Raum vor seinen Augen. Er enthielt Elemente aus dem Gemeinschaftsraum der Paare, die Harry gesehen hatte. Jetzt war das Bild ganz klar in seinem Kopf. Er hatte das Gefühl als stünde er mitten im Raum und war vollkommen abwesend, hörte die Stimmen der anderen Ravenclaws und fand alle wieder, an die er sich erinnern konnte. Harry hatte nicht das Gefühl seine Beine zu bewegen und doch schien er zu einem Stuhl zu laufen, sich zu setzen und eine Zeitschrift in die Hand zu nehmen. Es war schwer etwas zu erkennen, denn die Zeitung hielt er scheinbar über Kopf. Plötzlich wurde er wieder wach und Ron und Hermine standen vor ihm mit einem besorgten Gesichtsausdruck.

\enquote{Geht es dir gut, Harry?}, fragte Hermine.

\enquote{Ja}, antwortete er, \enquote{ich bin nur etwas eingenickt.}

\enquote{Du hast im Schlaf geredet Harry}, fügte Ron ein.

Harrys Augen weiteten sich. \enquote{Was hab ich denn gesagt?}

\enquote{Du hast irgendein Rätsel gelöst}, sagte ihm Hermine.

Harry fiel wieder ein, wie er vom Gemeinschaftsraum der Ravenclaws geträumt hatte, da hatte auch eine Stimme laut ein Rätsel gelöst. So langsam fühlte er sich ein wenig ertappt, denn es kam ihm in den Sinn, dass die Zeitschrift der Wortklauber sein musste. Und den las nur Luna. \enquote{Hab ich es denn geschafft?}, fragte Harry mit einer unschuldig aussehenden Miene im Gesicht.

\enquote{Weiß nicht. Ich glaube nicht}, sagte Ron.

\enquote{Ich glaube, ich geh' dann mal schlafen}, antwortete Harry und machte sich erneut auf den Weg zu seinem Schlafsaal. Er musste unbedingt mit Luna darüber reden. Aber darüber konnte er auch Morgen noch nachdenken.

Mitten in der Nacht wachte Harry auf und ging aufs Klo. Nachdem er fertig war, fühlte er sich unruhig und dachte sich: \gedanke{Lese ich noch ein bisschen was, Morgen ist wieder Wahrsagen bei Trelawney und die Lektüre, die sie uns immer gibt, ist so herrlich einschläfernd.} Er schlich vorsichtig in sein Zimmer, holte seinen Morgenmantel, ging die Treppen zum Gemeinschaftsraum hinunter und setzte sich in den Sessel. Unten fiel ihm ein, dass er kein Buch mitgenommen hatte und stand wieder auf. Plötzlich fühlte er eine Präsenz. Erschrocken drehte er sich um, konnte aber niemanden entdecken. Er blieb noch ein paar Minuten stehen und das eigenartige Gefühl verlor an Intensität. Er begab sich wieder nach oben, zog seinen Morgenmantel aus und schlief bis zum nächsten Morgen durch.

Nach dem Frühstück bemerkte er, wie Luna die Große Halle verließ. Er sprang auf und ging ihr hinterher. \enquote{Hast du ein paar Minuten Zeit, Luna?}, fragte er sie.

Luna drehte sich um und sah ihn an. \enquote{Ja Harry. Gerne.}

\enquote{Laufen wir ein paar Schritte}, sagte Harry und begann den Flur entlangzulaufen; mit Luna neben sich. \enquote{Gestern hatte ich einen seltsamen Traum von eurem Gemeinschaftsraum}, sagte Harry.

Luna sah in an und fragte ihn dann: \enquote{Welcher Art?}

\enquote{Nun ja}, sagte Harry, \enquote{Ich konnte ihn genau sehen. Ich konnte die Schachtische sehen, den Kamin auf der linken Seite mit den Raben als Stützen, die Teppiche an der Wand und die Marmorstatue von Rowena Ravenclaw.}

Luna schaute ihn musternd an und fragte ihn: \enquote{Sag mal Harry, warst du schon mal in unserem Gemeinschaftsraum?}

\enquote{Nein,} sagte Harry. \enquote{Ich hatte aber nicht das Gefühl, dass ich das selbst war. Ich hatte mehr das Gefühl, dass ich durch die Augen eines anderen schaue und durch die Ohren eines anderen höre.} Luna öffnete ihren Mund, sagte aber nichts und Harry fuhr fort. \enquote{Ich lief vom Porträt weg und setzte mich in einen Stuhl neben Ravenclaws Statue aus Marmor, nahm eine Zeitschrift und las sie \gst verkehrt herum, löste ein Rätsel und dann wurde ich aufgeweckt.}

\enquote{Verrückt}, meinte Luna. \enquote{Genau das habe ich gestern Abend gemacht, nachdem ich aus der Bibliothek gekommen bin. Und du hast unseren Gemeinschaftsraum recht gut beschrieben.} Luna schaute wieder auf den Boden und sagte mit einem Zittern in ihrer Stimme: \enquote{Mir ist gestern auch was passiert}, dann sah sie ihm direkt in die Augen: \enquote{Ich hatte das Gefühl, dass ich während der Zeit beobachtet wurde, dass jemand in mir steckt und erlebt, was ich erlebe.}

Harry schluckte, \enquote{Genau das Gefühl hatte ich gestern Abend, kurz nach Mitternacht, als ich in unserem Gemeinschaftsraum saß.}

\enquote{Der mit unserem Bild an der Wand?}, fragte Luna.

Harry fiel der Mund herunter und blieb offen stehen. Er drehte sich zu Luna und fragte: \enquote{Was? Wer hat dir das erzählt?}

\enquote{Keiner}, antwortete sie. \enquote{Ich hatte es einfach gesehen. Ich habe geträumt ich säße in einem Stuhl der Gryffindors. Dann bin ich aufgestanden und habe mich umgesehen \gst im Körper eines anderen.}

Harry begann langsam zu begreifen und er zog Luna in einen leeren Klassenraum und verschloss hinter sich die Tür. \enquote{Ist dir klar, was das bedeutet, Luna?}, fragte er sie.

\enquote{Ja, wir haben durch unsere gemeinsamen Nächte in diesen Räumlichkeiten irgendeine Art Verbindung zueinander aufgebaut.}

Harrys Kinn fiel wieder herunter. Diese Antwort hatte er nicht erwartet.

Die Schulglocke läutete die erste Stunde ein und Harry verließ mit Luna das Klassenzimmer. Während der gesamten Stunde bei Professor Flitwick musste ihn Hermine mehrmals in seine Seite kneifen, da er mit seinen Gedanken immer wieder abschweifte und die Konzentration verlor. Er dachte immer wieder über das nach, was Luna ihm gesagt hatte. Über die Verbindung zwischen ihnen und über ihre seelische Verbundenheit. Er war sich nicht sicher, ob er sich wieder mit ihr dort treffen sollte, wusste aber instinktiv, dass es das richtige war. Oder er glaubte es zumindest.

\trenn

\enquote{Dieser Lehrer ist mir irgendwie unheimlich}, sagte Hermine. \enquote{Hast du das gesehen?}, fragte sie.

\enquote{Was denn?}, fragte Ron und auch Harry sah sie verständnislos an.

\enquote{Die Tiere reagieren anders auf ihn. Schaut mal}, sagte sie. Sie zeigte auf eine entfernte Stelle und auf einen Baum, in dessen Nähe gerade ein Schüler auf einem Pfad entlang ging. Vögel flogen aus dem Baum heraus und ließen sich kurz darauf wieder auf dessen Zweige und Äste nieder, als die vermeidliche Gefahr vorüber war.

\enquote{Was ist daran so besonders?}, fragte Ron, der nicht wusste, worauf Hermine hinaus wollte.

\enquote{Warte es ab und schau weiter.}

Wieder ging jemand an dem Baum vorbei und wieder flogen die Vögel.

\enquote{Was soll daran schon toll sein?}, fragte Ron schon halb genervt.

\enquote{Warte doch einfach mal ab Ron, sei nicht immer so ungeduldig.}

Entnervt gab Ron auf und schaute weiterhin auf den Baum.

\enquote{Und nun, was soll jetzt so besonders daran gewesen sein?}, fragte er, als wieder jemand am Baum vorbeiging.

\enquote{Das war er, aber die Vögel sind nicht aufgeflogen. Nichts hat sich gerührt.}

\enquote{Sie werden nicht mehr dort sein}, schloss Ron. Doch sein Argument wurde sofort widerlegt, als eine weitere Person am Baum vorbeilief und die Vögel erneut davon flatterten.

\enquote{Und woher willst du wissen, dass er es ist?}, versuchte Ron einen letzten Versuch.

\enquote{Sein Gang, sein Auftreten\abs}

Ron plusterte. \enquote{Soweit ist es schon? Erinnerst du dich noch an dein zweites Jahr?}

\enquote{Ronald Weasley, das ist keine Schwärmerei. Ich bin nicht in ihn verliebt oder so, ich bin skeptisch, was ihn betrifft. Schau doch mal, was er uns beibringt.}

Ron dachte nach.




\begin{kommentar}
Schon der Titel des Kapitels ist eine nette Anspielung auf das Morsealphabet. Und gleich ein paar Zeilen später, nachdem Elber festgestellt hat, dass den Schülern viel fehlt, will er Dumbledore in den Hintern treten. Ein schönes Bild, wenn man sich das so vorstellt. Kurz darauf spricht Elber Harry an und möchte ihn ausbilden. Er scheint zu wissen, dass Harry Voldemort gegenübertreten muss.
\end{kommentar}

\begin{kommentar}
Elber geht mit Sprout und Flitwick zu einem Raum im Schloss, dem Rosenhof. Wenn man weiß, dass er das Schloss erbaut hat, wirkt es lustig, dass er sagt, er habe auf einem Plan einen Raum gesehen, dessen Zugang durch einen Teppich verhängt wurde.
\end{kommentar}
