\chapter{Geburtstagsfeier und Numensobligat}


\enquote{Überraschung}, kam es Harry entgegen.

Dieser kam gerade von oben aus seinem Zimmer, wo er von Arabella hin gelockt wurde, da in der Küche die Vorbereitungen noch nicht ganz abgeschlossen waren. Er konnte gar nichts mehr sagen. Da standen Ron, Hermine und Ginny die ihn alle anlächelten. Hermine und Ginny stürmten auf den verdutzt drein schauenden Harry zu und umarmten ihn. Dann gaben ihm beide gleichzeitig einen Kuss auf die Backe und drückten ihn nochmals.

\enquote{Alles Gute zum Geburtstag}, sagten Hermine und Ginny; und Ron fügte ebenfalls ein: \enquote{Alles Gute zum Geburtstag}, hinzu.

Harry fing an zu grinsen, denn so einen Geburtstag hatte er bisher noch nie erlebt. Seine besten Freunde Ron und Hermine waren da und auch Ginny, von der er wusste sie wollte unbedingt dabei sein, denn sie hatte immer noch dieses gewisse Leuchten in ihren Augen, wenn sie ihn sah. Harry würde sich für ihren Geburtstag etwas ganz Besonderes überlegen, dachte er. Als die Mädels ihn endlich wieder losließen, kam Ron auf ihn zu und umarmte ihn brüderlich. Harry drehte sich um und sah zu Arabella, die ihn anlächelte und sagte, \enquote{Alles Gute zum Geburtstag Harry.}

Er fühlte sich glücklich wie schon lange nicht mehr. Dann drehte er sich wieder zu seinen Freunden um. Hermine nahm seine beiden Hände und sagte zu ihm: \enquote{Wir haben noch eine Überraschung für dich.} Ginny, die jetzt hinter ihm stand, schlang von hinten ein Band um seine Augen, sodass er nichts mehr sehen konnte. Harry wusste nicht, was passierte, aber ließ es einfach geschehen. Hermine führte ihn nun an einer Hand zur Küche heraus und in das Wohnzimmer. Dort angekommen gab sie ihm zu verstehen, er solle sich doch setzen. Etwas unsicher und nach Halt suchend, ließ er sich schließlich in einen Sessel fallen. Er spürte einen Mund nahe seinem Ohr, konnte aber nicht sagen, ob es Ginny oder Hermine war. Seine Gedanken begannen zu kreisen und er bekam eine leichte Gänsehaut. Ginny und Hermine fingen leicht an zu kichern, denn er hatte nur ein kurzes Hemd an, und so konnte man seine Unterarme gut erkennen, die ebenfalls leichte Gänsehaut anzeigten. Plötzlich spürte er etwas, dass an seinen Ohren entlang fuhr. Es dauerte etwas bis er realisierte, dass Ginny und Hermine mit ihren Zungen seine Ohren umspielten. Er musste schlucken. Dann, so als hätten sie lange geübt, begannen sie jeweils ein Wort abwechselnd in seine Ohren zu sprechen.

\enquote{Das war nicht die Überraschung, Harry. Mach deine Augen auf.} Das Band wurde ihm vom Kopf gezogen und er erblickte den Wohnzimmertisch, auf dem vier Geschenke lagen. Harrys Augen weiteten sich.

\enquote{Für mich?}, fragte er erstaunt.

\enquote{Ja!}, antwortete Ginny.

Er hatte zwar schon von allen Geschenke erhalten, aber sie waren noch nie dabei gewesen, wie er sie ausgepackt hatte, denn Harry hatte seinen Geburtstag immer in den Ferien und so konnte er keinen einladen. Das hätten sein Onkel und seine Tante niemals zugelassen. Er gab Ginny und Hermine jeweils einen Kuss auf die Backe und meinte dann zu Ron: \enquote{Du bekommst aber keinen von mir.}

\enquote{Das macht nichts}, sagte Ginny, \enquote{ich mache das.} Als Ginny zu ihrem Bruder gehen wollte, hielt sie Hermine zurück und flüsterte ihr etwas ins Ohr. Sie begann zu kichern und Hermine ging zu Ron.

\enquote{Damit du nicht meinst, du würdest gar nichts bekommen} und gab ihm einen Kuss. Harry musste sich beherrschen, um nicht zu lachen, denn der Ausdruck auf Rons Gesicht als ihn Hermine küsste, blieb noch eine ganze Weile auf seinem Gesicht stehen. \gedanke{Bahnt sich zwischen den beiden was an?}, fragte sich Harry, aber er hatte Geschenke auszupacken und so lenkte er sich damit ab.

Zuerst packte er Hermines Geschenk aus. Er ahnte bereits, dass es irgendein Buch war. Zu seiner Überraschung fand er eine Schachtel darin. Er öffnete sie und staunte, als er etwas, das wie eine schwarze Scheibe aussah, betrachtete. Sie war etwas größer als eine Galleone.

\enquote{Was ist das?}, fragte er.

\enquote{Das ist ein magisches Tagebuch. Du kannst wichtige Termine eintragen und wirst automatisch daran erinnert. Funktioniert auf Sprachbasis}, sagte Hermine. \enquote{Lege die Medaille einfach auf das mitgelieferte Buch und der Text wird automatisch übernommen.} Harry zog das Schutzpapier beiseite und entdeckte in der Schachtel ein kleines Buch.

Harry zog beide Augenbrauen hoch, stand auf und bedankte sich bei Hermine mit einem kleinen Kuss auf ihre Wange nahe ihrem Mundwinkel. Er setzte sich wieder und hatte den Eindruck, als ob Hermine leicht errötete. Als Nächstes öffnete er Rons Päckchen. Von Ron bekam er silberne Erweiterungs-Medaillen für sein Tagebuch inklusive Beschreibung und bedankte sich. \gedanke{Die Textlänge ist wohl begrenzt!} Er wollte gerade das große weiche Päckchen öffnen, als Ginny ihm ihres in seine Hand drückte. Es war das kleinste Päckchen von allen. Harry hatte keine Ahnung was es sein konnte. Als er es öffnete versteinerte sein Gesicht. Er war richtig sprachlos. Seine Augen begannen zu leuchten als er es berührte. Seine Hand umschloss sanft seine Konturen. Es war ein kleiner Anhänger in Form eines Basilisken. Er hatte eine schwarze Oberfläche und die Augen glitzerten grün wie Smaragde. Er zog sich den Anhänger um und lächelte Ginny an. Leicht unsicher wendete sie ihren Blick ab, doch Harry öffnete seine Arme und ging auf Ginny zu. Er zog sie an sich und flüsterte in ihr Ohr.

\enquote{Danke Ginny. Ich liebe dich \gst wie eine Schwester, die ich nie hatte} und küsste sie auf den Mund. Ron verschlug es die Sprache und Ginny begann rot zu werden. Als eine einzelne Träne begann ihr Gesicht herunterzulaufen, fing Harry an \enquote{Tut mir leid Ginny, ich hätte dich nicht küssen sollen.}

\enquote{Nein, das ist es nicht Harry}, antwortete sie. \enquote{Es ist das, was du mir gesagt hast. Der Teil mit der Schwester die du nie hattest. Bei dir fühle ich mich richtig geborgen, Harry.} Sie drehte sich zu Ron um und sagte: \enquote{Nicht, dass ich mich bei dir nicht geborgen fühle Ron, aber Harry ist wie ein Bruder zu mir, der nicht immer und überall um seine kleinere Schwester besorgt ist.}

Ron senkte leicht seinen Blick, begann aber nach kurzer Zeit zu grinsen. \enquote{Du weißt ja, wie große Brüder manchmal sind. Vielleicht habe ich es auch nur übertrieben.}

Alle fingen an zu lachen. Harry setzte sich wieder in seinen Sessel und betrachtete seine Geschenke. Da fiel ihm das vierte auf, welches noch immer auf dem Tisch lag. Harry wunderte sich, von wem das denn sein könnte. Er nahm es und fand keine Karte, die auf den Absender schließen ließ.

\enquote{Das ist von mir}, hörte Harry von hinten. Er drehte sich um und sah Arabella.

\enquote{Von dir? Ja aber\abs}

\enquote{Mach es einfach auf}, sagte sie.

Harry öffnete das Päckchen. Besser gesagt er riss es auf und fand ein T-Shirt, eine Hose und ein Paar Schuhe.

\enquote{Nachdem du gerne joggen gehst, dachte ich mir, du könntest das brauchen. Sie sind bereits gewaschen.}

\enquote{Danke Arabella}, sagte Harry, stand auf und umarmte auch sie. \enquote{Gleich morgen früh probiere ich die Sachen aus.}

\enquote{So, jetzt wird erst einmal gefrühstückt}, sagte Arabella und verschwand in die Küche.

Die anderen folgten ihr und Hermine und Ginny begannen den Tisch zu decken, während Ron Harry zu seinem Stuhl delegierte und meinte: \enquote{Du hast heute Geburtstag, wir machen das.}

\enquote{Wie kommt ihr hierher? Woher wisst ihr, dass ich nicht bei meinem Onkel und meiner Tante bin?}

Ron antwortete: \enquote{Wir haben Eulen von Miss Figg bekommen. Sie bat uns heute herzukommen. Wir sind schon sehr früh da gewesen und halfen ihr beim Frühstück machen. Sie hatte Hedwig geschickt.}

Harry staunte nur und bekam seinen Teller serviert. Genüsslich biss er in seinen Schinken und sein Toastbrot, verschlang seine gebackenen Bohnen und blickte zu Ginny.

\enquote{Ist es wegen der Kammer?}, fragte Harry und deutete auf sein Amulett um den Hals.

\enquote{Ja}, antwortete sie und senkte ihren Blick. \enquote{Als ich es in einem Schaufenster sah, wusste ich einfach was ich zu tun hatte. Ich verspürte diesen Drang es dir zu schenken.} Harry lächelte und bedankte sich nochmals. Er wusste, dass es immer noch an ihr nagte.

\enquote{Ich werde es immer bei mir tragen.}

Ginny schaute ihm wieder in die Augen. Sie begannen zu leuchten.

\enquote{Du hast wunderbare Augen Ginny}, sagte Harry. Sie errötete wieder.

\enquote{Harry}, hörte er plötzlich Hermine sagen. \enquote{Wenn du so weitermachst, wird man in Hogwarts denken, sie hätte einen Sonnenbrand.} Jetzt begannen alle zu lachen. Selbst Ginny, die ihre Röte wieder verlor.

Harry fing an zu grinsen und sagte dann: \enquote{Vielleicht will ich ja, dass sie errötet.}

Ron gab ihm einen Knuff in seine rechte Seite und sagte dann: \enquote{Wir reden hier immerhin über meine Schwester.}

Wieder brach Gelächter aus.

Nach dem Frühstück führte Arabella dann alle in ihr Wohnzimmer. Sie deutete ihnen an, kurz zu warten und verschwand aus dem Zimmer. Nach ein paar Minuten kam sie wieder herein und hatte ein kleines Brett in der Hand, das etwa so groß wie der Wohnzimmertisch war. Sie stellt es auf dem Tisch ab, und meinte zu Harry, er solle doch mit seinem Zauberstab darauf tippen. Er zog ihn heraus und tippt auf das Spielfeld. Plötzlich fingen an drei Stangen auf den kurzen Seiten herauszuwachsen. Oben an den Stangen begannen sich Ringe auszubilden, und etwas, das wie Zuschauertribünen aussah, fing an ebenfalls aus dem Brett heraus zu wachsen. Als sich alles wieder beruhigt hatte, sah man ein Quidditch-Feld in Miniaturausgabe auf dem Wohnzimmertisch stehen.

\enquote{Wahnsinn, so etwas habe ich noch nie gesehen}, sagten alle fast gleichzeitig.

Arabella setzte sich und meinte: \enquote{Diese Art von Spielzeug wird auch nicht mehr hergestellt. Er ist viele Jahrzehnte alt. Schon mein Großvater hatte damit gespielt. Jeder von euch sucht sich eine Mannschaft und eine Position aus und sagt diese laut und deutlich. Dann könnt ihr mit euren Gedanken die Figuren steuern. Ich habe früher gerne damit gespielt. Nur alleine macht das keinen Spaß. Zwar werden die anderen Figuren von selbst ihren Weg finden, aber wenn man keinen hat, gegen den man spielen kann, verliert das Ganze seinen Reiz.}

Harry fand Gefallen daran. \enquote{Welche Mannschaften \abs}, aber dann entdeckte er die Schriftzüge auf seiner Seite und führte seinen Satz nicht fort. Er verstand und sagte dem Spielfeld laut und deutlich.

\enquote{MacCornahiew, Hüter.} Eine Figur, die Harry sehr ähnlich sah und die Robe der MacCornahiew-Mannschaft trug, erschien auf dem Spielfeld. \gedanke{Mal sehen, ob ich als Hüter auch gut bin}, dachte Harry. Ginny nahm sich die Position als Sucherin in Harrys Mannschaft und Ron und Hermine spielten in der anderen Mannschaft, der \accentuate{Panther von Loch Lumen}. Hermine nahm sich die Position des Treibers und Ron wählte ebenfalls die Position des Hüters aus.

Arabella Figg spielte den Schiedsrichter und sagte dann: \enquote{Spiel beginnen.}

Die verschiedenen Bälle und restlichen Figuren kamen zum Vorschein. Der Schnatz wurde losgelassen und die Klatscher fingen an sich in die Lüfte zu erheben, um auf den Moment des ersten Ballkontaktes zu warten. Der Quaffel schnellte hoch und die Jäger versuchten ihn zu erreichen. Das war für die Klatscher das Startsignal und sie schossen auf dem Spielfeld umher, wurden von einem Treiber zum andern geschlagen die versuchten die Spieler der gegnerischen Mannschaft zu treffen und von ihren Besen zu werfen.

Als Harrys Spielfigur, die von einem Klatscher getroffen wurde, abstürzte und alle ihr hinterherschauten, stand diese vom Boden wieder auf, wischte sich den Staub von ihrer Robe, drehte sich zu Harry um und begann ihn zu schimpfen. \enquote{Was fällt dir ein, mich einfach so fallen zu lassen. Den hättest du doch kommen sehen sollen. Du bist schließlich größer als ich.} Merklich empört, stieg die kleine Figur wieder auf ihren Besen und begann ihren Platz vor den Ringen einzunehmen.

Sie verbrachten den ganzen Vormittag und den halben Nachmittag damit, Mini-Quidditch zu spielen; nur durch das Mittagessen unterbrochen. Harrys Mannschaft verlor vier zu zwei. \enquote{Das nächste Mal spiele ich den Sucher}, bekräftigte Harry. Alle lachten.

Arabella sagte: \enquote{Spiel beendet}, und die Figuren verschwanden vom Spielfeld. Die Tribünen und die Ringtore verschwanden und es blieb nur noch das Brett übrig. \enquote{Ihr könnt Morgen weiterspielen, wir müssen noch das Abendessen herrichten.}

\enquote{Ihr bleibt bis Morgen?}, fragte Harry.

\enquote{Ja, kurz nach dem Mittagessen gehen wir wieder}, sprach Hermine und verschwand mit Arabella in der Küche.

Harry setzte sich auf das Sofa neben Ginny, lehnte sich zurück und ließ seine Gedanken schweifen. Ron verließ ebenfalls den Raum und ließ Harry mit seiner Schwester alleine.

Nach einigen Minuten sagte Harry: \enquote{Tut mir leid, wenn ich dich heute Morgen mit dem Kuss überrascht habe, aber ich war so glücklich über dein Geschenk. Es überkam mich einfach.}

\enquote{Das macht nichts, Harry. Ich war zwar zuerst überrascht, aber es war nicht unangenehm.}

Harry lächelte und fragte dann: \enquote{Sollen wir es wiederholen?}

Ginny drehte sich zu ihm um und hob eine Augenbraue. Während sie ihn kitzelte, meinte sie: \enquote{So war das auch nicht gemeint.} Harry fing an zu lachen. Nicht weil Ginny das gesagt hatte, sondern weil er  kitzlig war. Ginny schien das zu gefallen und machte weiter. Harry versuchte sich zu wehren, und an Ginny heranzukommen, um sie ebenfalls zu kitzeln. Ein paar Mal schaffte er es sogar, sie von ihm abzubringen. Aber Ginny war sehr geschickt darin seinen Attacken auszuweichen. Als sich beide eng ineinander umschlungen auf dem Boden wälzten, kam Hermine rein, die das Lachen gehört hatte und einfach mal nachschauen wollte. Zuerst blieb sie schockiert im Türrahmen stehen, als sie die beiden so sah. Harry lag am Boden und Ginny saß auf ihm. Hermine fing an ihre Hand vor ihren Mund zu halten, als Ginny zu Harry sagte. \enquote{Gibst du nun auf} und sie stupste wieder in paar Mal in seine Seite, \enquote{oder soll ich weiter machen?}

\enquote{Ich geb’ auf, ich geb’ auf}, konnte sich Harry mit Müh und Not sagen hören. Nachdem er sich beruhigt hatte, strich er Ginny über ihre Wange und sagte leise zu ihr: \enquote{Ich hab dich wirklich sehr gerne Ginny. Du bist mehr als nur eine Schwester für mich. Ich kann nur nicht sagen was genau.}

Ginny lächelte ihn an, stieg von Harry herab und drehte sich um, um in die Küche zu gehen, als sie Hermine sah.

\enquote{Wie lange stehst du schon da?}, fragte sie erschrocken. Harry richtete sich auf und lachte immer noch.

\enquote{Lange genug, um festzustellen, dass ihr nicht das gemacht habe, was ich zuerst dachte, als ich hereingekommen bin}, sagte Hermine.

Hermine drehte sich lächelnd um und Ginny folgte ihr in die Küche. Harry konnte Hermine nur noch sagen hören: \enquote{Jetzt weiß ich ja, was ich mit Harry machen muss, wenn er wieder nicht das tut, was ich von ihm will.}

Noch leicht außer Atem stand Harry auf und ging ebenfalls in die Küche. \gedanke{Das ist der schönste Geburtstag, den ich je hatte}, dachte er.

Nach dem Abendessen und einer langen Unterhaltung mit seinen Freunden ging Harry früh schlafen.

Er lag im Bett und konnte nicht einschlafen. Die Aufregung war immer noch da. Er umfasste Ginnys Amulett und schloss seine Augen.

\begin{traum}
Er war wieder im Raum der Wünsche, als er letztes Schuljahr seine Mitschüler in Verteidigung gegen die dunklen Künste unterrichtete. Er stand hinter Ginny und sah ihr dabei zu, wie sie ihren ersten gestaltlichen Patronus herbei schwor. Dieser hatte die Form eines Pferdes. Harry lief weiter, an Neville vorbei und ermutigte ihn, es gleich wieder zu versuchen, nachdem sein Versuch missglückt war. Er sprach der ganzen Gruppe Mut zu und sagte ihnen, dass ein gestaltlicher Patronus schwierig sei, aber auch abstrakte Formen gut seien. Er lief an Ron vorbei, dessen Hund-Patronus Neville umwarf. Dann sah er wie Hermines Otter-Patronus umherflog und sie dabei leicht kitzelte. Er ging weiter zu Luna und sagte, sie möge es versuchen. Sie erzeugte einen kleinen Hasen, der in der Luft umhersprang. Sie drehte sich zu ihm um und lächelte ihn an. Harry lächelte zurück.
\end{traum}

Dann verschwamm das Bild und Harry sah nur das Innere seiner Augenlider. Er ließ das Amulett wieder los, drehte sich um und schlief ein.

Am nächsten Morgen hatte Harry seine neuen Jogging-Klamotten an und ging in die Küche, um wie jeden Tag ein Glas Wasser zu trinken, bevor er loslief. Als er die Küche betrat, saß Ginny da.

\enquote{Darf ich dich begleiten?}, fragte sie.

Harry konnte ihr diesen Wunsch nicht ausschlagen, nicht nachdem sie bereits selbst in Jogging-Sachen da saß.

\enquote{Gerne Ginny.} Er trank sein Glas leer und die beiden machten sich auf.

Sie verließen das Haus und Harry zeigte Ginny, wie sie sich aufzuwärmen habe. Er hielt ihre Beine während sie Sit-ups machte und danach hielt sie seine. Dann begannen sie zu joggen. Rund um den Block, danach die Straße hoch und wieder zurück. Auf dem Rückweg liefen sie am Haus seines Onkels und seiner Tante vorbei. Harry blieb stehen und sagte ihr, dass er hier normalerweise wohnte.

\enquote{Dies ist das Haus meines Onkels und meiner Tante. Schrecklich. Sie mögen mich nicht besonders.}

Er lief wieder los und Ginny folgte ihm.

\enquote{Mein Onkel ist ganz schlimm. Er hat eine Art Eulen-Phobie. Immer wenn eine Eule Post bringt, ist er ganz aufgeregt. Er hasst alles, was mit mir oder der Zauberei zu tun hat.}

Ginny schaute ihn mit großen Augen an.

\enquote{Sind sie wirklich so schlimm?}

\enquote{Du kannst ja mal vorbeikommen und sagen, du kennst mich aus der Schule. Und dann wartest du ab wie sie reagieren.}

Ginny schüttelte den Kopf und meinte.

\enquote{Lieber nicht.}

Harry lächelte sie an und Ginny lächelte zurück. Obwohl sie nicht in ihn verliebt war, so dachte er zumindest, hatte sie immer dieses Leuchten in ihren Augen, wenn sie ihn ansah. Sie gingen wieder in Arabellas Haus und Harry ließ Ginny den Vortritt in der Dusche. Nachdem beide mit Duschen fertig waren, gingen sie in die Küche, wo schon Ron, Hermine und Arabella beim Frühstücken waren.

\enquote{Wo wart ihr?}, wollte Hermine wissen.

\enquote{Draußen. Joggen}, sagte Ginny. \enquote{War angenehm. Daran könnte ich mich gewöhnen.}

Sie lächelte Harry zu. \enquote{Machst du in der Schule weiter? Nimmst du mich wieder mit?}

\enquote{Ja}, sagte Harry, \enquote{ich mache weiter und du kannst gerne mitkommen.}

Ginny schien glücklich zu sein.

Nach einer weiteren Runde Mini-Quidditch, in der Harry nun Sucher spielte und die anderen auch eine andere Position einnahmen, kam auch schon die Zeit für das Mittagessen. Dieses Mal gewannen Harry und Ginny. \gedanke{Als Sucher bin ich wohl doch besser, im Gegensatz zur Position des Hüters.} Arabella hatte wie immer ausgezeichnet gekocht und Harry verabschiedete sich von seinen Freunden nach dem Mittagessen.

\enquote{Wir sehen uns dann im Zug}, rief er ihnen hinterher, als sie in einen Wagen einstiegen, der verdächtig wie einer der Ministeriumswagen aussah, die Rons Vater sonst organisiert hatte.

Die nächste Woche verlief relativ ruhig. Er joggte jeden Morgen und half Arabella bei ihrer Hausarbeit. Er nahm ihr durch seine Zauberei viel Arbeit ab und hinterließ ihr ein paar nützliche Gegenstände, die ihr das Arbeiten erleichterten. Mitten in der Woche kam eine Postkarte, die besagte, dass am Montag die Dursleys wieder ankommen würden. Harry packte also am Sonntag seinen Koffer und brachte ihn zurück in sein Zimmer im Haus seines Onkels und seiner Tante. Er wollte nicht, dass sie erfuhren, dass er alles bei Arabella dabei gehabt hatte.

Wieder zurück gab Harry Arabella den Wohnungsschlüssel, den sie an sich nahm und in ihrer Tasche verstaute.

\enquote{Ich nehme an, dass ich dich wieder sieze, wenn mein Onkel mich abholt.}

\enquote{Ja Harry, das wäre passend. Wenn wir uns ihnen gegenüber zu gut verstehen, kannst du nächstes Jahr nicht wieder kommen, falls sie mal wieder in den Urlaub wollen.}

Harry nickte und lächelte. Ihm hatten die drei Wochen bei Arabella gefallen. Und für die nächsten drei Wochen war sie wieder Miss Figg. Als Onkel Vernon ihn am nächsten Tag abholte, saß Harry in der Küche. Es klingelte und Miss Figg öffnete die Haustüre. An der Stimme konnte Harry Onkel Vernon erkennen, der sich mit Miss Figg zu unterhalten schien. Sie sagte ihm, dass er ab und zu mal etwas kaputt gemacht hatte, aber die Sachen eigentlich eh schon alt gewesen waren. Im Großen und Ganzen sei er anständig gewesen. Onkel Vernon bedankte sich bei Miss Figg.

\enquote{Harry, komm her, dein Onkel holt dich wieder ab}, herrschte sie ihn an. Harry verkniff sich ein Grinsen und kam aus der Küche in den Flur. Er lief auf Onkel Vernon zu und hatte dabei leicht seinen Kopf gesenkt.

\enquote{Komm mit}, raunte ihm sein Onkel zu. Harry ging hinterher und musste die nächsten drei Wochen bei seinem Onkel und seiner Tante verbringen.

Auf dem nach-Hause-Weg ließ er den vorigen Abend noch einmal Revue passieren. Gestern Abend hatte er mit einem Zauber in einer Ecke des Gartens ein paar Petunien gepflanzt und dafür gesorgt, dass sie zur richtigen Zeit blühen würden.

\begin{rueckblick}
\enquote{Professor Sprout? Haben sie einen Moment Zeit?}

\enquote{Aber sicher doch Mister Potter. Was gibt es?}

\enquote{Ich hätte gerne gewusst, ob es einen Zauber gibt, mit dem man Blumen pflanzen kann. Mit dem man dafür sorgen kann, dass Blumen zu einem bestimmten Zeitpunkt blühen.}

\enquote{Ja, so etwas gibt es. Wieso?}

\enquote{Ich würde diese Zauber gerne lernen.}

\enquote{Erhoffen sie sich davon mehr Erfolg bei den Mädchen?}

\enquote{Auch. Aber vor allem möchte ich etwas für meine Tante und meine Mutter tun. Ich dachte da an Lilien und Petunien.}

\enquote{Oh, das sind schöne Blumen.}

\enquote{Werden sie mir helfen?}

\enquote{Ja, kommen sie mit, Mister Potter. Die notwendigen Zauber sind nicht besonders schwer. Sie brauchen nur etwas Vorbereitung.}
\end{rueckblick}

Kaum war er wieder zu Hause, flog eine Eule durch das offene Küchenfenster. Sie warf einen hellen, oliv-grünen Umschlag auf den Boden und verschwand, ohne etwas zu erwarten. Sein Onkel war gerade damit beschäftigt aus dem Wohnzimmerfenster hinauszusehen und sich wieder über die Nachbarin mit ihrem Hund aufzuregen.

\enquote{Diese aufgetakelte Schnepfe. Ständig läuft sie mit ihrer Promenadenmischung vor unserer Einfahrt herum und lässt diesen Köter an unseren Zaun\abs}

\enquote{Vernon!}, ermahnte ihn seine Frau.

\enquote{Ja, Petunia, du hast recht.}

Harry sah auf den Umschlag, der an ihn adressiert war, und drehte ihn um. Der Absender war ein Notar.

\begin{brief}
Plaustein \& Söhne

Notariat
\end{brief}

Harry wunderte sich, was ein Notar der Zauberergemeinschaft von ihm wollte. Er öffnete den Umschlag und begann, nachdem er sich auf einen Küchenstuhl gesetzt hatte, den Brief zu lesen.

\begin{brief}
Sehr geehrter Mister Harry James Potter,

als Nachlassverwalter und Familiennotar der Familie Black, setzen wir sie gemäß den Bestimmungen der Zaubereibehörde und den Wünschen des verstorbenen Sirius Black in Kenntnis. Bitte treffen Sie pünktlich am kommenden Donnerstag in unserem Notariat ein.

Wir wurden gebeten, sie zur Eröffnung des Numensobligats einzuladen. Des Weiteren sind noch folgende Personen geladen: Albus Percival Brian Wulfric Dumbledore, Remus John Lupin, Nymphodora Tonks.

Bitte teilen Sie uns mit, falls sie nicht erscheinen können, oder eine Person benennen, die sie vertreten soll.
\signumspace
Hochachtungsvoll

Nymphodora Plaustein
\end{brief}

Harry ließ seine Hände sinken. \gedanke{Eröffnung eines Numensobligats}, ging ihm durch den Kopf. Ihm wurde wieder schwer ums Herz. Tränen begannen sich in seinen Augen zu bilden. Er faltete den Brief zusammen und steckte ihn in seine Hosentasche. Danach wischte er sich mit seinem Ärmel über die Augen, um seine Tränen wegzuwischen. \gedanke{Dumbledore, Remus und Tonks}, ging ihm durch den Kopf. \gedanke{Sie sind auch geladen worden. \gst Eingeladen worden}, korrigierte er sich.

Traurig und die Momente im Ministerium immer wieder in seinem Geist durchspielend, saß er auf seinem Stuhl in der Küche und blickte ins Leere. Seine Verwandten ignorierten ihn, da sie andere Sachen zu tun hatten. Sie packten noch ihre Koffer aus.

Plötzlich klingelte es an der Haustür und Harry stand auf, um sie zu öffnen. Er sah durch das Guckloch an der Haustür nach draußen. Der Mann sah aus wie Remus Lupin. Doch Harry war vorsichtig geworden. Seinen Zauberstab ziehend, stand er hinter der Tür.

\enquote{Wer da?}, fragte Harry hinter der Tür.

\enquote{Remus John Lupin. Freund und ehemaliger Hogwarts-Lehrer.}

Harry öffnete die Tür und hielt seinen Zauberstab auf Lupin.

\enquote{Du bist vorsichtig geworden}, sagte Remus.

\enquote{Bist du, du?}, fragte Harry bewusst.

\enquote{Nein, ich bin jemand anderes. Ich bin unser Freund Voldi.}

Harry bat ihn herein, hielt seinen Zauberstab aber immer noch auf Remus. Er hatte schon eine Weile darüber nachgedacht, wie er herausfinden konnte, ob Remus wirklich der Richtige ist. Plötzlich sah er Bildfetzen vor sich. Sie schienen aus Remus herauszukommen und durch den Raum zu schweben. Er sah den Biss, der Remus zum Werwolf machte.

\enquote{Wie genau bist du zum Werwolf geworden?}, fragte Harry. \enquote{Du bist von einem Werwolf ins Genick gebissen worden, richtig?}

\enquote{Woher weißt du das? Das habe ich dir nie erzählt.}

\enquote{Erzähl weiter. Was ist danach passiert}, forderte er.

\enquote{Na ja, ich habe geblutet und den Werwolf davonlaufen sehen. Meine Mutter hat mich damals gefunden. Sie hat mir auch geholfen, damit klarzukommen.}

\enquote{Was hatte sie an?}, fragte Harry.

\enquote{Warum willst du das wissen? Was soll das?}

\enquote{Was hatte sie an?}, forderte Harry mit Nachdruck.

\enquote{Ein gelbes Kleid}, antwortete Remus.

\enquote{Mit Blümchenmuster}, nickte Harry.

\enquote{Nein, es war nicht gemustert.}

\enquote{Ok, Remus.} Harry senkte seinen Zauberstab und bat ihn in die Küche.

\enquote{Wie kannst du davon wissen?}, fragte er Harry.

\enquote{Weiß nicht genau. Ich habe Bildfetzen gesehen, die aus dir herauskamen. Ich habe mir nämlich die Frage gestellt, wie ich am besten überprüfen kann, ob du auch du selber bist. Da ist das dann passiert.}  Dann wurde Harrys Gesichtsfeld für ein paar Sekunden eingeschränkt. Er sah schwarze Ränder und sein Gesicht erblasste etwas.

\enquote{Alles in Ordnung?}, fragte Remus.

Als sich Harry wieder beruhigt hatte, sagte er: \enquote{Geht schon wieder. War nur kurz benommen. Aber weswegen bist du hier? Geht es um diesen Numensobligat?}

\enquote{Ja, Dumbledore bat mich, dich übermorgen abzuholen. Ich bin deshalb vorher vorbeigekommen, damit du Bescheid weißt.}

Harry nickte. \enquote{Willst du was zu trinken?}

\enquote{Gerne.}

Harry schenkte ihm ein halbes Glas Wasser ein und füllte mit Saft auf.

\enquote{Ich werde dann übermorgen vorbeikommen und dich mitnehmen. Wir müssen allerdings vom Haus weg. Sonst können wir nicht apparieren.}

\enquote{Ich weiß, zwei Kilometer.}

\enquote{Du weißt wie groß der Bannkreis ist?}

\enquote{Warum nicht? Man muss seine Grenzen kennen!}, antwortete Harry mit einem etwas süffisanten Lächeln.

Remus schüttelte nur ungläubig den Kopf. Er wunderte sich nicht mehr über Harrys Möglichkeiten und Wege, etwas in Erfahrung zu bringen.

\enquote{Was ist denn ein Numensobligat?}

\enquote{So etwas wie ein Testament}, antwortete Remus, bevor er wieder ging.

Zwei Tage später kam Remus vorbei, um Harry abzuholen. Harry hatte in der Zwischenzeit seinem Onkel und seiner Tante erklärt, dass er einen wichtigen Amtstermin habe. Z-Sachen erklärte er ihnen, da Onkel Vernon immer einen Anfall bekam, wenn er zaubern sagte. Die kleine Unterhaltung zwischen Vernon, Petunia und Remus, die stattfand, als Remus Harry abholen wollte, war entsprechend knapp, aber dennoch höflich gewesen.

\enquote{Und Harry, wie sind die Ferien so?}, wollte Remus wissen, als sie zu Fuß unterwegs waren.

\enquote{Na ja, ging so. Als ich bei unserer Nachbarin untergebracht war, während meine Verwandtschaft Urlaub gemacht hat, hatte ich eine Menge Spaß. Ich habe mich mit ihr gut verstanden.}

Dann wurde es für eine Weile still. Harry genoss es einfach mal nur so neben Remus herzulaufen. Einfach nur Gesellschaft zu haben, ohne ständig reden und sich erklären zu müssen. Dann waren sie außerhalb der Apparitionsgrenze. Remus nahm Harry an der Hand und zog ihn mit sich. Beide apparierten in einer ruhigen Ecke. Remus zog, für Harry unbemerkt, eine Tüte aus seiner Tasche heraus. Harry würgte einmal kurz, musste aber nicht brechen. Etwas blass, sah er Remus an.

\enquote{Beeindruckend Harry. Die meisten müssen sich beim ersten Mal übergeben. Oder auch noch nach ein paar Mal danach.}

\enquote{Komisch, warum nur?}, fragt Harry sarkastisch nach. Mit verzogenen Gesicht schaute er Remus an und fragte schließlich: \enquote{Wohin?}

\enquote{Komm mit.}

Sie liefen ein paar Meter die Gasse entlang, als Harry entdeckte, wo sie waren. \enquote{Nokturngasse?}, fragte er.

\enquote{Ja. Nokturngasse, Ecke Winkelgasse. Hier entlang}, sagte Remus, als er sich sorgsam umsah und Harry dann hinter sich her zog.

Nach wenigen Metern standen sie vor der Tür. Daneben war ein grün angelaufenes Kupferschild mit glänzender, gravierter Schrift angebracht. Dort stand: \accentuate{Plaustein \& Söhne}

Harry trat vor Remus durch die Tür in das Treppenhaus und folgte den Hinweisschildern in den ersten Stock. Er trat durch die Glastür und auf die Dame am Empfang zu.

\enquote{Die Herren Potter und Lupin}, begann er.

\enquote{Dritte Tür Links. Sie sind bei der Chefin persönlich}, sagte die Dame, ohne aufzusehen.

Harry sah zu Remus und zog eine Augenbraue hoch. Sie gingen vor die Tür, welche die Dame genannt hatte, und klopften. Nach einigen Sekunden kam ein \enquote{Herein!}, durch die Tür.

Harry öffnete sie und er und Remus traten ein.

\enquote{Ah Mister Potter, Mister Lupin. Sie sind etwas zu früh.}

Harry sah auf seine Uhr. Fünf Minuten zu früh, fand er. \enquote{Fünf Minuten}, sagte er.

\enquote{Ich vergaß}, entschuldigte sich die Notarin. \enquote{Muggel nehmen es mit der Pünktlichkeit nicht so genau.}

Harry zog beide Augenbrauen hoch.

\enquote{Hexen und Zauber ziehen es vor höchstens eine Minute zu früh, oder zu spät zu kommen. In manchen Kreisen ist sogar eine Differenz von zehn Sekunden unhöflich.}

\enquote{Verstehe, sollen wir so lange draußen warten?}

\enquote{Nein, nein, bitte \gst setzen Sie sich}, bot sie den beiden einen Sessel an. \enquote{Mein Name ist Nymphodora Plaustein, Inhaberin dieses Notariats mit angeschlossener Kanzlei.}

Harry und Remus nahmen dankend an und setzten sich.

\enquote{Möchten Sie einen Tee? Kürbissaft? Gebäck?}

\enquote{Einen Tee, danke}, sagte Harry.

\enquote{Für mich nichts}, antwortete Remus.

Nachdem der Tee für Harry auf dem Tischchen neben seinem Sessel stand, kamen auch schon Dumbledore und Tonks herein. Die Begrüßung war kurz, aber herzlich. Nachdem alle saßen, begann die Notarin die Testamentseröffnung.

\enquote{Vielen Dank, dass Sie alle hier sind. Ich habe sie eingeladen, weil sie im Numensobligat erwähnt wurden\abs}

\enquote{Verzeihung Ma’am, aber was ist ein Numensobligat?}, fragte Harry noch einmal sicherheitshalber nach.

\enquote{Wie? Sie wissen nicht was\abs? Oh natürlich, Verzeihung. Ich hätte mich anders ausdrücken müssen bei Ihnen. Das ist ein Testament. Wir sind hier um den letzten Willen Sirius Blacks zu verlesen.}

\enquote{Oh, danke, ich ahnte schon etwas in der Richtung, wollte aber Gewissheit haben}, sagte Harry.

Die Notarin begann: \enquote{Numensobligat \gst Verzeihung \gst Testament von Sirius Black.} Es folgte eine kurze Pause. \enquote{Ich, Sirius Black, durch die Magie im Vollbesitz meiner geistigen Fähigkeiten bestätigt und durch den Zauber bestätigt, verfüge als meinen letzten Willen folgende Punkte. Zu meiner Testamentseröffnung (diesen Begriff habe ich wegen Harry gewählt) möchte ich folgende Personen hier haben: Albus Percival Wulfric Brian Dumbledore, Remus John Lupin, Nymphodora Tonks (verzeih mir Tonks) und Harry James Potter.} Erneut pausierte sie kurz. \enquote{Ich will nicht viele Worte verlieren, also nur das wichtigste. Jeder von euch bekommt etwas. Nymphodora \gst wenigstens jetzt kannst du mir nichts mehr anhaben \gst dir gebe ich ein besonderes Familienschmuckstück. Es schützt dich vor vielen Zaubern. Du wirst es schon noch herausfinden.}

Mrs Plaustein öffnete ihre Schublade, nahm ein kleines hölzernes Kästchen heraus und übergab es Tonks.

\enquote{Remus, mein treuer Freund. Dir vermache ich ebenfalls ein besonderes Schmuckstück aus meiner Familie.} Sie nahm ein weiteres Kästchen aus ihrer Schublade und überreichte es Remus.

Dann las sie weiter: \enquote{Bitte, meine Freunde. Öffnet es.}

Tonks und Remus öffneten ihre Kästchen und staunten über die Ketten, die darin waren. Sie waren leicht rosafarben. Ihr Wert war nicht genau zu erkennen, aber beide schätzten, dass Kupfer drinnen sein musste, oder die Ketten Rotgold enthielten.

\enquote{Diese Ketten}, fuhr Mrs Plaustein fort, \enquote{enthalten kein Kupfer oder Rotgold, wie ihr vielleicht annehmen werdet. Diese Ketten \gst und bitte behaltet sie \gst}

Dann sagte sie: \enquote{Fassen sie die Ketten bitte an}, forderte die Notarin.

Remus und Tonks taten dies, worauf die Ketten kurz zu Leuchten begannen.

\enquote{\gst sind aus einer Rhodiumlegierung und enthalten fast hundert Prozent dieses Metalls. Es ist ein wunderbarer Katalysator und wird euch auf eurem Weg gute Dienste leisten. Den wahren Wert, der über dem materiellen liegt, wird euch erst später bewusst werden.}

Remus und Tonks wollten die Ketten schon ablehnen, da hoben sie aus ihren Kästchen ab und legten sich um ihre neuen Träger. Das Wissen, das die Ketten mitbrachten, sickerte in die Köpfe der beiden, sodass sie ihren Widerstand aufgaben.

\enquote{Albus, dir gebe ich etwas, das du schon immer haben wolltest. Meine Sammlung an alten Heften, meine Schoko"-frosch-Samm"-lung und ein besonderes Rätsel.}

Mrs Plaustein übergab Dumbledore einen Würfel, der in sämtlichen Richtungen dreimal unterteilt schien. Harry erkannte ihn als Rubik-Würfel.

\enquote{Dieser Würfel wird dir sicher viel Spaß machen. Die anderen Sachen werden dir nach Hause geliefert.}

Dumbledore drehte begeistert den Würfel und besah ihn sich von allen Seiten.

\enquote{Zu guter Letzt, Harry, mein Patensohn. Dir vermache ich den ganzen Rest. Mein Verlies in Gringotts, mein Haus samt Kreacher. Leider liegt ein Zauber auf dem Haus, der verhindert, dass zu viele Leute die Erste Zeit sich im Hause aufhalten. Unsere Kartengruppe muss sich also die nächste Zeit einen anderen Ort suchen.}

Mrs Plaustein legte das Testament auf ihren Schreibtisch und überreicht den Schlüsselbund mit den beiden Schlüsseln an Harry. Zusätzlich gab sie ihm noch einen Brief in die Hand.

Harry öffnete ihn und las:

\begin{brief}
Lieber Harry,

bitte nimm dein Erbe an, sonst fällt Kreacher meiner Cousine zu und er wird ihr freudestrahlend alles erzählen, was wir in mühevoller Kleinarbeit alles erreicht haben.
\end{brief}

Dann löste sich der Brief auf. Harry verstand. Obwohl er noch nicht genau wusste, wie, nahm er sich vor, Kreacher zu resozialisieren und in die Gemeinschaft zurückzuführen. Er musste vom Zwang seiner alten Familie gelöst werden.

\enquote{Damit wäre der Numensobligat beendet. Ich danke für ihr Erscheinen.}

Die vier standen auf, gaben der Notarin die Hand und verließen ihr Büro.

\enquote{Ich bringe Harry dann heim, Remus.}

Remus nickte und verschwand mit Tonks, nachdem sie aus dem Gebäude getreten waren. Harry und Dumbledore liefen noch eine Weile nebeneinander in der Winkelgasse her.

\enquote{Hier war ich schon lange nicht mehr}, sagte Dumbledore plötzlich.

\enquote{Haben Sie auch bemerkte, dass zwischen Tonks und Lupin etwas läuft?}, fragte Harry.

\enquote{Du hast das bemerkt?}

\enquote{Ja, war nicht zu übersehen. Die Ketten passten zueinander und leuchteten kurz, als sie sie umlegten \gst Äh ja. Es zogen sich feine Linien zueinander hin. Es würde mich nicht wundern, wenn diese Ketten ihnen zu einer engeren Bindung und Beziehung verhelfen würden.}

\enquote{Wie kommst du darauf?}, fragte Dumbledore.

\enquote{Weiß nicht genau. Ich habe so ein Gefühl.}

Dumbledore streckte Harry einen Arm hin, worauf dieser ihn ergriff und beide disapparierten. Dieses Mal ohne Gesichtsverfärbung aber mit noch leichtem Aufstoßen. \enquote{Du scheinst das Apparieren recht gut zu vertragen.}

\enquote{Na ja, es gibt schlimmeres. Aber es ist schon recht unangenehm. Beim ersten Mal hätte ich mich fast übergeben.}

Dann liefen sie still nebeneinander her und genossen die Stille. Es war schön, mit Dumbledore einfach zu spazieren.

Kurz vor der Tür fragte Harry: \enquote{Wie sieht es mit einem Lehrer für \VgddK aus?}

\enquote{Na ja, ich bin dran. Ich habe jemand; fast. Er ist gut geeignet. Scheinbar ist er nicht sonderlich davon angetan, dass ihr letztes Jahr nichts gelernt habt.}

\enquote{Nichts ist nicht ganz richtig. Das, was wir selber gelernt haben, war schon viel.}

\enquote{Das galt aber nicht für alle.}

\enquote{Und die Theorie war zumindest ein verschwindend geringer Teil}, sagte Harry und grinste Dumbledore an.

\enquote{Ich habe mir sagen lassen, dass du die praktische Prüfung, ebenso wie ein Großteil aus deiner Jahrgangsstufe außer den Slytherins, hervorragend geschafft hast. Dolores hat getobt, habe ich gehört, als die Ergebnisse bekannt wurden.} Dann grinste er.

\enquote{Professor? Kommen Sie noch kurz mit rein? Dann sollten sie sich noch umziehen, sonst bekommt mein Onkel wieder einen Anfall.}

\enquote{Ich glaube nicht. Ich würde trotzdem auffallen.} Kurz vor der Tür verabschiedeten sich die beiden voneinander. \enquote{Mach’s gut Harry.}

\enquote{Bis zum September Professor}, antwortete Harry und ging ins Haus.

Dumbledore ging die Straße weiter und schaute noch auf einen Tee bei Arabella Figg vorbei.

\trenn

In der letzten Woche vor Schulbeginn flatterte am Dienstagmorgen eine Eule zum offenen Fenster herein. Onkel Vernon war gerade nicht da, und Dudley vergnügte sich anscheinend mit seinen Kumpels. In der Küche stand gerade Harry, der seiner Tante beim Abspülen half, als die Eule auf sich aufmerksam machte. Sie hatte einen Brief an ihrem Fuß, den Harry abmachte. Er gab ihr noch etwas von den Resten des Essens und die Eule flog davon. Harry öffnete seinen Brief.

\enquote{Das ist meine Einkaufsliste von \gst äh \gst Schulsachen, Tante Petunia}, sagte er seiner Tante, die ihn mit erhobener Augenbraue und zusammengekniffenen Lippen anschaute. Nervös drehte sie sich um, sah zu ihm zurück und sagte dann. \enquote{Was geht mich das an? Am Freitag habe ich Besseres zu tun. Du wirst mit mir einkaufen gehen. Ich habe ein paar Besorgungen zu machen. Schwere Sachen, die ich Dudley nicht heben lassen kann.}

Harry schluckte. Wie sollte er seine Schulsachen besorgen, wenn ihn keiner abholte? Er dachte kaum, dass seine Tante oder sein Onkel ihn zur Winkelgasse oder zumindest in die Nähe bringen würden.

Am Freitag darauf stand Harry auf und saß wie gewöhnlich am Frühstückstisch, als seine Tante zu sprechen anfing.

\enquote{Vernon, ich brauche heute den Wagen. Ich gehe mit Harry ein paar Besorgungen machen. Und Dudley kann sich dabei verletzen.}

Onkel Vernon entwich nur ein kurzes \enquote{Hm, ja.} Er aß unbekümmert weiter. Ihn störte es nicht, wenn sich Harry danach nicht mehr bewegen konnte.

Harry stand auf und legte seinen Teller in die Spüle. Tante Petunia lief ihm hinterher und ging dann durch den Flur zur Tür hinaus. Harry folgte ihr. Im Auto angekommen setzte sich Harry hinein und fuhr mit seiner Tante los. Er war erstaunt, dass sie nicht wie üblich in eines der Kaufhäuser fuhr, die sie sonst immer benutzte. Dieses Mal benutzte sie die Schnellstraße und fuhr Richtung London. \gedanke{Das kann ja heiter werden; London, und keine Möglichkeit Schulsachen zu kaufen.} Seine Tante fuhr durch die halbe Stadt, an vielen Kaufhäusern vorbei, von denen Harry dachte, da müsste er jetzt rein. Zu seinem großen Erstaunen fuhr sie auf einen Parkplatz in der Nähe des tropfenden Kessels. Wenn er dort nur für eine halbe Stunde verschwinden könnte, dachte er. Als sie anhielt, wollte er gerade aussteigen, als ihn seine Tante am Arm festhielt. Sie zog einen Zettel aus ihrer Tasche und reichte ihn Harry. Er entfaltete das Pergament und entdeckte seine Einkaufsliste.

\enquote{Du machst deine Besorgungen und ich mache meine. In einer Stunde treffen wir uns wieder hier. Und \gst kein Wort zu deinem Onkel oder zu Dudley. Verstanden?}

Harrys Augen weiteten sich und sein Mund stand offen. Er war sprachlos. Als er sich wieder fasste, brachte er nur ein \enquote{Ja, Tante Petunia} heraus. Er stieg aus dem Wagen aus und machte sich auf den Weg zum tropfenden Kessel. Seine Gedanken schwirrten umher. \gedanke{Wusste Tante Petunia, wo sich die Winkelgasse befand? War sie nur zufällig hier, weil sie in der Gegend einkaufen wollte?} Harry konnte es gar nicht so recht glauben. Ihm war es nur recht, dass er eine Gelegenheit hatte seine Sachen einzukaufen.

Nachdem er bei Gringotts wieder etwas Geld abgehoben hatte, ging er einkaufen. \gedanke{Hier dürfte mir nichts passieren}, dachte er. Er kaufte seine Bücher bei Florish \& Blotts, neue Tinte und reichlich Pergamentrollen. Außerdem frische Eulenkekse für Hedwig, die während der gesamten Ferienzeit keinen einzigen Brief für ihn zu transportieren brauchte und so recht entspannt wirkte. Lediglich Arabella hatte ein paar wenige verschickt. Nachdem er seine Einkäufe erledigt hatte, sprach ihn von hinten jemand an.

\enquote{Hallo Harry, auch beim Einkaufen?}

Er drehte sich um und sah Luna, die ihn anlächelte.

\enquote{Ja. Bin gerade fertig geworden.}

Neben Luna stand ein älter aussehender Mann.

\enquote{Das ist mein Vater}, sagte sie.

Er schüttelte Harry die Hand und meinte nur: \enquote{Schön Sie mal persönlich kennenzulernen Mister Potter. Meine Tochter sagte, Sie seien letztes Schuljahr ein ausgezeichneter Lehrer gewesen, als Sie gegen Miss Umbridge agierten. Und das Interview ließ die Auflage meines Magazins in die Höhe schnellen.}

Harry empfand Mister Lovegood als durchaus sympathisch.

\enquote{Danke Mister Lovegood. Das hat auch Spaß gemacht.}

\enquote{Ach nennen Sie mich doch Xenophilius.}

\enquote{Gerne, dann müssen Sie mich aber auch Harry nennen.}

\enquote{Geht klar.}

Die beiden verabschiedeten sich und Luna winkte ihm zum Abschied nochmals zu. Harry schaute auf seine Uhr, die er gekauft hatte und bemerkte, dass er noch etwas Zeit hatte. Langsam schlenderte er die Winkelgasse entlang, als sein Blick in das Schaufenster von Ollivanders fiel. Er betrat den Laden und Mister Ollivander, kam kurz darauf um die Ecke.

\enquote{Ah, Mister Potter. Schön Sie wiederzusehen. Wie geht es Ihnen? Ist mit Ihrem Zauberstab alles in Ordnung?}

\enquote{Ja}, meinte Harry. \enquote{Mir geht es gut. Aber was ich Sie fragen wollte. Wenn ich meinen Zauberstab verliere, gibt es dann irgendeinen Ersatz? Ist der dann auch so gut wie mein alter?}

Mister Ollivander schaute ihn an. Er atmete tief ein und dann wieder aus.

\enquote{Normalerweise ist ein Ersatz kein Problem, Mister Potter. Aber bei ihrem \gst nun ja. Wissen Sie, es ist so. Phönixfedern sind extrem selten. Ich habe keine mehr auf Lager. Sie müssten mir also eine Beschaffen, wenn Sie einen neuen oder einen Ersatz wollen.}

Harry schaute Mister Ollivander nur an. Er wusste nicht, was er sagen sollte. Dann fiel ihm etwas ein.

\enquote{Wissen Sie von welchem Phönix meine Feder stammt?}

\enquote{Ja. Aber Sie müssten ihn überreden, ihnen eine abzugeben. Anders als bei Einhörnern oder Drachen lassen sich nur freiwillig gegebene Phönixfedern in Zauberstäben verwenden.}

\enquote{Und welcher war es?}

Mister Ollivander Augen begannen leicht zu leuchten.

\enquote{Fawkes}, sagte er.

Harry fing an zu grinsen.

\enquote{Wieso grinsen sie Mister Potter?}, fragte Mister Ollivander nach.

\enquote{Nun ja}, sagte Harry, \enquote{ich kenne Fawkes. Vielleicht kann ich ihn irgendwann dazu überreden, falls ich mal Bedarf dazu habe.}

Harry schaute wieder auf seine Uhr. Er bemerkte, dass er langsam gehen musste.

\enquote{Ich muss leider gehen Mister Ollivander, meine Tante wartet sonst auf mich.}

Er ging durch die Tür, zurück in den Tropfenden Kessel und hinaus in die Welt der Muggel. Er fand seinen Weg zurück zum Auto, wo Tante Petunia gerade ihre Sachen einlud. Er verstaute ebenfalls seine Einkäufe und sie fuhren zurück. Zu Hause packte Harry seine Sachen schnell in sein Zimmer und half seiner Tante ihre Einkäufe auszuladen. Onkel Vernon kam in die Küche und staunte.

\enquote{Du hast ja kaum was dabei Petunia}, sagte Onkel Vernon.

\enquote{Sie haben die Sachen bedauerlicherweise nicht mehr da gehabt. Das müssen sie erst bestellen.}

Harry war sich nicht sicher, ob Tante Petunia log und ihn nur seine Schulsachen einkaufen ließ. Er war sich nicht einmal sicher, was sie wusste und was sie verdrängte. \gedanke{Irgendwie hatte sie der Brief von Dumbledore verändert}, bemerkte Harry. \gedanke{Aber was soll’s. Ich hab meine Schulsachen und Tante Petunia ihre Einkäufe.}

Plötzlich flatterte wieder eine Eule herein. Onkel Vernon zuckte zusammen und wollte schon wieder etwas sagen, hielt sich aber zurück. Harry nahm den Brief und las ihn.

\begin{brief}
Lieber Harry,

Ich schicke dir Sonntag um 9 Uhr einen Ministeriumswagen vorbei.

Halte deine Sachen gepackt.
\signumspace
Arthur Weasley
\end{brief}

\enquote{Ich werde am Sonntag abgeholt}, sagte Harry trocken und ließ den Brief auf den Tisch fallen. Er ging nach oben, kontrollierte, ob er auch alle Hausaufgaben gemacht hatte und verstaute seine Sachen in seinem Koffer.

Am Sonntag dann stand er auf und ging in die Küche um zu Frühstücken. Es läutete und Onkel Vernon öffnete die Tür. Harry saß am Frühstückstisch und konnte die Unterhaltung mit anhören.

\enquote{Guten Tag, ich komme Harry abholen}, hörte Harry eine Männerstimme sagen.

\enquote{Junge, komm her, hol deine Sachen und verschwinde}, sagte Onkel Vernon.

Harry stand auf, ging hoch in sein Zimmer, machte seinen Koffer wieder leichter, nahm ihn und Hedwigs Käfig und ging die Treppen hinunter. Der Mann nahm Harrys Koffer an sich und schmunzelte, als er bemerkte, dass er sehr leicht war. Er verstaute ihn in seinem Kofferraum und Harry verabschiedete sich von seinem Onkel und seiner Tante. Am Wagen angekommen, öffnete er ihn und sah, dass bereits Hermine im Inneren saß.

\enquote{Hallo Hermine.}

\enquote{Hallo Harry.}

Nach einer angenehmen Fahrt zum Londoner Bahnhof lud der freundliche Herr die Gepäckstücke aus und fuhr davon.

Im Zug sitzend warteten beide auf Ron, der sich kurze Zeit später blicken ließ. Während der ganzen Fahrt über unterhielten sich die drei über die vergangenen Ferienwochen. Nach einer Weile fasste Harry wieder sein Amulett an und schloss die Augen.

\begin{traum}
Er war wieder im Raum der Wünsche, als er letztes Schuljahr seine Mitschüler in Verteidigung gegen die dunklen Künste unterrichtete. Er stand hinter Ginny und sah ihr dabei zu, wie sie ihren ersten gestaltlichen Patronus herbei schwor. Dieser hatte die Form eines Pferdes. Harry lief weiter, an Neville vorbei und ermutigte ihn, es gleich wieder zu versuchen, nachdem sein Versuch missglückt war. Er sprach der ganzen Gruppe Mut zu und sagte ihnen, dass ein gestaltlicher Patronus schwierig sei, aber auch abstrakte Formen gut seien. Er lief an Ron vorbei, dessen Hund-Patronus Neville umwarf. Dann sah er wie Hermines Otter-Patronus umherflog und sie dabei leicht kitzelte. Er ging weiter zu Luna und sagte, sie möge es versuchen. Sie erzeugte einen kleinen Hasen, der in der Luft umhersprang. Sie drehte sich zu ihm um und lächelte ihn an. Harry lächelte zurück.
\end{traum}


Er öffnete seine Augen und dachte sich: \gedanke{Wieder das Gleiche. Jedes Mal sehe ich Lunas Gesicht. Aber Luna ist recht nett. Ich mag sie. Sogar sehr gerne.}




\begin{kommentar}
Die Idee zum Basiliskenanhänger kam mir, als ich eine andere Geschichte (Die übersinnliche Schlange) las. Dort konnte Harry Ginny sehen, wenn er den Anhänger in seiner Hand hielt. Hier stellt sich heraus, dass der Anhänger wirklich von Salazar Slytherin ist und einen kleinen Teil von Salazar selbst beinhaltet. Damit hat Harry Zugriff auf Teile Salazars Magie und auch Salazars Wissen selbst.
\end{kommentar}

\begin{kommentar}
Während Harrys Geburtstagsfeier bei Arabella hält diese ein kleines Brett heraus. Ein Mini-Quidditch-Spiel. Die Idee zu solch einem Spiel kam mir, als ich den Film >Zathura - Ein Abenteuer im Weltraum< gesehen hatte. Es ist eine Mischung zwischen diesem Spiel und >Jumanji<.
\end{kommentar}

\begin{kommentar}
Harry pflanzt im Garten seiner Tante Petunien. Eine schöne Anspielung auf ihren Namen.
\end{kommentar}

\begin{kommentar}
Wieder zurück bei den Dursleys bekommt Harry einen Brief und kurz darauf erscheint Remus Lupin. Harry fliegen Bildfetzen zu, von etwas, was Harry nicht über seinen ehemaligen Lehrer wissen kann. Hier greift zum ersten Mal die Magie des Amuletts, das Harry zum Geburtstag geschenkt bekommen hat.
\end{kommentar}

\begin{kommentar}
Der Notarin Mrs Plaustein habe ich Tonks Vornamen gegeben, weil ich es als nett empfand, jemanden zu haben, der diesen Vornamen auch trägt und ihn nicht so ablehnt, wie Tonks es tut.
\end{kommentar}

\begin{kommentar}
Später, als Harry seine Liste mit Schulsachen bekommt, fährt ihn seine Tante in die Nähe der Winkelgasse, da er sonst keine Gelegenheit bekommen hätte, seine Schulsachen zu kaufen. Wieder ein kleiner Hinweis, dass seine Tante mehr weiß, als sie zugibt.
\end{kommentar}
