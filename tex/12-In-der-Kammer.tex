\chapter{In der Kammer}

\begin{traum}
\enquote{Harry, wenn Voldemort uns damals nicht getötet hätte, \gst dann hättest du jetzt eine kleine Schwester}, sagte Harrys Mutter.

Harry musste das erst einmal verarbeiten. Er stand auf und setzte sich zwischen seine Eltern. Dann legte er seine Hand auf ihre. Sein Vater legte eine Hand auf seine Schulter. Ihre Berührungen fühlten sich so real an. Davon hatte Harry lange geträumt. Endlich seine Eltern in seinen Armen zu halten. Harry rutschte näher an seine Mutter heran und legte einen Kopf auf ihre Schulter. Seinen Vater zog er mit einer Hand zu sich, um ihm zu signalisieren, dass er näher kommen möge.

\enquote{Welchen Namen wolltet ihr meiner\abs Schwester geben?}, fragte Harry.

\enquote{Als wir noch lebten}, sagte sein Vater hinter ihm, \enquote{hatten wir noch keine Zeit dazu, aber jetzt haben wir uns auf Jamie geeinigt.}

Harry lächelte seine Mutter an und gab ihr einen Kuss auf die Wange. Eine Hand war noch immer auf ihrer, welche auf ihrem Bauch lag.

Dann begann sich Schwere in seinen Gliedern bemerkbar zu machen. Die Umgebung wurde langsam undeutlicher. \enquote{Bis bald, Harry}, sagten beide Eltern. Dann verschwand die Umgebung und Harry schwebte.
\end{traum}

Er öffnete seine Augen und wunderte sich, warum die Decke in seinem Bett plötzlich aus Stein bestand. Dann realisierte er, dass er keine Bettpfosten sah und auch das Bett fühlte sich anders an.

\enquote{Madame Pomfrey. Er ist wach}, hörte er eine Stimme sagen.

\gedanke{Hermine, das war Hermine. Ich bin im Krankenflügel.} Er hob leicht seinen Kopf und sah in Hermine und Rons Augen.

Madame Pomfrey trat an ihn heran und meinte: \enquote{Endlich, Mister Potter. Wir hatten uns schon Sorgen gemacht. Sie sind gar nicht mehr aufgewacht.}

Harry sah sie verwundert an. \enquote{Ich habe doch nur normal geschlafen. Wie spät ist es?}

\enquote{Fast Mittag}, sagte Ron.

Harry schaute Ron erstaunt an. \enquote{Dann hatte ich doch etwas länger geträumt.} Er richtete sich auf, hob die Hand vor seinen Bauch und öffnete sie. Darin lag noch immer sein Basilisken-Amulett. Er lächelte, als er an seinen schönen Traum dachte. Oder war es doch mehr? Hatte er wirklich Kontakt zu seinen Eltern hergestellt? Oder hatte das Amulett nur seine Träume beeinflusst und ihm die Illusion seiner Eltern gegeben? Er hatte nur eine Chance das zu erfahren. Er musste jemanden finden, der seine Eltern gekannt hatte und die von seiner ungeborenen Schwester wusste. Aber das war unmöglich. Er musste also später noch einmal diesen Traum erleben und sie etwas fragen, was er unmöglich über einen andere Wege erfahren konnte, um es sich bestätigen zu lassen. Denn andersherum würde es nicht funktionieren.

\trenn

Er saß alleine auf einer der vielen Treppen in Hogwarts, um über seine Vision nachzudenken. Er wusste nicht, wie er es herausfinden sollte, ob das, was er in seiner Vision gesehen hatte, wahr war, oder doch nur ein Traum. Ganz in Gedanken, seinen Blick durch die Mauer hindurch in die Unendlichkeit gerichtet, begann die Luft vor ihm zu flirren und Salazar erschien ihm wieder.

\enquote{Harry, hör mir bitte zu. Ich habe mich\abs na ja\abs etwas in deinem Kopf umgesehen und festgestellt, dass dir Okklumentik-Übung fehlt. Du hast hier einen guten Lehrer darin, nur seid ihr nicht gerade gut aufeinander zu sprechen. Gehe deshalb in meine Kammer, hole dort vom toten Basilisken Trankzutaten und besorge eines meiner Bücher. Sie sind auf Parsel geschrieben. Du wirst sie lesen können. Gib die Sachen deinem Lehrer. Ich denke, er wird dir dankbar sein und den Unterricht wieder aufnehmen, wenn du ihn darum bittest. Die Tränke sind sehr alt und wertvoll. Schreib ein paar davon ab \gst du wirst wissen, welche \gst und bring sie ihm mit den Zutaten.}

\enquote{Du bist in meinem Kopf?}, fragte Harry erstaunt.

\enquote{Du bist ein Teil von mir, Harry.}

\enquote{Ja, aber\abs}

\enquote{Kein aber, Harry. Du bekommst von mir ebenso viel zurück. Meine magische Stärke, meine Duellierfähigkeiten und mein Wissen, das ich über die Jahre angesammelt habe. Es liegt noch verborgen in dir. Du wirst darauf Zugriff haben, wenn du es brauchst. Danach kannst du es dauerhaft und wirst dich immer daran erinnern.}

Die Luft begann wieder zu flirren und Salazar verschwand.

Harry saß noch einige Minuten auf der Treppe und dachte nach, bis er schließlich auf sein Zimmer ging und zu packen begann.

Er holte seinen Rucksack heraus und packte Pergamente ein, sowie eine Feder und ein kleines Tintenfass. Er zauberte sich Phiolen und kleine Tücher zum Einwickeln herbei, kleine Holzkästchen für größere Gegenstände, diverse Werkzeuge zum Zerkleinern, verkleinerte alles und zog sich danach festes Schuhwerk an.

Salazar hatte ihm nicht gesagt, dass er Hilfe mitnehmen sollte, also ging Harry alleine.

Im Stockwerk mit der maulenden Myrte angekommen, betrat er das Mädchenklo.

\enquote{Myrte?}, fragte er in den Raum hinein, doch es kam keine Antwort.

Vorsichtig schaute er sich um, bis er schließlich feststellte, dass sie nicht da war. Er ging zum Waschbecken, suchte das spezielle mit der Schlange auf dem Wasserhahn und sprach in Parsel: \parsel{Öffne dich.}

Das Waschbecken gab den Zugang frei und Harry rutschte die Röhre hinab in die Dunkelheit.

Unten angekommen fiel er wie letztes Mal in zahlreiche Knochen und Knochenreste.

\gedanke{Mahlzeiten eines Basilisken}, dachte er amüsiert. Doch nach Entspannung, oder gar Frohsinn, war ihm nicht. Er hörte ein dumpfes \geraeusch{Klonk}. Der Zugang muss sich wieder verschlossen haben. \gedanke{Darüber mache ich mir später Gedanken}, dachte sich Harry und stieg über das Geröll und den Schutthaufen, den damals Lockhart durch seinen fehlgeschlagenen Zauber ausgelöst hatte. Kurz danach sah er schon die riesige Schlangenhaut.

Er holte ein Messer aus seinem Rucksack und schnitt ein großes Stück heraus, vergrößerte ein eingepacktes Tuch, wickelte die Schlangenhaut darin ein und verkleinerte es. Danach schob er es in seinen Rucksack. Nachdenklich sah er den Rest der Haut an und entschloss sich, einen Verkleinerungs-Zauber an der ganzen Haut zu probieren. Nachdem der Versuch geklappt hatte, steckte er diese Haut ebenfalls ein.

\gedanke{Ich denke, ein persönlicher Vorrat an Trank"-zu"-ta"-ten wäre vielleicht nicht schlecht.}

Er schritt weiter mit leuchtendem Zauberstab den gewundenen Gang entlang, bis er abermals vor der kreisrunden Tür mit den vielen Schlangen stand.

Wie schon in seinem zweiten Jahr befahl er der Tür erneut sich zu öffnen und trat danach in die Kammer. Am Ende konnte er noch immer den Körper der Schlange sehen. Er schien kaum verwest zu sein. Vor dem riesigen Tier angekommen, sammelte er alles ein, von dem er dachte, er brauche es. Er nahm sich Schuppen, Zähne, Blut, Speichel und Gift. Und wieder zweigte er sich einen Teil für seinen persönlichen Vorrat ab.

Dann versuchte er den Basilisken zu verkleinern und steckte ihn ebenfalls in seinen Rucksack. Langsam wurde er schwer, daher stelle er ihn an den Rand des Raumes und sah zu Salazars Statue.

Er dachte eine Weile nach, bis ihm die Idee kam. \gedanke{Das muss Salazars Wissen sein}, dachte er sich. Er rief ihn. \enquote{Salazar?}

Kurz darauf schwebte der alte Zauberer vor seiner eigenen Statue.

\enquote{Ja, Harry?}

\enquote{Wollen wir ein bisschen spazieren gehen?}

\enquote{Gerne, aber ich kann doch nicht lange\abs}

Harry unterbrach ihn. \enquote{Ich verfüge über dein Wissen und eine Statue von dir. Sagt dir das was?}

Salazar runzelte erst die Stirn, dann kam ihm die Erkenntnis. Er nickte, drehte sich kurz um, um sich seine Staute zu beschauen, und sah dann wieder zu Harry. Er stellte sich so wie seine Statue hin und wartete.

Harry zog seinen Zauberstab und murmelte einen Zauber, woraufhin es Salazars Geist nach hinten in seine Staute zog. Kurz darauf fing sie an zu blinken und bewegte sich. Es dauerte noch eine knappe Minute, bis der weiße Marmor Farbe bekam und es so schien, als ob Salazar Slytherin leibhaftig vor ihm stand. Beide wusste, dass es nicht von langer Dauer war, aber die gemeinsame Zeit wollten beide nutzen.

\enquote{Sag deinem Elfen Bescheid, Harry, dass er deine Trankzutaten für dich einlagern soll und den Rest für dich auf deinem Zimmer sicher verstauen soll. Du wirst es dort wieder finden.}

Harry nickte und rief nach Kreacher.

Dieser erschien und verneigte sich. \enquote{Sir Harry hat Kreacher gerufen}, krächzte der Elf.

Kreacher hatte sich in den letzten Wochen schwer verändert. Seit Harry ihm ein Erbstück von Meister Regulus und zu Weihnachten schließlich die Köpfe seiner Vorfahren geschenkt hatte, war der Elf glücklich und ihm gegenüber loyal geworden.

Harry erklärte ihm, was er von ihm verlangte, und Kreacher verschwand mit einem Kopfnicken samt Rucksack.

Dann winkte Salazar Harry heran und umarmte ihn erst einmal. Dicke und glückliche Tränen flossen an ihm herab. Nachdem er sie fort gewischt hatte, Harry traute sich nicht zu fragen, wieso, ging er voran und Harry folgte ihm durch einen der Gänge. Salazar zeigte auf einen Steinvorsprung und Harry fühlte, wie das Wissen in ihn strömte. Er öffnete die verborgene Tür mit einem sanften Vorbeifahren seiner Hand und trat nach Salazar hindurch. Dahinter war ein kleines Arbeitszimmer mit einem kleinen Bücherregal.

\enquote{Es gibt noch einen anderen Zugang, von meinen Privaträumen aus, aber den zeige ich dir ein anderes Mal.}

Harry nickte, da er überwältigt war, das Arbeitszimmer eines seiner Ahnen zu sehen. Salazar zeigte auf eines der Bücher und Harry fühlte wieder, wie ihm das Wissen seines Ahnen etwas zuflüsterte. Er legte einen Schutzzauber auf das gute Dutzend Bücher, um sie vor dem Zerfall zu schützen. Alle Bücher waren in Parsel geschrieben.

Harry wurde bewusst, dass niemand wusste, dass man Parsel auch schreiben konnte, aber die wenigen, die Parsel sprechen konnten, schienen miteinander verbunden zu sein, denn Salazar hatte herausgefunden, nachdem er Hogwarts verlassen hatte, dass alle, die Parsel sprechen konnten, dieselben Symbole verwendeten und sie auch lesen konnten.

Parsel hatte Ähnlichkeiten mit dem Arabischen, was die Symbolik anbelangte, wich aber doch davon ab. Kleine Schlangen schienen die Symbole zu sein.

Harry las: \parsel{Gesammelte Rezepte und Tränke mit und über Schlangen von Salazar Slytherin}

Sein Geist übersetzte und er verstand. \accentuate{Gesammelte Rezepte und Tränke mit und über Schlangen von Salazar Slytherin}

Harry hatte eine ungefähre Ahnung, was sich in den Büchern befand, und ignorierte die gefährlichen Bücher, die sich mit den dunklen Künsten beschäftigten. Er nahm das Buch mit den Tränken und schlug es auf. Zwar hatte Kreacher Harrys Rucksack mit den Pergamenten, der Feder und dem Tintenfässchen mitgenommen, aber auf dem Tisch lag genug davon. Nur die Tinte musste reaktiviert werden. Harry wollte schon eine der Federn eintauchen, entschied sich aber dann um und duplizierte das Tintenglas, legte das Original zu den Büchern und benutzte die Tinte aus dem duplizierten Glas, um einige Tränke abzuschreiben.

\enquote{Warum hast du das gemacht, Harry?}, fragte ihn sein Ahne.

\enquote{Alte Tinte ist kostbar. Diese hier ist über tausend Jahre alt. Wenn sie weg ist, gibt es nichts mehr davon. So kann ich die Rezepte abschreiben und meine ganzen schulischen Sachen mit deiner alten Tinte schreiben.}

Salazar bekam große Augen.

\enquote{Tust du mir einen Gefallen, wenn wir hier fertig sind?}, fragte er.

\enquote{Welchen?}

\enquote{Verkleinere meine Statue und nimm sie mit. Stelle sie bei dir auf deinen Nachttisch, oder tue sie in deinen Koffer. Nur lass mich hier nicht alleine zurück}, sagte er mit gewisser Sehnsucht in der Stimme.

Harry nickte und schrieb weiter. Kreacher kam mit einem Teller voll belegter Brote und zwei selbst auffüllenden Gläsern mit Kürbissaft. Beide aßen und tranken und Salazar beteiligte sich an Harrys Kopieraktion. Dann legte er das Buch zurück und beide verließen den Raum.

Salazar begleitete ihn noch bis zu der Röhre, die er heruntergekommen war, dann konnte er sich nicht mehr bewegen. Die Statue nahm die alte Form und Farbe an, Salazars Geist wurde herausgepresst und verblasste. Harry hörte ihn noch erschöpft in seinem Geist: \stimme{Danke, aber jetzt muss ich mich erst einmal einen Tag ausruhen. Wir können frühestens übermorgen wieder reden.}

Harry verkleinerte die Marmorstatue und schob sie ein. Er hatte kein schlechtes Gewissen, da sie als seinem Erben ja eh ihm gehörte. Außerdem hatte er Salazars Erlaubnis. Dann betrachtete er die Röhre und dachte nach, als er hinter sich ein Flügelschlagen hörte.

Er drehte sich um und da war Fawkes. Harry lächelte ihn an und hob ihm seine Hand hin. Der Vogel flog auf seine Schulter, wo ihn Harry streicheln konnte. Gerne ließ sich das Tier das gefallen und wippte nach einiger Zeit mit dem Kopf. Harry verstand, Fawkes hob ab und Harry hielt sich an seinen beiden Füßen fest. Dann flog der Phönix mit ihm die Röhre hinauf. Harry zischte auf gut Glück im passenden Moment Parsel und kam durch den offenen Zugang im Mädchenklo an.

Als er wieder auf seinen eigenen Füßen stand, sah Harry Myrte, wie sie schüchtern um eine Ecke sah.

\enquote{Hi Myrte. Schön dich zu sehen. Wie geht es dir?}

\enquote{Du kommst mich besuchen?}, fragte sie.

\enquote{Eigentlich wollte ich dich vorher mit in die Kammer nehmen, weil ich was brauchte, aber du warst nicht da.}

\enquote{Oh, ich war beschäftigt. Sitzung mit den anderen Geistern. Aber schön, dass du da bist.} Sie flog auf ihn zu und drückte ihm einen kalten Kuss auf die Wange. Danach wurde sie rot. Wenn man es für rot halten konnte, wenn man von einem Geist spricht.

Harry fuhr ihr über die Wange, was ihr einen Schauer über den Rücken laufen ließ. Er musste noch eine Menge an Magie in sich haben, nachdem er gerade eben erst aus der Kammer kam und mit Fawkes herauf geflogen war. Er lächelte ihr zu und verabschiedete sich von ihr. Myrte schien glücklich über die kurze Freundschaftsbezeugung zu sein.

Dann lief er zurück in sein Zimmer, die Pergamente und die Trankzutaten waren in einem kleinen neuen Koffer durch Kreacher bereitgelegt, und stellte Salazars Statue auf seinem Nachtschrank ab. Er versah sie mit einem Sicherungszauber, damit sie kein anderer mitnehmen konnte. Jeder konnte sie zwar berühren und hochheben zum Anschauen, aber das Zimmer konnte keiner mit ihr verlassen. Selbst ein Umstellen war nicht möglich. Ein Fremder wurde gezwungen, die Statue wieder auf Harrys Nachtschrank zu stellen.

\trenn
\onelineback % Anderenfalls werden 2 Leerzeilen gesetzt

\begin{traum}
Er lief einen dunklen Gang entlang. Immer wieder musste er abbiegen, bis er in einen runden Raum gelangte. Und dort stand er, sein Angstgegner. Instinktiv zog er seinen Zauberstab, um sich verteidigen zu können. Beide waren alleine, niemand konnte ihnen helfen. \gedanke{Endlich}, dachte er. Dieses Mal wird es vorbei sein. Es gibt kein Entkommen mehr. Zauber um Zauber schleuderte er ihm entgegen, doch alle wurden abgewehrt. Dann wollte er es beenden und griff zum äußersten. Doch selbst dieser Zauber wurde abgewehrt und der grüne Blitz flog abermals auf ihn zurück.

Schweißgebadet und schreiend saß er im Bett. Nach wenigen Sekunden wurde seine Tür aufgerissen und Bellatrix stand im Rahmen. \enquote{Alles in Ordnung, mein Lord?}, fragte sie besorgt.

Er sah sie nur an.

Gerade als sie gehen wollte, hielt er sie auf. \enquote{Bleib hier, Bellatrix. Komm her.}

Ihr Herz machte einen Hüpfer, als sie die Tür hinter sich schloss. Voldemort rückte ein wenig zu Seite und sagte: \enquote{Komm her, Bellatrix. Neben mich, auf die Decke.}

Sie kam wie eine Katze auf das Bett zu und legte sich hin.

Er beugte sich über sie und besah sich ihr Gesicht. Irgendwie musste er sich beruhigen und seinen Ausbruch wieder unter Kontrolle bringen. Mit seinen langen Fingern fuhr er ihre Gesichtskonturen nach.

Bellatrix bekam durch diese Behandlung eine Gänsehaut und schloss ihre Augen.

Vollkommen Gefühlslos beruhigte er damit seine Gedanken. Dann legte er sich hin und schlief ein.

Bellatrix lag glücklich an seiner Seite und begann sich nach kurzem Wachen an ihn zu kuscheln.
\end{traum}

Harry erwachte und dachte über das nach, was er eben geträumt hatte. Er war Harry und duellierte sich mit Voldemort. Nein, er war Voldemort und duellierte sich mit Harry. Er erwachte schweißgebadet in Malfoy-Manor. Nein, er war Harry und erwachte normal mitten in der Nacht. Nein, er war Voldemort, der Bellatrix sanft streichelte und seine Gedanken beruhigte. Er war Harry, der alleine in seinem Bett lag und noch immer die Berührungen von Bellatrix’ Haut spürte. Mit eigenartigem Gefühl schlief er wieder ein. Er musste sich erst einmal klar werden, wer er war. Solche Träume waren einfach zu verwirrend.

Noch am nächsten Morgen dachte Harry darüber nach, was er vergangene Nacht erlebt hatte. Beim Frühstück erzählte er Ron und Hermine darüber.

\enquote{Harry, du musst deine Okklumentik-Stunden wieder aufnehmen.}

Vorsichtig legte er einen Finger auf seine Lippen und sprach leise: \enquote{Mache ich schon, momentan geht es etwas schleppend.}

\enquote{Du machst was?}, fragte Ron.

\enquote{Ich arbeite an Okklumentik-Stunden mit Snape. Ich hoffe, ich habe demnächst Glück. Es geht langsam bergauf.}

Hermine verengte ihre Augen, um Harry intensiv zu durchleuchten, kam aber zu dem Entschluss, dass er doch recht hatte.

Vor der vorletzten Unterrichtsstunde griff sich Harry aus seinem Raum den Rucksack und nahm ihn mit zu Zaubertränke. Heute musst er, wie seine Mitschüler auch, alleine einen Trank brauen. Professor Snape nannte ihnen die geforderte Seite und auch Harry schlug sein Buch auf.

Dieser Trank kam ihm bekannt vor. Es wurden Teile einer Schlange verwendet. Er blätterte kurz durch seine Pergamente welche er aus Salazars Büchern abgeschrieben hatte, und erkannte, dass er in Salazars Rezept nur das Gift eines Basilisken, anstelle des Giftes einer normalen Schlange verwenden musste. Da er ja Schlangengift dabei hatte, war das kein Problem und obwohl ihm Snape immer wieder über die Schulter sah und ihn versuchte zu Triezen, ließ er sich nicht irritieren. Harry tropfte die geforderte Menge des Basiliskengiftes in einem ruhigen Moment in den Trank und braute weiter. Er las immer mal wieder das Rezept durch und erkannte die Unterschiede zu Salazars Aufzeichnungen. Da er schlechte Erfahrungen mit seinen Schulbüchern gemacht hatte, hielt er sich bei diesem Rezept an das von Salazar. Es musste anders umgerührt werden und auch die Reihenfolge von zwei Zutaten war anders.

Doch Harry hatte ein gutes Gefühl und auch die Farbe seines Trankes nahm den gewünschten Farbton an. Er konnte Snape hinter sich förmlich sehen, wie er seine Stirne runzelte, als er mal wieder in seinen Kessel sah. Doch dieses Mal sagte er nichts und ließ Harry weiter brauen.

Am Ende der Stunde gab er seinen Trank ab und packte seine Sachen zusammen.

\enquote{Mister Potter}, schnarrte Snape, \enquote{50 Punkte Abzug für Gryffindor wegen eines misslungenen Zaubertrankes.} In Harry begann Wut hochzukochen, aber er hielt sich unter Kontrolle, schließlich hatte er ein Ziel. Er schluckte seinen Ärger herunter und trödelte, da er mit seinem Lehrer noch sprechen wollte. Er schnappte seine Tasche und folgte Snape in sein Büro.

\enquote{Was wollen Sie hier, Potter? Soll ich Ihnen nochmals Punkte abziehen?}

\enquote{Nein, Sir. Ich habe etwas für Sie. Ich dachte, das könnte Sie interessieren.}

Harry öffnete seinen Rucksack und nahm eine kleine Holzkiste heraus. Er stellte sie auf den Boden vor Snapes Schreibtisch und vergrößerte sie. Dann entnahm er eine Phiole und stellte sie auf den Tisch. Skeptisch besah sich Snape den Inhalt. Danach nahm Harry ein Tuch heraus und schlug es auf. Auch dieses Exemplar besah sich Snape. Als Nächstes holte Harry seine Kopie seiner Aufzeichnungen heraus und legte sie Snape auf seinen Tisch. Dieser warf einen kurzen Blick auf das oberste Rezept und wand sich dann der Phiole wieder zu, als sein Blick wieder zurück zur Rezeptur flog.

\enquote{Das haben Sie also gebraut heute}, sagte er. \enquote{Nun gut. Fünfzig Punkte für Gryffindor wegen eines hervorragenden Trankes. Die Farbe stimmte und auch die Konsistenz war in Ordnung.}

Harry traute seinen Ohren nicht. Fünfzig Punkte von Snape. Entsetzt und skeptisch sah er seinen Professor an.

\enquote{Was wollen Sie dafür?}, fragte Professor Snape und holte so Harry wieder in die Realität zurück.

\enquote{Dafür? Direkt nicht\abs Ich möchte nur die Okklumentik-Stunden wieder aufnehmen; und zwar richtig.}

\enquote{Und wenn ich ablehne? Nehmen Sie die Sachen wieder mit?}

Harry verkleinerte die Phiole und die Schlangenhaut samt Tuch und verstaute sie wieder in der Holzbox. Er legte die Pergamente oben auf und schloss die Box. Dann stellte er sie auf Professor Snapes Tisch.

\enquote{Nein. Ich überlasse sie Ihnen. Vielleicht sind sie für den Unterricht interessant. Es ist genug für alle Klassen und genug für Ihren Vorrat. \gst Ich möchte die Stunden unabhängig von dem, was ich ihnen gerade gegeben habe, erhalten.}

\enquote{Sie haben mir also gerade etwas geschenkt und erwarten keine Gegenleistung dafür?}

\enquote{Exakt.}

\enquote{Warum also sollte ich die Mühe auf mich nehmen und Ihnen Okklumentik-Stunden geben?}

\enquote{Lassen Sie mich folgendermaßen antworten: Wenn Sie ablehnen, werde ich mir das benötigte Wissen aus Büchern und mit meinen Freunden erarbeiten müssen. Es wird mir schwerer fallen und es wird mich eine Menge Arbeit und Anstrengung kosten. Aber wenn Sie nach einem Grund fragen\abs} Er stützte sich auf dem Schreibtisch auf und schaute seinem Professor in die Augen, \enquote{sehen Sie in meine Augen und beantworten Sie sich die Frage selber. Dann tarnen Sie die Unterrichtsstunden mit Nachsitzen.}

Harry nahm seine Tasche auf, wandte seinen Blick ab und verabschiedete sich mit den Worten: \enquote{Ich muss zur nächsten Stunde.}

Nachdenklich sah Snape eine Weile seinem Schüler nach und durch die offene Tür in sein Klassenzimmer. Mit einem Schlenker seines Zauberstabes schloss er seine Tür und verriegelte sie. Danach holte er ein Glas aus seinem Schreibtisch und goss sich einen Finger breit Feuerwhisky ein. Das Glas nahm er und setzte sich in einen schweren bequemen Sessel vor dem Kamin in einem Nebenzimmer. Nach fünf Minuten nahm er den ersten Schluck aus seinem Glas und schaute es nachdenklich an. Es schien grün zu schimmern. Genauso wie die Flammen einen Grünton haben. Lilys Grün.

Sichtlich erschöpft schlief er ein. In dieser Nacht träumte Severus Snape von Lily. Sie stimmte ihm zu, Harry wieder Okklumentik-Unterricht zu geben. Das Glas fiel ihm aus der Hand, erreichte aber nicht den Boden. Ein Hauself, der spät in der Nacht den Raum säuberte, bemerkte es rechtzeitig und ließ das Glas auf den Schreibtisch schweben. Er hinterließ unter dem Glas einen Zettel mit seinem Namen, denn Severus hasste es, wenn die Dinge nicht dort waren, wo er sie verlassen hatte. So wusste er, dass das Glas von einem Hauselfen gerettet wurde und nicht jemand im Zimmer war, der es ihm aus der Hand genommen hatte.

\enquote{Potter, diesen Zaubertrank üben Sie bei mir so lange, bis Sie ihn können}, ranzte Professor Snape durch das ganze Klassenzimmer. \enquote{Montagabend fangen Sie damit an. Das gibt Ihnen die Zeit, übers Wochenende noch einmal das Rezept durchzusehen und sich zu überlegen, was Sie falsch gemacht haben.} Damit war die Stunde beendet und Harry wusste, dass sich Snape dazu herabgelassen hatte, ihm die gewünschten Stunden zu geben. Denn sein Trank war heute perfekt. Er hatte als einziger die richtige Farbe, nachdem er sich an Salazars Rezept gehalten hatte. Es war ein sanfter Grünton. Da aber alle anderen einen Blaustich hatten; sie arbeiteten mit dem Schulbuch-Rezept; hatte Snape die perfekte Ausrede für das Nachsitzen.

Hermine schüttelte nur den Kopf und Draco konnte sich ein Grinsen nicht verkneifen. Harry aber nahm es stoisch hin. Die Punkte, die er wieder offiziell verloren hatte, würden ihm wahrscheinlich eh nicht offiziell abgezogen werden, da sein Trank die gleiche Punktezahl bekommen würde. Also konnte sich Snape den Abzug sparen und er zog nur wieder eine Show ab.

Er würde zwar nie mit ihm gut Freund werden, aber es schien für Harry so, als respektierten sie einander. Harry begann zumindest ihn zu respektieren.

Abends begann er endlich in dem Buch über Dementoren zu lesen. Er schlug das Buch auf und entdeckte unter einer Zeichnung eines Dementoren den Namen der Autorin. \gedanke{Adriana de Mimsy-Porpington. Ob sie mit Sir Nicolas verwandt ist? Den muss ich mal fragen.}

Harry schlug die ersten Seiten um und begann das erste Kapitel zu lesen:

\begin{buch}
\block{Vom Inferi zum Dementor}

Ursprünglich von Inferi abstammend, sind Dementoren eine der gefürchtetsten Kreaturen der magischen Welt. Bei Nachforschungen bin ich auf diese interessante Tatsache gestoßen. Da ich von Natur aus Neugierig bin, wollte ich diese Entwicklung natürlich selbst sehen. Ich besorgte mir also einen Inferi. (Nein, ich habe ihn nicht selber erzeugt.) Dieser Inferi hatte seine Aufgabe schon erledigt und war somit antriebslos. Ich nahm ihn also mit nach Hause und beobachtete ihn. Immer wieder musste man ihn feucht halten, damit er nicht austrocknete. Reden konnte er nicht, denn sonst hätte er mir sicherlich gesagt, warum er eines Tages verschwunden war. Doch kaum waren ein paar Wochen vorbei, kam er mit einer weiblichen Inferi an. Sechs Monate später kam ein kleiner Inferi auf die Welt.

Ich war wieder beruhigt. Doch eines Nachts hatte ich richtig Angst. Ich erwachte mitten in der Nacht und sah in zwei rote Augen. Das wenige Licht, das ins Zimmer schien, lies es mir kalt den Rücken herunterlaufen. Auf meiner Bettkante am Fußende saß eine Kreatur, schwarz und mit leuchtenden roten Augen, Finger und Zehen spitz zulaufend und weiße Linien über den Körper laufend. Eine Hand griff nach mir.

Als ich den ersten Schreck verdaut hatte, griff die Gestalt nach einer kleinen Wolke, die sie aus meinem Kopf heraus zog. Dann nahm sie diese in ihren Mund auf. Sie sah mich noch eine Weile an und verließ dann auf vier Pfoten mein Zimmer. Neugierig folgte ich ihr und stellte nach einer Weile fest, dass es der weibliche Inferi gewesen war; über Nacht hatte sie sich verändert. Sie hatte sich von meinem Albtraum genährt. In den folgenden Jahren erlebte ich kaum noch Albträume.

Doch wieder änderte sich etwas über Nacht. Plötzlich begannen meine Inferi; oder sollte ich sie schon seit der letzten Transformation anders nennen; über dem Boden zu schweben. Das Tageslicht begannen sie zu scheuen und sie bedeckten sich immer mehr. Zuerst nur mit Hüten oder T-Shirts, später mit dünnen Leinentüchern. Dann begannen sie meine guten Erinnerungen zu verdauen, zuerst nur zaghaft, dann aber immer aggressiver. Ich konnte das nicht mehr lange machen. Glücklicherweise hatte ich den Patronus-Zauber gut drauf, sodass ich meine Studienobjekte noch lange Zeit halten konnte.

Aber nur solange, bis ich entdeckte, dass sich in einem Teil meines Gartens, wo sie sich immer aufhielten, neue Dementoren (Ja, ich nenne sie jetzt so.) wuchsen. Wie Geister entstiegen sie dem Feld, langsam aber beständig über mehrere Stunden. Dann verlangten sie sofort nach Nahrung. Jetzt wusste ich, was ich wissen wollte. Dementoren entwickeln sich aus Inferi, wenn sie nicht nach Erfüllung ihrer Aufgabe beseitigt werden. Die neuen Nachkommen waren leider nicht mehr so pflegeleicht, wie die Eltern. Da aber scheinbar keine familiäre Bindung bestand, wurden sie weggeschickt. Meine beiden behielt ich, bis ich sie eines Tages schweren Herzens abgeben musste. Ich schickte sie nach Askaban. Da hatten sie es besser als hier. Doch bevor ich sie wegschickte, gab ich ihnen noch zu verstehen, sie mögen ihre Tücher lüften. Ich wollte sie noch einmal ohne sehen. Sie unterschieden sich kaum von dem Wesen, das zum ersten mal auf meiner Bettkante saß. Nur hatte es keine leuchtend roten Augen mehr, sondern schorfige Augenhöhlen. Die Nase war platt und hatte nur zwei Löcher. Das Loch im Mund war größer geworden und konnte nicht mehr geschlossen werden.

Dann schickte ich sie endgültig los. Das Männchen, mein ursprünglicher Inferi, legte eine seiner Hände gegen meine Stirn und übermittelte mir Bilder. Dann schwebten sie davon.\\
Erst nach einigen Wochen erkannte ich das außergewöhnliche Geschenk, das sie mir gaben. Sie gaben mir die Gabe, sich gegen sie immun zu machen. Außerdem teilten sie mir mit, wie man sie zerstören konnte. Es gab nur zwei Möglichkeiten. Ein konzentrierter Lichtzauber mit einem Zauberstab direkt in ihre Mundöffnung, oder durch gelenkte Patroni, die nicht nur als Nahrung dienten, sondern sie aktiv angriffen. Man musste es seinem Patronus aber befehlen und es musste ein gestaltlicher sein.
\end{buch}

Harry legte sein Buch weg, legte sich hin und schlug die Decke über sich. Er dachte nach. Über das was er eben erfahren hatte. Diverse Gedanken schlugen unkontrolliert in seinem Bewusstsein auf. \gedanke{Hatte Professor Elber Inferi bei sich, die jetzt zahme Dementoren sind?} Dann sah er eine glatte Wasseroberfläche mit einer Menge toter Körper darin. \gedanke{Inferi.} Dann nickte er weg und begann zu träumen. In einem Traum sah er die Kreatur auf seiner Bettkante sitzen. Die roten Augen sahen ihn an und nahmen die schlechten Gedanken von ihm.

\trenn

Es war wieder Samstag und Harry übte mit Ron und Hermine auf dem Besen den freien Fall. Sein dritter Versuch war bereits perfekt, doch er machte noch ein paar Sprünge. Er verstand sich mittlerweile blind mit seinem Besen. Er wusste, was er ihm zumuten konnte, und der Besen wusste, wie weit er gehen konnte, ohne Harry zu schaden. Ein letztes Mal für heute stieg er von seinem Besen und fiel im freien Fall. Der Besen raste ihm hinterher und fing ihn vor dem Aufprall ab. Bei seinem nächsten Spiel würde er sich bei passender Gelegenheit hinunterstürzen und die anderen schocken. So hoffte er bei einem Rennen seinen Gegner zu schlagen. Durch Überraschung.

Abends verschwand Luna mit Harry nach dem Abendessen in den Gemeinschaftsraum der Paare. In allen Sesseln und Stühlen saßen Pärchen, die sich küssten und streichelten. Einige bemerkten Lunas und Harrys Anwesenheit gar nicht. Andere wiederum grüßten sie und machten sofort weiter. Es war hier vollkommen normal und keiner musste sich schämen. Einige waren mit ihren Händen unter den Roben ihres Partners oder ihrer Partnerin. Luna und Harry entschlossen sich, sich in ihr Zimmer zurückzuziehen. Er betrachtete die Truhe, die schon immer in der Ecke des Zimmers stand. Harry fiel auf, dass er sich niemals Zeit genommen hatte sie sich anzuschauen. Er öffnete den Deckel und fand zu seiner Überraschung ein Schachbrett. Nach ein paar Runden in denen er 2:4 verloren hatte, räumte er es beiseite.

Als er das Spielbrett in die Truhe legte, hörte er nur von hinten: \enquote{Bleib so stehen Harry und drehe dich nicht um. Warte kurz, ich habe eine Überraschung für dich.}

Harry stand da, das Spielbrett noch immer in den Händen. Er ließ es los und schloss den Deckel der Truhe. Dann richtete er sich auf und stand mit dem Gesicht zur Wand. Er musste warten.

Nach einer Weile meinte Luna. \enquote{Du kannst dich jetzt umdrehen.}

Harry drehte sich um und wusste nicht, was er sagen soll. Da stand sie nun im fahlen Mondschein. Sie hatte nichts an. Sie war vollkommen nackt. Und außer ihrem Haupthaar fand er beim herunterschauen an ihr kein weiteres. Harry ließ vor Staunen seinen Mund offen stehen. So hatte er sie noch nie gesehen und das schimmernde Licht, dass sich auf ihrer zarten hellen Haut brach, ließ sie noch viel schöner erscheinen als die ganzen Male zuvor. Harry lief das Wasser in seinem Mund zusammen und er spürte, wie sein Herz anfing schneller zu schlagen. Er spürte plötzlich ein Verlangen und versuchte, seine Gedanken abzulenken, sodass Luna nicht mitbekommen würde wie er anfing sie zu begehren. Mit jedem langsamen Schritt, den sie ging, nur einen Fuß vor den anderen setzend, hörte Harry sein wallendes Blut in seinem Körper zirkulieren. Es war so, als ob er seine Umgebung nur noch schemenhaft wahrnehmen konnte. Er war nur auf Luna fixiert. Er konnte einfach seinen Blick nicht mehr von ihrem lösen. Langsam ließ er seinen Blick an ihr heruntergleiten. Unten angekommen fing er wieder an, seinen Blick nach oben zu wenden. Er schaute ihr tief in die Augen und wusste, er sollte nicht diesen begehrenden Blick haben.

\begin{abAchtzehn}
Er wusste, er sollte seine Gedanken ändern. Wusste, sie würde seine Gedanken in dieser Sekunde lesen können. Luna blieb nur wenige Zentimeter vor ihm stehen und ging mit ihrem Oberkörper nach vorne. Automatisch zog es Harry ihr entgegen. Ihre Lippen trafen auf seine und beide versanken in einem langen und wundervollen Kuss. Sie ging noch einen Schritt nach vorne und ihre Brüste berührten seinen Oberkörper. Harry durchfuhr ein Schauer wie er ihn noch nie erlebt hatte. So intensiv, so schön. Als er begann sich daran zu gewöhnen, fing er wieder Lunas Gedanken auf und verstand, dass sie ihn genauso wollte. Er fing an seinen Mund leicht zu öffnen und mit ihrer Zunge zu spielen. Sie begann sein Shirt zu öffnen und ließ es einfach auf den Boden fallen. Jetzt kam sein nackter Oberkörper voll zu Geltung. Seine Hände wanderten ihre Arme entlang hoch und er vergrub sie in ihren Haaren. Diese schimmerten in diesem Licht als wären sie aus purem Gold. Die Kerzen im Raum fingen an, langsam, kaum merklich, heller zu werden, um eine Stimmung zu schaffen, die jedem sagen würden, das sei der perfekte Ort, die perfekte Zeit. Langsam und für Harry fast unbemerkt näherte sie sich mit ihren Händen seiner Hose und begann sie zu öffnen. Sie fiel herunter und Luna ging einen Schritt zurück, damit Harry heraussteigen konnte. Er brach den Kuss und zog sie sanft zum Bett.

Sie hatten Zeit, unendlich viel Zeit, fand er. Plötzlich kam ihm der Gedanke an Verhütung.

\enquote{Mach dir keine Sorgen, Harry}, hörte er Luna sagen. \enquote{Ich war vor einiger Zeit bei Madame Pomfrey. Sie hat mir etwas gegeben. Wirkt ein paar Monate.}

Harry war erleichtert. Er fuhr mit seinen Händen ihren Konturen nach und er spürte ein leichtes Zittern. Jetzt begann er ihren Hals zu küssen. Mit seinen Händen umkreiste er ihren Rücken. Seine Hände wanderten nach vorne und er küsste sie weiter. Sie erwiderte seinen Kuss und begann ihren Mund leicht zu öffnen. Seine Hände glitten über ihre zarte Haut nach vorne, über ihre Brüste. Ihre Zunge umspielte seine Zähne, bis sie seine Zunge traf. Ihre Hände glitten weiter und begannen Harry sein letztes Stückchen Stoff auszuziehen. Als er den Kuss löste, streifte Luna ihm seine Unterhose ab und Harry saß nun auch vollständig nackt auf dem Bett. Harry legte sich mit dem Rücken mitten auf das Bett und Luna saß auf seinem Bauch. Er zog sie sanft an sich und küsste ihre Brüste. Ein wohliges Schnurren entfloh Luna. Er fing an, ihre Brüste mit seiner Zunge zu umspielen. Luna entfloh ein leichtes Keuchen und ihre Gedanken und Gefühle begannen sich mit denen Harrys zu vermischen. Sie küsste ihn und begann mit ihrer Zunge seinen Hals entlangzufahren. Sie umspielte seinen Adamsapfel und Harry begann sich aufzurichten. Sie umklammerte seinen Rücken mit ihren Händen und seine Hüfte mit ihren Beinen. Harry sah Luna in die Augen, schloss die seinen und begann sie erneut zu küssen. Langsam drückte sich Luna an ihm hoch und rieb an seinem Bauch entlang. Ein wohliger Schauer durchfuhr ihn. Dann ließ sie langsam ab und er glitt ihn sie hinein. Für ihn quälend langsam ging sie nach unten bis es nicht mehr ging. Sie hatten sich vereint. Mit geschlossenen Augen saßen sie einige Zeit nur da, sich umarmend und küssend. Die Blockaden waren gefallen und Harry spürte, als sie begann sich auf und ab zu bewegen, wie er in ihren Körper glitt. Er hatte nicht mehr das Gefühl in ihr zu stecken, sondern in sich selbst. Er spürte, wie er in Lunas Körper war; so als wäre es sein eigener. Luna empfand ebenso. Das war eine Erfahrung, die er wohl mit niemand anderem jemals machen würde, dachte Harry und wünschte sich, der Moment würde nie vergehen.

Er glitt wieder in seinen Körper und beherrschte sich mit ganzer Kraft. Gemeinsam kamen sie dem Höhepunkt näher und ließen sich den Namen des anderen rufend auf das Bett fallen. Voll von Schweiß bedeckt, küssten sich Luna und Harry und schliefen glücklich ein.

\end{abAchtzehn}

\begin{safedivide}
\fskdivider
\end{safedivide}

Währenddessen ging es bei Philip weiter.

Madame Pomfrey hatte die letzten Tage viele Stunden damit verbracht, Bücher zu lesen. Sie frischte ihre Kenntnisse in Muggelmedizin auf, die sie nach ihrer Ausbildung als Medi-Hexe erlangt hatte. Sie hatte als angelernte Hilfskraft in einem Krankenhaus viel über Muggelmedizin gelernt. Dies kam ihr jetzt zugute. Philip saß auf einem Holzstuhl mit Lehne, der mit einem Kissen auf der Sitzfläche und einer dicken Wolldecke an der Lehne belegt war. Er saß bequem. Vor ihm saß Madame Pomfrey und erklärte ihm, wie der Vorgang ablief. Professor Elber stand daneben und assistierte.

\enquote{Also, Mister Allman}, begann sie. \enquote{Die Prozedur geht folgendermaßen: Während des Ablaufes werden Sie nichts sehen, denn ich muss leider Ihr anderes Auge transparent machen, damit ich die zweite Augenhöhle nachbilden kann. Die zweite Augenhöhle wird dann jucken. Dieses Jucken wird noch ein paar Stunden anhalten. Es ist wichtig, dass Sie den Reiz ignorieren. Sie dürfen nicht kratzen. Sie bekommen ein kühlendes Tuch in ihre Augenhöhle, um den Reiz zu lindern. Darüber lege ich Ihnen eine Augenklappe, die Sie bitte den restlichen Tag tragen werden. Kommen Sie morgen gleich vor dem Frühstück zu mir. Dann werde ich ihnen die Augenklappe kurz entfernen und das Kühlpack erneuern. Nach dem Frühstück geht es dann zu einem Arzt. Ich werde Sie begleiten.}

\enquote{Kommt Professor Elber mit?}

Dieser schüttelte den Kopf.

\enquote{Sind Sie bereit?}, fragte Madame Pomfrey. Philip nickte und sie begann.  Das kleine kühlende Kügelchen und die Augenklappe lagen bereit. Sie begann, indem sie einen Transparenzzauber auf das intakte Auge legte. Dann besah sie sich die intakte Augenhöhle lange und genau. Jeden Schritt teilte sie Philip mit, damit er beruhigt war, denn er sah nichts mehr. Auf der anderen Seite bildete sie mit ihrem Zauberstab die Augenhöhle nach, so wie sie sein sollte. Nur der hintere Teil der Höhle war nicht so tief, da die Verbindung zum Sehnerv wieder hergestellt werden musste, wenn das Auge fertig war. Zuletzt legte sie das kühlende Kügelchen aus Stoff in die neue Augenhöhle. Nachdem sie den Transparenzzauber entfernt hatte, legte sie ihm die Augenklappe an und Philip sah wieder etwas. Morgen stand der Termin bei dem Squib-Arzt an. Dieser würde eine Gewebeprobe nehmen, um das Auge zu züchten.

\trenn

\enquote{Professor Snape?}, rief Harry durch das Klassenzimmer.

\enquote{Büro. Erste Tür links}, kam die Antwort.

Harry betrat das Büro und suchte die erste Tür auf der linken Seite. Sie war ganz hinten. Er betrat einen Raum mit allerlei Trankzutaten und Kesseln. Unter einigen brannte ein Feuer, andere waren mit einer Flüssigkeit gefüllt. An den Wänden standen Regale mit unzähligen, fein sortierten Trankzutaten. Flüssige und feste Zutaten standen in Gläsern oder Phiolen da. Alle waren sauber beschriftet in Snapes üblicher Handschrift. Harry stieg ein herber Duft in die Nase. Ein Geruch der gerade von einem der Kessel ausging.

\enquote{Was brauen Sie?}, fragte Harry seinen Lehrer.

\enquote{Einen Ihrer Tränke. Woher haben Sie überhaupt die Rezepte? Sie sind scheinbar sehr alt. \gst Und danke, für die Zutaten.}

Das hatte Harry jetzt nicht erwartet. Ein Lob von Snape. Zwar ein kurzes und knappes, aber dennoch ein Lob. Er dachte kurz nach und antwortete knapp: \enquote{Sie wissen, was es ist?}

Snape nickte. \enquote{Große Schlange.}

\enquote{Basilisk}, antwortete Harry, leicht amüsiert über Snapes Antwort. Denn ein Basilisk unterschied sich doch sehr von einer Schlange.

\enquote{Große Schlange, wie ich sagte. Aber woher haben Sie die Sachen?}

\enquote{Kammer des Schreckens} antwortete Harry knapp. \enquote{Ich war noch einmal dort.}

\enquote{Wie kamen Sie auf diese Idee?}

Harry schwieg. Dann sagte er: \enquote{Finden Sie es während unserer Stunden heraus. Das macht die Sache interessanter. Für Sie, weil Sie es wissen wollen, und für mich, weil ich es nicht unbedingt sagen möchte.}

Snape nickte und meinte dann: \enquote{Ich brauche noch fünf Minuten. Der Trank ist gleich fertig. Ich denke, dass Sie heute und morgen wegen des misslungenen Tranks \gst dafür übrigens die abgezogenen Punkte retour und noch zehn darauf \gst Nachsitzen werden. Beim nächsten Mal behalte ich Sie die ganze Stunde im Auge, was Sie wieder zweimal Nachsitzen kosten wird. Dann sehen wir weiter. Sie sollten ihre Braukünste dann etwas verbessert haben. Ich nehme dann den Trank der lebenden Toten. Wir werden ihr Rezept brauen und nicht das aus dem Schulbuch. Ich erwarte eine gute Leistung. Sie werden keine Punkte im Unterricht dafür bekommen, da Sie meine Unterlagen gesehen haben und sich darauf vorbereiten konnten. Das macht wieder einmal Nachsitzen. Dann haben wir schon fünf Termine, in denen ich Ihnen hoffentlich etwas beibringen konnte.}

Harry nickte und rührte den Kesselinhalt für seinen Professor dreimal um.

\enquote{Gut bemerkt}, meinte Professor Snape.

Harry nickte nur. Dann stellte er seine Tasche ab und wartete, bis Snape den Inhalt umgefüllt hatte. Dann leerte er den Kessel, löschte das Feuer und wischte sich seine Hände ab. Harry machte sich bereit für den Angriff. Snape zog seinen Zauberstab und sprach: \zauber{Legilimens!}

Harry konzentrierte sich und versuchte seinen Geist zu entleeren. Die erste halbe Minute schaffte er es ganz gut und konzentrierte sich auf einen runden Raum mit vielen Türen. Denselben Raum, den er im Ministerium gesehen hatte. Doch dann zog es ihn zu einer der Türen, sie öffnete sich und er schritt hindurch. Er stand in der Kammer des Schreckens. Der Basilisk war nicht mehr dort. Harry hatte keine Zeit mehr, sich darüber zu wundern, denn die Verbindung brach ab.

Er hatte leichte Kopfschmerzen und setzte sich auf einen Stuhl. Snape nahm ebenfalls Platz und sah ihn an.

\enquote{Wie sind Sie so weit gekommen?}, fragte Snape.

\enquote{Trotz allem, was letztes Jahr passiert ist, habe ich bedingt weiter gemacht. Zwar nicht oft, aber wenn, dann habe ich vor dem Einschlafen immer gedacht, dass ich in einem leeren schwarzen Raum in einem Bett liege und träume. Mein Traum war, dass ich in einem leeren schwarzen Raum liege, schlafe und träume. Mein Traum war\abs und so weiter.}

\enquote{Interessante Technik. Davon habe ich noch nie gehört. Es scheint aber, wenn Sie sonst nichts unternommen haben, dass Sie dann damit erfolgreich waren.}

Harry nickte. \enquote{Ich habe nach dem Ende der ersten \accentuate{Runde} noch etwas gelesen. Über Geistentleerung. Leider hat mir das nicht viel weitergeholfen, bin dann aber eher durch Zufall auf eine Beruhigungs-Technik gestoßen, falls man aufgewühlt sein sollte. Es war ein Medizinbuch. Ich habe mir nichts dabei gedacht, aber, als ich Probleme beim Einschlafen hatte, es doch ausprobiert. Am nächsten Tag merkte ich, dass ich besser geschlafen hatte. Ich hatte die ganzen Ferien über Zeit, diese Art für mich zu perfektionieren. Hier in der Schule ist sie wieder etwas in den Hintergrund gerückt, da ich hier keine Schlafprobleme habe. Ich habe erst vor zwei Wochen wieder damit angefangen. Zusätzlich habe ich mir überlegt, wie ich bei Binns Unterricht nicht einschlafe. Doch das hat nicht so gut geklappt, aber das Resultat um etwa zehn Minuten verzögert.}

Snape nickte. \enquote{Bereit?}, und fing wieder an. \zauber{Legilimens!}

Wieder schaffte es Harry, sich zu konzentrieren. Wieder stand er in dem runden Raum und er drehte sich langsam um seine eigene Achse. So, als ob er die richtige Tür suchen würde. Nach knapp einer Minute wurde er wieder durch einen Drang zu einer der Türen gezogen. Wider schritt er durch die sich öffnende Tür und stand wieder in der Kammer des Schreckens. Dieses Mal vor Salazar Slytherins Statue. Harry dachte an ein blaues Auto, um seine Gedanken abzulenken. Doch das Einzige was passierte war, dass sich Salazars Marmorstatue blau färbte. Er spürte abermals den Drang sich zu bewegen, blieb aber standhaft. Doch nach einigen Minuten ging er doch auf die Statue zu, die ihn begrüßte und sich ebenfalls zu bewegen schien und auf ihn zulief. Dann brach die Verbindung wieder ab.

\enquote{Enorm, für den ersten Versuch. Sie sind noch sehr schwach und der Dunkle Lord würde Ihre Barriere schneller durchbrechen. Ehrlich gesagt ich auch, wenn ich möchte. Aber sei’s drum. Ich möchte, dass Sie sich eine andere Technik zu Gemüt führen. Mal sehen, ob die bei Ihnen anschlägt, oder ob die normalen Wege bei Ihnen nicht zum Erfolg führen und Sie sich ihren eigenen Weg suchen müssen. Ich möchte, dass Sie vor dem Schlafengehen und auch sonst bei jeder Gelegenheit ein Lied in Gedanken singen. Das müssen Sie immer auf Abruf haben, damit der Angreifer nur das hört. Sie müssen sich darauf konzentrieren nur an dieses Lied denken zu können. Versuchen Sie es gleich heute Abend. Ich möchte sehen, ob es morgen einen Effekt hat. \gst Wer war übrigens die Staute? Sie wirkte irgendwie fehl am Platz. Und was war das für ein Raum, den Sie mir zeigten, den ich gesehen habe?} Die letzten Fragen kamen ziemlich unsicher von Professor Snape. Er wusste wohl nicht, ob er reale Bilder sah, oder von Harry inszenierte, war aber doch eher der Meinung, dass sie real waren.

\enquote{Wollen Sie es wissen, oder wollen Sie es, wie von mir vorgeschlagen, herausfinden?}, fragte Harry.

\enquote{Ich werde darüber nachdenken. Wir haben morgen ja noch einmal einen Termin. Sie sind entlassen.}

Harry nickte. \enquote{Professor? Ich habe unter Aufsicht noch einmal den Trank von heute gebraut?}, fragte er nach.

Snape nickte und Harry verließ mit seiner Tasche den Tränkeraum, durchquerte Snapes Büro und ging durch den Gemeinschaftsraum in sein Zimmer, um sich schlafen zu legen.

Er dachte nach, welches Lied er sich suchen konnte. Musste es ein leichtes sein, damit es eingängig sein würde, oder musste es ein schweres sein, damit es verwirrend sein würde? Diverse Titel gingen ihm durch den Kopf.

Dann meldete sich Salazar: \stimme{Nimm ein leichtes. Das ist eingängig und so schaffst du es auch, es im Kanon mit dir selbst zu singen.}

Harry drängte sich ein sehr altes ihm unbekanntes Lied in seine Gedanken. \gedanke{Es muss von Salazar sein}, dachte er noch, sang es in seinen Gedanken und schlief ein.

Am selben Tag als Harry bei Snape war, ging Philip kurz vor dem Frühstück zu Madame Pomfrey, um seinen Verband prüfen zu lassen. Sie nahm den kleinen Wattebausch heraus und ersetzte ihn durch einen neuen. Dann schickte sie ihn zum Frühstück und bat ihn, danach gleich wiederzukommen.

Nachdem er wieder da war, reisten sie per Kamin in den Tropfenden Kessel. Madame Pomfrey hatte eine Ausnahmegenehmigung von Dumbledore erhalten, der ihr den Kamin für diese beiden Reisen extra frei schaltete. Im Tropfenden Kessel angekommen, begrüßten sie kurz Tom und verschwanden dann durch die Tür ins Muggellondon. Sie nahmen die U-Bahn zur entsprechenden Klinik und betraten nach kurzem Fußmarsch den Eingangsbereich. In Muggelkleidung, welche sie sich vor ihrer Abreise angezogen hatten, fielen beide nicht auf. So konnten sie sich unbemerkt bewegen. Philip, für den alles neu war, beherrschte sich und sah sich nur staunend um.

Dann saßen sie in einem Behandlungszimmer und warteten auf den Arzt. Madame Pomfrey erklärte Philip die verschiedensten Muggelerfindungen, soweit sie sich daran erinnern konnte. Dann ging die Tür auf und der Arzt kam herein.

\enquote{Doktor Carrow?}, fragte Madame Pomfrey nach.

Dieser nickte und setzte sich auf seinen Untersuchungsstuhl. \enquote{Poppy Pomfrey und Philip Allman, richtig?}, fragte er nach. Beide nickten. \enquote{Gut. Das Ganze geht recht schnell, wenn Sie Magie einsetzen. Dann können Sie das Auge schon in knappen zwei Stunden mitnehmen. Ich werde die Schwester rufen und dann machen wir eine Biopsie an Ihrem intakten Auge. Das heißt: Ich werde ihnen ein Betäubungsmittel in das Auge tropfen. Dann wird eine winzige Gewebeprobe entnommen und inkubiert. Dann werde ich Ihnen das Gegenmittel tropfen. Der ganze Vorgang dauert etwa fünf bis zehn Minuten. Danach wird das Auge mithilfe von Magie in einem Brutschrank herangezüchtet. \gst Die Schwester weiß übrigens nichts davon. Die weiß nur von einer Vorsorgeuntersuchung wegen des grauen Stars. Das ist eine Erkrankung der Augen, bei der man zunehmend einen Schleier vor den Augen bekommt.}

Philip nickte, sobald er etwas verstanden hatte. Dann rief der Doktor nach der Schwester, indem er auf einen Knopf an einer der Apparaturen drückte. \enquote{Schwester Margret, kommen Sie bitte in Behandlungszimmer vier. Augenbiopsie.} Nach knapp einer Minute kam die Schwester mit einer Nierenschale, zwei Kunststofffläschchen, einem Skalpell, sowie einer Lupe mit Kopfband und einem kleinen Glasgefäß mit Korkverschluss.

Philip musste sich auf die Liege hinter ihm legen und der Arzt tropfte einen Tropfen auf das Auge. Nach etwas mehr als zwanzig Sekunden meinte er: \enquote{Jetzt nicht mehr bewegen und versuchen Sie, das Blinzeln zu unterdrücken.} Er setzte die Lupe auf und nahm das Messer. Dann schabte er vorsichtig eine hauchdünne Schicht ab und legte diese in das Glasfläschchen, welches die Schwester sofort verschloss. Der Doktor wartete noch ein paar Sekunden, dann tropfte er das Gegenmittel auf das Auge. Philip konnte wieder blinzeln und ihm wurde ein kleines Papiertaschentuch gereicht, um seine Tränenflüssigkeit abzuwischen.

\enquote{Kommen Sie in zwei Stunden wieder}, sagte der Arzt, als er der Schwester das Glasfläschchen abnahm. \enquote{Dann machen wir die andere Untersuchung, weswegen Sie hier sind.} Dann ging er. Die Schwester folgte ihm.

\enquote{Welche andere Untersuchung?}, fragte Philip nach.

\enquote{Ich denke, er hat das nur zur Schwester gesagt, damit sie sich nicht wundert, wenn wir wieder kommen. Sie kann dann diesen Termin einplanen.}

Philip nickte und sie verließen das Krankenhaus. Madame Pomfrey zog eine Karte hervor und besah sich diese. Dann führte sie Philip in ein kleines Café, in dem es auch Eis gab. Der Becher war leider viel zu schnell geleert, deshalb schlenderten die beiden die Einkaufsstraße zuerst in eine Richtung und danach in der anderen Richtung.

Pünktlich nach zwei Stunden kamen sie wieder an der Anmeldung an und wurden abermals in den vierten Stock geschickt. Die Schwester erkannte die beiden wieder; ein Junge mit Augenklappe fiel halt auf; und bat sie gleich wieder in ein Behandlungszimmer. Nach knappen zwei Minuten kam der Arzt wieder.

\enquote{Danke, Schwester. Für diese Untersuchung brauche ich Sie nicht.}

\enquote{Dann werde ich weiter Ordnung im Archiv schaffen. Falls Sie mich brauchen sollten.} Dann verschwand sie.

Der Doktor schloss die Tür und stellte das bedeckte Auge im Glasfläschchen auf der Nierenschale auf dem Tisch ab. \enquote{Soll ich Sie alleine lassen?}, fragte er Madame Pomfrey, nachdem er das Tuch entfernt hatte und das Auge durch das Fläschchen zu sehen war.

\enquote{Nein. Sie können uns aber nachher zu einem Optiker schicken. Dann können wir gleich die Augen untersuchen lassen.} Doktor Carrow nickte und wartete. Madame Pomfrey nahm das Glasfläschchen und entkorkte es. \enquote{Sie kennen die Prozedur ja schon, Mister Allman.} Dieser nickte und sie begann. Wieder erklärte sie alles, was sie tat, da Philips Auge wieder einmal transparent war und er nichts mehr sah. Jetzt passte sie die Augenhöhle exakt dem an, was sie sah, und zusätzlich noch der Eigenart des neuen Auges. Dann setzte sie es ein und sprach einen normalen Anwachszauber für Pflanzen.

\enquote{Das juckt}, beklagte sich der junge Zauberer. \enquote{Fürchterlich.} Doch er widerstand dem Drang zu kratzen. Nach mehr als einer Minute hörte es auf.

Madame Pomfrey legte die Augenklappe wieder auf das Auge und löste den Zauber auf dem anderen. \enquote{Sehen Sie mich wie bisher?}, fragte sie. Philip nickte. \enquote{Und auf der anderen Seite? Irgend eine Veränderung?}

Er nickte. \enquote{Kleine, leicht rötlich schimmernde Punkte.} Madame Pomfrey nahm die Augenklappe ab. Philip blinzelte etwas und nahm nun deutlicher seine Umgebung wahr. Ihm wurde leicht schummerig.

\enquote{Halten Sie das andere Auge zu}, sagte Doktor Carrow. Philip tat dies und es wurde besser. \enquote{Sie müssen sich daran erst noch gewöhnen. Nach so einer langen Zeit mit nur einem Auge passt sich das Gehirn an. Nehmen Sie am Anfang die Augenklappe nur beim Essen oder Abends ab. Nicht während des Unterrichts. Wenn es dann besser wird, dann lassen Sie sie länger weg, bis Sie ganz auf sie verzichten können. Sie werden es selbstständig merken. \gst Decken Sie jetzt besser wieder ihr neues Auge zu, bis Sie im Fahrstuhl sind. Mein Kollege weiß beschied, zweiter Stock.} Dann stand er auf und ging zur Tür. Als Philip die Augenklappe wieder aufhatte, öffnete der Doktor die Tür und sagte noch: \enquote{Viel Glück}, bevor er verschwand.

\enquote{Danke}, riefen ihm beide hinterher.

Philip sah seine Krankenschwester an. \enquote{Was erwartet mich jetzt noch?}

\enquote{Nur eine einfache Augenuntersuchung. Es wird deine Sehstärke überprüft.}

Dann gingen beide hinaus und in Richtung Fahrstuhl.




\begin{kommentar}
Gleich am Anfang des Kapitels sagt Harrys Mutter zu ihm, dass er eine Schwester hätte haben können, wenn sie nicht von Voldemort umgebracht worden wären. Die Idee dazu habe ich aus einer anderen Fan-Geschichte, in der Harry durch einen Unfall Jahre in die Vergangenheit zurückgeworfen wurde. Dort lebte seine Mutter und er hatte eine Schwester. Zudem hat es JKR selbst einmal gesagt.
\end{kommentar}

\begin{kommentar}
Nachdem Harry aus der Kammer ein paar Zutaten vom Basilisken geholt hat, gibt es einen kurzen Schwenk zu Voldemort, der schweißgebadet aufwacht und sich danach Bellatrix ins Bett holt. Ein kleiner Seitenhieb auf eine meiner anderen Geschichten, 'Ihr größter Wunsch'.
\end{kommentar}

\begin{kommentar}
Poppy Pomfrey begleitet Phillip Allman zu einem Muggelarzt. Doktor Carrow. Eine nette Anspielung auf die Carrow-Zwillinge aus dem siebten Band.
\end{kommentar}
