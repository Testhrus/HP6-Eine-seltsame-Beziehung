\chapter{Abschiede}


Am nächsten Tag in der Früh fuhr wieder der Hogwarts-Express, um alle Schüler nach Hause zu bringen. Das schlechte Wetter war genauso schnell verschwunden, wie es gekommen war. Harry lief im Trainingsanzug einige Runden auf der sandigen Piste auf dem Quidditchgelände. Er fragte sich, warum er noch so kurz vor Schuljahres-Ende eine Trainingseinheit absolvieren sollte. Und vor allem, wer würde sein Gegner sein. Er hatte gerade eine erneute Runde beendet, als Dumbledore hereinkam. Harry stutzte, denn er hatte eine altertümliche Robe in Schwarz an. Er stellte sich in den Kreis, der die Arena abgrenzte, und wartete auf Harry.

\enquote{Wie jetzt?}, fragte dieser. \enquote{Du? \gst Sie?}

\enquote{Frederick weiß Bescheid.}

\enquote{Oh\abs okay. Du bist mein Gegner?}

\enquote{’N Problem damit?}

\enquote{Nein. Ich bin nur überrascht.} Harry ging auf den Kreis zu und atmete einmal kräftig durch. Dann trat er hinein.

Die beiden Kontrahenten verbeugten sich und der Kampf begann. Harry war zuerst zögerlich und blockte vorwiegend ab, anstatt anzugreifen. Doch nach und nach kamen ihm die vergangenen Ereignisse wieder in den Sinn. Der Kampf Voldemort mit Dumbledore, der Kampf Dumbledore mit Elber und schließlich die vielen Trainingseinheiten und die Erinnerungen von Salazar, die er ihm zeigte.

Harry entschied sich jetzt aktiver zu sein. Zug um Zug griff er mehr und stärker an. Solange, bis Dumbledore ihm einen mächtigen Zauber schickte, den er mit einem Schild blocken konnte. Beständig floss die magische Energie aus Dumbledores Zauberstab heraus und auf Harrys Schild zu. So konnte er natürlich nicht mehr angreifen. Doch er hatte eine Idee. Er nahm seine andere Hand und hielt sie so, als ob er etwas blocken wollte. Dann führte er seinen Zauberstab auf seine Hand und diese begann kurz zu leuchten. Der Zauber wurde übertragen und Harry konnte nun angreifen, obwohl er einen Schild hatte.

Dumbledore war darüber so erstaunt, dass er seinen Stab nicht mehr halten konnte. Er flog auf Harry zu, der ihn zielsicher wie ein Sucher fing. Er musste sich ein Grinsen verkneifen, da er Dumbledore besiegt hatte. Zwar zweifelte er daran, dass er in einem echten Kampf eine Chance gehabt hätte, aber dieser Überraschungsmoment war ihm doch gelungen.

Harry verbeugte sich abermals und ging dann auf Dumbledore zu. \enquote{Ich glaube, das ist deiner}, sagte er und reicht ihm seinen Stab. Dumbledore lächelte und klopfte Harry anerkennend auf die Schulter. Schwatzend liefen sie nebeneinander zurück zum Schloss und ließen den Kampf noch einmal Revue passieren.

\trenn

Es war der letzte Abend im Schloss. Die Halle war wie immer voll und alle Schüler aßen zu Abend. Nachdem das Essen wieder von der Tafel verschwunden war, wollte gerade Dumbledore seine Abschlussrede halten, doch Professor Elber stand bereits und hielt sie seinerseits.

\enquote{Ich danke Ihnen allen, dass ich dieses Schuljahr hier auf Hogwarts verbringen durfte. Dass ich Sie am Jahresanfang verschreckt habe, tut mir leid. Aber dafür haben Sie ja dieses tolle Konzert als Entschädigung erhalten. Ich möchte Ihnen aber noch etwas auf den Weg mitgeben, etwas, dass Sie immer an Ihre Schulzeit erinnert, etwas, dass Ihnen sagt: \inner{Übe, und du wirst ein wahrer Meister.} Zusammen mit ein paar Schülern habe ich eine kleine Demonstration vorbereitet. Wenn diese einmal zu mir kommen würden?}

Es dauert wenige Sekunden, als alle gleichzeitig vor ihm erschienen. Aus dem Nichts. Sie waren lediglich von ihren Bänken aufgestanden, hatten sich einen Schritt in den Gang entfernt und apparierten scheinbar nach vorne, vor den Lehrertisch. Nun standen Draco Malfoy, Hermine Granger, Ronald Weasley, Susan Bones und Ernie McMillian vorne. Das löste tumultartige Zustände im Raum aus. Die Lehrer standen fassungslos auf; von überall hörte man: \enquote{Das ist unmöglich. Man kann im Schloss und auf den Ländereien nicht apparieren.}

Es verstrichen einige Sekunden und alle schauten wie gebannt zu den Lehrern nach vorne, die genauso ratlos aussahen. Nur einer stand da und rührte sich nicht. Snape. Dumbledore sah mit kindlichem Interesse zu. Nachdem er den ersten Schreck überwunden hatte, erinnerte er sich an das, was ihm sein Kollege erzählte. \accentuate{Virtuelles apparieren.} Dann musste er ein Schmunzeln unterdrücken.

\enquote{Die Schüler hier sind natürlich nicht wirklich appariert. Es sah nur so aus. Sie haben ein Trugbild von sich projiziert. In Wahrheit stehen sie immer noch an ihren alten Stellen. Ihre Mitschüler mögen es bitte einmal probieren, sie anzufassen.}

Und tatsächlich. An den Stellen, an denen Ron und Hermine verschwunden waren, spürte man immer noch einen Widerstand. Und auch die Schüler der anderen Häuser konnten die Körper ihrer Kameraden spüren.

\enquote{Ich hoffe, diese kleine Demonstration bleibt Ihnen noch eine Weile in Erinnerung. Vielleicht sehen wir uns irgendwann noch einmal. Bis dahin, lernen Sie fleißig und üben Sie. Denn das Dunkle wartet nicht, bis sie gerüstet sind. Es ist da draußen und wartet nur auf einen passenden Zeitpunkt.}

Dann öffneten sich die Türen der Großen Halle und die Schüler begannen, sich für den nächsten Morgen vorzubereiten, den Tag der großen Abreise.

Die Sonne ging langsam unter und der Himmel begann sich rot zu färben. Harry stand im Torbogen zum Eingang von Hogwarts und sah nach draußen. Von hinten kamen Professor Elber und Professor Dumbledore näher. Er drehte sich nicht um, hörte aber deren Unterhaltung mit.

\enquote{Also, Albus, es war schön hier. Wir sehen uns.} Harry hörte, wie sich die zwei die Hände gaben und sie sich schüttelten.

Dann wurde er durch eine andere Unterhaltung abgelenkt.

\enquote{Kann ich die Ferien über bei dir bleiben, Astoria?}, fragte Pansy ihre Mitschülerin. \enquote{Ich habe gerade einen Brief von meinen Eltern erhalten}, schluchzte sie weiter. \enquote{Sie haben mich hinausgeworfen, weil ich Dracos Freundin bin. Ich kann nirgendwo hin.}

\enquote{Deine Großeltern, oder andere Verwandte?}, fragte sie nach.

\enquote{Väterlicherseits sind sie schon im ersten Krieg gestorben, mütterlicherseits kenne ich sie nicht. Ich habe keine Ahnung, ob noch jemand lebt. Sonstige Verwandte sind mir nicht bekannt. \gst Doch, ein Cousin in Australien. Aber der ist ständig unterwegs.}

\enquote{Tut mir leid, Pansy, aber wir sind die erste Woche in Frankreich, bei meinem Onkel.}

\enquote{Der, den ihr nicht leiden könnt?}

\enquote{Meine Mutter und mein Vater können ihn nicht leiden. Ich finde ihn gar nicht mal so schlecht. Aber meine Eltern sagten sich: \inner{Er bezahlt uns immerhin die Reise und die Unterkunft. Und eine Woche Frankreich. \gst Paris sehen und so. \gst Das lassen wir uns nicht entgehen.} Danach werden wir nach Amerika gehen.} Harry drehte sich jetzt um und sah Pansy richtig bedrückt. \enquote{Was ist mit Draco?}

\enquote{Vergiss es. Sein Vater\abs} Sie sprach nicht mehr weiter.

Astoria nickte nur. \enquote{Und andere Mitschüler?}

\enquote{Sieht schlecht aus. Ich habe noch Millicent und Lemia nicht gefragt. Aber ich habe kaum Hoffnung. Vermutlich muss ich unter irgendeiner Brücke schlafen.}

Harry hatte plötzlich Mitleid mit Pansy und fühlte mit ihr. Dann konnte er nicht mehr zu ihr sehen und sah zu Dumbledore und Elber, dessen besorgtes Gesicht auf Pansy gerichtet war. Als sich Pansy ihm zu wandte, blickte er schnell zu Dumbledore. Harry drehte sich wieder um, sah nach draußen und genoss weiterhin das Wetter.

Dann spürte er eine Hand auf seiner Schulter. \enquote{Harry, alles Gute weiterhin in der Schule.}

\enquote{Danke, Professor}, sagte Harry und lächelte. Professor Elber ging an ihm vorbei und lief den Weg nach Hogsmeade hinunter. In der Hand hatte er einen kleinen Koffer, der Harry an eine Aktentasche erinnerte, und einen Regenschirm, der so schmal wie ein Spazierstock war. Harry sah ihm nach.

\enquote{Professor?}, fragte er Dumbledore. \enquote{Haben Sie schon einen neuen Lehrer für nächstes Jahr?}

\enquote{Ja.}

\enquote{Professor?}, fragte er Dumbledore.

\enquote{Ja, Harry}, gab Dumbledore zurück.

\enquote{Woher kennen Sie Professor Elber?}, fragte Harry.

Dumbledore schaute nun ebenfalls den Pfad entlang hinunter. Nach einer Weile fing er zu erzählen an.

\enquote{Weißt du, Harry, bevor Voldemort deine Eltern ermordet hat und verschwunden ist, gab es einige Prozesse gegen Todesser. Ich wohnte damals allen Verhandlungen als Zuhörer bei. Mit Ausnahme der von Snape und Karkaroff. Aber Karkaroffs Verhandlung hast du bereits gesehen.} Harry nickte. \enquote{Auf jeden Fall, als es dann zu den Verurteilungen kam, fragte der zuständige Vorsitzende die Jury. \inner{Irgendjemand für eine Verurteilung?} Und Professor Elber war der Einzige, der seine Hand erhob. Er saß in der ersten Reihe. Die anderen aus der Jury saßen alle hinter ihm. Er konnte sie nicht sehen. Der Vorsitzende fragte weiter: \inner{Irgendjemand für einen Freispruch?} Keiner hob die Hand. \inner{Irgendwelche Enthaltungen?}, fragte der Vorsitzende. Alle Anderen hoben die Hand.}

Harry sah Dumbledore erstaunt an.

\enquote{Ich muss dir nicht sagen, dass die Todesser natürlich heftig protestierten. Aber sie waren verurteilt worden. Einer hatte damals Professor Elber geschworen, dass der Dunkle Lord ihn persönlich zur Strecke bringen würde. Das hat ihn eigenartigerweise nicht besonders interessiert. Er stand auf und sagte klar und deutlich: \inner{Wenn Lord Voldemort etwas von mir will, dann soll er selber kommen.} Danach setze er sich wieder.}

Harry sah wieder Richtung Pfad.

\enquote{So ist es bei jeder Verhandlung gelaufen. Er war der Einzige, der für Schuldig gestimmt hat. Alle Anderen enthielten sich.} Jetzt musste Dumbledore lachen. \enquote{Der Vorsitzende hatte jedes Mal danach gesagt: \inner{Einstimmig verurteilt.}}

Dumbledore drehte sich jetzt zu Harry. \enquote{Er ist nach jeder Verhandlung immer gleich verschwunden. Durch ein paar Kontakte konnte ich schließlich erfahren, wo er sich eingemietet hatte. Eines Abends bin ich dann zu ihm gegangen. Ich wollte ihm gratulieren, dass er immer dafür war. Als ich dann im Dunklen die Gasse entlang ging, sah ich von weitem einen Fremden in das Haus gehen. Ich dachte schon ein Todesser und beschleunigte meine Schritte. Aus einem Fenster sah ich grüne Blitze. Danach tat es einen Schlag und der Fremde brach durch die Haustür. Professor Elber kam mit gezogenem Zauberstab heraus und sagte zu ihm etwas in der Art: \enquote{Wenn Voldemort etwas von mir will, dann soll er gefälligst selber kommen.} Das hatte mich beeindruckt.}

Dumbledore drehte sich nun um und lief langsam in das Innere des Schlosses, da es durch aufkommende Wolken kälter wurde. Harry lief neben ihm her.

\enquote{Er ging danach zurück und die Haustür nahm ihren früheren Platz ein. Als ich dann klopfte, musterte er mich eine Zeit lang, bis er mich hereinbat. Seit der Zeit kennen wir uns. Und ich konnte ihn endlich überreden, hier zu unterrichten. Das wollte ich schon letztes Jahr. Aber du weißt ja selber.}

\enquote{Umbridge}, entgegnete ihm Harry. Dumbledore nickte. Sie betraten die Große Halle und Dumbledore setzte sich Harry gegenüber an seinen Haustisch. \enquote{Meinst du, dass er Nicolas Flamel kennt?}, fragte Harry seinen Schulleiter.

\enquote{Wie kommst du darauf Harry?}

\enquote{Weil er mir gegenüber einige Andeutungen gemacht hat, beziehungsweise er sich versprochen hat, dass mich das vermuten lässt. Zudem hat er dich unten in der Kammer ganz schön schwitzen lassen. Er sah keinen Zauber als ein Problem an. Er hat uns über die unverzeihlichen Flüche aufgeklärt, als wäre es das Normalste der Welt. Und er hat mir Sachen beigebracht, die weit über das hinausgehen, was man hier normalerweise lernt.} Erst bei seinem letzten Satz merkte er, was er gesagt hatte. \enquote{Entschuldigung, Albus.}

Doch Dumbledore lachte. \enquote{Nein Harry, mir ist nicht bekannt, dass er Nicolas und Perenelle kennt. Wie kommst du nur auf diesen Gedanken?}

\enquote{So wie er auftritt und vor allem sein Wissen über die Magie. Vor allem, wie er Bellatrix behandelt hat.}

Dumbledore, der sich in der Zwischenzeit in einen Hähnchenflügel biss, legte ihn beiseite, schluckte seinen Bissen runter und fragte dann: \enquote{Wie meinst du das?}

\enquote{Na ja, ich hatte einen Traum, oder eine Vision. In der spielte er mit Lucius Malfoy Schach, als Draco ganz aufgeregt durch das halbe Schloss schrie. Sofort ist er in den Saal gesprungen und sah, wie Bellatrix Dracos Schwester Tamara mit dem Cruciatus-Fluch belegte.}

Dumbledores Augen wurden größer.

\enquote{Er entwaffnete sie mit einer Handbewegung und sagte Draco, er solle seine Schwester nach oben tragen und dort warten. Als sie alleine waren, versuchte Bellatrix ihn mit dem Cruciatus-Fluch zu belegen, da er sich kurz umdrehte. Er zuckte nur kurz zusammen und entwaffnete sie abermals. Dann dachte ich, dass seine Haare weiß wurden und länger, seine Augen rot und seine Fingernägel leicht wuchsen. Dann zuckten aus seinen Fingern blau-violette Blitze, die er auf Bellatrix schleuderte. Nachdem er sie gefoltert oder bestraft hatte, während dessen ich dachte, Teile ihrer Haut würden verbrennen, hörte er auf. Er nahm seinen Zauberstab heraus und ließ sie wie eine bewusstlose Puppe in einen Sessel schweben und legte einen Finger auf ihre Stirn. Sie sah danach irgendwie anders aus und er sagte so etwas wie, dass sie eine halbe Stunde für sich hätte.}

Dumbledore sah in ausdruckslos an. Er biss wieder in seinen Flügel, kaute langsam und blickte nachdenklich Harry an.

\enquote{Vielleicht\abs hat er einen eigenen Stein. \gst Wie alt schätzt du ihn?}

\enquote{Na ja, er scheint zumindest Schüler zu kennen, die ungefähr dreißig Jahre vor dir zur Schule gingen. Und so wie er immer darüber erzählt, als ob er selbst dabei gewesen wäre. Ich denke, er ist mindestens ein paar hundert Jahre alt.} Jetzt biss auch Harry in einen Chicken-Wing. \enquote{Darf ich Ihnen eine etwas eigenartige Frage stellen, Professor? Und bitte, fragen Sie mich nicht, wie ich darauf komme}, fragte Harry, nachdem sich die Schüler in ihrer Umgebung mehrten und sie nicht mehr wirklich alleine waren.

\enquote{Sicher Harry. Ich halte es da wie mein Kollege. Du darfst mich alles Fragen, aber du erhältst nicht immer eine Antwort.}

\enquote{Es geht um Geister.}

\enquote{Ah ja.}

\enquote{Gibt es gewisse Zauber \gst oder Umstände \gst wissen Sie, normalerweise kann man ja durch einen Geist durchgehen, seine Hand in ihn halten \gst Aber gibt es eine Möglichkeit, dass man einen Geist anfassen kann, dass er zeitlich begrenzt anfassbar, also materiell, wird?}

Dumbledore brach seinen Bissen, den er nehmen wollte, ab und sah Harry an. Dieser wurde rot.

\enquote{Ist dir das schon mal passiert?}, fragte er nach.

\enquote{Ich sagte doch: \inner{Fragen Sie mich bitte nicht}}, gab er ausweichend zurück.

Dumbledore überlegte. \enquote{Nun ja, dass Geister scheinbar materiell werden können, davon habe ich schon einmal gehört. Es bedarf dazu, soweit ich mich erinnern kann, sehr starker Gefühlsregungen. Ein einfaches Verlangen reicht dazu nicht aus.}

Harry nickte. \enquote{Wenn also ein Geist erregt ist, im positiven oder negativen Sinne, dann kann das passieren?}

\enquote{Ja. Das muss aber eine sehr starke Erregung sein.}

\enquote{Kann man mit Zaubern nachhelfen, also die Erregungsschwelle senken?} Harry kam sich langsam komisch vor, bei seinen Fragen.

Dumbledore dachte nach und sah gedankenverloren in die Runde. \enquote{Sir Nicolas, würden Sie mal herüber schweben?}

Der Geist schwebte zu den beiden und Dumbledore erzählte ihm kurz, was Harrys Frage war und wo beide jetzt standen.

Dieser kratzte sich am Ohr und dachte nach. \enquote{Bestimmte Zauber\abs}, murmelte er vor sich hin. Er rieb nun mit der anderen Hand an den inneren Konturen seines Ohres herum und sah auf den gedeckten Tisch. Er roch an einem einzelnen Pommes und kaute dann in der Luft.

Dann verschränkte er seine Finger ineinander, legte sie auf dem Tisch ab und legte die Daumen gegeneinander. Er sah zu Harry hoch und meinte dann: \enquote{Es gibt Zauber, die haben auf Geister durchaus eine Wirkung. Aber keine so starke, dass sie so etwas auslösen würden.} Er dachte kurz nach. \enquote{Eine Kombination von Zaubern könnte diesen Effekt etwas anheben \gst aber der Geist braucht noch immer eine gewisse Menge an Konzentration und Gefühlsregungen.} Er fuhr sich mit seinem Finger die Lippen entlang. \enquote{So etwas wie Ihr Zustand vor einigen Wochen könnte so etwas hervorrufen. Es ist eine Art der Gefühlsaufladung. Möglich, dass es im Zusammenhang mit Zaubern, die auf einem Ort wirken, auf Geister so eine Auswirkung hat. \gst Aber das ist nur eine Vermutung.} Ohne dass Dumbledore es merkte, zwinkerte er ihm zu.

Jetzt durchzuckten Harry mehrere Gedanken. Er erinnerte sich an Salazars Statue und wie er ihm die Möglichkeit zur Stofflichkeit gab. Ihm viel erst jetzt richtig auf, dass während des Geschlechtsverkehrs mit Myrte seine erste Ladung Sperma durch sie hindurch flog, dann aber ein feiner Nebel sich auf sie zu bewegte und so Myrte zu kurzer Stofflichkeit verhalf. \gedanke{Biologische Materie, na klar.} Am liebsten hätte er sich vor den Kopf geschlagen. Jetzt musste er leicht Schmunzeln. Es war so einfach. Geister konnten durch Übergießen mit biologisch angereicherten Flüssigkeiten und einer starken Gefühlsregung und unter Zuhilfenahme von Zaubern, die Gefühls-fördernd sind, für begrenzte Zeit Stofflichkeit erhalten.

Vor allem reizte ihn die Möglichkeit, seinen Verwandten Salazar einmal vorzustellen. Mit diesem Gedanken aß er weiter.

\trenn

Am nächsten Morgen wartete Harry auf seine Freunde um mit ihnen zum Zug zu laufen, als ein Hauself vor Pansy erschien und ihr einen Brief in die Hand drückte. \enquote{Hier Miss, für Sie.}

Pansy nahm den Brief entgegen und las ihn vor, da ihre Freundin ihr gegenüber stand und er nichts Privates zu enthalten schien. Harry konnte den Monolog mit anhören.

\begin{brief}
Liebe Pansy,

ich habe erfahren, dass dich deine Eltern hinausgeworfen haben.

Als dein Pate bin ich mir der Verantwortung bewusst, die ich jetzt auf mich nehme. Wenn du die Ferien woanders verbringen möchtest, dann sag meinem Elfen einfach Bescheid. Er wird dann wieder gehen. Falls du noch Optionen offen hast und lieber die Zugfahrt abwarten möchtest, dann kannst du das ebenso. Mein Elf wird dich dann am Bahnhof erwarten und deine Entscheidung entgegennehmen. Etwaige Fragen an mich können wir persönlich klären, wenn du deine Ferien bei mir verbringen möchtest.
\signumspace
Liebe Grüße

Dein Pate
\end{brief}

Pansy ließ den Brief in ihren Händen sinken und sah nachdenklich den Elfen an, der ihren Blick wartend erwiderte. \gedanke{Er steht einfach da und wartet. Keine Spur von Unterwürfigkeit. Er schaut mich selbstbewusst an und wartet.} \enquote{Ich warte noch die Zugfahrt ab. Bitte erwarte mich am Bahnhof und gehe davon aus, dass ich mitkomme. Allerdings kann sich während der Zugfahrt noch was ändern.}

Der Elf nickte und wartete noch einen kurzen Moment. Dann fragte er nach. \enquote{War’s das Miss?}

Pansy nickte und der Elf verschwand.

In der Zwischenzeit waren Harrys Freunde angekommen und sie liefen zusammen zum Zug. Er hing noch seinen Gedanken nach, bis sie endlich im Zug saßen.

Die Pfeife des Zuges pfiff und der Zug setzte sich in Bewegung. Kurz bevor der Zug losfuhr, poppte es und ein kleiner Elf tauchte im Abteil auf. Neugierig sah er Harry an. Der Zug fuhr bereits, war aber immer noch im Bahnhof. Ein weiterer Elf poppte auf dem Bahnsteig auf. Nachdem der kleine Elf seinen Vater entdeckt hatte, winkte er ihm aus dem Zug zu.

Harry sah kurz zwischen beiden hin und her. Dann öffnete er das Fenster und rief hinaus: \enquote{Soll ich Ihnen Ihren Sohn hinausgeben, oder holen Sie ihn in London ab? Ich passe dann so lange auf ihn auf, dass er keinen Unsinn macht.}

\enquote{Der Zug ist schon zu schnell. Das schaffen Sie nicht mehr. Ich hole ihn in London ab}, sagte der Elfenvater.

Harry nickte und schloss das Fenster wieder. Dann setzte er sich wieder und nahm den kleinen Elfen auf seinen Schoß, da er ihn wartend ansah.

Ginny saß neben Harry im Abteil und hatte ihre Hand auf seinem Fuß. Seine Hand lag darüber. Ihre Köpfe aneinander gelehnt sahen sie Ron und Hermine zu. Hermine hatte vor Aufregung gestern kaum geschlafen und lag mit ihrem Kopf in Rons Schoß. Verträumt fuhr er durch Hermines Haar. Neville und Luna saßen sich gegenüber an der Fensterreihe und sahen nach draußen auf den Bahnhof. Hagrid blickte noch durchs Fenster und winkte. Der Zug fuhr aus dem Bahnhof und um die erste Ecke. Heute Abend würden sie wieder Zuhause sein. \gedanke{Zu Hause}, dachte Harry. Die ganzen Ferien über wäre er wieder bei seinem Onkel und seiner Tante. Aber für den Moment lag es noch Stunden entfernt. Jetzt hatte er Ginny neben sich und war glücklich.

Er schloss die Augen und lauschte Lunas Geist. Er sah das Konzert vor sich, das er verpasst hatte, weil er in der Krankenstation war. Er war hinter der Bühne mit den Künstlern. Und er drehte sich und tanzte unbeschwert zur Musik. Er fühlte sich, als ob er es selber erlebt hatte, aber es waren doch nur Lunas Erlebnisse. Stumm lächelte er in sich hinein.

Nach diesem kurzen Überblick fing das Ereignis an.

\begin{rueckblick}
Professor Elber stand vor dem Lehrertisch und hielt wieder eine seiner ungeliebten Reden.

\enquote{Heute Abend ist das große Ereignis, das lediglich als \accentuate{Magie in Konzert} ausgeschrieben wurde. Hierbei handelt es sich, wie der Name schon sagt, um ein Konzert. Allerdings spielt nicht eine Band, sondern mehrere. Es ist eine Art Benefizveranstaltung, da Aufnahmen gemacht und verkauft werden. Diese Veranstaltung ist mit Technik der Muggel aufgemacht. Also keinerlei Zauberei, wenn ich bitten darf. \gst Da Sie das ganze Jahr über sehr gut mitgearbeitet haben und stetig Interesse gezeigt hatten, dachte ich mir, ich revanchiere mich damit. Es gaben mir sogar viele Schüler Rückmeldung, was sie vom Unterricht gehalten haben und was sie noch lernen möchten. Deswegen, als großes Danke, habe ich dieses Ereignis, mit Erlaubnis unseres Schulleiters, organisiert. Außer den Schülern werden auch eine gewisse Anzahl an Muggeln dem Ereignis beiwohnen, Muggel, die Sie kennen. Allgemein gesagt, Leute, die schon Weihnachten hier waren. \gst Die \accentuate{Veranstalter} sind alles Muggel, die hinterher ein paar Details vergessen werden. Die werden sich aber an einen großen Teil erinnern. Dafür sollten eigentlich Auroren sorgen. Da diese aber}, er pausierte kurz, \enquote{nicht zur Verfügung stehen, wird das Lehrerkollegium diese Aufgabe übernehmen. Passen Sie trotzdem auf, was Sie in Gegenwart des Personals sagen. Die Festivalteilnehmer dürften unproblematischer sein. Alle haben in gewisser Weise mit der magischen Gemeinschaft zu tun. In acht Stunden geht es los. Eine Stunde vorher ist Einlass.}

Jetzt begann aufgeregtes Geschnatter in der Großen Halle. Unzählige Diskussionen entstanden in der Großen Halle. Harry bekam viele Diskussionsfetzen mit. Mehr, als wenn er selbst in der Halle gesessen hätte. Diese Erfahrung war für ihn neu.
\end{rueckblick}

Von dem quengelnden Elfen wieder in die Realität zurückgekehrt, überlegte er, was er dem Elfen als Beschäftigung geben könnte. Er zauberte eine Rassel hervor, die den kleinen für eine gute Stunde beschäftigte.

Die Rassel wirkte auf ihn beruhigend, sodass Harry wieder seine Augen schloss und seinen Kopf zurücklehnte. Luna übersprang die sieben Stunden und fing wieder an, als es auf das Quidditch-Feld ging.

\begin{rueckblick}
Ein leichtes Kribbeln war zu spüren, als Luna durch den Eingang auf das Feld trat. Auf beiden schmalen Seiten der Arena waren Bühnen aufgebaut. In etwa einen Meter und fünfzig standen Verstärker und Lautsprecher. Im hinteren Teil der Bühne war etwas zu sehen, was an Orgelpfeifen erinnerte. Sie suchte sich einen guten Platz in der Mitte, da Luna vermutete, dass auf beiden Seiten Auftritte waren. Ein Teil der Schüler ging näher an die Bühne heran. Luna sah sich um. Sie drehte sich einmal im Kreis. Es hatte sich außer den beiden Bühnen an den Stirnseiten nichts geändert. Vom Eingang her kam ihr Vater auf sie zu. Sie lächelte ihn an.
\end{rueckblick}

Dann gab es einen Schnitt. Die Ansicht wechselte und Professor Dumbledore stand auf der hinteren Bühne und fing an zu erzählen. Vermutlich war die Unterhaltung zu privat.

\begin{rueckblick}
\enquote{Ich freue mich, dass wieder einmal viele Eltern, Verwandte und Bekannte hierhergekommen sind. Erneut gibt es etwas hier auf diesem schönen Feld zu feiern. Auch wenn es abseits gelegen ist. Aber das kann nur von Vorteil sein. Wir können nicht wegen Ruhestörung oder zu lauter Musik belästigt werden. Ich will euch nicht lange aufhalten. Ihr wartet sicher schon alle gespannt auf den ersten Gig. Damit gebe ich die Bühne frei. Hinter euch fängt nun die erste Gruppe an, während auf dieser Bühne aufgebaut wird. Die Lieder finden also abwechselnd auf beiden Bühnen statt. Viel Spaß.}

Dann begann die Musik und Luna drehte sich mit dem Rest der Konzertteilnehmer um.

\enquote{Hallo Freunde}, begann der Sänger der \accentuate{Wirbelnden Kröten}. \enquote{Wir freuen uns, dass wir zu so einem bedeutenden Ereignis eingeladen wurden. Deshalb haben wir jetzt erstmalig für euch unser neustes Lied. \gst Federleicht.}

Die Musik begann mit einem Schlagzeug, das das gesamte Lied über denselben eingängigen Bolero-artigen Takt vorgab. Doch dieses Lied wurde von der Gitarristin gesungen. Mit heller Stimme fing sie an.

\begin{lied}
Wie eine Feder

getragen durch Magie

schweb’ ich durch die Welten

doch ich finde nie

das, das was ich suche

liegt doch klar vor mir, doch mein Ziel erreiche

ich wohl doch nie\abs
\end{lied}

Das Lied war leicht, fand Luna. Sie sah zu ihrem Vater, der selig lächelnd der Musik zuhörte und einen Blick hatte, der scheinbar die Umgebung ausblendete und nur auf die Bühne gerichtet war.
\end{rueckblick}

Immer abwechselnd auf beiden Bühnen kamen die Lieder dran. Aber eines davon beeindruckte Harry besonders. Es wurde extra angekündigt, damit die Bühnen-Crew, die das gesamte Konzert mitfilmte, nicht verwundert schaute.

Harry warf einen Blick auf Frodo, der immer noch mit der Rassel spielte und ab und an zum Fenster auf die vorbeiziehende Landschaft schaute.

\begin{rueckblick}
\enquote{Das nächste Lied ist etwas Besonderes. Wie ich einigen von euch schon mitgeteilt hatte, testen wir heute unter realen Bedingungen ein neues Stück Technologie, das realistische Bilder projizieren kann.} Wieder kam ein Lied, das mit einem Schlagzeug anfing. Dann kam von unter der Bühne ein kreisrunder Stempel hervor mit einem kleinen Geschöpf darauf. Es war ein Medusoner. Mit hoher Stimmlage begann sie. \enquote{Das Lied nennt sich Stardust.} Das Schlagzeug begann und eine Gruppe weiterer Medusoner trat auf die Bühne. Jeder im Publikum wusste, dass die Ansprache von Dumbledore nur dazu diente, die Crew zu beruhigen.

\begin{lied}
We are waiting for the sun

We are waiting for the sun

to come out and play

all these rainy days

I getting all

we are waiting for the stars

we are waiting for the stars\abs
\end{lied}
\end{rueckblick}

Das hatte Harry beeindruckt. So ein kleines Wesen, dass sich traute, vor so einem großen Publikum aufzutreten. Die nachfolgenden Lieder hörte er überwiegend in seinem Geist, da er etwas Bewegung brauchte. Ginny unterhielt sich mittlerweile mit Hermine und Ron aß gerade. Luna schien Neville zu beobachten, der vor sich hin döste. Er wollte gerade umdrehen und Luna Bescheid geben, dass sie kurz auf Frodo aufpassen solle, als er ihn schon an seiner Hand spürte, wo er sich festhielt.

\enquote{Frodo mitgehen}, sagte der Elf, der seine Hand hielt und ihn fröhlich ansah.

Harry sah lächelnd zu ihm herab und nickte. \enquote{Also gut, gehen wir ein wenig.}

Durch seine Größe fiel Frodo niemandem auf, der in den Abteilen saß. Nur als er ein Großraum-Abteil durchquerte, erregten sie Aufmerksamkeit. Wieder einmal gab es Getuschel. Frodo störte das nicht und Harry ignorierte es. Er war das gewohnt. Als sie der Hexe mit dem Süßigkeiten-Wagen begegneten, kaufte Harry etwas für Frodo. Er war ein wenig wie ein kleines Kind.

Erst als er Katharina und Alina entdeckte, die gerade ihre Köpfe zusammen gesteckt hatten, realisierte er, dass er im Slytherin-Abteil angekommen war.

Beide sahen zu ihm auf und begrüßten ihn. \enquote{Hallo, Harry. Wen hast du denn mitgebracht?}

\enquote{Das hier ist Frodo. Er hat sich eingeschlichen. Ich passe auf ihn auf, bis ihn sein Vater in London abholt.}

Beide rückten etwas zusammen und machten Harry Platz. Er setzte sich auf die Gang-Seite und nahm Frodo, der ihn bettelnd ansah, auf seinen Schoß.

\enquote{Und was macht ihr?}, fragte Harry.

\enquote{Wir verabreden uns. Wir wollen uns in den Ferien mal treffen. Nachdem sie von der Krankenstation wieder gekommen war, habe ich ihr beim Schulstoff geholfen. Da haben wir uns angefreundet. Jetzt besuchen wir uns in den Ferien.}

Dann begann der Zug zu bremsen. Es war fast eine Notbremsung. Innerhalb hundertfünfzig Metern hielt er an. Katharina öffnete das Fenster und sah nach draußen. Vor den Schienen schien etwas zu liegen. Eine Sperre. In der Ferne sah man Dementoren auf den Zug zufliegen und eine Menge schwarzer Gestalten standen da und warteten.

Dann ertönte ein Pfiff und eine Menge Elfen tauchte auf und verteilte sich rings um den Zug. Sie warteten auf die Dementoren. Einige sahen sich um warfen ihren Blick auf den Zug und ihre Insassen. Eine der weiblichen Elfen entdeckte Harry und poppte in das Innere des Zuges.

\enquote{Mister Potter, Sir. Würden Sie bitte die Dementoren mit einem Patronus vertreiben? Um die anderen werden wir uns kümmern. Nur können wir nicht gleichzeitig gegen beide vorgehen.}

Harry verstand. \enquote{Passt ihr so lange auf Frodo auf?}, fragte er Katharina und Alina. Beide nickten. \enquote{Gut, ich komme mit.}

Die Elfe nahm ihn bei der Hand und poppte zurück zu den anderen.

\enquote{Mister Potter wird sich um die Dementoren kümmern}, sagte die Elfe.

Harry wartete, bis er die übliche Kälte spürte. Dann stelle er sich die schönsten Szenen vor. Er hätte erwartet, Luna zu sehen. Doch dann kam ihm Ginny in den Sinn und vor sein geistiges Auge. Er sah klar und deutlich die Szene im Bad vor sich, als sie über ihn stieg. Dann schwenkte sein Verstand zu Luna und anschließend zu Lavender. Und dann zu Myrte. Diese Kombination gab ihm die notwendige Kraft. Er zog seinen Zauberstab und heilt ihn nur noch nach oben. Sofort löste sich eine gewaltige blau schimmernde Welle und breitete sich kuppelartig aus. Welle um Welle trennte sich vom Zauberstab und näherte sich den Dementoren. Nach vergeblichen Versuchen durchzubrechen kehrten sie um und verließen den Schauplatz.

Die Elfe brachte Harry wieder in den Zug zurück, wo sofort Frodo wieder bei ihm war.

Die Angreifer, die bislang nur dastanden und hofften, dass die Dementoren die Zuginsassen so weit schwächten, dass sie nur noch den Zug durchstreifen mussten, waren für wenige Sekunden verwirrt. Dies nutzten die Elfen und fingen an, die ersten mit Seilen zu fesseln und am Apparieren zu hindern. Sie hatten rund ein Drittel gefesselt, als sich der Rest anfing zu wehren. Es war ein ungleicher Kampf. Die Angreifer hatten gegen die Elfen kaum eine Chance. Rund ein Drittel des Restes starb durch zurückprallende oder umgelenkte Flüche, oder weil die Elfen sich duckten oder zur Seite sprangen und der Zauber einen Kollegen traf.

Der Rest disapparierte. Die Elfen warteten noch einige Minuten. Dann begann ein Teil von ihnen die Schienen freizuräumen und scheinbar dem Zugführer Bescheid zu sagen. Bilbo tauchte auf und Harry freute sich, da er die Verantwortung für Frodo abgeben konnte.

\enquote{Danke, dass Sie auf meinen Sohn aufgepasst haben.}

\enquote{Keine Ursache, Mister Bilbo.}

Die Zugpfeife ertönte und Bilbo verabschiedete sich. \enquote{Wir sehen uns, Mister Potter.} Dann poppte er und war verschwunden.

Der Zug setzte sich wieder in Bewegung und Harry setzte sich neben Katharina.

\enquote{Wer wohl die Elfen geschickt hat?}, fragte Katharina.

\enquote{Weiß nicht}, meinte Harry und hob seine Schultern. \enquote{Vielleicht war es Professor Elber, ich bin mir aber nicht sicher.}

\enquote{Hast du wieder eine Freundin? Nachdem du dich ja von Luna und Pansy getrennt hast\abs}

\enquote{Ja.}

\enquote{Und wen?}

\enquote{Ginny.}

\enquote{Weasley?}

\enquote{Ja.}

\enquote{Und wie läuft es?}

\enquote{Ganz gut. Wir haben uns mehrmals ausgesprochen. Das liegt wohl daran, dass ich nach meinem dreiwöchigen Aufenthalt im Krankenflügel an Erinnerungslücken litt. Ich musste mir erst einmal sagen lassen, was ich vergessen habe. Deshalb habe ich mich mit Ginny unterhalten. Dadurch ist unsere Verbindung enger geworden, finde ich. Wir sind aneinander gewachsen.}

\enquote{Du meinst wohl ineinander, Potter}, meldete sich plötzlich Draco Malfoy. Bisher hatte er sich ruhig verhalten, aber er muss ja seiner Rolle gerecht werden.

\enquote{Du weißt doch, dass ich es manchmal mit den Worten nicht so genau nehme, Malfoy}, konterte er.

Das sollte ihm genug Zeit verschaffen, sich zu verabschieden.

\enquote{Wir sehen uns}, sagte er zu Katharina und Alina. Dann ging er zurück in sein Abteil.

\enquote{Wo ist der Kleine?}, fragte Luna sofort, da er ohne zurückkam.

\enquote{Sein Vater hat ihn abgeholt, als der Zug angehalten hat.}

\enquote{Was ist denn passiert? Wir haben nur mitbekommen, dass der Zug angehalten hat. Dann dauerte es etwas. Dann kamen Dementoren, die durch einen Patronus vertrieben wurden.} Harry zeigte auf sich und nickte. \enquote{Dann gab es einen ungleichen Kampf zwischen den Elfen und den Todessern? Oder was waren die? Dann ging es wieder weiter. Habe ich was vergessen?}

\enquote{Nein, Luna. Das war alles.}

\enquote{Dann ist ja gut.} Sie widmete sich wieder Neville, neben dem sie jetzt saß.

Ginny stand auf und deutete Harry an, er möge sich auf ihren Platz setzen. Dann setzte sie sich auf seinen Schoß.

Harry schloss wieder die Augen, nachdem Ginny sich an ihn geschmiegt hatte und lauschte Luna Erlebnissen.

\begin{rueckblick}
Dann wurde eine Wand am Bühnenende gesenkt und ein eigenartiges Instrument kam zum Vorschein. Ein alter Mann trat mit Anzug auf die Bühne. Er begann mit einer sehr kurzen Erklärung seines Instrumentes. \enquote{Dies hier ist ein Trautonium, entwickelt von Professor Trautwein um 1930. Das folgende Lied heißt \accentuate{Canon Caprice 2}.} Dann setzte er sich und begann zu spielen.\\
Es war ein eigenartiges Lied. Elektronische Musik hatte bisher keiner der Zauberer, die nicht bei Muggeln aufgewachsen waren, gehört. Und selbst die anderen hatten derartige Klänge noch nie gehört.
\end{rueckblick}

Der Zug hielt an und Harry stieg aus. Er verabschiedete sich von Ron und Hermine, begrüßte Mr. und Mrs. Weasley, blieb mit Ginny zurück und küsste sie zum Abschied noch einmal. Er sah Pansy, wie sie mit dem Hauselfen sprach, der sie auch schon vor Hogwarts angesprochen hatte, ihm ihre Hand gab und dann mit ihm verschwand. \gedanke{Sie geht wohl zu ihrem Paten}, dachte Harry. Dann durchschritt er die Absperrung zur Muggelwelt, wo ihn schon seine Tante erwartete. Harry sah seinen Onkel und auch seinen Cousin nicht. Sie nahm ihn mit zum Parkplatz wo er seinen Koffer einladen musste. Danach fuhr sie mit ihm schweigend nach Hause.

Etwas hatte sich mit seiner Tante verändert, dachte er noch, als er in seinem Bett im Ligusterweg 4 einschlief und zu träumen begann. Doch zuvor bekam er noch den Abschluss des Konzerts mit, den Luna ihm noch übersandt hatte und den er bis jetzt ignorierte.

\begin{rueckblick}
Seitlich durch den Haupteingang kam eine Gruppe von Dudelsack-Spielern. Sie fingen an mit einem Riff, den man eher von Gitarren gewohnt war.\\
Während die eine Hälfte den Rhythmus spielten, dudelte die andere Hälfte die Melodie.
\end{rueckblick}

Harry musste grinsen. Er kannte das Lied. \liedinline{Thunderstruck}. Auf dem Dudelsack hörte sich das einfach unglaublich an.

\trenn

Während Harrys Fahrt nach Hause, apparierte der Elf mit Pansy in ein kleines Anwesen. \enquote{Das hier ist ihr Zimmer, Miss}, sagte der Elf und zeigte auf die Tür vor ihnen. \enquote{Morgen früh wird sich ihr Pate um Sie kümmern. Ich wünsche Ihnen eine gute Nacht.} Dann verschwand der Elf. Pansy stand noch staunend vor der Tür und griff schließlich an die Klinke, um sich ihr Zimmer anzusehen. Sie ließ ihre kleine Tasche fallen, die sie noch in der Hand hielt, und ihr Mund klappte auf. Das war etwas ganz anderes als zu Hause, oder in Hogwarts. Zu Hause hatte sie ein mittelgroßes Zimmer mit einem Schreibtisch, einem großen Schrank für ihre Kleidung, ein Bett und ein großes Regal, in das sie ihre Schulbücher legte und andere Bücher zum Lesen oder Studieren hatte. In Hogwarts musste sie ihr Zimmer mit anderen Mitschülern teilen und konnte ihre Hausaufgaben im Gemeinschaftsraum machen.

Doch dieses Zimmer sah anders aus. Sie fand ein ähnliches Bett wie ihres in Hogwarts vor. Nur war es etwas größer. Das Zimmer war doppelt so breit und anderthalbmal so lang wie ihr Zimmer bei ihren Eltern. Eine ganze Wand bedeckte ein Regal, in dem nicht nur ihre gesamten Schulbücher lagen, sondern auch noch jede Menge andere Bücher, die sehr wichtig aussahen. In die Regale integriert war etwas, das nach einem Wandschrank aussah.

Sie betrat den Raum, ging auf den Schrank zu und öffnete ihn. Im Inneren waren ihre sämtlichen Kleidungsstücke, die sie bei ihren Eltern lassen musste. Betäubt durch die Eindrücke schloss sie die Schranktür wieder und sah ganz erstaunt auf den Elfen neben sich, er war aus dem Nichts aufgetaucht, der noch ein paar Bücher in das Regal stellte. Es waren Pansys Lieblingsbücher.

\enquote{Sind etwa meine gesamten Sachen aus meinem Zimmer hier?}, fragte Pansy ganz erstaunt.

\enquote{Dies \accentuate{hier ist} ihr Zimmer, Miss. Aber ja, alle Sachen aus ihrem \accentuate{alten} Zimmer sind nun hier.}

\enquote{Aber, wie bist du\abs}

\enquote{Der Elf der Familie Parkinson war sehr hilfsbereit. Er muss Sie sehr mögen.}

Pansy taumelte nach hinten und ließ sich auf ihr Bett fallen.

\enquote{Alles gut, Miss?}, fragte der Elf besorgt nach.

Pansy nickte nur, immer noch nicht fassend, was gerade passiert war. \gedanke{Meine Eltern haben mich hinausgeworfen. Mein Vater konnte mich ja noch nie richtig leiden. Aber jetzt bin ich bei meinem Paten, den ich nicht einmal kenne, und habe ein Zimmer, das schöner ist als alle Zimmer, die ich bisher gesehen habe. Es passt zu mir.} Der Elf war wieder verschwunden und Pansy zog sich aus und danach ihr Nachthemd an, sie verschwand noch einmal ins Bad und stieg danach ins Bett um grübelnd einzuschlafen.

Als sie am nächsten Tag erwachte, ging ihr erst einmal ein Gedanke durch den Kopf. \gedanke{Das war ein schöner Traum gestern. Ich habe in einem bequemen Bett in einem Zimmer\abs}, dann schlug sie die Augen auf und blickte sich um. \gedanke{Kein Traum. Ich bin wirklich hier.} Sie sah sich um und entdeckte ihren Hogwarts-Koffer. \gedanke{Habe ich den gestern gar nicht mehr gesehen, oder ist der in der Nacht hereingekommen?}, fragte sie sich. Unsicher und noch leicht schläfrig stand sie auf und ging zu ihrem Schrank. Dort waren nun mehr Kleider und Anziehsachen, als am Abend zuvor. Nachdenklich schloss sie den Schrank wieder und richtete sich für das Frühstück, das ihr Magen lautstark erwartete. Sie stieg die Treppe hinunter in das Erdgeschoss und lief die Spur aus einzelnen Blumenblättern nach. Es waren Lilienblüten. \enquote{Sind das nicht auch die Lieblingsblumen von Harrys Mutter?}, fragte sie sich. Dann hatte sie die Schiebetüre erreicht, die durch zwei auf zierlichen Holztischchen stehenden Blumenvasen eingerahmt waren und öffnete sie. Der Tisch war bereits gedeckt und Pansy blickte durch die Fensterwand auf der gegenüberliegenden Seite des Raumes hinaus in den Garten.

Dann setzte sie sich im weiß gestrichenen Zimmer auf einen Stuhl und begann sich ihren Teller zu beladen. \enquote{Nur meine Lieblingssachen, die ich gerne Frühstücke}, ging es ihr erneut durch den Kopf. \enquote{Das Himmelbett hatte dieselbe Zeichnung auf dem Baldachin wie in der Schule. Das kann doch kein Zufall sein. Was weiß mein Pate alles über mich, was ich nicht weiß? Wieso hat er sich nicht mehr seit meinem dritten Geburtstag gemeldet? Wieso kann ich mich nicht mehr an sein Gesicht erinnern?}

\enquote{Guten Morgen, Miss}, schreckte sie der Hauself aus ihren Gedanken. \enquote{Das hier ist für Sie.}

Pansy nahm den Brief, den ihr der Hauself gab, und las ihn sofort. Sie registrierte nicht die kleinen tippelschritte und das Knarzen des Stuhles, da sie so in ihrem Brief fixiert war.

\begin{brief}
Liebe Pansy,

tut mir leid, dass ich nicht zum Frühstück da sein kann, aber ich muss unbedingt noch vorher etwas erledigen.

Bin gegen 10 Uhr \gst
\end{brief}

Pansy sah auf ihre Uhr. Es war kurz nach 8 Uhr.

\begin{brief}
Bin gegen 10 Uhr wieder da und leiste dir ab da dann Gesellschaft. Ich werde dir dann deine Fragen, die dir sicher schon auf deiner Zunge brennen, beantworten.
\signumspace
Liebe Grüße

Onkel Fred
\end{brief}

\gedanke{Onkel Fred}, ging ihr durch den Kopf. \gedanke{Wenigstens weiß ich jetzt seinen Namen, seinen Vornamen.} Sie legte den Brief beiseite und biss wieder in ihr Brötchen, als sie den Elfen ihr gegenüber sitzend erblickte und kurz zusammenzuckte. Stumm und ausdruckslos, mit dem Brötchen im Mund, sah sie ihn an. Der Elf blickte kurz zurück und sah dann verständnislos an sich herunter. Dann sah er wieder zu ihr und schließlich über den Tisch, ob auch alles in Ordnung sei, bis es ihm schließlich dämmerte und er meinte zu wissen, was los sei.

\enquote{Oh, Verzeihung Miss, Remmy vergaß, dass Sie es offensichtlich nicht gewöhnt sind, mit Hauselfen zu frühstücken.}

Pansy biss den Teil des Brötchens in ihrem Mund ab und schüttelte dann ihren Kopf.

\enquote{Remmys Chef meint, dass die Familie, wenn möglich, zusammen essen sollte. Da Remmy normalerweise mit ihrem Chef und den anderen zusammen frühstückt, die aber nicht da sind, ist es Remmys Aufgabe, dies für Sie zu tun.}

\enquote{Du musst aber nicht\abs}, sagte sie, als sie ihren Bissen herunterschluckte, \enquote{mit mir frühstücken\abs}

\enquote{Remmy hatte heute selbst noch nichts. \gst Ich genieße das gemeinsame Frühstück mit der Familie}, fügte der Elf hinzu, nachdem er durch das Fenster in den Garten sah.

\gedanke{Ich genieße das}, dachte Pansy. \gedanke{Seit wann reden Elfen denn in der Ich-Form von sich?}

\enquote{Remmy?}, fragte Pansy vorsichtig nach und der Elf sah sie wieder an. \enquote{Du hast gerade \inner{Ich genieße das} gesagt.} Der Elf bekam große schuldbewusste Augen und wollte gerade etwas erwidern, doch Pansy unterbrach ihn. \enquote{Mich stört das nicht, nur ist es mir gerade aufgefallen.}

Remmy zog den Kopf leicht ein und meinte dann leise: \enquote{Unser Chef legt keinen Wert auf solche Unterwürfigkeit. Wir sollen uns nicht selbst beim Namen nennen. Ich tat das nur, um sie nicht zu sehr zu irritieren.} Und dann, nach einer kleinen Pause, fügte er leicht beschämend hinzu: \enquote{Er würde uns sogar Kleidung geben, wenn wir welche wollen}, doch er unterbrach sich, schlug sich die Hand vor den Mund und eilte hinaus. \enquote{Hab noch viel zu tun, Miss.}

Grinsend sah sie ihm nach. \gedanke{Hast wohl zu viel über deinen Herrn verraten}, ging ihr durch den Kopf.

Als sie fertig war mit dem Frühstücken, stand sie auf, öffnete die Fensterwand und trat einige Schritte auf die Terrasse hinaus. Nachdenklich schaute sie in den Garten, der aus allerlei verschiedenen Pflanzen bestand. Es schien keine Pflanze zu geben, die hier nicht wachsen würde.

Es klingelte an der Haustür und Pansy hörte ein leises \geraeusch{Plopp}. Die Haustür wurde geöffnet und eine Stimme sagte: \enquote{Ich habe hier ein Paket für E\agst}

\enquote{Schon gut, ich nehme es}, sagte der Elf. Kurz darauf fiel die Haustür ins Schloss und es war wieder still.

Dann, so gegen 10 Uhr, ging die Haustür auf und eine Stimme sagte: \enquote{Ich bin wieder da.} Pansy kam die Stimme bekannt vor, aber sie wusste nicht, woher sie sie kannte. Dann hörte sie Schritte. Sie traute sich nicht, sich umzudrehen, bis sie eine Hand auf ihrer Schulter spürte. \enquote{Es freut mich, dass du nicht unter einer Brücke schlafen musst. Dir wäre es vielleicht lieber gewesen, bei einer deiner Freundinnen oder einer Schulkameradin oder -kameraden zu sein, als bei mir.} Jetzt lief ihr ein kalter Schauer über den Rücken. Diese Stimme kannte sie genau. \enquote{Es tut mir leid, dass ich dich das ganze Schuljahr über getriezt und unfair behandelt habe, aber ich kann dir erklären, warum.}

Sie drehte sich leicht zitternd um und sah Frederick Elber in die Augen, ihrem ehemaligen Lehrer in \VgddK. \enquote{Sie\gst}, begann sie.

\enquote{Du, Pansy. Nenn mich Frederick, oder Fred oder Onkel Fred, aber siez mich nicht. Setz dich. Ich denke, ich bin dir eine Erklärung schuldig.}

Er nahm sie bei der Hand und führte sie in eine etwa zehn Metern entfernte eiche-lasierte Holzlaube und setzte zuerst Pansy hin und danach sich ihr gegenüber. Er sah auf die Tischplatte und begann zu erzählen.

\enquote{Es war kurz vor deiner Geburt, als deine Mutter zu mir kam und mich bat, deine Patenschaft zu übernehmen. Dein Vater war nicht sonderlich begeistert, stimmte aber zu. Dann, einen Tag vor deiner Geburt, musste er dringend geschäftlich weg. Also war ich an seiner Stelle da und stand deiner Mutter bei. Ja, ich war da}, sagte er, als er kurz aufsah und Pansys Blick einfing. \enquote{Ich war bei deiner Geburt dabei und sah dich. Klein und schmutzig.} Er lächelte sie an. Dann sah er in ihre Augen und fuhr fort. \enquote{Du hast große Ähnlichkeiten mit deiner Großmutter. Auf jeden Fall kam dein Vater einen Tag danach wieder zurück und nahm dich glücklich in seine Arme. Ich bot ihm an, in einem Denkarium deine Geburt zu sehen, doch er lehnte ab.}

\enquote{Was ist ein Denkarium?}, fragte Pansy nach.

\enquote{In ein Denkarium legt man Gedanken, die man sich noch einmal ansehen will. Man ist noch einmal am selben Ort, den man schon einmal erlebt hat. Es ist eine Art Illusion, die aus den eigenen Gedanken erzeugt wird. Man kann sich so auf Details konzentrieren, die man sonst leicht übersieht.}

Pansy nickte.

\enquote{Dann kam dein erster Geburtstag, dein zweiter Geburtstag und schließlich dein dritter. Dann brach der Kontakt ab und ich erhielt später einen Brief, dass ich nicht mehr kommen brauche. Ich bin persönlich bei ihm gewesen und habe ihn gefragt, doch er warf mich raus. \gst Ich schrieb dann ein paar Tage später einen Brief, in dem ich nachfragte, was ich falsch gemacht hatte, warum ich nicht mehr zu euch kommen durfte. \gst Als Antwort bekam ich einen Heuler, in dem er mir androhte, dir etwas anzutun, wenn ich dir zu nahe kommen würde und mich mit dir anfreunden würde. \gst Deshalb bin ich in der Schule immer so gemein zu dir gewesen. Ich hatte Angst um dich. Angst, dass er dir etwas tun würde, wenn in einem Brief mein Name fallen würde und du zu sehr Gutes von mir berichten würdest.}

Es sah Pansy wieder genau in die Augen, nachdem er während seines Vortrages ihr Gesicht abgesucht hatte.

\enquote{Und das soll ich dir glauben?}, fuhr sie ihn an.

\enquote{Schreib nach Hause, Pansy, wenn du mir nicht glaubst. Einen Brief an deinen Vater und einen an deine Mutter. Dann warte ab. Du kannst meine Eulen nehmen, oder deine Briefe an meinen Hauselfen geben.}

\enquote{Das werde ich auch tun}, sagte sie zornig, stand auf und ging in ihr Zimmer, um ihren Eltern zu schreiben.

Sie begann zwei identische Briefe aufzusetzen, die sich nur in der Anrede unterschieden.

\begin{brief}
Hi Mum,

ich sitze gerade bei einer Freundin, bei der ich die kommende Woche verbringen werde. Wir haben uns gestern Abend noch ein paar alte Fotos angesehen, als sie mich fragte, warum denn mein Pate keine Zeit für mich habe. Da fiel mir ein, dass ich ihn schon lange nicht mehr gesehen hatte. Was macht er überhaupt? Warum hat er sich von uns zurückgezogen?

Bitte antworte mir.
\signumspace
Pansy
\end{brief}

Sie rief nach Remmy und fragte ihn, ob er die Briefe übermitteln könnte. Er nickt, nahm die Briefe an sich und verschwand.

Der Rest des Tages verlief recht schweigsam, da Pansy die meiste Zeit in ihrem Zimmer verbrachte und sich mit ihren Hausaufgaben ablenkte. Sie kam erst gegen Abend zum Essen herunter und schaufelte daher ihre Portion gierig in sich hinein.

Ihr Pate saß mit seinem Elfen still am Tisch und sprach nichts mit ihr. \gedanke{Sie wird sich schon melden, wenn sie ihre Antwort hat}, dachte er.

Mitten im Essen tauchte eine Eule auf, die einen Heuler mitbrachte. Er war für Pansy. Sie band ihn los und wollte damit in ihr Zimmer rennen, doch er qualmte bereits verdächtig. Als sie die Vorhalle bereits zur Hälfte durchquert hatte, riss sich der Heuler los und explodierte. Sie hörte die Stimme ihres Vaters.

\extase{Dein sauberer Pate ist tot. Dieser Wahnsinnige hat sich selbst umgebracht, als er an neuen Tränken forschte. Er war ein Arsch. Frage ja nie wieder nach ihm und mich überhaupt nichts mehr du Todesser-Hure. Du bist für mich gestorben, du Flittchen.}

Geschockt stand Pansy nun da und schaute ängstlich und mit Tränen in den Augen zurück. Frederick hob seinen Kopf und schaute sie an, als ob er fragen würde: \accentuate{Wirst du weglaufen, wenn ich dir zu helfen versuche?}

Pansy kniff ihre Augen zu und die Tränen bahnten sich ihren Weg über ihr Gesicht. \gedanke{Todesser-Hure hat er mich genannt. Mein eigener Vater.} Doch sie realisierte sehr bald, dass das nur eine weitere Ausprägung seiner wütenden Art war. Sie spürte, wie sich ein Stück Stoff über ihre Wangen bewegte. Der Spur der Tränen aufwärts folgend. Dann fühlte sie Finger an ihrem Kinn, das leicht angehoben wurde. In ihrem Gesicht spürte sie einen warmen Atem, bevor ihre Stirn geküsst wurde. Dann spürte sie noch, wie etwas in ihre Tasche gesteckt wurde und wie sich jemand einen halben Schritt weit entfernte.

Langsam öffnete sie ihre tränenden Augen und sah zu ihrer Tasche. Dort ragte ein Stoff-Taschentuch heraus. Sie nahm es und erstickte neue aufkommende Tränen im Keim. Eine Eule flatterte durch das Fenster und setzt sich auf Pansys Schulter. Pansy nahm ihr den Brief ab und ihr Pate sagte der Eule, dass auf dem Tisch etwas Speck für sie liegen würde. Sie könnte sich auch in dem Eulenhäuschen eine Weile ausruhen, falls sie es denn wolle. Die Eule schuhute und flog in das Nebenzimmer, wo sie einen Streifen Schinken nahm und in die Nacht verschwand. Wohin, das konnte man nicht mehr sehen.

Pansy öffnete ihren Brief und wischte sich noch einmal die Tränen fort, bevor sie las. Ihr Pate stand immer noch vor ihr.

\begin{brief}
Liebe Pansy,

du weißt, dass ich gegen die Entscheidungen und Taten deines Vaters nichts ausrichten kann. Mir droht aber keine Gefahr. Er hat sich tierisch aufgeregt, als er erfahren hatte, wer dein Freund ist. Ich hoffe, du hast die anderen Wochen auch einen geeigneten Unterschlupf.

Von deinem Paten habe ich schon lange nichts mehr gehört. Sein Name ist aber, falls du ihn brauchen solltest, Frederick Elber. Dein Vater hat mir nie gesagt, warum er den Kontakt zu ihm abgebrochen hat. Aber du kannst froh sein, dass er sich darüber gefreut hat, wie er dich behandelt, als du uns geschrieben hast, sonst hätte er dich vermutlich verprügelt.

Ich habe unseren Elfen angewiesen, dass, falls du etwas brauchen solltest, er es dir bringen soll. \gst Er steht gerade neben mir und sagt mir, dass deine Sachen bereits abgeholt wurden. Ich beauftrage ihn, deine restlichen Sachen \gst sollten noch welche übrig sein \gst dir zu schicken. Und ein paar Erinnerungsfotos werde ich ihm auch noch mitgeben. Das dauert allerdings noch ein paar Tage.
\signumspace
In der Hoffnung, dass es dir gut geht\gst

Deine dich liebende Mum
\end{brief}

Erneut liefen Pansy Tränen über ihr Gesicht. Sie sah ihren Paten stumm an und er schien ihre Sehnsucht zu begreifen. Er ging auf sie zu und nahm sie fest in den Arm.

Als sie am nächsten Morgen aufwachte, erinnerte sie sich nur noch, dass sie nach dem Lesen des Briefes ihrer Mutter an der Schulter ihres Paten geheult hatte und dann wohl eingeschlafen war. Sie zog sich um und ging dann nach unten um zu frühstücken. Freundlich erwartete sie bereits Onkel Fred, der ihr zulächelte. Sie gab ein mattes Lächeln zurück, traurig, dass sie verstoßen wurde, aber doch glücklich, scheinbar eine neue Heimat gefunden zu haben.

\enquote{Wie ist dein Verhältnis zu Harry Potter?}, fragte sie ihr Pate während des Frühstücks.




\begin{kommentar}
Das erste Lied des Konzerts habe ich mir selber ausgedacht. Dann kommt das Lied, das ein Medusoner singt. Später wird es Lena Mayer-Landruth singen. Es wird die Titelmelodie zum Film Jesus liebt mich.
\end{kommentar}

\begin{kommentar}
Und zu den Dudelsack-Bläsern: Sucht einfach auf Youtube nach Thunderstruck und Dudelsack.
\end{kommentar}

\begin{kommentar}
Ich finde, der letzte Satz ist eine sehr schöne Überleitung zur nächsten Geschichte.
\end{kommentar}
