\chapter{Erwachen}


\enquote{Brauen Sie Ihren Trank}, fuhr ihn Snape an, \enquote{ich habe noch zu tun.}

Harry nickte und stellte seinen Rucksack ab, holte einen Kessel und füllte ihn mit Wasser. Während das Feuer das Wasser erwärmte, packte er seine Zutaten aus und legte die Pergamentrolle mit den Zutaten und der Brauanweisung daneben. Ruhig und gewissenhaft warf er seine Zutaten nacheinander in den Kessel und rührte an den entsprechenden Stellen um.

Eine Portion füllte er in eine Glasflasche die er verkorkte. Den Rest füllte er in einen großen Becher, den er heute noch in die Kammer tragen würde. Anschließend räumte er sorgfältig auf und leerte seinen Kessel mit einem \spruch{Evanesco}.

Am Büro seines Lehrers angekommen fragte er: \enquote{Dauert es noch lange, Professor?}

\enquote{Es dauert so lange es eben dauert.}

\enquote{Länger als eine halbe Stunde? Dann könnte ich noch diesen Trank hier\abs}

\enquote{Was haben Sie damit denn vor?}

\enquote{Das werden Sie nachher erfahren. Wir üben doch heute bewusstes Lenken von Gedanken.} \gedanke{Ich werde es darin unterbringen}, fügte er in Gedanken hinzu.

Snape nickte nur und gab ein undefiniertes Grunzen zurück. Harry machte sich auf den Weg zur Kammer. Dort angekommen legte er das Ei vorsichtig in den Becher. Nun musste er bis morgen warten und dann das Ei wieder in das Nest legen, noch ein kleiner Meldezauber kombiniert mit einem Verzögerungszauber und er konnte der schlüpfenden Kreatur zusehen und sie auf sich prägen, dachte er.

Zurück bei Snape ging es auch schon los. Wieder einmal drang er in Harrys Geist ein. Und wieder sah er den Gang, der zu dem runden Raum mit den Türen führte. Eine dieser Türen öffnete sich und er stand in einer mit Stroh gefüllten Kammer. Ein Ei lag in einer nestartigen Ansammlung von Stroh und anderen Materialien. Snape ging auf das Ei zu, nahm aus seiner Tasche ein Glas und eine kleine Flasche mit einer Flüssigkeit. Dann nahm er das Ei, legte es in das Glas und schüttete die Flüssigkeit darüber. Dann drehte er sich um und fand sich im selben runden Raum wie am Anfang wieder. Und wieder einmal stand diese steinerne Staute von Salazar Slytherin vor ihm. Ein paar kurze Bildfetzen durchzogen Snape. Er sah sich in einem Sessel im Gemeinschaftsraum der Gryffindors sitzen und einen Geist anschauen, der Ähnlichkeiten mit der Statue hatte. Dann lag er plötzlich in einem unbekannten Bett und schaute an die Decke. Er stand vor einer kristallenen Säule mit einer Vertiefung und schaute durch eine klare Flüssigkeit, die in der Mulde lag, auf ein unscharfes Amulett. Dann stand er dicht hinter einem Mädchen mit blonden Haaren und umfasste ihre Schultern. Dann brach die Verbindung ab.

\enquote{Ab wann}, schnaufte Snape, \enquote{habe ich gesehen, was sie mir nicht zeigen wollten?}

\enquote{Ab dem Sessel im Gemeinschaftsraum}, antwortete Harry.

\enquote{So langsam geben die Bilder einen Sinn. Mittlerweile weiß ich, dass es sich bei der Statue und bei dem Geist um Salazar Slytherin handelt. Ich wusste gar nicht, dass Sie eine derartige Vorstellungskraft haben und Sie Slytherin plastisch sehen können.}

\enquote{Das war nicht vorgestellt}, sagte Harry.

\enquote{Wie darf ich das verstehen?}

\enquote{Was ich Ihnen jetzt sage, darf diesen Raum nicht verlassen. Und es darf Ihren Mund im Beisein anderer nicht verlassen.}

Snape nickte. \enquote{Einverstanden.}

\enquote{Den Geist, den Sie gesehen haben}, sagte Harry, der sich jetzt in einen Sessel am Kamin gesetzt hatte; wobei Snape ihm gegenüber Platz nahm; \enquote{war wirklich Salazar Slytherin. Ich weiß nicht, wie es geht, aber er ist wirklich da. Ich kann mich mit ihm unterhalten.}

\enquote{Sie sind sich sicher, dass Sie sich nicht den Kopf angestoßen haben?}, fragte Snape.

\enquote{Ganz sicher}, erklang es hinter Snape.

Überrascht drehte er sich um und sah dem Geist ins Gesicht. \enquote{Wie?}

\enquote{Ich bin wirklich da. Ich weiß nicht wie, aber ich weiß warum. Harry braucht alle Hilfe, die er bekommen kann. Er kann Ihnen noch viel mehr erzählen. Ich denke, wenn er Ihnen soweit vertraut, dass er Ihnen zeigt, was es mit dem Ei auf sich hat, dann sollten Sie ihm vertrauen. Beim nächsten Mal dürften Sie mehr erfahren.} Dann verschwand er wieder.

Snape sah Harry nachdenklich an. \enquote{Dann bis zum nächsten Termin}, sagte er.

Harry stand auf und ging.

Mitten in der Nacht erwachte er durch seinen Alarmzauber. Er schlich sich aus seinem Zimmer und kontrollierte die Karte des Rumtreibers im Gemeinschaftsraum, als sich Salazar zeigte und meldete. \enquote{Neben dem Ausgang, Harry.} Harry erschrak, als er Salazars Stimme hörte. \enquote{Neben dem Ausgang, Harry}, sagte er erneut. Harry löschte die Karte und packte sie ein. Dann ging er zu Salazar und stellte sich hinter ihn. Sein Urahn zeigte auf einen Stein den Harry mit seinem Zauberstab berühren sollte. Harry tat wie ihm geheißen und berührte den Stein. \enquote{Schnell, Harry, den darunter auch und dann einfach durchlaufen.}

Harry folgte brav und durchschritt die scheinbar massive Wand. Harry stand im Dunkeln. Sofort ließ er seinen Zauberstab aufleuchten und sah sich um. Er stand in einem schmalen Gang vor einer Steinwand.

Als er sich wieder umgedreht hatte, sah er nach ein paar Metern eine Treppe, die beständig nach unten führte. Harry folgte der Treppe die gewendelt war, dann wieder ein Stück gerade ging, oder leicht schräg nach unten, oder über Treppen führte. Er kam einem Abzweig näher. Auf seiner Strecke zeigte plötzlich ein roter Pfeil in seine Richtung.  Auf dem Boden eines Abzweiges war ein blauer Pfeil, der weg zeigte.

\enquote{Der Weg zu Ravenclaw?}, fragte Harry.

\enquote{Gut erkannt}, sagte Salazar, der neben ihm kurz erschien, da es so länger hielt und er sich Harry immer wieder kurz zeigen konnte.

Auf seinem Weg weiter nach unten traf er auch die Abzweigungen von Hufflepuff und Slytherin. Am Ende angekommen zeigte ihm Salazar wieder einen Stein, damit er neben der Röhre raus kommen würde. Er sah sich kurz im Schein seines Zauberstabes um und schüttelte ihn ein paar mal, sodass kleine Lichtkugeln um ihn herum schwebten. Wieder in der Kammer kroch er in die Röhre, aus der der Basilisk kam und wartete, bis die Eierschale zu knacken anfing. Dann schloss er seine Augen und wartete.

Salazar sagte ihm in seinem Geiste: \stimme{Ich seh’s mir an und gebe dir dann meine Erinnerungen daran. Da man mich nicht sieht, passiert mir nichts. Außerdem hatte ich auch mal einen Basilisken. Sie wurde aber leider böse und du hast die Basiliskendame getötet. Ich wünschte, ich hätte noch die Gelegenheit gehabt, sie abzuholen und mit ihr zu sterben.}

Nach wenigen Minuten war es so weit und der kleine Basilisk war geschlüpft. Harry zuckte zusammen, als er die Berührung der kleinen Zunge spürte.

\parsel{Schließe deine Augen.}

\parsel{Warum, Meisster?}

\parsel{Tu es. Ich erkläre es dir gleich. Lass sie zu, bis ich es dir sage.}

\parsel{Ok, Meisster.}

\stimme{Er hat seine Augen geschlossen}, vernahm Harry in seinem Geist.

Harry öffnete seine Augen und richtete seinen Zauberstab auf die Augen des kleinen Basiliskenmännchens. Nacheinander sprach er über jedem der beiden Augen einen Zauberspruch. Dann wagte er den letzten Schritt.

\parsel{Du kannst deine Augen jetzt öffnen.}

Der kleine Basilisk öffnete seine Augen und sah Harry an. Dann schloss er sie schnell wieder. \parsel{Dass war gefährlich, Meisster. Ich hätte euch töten können.}

\parsel{Nein, ich habe Vorkehrungen getroffen.} Der Basilisk öffnete wieder seine Augen und sah Harry erstaunt an. \parsel{Ich habe einen Zauber auf deine Augen gelegt, damit mich dein Blick nicht tötet. Ich werde dir noch einen Trank geben, damit auch andere durch dich nicht getötet werden.}

Mit einem seligen Ausdruck schlängelte sich das Männchen in Harrys Schoß, der sich in der Zwischenzeit auf das Stroh gesetzt hatte. Vorsichtig strich er über den kleinen Basilisken.

\parsel{Wie isst euer Name, Meisster?}, fragte der Basilisk.

\parsel{Ich heiße Harry}, antwortete Harry.

\parsel{Und ich? Wie isst mein Name?}

\parsel{Tja, \gst wie wäre es mit Marcel?}

\parsel{Marssel? \gst Klingt gut. Marssel! \gst Ich bin müde, Meisster Harry.}

\parsel{Harry reicht, du brauchst mich nicht Meister zu nennen.} Dann nahm er die kleine Schlange vorsichtig hoch und legte sie auf dem Stroh ab. Er sprach noch einen Wärmezauber und verabschiedete sich. Marcel schloss seine Augen, rollte sich zusammen und schlief.

Harry stand auf, verließ die Geburtskammer und machte sich auf den Weg zurück.

\enquote{Warte, Harry. Hole noch eines der Bücher. Du brauchst Informationen über die Aufzucht und die Pflege von Basilisken.}

Harry nickte, holte das Buch und schleppte sich mit schmerzenden Augen in den Weg nach oben und in sein Zimmer zurück.

\trenn

Gedanklich ging er noch einmal die Okklumentik-Stunde mit Snape durch, die er, kurz nachdem er das Ei in den Trank gelegt hatte, absolviert hatte. Sein Geist schweifte dabei ab.

\begin{rueckblick}
Sein Lehrer drang wie üblich in seinen Geist ein, den er zunehmend besser kontrollieren konnte. Dieses Mal zeigte er ihm bewusst die Bilder, die er ihm zeigen wollte. Ein Ei auf dem Stroh, dann wieder diese Statue aus Marmor. Dann das Ei, das er in den Becher mit der gebrauten Flüssigkeit legte. Dann, wie er ihn braute; voll konzentriert und auf den Punkt gebracht. Dann begann er Snape zurückzudrängen, doch noch gab dieser nicht auf. Kurz dachte er über seine Mutter und schickte für einen Augenblick ein Bild seiner Mutter, das er in seinem Album gesehen hatte, Snape zu. Doch er entschied sich um und zeigte ihm stattdessen etwas, was er sich gedanklich zusammen reimte. Dann kamen auf Snape Bildfetzen zu, die Harry nicht kontrollieren konnte.

\enquote{Nehmen Sie keine Rücksicht darauf, wenn sich ein Feind in ihrem Geist aufhält.}

\enquote{Aber wir üben doch nur.}

\enquote{Wir gehen reale Bedingungen durch. Wenn Sie einen Feind so schlagen können, dann tun Sie das auch.}

Er ließ sich diesen Gedanken noch einmal durch den Kopf gehen, was zur Folge hatte, dass Snape sich selbst sah, wie er es Harry mitteilte. Harry zog amüsiert einen Mundwinkel hoch, als er Snapes Gesicht sah.

Plötzlich durchflutete ihn eine Erinnerung und obwohl er wusste, dass er sie so nicht gesehen haben konnte, weil er auf sich selbst blickte und noch zu klein war, wusste er, dass sie real war.

Er stand sich selbst gegenüber. Seinen weißen Zauberstab in der Hand auf sein kleineres selbst gerichtet und den Tötungsfluch sprechend, der an ihm abprallte und ihn selber traf. Er hörte einen unmenschlichen Schrei und Schmerzen durchfuhren ihn. Schmerzen, wie er sie noch nie erlebt hatte. Schlimmer als der Cruciatus-Fluch. Doch so schnell, wie der Schmerz kam, ging er auch wieder. Die Verbindung zu Snape brach ab und Harry sackte zusammen. Schwer atmend lag er bei vollem Bewusstsein auf dem Boden und atmete stoßartig. Langsam schwenkte er seinen Kopf zu Snape hinüber und sah ihn in einem Sessel sitzen und sich die Hand auf die Brust legte.

\enquote{Alles in Ordnung, Potter?}, fragte er.

\enquote{Geht so, Professor, und bei Ihnen?}, antwortete Harry fragend.

\enquote{Was war das?}, wollte er wissen.

\enquote{Voldemort, wie er versuchte mich zu töten. Denke ich. Ich\abs Ich habe einmal davon geträumt, aber ohne\abs Schmerzen.}

Dann fielen Harrys Augen endgültig zu.

Um zwei Uhr morgens wachte er auf und fand Snape schlafend im Sessel sitzend vor. Harry stand auf und suchte auf Snapes Schreibtisch ein leeres Pergament. Dann schrieb er darauf:

\begin{brief}
Bin ins Bett gegangen, gute Nacht Professor.
\signumspace
Harry Potter
\end{brief}

Dann schlich er sich vorsichtig zurück und schlief in seinem Bett ein.
\end{rueckblick}

Er überlegte, ob er Snape und seinen Freunden zumindest einen Teil seines Wissens zeigen sollte. Er könnte ihnen immerhin die privaten Räume Slytherins zeigen. Müde und mit geschlossenen Augen dachte er in den Raum hinein.

\gedanke{Salazar?}

\stimme{Ja, Harry.}

\gedanke{Ich habe mich gefragt, ob ich ein paar Leuten deine privaten Räume hier im Schloss zeigen darf. Professor Snape weiß immerhin von deiner Existenz. Und da ich nachweislich von dir abstamme, könnte mich das Amulett immerhin in seine Räume geführt haben. Ich werde weiterhin zu dir schweigen, aber deine Räume würde ich schon gerne\abs}

\stimme{Von mir aus gerne, Harry. Aber wie willst du es dem Bild am Eingang beibringen?}

\gedanke{Ich dachte, dass du\abs immerhin seid ihr zwei ja ein und dieselbe Person. Du könntest dich doch mit seinem Bild verbinden und ihn so überzeugen.}

\stimme{Ja, das stimmt. Wir sind ein und dieselbe Person. Aber ich weiß nicht, ob das, was du vorschlägst, auch klappt. Es käme auf einen Versuch an. Dann wäre mein anderes Ich nicht so komisch auf dich zu sprechen.}

\gedanke{Danke.}

\stimme{Wofür?}

\gedanke{Dafür, dass du es versuchst.}

\stimme{Für Versuche brauchst du dich nicht zu bedanken. Zumindest nicht bei mir. Danke mir nur für gelungene Taten.}

\gedanke{Verstanden Sal. \gst Verzeihung. Salazar.}

\stimme{Sal ist schon in Ordnung, wenn wir alleine sind.}

\gedanke{Aber wir sind doch immer alleine.}

\stimme{Und wenn dein Professor dabei ist? Oder wenn noch jemand dazukommt? \gst Später?}

\gedanke{Du bist doch derjenige, der sagt, ich darf nichts erzählen.}

\stimme{Noch nicht, Harry. Später.}

\gedanke{Wann ist später?}

\stimme{Falsche Frage, Harry.}

Dann kehrte Ruhe ein.

\trenn

Nach Wochen im Koma zeigte am Dienstag Professor Elber erstmals wieder eine Reaktion. Er öffnete die Augen und stöhnte. \enquote{Mein Kopf. \gst Was ist passiert?}

Die Tür zu Madame Pomfreys Büro ging auf und sie kam herein. \enquote{Ah, Professor Sie sind schon auf?} und schaute ihn dabei an.

\enquote{Wen meinen Sie?}

\enquote{Na Sie, Frederick. Haben Sie irgendwelche Beschwerden?}

\enquote{Nein, aber \gst wo bin ich?}

\enquote{Sie sind im Krankenflügel in Hogwarts.}

\enquote{Ah ja}, antwortete Professor Elber. Er erhob sich leicht und stützte sich auf seinen Ellenbogen und seinen Unterarmen auf. \enquote{Wenn Sie mir jetzt noch sagen können, wie ich hier hergekommen bin und was eigentlich passiert ist\abs}

In diesem Moment ging die Tür auf und Dumbledore kam herein.

\enquote{Und Frederick, alles wieder in Ordnung? \gst Vor allem, wie geht es dir?}

\enquote{Mir geht es gut, aber wer sind Sie? Ich verstehe, dass sich meine Frau um mich sorgt}, er zeigte auf Madame Pomfrey, \enquote{aber jemand den ich nicht kenne und mich besuchen kommt?}

Professor Dumbledore hob seine Augenbrauen und meinte dann zu Madame Pomfrey. \enquote{Kann ich Sie mal kurz sprechen} und dann zu Professor Elber gewandt: \enquote{Sie bekommen sie gleich wieder.}

Professor Elber hob seinen Oberkörper weiter an und saß nun im Bett, während Professor Dumbledore und Madame Pomfrey in ihr Büro verschwanden. Etwas benommen schaute er auf die andere Seite des Zimmers und durch das Fenster hindurch.

Nach einigen Minuten kam Professor Dumbledore mit Madame Pomfrey wieder aus ihrem Büro heraus. Professor Elber entdeckte sie und meinte nur: \enquote{Hallo Krankenschwester, wann gibt es was zu essen?}

\enquote{Sie wollen von ihrer Frau was zu essen?}, fragte Professor Dumbledore.

Professor Elber stutzt und meinte: \enquote{Ich bin nicht verheiratet, aber ich habe Hunger und fragte gerade die Krankenschwester, wann es was zu essen gibt.}

\enquote{Aber vor fünf Minuten haben Sie noch behauptet sie wäre ihre Frau.}

\enquote{Wer?}

\enquote{Sie.}

\enquote{Ich?}

\enquote{Ja.}

\enquote{Ne.}

\enquote{Doch.}

\enquote{Ohh.}

\enquote{Erinnern Sie sich nicht mehr daran?}

\enquote{Nein}, antwortete Professor Elber.

Madame Pomfrey brachte ihm eine Suppe und Professor Dumbledore verließ den Krankenflügel auf dem Weg zur Großen Halle, da es bereits Zeit für das Abendessen war. Er würde seinen Schülern diese großartige Nachricht noch heute mitteilen.

Währenddessen war Harry in der Kammer des Schreckens und gab Marcel seinen neuen Trank, damit keiner durch den Blick des kleinen Basilisken versteinert oder getötet werden würde. Das Brennen in seinen Augen verschwand und die Schlange konnte mit Harry die Kammer verlassen. Er nahm sie auf dem Arm mit nach oben.

Doch vor dem Feuer im Gemeinschaftsraum oder in seinem Zimmer fühlte sich Marcel nicht besonders wohl.

\parsel{Bitte, bring mich zurück. Hier fühle ich mich nicht so wohl.}

\gedanke{Scheinbar hat der Trank sein Marcels lispeln behoben}, dachte Harry.
\parsel{Warum? Möchtest du lieber alleine sein?}

\parsel{Ja. Ich ziehe die Einsamkeit oder die Gesellschaft anderer meiner Art vor.}

\parsel{Und Schlangen?}, fragte Harry nach.

\parsel{Die auch, wieso?}

\parsel{Dann habe ich eine passende Umgebung für dich.}

Harry nahm Marcel wieder auf den Arm und verdrückte sich durch die Gänge des Schlosses Richtung Salazars private Räume. Dort setzt er ihn ab und erklärte ihm, wie er in die Kammer kommen könnte. Der kleine Basilisk schien glücklich zu sein.

\parsel{Tust du mir noch einen Gefallen, Marcel?}, fragte Harry den Basilisken.

\parsel{Wenn ich kann.}

\parsel{Gib mir etwas deines Giftes.}

\parsel{Wie?}

Harry zauberte ein kleines Reagenzglas hervor und überzog es mit einer Gummimembran. Dann hielt er es seinem Freund hin. Marcel zögerte kurz, biss dann aber doch zu und wenige Tropfen seines Giftes rannen im Inneren des Reagenzglases entlang zu Boden.

\parsel{Danke Marcel.}

\parsel{Gerne geschehen, Harry.}

Harry verabschiedete sich und machte sich auf den Rückweg. In seinem Zimmer holte er noch eine Ampulle Basiliskengift der alten Schlange und nahm vorsichtig ein paar Tropfen von Marcels Gift, um es in seinen privaten Vorrat zu lagern. Er hielt es trotzdem für viel und nahm sich vor Kreacher zu sagen, dass er einen großen Teil davon im Grimmauldplatz in seinem Labor lagern sollte. Durch ein paar Bücher und die wenigen Besuche dort, hatte er einen guten Überblick über die Räumlichkeiten dort.

Nach getaner Arbeit, machte er sich auf den Weg zu Madame Pomfrey, um ihr etwas von dem Gift zu geben. Er vergewisserte sich vorher, dass sie nicht nachfragen würde, woher er das habe, was er ihr nun geben würde. Schließlich sagte sie zu und Harry überreichte ihr die kleine Phiole des Basiliskengiftes und das Reagenzglas mit den wenigen Tropfen.

Er vermutete, dass sie eine Ahnung hatte, woher das Gift in der Phiole sein könnte, dass sie aber keine Ahnung hatte, woher das Gift des jungen Basilisken war. Und so sollte es auch sein.

\trenn

Schweißgebadet wachte Harry auf. Er schrak aus seinem Traum hoch und schwitzte überall an seinem Körper. Er starrte auf einen imaginären Punkt weit draußen vor dem Schloss. Seine Hand fuhr auf seine Brust um zu fühlen, ob er sein Amulett noch bei sich trug. Er hatte es seit es ihm Ginny geschenkt hatte nicht abgelegt. Doch er fühlte sein Amulett nicht. Panisch suchte er danach. Es lag auf seinem Nachttisch. Er nahm es auf und legte es sich um. Wie auf ein unsichtbares Kommando wurde er ruhiger und ausgeglichener. Dann begann er sich wieder zu erinnern, was letzte Nacht passiert war.

Schlaftrunken hatte er, bevor er ins Bett stieg, statt seiner Brille sein Amulett abgenommen, hatte es auf seinen Nachttisch gelegt und sich danach hingelegt. Dann schlummerte er ein.

\begin{traum}
Voldemort war seit vier Jahren Tod und es stand ein Frühstücks-Treffen bei den Weasleys an. Draco Malfoy war auch dabei, da er sich mittlerweile von seinem Vater abgewandt hatte und sich in letzter Zeit besser mit Harry und seinen Freunden verstand. \gst

Harry und Draco saßen spät am Abend noch draußen und unterhielten sich über alle möglichen Dinge. Ihre Hände berührten sich und beide zogen den anderen mit den Augen aus. \gst

Draco lag mit Harry im Bett und Harry liebkoste Dracos freie Brust mit seiner Zunge. Draco keuchte unter Harrys warmer, feuchter Zunge. Sie begehrten sich so sehr. Sie spielten miteinander. Sie konnten sich kaum noch halten. Dann war es so weit. Harry zog Dracos Boxershorts herunter und glitt immer weiter mit seinem Mund an ihm herunter. Draco griff mit seinen Händen in die Matratze und stöhnte auf, als ihm Harry \gst
\end{traum}

So langsam kamen in Harry die Erinnerungen an seinen Traum wieder hoch. Er schaute auf seine Uhr und merkte, dass es sich um sieben Uhr morgens nicht mehr lohnte, sich hinzulegen. Er duschte erst einmal, zog sich danach an und ging nach unten.

Tamara saß in einem der Sessel und las. Sie war immer eine der Ersten, die morgens wach war. \enquote{Guten Morgen, Harry}, trällerte sie ihm entgegen. Mittlerweile war sie für ihn wie eine kleine Schwester.

\enquote{Guten Morgen, Tamara.}

\enquote{Schlecht geträumt?}, fragte sie ihn.

\enquote{Ja. Von Draco.} Doch das wollte Harry nicht sagen.

Interessiert horchte sie auf, klappte ihr Buch zu, nachdem sie ihr Lesezeichen hineingelegt hatte, und ließ das Buch mit ihrem Zauberstab an die Decke schweben. Dort sah es keiner und so nahm es ihr auch keiner weg. Harry hatte ihr gezeigt, wie sie es machen müsste. Und auch, dass ihr keiner dabei zusah.

Sie hob eine Augenbraue hoch und bohrte so lange nach, bis Harry ihr ziemlich grob und oberflächlich die Zusammenfassung sagte. Tamara bekam große Augen.

\enquote{Hast du meinen Bruder schon einmal nackt gesehen?}, fragte sie.

\enquote{Nein}, gab Harry empört zurück.

\enquote{Du hast ihn ziemlich genau beschrieben}, sagte sie. \enquote{Gehen wir frühstücken. Es sollte schon etwas da sein.}

Sie nahm Harry bei der Hand und zog ihn hinter sich her. Wenige Meter, nachdem sie das Porträt hinter sich gelassen hatten, lies sie von ihm ab und lief nun neben ihm her. Weiter unten trafen sie auf Draco. Seine Schwester begrüßte ihn und sah ihm in die Augen. Sie ließen sich etwas zurückfallen, um sich ungestört zu unterhalten. Plötzlich hörte er laufende Schritte hinter sich. Zuerst nur ein paar Füße, danach zwei Paar.

\enquote{Harry}, hörte er Tamaras Stimme. Doch weiter kam nichts. Harry drehte sich um und merkte, wie ihr Malfoy den Mund zu hielt. Doch Tamara konnte sich wehren, sie stieg ihm auf den Fuß und versuchte ihn zu beißen. Draco ließ von seiner Schwester ab, worauf sie gleich sagte: \enquote{Er hatte denselben Traum wie du, Harry.}

Er konnte sich jetzt nicht mehr bewegen. Steif stand er da. Schrittweise bewegte er seine Augen von Tamara hinauf zu Draco und sah ihm direkt in die Augen. Mechanisch drehte er sich um und lief die restlichen Meter in die Große Halle. Er setzte sich auf die erstbeste Bank und starrte in die Ferne. Er saß auf der Bank der Ravenclaws. Draco und Tamara kamen um die Ecke und setzen sich neben ihn. Harry war in der Mitte. Nach einer Minute, die Harry brauchte um sich zu beruhigen, sah er zu Draco. Dieser sah ebenfalls an die Wand. Dann drehte sich Harry um und nahm sich Croissant. Er biss hinein und nahm etwas Orangensaft dazu, um es hinunterzuspülen.

\enquote{Madame Pomfrey!}, sagte Harry matt und stand auf. Draco nahm ebenfalls ein Croissant, trank schnell etwas und trabte Harry hinterher.

In der Krankenstation angekommen war Madame Pomfrey gerade dabei aufzuräumen. \enquote{Mister Malfoy, Mister Potter. Was kann ich für Sie tun?}

\enquote{Nun ja Madame Pomfrey}, druckste Harry herum. \enquote{Haben sie etwas gegen Alpträume?}

Draco nickte nur stumm und betreten.

Madame Pomfrey hob eine Augenbraue und schaute die beiden an. \enquote{Welcher Art waren Ihre Träume?}, fragte sie. Die beiden schauten die Krankenhexe erstaunt an. \enquote{Für die Behandlung}, fügte sie rasch hinzu, als sie die beiden betrachtete.

\enquote{Unangenehme erotische}, sagte Draco.

Harry sah ihn fassungslos an.

\enquote{Und bei Ihnen, Mister Potter?}

Mechanisch und ohne zu überlegen, sagte er ebenfalls: \enquote{Unangenehme erotische.} Dabei sah er Draco unentwegt an. Wieder sah er zu Madame Pomfrey, die ihn jetzt mit beiden hochgezogenen Augenbrauen ansah und immer wieder zwischen den beiden hin und herwechselte.

\enquote{Oh}, sagte sie plötzlich. \enquote{Sie hatten denselben Traum?}, fragte sie. Keiner der beiden sagte ein Wort.

\enquote{Ich bin gleich wieder da. Setzen Sie sich.} Dann drehte sie sich um und ging in ihr Büro.

Harry und Draco nahmen sich jeder einen Stuhl und warteten. Nachdem sie ihren Trunk erhalten hatten, gingen sie in die Große Halle, wo Tamara schon am Slytherin-Tisch saß und auf ihren Bruder wartete.

Nach dem Frühstück hatte Harry zwar noch Hausaufgaben zu erledigen, aber ihm war im Moment überhaupt nicht danach diese auch wirklich zu erledigen. Ziellos streifte er durch das Schloss. An Klassenzimmern und anderen Räumen vorbeilenkten ihn seine Füße ohne bestimmtes Ziel. Erst als er stehen blieb, ohne dass er sich dessen bewusst geworden war, hob er seinen Kopf. Er stand genau unter einem dieser steinernen Bogen, die es in Hogwarts massenweise gab.

Sein Kopf wurde langsam klarer und als ihm unbekannte Stimmen so langsam leiser wurden, drehte er sich und drückte einen bestimmten Stein in der Wand. Nach ein paar Sekunden öffnete sich dieselbe und Harry trat wie immer in den schmalen kleinen Raum ein.

Er drehte sich um und wartete, doch nichts geschah. Also begann er die Zeichen abzusuchen. Er hatte vor einer Woche genügend Zeit gehabt, all diese Symbole auf ein Pergament zu übertragen, welches er fast immer bei sich trug und auf das Ron ihn einmal angesprochen hatte. Er hatte ihm nur erzählt, dass er bald mehr darüber erfahren würde. Es wäre eine Überraschung.

Mit seinem Finger fuhr er über die Symbole und drückte erschrocken auf eines, nachdem er Stimmen hörte die näher gekommen waren. Harry hatte den Eindruck, sie waren in seinem Gang zu hören. Die Wand schloss sich und der Boden vibrierte leicht. Harry schaute noch auf das Symbol. Es sah so aus, als ob es Gitterstäbe darstellen würde. \gedanke{Snape. Die Kerker. Sein Büro?}, fuhr es Harry durch den Kopf. Es kam in ihm leichte Panik auf, doch als sich die Türen öffneten, war keine Spur von einem Kerker, oder von Snape zu sehen.

Ein laues Lüftchen umspielte sein Gesicht. Er ging nach draußen und ohne es zu bemerken, schloss sich hinter ihm die Wand. Harry stand nun im Freien. Er stand auf einer steinernen halbrunden Plattform. Etwa einen Meter lang und anderthalb Meter breit, in der Schlossmauer eingelassen. Den Halbkreis umgab ein schmiedeeisernes Geländer, welches an manchen Stellen mit Moos bewachsen war und auch schon leicht zu rosten anfing.

Er genoss den Augenblick, genoss den Ausblick. Dann blickte er zu Boden. Er trat an das Geländer und sah nach unten. Dort war nichts. Es ging über hundert Meter in die Tiefe. Man konnte den Boden kaum sehen. Nur erahnen. Der nackte Fels schloss nahtlos an die Schlossmauer an und Harry hatte den Eindruck, dass eines in das andere überging. Nach einigen Sekunden trat er einen Schritt zurück und nahm so langsam über seine Ohren Vogelgesänge und Gezwitscher wahr. Es war ganz leise, sodass er es kaum hörte.

Er sah nach links. Dort war nur Mauerwerk zu sehen, ebenso rechts. Dort konnte er jedoch Fenster und in der Ferne den Rand eines Turmes erahnen.

Wo war er bloß? In welchem Teil des Schlosses trieb er sich bloß herum? Er würde mit der Karte des Rumtreibers wieder kommen und dann schauen, wo im Schloss er sich befand. Er hielt sich am Geländer fest und sah nach oben. Dort konnte er einige Türme erkennen. Aber von außen war es schwer zu sagen, um welche Türme es sich handelte.

Harry rutschte das Herz in die Hose, als er wieder geradeaus blickte. Die Wand, aus der er herausgetreten war, war wieder verschlossen und Harry hatte absolut keine Ahnung, wie er zurückkommen sollte. Er tastete die Wand ab, doch er blieb erfolglos. Also begann er seine anderen Optionen in Betracht zu ziehen. Er drehte sich wieder um und sah über das Geländer. Er könnte ein langes Seil heraufbeschwören und es am Geländer festmachen. Danach könnte er sich einfach abseilen. Doch unter ihm war nichts, worauf er am Ende seiner Reise stehen konnte. Keine Plattform, die um das Schloss herum führte und auf der er ins Innere des Schlosses zurückkehren konnte. Er könnte natürlich auch seinen Besen aufrufen.

Die Vogelgesänge wurden lauter. Nachdenklich sah Harry nach oben zu den Vögeln. Langsam beruhigte er sich wieder, da er, zwar nur unbewusst aber doch deutlich genug, Gesänge eines Phönix' wahrnahm. Oben am Himmel zog gerade ein Vogel seine Kreise und es schien, als würde er immer weiter sinken und auf Harry zufliegen. Mit wenigen Flügelschlägen kam er an und setzte sich auf das Geländer. Harry presste sich an die Wand. Der rote Vogel war doch ziemlich groß und unheimlich. Irgendwie kam der Vogel Harry bekannt vor. Dieser legte seinen Kopf leicht schräg, so als ob er versuchen würde Harry einzuschätzen. Harry trat nun langsam auf den Vogel zu und sah ihm direkt in die Augen. Dann fiel ihm ein Stein vom Herzen. Es war Fawkes. Professor Dumbledores Phönix. Er ging auf ihn zu und strich ihm durch sein Gefieder. Der Phönix stimmte ein Lied an und Harry durchfuhr eine innere Ruhe und Behaglichkeit.

\enquote{Schön, dich mal wieder zu sehen, Fawkes.} Der Vogel sah ihn an und nickte. Harry hatte den Eindruck, der Vogel verstünde ihn. Er strich ihm noch einige Minuten durch sein Gefieder, was der Vogel durch seine schönsten Gesänge erwiderte. Harry durchfuhr ein wohliger Schauer. \enquote{Kannst du mir helfen, Fawkes? Ich komme hier nicht weg.}

Der Vogel sah ihn an und gab einen kleinen Laut von sich. Harry war erleichtert, würde ihn der Vogel doch zurück ins Schloss fliegen. Doch das tat er nicht. Er sah an Harry vorbei und nickte mit seinem Kopf in Richtung Mauer. Harry drehte sich verunsichert um.

Die Mauer sah nicht anders aus, als vorher. Jeder Stein hatte dieselbe Farbe. Er schaute wieder fragend zu Fawkes. Dieser nickte weiterhin mit seinem Kopf zur Mauer. Also ging Harry dort hin und zeigte mit seinem Finger auf einen beliebigen Stein. Der Kopf des Phönix wippte zu Seite, um Harry zu verdeutlichen, er möge seinen Finger auf einen anderen Stein bewegen. Als der Phönix schließlich nickte, wusste Harry, dass er den richtigen Stein hatte. Er drückte ihn leicht, und der Stein gab nach. Er zählte von der linken Seite des Geländers zwei nach oben und drei nach rechts und notierte sich diese Position auf seinem Pergament. Dann drückte er den Stein und die Wand teilte sich. Harry trat ein und drehte sich um. Fawkes kam auf ihn zu geflogen und Harry konnte gerade noch seinen Arm hochheben, damit der Phönix darauf landen und sich setzen konnte. Er lief den Arm entlang hoch zu seiner Schulter und schmiegte sich danach an Harrys Kopf.

Harry wollte schon einen Knopf drücken, als ihm Fawkes zart in sein Ohr biss. \enquote{Au Fawkes. Lass das}, sagte Harry und sah ihn verärgert an. Doch Fawkes schüttelte den Kopf und nickte zweimal nach links und einmal nach unten. Harry stutzte. Dann sah er wieder auf die kleinen Knöpfe. Er folgte mit seinem Blick den Anweisungen Fawkes und sah einen Knopf, der wie ein Fernrohr aussah. Doch da gab es noch einen, der ähnlich war und daneben ein Turm abgebildet war. Harry legte seinen Finger auf den Knopf und sah Fawkes an. \enquote{Diesen hier?} Der Phönix nickte und Harry drückte ihn (den Knopf). Die Fläche oberhalb der Knöpfe leuchtete gelblich auf. Harry wusste nicht, was das zu bedeuten hatte. Nach einigen Sekunden hörte das Leuchten auf und der Knopf sprang wieder heraus. Fawkes gab einen enttäuschenden Laut von sich. Harry sah den Vogel noch einmal an und drückte danach erneut den Knopf. Wieder leuchtete die Fläche auf, aber dieses mal legte Harry seine Hand darauf. Das Leuchten veränderte sich und begann zu pulsieren. Danach leuchtete es in einem dauerhaften Rotton und der Knopf sprang erneut heraus. Resigniert wollte Harry einen anderen Knopf drücken, als sich Fawkes mit seinen Krallen in Harry Schulter fester hielt und ihn zusammenzucken ließ.

Sauer sah er den Vogel an, der nun den Kopf schüttelte und einen kleinen Laut von sich gab. \gedanke{Also gut}, dachte Harry, \gedanke{dann eben noch einmal.} Harry drückte abermals den Knopf, welchen Fawkes ihm gezeigt hatte und legte danach seine Hand auf die leuchtende Fläche. Dieses Mal jedoch begann Fawkes seine Gesänge und die Fläche leuchtete Grün auf.

Eine unheimliche Stimme erklang. \stimme{Berechtigung erteilt.} Danach verstummte sie wieder.

Harry hörte sie klar und deutlich, hatte aber das untrügliche Gefühl, nur er würde sie hören. Aber das konnte er nicht nachprüfen, da außer Fawkes niemand sonst hier war.

Die Fläche leuchtete nun grün. Harry spürte ein leichtes, aber kurzes Kribbeln. Der Boden begann leicht zu vibrieren und Harry nahm seine Hand herunter. Danach verschwand das grüne Leuchten.

Kurz darauf öffnete sich die Wand vor ihm und Harry trat vorsichtig heraus. Fawkes hatte inzwischen seine Krallen wieder etwas eingezogen und hatte sich wenige Zentimeter von Harrys Schulter entfernt, um sich daneben festzuhalten. Harry sah ein Fernrohr, dessen Ende in eine Halbkugel zu führen schien. Sie bildete eine Art Sessel, erkannte er, als er sich um sie herum bewegte. Er befand sich im obersten Stockwerk eines Turmes. Ihm kam der Ort unbekannt vor. Er war verlassen. Harry hörte keine Stimmen. Es lag auch kein Staub herum. \gedanke{Entweder wird der Ort noch benutzt, oder er wurde verzaubert, dass sich kein Staub darauf absetzen konnte.}

Außer dem Fernrohr und kahlen Wänden, an denen ein einzelnes Poster über die verschiedenen Mondphasen hing, war der Raum leer. Er entdeckte eine Wendeltreppe, welche er hinunter in das nächste Stockwerk ging. Dort standen einige Reihen voller Bücher sämtlicher Stilrichtungen. Harry konnte auch einiges an Muggelliteratur erkennen. Auch hier waren die Wände kahl und der Raum sonst leer. Ein Stuhl stand vor einem Schreibtisch und einem Fenster, sodass das Licht direkt auf die Bücher fiel, die da lagen und somit leicht gelesen werden konnten. Harry konnte jetzt erkennen, dass der Raum in der nächst tieferen Etage breiter wurde, aber er konnte sich nicht über das Geländer lehnen, um runter zu sehen.

Also trat Harry die Wendeltreppe eine weitere Etage hinunter. Er hielt sich am Geländer fest, da Fawkes doch auf seiner Schulter drückte. So langsam dämmerte ihm, wo er war. Als er den Boden erreicht hatte, sah er wieder die ihm bekannten storchenbeinigen Tische mit den filigranen silbernen Gerätschaften und der schweren Eichentür. Sein Blick erhob sich und er sah die ebenso bekannten Bilder der schlafenden Direktoren Hogwarts. \enquote{Ich bin in Dumbledores Büro}, kam es aus Harry vor Staunen raus. Er drehte sich so, dass Fawkes auf seine Stange laufen konnte und drehte sich wieder um. Sein Blick wanderte durch den leeren Raum. \gedanke{Ich muss hier raus}, dachte Harry. Plötzlich spürte er etwas Warmes auf seiner Schulter. Er schaute sie an und entdeckte, das Fawkes wenige Tränen auf seiner Schulter niederließ. Es brannte leicht, doch Harry wich nicht zurück. Als Fawkes wieder seinen Kopf erhob, drehte sich Harry zu ihm und sah ihn dankbar an. Er fuhr mit seiner Hand über seine Schulter. Sie war scheinbar vollständig geheilt. Unter seiner Kleidung konnte er allerdings nicht mehr erkennen. Mit beiden Händen strich er Fawkes jetzt über sein Gefieder. Der Phönix stimmte wieder seinen Gesang an. Dieses Mal einen anderen. Er steckte beide Flügel weit von sich und Harry fuhr ihm knapp unter den Flügeln über seinen Bauch. Er spürte ein tiefes inneres Gefühl der Ruhe und Ausgeglichenheit. Dann strich er mit Daumen und Zeigefinger auf der Ober- und Unterseite an Fawkes Flügeln entlang. Der Phönix schaute ihn verträumt an und Harry konnte nicht anders, als nur zu lächeln und ihn ebenso verträumt anzusehen.

Die beiden waren so miteinander beschäftigt, dass sie nicht mitbekamen, dass hinter ihnen die Tür aufging und Professor Dumbledore mit Professor McGonagall in den Raum kam. Sie schienen sich zu unterhalten, als sie plötzlich mitten im Satz verstummten, da sie Harry sahen, wie er Fawkes über sein Gefieder strich.

\enquote{Harry?}

Plötzlich wurde er aus seinen Gedanken gerissen. Fawkes klappte seine Flügel schuldbewusst zusammen und Harry wirbelte herum. \enquote{Professor, Professor!} war alles, was er herausbekam. \enquote{Fawkes hat\abs ich war\abs bin\abs} Harry atmete einmal tief durch, während er interessiert den Boden betrachtete. Dann sah er Dumbledore direkt in die Augen und begann zu erzählen. \enquote{Ich bin gerade durchs Schloss gelaufen, als ich nicht mehr wusste, wo ich war. Dann bin ich Fawkes begegnet. Ich habe ihn gestreichelt und wollte ihn zurückbringen. Er hat mich dann hereingebracht.} Danach sah er wieder zu Boden. \enquote{Ich werde jetzt gehen und nachher meine Strafarbeiten bei Professor McGonagall abholen.} Er drehte sich nochmals um und sagte. \enquote{Danke, Fawkes.} Dann durchquerte er Dumbledores Büro.

Er hielt den Türgriff in der Hand, als Dumbledore sagte: \enquote{Harry.} Dieser drehte sich um.

Dumbledore und McGonagall hatten sich bereits gesetzt und Professor McGonagall sagte ihm wortlos, nur durch ihre Gestik, dass er sich neben sie setzen möge. Dumbledore saß wie immer in seinem Stuhl hinter dem Schreibtisch und hatte seine Fingerspitzen gegeneinander gelegt. Harry trottete mit einem Kloß im Hals auf die beiden zu und setzte sich.

\enquote{Wie sind Sie hier hereingekommen, Mister Potter?}, fragte ihn Professor McGonagall.

Dann sah Harry kurz zu Dumbledore und danach wieder zu Professor McGonagall. \enquote{Da ich schon mal hier bin, können Sie mir auch gleich sagen, was ich für Strafarbeiten bekomme.}

Professor McGonagall hob eine Augenbraue hoch.

\enquote{Hast du was angefasst?}, fragte ihn Dumbledore. \enquote{Nein\abs} Harrys Kopf wirbelte herum. \enquote{Doch\abs den Türgriff und\abs das Geländer.}

Dumbledore sah in an. \enquote{Was noch?}, fragte er. Harry sah zu Fawkes. \enquote{Außer Fawkes, der wohl nicht zählt\abs bei diesem Besuch nichts.}

Dumbledore lächelte Harry an. \enquote{Nun, Minerva. Harry ist in Ihrem Haus.} Er sah zu McGonagall.

Diese ließ ihren Blick nicht von Harry ab. Schweigend sah sie Harry an. Am liebsten würde er jetzt im Boden versinken.

\fluestern{Wenn sie doch nur schreien würde, oder zumindest irgendetwas sagen würde.}

\enquote{Oh, ich sage etwas, Mister Potter.}

Jetzt war Harry wieder bei vollem Bewusstsein. \enquote{Habe\abs ich\abs etwa\abs etwa laut geredet Professor?}

\enquote{Ja Mister Potter, zwar nicht sehr laut, aber ich habe es doch gehört.}

Harry schluckte wieder, zwang sich aber Professor McGonagall anzusehen. Er konnte für den Bruchteil einer Sekunde ein Lächeln erkennen. Jetzt riskierte er es. \enquote{Das habe ich gesehen}, sagte Harry frech.

\enquote{Was?} fragte Professor McGonagall.

\enquote{Dass Sie mich angelächelt haben. Was ist jetzt mit meiner Strafarbeit?}

\enquote{Gibt es keine und nun gehen Sie Mister Potter}, sagte sie.

\enquote{Danke, Professor.} Harry sprang auf und gab ihr einen Kuss auf ihre Wange. Dann rannte er zur Tür, öffnete sie und sah noch einmal kurz zu Dumbledore. Dann schloss er die Tür, lauschte aber an selbiger.

\enquote{Mir scheint, dass Sie einen Verehrer gefunden haben, Minerva}, sagte Dumbledore.

\enquote{Seien Sie ruhig, Albus}, schnauzte sie Dumbledore an.

Harry musste grinsen. Er verließ seinen Horchposten und ging kurz in seinem Zimmer vorbei, um seine Schulsachen zu holen. Dann setzte er sich in die Bibliothek zu seinen Freunden, um die Hausaufgaben zu erledigen.

Kurz nach dem Abendessen ging er in einen selten benutzen Flügel des Schlosses. Er sah zu einem Fenster hinaus auf den See, wo der Krake gerade seine Bahnen zog.

\enquote{Kreacher?}, rief Harry mehr fragend als bestimmend in die kühle Luft hinaus.

Der Elf erschien und verbeugte sich. \enquote{Was kann Kreacher für seinen Herrn tun?}

\enquote{Erzähl mir etwas über den Umhang, den du mir zu meinem Valentins-Ausflug gegeben hast.}

\enquote{Sir Harry will sicherlich etwas über diese Runen wissen.}

\enquote{Ja Kreacher. Und warum die Familie Black solch eine Angst hervorruft. Als ich in einem Laden in Hogsmeade stand}, er sah Kreacher nun an, \enquote{hatte der Verkäufer scheinbar Angst vor mir.}

\enquote{Da hat der Verkäufer richtig gehandelt, falls er euch täuschen wollte. Diese Kleidungsstücke sind sehr selten und kostbar. Es geht eine Magie von ihnen aus, die den Träger spüren lässt, wie der gegenüber eingestellt ist. Die Familie Black hat ihn in früheren Zeiten immer zu Verhandlungen angezogen. Kreacher dachte, es sei ein passendes Kleidungsstück für ein erstes Date. Viele Frauen wollen sich einem nur des Geldes und des Ruhmes wegen nähern.}

\enquote{Was passiert, wenn einem der Gegenüber wohlgesonnen ist und keine bösen Gedanken oder Absichten hat?}

\enquote{Dann wirkt die Magie nicht. Man spürt nichts. Es ist so, als wenn man den Umhang nicht tragen würde.}

\enquote{Dann hat Katharina keine Hintergedanken}, meinte Harry mehr zu sich selbst und sah wieder zum Fenster hinaus.

\enquote{Das hat Kreacher auch gespürt}, sagte der Elf und zog ruckartig den Kopf zwischen seine Schultern, bevor er ihn wieder herausnahm.

Harry bekam davon nichts mit. Er reagierte kurz darauf, indem er wieder zu Kreacher sah. \enquote{Du hast sie überwacht?}

\enquote{Kreacher wäre ein schlechter Elf, wenn ich die gröbsten und groben Sachen, die man leicht herausfinden kann und die meinen Herrn vor Schaden bewahren können, nicht machen würde. Kreacher hat nicht in den privaten Sachen von Miss Chapel geschnüffelt, oder andere Verletzungen der Privatsphäre vorgenommen. Er hat sich freiwillig gemeldet, um ihr Zimmer ein paar Mal zu säubern. Und auch sonst hat er ein paar Gespräche mitbekommen. Kreacher schweigt aber über das Erfahrene, sofern es nicht seinem Herrn schadet.}

Harry nickte. \enquote{Das hast du gut gemacht, Kreacher. Über die kleine Einmischung hättest du mich aber vorher informieren können.}

Kreacher nickte und verstand.

\enquote{Was tun wir heute noch?}, fragte Harry, als er wieder zum Fenster hinaussah. Er meinte damit nicht einmal seinen Elfen, sondern mehr sich selbst.

\enquote{Kreacher würde gerne zum Grimmauldplatz zurückkehren, um dort seinen privaten Besitz zu holen.}

Harry sah seinen Elfen wieder an. \enquote{Verrätst du mir, was es ist?}

\enquote{Kreacher würde es vorziehen, es für sich zu behalten.}

\enquote{Essensreste, oder vielleicht alte Lumpen?}

Kreacher schüttelte seinen Kopf.

\enquote{Von mir aus. Wenn es dir gehört. Aber warum fragst du mich?}

\enquote{Sir Harry hat das Aufenthalts-Bestimmungsrecht für Kreacher. Wenn Sir Harry sagt, dass Kreacher in Hogwarts bleiben soll, dann darf Kreacher nicht von sich aus woanders hin. Es sei denn, sein Herr schwebt in Gefahr und diese kann dadurch abgewendet werden.}

Harry verstand. \enquote{Du darfst jederzeit in den Grimmauldplatz, um etwas zu holen, abzulegen, oder etwas zu erledigen, falls es mir oder dem Haus nicht schadet. Auch keinen Freunden von mir.}

\enquote{Das würde Kreacher niemals tun, Sir. Kreacher dankt euch.} Dann verschwand der Elf.

\trenn

Bald war wieder Frühlingsanfang, die Blumen begannen schon seit einiger Zeit ihre Blüten zu zeigen und Harry saß wieder im Unterricht. Professor Snape hielt gerade einen Vortrag über Kneuzers, als die Tür zum Klassenzimmer aufging und unter großem Staunen und Gemurmel Professor Elber das Zimmer betrat.

\enquote{Lassen Sie sich nicht stören Severus. Ich will mir nur einen kurzen Überblick über die vergangenen Wochen verschaffen und komme erst gegen morgen Nachmittag zum Unterrichten. Poppy hat mich zwar entlassen, mir aber noch einen Tag Ruhe verordnet. Also bringe ich mich erst einmal auf den neuesten Stand.}

Leichtfüßig schritt er durch die Klasse, obwohl er einen leicht müden und erschöpften Eindruck machte. Er ging an Professor Snape vorbei und öffnete die Schublade des Pults, nahm das kleine Buch heraus und ging die Treppen zu seinem Büro hoch, nachdem er die Schublade wieder geschlossen hatte. Er verschwand darin und ließ Professor Snape die Klasse weiter unterrichten. Professor Snape unterrichtete weiter, als ob nichts geschehen wäre. Nach einigen Minuten kam Professor Elber wieder aus seinem Büro, legt das kleine Buch zurück in das Pult, bedankte sich bei Professor Snape und verschwand wieder.

Professor Snape wandte sich wieder der Klasse zu und führte seinen Unterricht über die Kneuzers fort. \enquote{Diese Kreaturen haben die Eigenschaft, Halluzinationen und Verwirrung hervorzurufen. Je näher man einem Kneuzer kommt, desto realer werden diese Irritationen.} Die Schulglocke läutete und Professor Snape fügte hinzu. \enquote{Bis morgen Früh eine Zeichnung über Kneuzers mit exakter Beschriftung der einzelnen Körperteile. Sie dürfen gehen.} Alle stöhnten und verließen das Klassenzimmer. Draußen wartete Professor Elber, der jetzt ein paar Notizen in der Hand hielt. Professor Snape kam nun aus dem Klassenzimmer und Professor Elber fing ihn ab.

\enquote{Severus? Auf ein Wort, oder zwei. Ich habe mir den Stoff, den Sie unterrichteten, angeschaut und muss sagen, ich bin beeindruckt. Die Änderungen, die Sie gemacht haben, haben dem Ganzen noch die nötige Würze gegeben.}

\enquote{Ich freue mich, dass es Ihnen gefällt.}

\enquote{Ja, durchaus. Aber wie sieht es mit den Samstagskursen aus? Haben sie die auch gehalten?}

\enquote{Die hat Professor Dumbledore übernommen.}

\enquote{Ach}, und er grinste.

\trenn

\enquote{Ist es jedes Jahr hier so stressig für die Lehrer?}, fragte Elber Dumbledore.

\enquote{Nein, nein}, kam von Dumbledore.

\enquote{Aber wieso sind dann dieses Jahr so viele schwarz-magische Vorgänge zu finden? Wir sind doch auf dem Weg zur Krankenstation, also muss wieder etwas vorgefallen sein \gst Albus, das ist mir nicht geheuer. Ich bin mittlerweile mehr mit anderen Sachen, als mit dem Lehrauftrag beschäftigt. Ich bin dabei, Harry zu unterrichten, er lernt übrigens sehr gut, habe aber sonst Sachen zu tun, für die ich nicht hier bin.}

\enquote{Aber, dass Sie Poppy dazu gebracht haben mit Einschränkungen schwarze Magie zu akzeptieren und es sogar bei mir mit Abstrichen geschafft haben \gst das rechne ich Ihnen hoch an.}

Elber lächelte ihn nur an und meinte: \enquote{Wie ich bereits meinen Schülern sagte: \inner{Es gibt keine schwarze Magie. Es gibt nur die Intention desjenigen, der einen Zauber ausspricht und die Ansichten der Auswirkungen. Magie hat keine Farbe.}}

\enquote{Wie ich hörte, haben Sie Harry viel beigebracht.}

\enquote{Ja.}

\enquote{Er ist richtig gut geworden und weiß schon viel.}

\enquote{Das sind lediglich Grundlagen.}

\enquote{Grundlagen?} Dumbledore blieb stehen. \enquote{Grundlagen?}

Auch Elber bleib stehen und sagte: \enquote{Ja. Wir fangen gerade an, uns in den fortgeschrittenen Bereich zu arbeiten.}

\enquote{Was verstehen Sie unter \accentuate{fortgeschrittenen Bereich}?}

Elber drehte sich wieder um und sah Richtung Krankenflügel. \enquote{Wie lange ist der Weg noch?}

\enquote{Etwa dreihundert Meter.}

Elber schnippte mit den Fingern und beide standen vor dem Krankenflügel. \enquote{Das meine ich mit \accentuate{fortgeschrittenem Bereich}}, antwortete er.

\enquote{Sie beherrschen die Magie der Elfen?}

\enquote{Warum auch nicht. Wir können viel voneinander lernen. Ich habe meinen beispielsweise Tricks beigebracht, die nur von Zauberern beherrscht werden. Auch können sie mit Zauberstäben umgehen. Es entlastet sie bei schwierigen Aufgaben.}

\enquote{Aber, Elfen dürfen doch gar keine Zauberstäbe\abs}

\enquote{Wer sagt das?}

\enquote{Das ist Gesetz.}

\enquote{Von wann?}

\enquote{1640, oder so.}

\enquote{Dann ist es nicht einmal gültig. Es gibt ein Gesetz von 932, das ausdrücklich jedem magischen Wesen den Gebrauch von Hilfsmitteln, magisch oder nicht, gestattet, wenn es eine Erleichterung darstellt und nicht zum Schaden der Gesellschaft ist. Also ist das neue Gesetz unwirksam. Es wurde meines Wissens nicht aufgehoben. Zudem müsste man für solch eine Aktion das gesamte Gamot-Kremium mit allen Mitgliedern einberufen, was seit 1602 nicht mehr stattgefunden hat.}

\enquote{Woher wissen Sie das?}

\enquote{Ich habe\abs darüber gelesen. In der Bibliothek meiner Familie gibt es viele Bücher über Geschichte, Gesetze und allerlei anderes.} Er drehte sich um und öffnete die Tür.

Drinnen saßen Madame Pomfrey und ihre Patientin. An der Schuluniform konnte man das Haus Slytherin erkennen. Ihre Haare verbarg sie unter einem bunten Handtuch. Als sie sich umdrehte um die beiden anzusehen, sah ihre linke Gesichtshälfte wie versteinert aus. Die andere Gesichtshälfte sah nicht glücklich aus.

\enquote{Alraunensaft?}, war Dumbledores erste Frage.

\enquote{Hilft nicht.}

\enquote{Was ist mit dem Handtuch?}

\enquote{Das hat sie zu einem Turban gebunden, um die anderen nicht zu versteinern. Auf ihrem Kopf haben sich lauter Schlangen gebildet.}

\enquote{Welche anderen?}, fragte Elber.

\enquote{Die anderen hinter dem Vorhang sind ganz versteinert. Miss Chapel hier versicherte mir glaubhaft, dass es was mit ihren Haaren zu tun hat. Sie wurde vermutlich Opfer eines Streiches. Ihre Haare sind laut ihrer Aussage zu Schlangen geworden.}

\enquote{Medusa?}, fragte Elber.

\enquote{Ja}, antwortete Katharina Chapel.

\enquote{Vermutlich ein schlechter Scherz eines Mitschülers}, meinte Madame Pomfrey.
