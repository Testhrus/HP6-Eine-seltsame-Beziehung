\chapter{Magie sammeln}


Wieder in der Stadthalle von Hogsmeade wurden die Paare für die Apparitionsversuche ausgelost. Die Apparatur spuckte gerade zwei weitere Kugeln aus. \enquote{Harry Potter}, ließ die kleine Kugel verlauten, als sie zerplatzte. Harry trat vor und wartete auf seinen Partner. Eine weitere Kugel schoss hervor und tönte \enquote{Luna Lovegood.} Luna trat ebenfalls aus der Menge hervor.

Harry hörte von Professor Elber ein: \enquote{Das könnte interessant werden. Auf drei \gst Eins. Zwei. Drei.} Luna und Harry apparierten gleichzeitig zur anderen Seite der Halle. Dort angekommen schienen beide verwirrt zu sein und waren etwas wackelig auf den Beinen. Professor McGonagall und Professor Elber liefen sofort zu den beiden, welche sich bereits anschauten und lauthals zu Lachen anfingen.

\enquote{Was ist los, Mister Potter, Miss Lovegood}, rief Professor McGonagall.

\enquote{Ich bin wieder in meinem Körper}, sagten Luna und Harry fast gleichzeitig.

\enquote{Sind Sie sicher?}, fragte Professor McGonagall nach.

\enquote{Aber ja}, antwortete Luna.

Professor Elber zog seinen Zauberstab, richtete ihn auf die beiden und sprach denselben Spruch wie schon einmal, um festzustellen, wie beide miteinander verbunden waren. Dann sagte er sehr ernst. \enquote{Ich muss euch beide enttäuschen. Das hat die Sache nur noch verschlimmert.}

Professor McGonagall schaute ihn nur an. \enquote{Wie meinst du das?}, fragte sie ihn.

\enquote{Die Signale, die vorher vom anderen Körper empfangen wurden, wurden nicht etwa zurückgesetzt, wie es sein sollte. Sie werden jetzt wieder reflektiert. Es scheint nur so, als ob jeder in seinem eigenen Körper ist. \gst Harry, apparieren Sie mal alleine.}

\enquote{Meinen Sie, dass das klug ist Professor?}, fragte Harry.

\enquote{Keine Ahnung}, sprach Professor Elber. \enquote{Aber wenn Sie nicht apparieren können, dann ist der Unterricht so lange für Sie zu Ende, bis Ihr Zustand aufgelöst wurde.}

Harry apparierte nun an das andere Ende der Halle. Doch er blieb in seinem Körper.

\enquote{Und jetzt Sie, Luna.} Luna tat dasselbe und es passierte ebenfalls nichts anderes. \enquote{Nun ja, es scheint, dass sie zumindest alleine apparieren können. Immerhin. Sie müssen sich halt abstimmen. Aber das dürfte für Sie kein Problem sein.} Er grinste.

Er war wieder in seinem Körper und hatte erst einmal Hunger. Luna hatte in seinem Körper anscheinend noch nichts gegessen. Also schlich er sich nach der Stunde erst einmal in die Küche. Er ging den Gang vor der Küche entlang, bis er zum Bild mit der Obstschale kam. Er streichelte die Birne, aus der sofort ein Türgriff wuchs. Er betrat die Küche und stieg die wenigen Stufen hinunter.

Die kleinen Elfen schauten ihn an. \enquote{Wer du sein?}, fragte ihn ein unbekannter Elf.

\enquote{Harry. Harry Potter}, antwortete er ihm.

Die anderen gingen sofort ihrer Arbeit nach. Harry setze sich an den Rand des Tisches und der kleine Elf brachte ihm sofort Essen. Harry nahm sich eine Scheibe Brot, etwas Hähnchen und goss aus einem großen Krug Kürbissaft in seinen Becher. Er stoppte in der Mitte und füllte den Rest mit klarem Wasser auf. Er war gerade mit Essen fertig, als neben ihm Dobby erschien.

Sie schauten sich fragend an.

Dann sagte Harry: \enquote{Ich bin's, Harry.}

\enquote{Harry Potter, Sir}, gluckste der kleine Elf. \enquote{Schön Sie wiederzusehen.} Er nahm ihm gegenüber Platz. \enquote{Es freut Dobby, dass Sie wieder in Ihrem Körper sind.}

\enquote{Danke Dobby}, antwortete Harry. \enquote{Aber unser Problem ist noch immer nicht vollständig gelöst.}

\enquote{Nicht gelöst?}, fragte Dobby. \enquote{Aber Harry Potter ist wieder in seinem eigenen Körper.}

\enquote{Schon Dobby. Aber Professor Elber sagte uns, dass unsere Sinneseindrücke jetzt nicht nur einmal, sondern zweimal reflektiert werden.}

Dobby machte große Augen.

Harry wischte sich seine Hände mit einer Serviette ab und Dobby verschwand, nur um kurz darauf mit einer Schale klaren Wassers, einem Stück Seife und einem Handtuch neben ihm aufzutauchen. Harry wusch sich die Hände und bedankte sich bei Dobby, bevor er die Küche wieder verließ.

Es wurde Zeit, mit Professor Snape über seine Erlebnisse zu reden. Es war gerade noch Unterricht und er könnte eventuell in seinem Büro sitzen. So suchte Harry die geheime Passage auf und machte sich kurz bemerkbar. Er horchte an der Tür und öffnete diese. Das Regal schwang zur Seite und Professor Snape sah auf.

\enquote{Potter. Was verschafft mir die Ehre?}, fragte er.

\enquote{Ich habe mir Ihre Erinnerungen angesehen.}

\enquote{Ah, ja}, gab er emotionslos zurück.

\enquote{Ich habe eine Frage.}

\enquote{Ich werde Ihnen keine Fragen zu Ihrer Mutter, geschweige denn Ihrem Vater, beantworten.}

\enquote{Das will ich auch nicht wissen. Ich möchte etwas anderes wissen. Eigentlich zwei Dinge. Eine Sache zur Erinnerung und eine zu einem ähnlichen Thema. \gst Als Sie mit meiner Mutter im Zug saßen, war da noch eine andere Person anwesend.} Harry drückte es als eine Tatsache aus, da er wusste, dass Snape wie üblich sagen würde, er hätte die Erinnerung doch selbst gesehen, sollte er fragen, ob noch eine weitere Person anwesend gewesen wäre. \enquote{Wissen Sie, wer es war?}

Snape stutzte. Solch eine Frage zu einer seiner Erinnerungen kam ihm doch etwas merkwürdig vor. Er durchforstete sein Gedächtnis, bis er zu dem Zeitpunkt kam, indem er Lily im Zug gegenüber saß. Sofort zog sie ihn in seinen Bann. Er musste sein Gesicht gedanklich abwenden und im Abteil umherschauen. Aber außer einer verzerrten und einfarbigen grauen Masse in Menschengestalt, saß dort niemand.

\enquote{Nein}, antwortete er.

\enquote{Auch nichts Verschwommenes, menschenähnliches?}, fragte er nach.

Snapes Auge zuckte kurz und Harry empfing ein Bild aus seinen Gedanken. Die Figur, so wie Snape sie gesehen hatte, und nicht, wie man sie in einem Denkarium sah. Doch die Konturen waren nicht klarer oder schärfer.

\enquote{Nein}, antwortete er abermals.

Doch Harry wusste es bereits. \enquote{Noch eine Frage, dann sind Sie mich los.}

\enquote{Schön. Dann habe ich endlich Ruhe vor Ihnen}, kam es wieder sarkastisch. Immer wieder brach es bei ihm durch. Er durfte ihn nicht zu sehr an ihn heranlassen. Er hatte immer noch eine Aufgabe.

\enquote{Wissen Sie, ob es einen Unterschied gibt, ob man eine Erinnerung mit dem Finger, oder dem Zauberstab in einem Denkarium umrührt?}

\enquote{Wieso glauben Sie, dass ich so etwas weiß? Immerhin hat unser Schulleiter eines der wenigen noch existierenden Denkarien, weltweit.}

\enquote{Ein einfaches Nein hätte es auch getan. Danke Professor und gute Nacht. \gst Halt, da fällt mir noch was ein. Kurz vor Ihrem Tod hat ihnen da meine Mutter ein Geheimnis anvertraut? Etwas, was Sie sonst niemandem erzählt hatte?} Snape sah ihn einfach nur an. \enquote{Etwas über ein Geschwisterchen?}

\enquote{Sie hatte mal erwähnt, dass sie sich noch ein weiteres Kind wünscht, mehr nicht. \gst Gute Nacht.}

Harry bedankte sich. Er wusste, dass er jetzt gehen musste. Dann verabschiedete er sich und ging. Zurück im Gang sinnierte er: \gedanke{Entweder hatte er bisher keine Zeit, den Gang zu verschließen, oder er konnte es nicht. Oder er möchte, dass ich ständig zu ihm kann, falls ich etwas wissen möchte. Andererseits, wenn ich mich ohne triftigen Grund dort aufhalten sollte, gibt es Strafarbeiten, oder ich fliege von der Schule.}

Nachdem er den Gemeinschaftsraum durchlaufen hatte und in seinem Zimmer war, legte er sich ins Bett und dachte nach. \gedanke{Sprouts Aufsatz kann ich morgen früh machen, den für Snape habe ich schon fertig. Und für Verwandlung habe ich noch kein Thema. Aber darüber kann ich mich morgen mit Hermine unterhalten. Vielleicht fällt ihr etwas ein. Meinen VgddK-Aufsatz werde ich Morgen auch noch beenden. \gst Und was meine Vision von meinen Eltern anbelangt, da bin ich so schlau wie zuvor. Schade, dass keiner weiß, ob meine Mutter schwanger war.} Er stand auf und ging wieder nach unten.

Dean überredete ihn zu einer Runde Spreng-Schnipp-Schnapp. Nach vier verlorenen zwei gewonnen Runden und angesengten Augenbrauen wollte Dean noch seine Schachkenntnisse verbessern und spielte gegen Harry.

Als Harry müde wurde, bedankte er sich und bereitete sich fürs zu Bett gehen vor. Er stieg in sein Himmelbett und ließ die Vorhänge offen.

Ron, der ebenfalls schon im Bett lag, meinte: \enquote{Ich habe mir mal ein paar Punkte von deiner Karte angesehen. Du weißt schon\abs Auf jeden Fall gibt es eine Tür im zweiten Kellergeschoss, die ich nicht aufbringe. Du kannst ja mal schauen.}

Harry nickte, sodass Ron wusste, er habe ihn verstanden. Danach drehte er sich zur Seite und fiel in einen traumlosen Schlaf.

Am nächsten Tag stand er um fünf Uhr bereit, um seine beiden neuen Mitschüler für die DA zu empfangen. Pünktlich kamen sie an und schauten sich erstaunt um. Im Raum lagen Kissen auf dem Boden, Übungspuppen standen herum und ein großer alter Spiegel stand in der Ecke. Poster mit verschiedenen Figuren und Bewegungen hingen an der Wand.

\enquote{Schön, dass ihr da seid, dann können wir ja anfangen.}

\enquote{Wo sind denn die anderen?}, fragte Adrian.

\enquote{Die kommen erst in einer Stunde. Ich habe euch schon jetzt hier herbestellt, damit ich weiß, was ihr könnt.} Damit zog er seinen Zauberstab. \enquote{Und, um euch auf Spur zu bringen. Die ersten Termine werden etwas stressig für euch.}

Dann warf er einen Verwirrungszauber auf die beiden. Er trat um sie herum und wartete. Sichtlich geschockt und verwirrt von der Aktion zogen sie ihre Zauberstäbe und zielten auf den vermeidlichen Harry vor ihnen. Doch der Zauber hatte keine Wirkung. Auch schien sich der Harry vor ihnen kaum zu bewegen. Er machte keine Anstalten auszuweichen, oder sonst eine Aktion gegen sie zu starten.

\enquote{Das ist ein einfacher Verwirrungszauber}, hörten sie plötzlich eine Stimme hinter sich. Da sie die Position mit ihren Ohren recht gut zuordnen konnten, drehten sie sich um und sahen in die Richtung, aus der das Geräusch kam. Langsam wurde er sichtbar und die beiden zögerten nicht und versuchten ihn zu entwaffnen.

Harrys Zauberstab zog nur leicht an seiner Hand, sodass er keine Mühe hatte ihn festzuhalten. Der Expelliarmus war einfach zu schwach. Dann führte Harry ihn aus und hatte kurz darauf die beiden Zauberstäbe in seiner Hand. Er löste den Zauber auf und gab die Stäbe wieder zurück.

\enquote{Den werden wir als Erstes üben. Danach werden wir Vergrößerungs- und Verkleinerungszauber durchnehmen. Die solltet ihr in einer Stunde beherrschen, denn das, was ich heute geplant habe, baut darauf auf. Die anderen haben mit diesen Zaubern schon viel Erfahrung, also strengt euch an.}

Zuerst dauerte es eine Weile, bis die beiden begriffen hatten, wie man die Zauber richtig ausführt, aber dann klappte es recht gut. Adrian war damals zu jung, um in das Inquisitionskommando aufgenommen zu werden; er hatte es nicht einmal versucht, und Katharina wollte nicht. Sie hatte keine Lust auf Streifgänge durchs kalte abendliche Schloss. Kurz vor sechs waren beide erschöpft aber glücklich.

\enquote{Die nächsten drei vier Termine werden wir noch vorher üben.}

Die beiden nickten.

Dann schlug die Uhr Sechs und die ersten der DA kamen herein. Was keiner der Gruppe wusste war, dass die Hausgeister davon Wind bekommen hatten und sich in einer Bücherreihe versteckten. Lediglich deren Augen sah man über den Büchern, falls man wusste, wonach man suchte. Geduldig und mit Spannung beobachteten sie das Treffen.

Katharina und Adrian standen neben Harry und sahen auf den Eingang. Nach und nach kamen die anderen der DA in das Klassenzimmer. Katie und Susan waren die einzigen, die sie anlächelten, was den beiden die Anspannung etwas nahm. Sie würden nachher mit ihnen üben. Der Rest war vor allem eher desinteressiert. Nur wenige zeigten eine noch offene Ablehnung. Darunter auch Zacharias Smith. Der sah immer noch nicht begeistert aus. Rons Gedanken konnte man momentan gar nicht aus seinem Gesicht deuten.

\enquote{Ist es jetzt also so weit, dass man Slytherin aufnimmt?}, fragte Zacharias provokant.

\enquote{Ich finde es beispiellos, wie rassistisch und arrogant Sie gegenüber allen Slytherin sind}, sagte eine Stimme aus Richtung des Bücherregals. Der Geist des blutigen Barons kam hervor und sah Zacharias bedrohlich an. \enquote{Wie können Sie alle Schüler meines Hauses so hassen? Ist es Ihnen egal, wer hinter der Uniform steckt?} Beständig flog er auf ihn zu, was Zacharias dazu veranlasst, langsam rückwärtszugehen. \enquote{Merken Sie sich die Gesichter bei der Auswahlzeremonie und fangen dann an sie zu hassen, sobald der Hut den Namen Slytherin ausruft?} Jetzt stand Zacharias mit dem Rücken an der Wand und der blutige Baron schwebte nur wenige Zentimeter vor ihm. Er hielt seine Hände links und rechts von seinem Gesicht und es sah so aus, als ob er sich an der Wand abstützte. Seine Hände waren dabei nur wenige Zentimeter in der Wand. Die drei anderen Hausgeister kamen nun ebenfalls hervor und sahen ebenso beleidigt zu Zacharias und den anderen, die den beiden Slytherin ablehnend gegenüber standen. \enquote{Vielleicht sollte ich dafür sorgen, dass Sie eine Weile zu den Slytherins gesteckt werden, damit Sie mal am eigenen Leib zu spüren bekommen, wie es ist, von drei Häusern gehasst zu werden.}

Zacharias wollte gerade etwas dagegen sagen, doch der blutige Baron schnitt ihm mit einer Handbewegung das Wort ab. Er gab ihm durch seinen Kopf hindurch eine schallende Ohrfeige. Danach schwebte er durch ihn hindurch und verließ den Raum. Zacharias’ Gesichtsfarbe wurde blauer und es bildeten sich Eiskristalle. Er konnte nicht mehr blinzeln.

\enquote{Geschieht Ihnen recht}, sagten die drei anderen Hausgeister und schwebten neben Zacharias durch die Wand. Hermine und Alicia kamen auf ihn zugestürmt und versuchten, die Erfrierungen zu lindern und Zacharias zu helfen. Dann brachten sie ihn zum Krankenflügel.

\enquote{Und wir machen weiter}, sagte Harry. \enquote{Heute werden wir uns eine andere Schicht der Magie ansehen. Sucht euch einen Platz und jemanden zum Üben.}

Katie und Susan sahen sich kurz an, nickten einander zu und schnappten sich Adrian und Katharina. Die anderen teilten sich entsprechend auf und bildeten mal wieder neue Paarungen, um möglichst verschiedene Angriffstechniken und Bewegungen zu kennen.

Nachdem sich alle hingesetzt hatten, begann Harry sich zu konzentrieren und seine Augen zu schließen. Dann sagte eine Stimme hinter den anderen: \enquote{Illusionen und Täuschungen sind wichtige Techniken.} Alle drehten sich um und sahen Harry. Als sie wieder an die alte Stelle blickten, war er nicht mehr da. Der andere Harry stand auf und lief durch den Raum. \enquote{Tarnen und Täuschen ist wichtig, wenn man sich verteidigen will. Nicht nur für Auroren, auch, wenn man sich zu Hause verteidigen will. Es sind schwere Zeiten, in denen wir sind und denen wir entgegensteuern. Wenn ihr so etwas könnt, dann könnt ihr eure Familie gegenüber den Todessern verbergen. \gst Fangen wir an.}

\trenn

Lucius und Severus saßen in Lucius’ Arbeitszimmer und hatten den Raum magisch abgeschirmt.

\enquote{Und, wie war es bei deiner Familie?}, fragte ihn Severus ohne Umschweife, nachdem sich beide hingesetzt hatten.

Überrascht sah Lucius sein Gegenüber an. Unsicher darüber, was er sagen sollte, fragte er nach: \enquote{Woher glaubst du zu wissen, dass ich meine Familie, die mich verlassen hat, besucht habe?}

\enquote{Frederick}, sagte Severus.

Lucius fiel ein Stein vom Herzen. \enquote{Es war schön, mal wieder alle zu sehen. Weißt du Severus, als mir Tamara von diesem Brauch der Muggel erzählte, musste ich mir einen bösen Kommentar verkneifen. Der Dunkle Lord geht mir langsam\abs} Den Rest des Satzes sprach er nicht mehr aus.

Severus saß emotionslos da und sah ihn an. \enquote{Willst du nicht auch hier raus? Gefangen im eigenen Haus. Wie lange machst du das noch mit?}

\enquote{Der Dunkle Lord lässt mich nicht gehen. Wenn ich weg bin, dann kommen die anderen nicht raus. Als ich die Sachen für den Dunklen Lord besorgen musste, stand ich unter Beobachtung.} Dann stutzte er. \enquote{Wieso ist mir das nicht gleich aufgefallen? Die ganzen vier Stunden war ich praktisch nicht auffindbar. Wieso hat er ihm dann berichtet, dass ich meine Aufgabe erfüllt habe?} Severus streckte einen Zeigefinger in die Luft. \enquote{Du warst das? Was hast du gemacht? Imperius?}

Severus schüttelte den Kopf. \enquote{Falsche Erinnerungen. Den Zauber habe ich in einem Buch gefunden. Man kann zwar, wenn man geschickt ist, herausfinden, dass man manipuliert wurde, aber die originalen Erinnerungen wieder herzustellen ist schwierig. Und da unser Freund keine Veranlassung hat, nachzuschauen\abs}

Lucius grinste. \enquote{Bin ich froh, dass ich dich habe. Wenn ich daran denke, wie ich dich damals angeworben habe. Ich muss mich heute noch bei dir dafür entschuldigen.}

\enquote{Das hast du schon getan. Immerhin durfte ich Dracos Pate werden.}

\enquote{Das lag nicht an mir. Da hatte Narcissa ihre Finger im Spiel. Ich war anfangs dagegen. Aber mittlerweile bin ich auch darüber froh. So hatte er auf Hogwarts immer jemand, dem er vertrauen konnte. Da bin ich dir schon wieder etwas schuldig.}

Severus nickte nur. \enquote{Wie sieht es jetzt aus? Verlässt du das Manor?}

\enquote{Wie denn? Wenn ich versuche zu apparieren, wird der Dunkle Lord mir sofort auf den Fersen sein. Dann verbringe ich die Tage und Nächte in meinem eigenen Verlies.}

\enquote{Darüber solltest du dienstags mal laut nachdenken.}

\enquote{Wieso dienstags? Da ist doch immer mein Schach\aabs Du meinst Frederick? Was kann der schon\abs Der appariert hier einfach rein und raus., er ist ja Tamaras Pate. Auch wieder meine Frau. Du meinst, er könnte mir dabei helfen? Aber wie?}

\enquote{Hast du nicht mitbekommen, was er mit Bellatrix gemacht hat, als sie auf deine Tochter eingewirkt hat?} Lucius schüttelte nur den Kopf. \enquote{Ohne Zauberstab hat er so etwas wie den Cruciatus auf sie geworfen. Kurz danach ist er mit beiden Kindern gegangen. Er hat sie in Sicherheit gebracht.}

\enquote{Dass er sie mitgenommen hatte, weiß ich. Auch, dass es zu ihrer Sicherheit war, aber nicht warum genau. Bellatrix hat mir\abs Klar, wenn sie es selbst betrifft.} Lucius dachte nach. \enquote{Ich habe das Gefühl, dass es bald gegen Hogwarts geht. Wenn es so weit ist, dann werde ich wohl den Entschluss fassen, zu gehen. Alle, die im Manor sind, werden festgehalten. Leider gilt das nicht für den Dunklen Lord.}

\enquote{Das ist wohl wahr! Und ich kenne weder eine Möglichkeit, ihn hier festzuhalten, noch jemand, der es vermöchte. Vielleicht schafft er es auch, diesen Zauber aufzuheben. Dann wären die anderen wieder frei.}

\enquote{Meinst du, Frederick schafft es trotzdem durch? Falls du eingeschlossen sein solltest?}

\enquote{Ich habe ihn das ganze Jahr über in Aktion erlebt. Sein Wissen über Magie ist\abs eigenartig.}

\enquote{Was meinst du mit eigenartig?}

\enquote{Wie soll ich es sagen.} Severus überlegte. \enquote{Weißt du, was er den Siebtklässlern in der ersten Stunde gezeigt hat?} Lucius schüttelte den Kopf. Doch inmitten der Bewegung hielt er inne und nickte. \enquote{Dämonenfeuer. Doch es verhielt sich nicht so, wie du und ich es kennen. Wie wir es gelernt haben.} Severus erschuf mit seinem Stab eine kleine brennende Schlange. Lucius zuckte zuerst, beruhigte sich dann aber wieder, als Severus keinerlei Anstalten machte, sich darum zu kümmern, außer der kleinen Schlange zuzusehen, wie sie über den Tisch kroch. \enquote{Alles eine Frage der Intention, sagte er. Dieses Feuer hier ist harmlos, obwohl es Dämomenfeuer ist. Eigenartig, nicht? Deswegen nennt er es wohl lebendiges Feuer.}

\enquote{Hat er außer dieser Aktion noch etwas gemacht, das das Wort eigenartig rechtfertigt?}

\enquote{Er hat vor der gesamten Schule über die unverzeihlichen Flüche referiert. Er beantwortet Fragen zu dunkler Magie, wo selbst ich keine Antwort geben würde.}

\enquote{Meinst du nicht könnte, Severus?}

\enquote{Das manchmal auch. Also habe ich ihm auch mal einige Fragen gestellt. Entweder er hat sie gleich beantwortet, oder er kam ein paar Stunden später mit einer Antwort. Sowohl dunkle, als auch helle Magie.}

\enquote{Draco und Tamara sagten mir, dass es keine dunkle Magie gibt. Ebenso keine helle. Ehrlich gesagt, habe ich das nicht so recht verstanden. \gst Warte. Es kommt auf die Intention an.}

Severus nickte.

\trenn

Als das heutige Treffen vorbei war, drückte Katharina Harry noch ein Pergament in die Hand, bevor sie, wie die anderen auch, ging. Harry entfaltete es, als er nur noch mit Ron, Hermine und Ginny alleine war.

\begin{brief}
Das ist mein Familienwappen, Harry. Vielleicht hilft es dir bei deiner Suche.
\signumspace
Katharina
\end{brief}

Darunter war das Bild eines Mondes. Er war dreiviertel gefüllt. Darüber lag ein Zauberstab, der nur über der hellen Seite des Mondes zu sehen und quer über dem Mond gezeichnet war. Einen Schritt weiter brachte ihn das vielleicht. Er musste nur die Mondphase herausfinden, dann konnte er sich darauf konzentrieren, einen Zugang zu finden. Er teilte den dreien seine Gedanken mit und auch das, was er herausgefunden hatte.

Dann lief er nachdenklich durch das Schloss.
Er brauchte etwas Zeit für sich. So ging er in den Aufzug und fuhr an dieselbe Stelle, an der er Fawkes getroffen hatte, saß auf dem kleinen Balkon an der Westseite des Schlosses, hatte den Steinboden angewärmt und ließ die Beine durch das Geländer baumeln. Er brauchte Zeit nachzudenken und hatte einen Platz gefunden, der nur ihm bekannt war. Er hatte schon mehrmals diesen Platz aufgesucht, an dem er Ruhe und Entspannung erfahren hatte, hatte seine Augen geschlossen und ließ die Magie durch ihn hindurch fließen. Dann dachte er an die Worte, die noch immer deutlich in seinem Kopf zu hören waren.
% "Nach meiner Grösse beurteilst du mich? Tust du das? Aber das solltest du nicht, denn die Macht ist mein Verbündeter, und ein mächtiger Verbündeter ist Sie. Das Leben erschafft Sie, bringt Sie zur Entfaltung. Ihre Energie umgiebt uns, verbindet uns mit allem. Erleuchtete Wesen sind wir, nicht diese rohe Materie. Du musst Sie fühlen die Macht die dich umgibt, hier, zwischen dir, mir dem Baum, den Felsen dort, allgegenwärtig ja, selbst zwischen dem Sumpf und dem Schiff."

\accentuate{Die Magie ist mein Verbündeter, und ein mächtiger Verbündeter ist sie. Ihre Energie umgibt uns, verbindet uns mit allem. Erleuchtete Wesen sind wir, nicht diese rohe Materie. Sie müssen sie fühlen, die Magie die sie umgibt, hier, zwischen Ihnen, mir, dem Baum, den Felsen dort, allgegenwärtig ja, selbst zwischen dem See und dem Stein auf seinem Grund.}

Er fühlte sich leicht und warm. Spürte, wie ihn ein warmer Energiestrom durchflutete. Er war vollkommen entspannt. Er hörte die Vögel, die am Himmel über ihm flogen, spürte den Wind auf seiner Haut und einen Gecko neben ihm atmen. Er sah durch seine geschlossenen Augen die Vögel auf ihn und auf das Geländer zufliegen, um sich dort, oder in seinem Schoß niederzulassen. Er öffnete die Augen und nahm einen Vogel auf seine Hand. Dieser schien keine Angst vor ihm zu haben. Beide Lebewesen betrachteten sich gegenseitig. Der Vogel wusste wohl, dass Harry ihm nichts tat.

Die Wand hinter ihm ging auf und zwei Gestalten traten erstaunt ins Freie.

\enquote{Hallo Hermine, Ron.}

\enquote{Woher\abs} stammelte Hermine.

\enquote{Woher wusstest du, dass wir das sind}, fragte Ron.

\enquote{Ich habe es gespürt}, sagte Harry. Der Vogel saß noch immer in seiner Hand. Hermine setze sich rechts von Harry neben den Gecko und Ron auf die andere Seite. Einige Vögel flogen davon und auch der Gecko verzog sich wieder, doch der Vogel in seiner Hand und wenige andere blieben und betrachteten die Neuankömmlinge neugierig.

\enquote{Was ist mit den Vögeln los, Harry?}, fragte Ron.

\enquote{Das ist ein Zauber, Ron}, sagte Hermine.

\enquote{Nein, kein Zauber. Magie.}

\enquote{Das meinte ich, Harry}, sagte Hermine.

Jetzt sah er Hermine an, nahm ihre Hand in seine und schaute ihr tief in die Augen.

\enquote{Harry}, sagte sie, als sie leicht rot wurde.

Er legte ihr den Vogel in ihre Hand und lies Hermines Hand wieder los. \enquote{Fühlt sich so ein herbeigezauberter Vogel an?}, fragte er sie.

Hermine sah den Vogel an und strich ihm über sein Gefieder.

\enquote{Nein}, sagte Hermine verwundert.

Dann sah er wieder durch das Geländer in die Ferne. \enquote{Wie seid ihr eigentlich hier hergekommen?}, fragte sie Harry nun.

\enquote{Wir machten uns Sorgen um dich}, sagte Ron. \enquote{Wir wussten nicht, wo du warst, also haben wir auf der Karte nachgesehen. Du warst\abs Harry du warst\abs außerhalb des Schlosses.}

Und Hermine fuhr fort: \enquote{Du warst einfach knapp außerhalb des Schlosses. Westflügel. Dritter Stock.}

Jetzt wusste Harry zumindest, dass ihn die Karte immer noch anzeigte. Er war immer noch entspannt und das warme Gefühl durchdrang ihn immer noch.

\enquote{Wir wollten irgendwie zu dir gelangen, fanden aber keinen Weg. Wir dachten, du seist in Schwierigkeiten. Also sind wir durchs Schloss gelaufen auf der Suche nach einem Lehrer. Wir fragten den Ersten, dem wir begegnet sind. Wir drückten unsere Sorge aus und sagten, dass wir zwar wissen, wo du bist, aber nicht, wie wir zu dir kommen können, oder wie du dort hingelangt bist. Harry, wir hatten uns Sorgen um dich gemacht.}

Und Ron erzählte weiter: \enquote{Wir haben ihm gesagt, dass du im dritten Stock im Westflügel bist, knapp außerhalb des Schlosses. Wir konnten uns nicht erklären, wie du dort hingekommen sein könntest. Der Professor sah uns erst ungläubig an, doch dann überlegte er. Er fing an leicht zu schmunzeln und stand dann auf. Er winkte uns zu und ging mit uns um ein paar Ecken. Dann drückte er einen Stein in der Wand, worauf diese sich teilte. Er gab uns zu verstehen, dass wir in den kleinen Raum eintreten sollten. Dann lehnte er sich herein und drückte auf einen kleinen Knopf. Die Wand verschloss sich, als er sich wieder aufrichtete. Wir beide erschraken, als der Boden zu vibrieren begann. Dann öffnete sich die Wand wieder und du warst da.}

Harry grinste. Er schloss wieder seine Augen und ließ die Magie ihn weiter durchströmen. An keinem anderen Ort im Schloss hatte er dieses Gefühl erhalten, selbst dann nicht, wenn es absolut ruhig war und er sich entspannen konnte. Jetzt endlich wusste er, wozu dieser Ort da war.

\enquote{Harry, die Wand}, schrie Ron plötzlich und schüttelte Harry am Arm. Der erschrak, ließ sich aber sonst nicht aus der Ruhe bringen.

\enquote{Harry}, ermahnte ihn Hermine, doch er reagierte nicht.

\enquote{Wie kannst du nur so ruhig sein, Harry}, brauste Ron auf.

\enquote{Ich nehme an, Harry wird seine Gründe haben}, sagte Hermine, sah Harry dabei an und setzte sich ebenfalls so wie Harry hin, schloss ihre Augen und entspannte.

Doch Ron ließ sich nicht beruhigen. Er stand vor der Wand und versuchte rein zu kommen. Harry seufzte resigniert. Er spürte den Stein fast auf seiner Hand, als er sie leicht nach vorne schob. Der Stein in der Wand versank und kurz darauf öffnete die Wand sich. Ron stolperte einige Schritte vorwärts, da er sich gegen die Wand gelehnt hatte. Im Inneren des Aufzuges wurde durch eine unsichtbare Hand ein Knopf gedrückt und die Wand schloss sich erneut.

Nun herrschte Ruhe.

\enquote{Wo ist Ron hin?}, fragte sich Hermine und schaute sich um.

\enquote{Wieder im Aufzug. Er hatte mich gestört.}

\enquote{Wie? Er hatte dich gestört.} Sie sah auf Harrys Hände. \enquote{Ohne Zauberstab?}, fragte sie ungläubig.

Harry öffnete die Augen und nahm Hermines Hand in seine. \enquote{Wie läuft es zwischen dir und Ron?}, fragte er sie. Hermine wurde warm um ihr Herz. \enquote{Es wird immer besser. So langsam kann er richtig entspannen, wenn wir alleine sind. Seine warmen Lippen fühlen sich gut an.} Dann sah sie auf ihre Hand, die Harry noch immer festhielt. Sie wollte sie zurückziehen, doch Harry hielt sie fest. Er griff an ihr Handgelenk und schloss sanft ihre Hand. Mit ihren Fingern bildete sie einen kleinen Hohlraum. Jetzt lag seine Hand sanft auf der Außenseite ihrer Finger. Dann spürte sie in ihrer Hand etwas Weiches. Harry nahm beide Hände von ihr und legte sie zurück in seinen Schoß.

Hermine öffnete langsam ihre Hand und darin lag ein kleiner Minimuff. \enquote{Harry}, sagte Hermine ganz erstaunt. \enquote{Wie hast du das gemacht?} Sie sah ihn erstaunt an.

Ihre Augen sahen einfach umwerfend aus.

\enquote{Du weißt, dass Ron sich glücklich schätzen kann, dich zu haben?}

\enquote{Lenk' jetzt nicht ab, Harry. Wie machst du das?}

\enquote{Ich weiß es nicht\abs noch nicht. An diesem Ort fällt es mir so leicht, das zu tun. Ich habe es schon in meinem Zimmer versucht, aber dort klappt es nicht annähernd so gut.}

\enquote{Du kannst so etwas in deinem Zimmer? Ohne Zauberstab?}

\enquote{Nicht so gut wie hier}, sagte Harry entschuldigend.

Hermine bekam große Augen. \enquote{Ich habe nicht einmal Dumbledore so etwas ohne Zauberstab machen sehen. Wir sollten zu ihm und es ihm sagen.}

\enquote{Und dann?}, fragte Harry. \enquote{Was machen wir dann? Werde ich Schulleiter?}

\enquote{Harry, das ist nicht witzig.}

\enquote{Nein ist es nicht, aber was will Dumbledore machen?}, fragte er.

\enquote{Lass uns erst einmal zu ihm gehen.}

Hermine stand auf und stand nun vor der Wand. Also blieb Harry nichts anderes übrig. Sie würde ihn physisch so lange traktieren, bis er nachgab. Er stand auf und sah den Stein in der Wand an, der sich nach innen bewegte und so die Wand zum Öffnen brachte. Hermine sah ihn mit leicht offenem Mund an. Im Inneren des kleinen Raumes drückte er den Knopf von Hand und nach einer kurzen Fahrt waren sie noch etwa sechzig Meter vom Büro des Schulleiters entfernt.

\enquote{Woher weißt du von diesen Aufzügen?}, fragte Hermine ihn, als sie unterwegs waren.

\enquote{Professor Elber ging mit Dumbledore zum Astronomieturm, als ich dort hin wollte. Ich lief fast in sie hinein. Also bot Dumbledore mir an, ihn zu begleiten, da ich eh schon spät dran war. Dann bog Professor Elber ab und zog uns mit sich, als wir sagten, dass es da nicht lang ging. Dumbledore war ebenso erstaunt über diese Aufzüge wie ich. Ich habe leider noch nicht alle Wege erkundet. Aber sobald ich alle habe, sage ich Ron und dir Bescheid. \gst Äh, wollte ich Ron und dir Bescheid sagen}, korrigierte er sich. \enquote{Aber sag mal, Hermine, woher weißt du davon?}

\enquote{Ron und ich sind auf Professor Elber gestoßen. Er hat uns einfach in den Aufzug gesteckt und uns direkt zu dir geschickt.}

Sie gaben dem Wasserspeier das aktuelle Passwort, das Harry durch Zufall in Erfahrung gebracht hatte und klopften kurz darauf an die Tür des Schulleiters.

\enquote{Herein}, erklang Dumbledores Stimme durch die Tür. Beide traten ein und sahen bereits Ron und Professor Elber. Ron saß auf einem Stuhl und Professor Elber betrachtete gerade interessiert ein paar Gegenstände in einem der Schränke in Dumbledores Büro.

Dumbledore sah beide fragend an. Dann zog Hermine Harry an seinem Arm und setzte ihn unsanft auf einen Stuhl neben Ron. Nun saß er gegenüber Dumbledores Tisch. \enquote{Professor? Harry möchte ihnen etwas erzählen.}

Professor Elber drehte sich um und lehnte jetzt mit dem Rücken an einem der Schränkchen.

Dumbledore wartete ab, was Harry auf dem Herzen lag. Doch dieser wusste nicht, wie er anfangen sollte.

\enquote{Zeig es ihm}, sagte Hermine. Noch immer kam keine wirkliche Reaktion von Harry. \enquote{Zeig es ihm}, fuhr ihn Hermine jetzt deutlicher an.

Harry nahm langsam seine Hand in seinen Schoß und drehte die Innenseite nach oben. Dann krümmte er seine Finger und schloss sie. Er hatte jetzt eine kleine Kuhle in seiner Hand. Er konzentrierte sich und erfuhr wieder das warme Gefühl. Es schien ihm unglaublich leicht von der Hand zu gehen. Als er sie wieder offen hatte, lag in seiner Hand ein Minimuff. Dann sah er zu Dumbledore. Er hatte einen Gesichtsausdruck, wie ihn Harry noch nie gesehen hatte.

\enquote{Harry}, war alles, was Dumbledore herausbrachte und als er sich danach wieder in seinen Stuhl fallen ließ, den er verlassen hatte, als Harry seine Hand öffnete und Dumbledore den Minimuff sah, musste er schlucken.

Dieser rollte nun in Harrys Hand vergnügt umher und gab surrende und schnurrende Geräusche von sich.

Plötzlich begann Harrys Kopf zu schwirren. \enquote{Mir wird schlecht und schwarz vor Augen}, sagte er noch und sackte dann zusammen. Er spürte noch, wie Hermine ihn aufhielt und verhinderte, dass er mit seinem Kopf auf dem Boden aufschlug. Er konnte nichts mehr sehen, nichts mehr schmecken, nichts mehr riechen und nichts mehr spüren, konnte nur noch hören.

\enquote{Holt Madame Pomfrey.}

\enquote{Nein. Sie kann nichts für ihn tun. Er muss sich von alleine wieder erholen.}

Plötzlich wurde ihm warm an einer Stelle, an der normalerweise seine Brust war. Er spürte keine Hand, aber da musste eine sein. Er merkte nur, wie ein Handflächen-großes Stück auf seinem Bauch warm wurde.

\enquote{Ah, was ich vermutet hatte.} Es war wieder Professor Elber. Er hatte schon verhindert, dass man Madame Pomfrey rief. \enquote{Ja, das ist gut. Sehr gut. Er wird schwächer.}

\enquote{Was soll daran gut sein, dass Harry schwächer wird}, schrie Hermine jetzt. Harry hatte sie noch nie so mit einem Lehrer sprechen sehen. Nein hören, korrigierte er sich.

\enquote{Ich meine nicht Harry. Ich meine\abs}, doch er verstummte. \enquote{Er muss erst einmal wieder seine Kräfte finden. Der abgetrennte Teil der Magie ist rastlos in seinem Körper. Das kann ich spüren.} Die Wärme bewegte sich nun höher auf seinen Brustkorb zu. \enquote{Die Magie muss erst einmal Ruhe finden und sich mit jeder Faser, ja jeder Zelle seines Körpers verbinden. Er wird bald wieder sehen können. Seine Sinne werden wieder zurückkehren. Nicht wahr Harry?}

\gedanke{Ja}, dachte er. \gedanke{Ich beginne langsam wieder meinen Körper zu spüren.}

\enquote{Sie sollten bis zum neuen Schuljahr diesen Ort meiden, Harry. Kehren Sie nicht mehr dorthin zurück}, sprach wieder Professor Elber.

Harrys Wahrnehmungen wurden immer besser. Er konnte jetzt wieder schemenhafte Gestalten erkennen. Dann \gst war er wieder da.

\enquote{Wie geht es dir?}, fragte ihn Hermine, während Ron ihn leicht unsicher ansah.

\enquote{Ganz gut}, sagte er zu Ron gewandt. \enquote{Leichte Kopfschmerzen.}

Dann sah er zu Professor Elber. \enquote{Er ist in mir, richtig? Ein Teil von ihm.} Dann sah er zu Dumbledore. \enquote{So wie das Tagebuch. Ein Teil von Voldemort ist in mir.}

Ron und Hermine sahen Harry ungläubig an. \enquote{Du warst weggetreten, Harry}, sagte Ron.

\enquote{Du fantasierst}, fügte Hermine hinzu.

Harry sah wieder zu Professor Elber und richtete sich auf. Professor Elber ging in die Hocke und sah Harry an. \enquote{Wie kommen Sie darauf?}

\enquote{Ich habe es irgendwie gespürt als ich da saß und die Magie mich durchströmt hat. Ein kleiner Teil von mir blieb kalt, während der Rest sich erwärmte. Ein kleiner Teil, der mich an Voldemort erinnerte, als ich versuchte ihn zu erfassen. Außerdem habe ich schon einmal etwas in mir gespürt. Ich konnte es aber nicht richtig zuordnen.}

Professor Elber zeigte ein sanftes Lächeln. \enquote{Sie wissen also, was mit Ihnen los ist? Dass Voldemort deshalb nicht sterben kann?}

Harry nickte. \enquote{Erst das Tagebuch, dann ich. \gst Meinen Sie er hat noch mehr \gst Seelenabspaltungen vorgenommen?}

Hermine und Ron verschlug es die Sprache. \enquote{Harry, wie kannst du nur so ruhig über solche Sachen reden. Das ist gefährliche schwarze Magie. Woher weißt du von diesen\abs diesen Horkruxen.}

Hermine hätte sich am liebsten selbst auf die Zunge gebissen. Doch Dumbledore schien sie nicht gehört zu haben. Bei Elber hatte sie das Gefühl, dass, falls er es gehört hatte, er einfach darüber weggehen würde. Das war merkwürdig. Er war schwer einzuschätzen. Irgendwie interessierte es ihn nicht, wenn ein Schüler etwas wusste, das er nicht wissen sollte. Das hatte Hermine schon recht bald bemerkt.

\enquote{Wie kann ich ihn bekämpfen?}, fragte Harry nun.

\enquote{Tja}, antwortete Dumbledore.

\enquote{Lassen Sie es einfach über sich kommen, Harry. Sie sind mittlerweile so stark, dass Sie seine Magie anziehen. Wenn Sie die gesamte Magie des Seelenteiles aufgenommen haben, dann wird der Rest absterben und Voldemort wird, sofern er keine weiteren Horkruxe angefertigt hat, sterben können. Dann ist der Zeitpunkt gekommen, ihn zu töten}, machte Professor Elber weiter.

Harrys Kopf fuhr nach oben. Er ignorierte den aufsteigenden Kopfschmerz. Er krabbelte auf seinen Professor zu. Seine Nasenspitze war wenige Zentimeter von ihm entfernt. Wütend schrie er ihn an. \enquote{Ich soll ihn töten, stimmt’s? Es liegt an mir!} Wut spiegelte sich in seinem Gesicht.

\enquote{Was wollen Sie machen? Mich schlagen?}, fragte Professor Elber Harry provozierend und stieß ihm mit seinem Finger in die Brust. Das war für Harry zu viel. Er stürzte sich auf seinen Lehrer und gab ihm eine Ohrfeige.

\enquote{Harry!!!}, schrien Ron und Hermine außer sich vor Entsetzen.

Doch als Harry begriff, was er getan hatte, rollte er sich von ihm, in eine Fötus-Haltung und begann bitterlich zu weinen.

\enquote{Harry, du kannst doch nicht einfach so\abs}, doch Hermine verstummte, als ihr Professor Elber einen Arm auf ihre Schulter legte.

\enquote{Das musste sein, Hermine. Es beschleunigt das Ganze. Er muss seine Wut herauslassen.}

\enquote{Wie meinen Sie das?}, fragten nun Ron und Dumbledore.

\enquote{Der Prozess entzieht Voldemorts Seelenteil gerade eine Menge Magie. Es wird schwächer.}

\enquote{Aber}, begann Hermine wieder. \enquote{Er hat sie geohrfeigt.}

\enquote{Ich habe es auch darauf angelegt, oder etwa nicht?}

\enquote{Sie wollten, dass er das tut?}

\enquote{Das, oder etwas anderes. \gst Albus, ich glaube, wenn Harry sich beruhigt hat, dann schulde ich Hermine, Ron und Harry eine Erklärung.}

Nachdem sich Harry beruhigt hatte, saß er wieder auf einem Stuhl in Dumbledores Büro. Er hatte sich von seinem Zusammenbruch erholt.

Professor Elber fuhr mit seiner Erzählung fort. \enquote{Es ist wichtig, dass Sie das verstehen, Harry. Und es tut mir leid, dass Sie das erfahren müssen. Ich habe zwei Theorien, die ein Grund für ihren Zustand seien können. Zumindest bei der zweiten stimmt Albus mir zu.}

Professor Elber setzte sich nun auf den Stuhl gegenüber Harry. Ron und Hermine standen hinter Harry und legten jeweils eine Hand auf seine Schulter. Harry hielt sie fest.

\enquote{Ich muss dazu sagen, dass es sich um bloße Vermutungen handelt. Und ich mache Ihnen keinen Vorwurf, wenn Sie, nachdem ich fertig erzählt habe, sofort den Raum verlassen, weil sie wütend sind.}

Harry wurde leicht unwohl. Wenn schon sein Professor mit so einem Ausbruch rechnet, wie würde er dann wirklich reagieren?

\enquote{Meine erste Theorie ist, dass Sie ohne eine Spur von magischem Können auf die Welt gekommen sind. Sozusagen als Squib. Erst durch den missglückten Mord an Ihnen, bei dem ja ein Seelenteil Voldemorts unbewusst abgetrennt worden ist und das jetzt in Ihnen verweilt, sind Sie zu einem Zauberer geworden. Sie haben bisher auf die Magie in diesem Seelenteil zugegriffen, als sie zauberten. So langsam aber fängt ihr Körper an, diese Magie in sich aufzunehmen. Ein ganz normaler natürlicher Prozess. Das Seelenteil stirbt dabei ab, obwohl es sich verzweifelt versucht dagegen zu wehren. Deswegen auch die Ohnmachtsanfälle, welche in nächster Zeit noch zunehmen werden.}

Harry wurde ganz bleich im Gesicht. Hermine hielt sich die Hand vor den Mund und zog dabei ihre Hand von Harrys Schulter. Schuldbewusst legte sie sie gleich wieder zurück, um Harry Kraft zu geben, als sie das bemerkte. Ron keuchte nur.

\enquote{Die zweite Theorie, bei der Albus mit mir übereinstimmt, ist, dass Sie als ganz normaler Zauberer auf die Welt gekommen sind. Aber durch den versuchten Mord und dem besagten Seelenteil in Ihnen, haben Sie sich dessen Magie zunutze gemacht, da das Seelenteil Ihre Magie unterdrückt hatte. Erst langsam, jetzt da Ihre eigene Magie stärker wird, da sie sich durch das Unterdrücken des Seelenteiles in ihnen wehren kann, bekämpft es die fremde Magie und zieht sie zu sich herüber.}

Harry warf kurze Blicke zu seinem Schulleiter und sah dessen besorgte Blicke.

\enquote{In beiden Fällen wandert die Magie, egal welche, in Ihren Körper und verbindet sich auf natürliche Art und Weise mit Ihren Zellen.}

Harry würde am liebsten weinen. Wortlos stand er auf und verließ Dumbledores Büro. Er schloss die Tür hinter sich und setzte sich nun gegenüber dem Wasserspeier an die Wand auf den Boden, als er die Wendeltreppe herunter getreten war.

\enquote{Probleme?}, fragte ihn dieser.

Harry erschrak. Es war das erste Mal, dass ihn der Wasserspeier etwas fragte. Sonst kam von ihm nur das monotone \enquote{Passwort}.

Harry hörte leise Stimmen in seinem Kopf. Oder hörte er sie wirklich? \enquote{Er wird gleich wieder kommen, Hermine. Er braucht nur kurz Ruhe, um sich zu sammeln.} Harry meinte, Dumbledore gehört zu haben. \gedanke{Ja}, dachte er, \gedanke{ich brauche einfach nur kurze Ruhe.} Aber er kam nicht so weit.

\enquote{Probleme?}, fragte ihn der Wasserspeier erneut. Harry gab ihm eine kurze Zusammenfassung und sah ihn danach an. Der Wasserspeier schaute kurz nach links und dann nach rechts und ging dann auf Harry zu.

\enquote{Ihr könnt euch bewegen?}, fragte Harry ganz ungläubig.

\enquote{Na na, Harry. Du kannst mich ruhig duzen. Du brauchst mit mir nicht so zu reden, wie man es mit adeligen macht.}

\enquote{Nein}, antwortete Harry. \enquote{Ich habe euch alle gemeint. Alle Figuren Hogwarts, auf denen ein Zauber liegt.}

\enquote{Ja}, antwortete der Wasserspeier, \enquote{wir können reden. Aber wir sind nicht einzelne Geschöpfe, auf denen ein Zauber liegt. Wir sind eins. Wir sind das Schloss, seine Seele. Wir sind Hogwarts, bekommen seit Jahrhunderten die Sorgen und Probleme, aber auch die Freude und die Liebe, sowie den Hass innerhalb dieser Mauern mit. Das hat uns geprägt. Lass mich dir etwas von dem, was deine Mitschüler und du uns im Laufe der Jahrhunderte gegeben habt, zurückgeben.} Der Wasserspeier legte seine Hand auf Harrys Stirn. Sofort wurde ihm warm und sein Zorn verflog zunehmend. Als der Wasserspeier Stimmen hörte, ging er wieder an seinen Platz und erstarrte. Noch einmal bewegte er seinen Kopf und sah zu Harry. \enquote{Kein Wort zu irgendjemand.} Dann sagte der Speier nichts mehr.

Harry stand auf und ging zurück in Dumbledores Büro.

\enquote{Harry, du solltest noch etwas wissen}, fing Dumbledore an. \enquote{Du wirst in nächster Zeit noch öfters solche Ohnmachtsanfälle haben. Immer, wenn das passiert, ziehst du von Voldemorts Seelenteil etwas ab und schwächst es somit bis es endgültig stirbt.} Harry nickte verstehend und sah Dumbledore an, der mit den Fingerspitzen gegeneinander gelehnt in seinem Stuhl saß. \enquote{Aber \gst immer, wenn du nach deinem Anfall wieder aufwachst, musst du dich beherrschen. Du darfst dich nicht aufregen, oder andere starke Gefühle haben. Sonst könntest du unbewusst, während die Magie noch auf der Suche nach einem Platz in dir ist, Zauber bewirken, die du nicht kontrollieren kannst.}

\enquote{Ich sage meinen Kollegen Bescheid}, wandte Professor Elber ein.

Harry nickte und Dumbledore gab ihnen zu verstehen, sie dürfen gehen.

\gedanke{Dann bin ich ja genauso mächtig wie Voldemort vor seinem Sturz war. Oder im zweiten Fall noch mächtiger?}, dachte Harry nach.

\enquote{Du nimmst ihn mit Albus, habe ich recht?}, fragte Professor Elber Dumbledore.

\enquote{Es ist notwendig, ja}, antwortete er.

\enquote{Dann lass ihn den Horkrux spüren. Er hat die Fähigkeit zu spüren, ob ein Gegenstand ein Horkrux ist. Durch Voldemorts Seelenteil.}

Harry, Ron und Hermine schauten wie vom Donner gerührt zu Professor Elber und Dumbledore. Dieser nickte, schaute Harry kurz an und danach stumm auf die Tür. Harry verstand.

Auf dem Weg nach unten fragte er Professor Elber ganz leise: \enquote{Sie haben damals, als Hermine und ich Sie im Schloss gefunden hatten und kurz bevor Sie zusammen gebrochen sind, also bevor Sie für längere Zeit auf der Krankenstation verbracht haben, Ihren eigenen Horkrux zerstört und Ihre Seele repariert.} Elber nickte nur. \enquote{Sind Sie deshalb so alt geworden?}

Er schüttelte den Kopf. \enquote{Das habe ich erst später gemacht. Heute weiß ich, dass das eine Dummheit war. Ich wollte mich so zu sagen absichern.}

Sie verließen zu viert das Büro und machten sich auf den Weg zum Gemeinschaftsraum. Harry dreht sich auf dem Gang noch einmal um und sah, wie Professor Elber vor dem Wasserspeier stand und eine Hand von seiner Schulter nahm. Er hatte den Eindruck, er nicke. Dann drehte er sich um und ging in die entgegengesetzte Richtung. Er drehte sich dabei nicht noch einmal um.

Helena schwebte um die Ecke und Harry hörte nur etwas, was er als \accentuate{Gramps} deutete. Dann ein \enquote{Shhh!} Dann waren die beiden zu weit entfernt. Vermutlich hatte sein Gehirn nur diese Worte gebildet und in Wahrheit bedeuteten sie etwas ganz anderes.

\trenn

Harry dachte nach. \gedanke{Alle Zauber und Flüche sind dort versammelt. \gst In diesem Buch sind sämtliche Sprüche aufgelistet. \gst Grünes Index-Buch \gst Die Magie ist mein Verbündeter.} Er blieb mitten auf den Weg stehen und hing seinen Gedanken hinterher.

\enquote{So nachdenklich?}, hörte er hinter sich.

Er musste schmunzeln, denn Professor Sinistra kam den Gang entlang.

\enquote{Ja Professor.} Dann fiel ihm etwas ein. \enquote{Kann ich Sie etwas fragen?}

\enquote{Wenn’s nichts Unanständiges ist.}

\enquote{Ich habe ein Bild eines Mondes. Kann man die Phase bestimmen? Ich meine, den Tag eines Mondzyklus?}

\enquote{Sicher, wenn die Darstellung des Mondes gut genug ist.}

\enquote{Jetzt gleich?}

\enquote{Kommen Sie in mein Büro.}

Harry folgte seiner Professorin, die gerade dabei war, zu ihrem Büro hoch oben in einem der Türme zu gelangen.

\enquote{Hat Ihnen Professor Dumbledore nicht gesagt, wie Sie schneller zu Ihrem Büro kommen?}, fragte sie Harry, als er merkte, dass sie die Treppen ansteuerte.

Professor Sinistra blieb stehen und wurde leicht rot. \enquote{Woher\abs wollen Sie wissen, ob\abs}

Er lächelte sie an, da er sie ertappt hatte. Sie wusste es, sollte es aber wohl geheim halten. Er ging zu einem der üblichen Plätze und drückte den passenden Stein.

Als sich die Wand teilt und er vor seiner Professorin die kleine Kabine betrat, fragte sie ihn: \enquote{Woher wissen Sie davon, Mister Potter?}

\enquote{Ich war dabei, als es Professor Dumbledore erfahren hatte. Ich war zu einer Ihrer Stunden zu spät dran, als Ihnen Professor Elber Dokumente überreicht und sie\abs Äh, das möchte ich jetzt nicht wiederholen.}

\enquote{Als er heftig mit mir geflirtet hat, meinen Sie?} Harry nickte. \enquote{Dann haben Sie das an Weihnachten gar nicht mitbekommen?}

\enquote{Da bin ich fast hinter Ihnen gestanden. Ich hatte fast den Eindruck, er schläft\abs} Harry brach ab und wurde rot.

Professor Sinistra lachte ihn herzlich an. \enquote{Probleme damit haben, dass er mich Schätzchen nennt, aber mir eiskalt ins Gesicht sagen, ich hätte mit ihm geschlafen}, meinte sie und lachte.

\enquote{So war das nicht gemeint}, wehrte sich Harry.

\enquote{Schon klar}, winkte sie ab. \enquote{Also, was ist jetzt mit dem Mond?}

Harry zog das Pergament aus seiner Tasche und zeigte ihr die Zeichnung.

Sie sah es sich an und dachte nach. Dann zog sie ein Stück Pergament heraus und legte es daneben. \enquote{Setzen Sie sich ruhig neben mich. Dann lernen Sie gleich was. Das nehmen wir nämlich nicht im Unterricht durch. \gst Sehen Sie es als mündliche Note an, falls Sie auf der Kippe stehen sollten und ich Sie danach fragen werde.}

Harry nahm einen Stuhl und setzte sich neben seine Professorin.

Jeden Schritt erklärte sie ihm. Harry fragte ab und an nach, was genau sie machte, verstand aber recht schnell den Gedanken. Aufgrund der Abbildung konnte man um den einundzwanzigsten Tag schätzen. Nach einer kurzen Messung mit dem Lineal und einem Vergleich von Beobachtungskarten und der Zeichnung kam der zweiundzwanzigste heraus.

\enquote{Wofür brauchen Sie das überhaupt?}, fragte sie.

\enquote{Ich bin einem Rätsel auf der Spur.}

\enquote{Lassen Sie mich daran teilhaben?}

\enquote{Mondbibliothek.}

\enquote{Sie jagen dieser Legende nach?}

\enquote{Bisher habe ich keine Hinweise darüber gefunden, dass es eine Legende ist. Ich habe nur spärliche Hinweise gefunden. Zumeist Ideen, wen ich fragen könnte, denn die letzten Hinweise haben mich auf eine Spur gebracht.} Er breitete das Buch aus und zeigte ihr den entsprechenden Abschnitt. \enquote{Das mit den Zaubersprüchen habe ich schon mal gehört. Es gibt ein Buch, in dem alle Zaubersprüche mit Namen und kurzer Erklärung aufgeführt sind. Ich will deshalb mit Professor Elber sprechen. Vielleicht hat er einen Hinweis mehr. Außerdem hat er was über die Magie gesagt. Sie sei mein Verbündeter. Allgegenwärtig ist sie.}

Professor Sinistra dachte eine Weile nach. \enquote{Das mit der Magie hat er uns auch immer wieder gesagt. Und jetzt, wo Sie es erwähnen, dieses grüne Buch, das er da hatte\abs vielleicht weiß er etwas. Zumindest ist es den Versuch wert. Lassen Sie es mich wissen, wenn es etwas Nennenswertes gibt?}

Harry nickte. \enquote{Ich werde vielleicht auf Ihre Hilfe eingehen. Wenn ich den Namen wörtlich nehme, könnte sich diese auf dem Mond befinden.}

\enquote{Aber wie wollen sie dorthin gelangen?}

Harry hob die Schultern. \enquote{Apparieren?}, fragte er leicht ungläubig.

\enquote{Über vierhunterttausend Kilometer? Wohl kaum.}

\enquote{Genau das stört mich daran.}

\trenn

\enquote{Guten Morgen, Luna}, kam es ihm entgegen.

Verschlafen drehte er sich in Richtung der Stimme, öffnete seine Augen und sagte dann: \enquote{Guten Morgen Elisabeth.} Er drehte sich wieder auf die Seite, als ihm plötzlich klar wurde, dass er nicht mehr in seinem Bett lag. Erschrocken setze er sich im Bett auf und schaute sich um.

\enquote{Was ist, Luna?}, fragte ihn Klara, Lunas zweite Zimmergenossin.

Er sah sich um und bemerkte die blauen Bettdecken und Kopfkissen auf den weißen Bettlaken. In den Mädchenschlafsälen der Ravenclaws waren keine Himmelbetten wie in Gryffindor. Er sah Elisabeth wieder in die Augen. Sie hatte lange schwarze, leicht gewellte, schulterlange Haare und grüne Augen. Sie hatte dieselbe Augenfarbe wie Harry. Er stand auf und meinte: \enquote{Danke Elisabeth, aber ich bin Harry.}

\enquote{Du willst uns veralbern}, sagte Klara.

\enquote{Nein Klara}, antwortete Harry.

Jetzt grinste sie. \enquote{Ich glaube dir kein Wort, Luna, Harry kennt meinen Namen nicht.}

\enquote{Doch}, sagte Harry leicht verärgert. \enquote{Luna beschrieb euch mir recht gut}, log er. Er kannte sie, da er sie bereits durch Lunas Augen gesehen hatte und Luna ihm mitteilte, mit wem sie im selben Zimmer schlief. Er stand auf und sah Elisabeth an.

Sie war sehr hübsch. Sie kam auf ihn einen Schritt zu und legte ihre Hände um ihn. \enquote{Also Harry}, sagte sie und kam ihm näher. \enquote{Was machen wir jetzt?}, fragte Elisabeth.

Harry konnte die anderen Mädchen leise kichern hören. Er legte seine Hände auf ihren Oberkörper und hielt sie auf Abstand. \enquote{Lass es, Elisabeth. Ich bin im Körper eines Mädchens und habe daher kein Interesse, eine Beziehung anzufangen.}

Das Gekicher der Mädchen wurde nun lauter.

Elisabeth sah ihn eigenartig an. \enquote{Magst du mich, Harry?}, fragte sie ihn.

\enquote{Natürlich}, gab er zurück. \enquote{Aber ich fühle mich etwas bedrängt}, sagte Harry. \enquote{Du machst mich verlegen.}

Jetzt lächelte ihn Elisabeth an. Ihre Hände immer noch an seiner Seite, seit er sie auf Abstand hielt. Irgendetwas kam ihm komisch vor, als sie so dastanden, doch er wusste nicht was. Seine Hände fingen leichte massierende Bewegungen an. Dann stoppte er abrupt und es dauerte etwas, bis die Nervensignale von seinen Fingern im Gehirn ankamen und dort ausgewertet wurden \gst bis er bemerkte, wo er seine Hände hatte. Sie lagen die ganze Zeit über auf Elisabeths Brüsten. Er versuchte keine Miene zu verziehen, um dieses Gefühl noch etwas länger zu erhalten. Doch er konnte nicht. Er zog seine Hände schnell zurück und krümmte seine Finger ein.

Elisabeth drehte sich zu Klara und den anderen und meinte dann. \enquote{Lasst uns alleine.} Kichernd verließen sie den Raum und schlossen die Tür hinter sich. Elisabeth lief langsam auf Harry zu. Er zog seine Hände weiter zurück, denn er wollte nicht wieder ihre Brüste berühren. Als sie kurz vor ihm stand, dicht an ihm, konnte er seine Hände nicht mehr herunternehmen, da sie sich dicht an ihn gepresst hatte. Und so berührten seine Hände wieder ihre Brüste durch den knappen Stoff.

\enquote{Harry}, flüsterte sie in sein Ohr. \enquote{Wenn du mir einen kleinen Gefallen tust, dann erzähle ich niemandem, wie du meine Brüste berührt hast und es dir gefiel.}

\enquote{Ich habe nicht}, protestiert Harry, \enquote{nicht bewusst}, fügte er hinzu. \enquote{Na ja.}

Elisabeth lachte. Ihre Stimme ließ ihn ein wohliges Gefühl durchleben. Sie entfernte sich etwas von ihm, sodass er seine Hände herunternehmen konnte. Dann presste sie sich wieder an ihn. Er wusste mit seinen Händen nichts weiter anzufangen, also legte er sie ungezwungen um ihre Taille. Sie gab ein leises Schnurren von sich, kurz aber durchdringend. Er wollte sich schon zurückziehen, da sie nicht merken sollte wie sehr es ihn erregte, aber da regte sich nichts. Er war schließlich in Lunas Körper, fühlte nur etwas Angenehmes in seinem Unterkörper und wie sein Schritt leicht feucht wurde. Es war angenehm, als ihre Brüste gegen den Körper drückten, welchen er gerade bewohnte.

Wieder flüsterte sie in sein Ohr. \enquote{Harry, wenn du wieder in deinem Körper bist \gst und ich merke das, falls du wieder den Körper wechseln solltest \gst kommst du zu mir. Ich wollte dich schon lange einmal küssen.}

Er erschrak und versuchte sich aus ihrem Griff zu lösen. Eindringlich sah er in ihre Augen. \enquote{Das ist nicht dein Ernst}, meinte er.

\enquote{Soll ich den anderen erzählen, wie du es genossen hast, meinen Körper zu berühren?}

\enquote{Nein}, stammelte Harry. Und dann, nach einer kleinen Pause meinte er: \enquote{Das ist Erpressung. Also gut.} Er musste ein leichtes Grinsen überspielen, da er den Gedanken, Elisabeth zu küssen, nicht gerade abstoßend fand.

Sie löste sich von ihm und lief zurück zu ihrem Bett. Da sie auch noch ihren Schlafanzug trug, begann sie sich umzuziehen. Harry tat es ihr gleich, nachdem er sich in Lunas Zimmer umgesehen hatte. Immer wieder blickte er zu Elisabeth herüber. Irgendwann erblickte er sie von der Seite. Sie hatte nichts an. Ihre Haut war glatt und vereinzelte Leberflecken zeichneten sich ab. Auf ihrem rechten Oberschenkel konnte er eine klare Form erkennen. Es schien, also ob es wie ein kleines Herz aussah. Er ließ seinen Blick nach oben schweifen und entdeckte ihre Brüste. Sie sahen so aus, wie sie sich anfühlten. Für ihr Alter hatte sie eine beachtliche Oberweite. Als sie zu ihm blickte, warf er seinen Kopf zurück und machte sich daran, sich ebenfalls anzukleiden. Er dachte sich nichts dabei, dass er sich offen umkleidete. Sie war ebenso ein Mädchen wie er.

\gedanke{Wie ich}, dachte er.

\enquote{Gefällt dir mein Körper?}, fragte ihn Elisabeth. Ihm kam wieder Luna in den Sinn, die ihm dieselbe Frage auf der Krankenstation gestellt hatte.

Harry entschloss sich die Wahrheit zu sagen. \enquote{Ja.}

Elisabeth warf ihren Kopf herum und sah Harry erschrocken an. \enquote{Ja?}, antwortete sie leicht unsicher.

\enquote{Ja}, wiederholte Harry und lachte. \enquote{Du hattest wohl ein Nein erwartet.}

Plötzlich wurde ihm wieder schlecht. Er setze sich auf das Bett. Gerade noch rechtzeitig. Er spürte wie ihm schwarz wurde und er auf das Bett zurückfiel.

\enquote{Luna? Alles in Ordnung?}, hörte er, als er wieder seine Augen öffnete. Er war im Gemeinschaftsraum. In seinem Gemeinschaftsraum. Ron stand über ihm und fragte ihn erneut: \enquote{Luna? Alles in Ordnung?}

\enquote{Nein, Harry}, gab er zurück.

\enquote{Oh Mann}, sagte Ron und reichte ihm eine Hand, um ihn zu sich zu ziehen.

Nachdem er aufgestanden war, meinte Harry: \enquote{Luna ist wohl auch zusammengebrochen.}

\enquote{Ja Mann. Was meinst du wie sie mich beim Umziehen angesehen hat? Erst als ich bemerkte, wer sie war\abs}

Harry grinste. \enquote{Sie hat wohl zu viel gesehen}, meinte Harry.

\enquote{Für meinen Geschmack zu viel}, gab Ron zurück.

Er hatte in letzter Zeit öfters mit Luna gewechselt. Einmal sogar während des Unterrichts. Aber zum Glück bekam es keiner mit, da es nur mehrere Minuten gedauert hatte. Harry musste an Elisabeth denken. \gedanke{Und ich merke das, falls du wieder den Körper wechseln solltest}. Das könnte lustig werden.

\trenn

Gemeinsam mit den anderen ging er wieder den Schotterweg hinunter nach Hogsmeade. Sie hatten wieder eine Apparierstunde. Er ließ sich mit Elisabeth etwas zurückfallen und zog sie, nachdem sie in sicherer Distanz zu den anderen waren, in die Büsche. Es war Freitagnachmittag und passenderweise ein Hogsmeade-Wochenende. Sanft drückte er sie an den Baum und kam ihr näher. \enquote{Du weißt, ich löse meine Schulden bald ein. Und wenn, dann mache ich es richtig.} Dann verließ ihn der Mut. Er blieb stehen und sah sie nur an.

Dann passierte etwas, was er nicht erwartet hatte. Sie legte ihre Hände um ihn und drehte sich mit ihm, sodass er mit dem Rücken zum Baum stand. Dann drückte sie ihn gegen den Baum und trat an ihn heran. Er spürte ihre Brüste auf seinem Oberkörper. Automatisch legte er seine Hände um ihre Taille. Sie kam ihm näher, bis sie kurz vor seinem Mund aufhörte. Er konnte ihre Lippen auf seinen spüren, als sie sprach \enquote{Harry, erfülle deine Schulden. Jetzt.}

Also öffnete er seinen Mund und näherte sich ihr. Dann drang er in sie ein. Sobald er ihre Lippen ganz berührt hatte, konnte er sich nicht mehr zurückhalten. Er küsste sie, als gäbe es keinen Morgen mehr. Dann ließ er wieder von ihr ab und sah ihr ins Gesicht.

Sie schaute ihn mit erröteten Wangen sprachlos an. \enquote{Harry}, stammelte sie. \enquote{Was war das?}

\enquote{Ein Kuss, so wie du es wolltest}, sagte er.

\enquote{Nein}, antwortete sie. \enquote{Das war kein Kuss. Das war mehr, war pures Verlangen.} Jetzt übernahm sie die Führung und küsste ihn zurück. Dann löste sie sich von ihm und meinte: \enquote{Schuld mehr als nur eingelöst. Ich glaube, ich schulde dir nun etwas. Lass uns zu den anderen gehen. Du bist spät dran.}

\gedanke{Ein Verlangen liegt jetzt in ihren Augen}, dachte Harry, \gedanke{wie ich es noch nie gesehen habe.}

Sie nahm ihn an der Hand und zog ihn zurück auf den Schotter. Den Rest des Weges liefen sie, ohne ein Wort zu sagen nebeneinander her.

Vor der Stadthalle angekommen, legte sie einen Arm auf seine Schulter und nahm den Türgriff in die Hand. Dann sah sie ihn an und sagte zu ihm: \enquote{Ich habe mir den Fuß gezerrt und du hast mir hier her geholfen.} Dann öffnete sie die Tür und humpelte mit Harry neben sich herein. Er hielt sie in der Zwischenzeit an ihrer Taille und betrat mit ihr den Raum.

\enquote{Entschuldigung für die Verspätung. Sie hat sich den Fuß etwas gezerrt und ich wollte sie nicht den weiten Weg ins Schloss bringen}, sagte er.

\enquote{Harry half mir und stützte mich auf dem Weg hierher}, meinte Elisabeth.

Professor Elber zog seinen Zauberstab und zeigte damit auf Elisabeths Knöchel. Um ihn herum bildeten sich kleine grüne Funken, die ins Innere des Knöchels sickerten. Harry fiel auf, dass Professor Elber seinen Blick zwischen Elisabeths Knöchel und ihrem Gesicht hin-und-her wechselte. Danach schaute er sie an. \enquote{Sie sollten nun wieder normal laufen können}, sagte Professor Elber. So als ob nichts wäre, führte er seinen Unterricht fort.

\enquote{Er weiß es}, flüsterte Harry Elisabeth zu. \enquote{Ich seh' es an seinem Blick.}

\enquote{Was?}, keuchte Elisabeth leise.

\enquote{Dass dein Knöchel nicht verstaucht war. Geh am besten wieder zurück. Und sprich ihn nicht deswegen an.}

Am Nachmittag dann war das große Quidditch-Spiel gegen Slytherin. Heute könnte er eine gute Gelegenheit haben, seinen Sprung zu versuchen. Immer wieder hatte er ihn geübt und Dumbledore wusste Bescheid. Er hatte ihn einmal davon abgehalten zu sehr verletzt zu werden, als ihn Dementoren von seinem Besen holten. Harry hoffte, dass er heute nicht eingreifen müsste. Er hatte mit seiner Mannschaft ein paar Spielzüge auf dem Mini-Quidditch-Brett von Arabella ausprobiert. Die Sicht von außen auf das eigene Spielgeschehen war etwas vollkommen Neues. Aber es war sehr hilfreich. Um seine eigene Mannschaft vorzuwarnen, sagte er ihnen, was er geübt hatte, damit sie sich keine Sorgen machen mussten.

Schließlich standen alle Mannschaftsmitglieder bereit. Auf einen Pfiff von Madame Hooch hin bestiegen sie ihre Besen und hoben ab. Sie umkreisten einmal das Spielfeld, um sich kurz zu akklimatisieren und dann in Formation in der Luft stehenzubleiben. Die Kapitäne der Mannschaft schwebten bis auf Bodennähe und warteten, bis Madame Hooch den Quaffel in die Luft warf.

Der Kapitän der Slytherin-Mannschaft schaffte es, den Ball zu erwischen und warf ihn seiner Mannschaft zu, währenddessen kreisten Harry und Draco über dem Spielfeld. Einige Male standen sie in der Luft dicht beieinander und suchten das Feld ab. Sie nickten sich knapp zu und umrundeten das Feld wieder. Wie schon so oft wunderte sich Harry, dass Slytherin dieses Jahr extrem gut war. Doch er konnte sich nicht auf seinen Gedanken ausruhen. Er erspähte den Schnatz und flog auf ihn zu. Doch viel zu schnell war er auch schon wieder seinem Blick entflohen. Also stieg Harry mit seinem Besen wieder in die Lüfte.

Er versuchte sich von allem anderen unnützen zu befreien und den Schnatz magisch zu erfassen. Doch so einfach es auch in seinen Übungen war; auf dem Spielfeld mit einem kleinen fliegenden Ball, vielen schreienden und tobenden Zuschauern und den Kommandos der Mannschaften, sowie der Durchsage von Lee Jordan; machte es ihm dies nicht gerade leicht, den Schnatz zu fühlen. Seine Konzentration hätte ihm fast zu einer Begegnung mit einem Klatscher verholfen. Er konnte gerade noch ausweichen und musste dabei feststellen, dass Draco anscheinend den Schnatz gesehen hatte. Er hatte nur noch eine Chance. Er musste springen.

Sein Besen stellte sich senkrecht und schob ihn kurz an. Im freien Fall und mit dem Besen hinterher raste er auf den Schnatz zu. Das Publikum hielt den Atem an. Es wurde richtiggehend still. Wenn man einmal von den Kommandos der Mannschaften absah, die nicht wirklich realisierten, was gerade um sie herum passierte. Harry fiel weiter auf den Schnatz zu und wurde von seinem Besen überholt. Da er keine Masse zu beschleunigen hatte, war er schneller unterwegs. Harry fing den Schatz aus der Luft und wurde von seinem Besen, der ihn eine halbe Sekunde vorher überholt hatte, sanft abgefangen. Harry war gerade auf der richtigen Höhe, um mit Draco, der über ihm flog, nicht zusammenzutreffen.

Als Harry den Schnatz in seiner Hand hielt und nach oben in die Luft streckte, löste sich die Anspannung im Publikum und Jubel brach aus. Sie hatten Slytherin mit zehn Punkten Vorsprung geschlagen. Harry atmete noch immer etwas schwer und hörte gar nicht richtig hin, als seine Hauslehrerin mit mahnenden Worten auf ihn zukam.

\enquote{Wollen Sie einen Siegerkuss?}, fragte er sie, bevor ihn seine Mannschaft auf die Schultern hob und davon trug. Das Gesicht, das Professor McGonagall machte, bekam er gar nicht mehr mit. Nur Colins Foto hielt die Szene fest. Dafür würde er sicher noch einen Rüffel kassieren.

\trenn

\enquote{Meinst du, wir schaffen das?}, fragte Harry. \enquote{Ich muss in einer halben Stunde am Bahnhof sein.}

\enquote{Kein Problem, Harry.} Luna zog ihn mit um eine Ecke und in ein leeres Klassenzimmer. Sie verschloss die Tür und forderte endlich sein Versprechen ein. \enquote{Hose runter. Du schuldest mir noch was.}

Eine halbe Stunde später stand die Apparitionsklasse unten am Bahnhof und wartete auf den Zug, der sie nach London in das Ministerium bringen sollte, um ihre Prüfung abzuhalten. Plötzlich kam Professor McGonagall auf den Bahnsteig und ging voller Aufregung auf Professor Elber zu. \enquote{Frederick, es ist etwas passiert. Die Prüfung wird nicht stattfinden.}

\enquote{Was heißt das, nicht stattfinden?}

\enquote{Nun ja, es gibt keinen im Ministerium, der die Prüfung abnimmt. Ich habe es gerade erst erfahren. Das Ministerium ist scheinbar infiltriert worden.}

\enquote{Was?}, schrie Professor Elber. \enquote{Das kann nicht sein. Das Ministerium hat trotz allem die Verpflichtung die Prüfung abzunehmen. Egal wer jetzt die Abteilung leitet. Wir werden trotzdem dorthin fahren. Die Schüler haben ein Recht auf ihre Prüfung und das Ministerium hat die Pflicht diese abzunehmen.}

\enquote{Aber\gst}

\enquote{Kein Aber. Ich werde mich darum kümmern, sobald wir dort sind. Um die Sicherheit kümmere ich mich schon.} Der Zug kam an und Professor Elber zeigte seinen Schülern an, dass sie einsteigen sollen. \enquote{Minerva, steig ein}, sagte Professor Elber zu Professor McGonagall. Leicht irritiert stieg sie ein.

\gedanke{Ihr ist mulmig}, dachte Harry. \gedanke{Aber warum fügt sie sich dann? Sie ist doch sonst nicht so!}

Die Lokomotive setzte sich schwer schnaufend in Bewegung und fuhr Richtung London los. Harry hatte schon so oft diesen Zug benutzt, aber noch nie während des Schuljahres. Es war Freitag-Morgen und gegen Abend würden sie in London angekommen sein, dachte er. Die Fahrt dorthin war recht geruhsam, aber nicht langweilig.

Angekommen in London und durch die Absperrung von Gleis 9 3/4 hindurch, erregte die Gruppe bald eine gewisse Aufmerksamkeit. Professor Elber entging das nicht und er drehte sich plötzlich um und lief rückwärts. \enquote{Also Klasse}, sprach er in einem schottischen Akzent. \enquote{Zusammen bleiben. Ihr wisst, dass der Ausflug nach London ein Privileg ist. Wenn wir mit der U-Bahn nach Westminster fahren, bleibt bitte zusammen.}

\gedanke{Westminster}, dachte Harry. \gedanke{Dort ist das Zaubereiministerium.} Professor Elber drehte sich wieder um und lief nun wieder vorwärts. Man merkte, wie das Aufsehen, welches die Gruppe bis vor kurzem noch erregt hatte, abebbte. Die Muggel hielten sie für eine normale Schulklasse. Sie bewegten sich auf die U-Bahn-Station zu und verschwanden im Untergrund. Professor Elber zog vor den Drehkreuzen eine Karte heraus und ermöglichte durch das Entwerten dieser der Gruppe die Benutzung der U-Bahn.

Nachdem sie an der richtigen Station ausgestiegen waren, begann Professor Elber wieder mit seinem schottischen Akzent. \enquote{So, jetzt geht noch jeder von euch auf die Toilette und dann machen wir weiter.} Keiner widersprach. Professor Elber flüsterte Professor McGonagall noch etwas ins Ohr und führte die Jungs dann auf die Toilette. Keiner der umstehenden Muggel nahm wirklich Notiz von ihnen. Er erklärte auf dem Männerklo den Jungs, dass sie sich die Toilette herunterspülen müssten.

Im Ministerium angekommen war es still. Professor Elber ging voraus und führte die Gruppe durch viele Gänge und Treppen zu ihren Schlafsälen. Danach verriegelte er die große Tür und bereitete sich für die Nacht vor. Es war ein großer Raum, in dem viele Matratzen auf dem Boden lagen. Darauf jeweils ein Kissen und eine Decke. Harry wunderte sich, dass sich sein Professor hier so gut auskannte.




\begin{kommentar}
Katharina und Harry lesen in Hogwarts etwas über die Mondbibliothek. Ein Hinweis ist auch gleichzeitig eine Anspielung auf Monopoly, das Harry mal mit Tamara gespielt hatte. Ein anderer Hinweis auf die Bibliothek stammt von Indiana Jones und der letzte Kreuzzug. Und ein weiterer stammt von Krieg der Sterne.
\end{kommentar}
