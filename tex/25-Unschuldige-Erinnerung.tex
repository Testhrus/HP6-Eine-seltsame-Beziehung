\chapter{Unschuldige Erinnerungen}


\enquote{Fassen Sie sich nun an den Händen. Minerva, lässt du dich führen? Wir wollen an das Ende des Dorfes. Es geht um die Unterschiede zwischen den beiden Techniken. Lasse dich leiten und ziehe einfach mit. Das macht es mir leichter.}

Minerva nickte mechanisch und die Reise begann. \enquote{Lasst gleich nach dem Ankommen los, verstanden? \gst Drei \gst Zwei \gst Eins\abs}

\geraeusch{Wupp.} Die Gruppe war verschwunden und tauchte nach Bruchteilen einer Sekunde wieder auf. Sofort lösten sich die Hände der Schüler und etwa ein Drittel begann sofort auf die Knie zu sinken und in die vor ihnen stehenden Eimer zu brechen. Nachdem Tücher herumgereicht wurden, damit sich die Schüler ihre Münder putzen konnten, fassten sie sich auf ein Zeichen wieder an den Händen und die Reise zurück begann. In der Halle angekommen mussten nur noch zwei brechen.

\enquote{Sehr gut. Damit dürften die Brechanfälle für Sie vorbei sein. Mehr als zweimal musste bisher keiner brechen. \gst Sie wurden jetzt also zweimal nach der Drei-Punkt-Regel appariert. Sie wissen also, wie man sich fühlt, wenn man von jemand mitgenommen wird, der diese Technik beherrscht, oder selber appariert und diese Technik anwendet. Wir werden jetzt nochmal hin und wieder zurück apparieren. Allerdings nach der Fünf-Punkt-Regel. \gst Bereit?}

Die Schüler und Minerva nickten und fassten sich wieder an den Händen.

\enquote{Drei \gst Zwei \gst Eins.}

\geraeusch{Wupp.} Wieder verschwand die Gruppe aus der Halle und tauchte innerhalb von Sekundenbruchteilen am Ende von Hogsmeade auf. Dann ging es wieder zurück.

\enquote{Und? Haben Sie einen Unterschied festgestellt?} Die Klasse nickte. \enquote{Was war angenehmer?}

\enquote{Die dritte und vierte Reise}, sagte Marcel, ein Ravenclaw.

\enquote{Hatten Sie das Gefühl, dabei brechen zu müssen?}

\enquote{Nein.}

\enquote{Warum nicht?}, fragte er provozierend nach.

\enquote{Ich hatte nicht das Gefühl, durch einen Schlauch gepresst zu werden. Es war vielmehr ein Verschwinden und Auftauchen an einem anderen Ort. Wie ein normaler Spaziergang.}

\enquote{Sehr schön, das meinte ich. Fünf Punkte für Ravenclaw. \gst Kommen wir nun zur Drei-Punkt-Regel. Wir werden heute schon damit anfangen, uns darauf zu konzentrieren und die ersten Versuche zu unternehmen. Allerdings einzeln und nur von einem Kreis in den anderen.}

Auf dem Boden erschienen zwei Kreise, die nur zwei Meter voneinander entfernt waren.

\enquote{Fangen wir an mit dem ersten der drei Punkte.}

Elfen erschienen und brachten Kissen mit, die sie im Raum verteilten. Professor Elber setzte sich auf eines der Kissen und machte weiter.

%Ziel. Wille. Bedacht.			%Direktheit und Bequemlichkeit
\enquote{Fangen wir mit dem Ziel an. Sie müssen sich das Ziel genau vorstellen. Je genauer sie das Ziel vor Ihrem inneren Auge haben und sich vorstellen können, desto besser ist es. Prägen Sie sich also die Halle einmal ein. Versuchen Sie sich vorzustellen, wie es aussieht, wenn Sie in diesem Kreis stehen. Je genauer Sie sich den Zielort vorstellen können, desto besser ist es. Wenn Sie zum Beispiel geografische Daten haben, also Länge und Breite auf dem Globus, dann können Sie sich auch dort hin apparieren. Egal, wie Sie ihr Ziel benennen. Je genauer Sie den Ort benennen können in Ihren Gedanken, desto besser ist es. Denken Sie während der nächsten fünf Minuten darüber nach.}

Dann herrschte Stille. Minerva nahm ihr Kissen und setzte sich neben Professor Elber. Sie tuschelten miteinander so leise, dass sie die anderen nicht störten, die konzentriert da saßen.

\enquote{Als Nächstes kommt der Wille. Sie müssen dorthin wollen. Sie müssen in sich einen Drang auf das Ziel aufbauen. Es darf nichts Wichtigeres geben, als dort hin zu wollen. Sie müssen sich vorstellen, dass Sie von hier nach da gelangen, ohne sich zu bewegen. Und der dritte Punkt. Die Bedacht. Seien Sie Präzise in Ihren Ausführungen. Arbeiten Sie Sorgfältig, denn sonst zersplintern Sie.}

Er griff in die Luft und es kam ein Bündel an Papier zum Vorschein. \enquote{Bitte sehen Sie sich diese Fotos an und reichen Sie sie herum. Bitte einzeln.} Er reichte die Fotos einem Schüler in seiner Nähe. \enquote{Wenn Sie nämlich nicht sorgfältig sind, dann werden Sie so aussehen, wie die Personen auf den Bildern. Deswegen ist es auch erforderlich, dass Sie eine Lizenz erwerben. Illegales Apparieren ist strafbar. Sie können sich denken warum, wenn Sie diese Fotos sehen.}

Dann stand er auf und belegte die Halle mit einem Zauber. \enquote{Bitte versuchen Sie es alle einmal. Stellen Sie sich Ihr Ziel vor, haben Sie den Willen, dorthin zu gelangen, und seinen Sie präzise mit Ihren Körperteilen und Ihrer Kleidung. Es kann nämlich passieren, dass Sie diese verlieren. Das ist schon mal passiert, dass jemand nackt auftauchte. Die Halle wurde mit einem Anti-Apparitionsschild belegt. Sie werden ein Kribbeln spüren, wenn Sie zu apparieren versuchen. Wenn Sie sämtliche Körperteile spüren und alles an Ihnen kribbelt, dann würden Sie komplett an ihrem Ziel erscheinen. Leider können Sie das nicht für Ihre Kleidung und Ihre Körperbehaarung feststellen. Ich habe schon von Personen gehört, die nach einer Apparition eine Glatze hatten, oder deren Schambehaarung zurückblieb.}

Einzelne Schüler schmunzelten. Dann begannen sie sich zu konzentrieren und zuckten immer wieder zusammen und rieben sich verschiedene Körperteile.

Während dessen tuschelte Elber mit McGonagall. Ab und an gestikulierte er.

\fluestern{Du musst beim Fünf-Punkte-Apparieren folgendes beachten. Du willst direkt zum Ziel und nicht über einen Umweg, oder einen Pfad, der dir vorgegeben wird. Denke dir eine Linie direkt zum Ziel. Das ist wichtig. Sonst wirst du so durch den Raum geleitet, wie es der Magie passt. Das kann unter Umständen der vielfache Weg sein. Da die Magie ja den Raum belegt, in dem wir uns befinden, wird sie versuchen, uns nur eine Passage so klein wie Möglich zur Verfügung zu stellen. Das geht dann natürlich an die Substanz. Außerdem musst du auf die Bequemlichkeit acht geben. Stell dir die Reise so bequem wie nur möglich vor. Du willst schließlich nicht gequetscht werden. Dann wird dir auch eine größere Passage zugeteilt.}

McGonagall sah ihn abwechselnd mit Staunen dann wieder mit Unverständnis, aber auch mit verstehendem Blick an.

Nachdem die erste Stunde bereits vorbei war, stand Professor Elber wieder auf und schwang erneut seinen Zauberstab.

\enquote{So, jetzt wird es ernst. Jeder hat genau einen Versuch, zu apparieren. Die Halle wurde präpariert, damit ein heraus oder herein apparieren nicht möglich ist. Sie können nur innerhalb der Halle apparieren. Professor McGonagall und ich stehen sofort bereit, falls ein Unfall passieren sollte. Wollen wir hoffen, dass so etwas nicht passiert. Es ist egal, wer anfängt. Also, die erste freiwillige Person vor.}

Lange passierte nichts, bis eine mutige Hufflepuff vortrat.

\enquote{Ich sollte jetzt von Gryffindor fünf Punkte abziehen. \gst Das werde ich jetzt auch machen.}

\enquote{Frederick, wie kannst nur\abs? Wieso?}

\enquote{Bist du mutig und so\abs dann ist Gryffindor eine gute Heimat. So heißt es doch. \gst Fünf Punkte von Gryffindor wegen Feigheit.}

\enquote{Hrmpf}, war das Einzige, was McGonagall noch an Tönen hervorbrachte.

\enquote{Also Miss Mariola. Hinein in den Kreis.}

Die Schülerin kam und nahm im Kreis Platz.

\enquote{Ziel. Wille. Bedacht}, sagte Elber.

Mariola nickte und konzentrierte sich. Kurz darauf stand sie im Zielkreis. Nur ihre Unterhose war zurückgeblieben und fiel zu Boden.

\enquote{Violette Unterwäsche. Nett}, scherzte er.

Erschrocken drehte sich Mariola um und folgte dem Blick ihres Lehrers. Peinlich berührt nahm sie ihre Unterhose auf und schob sie in ihre Tasche. Dann traute sich auch der Rest der Klasse. Nicht bei jedem lief es so gut. Einer apparierte sogar ohne Kleidung. Da die anderen aber alle dahinter standen, war das nur bedingt peinlich, da sofort ein Bademantel gereicht wurde. Einem Schüler fehlte nach dem Apparitionsvorgang sogar ein Finger, der aber sofort wieder von beiden Professoren angezaubert wurde. Sie umhüllten den Arm der Schülerin mit einem rosa Nebel und fügten den Finger wieder an. Durch den Schock, spürte sie keine Schmerzen und bevor die Nervensignale ihr Gehirn erreichten, war der Finger schon wieder dran. So spürte sie nur ein kleines Kribbeln. Nach etwa der Hälfte war Harry dran. Er apparierte erfolgreich im Ganzen und ohne Verluste. Leider kam er außerhalb des Zielkreises an. Er war zu weit appariert, was ihn auf die Entfernung bei einer Prüfung durchfallen ließ.

Insgesamt war es ein durchwachsenes Ergebnis, einige verlorene Finger, ein paar verlorene Kleidungsstücke und einen Fingernagel, sowie Augenbrauen.

\enquote{Für den ersten Versuch war das schon ganz in Ordnung. Entspannen Sie sich jetzt und beruhigen Sie sich wieder. Setzen Sie sich noch ein paar Minuten auf Ihre Kissen, damit Sie sich beim Verlassen der Halle nicht aus Versehen ver-apparieren}, scherzte Professor Elber. \enquote{Die Stunde ist beendet. Bis nächste Woche. Dann werden wir die Übung wiederholen. Die erste viertel Stunde werden Sie dann wieder auf den Kissen Ihre geistigen Fähigkeiten stärken und dann wird wieder appariert. Erst wenn Sie quer durch die Halle die Apparitionen schaffen, werden wir auf die Fünf-Punkte gehen.}

\trenn

Harry ging in Richtung der Kerker. Unten angekommen lief er durch sein Tränkeklassenzimmer und klopfte wie üblich an die Tür zu Snapes Büro. Die Tür wurde geöffnet und Snape saß wie üblich hinter seinem Schreibtisch. Wie er es in der Zwischenzeit schon gewohnt war, setzte er sich in den Sessel und konzentrierte sich. Er wusste, dass Snape bald so weit war; er hatte nur noch wenige Minuten.

\enquote{Sehr gute Hausarbeit, Potter, fünfzig Punkte dafür.}

Harry zeigte keine Reaktion. Und dann, ohne Vorwarnung, kam sein Angriff.

\enquote{Legilimens.}

Harry konzentrierte sich auf einen Gang, den er entlang lief. An dessen Ende stand eine Tür. Die Tür öffnete sich und dahinter war ein runder Raum mit vielen Türen. Er ging auf eine zu und kam wieder in einen Gang. Dieser war genauso lang wie der Erste. An dessen Ende war wieder eine Tür und wieder ein runder Raum. Dieses Mal nahm er eine andere Tür und trat wieder in einen Gang.

Dann begann sein Widerstand langsam zu bröckeln und wurde schwächer. Harry dachte an Hogwarts und dessen Gänge. Er verabschiedete sich von den schwarzen eintönigen Ministeriumsgängen und die Umgebung verwandelte sich. Dann brach sein Widerstand komplett.

Snape brach ab und gab ihm einen Trank. Harry trank ihn wie üblich und Snape setzte sich ihm gegenüber in einen weiteren Sessel.

\enquote{Nicht schlecht, Potter. Über eine Stunde. Sie bessern sich merklich. Sie sind dran.}

Dann erinnerte sich Harry an Worte, die ihm einmal gesagt wurden. \enquote{\accentuate{Wir haben viel, was Magie betrifft, vergessen. Zaubern ohne Stab war früher selbstverständlich.}}

Harry musste sich ein Grinsen verkneifen. Er konzentrierte sich kurz und spürte die Magie. Er hob seine Hand und zeigte mit seinen mittleren drei Fingern seiner rechten Hand auf Snape und murmelte leise: \enquote{Legilimens.}

Da Snape nicht darauf vorbereitet war, konnte er einige Sekunden lang in Snapes Vergangenheit eindringen.

\begin{traum}
Er sah ihn, wie er mit seiner Mutter gemeinsam in Hogwarts spazieren ging, wie sein Vater ihn niedermachte, aber auch, wie er ihm das Leben rettete, als ihn Sirius in Remus' Werwolffänge locken wollte, während er sich bei Vollmond im Gang wieder verwandelt hatte \gst im Gang unter der peitschenden Weide.
\end{traum}

Dann konnte Snape seine Gedanken verschließen und entließ Harry \gst das hieß, Harry stand auf, bedankte sich und ging freiwillig, bevor er hinausgeworfen wurde.

Er ging schlafen, doch konnte es nicht. Dann versuchte er sich auf seine Okklumentik zu konzentrieren, doch er erreichte genau das Gegenteil. Die Bilder, die er sah, kamen wieder hoch, so als seien sie die einzigen Erinnerungen, die er hatte. Langsam sickerten diese wenigen Momente, in denen er seine Mutter lachen sah, in seine Gedanken und seine Erinnerungen ein und er verstand langsam, verstand, dass seine Mutter damals mit Snape ausging; doch das war nur ein Gedanke, ein Gefühl. Sein Verstehen lag erst am Anfang.

Er wachte mitten in der Nacht auf. Ihn beunruhigte etwas. Er nahm die Karte des Rumtreibers heraus und suchte \accentuate{Severus Snape} auf ihr. Er fand ihn in der Nähe des Gryffindorturmes. Er faltete die Karte zusammen, zog seinen Bademantel und seine Schuhe an und ging hinunter, durch den Gemeinschaftsraum und verließ ihn durch das Porträt.

\enquote{Sperrstunde Potter, sie sollten schlafen}, hörte er Snape sagen.

Harry drehte sich herum und antwortete: \enquote{Ich weiß. Ich habe nach ihnen gesucht. Ich\abs} Er überlegte, wie er es am besten sagen konnte. \enquote{Das, was ich gesehen habe\abs Ich habe darüber nachgedacht\abs seit ich heute schlafen gegangen bin und meine Gedanken verschließen wollte\abs Ich konnte nicht\abs Meine Gedanken drehten sich nur um das, was ich gesehen hatte.} Er stockte kurz. \enquote{Sir, ich beginne so langsam zu verstehen. Ich habe aber eine Frage, die mich beschäftigt.} Dann sah er Snape direkt in die Augen. Emotionslos sah er ihn an. \enquote{Wie haben sie und meine Mum\abs}

Snape hob seine Hand. \enquote{Kommen Sie an ein Denkarium ran?}, fragte er ihn.

Harry war sich nicht sicher, nickte aber.

Snape griff in seinen Umhang und zog eine kleine Phiole mit einer klaren, leicht bläulichen Flüssigkeit heraus. \enquote{Hier. Das ist eine Kopie. Vielleicht wird ihnen das dann klar.} Harry nahm die Phiole vorsichtig entgegen. Er hielt sie wie einen Schatz. Dann bedankte er sich bei Snape und drehte sich um, um zurückzugehen. \enquote{Ach, bevor ich es vergesse. \gst Fünf Punkte Abzug wegen Aufenthalt außerhalb der erlaubten Zeiten}, sagte Snape. Dann rauschte er davon.

Harry grinste und betrat den Gemeinschaftsraum. Er hatte einen Weg gefunden, ohne die schlafende Dame aufzuwecken. Er berührte ihre Stirn und übermittelte ihr das Passwort mit Legilimentik. Das hatte den Effekt, dass das Porträt aufschwang, da es das Passwort erkannte und akzeptierte. Er musste die fette Dame also nicht wecken.

Er ging nach oben, zog sich den Bademantel aus, fischte die Phiole heraus und sicherte sie magisch. Dann legte er sie sicher in seinen Koffer, legte sich ins Bett und schlief beruhigt ein.

\trenn

Professor Elber traf auf Luna und Harry, die am See spazieren gingen, um sich über ihre körperlichen Bedürfnisse und sonstige Notwendigkeiten zu unterhalten. Luna musste sich jetzt immerhin rasieren und hatte absolut keine Ahnung, wie man das machte. Harry zauberte einen Spiegel hervor und zeigte Luna einen Rasurzauber. Dann fuhr er mit dem Zauberstab an seinen Barthaaren entlang. Die eine Hälfte machte er, die andere überließ er Luna, die sich bemühte es ordentlich hinzubekommen. Es sah lustig aus, wie Lunas Körper dem von Harry erklärt, wie man sich zu rasieren habe.

\enquote{Gute Neuigkeiten}, sprach Professor Elber die beiden an. \enquote{Es ist nicht notwendig, etwas mit Ihren Seelen zu machen. Ich habe noch etwas anderes gefunden. Es zeigt an, wohin die Sinneseindrücke übermittelt werden. Ein einfacher Zauber. Ungefährlich.} Harry und Luna sahen ihren Lehrer freudig an. \enquote{Bereit?}, fragte er.

\enquote{Ja}, antworteten beide.

Ihr Lehrer fuhr beiden mit dem Daumen über ihre Stirn, zog seinen Zauberstab und murmelte einen Spruch, den die beiden nicht verstanden. Dann kamen aus ihrer Stirn feine Linien, die zum jeweils anderen Körper gingen. Erleichtert atmete Professor Elber aus.

\enquote{Das war Hebräisch, richtig?}, fragte Luna.

\enquote{Ja}, antwortete Professor Elber. \enquote{Es ist jetzt leichter. Entweder das geht schnell vorbei, oder wir müssen etwas nachhelfen. Da muss ich noch weiterlesen. Ich werde Severus Bescheid sagen, dass er seinen Trank für etwas anderes verwenden kann. Das müsste noch gehen.} Dann ging er wieder.

\enquote{Wie waren eigentlich die Stunden bei meinem Vater, Harry?}, fragte ihn Luna, denn beide hatten in der Zwischenzeit kurzzeitig ihre Körper getauscht. So hatte Harry einen halben Tag bei Xenophilius Lovegood verbracht.

\enquote{Interessant}, antwortete Harry. \enquote{Ich war plötzlich in deinem Bett und sah an die Decke. Ich musste mich erst einmal orientieren. Dann habe ich mich im Zimmer umgesehen. Das Wandgemälde fand ich besonders beeindruckend. Hast du es selbst gemalt?}

\enquote{Ja.}

\enquote{Du hast mich, Ron, Hermine, Ginny und Neville an die Wand gemalt.} Luna nickte. \enquote{Und das Band. Ich hielt es erst für ein Band. Bis ich näher kam und eine feine Schrift entdeckte. \accentuate{Freunde}, stand da. Eng nebeneinander und in drei Reihen untereinander. Goldene Schrift. Hast du das auch gemacht?}

\enquote{Ja. Alles mit einem feinen Pinsel. Ich hatte Zeit und mir war danach. Es war letztes Jahr während der Ferien. Dad war ein paar Tage beschäftigt. Also habe ich mich daran gesetzt und gemalt. Er findet es schön.}

\enquote{Das hat er mir gesagt, als er in deinem Zimmer auftauchte und mich sah, wie ich das Bild betrachtete.}

\enquote{Was hast du sonst noch so erlebt?}

\enquote{Ich habe für ihn gekocht.}

\enquote{Hat er keinen Verdacht geschöpft?}

\enquote{Nein. Ich habe das Rezeptbuch hergenommen und nach einem gesucht, dessen Zutaten ich kannte. Zu Hause habe ich auch immer mal wieder gekocht. Daher hatte ich keine großen Probleme. Dein Dad hat nur einmal komisch geschaut, und gemeint: \enquote{Seit wann schmeckt dir denn das? Das hast du doch sonst immer ungern gegessen.} Ich habe ihm erzählt, dass ich mich heute anders fühle und meine Geschmacksknospen wohl etwas anderes wollen. Das hat er mir abgenommen. \gst Du hast übrigens ein sehr schönes Haus. Nur für immer würde ich mit deinem Vater dort nicht leben wollen.}

\begin{abAchtzehn}

\enquote{Ich würde jetzt gerne mit dir schlafen, Harry}, sagte sie plötzlich.

\enquote{Du weißt aber schon, dass wir nicht mehr zusammen sind, oder?}

\enquote{Ja. Stört dich das? Du bist doch noch solo, oder?}

\enquote{Ja schon, aber\abs} Harry wusste nicht mehr weiter.

\enquote{War das jetzt die Antwort auf meine erste Frage, oder auf meine zweite Frage?}

\enquote{Wie bitte?}

\enquote{Ich wollte wissen, ob das die\abs}

\enquote{Schon gut, ich habe dich verstanden. \gst Das war die Antwort auf deine zweite Frage.}

\enquote{Ah. Also stört dich die Tatsache, dass wir nicht mehr zusammen sind. Deshalb möchtest du nicht mehr mit mir schlafen.}

Harry wusste nicht genau, was er sagen sollte.

Nach einer Weile fragte Luna nach. \enquote{Was ist jetzt? Ausziehen und ins Wasser? Wir können es ja als Schwimmen tarnen.} Kaum hatte Luna ihren Satz beendet, entledigte sie sich schon ihrer Kleidung und stieg nackt ins Wasser. \enquote{Komm schon, Harry. Das Wasser ist herrlich.}

Harry wusste nicht genau, wie er reagieren sollte, folgte ihr aber brav ins kühle Nass, nachdem er sich ebenfalls ausgezogen hatte, die Kleidung zusammengelegt und gegen Diebstahl gesichert hatte.

Sie schwammen am Ufer entlang und zwischen das Schilf, wo Luna ihre Arme um Harry legte und ihn zu küssen anfing. \enquote{Weißt du, auch wenn wir nicht mehr ein Paar sind, würde ich gerne meine Bedürfnisse stillen. Mit dir.}

Harry konnte es nicht leugnen und genoss die Zärtlichkeiten mit Luna. Er legte nun ebenfalls seine Arme um sie.

\enquote{Ich muss an Ginny denken}, sagte er.

\enquote{Dann tu es, wenn du mich küsst. Ich habe nichts dagegen.} Und als sich nach kurzer Zeit etwas bei Harry regte und sie seine Erregung zwischen ihren Beinen spürte, fügte sich hinzu: \enquote{Du scheinst Ginny sehr gerne zu haben. Das kann ich bis hierher spüren.}

\enquote{Ha. Ha}, sagte er. \enquote{Du bist aber auch schon ganz schön feucht zwischen den Beinen.}

Dies veranlasste Luna herzlich zu lachen. \enquote{Schaffst du auch den nächsten Schritt, wenn du an Ginny denkst?}, fragte sie vorsichtig nach.

\enquote{Tut mir leid, Luna. Das kann ich nicht.}

\enquote{Gibst du mir dann auf andere Art und Weise, was ich brauche?}

Harry nickte und seine Finger fuhren an ihrem Körper entlang Richtung Po. Eine Hand massierte ihren Hintern, während die andere ihr Schneckchen aufsuchte und sie zwischen den Beinen stimulierte. Luna legte ihre Beine um Harrys Hüften, was es ihm erschwerte, seine Hände zwischen den Körpern zu halten. Also zog er sie dazwischen heraus, nahm Luna noch ein Stückchen höher und kam von hinten. Immer wieder wechselte sein Finger zwischen ihren Schamlippen und ihrer Klitoris hin und her. Lunas Erregung wurde immer heftiger. Immer wenn sie meinte, dass sie einen Aufschrei nicht länger unterdrücken konnte, versuchte sie ihm einen Zungenkuss aufzudrücken, um ihre Schreie zu ersticken. Sie wollte nicht zu laut sein.

Als sie ihren dritten Orgasmus innerhalb einer viertel Stunde hinter sich hatte, brach sie den Kuss und meinte: \enquote{Es ist genug, Harry. Das war der Wahnsinn. Das reicht für mindestens\abs Auf jeden Fall lange.} Sie nahm ihre Hände von seiner Hüfte und legte sie um seinen Kopf. Dann küsste sie ihn erneut. \enquote{Darf ich dir nun einen Gefallen tun?}, fragte sie.

\enquote{Später. Vor allem nicht im Wasser.}

\enquote{Oh. Ich kenne aber einen guten Zauber, um ein paar Minuten die Luft anzuhalten und wenn dich dein Sperma im Wasser stört, dann schlucke ich einfach.}

\enquote{Das ist ein verlockendes Angebot, Luna. Aber, heute nicht.}

\enquote{Ok}, sagte sie und lies von ihm ab.

Das war das Schöne an Luna. Sie schien nie richtig beleidigt zu sein. Sie fuhr ein paar mal mit hartem Griff an seiner Männlichkeit auf und ab und schwamm dann zurück. Als sie zurückkamen lagen bereits Handtücher zum Abtrocknen bereit.

Nachdem sie angezogen waren, erschien Kreacher und nahm die Handtücher wieder mit.

\enquote{Richtig fürsorglich, dein Elf}, bemerkte Luna.

\enquote{Ja, das ist er}, antwortete Harry.

\enquote{Und? Wann darf ich dir nun einen blasen?}, fragte Luna unverfroren.

\enquote{Erwartest du von mir ein Datum?}

\enquote{Wäre nett}, sagte sie, während beide wieder zurück zum Schloss gingen.

\end{abAchtzehn}

\begin{safedivide}
\fskdivider
\end{safedivide}

Harry dachte nach. Er wusste noch immer nicht, wie genau er in Voldemorts Geist eindringen sollte. Dann hatte er einen Einfall. Es klang so logisch. Er konnte deshalb keine Verbindung herstellen, da er mit sich selbst keine Verbindung herstellen konnte. Er holte sein Glasröhrchen aus seiner Tasche, das er immer mit sich trug, und entkorkte es wieder.

Luna blieb stehen und sah ihn interessiert an.

Harry lächelte sie an und dachte nach. Er versuchte sich zu erinnern. Er schloss seine Augen und holte seinen Stab hervor. Als er die Szene klar und deutlich vor sich sah, kopierte er sie, indem er seinen Stab an seine Schläfe hielt und den Gedankenfaden herauszog. Dann legte er ihn vorsichtig in das Glasröhrchen und verkorkte es. Jetzt musste er es nur noch jemanden zeigen.

\enquote{Luna? Morgen}, sagte er. Dann machte er sich weiter auf den Weg zum Schloss. \enquote{Ich habe noch was vor. Ich muss zu Dumbledore.} Dann verschwand er schnellen Schrittes. Er holte seine Erinnerung von Snape aus seinem Koffer und machte sich damit zu Dumbledores Büro. Er hoffte, dass er Zugang bekommen würde, hoffte, dass Dumbledore ihm die Gelegenheit geben würde, die Erinnerung von Snape zu sehen und durch die andere Hagrid zu entlasten.

Vor Dumbledores Büro angekommen, stand bereits Professor Sprout vor dem Eingang und versuchte ihn zu öffnen.

\enquote{Professor}, begrüßte er sie.

\enquote{Hallo, Mister Potter.}

\enquote{Versuchen Sie zu Dumbledore zu gelangen?}

Professor Sprout nickte. \enquote{Ich schaffe es nicht zu ihm zu gelangen.}

Harry stutzte und schaute zum Wasserspeier. \enquote{Hallo}, begrüßte er ihn. \enquote{Warum haben wir keinen Zugang?}

\enquote{Es gibt kein Passwort, um die Tür zu öffnen}, antwortete dieser.

\enquote{Wie? Kein Passwort}, fragte Professor Sprout. \enquote{Es gibt immer ein Passwort.}

\enquote{Es sollte geändert werden. Aber es kam kein neues nach. Also kann ich euch nicht hereinlassen.}

\enquote{Kannst du feststellen, ob es ihm gut geht?}, fragte Harry nach.

\enquote{Ja, das kann ich}, antwortete der Wasserspeier.

\enquote{Und, wie geht es ihm?}, fragte Professor Sprout nach.

\enquote{Das kann ich euch nicht beantworten}, sagte der Wasserspeier.

Mittlerweile kam auch Professor McGonagall an. Sie wollte wohl ebenfalls zum Schulleiter.

Harry hielt sich zurück, als Professor Sprout und Professor McGonagall sich über das Problem unterhielten. Seine Gedanken rasten. Kurz, nachdem Professor McGonagall einen Patronus losgeschickt hatte, kam Madame Pomfrey heran und wollte wissen, was los sei. Sie konnte nur feststellen, dass der Direktor bewusstlos am Boden seines Büros lag. Harry bekam das mit. Er erinnerte sich an das eine mal, an dem er Madame Pomfrey mitnahm, um Parvati und Lavender zu helfen. Er lächelte.

\gedanke{Na klar, die Aufzüge}, dachte er und ging auf einen Bogen zu, drückte den passenden Stein und trat ein. Er drückte den passenden Knopf und legte seine Hand auf die Schaltfläche.

Nach einer kurzen Fahrt stieg er im obersten Stockwerk des Turmes aus, in dem das Schulleiterbüro war. Er trat auf die oberste Ebene, auf der das Fernrohr stand. Er ging die Stufen schnellen Schrittes hinunter. Als er unten ankam, sagte er nur schnell \enquote{Hallo Fawkes} und sah kurz darauf Dumbledore am Boden liegen. Er fühlte seinen Puls und seinen Herzschlag. Beides war schwach, aber vorhanden. Er öffnete die hölzerne Tür und rief nach unten: \enquote{Öffnen.} Nachdem er das Geräusch der sich bewegenden Steine vernommen hatte, lief er zurück zu Dumbledore und untersuchte ihn mit seinem Zauberstab, damit er Madame Pomfrey eine kurze Analyse geben konnte. Einerseits brachte dies wieder eine mündliche Note, anderseits war es eine prima Übung. Aber am wichtigsten war die ersparte Zeit.
%Blutdruck Normalwerte
%Einteilung der Blutdruck-Werte laut WHO (Weltgesundheitsorganisation):
%						systolisch (mmHg)		diastolisch (mmHg)
%optimal				  < 120						 < 80
%normal					  < 130						 < 90
%hochnormal				130-139						85-89
%Hypertonie Grad 1		140-159						90-99
%Hypertonie Grad 2		160-179						100-109
%Hypertonie Grad 3		>= 180						>= 110
%Hypertonie = Bluthochdruck

Als er Madame Pomfrey die Stufen herauf steigen hörte, rief er ihr schon entgegen: \enquote{Puls ist schwach, aber vorhanden Herzschlag bei 75 und Blutdruck 55 zu 95.}

Madame Pomfrey kam durch die Tür und nickte Harry zu. Dann schwang sie ihren Zauberstab und untersuchte weiter. Nach kurzem nahm sie Dumbledore mit und zu Viert verließen sie das Büro. Harry blieb alleine zurück. Unschlüssig sah er sich um. Außer ihm war nur Fawkes im Raum. Sein Blick schweifte durch den Raum und blieb an der Tür hängen, hinter der das Denkarium stand. Harry überlegte kurz und  öffnete die Tür. Nach einem kurzen, entschuldigenden Blick zu Fawkes, nahm er den Gedankenfaden von Snape heraus und leerte ihn in das Denkarium. Das leere Glasröhrchen verschloss er wieder und schob es in seine Tasche. Mit dem Finger rührte er einmal kurz um und tauchte in die Erinnerungen ein.

\begin{traum}
Zuerst hatte er das Gefühl, zu fallen. Doch dieses Mal war er darauf vorbereitet. Er landete zwar immer noch unsanft, konnte sich aber abfangen, indem er in die Hocke ging. Die Umgebung verfestigte sich und eine Tür hinter ihm wurde geöffnet. Ein kleiner Junge von etwa neun oder zehn Jahren kam aus einem Haus.

\enquote{Severus, bleib nicht zu lange fort. Es gibt bald Mittagessen.}

\gedanke{Wahrscheinlich seine Mutter}, dachte Harry. Er konnte eine gewisse Ähnlichkeit feststellen.

Die langen schwarzen Haare hingen glatt aber sauber herunter. Er hatte ein weißes Hemd an, das ihm zwei Nummern zu groß war und darüber eine Jacke. Ebenfalls zu groß. Er lief durch ein paar Straßen eines dreckigen Arbeiterviertels zu einer Wiese. Das Viertel, in dem der junge Severus wohnte, war in einem Teil der Stadt, in dem ein Kohlekraftwerk stand. Die umliegenden Bewohner arbeiteten in eben diesem.

Auf der Wiese angekommen, sah er ein junges Mädchen mit roten Haaren.

\enquote{Hallo Severus. Schön, dass du da bist.}

\enquote{Hi Lily. Ich habe leider nicht viel Zeit, da ich bald zum Mittag muss. Aber eine Stunde ist schon noch drin.}

\gedanke{Lily. Sie heißt wie meine Mum. Und sie sieht ihr auch noch ähnlich.} Er schaute sie begeistert an und interessierte sich nicht für die Worte, die sie wechselten. Es ging hauptsächlich darum, dass Lily etwas über Hogwarts erfahren wollte.

Nach einer Weile kam ein anderes Mädchen zu den beiden. Sie hatte blonde Haare.

\enquote{Hi Tunia. Setz dich zu uns.}

Die Blonde hatte aber kaum Interesse daran. \enquote{Es gibt bald Essen, Lily. Komm mit. Du kannst dich nachher noch mit Severus unterhalten. Reicht es dir nicht, was du schon weißt. Den Rest wirst du schon noch erfahren.}

\enquote{Du hast recht}, sagte Lily und stand auf. \enquote{Bis später, Severus.}

Severus nickte und verabschiedete sich von Lily mit einer kurzen Umarmung. Petunia schüttelte er kurz die Hand und ging dann zurück.

Harry wartete bereits auf den grauen Schleier, der ihn zur nächsten Erinnerung bringen sollte. Doch es passierte nichts. Also folgte er Severus. Vor dem Haus seiner Eltern angekommen, öffnete der Junge die Tür und ging hindurch. Wenn Harry in diesem Moment stofflich gewesen wäre, hätte er die Tür voll auf die Nase bekommen, doch so konnte er einfach hindurchlaufen. Zum ersten Mal sah er etwas Privates von Snape, etwas sehr Privates! Der Gang, in dem er stand, war klein und schmal. Er war zwar anders eingerichtet, aber es erinnerte ihn an das Haus seines Onkels und seiner Tante. Die Tapete war dunkel und schälte sich schon von oben um einige Zentimeter von der Wand herab.

Im kleineren Esszimmer, etwa halb so groß wie zu Hause und durch eine Wand vom Wohnzimmer und der Küche getrennt, saßen nun Severus, seine Mutter und sein Vater. Seine Mutter war eine gutmütig aussehende Frau mittleren Alters. Sie hatte vereinzelt graue Haare, außerdem hatte sie ein paar Pfunde zu viel auf den Hüften, wirkte aber nicht dick.

Sein Vater hatte einen Drei-Tage-Bart und sah aus, als ob er dem Alkohol zugetan war. Er hatte müde Augen. Eventuell hatte er gerade eine Nachtschicht hinter sich gebracht, denn direkt nach dem Essen gab er seiner Frau einen Kuss, drückte seinen Sohn kurz und schaute über seine Hausaufgaben. Dann ging er ins Bett.

Severus half seiner Mutter. Er hatte schon ein paar Zauber drauf, die er aber laut seiner Mutter nicht vor seinem Vater anwenden durfte.

\enquote{Denk an Vater. Pass auf. Er mag es nicht, wenn wir zaubern.}

Der kleine Severus nickte und schwang Mamas Zauberstab.

\enquote{Wann ist es denn so weit?}, fragte er.

\enquote{In vier Wochen. Dann fahren wir in die Winkelgasse. Dein Brief von Hogwarts ist vor einer Stunde angekommen. Er liegt auf deinem Bett unter der Decke.}

Der junge Severus rannte aus dem Zimmer und Harry wollte ihm folgen, doch die Umgebung veränderte sich in einen grauen Schleier.

Es dauerte mehrere Sekunden, bis eine neue Umgebung erschien. Severus saß mit Lily im Hogwarts-Express. Beide saßen in einem Abteil. Sie saßen sich gegenüber. Harry nahm noch eine weitere Person wahr. Sie war aber ganz verschwommen. Man konnte nur eine Silhouette erkennen, die aber an den Rändern schon mit dem Hintergrund zu verschmelzen schien. Die beiden schienen sie nicht zu beachten.

\enquote{Ich bin so aufgeregt}, plapperte Lily, als hinter Harry, der im Abteil stand, die Tür aufging. Er drehte sich um und sah einen Jungen, der eine ähnliche Brille aufhatte wie er selber.

\enquote{Hi junges Fräulein, was willst du denn bei dem Langweiler? Komm doch lieber zu uns. Da ist was los.}

Hinter dem Jungen waren noch zwei weitere; einer mit schwarzem gelocktem Haar und einer mit braunen Haaren und scheinbar dreckiger Oberlippe. Aber wenn man genauer hinsah, bemerkte man schon etwas Bartwuchs. Es dauerte ein paar Sekunden, bis Harry erkannte, dass sein Vater vor ihm stand. Hinter ihm Sirius und Remus.

\enquote{Was ist jetzt. Oder möchtest du weiter bei Schniefelus bleiben?}, fragte er erneut nach und warf einen gehässigen Blick zu Severus.

\enquote{Hau bloß ab. Ich entscheide selber, mit wem ich meine Zeit verbringen. Mit dir aber garantiert nicht.}

Für den Moment abgestraft trollten sich die drei wieder. Es sollte aber nicht das letzte Treffen sein.
\end{traum}

Neben der einen Erinnerung, die Harry von Professor Snape sah, wurde ihm langsam bewusst, dass sein Vater in jungen Jahren genau so war, wie ihn Severus Snape beschrieben hatte, denn schon in der nächsten Szene kam eine weitere Gemeinheit.

\begin{traum}
Die Umgebung verschwand wieder in einem grauen Schleier und offenbarte kurz darauf die Große Halle. Dumbledore, etwas jünger, saß in seinem Stuhl und wartete darauf, dass sich die große Flügeltür öffnete. Nach kurzer Wartezeit tat sie dies auch und hereintraten, hinter Professor McGonagall, die neuen Erstklässler. Harry erkannte seine Mutter, seinen Vater, Remus, Sirius und Severus. Harry sah, wie Severus schweren Herzens zusehen musste, wie der Hut seine Mutter in das Haus Gryffindor brachte, denn für Severus stand es außer Frage, dass er selber in Slytherin landen würde.

Und da war wieder diese Gestalt, die gleich nach seiner Mutter aufgerufen wurde. Professor McGonagalls Lippen bewegten sich zwar, doch er hörte keinen Ton und auch der Mund war wie durch einen Hitzeschleier unkenntlich gemacht, sodass er nicht einmal von den Lippen lesen konnte. Die Person wurde ebenfalls nach Slytherin gebracht. Harry kam das alles sehr merkwürdig vor. Wollte Snape nicht, dass er erfuhr, wer noch mit ihm in derselben Klasse war?

Die Szene mit Severus, der in der Luft bis auf die Unterhose ausgezogen baumelte, übersprang er mit mäßigem Interesse. Er hatte sie schon einmal gesehen. Er wunderte sich zwar, warum er sie ihm zeigte, aber andererseits hatte er sie schon einmal gesehen. Es machte keinen Unterschied. Zumindest aus Sicht von Severus Snape. Harry hätte darauf aber verzichten können.

Der graue Nebel kam ein weiteres Mal. Dann lichtete er sich wieder. Severus war außerhalb des Schlosses. Es war gerade Vollmond. Severus schlich vier Gestalten hinterher. Zum ersten Mal sah er, wie Sirius ihn in den Gang lockte, durch den kurz zuvor Remus gegangen war. Nur durch das beherzte Eingreifen seines Vaters konnte Severus dem sicheren Tod entgehen. Dieser zog ihn mit aller Gewalt von dort weg, was einen Beinbruch bei Severus zur Folge hatte.

Ein erneuter grauer Schleier und die beiden lagen auf der Krankenstation. Lily kam herein und trat zu James Potter heran.

\enquote{Warum hast du ihn gerettet?}, fragte sie ihn.

\enquote{Weil dich sein Verlust sehr traurig gemacht hätte}, gab er ihr als Antwort.

Als sie nach einer kurzen Unterredung mit Severus, der Harry nicht zuhörte, wieder ging, sah sie noch einmal zu James Potter, bevor sie aus dem Blickfeld verschwand.
\end{traum}

\gedanke{Das muss der Zeitpunkt gewesen sein, an dem sich meine Eltern näher gekommen waren}, dachte Harry.

\begin{traum}
Dann folgte die letzte Szene. Es gab einen Streit zwischen Lily und Severus, infolgedessen er sie Schlammblut nannte. Das hinterließ eine tiefe Narbe auf ihrem Gesicht. Zwar keine physische, aber der seelische Schmerz war ihr in diesem Moment genau anzusehen.
\end{traum}

Die Erinnerung war zu Ende und Harry hatte das Gefühl aufzusteigen.

Mechanisch holte er seinen Stab und das Glasröhrchen, entkorkte es und holte die Erinnerungen aus dem Denkarium heraus, legte sie sorgfältig in das Gefäß und verschloss es. Danach räumte er es mit seinem Stab zurück in seine Tasche. Nachdenklich ging er zu Fawkes und streichelte ihn. Der bunte Vogel ließ sich das gefallen und zwitscherte und sang ohne Unterbrechung. Die dunklen Gedanken wurden für den Moment von Harry genommen. Er genoss es, wieder einmal einen dieser seltenen Momente erleben zu dürfen, in denen er einfach nur glücklich war.

Dann spürte er, wie sich eine Person der Pforte näherte. Obwohl ihm dies noch nie zuvor untergekommen war, war er sich sicher, dass es Dumbledore war. Vor der Wand angekommen, bat er um Einlass. Harry musste schmunzeln bei diesem Gefühl. Ihm war so, als ob der Schulleiter nicht mehr in sein eigenes Büro kam. Er ging zur Holztür, öffnete sie und sagte wie auch schon zuvor: \enquote{Öffnen.}

Dann ging er zurück und setzte sich auf einen Stuhl an Dumbledores Schreibtisch. Gedankenverloren sah er zu Fawkes, der ihn mit schrägem Kopf ansah. Keiner der beiden hatte eine Ahnung, was der andere gerade dachte. Aber genau das gab Harry Kraft. Einfach nur dazusitzen und den schönen Vogel anzusehen.

\enquote{Harry?}, fragte Dumbledore ganz erstaunt, als er sein Büro betrat. \enquote{Hast du mir aufgemacht? Was machst du hier?}

Harry sah Dumbledore an. \enquote{Ich halte hier die Stellung, da sonst keiner mehr in das Büro kommt, solange die Wand kein Passwort hat und ich weiß nicht, wie man eines setzt}, gab er ehrlich zur Antwort.

\enquote{Aber, wie bist du hereingekommen?}

Statt eine Antwort zu geben, sah Harry zu Fawkes. Dieser gab nur einen tirilierenden Laut von sich und schaute zurück. Dumbledore sah zwischen beiden hin und her. Er schien die Antwort nicht weiter zu hinterfragen, obwohl es den Anschein erweckte, dass er vermutete, nicht die ganze Wahrheit erfahren zu haben. Zumindest hatte Harry nicht gelogen, denn es war immerhin Fawkes, der ihm damals den Zugang zum Büro ermöglicht hatte. Außerdem würde er ihn nie missbrauchen.

\stimme{Du hast sogar immer Zutritt zum Büro des Schulleiters, Harry. Alle Erben der Gründer haben das. Du musst dich dem Wächter nur offiziell zu erkennen geben. Allerdings wissen die anderen das dann auch. Achte also darauf, falls du in Versuchung kommen solltest}, hörte er Agathas Stimme.

\gedanke{Wissen die das nicht sowieso?}

\stimme{Nein. Bei einigen ist es eine Vermutung, die sie aber für sich behalten. Sonst geben sie sich der Lächerlichkeit preis, falls es nicht stimmen sollte.}

\enquote{Warum bist du noch hier, Harry?}, fragte ihn Albus.

\enquote{Ich wollte dich eigentlich um etwas bitten. \gst}

\enquote{Aber?}

\enquote{Deine temporäre Abwesenheit\abs}, bei diesem Ausdruck musste Albus schmunzeln, \enquote{hat mich veranlasst, mir dein Denkarium kurzfristig auszuleihen}, antwortete er nun ehrlich. \enquote{Ich hoffe, du bist nicht sauer.}

Albus schüttelte nur den Kopf. \enquote{Warum sollte ich? Es steht eh die meiste Zeit nur herum. Außerdem hast du schon Erfahrungen gesammelt. Ich habe schon mal mit dem Gedanken gespielt, es an einem etwas zugänglicheren Ort aufzustellen. Aber der eventuelle Ansturm macht mir noch Angst. \gst Gibt es noch etwas?}

\enquote{Ja. Es geht um Hagrid und sein Verbot, zu zaubern.}

\enquote{Da kann ich dir leider nicht helfen. Es gibt keine Beweise, dass er diese Tat, die ihm zur Last gelegt wird, nicht begangen hat.}

Harry zog das Glasröhrchen mit der Erinnerung heraus und stellte es auf den Tisch. \enquote{Das ist so nicht ganz richtig.}

\enquote{Wessen Erinnerungen sind das?}, fragte Albus nach.

\enquote{Toms Erinnerungen}, antwortete Harry. \enquote{Hagrid hat nicht den Basilisken frei gelassen. Er hat eine Agramantula im Schloss versteckt, welche später in den Wald umgezogen ist. Tom Riddle hat das herausgefunden und die Spinne verscheucht. Zur gleichen Zeit hat der Basilisk ein junges Mädchen umgebracht; Myrte. Er hat danach Hagrid beschuldigt und alle Beweise sprachen gegen ihn. Zwar reichten sie nicht für eine Verurteilung und Askaban aus, aber der Verweis von der Schule und das Zerbrechen seines Zauberstabes taten es als Strafe.}

Albus schien sprachlos zu sein. Vorsichtig, als würde es beim bloßen Ansehen zerbrechen, hob er das Glasröhrchen hoch und betrachtete den schimmernden Faden darin. \enquote{Das hast du dir in meinem Denkarium angesehen?}

\enquote{Nein}, antwortete Harry. \enquote{Aber ich würde mir diese Erinnerung gerne\abs einmal anschauen.} Beinahe hätte er \accentuate{noch einmal} gesagt.

Albus stand auf und holte das Denkarium heraus. Er nahm eine schmale metallene Schale vom steinernen Sockel und ließ sie auf den Tisch schweben. Dann entkorkte er das Glasfläschchen und rührte mit seinem Zauberstab um.

\enquote{Was passiert eigentlich, wenn man das mit seinem Finger tut}, fragte Harry.

\enquote{Nichts. Dann kann man die Erinnerung nicht ansehen. Es muss mit einem Stab gemacht werden.}

\enquote{Hat es schon einmal jemand versucht?}, fragte Harry weiter.

\enquote{Nicht, dass ich wüsste.}

\gedanke{Also könnte die komische Gestalt nur ein Fehler sein, weil ich mit dem Finger umgerührt habe}, ging Harry durch den Kopf.
% Später wird Harry erfahren, dass diese komische Gestalt seine Tante war.

\enquote{Wollen wir?}, fragte Dumbledore.

Harry nickte und beide berührten mit der Nasenspitze die Oberfläche der Flüssigkeit. Beide fielen sie, bis sie im Schloss landeten, Harry fast so elegant wie Dumbledore.

\begin{rueckblick}
Dumbledore war um etwa fünfzig Jahre jünger und Tom Riddle schlich eine Treppe hoch. Er sah, wie eine Trage mit einem Leichentuch und einem Körper darunter vorbeigetragen wurde.

\enquote{Tom? Was machen Sie hier? Wissen Sie nicht, dass eine Ausgangssperre besteht?}

\enquote{Doch Professor Dumbledore, aber ich wollte es mit eigenen Augen sehen, ob die Gerüchte stimmen.}

\enquote{Leider ja, Tom. Aber Sie sollten jetzt zu Bett gehen.}

Tom Riddle nickte, blieb aber noch für eine Frage stehen, bevor er ging. \enquote{Was passiert eigentlich mit dem Verantwortlichen?}

\enquote{Er wird wohl von der Schule fliegen. \gst Wissen Sie vielleicht etwas darüber?}

\enquote{Nein Sir. \gst Gute Nacht.}

\enquote{Gute Nacht, Tom.}

Tom ging die Stufen hinab und Harry und Albus folgten ihm. Sie gingen die Treppen bis ganz nach unten. Dort schlich sich Tom an einen großen Jungen heran. Es war Hagrid. Es folgte ein kleiner verbaler Schlagabtausch, dann ein Blitz und der Deckel einer Kiste, vor der Hagrid stand, flog auf. Heraus kam in einem Tempo, sodass man das Geschöpf fast nicht sehen konnte, eine Spinne und verschwand in den dunklen Gängen.

\enquote{Professor Dippet wird sich sehr freuen, dass du so gefährliche Kreaturen ins Schloss bringst.}

Dann verschwand Tom und die Szene löste sich auf. Sie manifestierte sich wieder vor dem Eingang zur Kammer im Mädchenklo. Der junge Tom sprach Parsel und der Zugang gab eine Röhre frei. Nacheinander rutschten die drei hinunter. Harry und Albus durch den jungen Tom hindurch, da er nach der Landung nicht gleich loslief. Vor dem versiegelten runden Tor angekommen, sprach Tom etwas auf Parsel. Danach ging er durch die sich öffnende Tür und durch den langen Gang bis an das Kopfende. Er sprach unentwegt auf Parsel.

\enquote{Komm zu mir}, übersetzte Harry.

Als der Basilisk herauskam und mit geschlossenen Augenlidern auf Tom traf, wirbelte die Umgebung wieder und wurde grau.
\end{rueckblick}

Danach stiegen beide Besucher auf und landeten in Albus’ Büro.

\enquote{Wird das ausreichen, um Hagrids Unschuld zu beweisen?}

\enquote{Bei dem Zustand im Ministerium? Nur wenn wir den richtigen Mitarbeiter erwischen.} Harry nickte und schaute betrübt zu Boden. \enquote{Woher hast du die Erinnerung eigentlich?}, fragte Albus.

Harry spürte wieder dieses Kratzen. Das Einzige, was ihm einfiel, war Marcel, den er in seine Gedanken projizierte und Albus zeigte. Dadurch irritiert oder aufgeschreckt, rückte er schlagartig nach hinten und fiel fast von seinem Stuhl.

\enquote{Ich muss noch Schulaufgaben machen}, sagte Harry, stand auf und verließ eiligst das Büro.

Den ganzen Weg zurück zum Gemeinschaftsraum musste er in sich hinein grinsen. \gedanke{Das muss ich mir merken. Wer auch immer in meinem Geist herumwühlt, bekommt einen Basilisken zu sehen.}

Gerade eben lief er am Bild von Adriana vorbei, als ihm wieder einfiel, dass er sie schon seit längerem etwas fragen wollte. Er wollte wissen, ob sie es war, die das Buch über die Dementoren geschrieben hatte. Er drehte noch einmal um und lief die wenigen Meter zurück. Dann sah er auf das Bild, an dem er immer wieder vorbeiging. Er sah die Frau mittleren Alters an. Sie schien gerade in einem Stuhl sitzend zu schlafen. \enquote{Mrs De Mimsy-Porpington?}, fragte er. Die Frau öffnete ein Auge. \enquote{Mrs Adriana de Mimsy-Porpington?}, fragte er genauer nach. Die Frau öffnete das andere Auge und richtete sich in ihrem Stuhl auf. Dann nickte sie. \enquote{Haben Sie das Buch über die Dementoren geschrieben?}, fragte er. Adriana hob nur fragend eine ihrer Augenbraue. \enquote{Haben Sie das Buch \buchtitel{Vom Inferi zum Dementoren} geschrieben?}, fragte er genauer nach. Adriana nickte. Harry kam das komisch vor. \enquote{Können Sie nicht sprechen?}, fragte er nach. Adriana nickte erneut. Dann begann sie mit Gebärdensprache. Doch Harry konnte damit nichts anfangen. Resignierend hörte sie wieder auf. Harry wusste jetzt zwar, dass sie die Autorin war, aber er konnte mit ihr keine Unterhaltung führen. Dann fragte Harry sie nach zwei Begriffen, die sie ihm durch Gebärden mitteilte. Harry bedankte sich und nahm sich vor, in der Bibliothek von Hogwarts danach zu suchen. Dann ging er weiter zu seinem Gemeinschaftsraum.

Doch er drehte plötzlich um und wäre fast mit jemandem zusammen gestoßen. \enquote{Tut mir leid, Katharina}, sagte er, als sie in seinen Armen lag, da er instinktiv zulangte. Dann ließ er sie los.

\enquote{Du warst wohl in Gedanken?}, fragte sie.

Er nickte nur. \enquote{Mir ist gerade etwas eingefallen. Ich wollte zur Bibliothek.}

\enquote{Sagst du’s mir, oder ist das Geheim?}

Harry sah sie erst eine Weile an. Er überlegte, ob er ihr vertrauen konnte. Aber die Tatsache, dass er ihr bei ihrem Medusen-Problem geholfen hatte, weil sie ihn darum bat, brachte den Ausschlag. \enquote{Komm mit, wenn du willst.} Zusammen liefen sie zur Bibliothek. \enquote{Ich suche etwas zur Mondbibliothek.}

\enquote{Wozu? \gst Warte mal. Mein Großvater hat diesen Begriff ein paar Mal verwendet. Er konnte mir aber nicht viel dazu sagen.}

\enquote{Jede Information ist mir hilfreich}, antwortete Harry.

\enquote{Ich kann ihn leider nicht mehr fragen}, sagte Katharina, als sie angekommen waren. Sie folgte Harry durch die Gänge. \enquote{Er ist letztes Jahr gestorben.}

\enquote{Tut mir leid}, gab Harry erschrocken und ehrlich betroffen zurück.

\enquote{Das muss es nicht. Er war schon lange Krank.}

Harry nickte und lief weiter. Nach ein paar Biegungen stoppte er und überlegte, wo er suchen könnte.

\enquote{Was suchst du?}, fragte ihn Katharina.

\enquote{Wonach ich suchen muss. Mondbibliothek, Bibliothek oder ein anderer Begriff.} Harry entschied sich, erst einmal, nach dem \accentuate{M} zu suchen. Er ging in den Bereich der Bibliothek, der nicht nach Themen, sondern nach dem Alphabet sortiert war. Hogwarts’ Bibliothek war nämlich zwei geteilt, es gab einen Bereich mit Themen-Sortierung und einen Bereich mit alphabetischer Sortierung. Er suchte im Alphabet die Stelle, in der die Bücher mit \accentuate{M} enthalten waren. Als er sie gefunden hatte, stand er zwischen zwei Regalreihen vor einer Holzpaneele. Sie zeigte wie üblich zwei Buchstaben, die neben einem runden Knopf in der Mitte angebracht waren. \accentuate{L} und \accentuate{M}. Harry drückte das \accentuate{M} und danach den runden Knopf in der Mitte. Nur für ihn teilte sich optisch das Holzpaneele. Er schnappte sich Katharinas Hand und zog sie gerade aus. Im inneren angekommen, ließ er sie los und sah sich nach einem Buch um, das den passenden Titel hatte. Oder eines, das nahe dran war.

Katharina staunte. \enquote{Ich habe gar nicht gewusst, dass es hier versteckte Reihen gibt}, sagte sie.

\enquote{Das weiß ich auch noch nicht so lange. Ehrlich gesagt, habe ich es nur erfahren, weil ich mit Luna\abs Na ja. Auf jeden Fall sind wir unserem Lehrer hinterher und so habe ich erfahren, dass es diese Reihen gibt. Das ist erst das zweite Mal, dass ich hier bin. Die anderen Sachen findet man im normalen Bereich.} \gedanke{Oder in der verbotenen Abteilung}, fügte er gedanklich hinzu.

\enquote{Oder in der verbotenen Abteilung}, sagte Katharina und sah ihn spitzbübisch lächelnd an.

Mit einem Unschuldsgesicht sah er sie an. \enquote{Wo du dich überall herumtreibst}, sagte er.

Ihr Lachen auf ihrem Gesicht brachte ihn ebenfalls zum Lachen. Nachdem sich beide beruhigt hatten, suchten sie diesen Gang ab. Doch weder in diesem, noch im Gang mit dem \accentuate{B} wurden sie fündig.

\enquote{Wie wäre es mit \accentuate{W} wie \accentuate{Wissen}?}, fragte sie.

\enquote{Gut, versuchen wir es}, meinte Harry.

Dort angekommen, hatte Katharina das Tor geöffnet, nachdem ihr Harry gesagte hatte, was sie zu tun habe. Sie machten sich auf die Suche.

\enquote{Wie wäre es hiermit?}, fragte Katharina und zog ein Buch heraus. Sie schlug es auf und überflog das Inhaltsverzeichnis.

Harry kam näher und verdrehte sich, nachdem er sich vorgebeugt hatte, um den Buchtitel vom Umschlag zu lesen. Katharina schlug den Buchdeckel wieder zu, damit Harry aufrecht stehend lesen konnte. \buchtitel{Wissenswertes über den Umgang mit der Magie}

Im Inhaltsverzeichnis stand unter anderem \accentuate{Aufbau der Magie} und \accentuate{Wissenssammlungen}.

Katharina blätterte an die erste Stelle und las etwas über den Aufbau der Magie. Es war eine ausführlichere Beschreibung dessen, was ihnen Professor Elber einmal während einer Vertretungsstunde gesagt hatte. Dann blätterte sie weiter und beide lasen. Das Kapitel war wie ein Bericht eines Forschers geschrieben, der mit jemanden per Brief kommuniziert.

\begin{buch}
Es gibt drei große Sammlungen von Wissen auf der Welt. Wenn man die Bibliothek von Hogwarts hinzunimmt, dann sind es sogar vier. Die eben genannte Bibliothek ist die wohl größte, frei zugängliche Sammlung an Wissen über Magie. Die anderen Sammlungen sind die chinesische Hang-Wu-Schriftensammlung, zu denen nur ein ausgewählter Personenkreis Zugang hat. Sie beinhaltet einen großen Teil der neueren Zauber und Flüche unserer Zeit. Im arabischen Raum gibt es eine Ibn-Al-Flachai genannte Sammlung von Sprüchen aus älteren Tagen. Nicht alle dieser in der, ebenfalls recht restriktiven, jedoch etwas freizügigeren und nach überprüfter Anmeldung zugänglichen, Bibliothek vorhandenen Schriften werden der guten Seite zugeordnet. Etwa die Hälfte der Schriften werden zu den dunklen Künsten gezählt.

Doch die wohl geheimnisvollste ist wohl die Mondbibliothek. Sie soll laut Angaben sämtliches magisches Wissen seit dem Anbeginn der Magie enthalten. Jeden Zauber, Gegenzauber, Fluch und Gegenfluch soll sie enthalten. Sämtliche Tränke sollen dort vorhanden sein. Es ranken sich abenteuerliche Mythen um dieses Machwerk. Eine zuverlässige Quelle berichtet, sie sei von einem schwarzen Magier erschaffen worden. Eine andere zuverlässige Quelle berichtet, es sei die Magie selbst, die sich diesen Ort des Wissens geschaffen hat. Andere ebenso zuverlässige Quellen behaupten, einer der ersten Magier sei so langlebig, dass er sich hier ein Gedächtnis geschaffen hat, mit dem er verbunden sei, um nicht zu vergessen. Er soll sie so verzaubert haben, dass sie alles Wissen sammelt und ihm dann mitteile, wenn er es denn brauche.

Sie sehen also, es ist nicht leicht, etwas darüber zu erfahren. Ich habe es mir zur Aufgabe gemacht, verlässliche Daten zu sammeln und niederzuschreiben. Doch ich habe kaum welche gefunden \gst zumeist in Form von Rätseln. Das aussagekräftigste Rätsel habe ich mir notiert.

Es lautet: \enquote{Um dorthin zu gelangen, nimm den direkten Weg. Wähle den richtigen Zeitpunkt und du wirst dein Ziel erreichen. Übe das Reisen ohne Zeit und du wirst erkennen, wann und wohin du musst. Das Wissen wartet auf dich und wird ständig wachsen. Ein Leben reicht nicht aus, um alle Geheimnisse zu ergründen. Aber hüte dich vor dem Wächter. Wenn du ihn gegen dich hast, hilft auch keine Flucht mehr. Dann hilft nur noch beten und der Übergang in das Danach.}

In einem ersten naiven Versuch an das Rätsel heranzugehen, habe ich erst einmal alles wörtlich genommen. Mondbibliothek. Mir fallen zwei Sachen dazu ein. Eine Bibliothek in einer Kugel. Oder eine Bibliothek auf dem Mond, beziehungsweise unter dessen Oberfläche. Es heißt schließlich: \enquote{\aabs nimm den direkten Weg.} Der nächste Hinweis: \enquote{Wähle den richtigen Zeitpunkt\abs} könnte auf eine bestimmte Mondphase hinweisen. Neumond wäre wohl nicht praktikabel, da man nichts sieht. Vollmond wäre da praktischer. Auf den Versuch, dort hin zu apparieren, könnte der Hinweis \enquote{Übe das Reisen ohne Zeit} deuten. Apparieren braucht ja keine Zeit. Dass man für das ganze Wissen mehr als ein Leben braucht, scheint zu beweisen, oder zumindest darauf hinzudeuten, dass diese Bibliothek schon sehr lange besteht. Und der Hinweis mit dem Wächter könnte auf ein Fabelwesen deuten, das Zauberern und Hexen gefährlich werden könnte. Mir fallen da nur Drachen, Greife, Harpyien und Furien ein. Mit den restlichen gefährlichen Tieren kommt ein durchschnittlich begabter Zauberer zurecht.

Da ich begeisterter Hobbyastronom bin und mich mit dem Mond auskenne, zumindest behaupte ich das von mir, habe ich ihn mal abgesucht, aber nichts gefunden. Entweder ist das mit dem Namen nicht wörtlich zu verstehen, oder aber der Zeitpunkt war falsch. Es könnte aber auch von hier nichts zu sehen sein.

Andere Rätsel waren leider nicht so aussagekräftig wie das eben genannte. Der Vollständigkeit halber führe ich sie einmal auf.

\enquote{Reise direkt, ohne Zeit zu verlieren, dorthin, wo du, Reisender, das Wissen zu suchen gedenkst. Reise nicht über Los. Gib deswegen kein Geld aus.}

\enquote{Betritt den Ort, an dem du Wissen suchst, unbewaffnet und ohne Scheu. Der Wächter wird dich empfangen und in dein Herz blicken. Nur, wenn er dich für würdig hält, wird er dir den ersehnten Zugang gewähren.}

\enquote{Nur der bußfertige Mensch wird sie finden. Die Bibliothek des Wissens, die alle Geheimnisse enthält.}

\enquote{Ein erleuchtetes Wesen musst du sein. Nicht diese rohe Materie. Sie muss dich umgeben, dich mit allem Verbinden. Dem Stein, dem Baum und dem Wasser, das dich umgibt. Allgegenwärtig ist sie.}
\end{buch}

Harry und Katharina schauten sich an. Sie standen wenige Zentimeter nebeneinander. Die Luft von Erotik geladen, doch jeder der beiden wusste, dass sie nicht der anderen Person in diesem Raum galt. Sie lächelten sich an.

\enquote{Weißt du, dass ich dich noch nie geküsst habe?}, fragte Harry.

\enquote{Was war mit unserem Valentiskuss?}, fragte Katharina nach.

\enquote{Der war verabredet und nur Show. Der zählt nicht.}

\enquote{Und der, als du diese Anziehungskraft hattest?}, fragte sie nach.

Harry öffnete seinen Mund, dachte aber noch einmal kurz nach. \enquote{Da haben wir uns nicht geküsst.}

\enquote{Stimmt}, gab sie zurück. \enquote{Na dann.}

\enquote{Aber dann ist der Augenblick des Moments weg}, warf Harry ein.

\enquote{Und wenn du es nicht tust, dann machst du dir bestimmt Vorwürfe.}

\enquote{Und wenn wir ihn verschieben, falls ich mir Vorwürfe mache?}

Katharina dachte kurz nach. \enquote{Und was ist mit mir?} Sie stellte das Buch an ihren Platz zurück und nahm dann Harrys Kopf zwischen ihre Hände. \enquote{Diese Gelegenheit kommt nie wieder, glaube mir.} Sie küsste ihn sanft. Es war ein unschuldiger Kuss. Ein Kuss, den eine Schwester ihrem Bruder geben würde, um ihm zu zeigen, dass sie immer hinter ihm steht, egal was passiert.

Harry lächelte dankbar und umarmte sie. Er sog den Duft ihrer Haare ein, die für wenige Sekunden sich wieder in kleine Schlangen verwandelten, weil sie die Beherrschung verlor. Als er sie wider losließ, sah er gerade noch die Schlangen, die sich wieder in Haare verwandelten. \enquote{Daran musst du noch arbeiten. Es ist gefährlich, wenn du deinen Freund küsst, oder mit ihm schläfst und er dann versteinert, weil du einen unglaublichen\abs}

Sie hielt ihm den Mund zu. \enquote{Ich kann’s mir vorstellen\abs Nein, das will ich besser nicht.} Sie ließ ihn wieder los. \enquote{Aber du hast recht. Sag mal, hast du eine Idee, wie ich das machen könnte?}

\enquote{Übe.}

\enquote{Du weißt dann aber schon, wie hoch die Zahl der Jungs auf der Krankenstation sein wird, wenn ich mit jedem erst einmal schlafen muss. Madame Pomfrey wird sich bei mir in doppelter Hinsicht bedanken.} Harry hob nur eine Augenbraue. \enquote{Erst einmal die Menge an Alraunensaft und die vielen belegten Betten, und dann noch die Menge an Verhütungstränken die ich brauche, wenn ich\abs} Sie brach ab. \enquote{Du hast das anders gemeint, stimmt’s?}

Harry nickte. \enquote{Du musst lernen, deine Gefühle zu beherrschen. Deine Gefühle dürfen deinen Kopf nicht erreichen, sie müssen am Sitz der Gefühle bleiben und nur deinen restlichen Körper erreichen. Deine Kopfhaut muss davon unbeeindruckt bleiben.}

Mit einer leicht angehobenen Augenbraue fragte sie Harry: \enquote{Übst \accentuate{du} mit mir?}

\enquote{Warum fragst du da ausgerechnet einen Gryffindor?}

\enquote{Weil du der einzige bist, dem meine Schlangen nichts ausmachen.}

\enquote{Du meinst, ich soll dich sexuell stimulieren? Würdest du dich einfach so vor mir ausziehen und dich befingern lassen?}, fragte er ungläubig nach.

Katharina wurde rot, was Harry ein Lächeln entlockte. \enquote{Ich dachte eher daran, dass du auch nackt dabei\abs Vergessen wir das. Das kann ich nicht von dir verlangen.}

\enquote{Ist dir das zu peinlich?}

\enquote{Nein, aber es wirft kein gutes Bild auf mich.}

\enquote{Du meinst, die Leute wissen schon davon?}

\enquote{Das nicht, aber\abs}

\enquote{Was ist es dann?}

\enquote{Ich mag dich nicht}, er hob eine Augenbraue an, \enquote{so sehr, dass ich mich dir vollkommen nackt präsentieren könnte.}

Harry trieb das Spielchen weiter. \enquote{Setze einen Hut auf, dann bist du nicht nackt.}

Katharina patschte ihm nun mit der flachen Hand auf seine Brust. \enquote{Du bist unglaublich,\abs unmöglich, weist du das?}

\enquote{Dass ich unglaublich bin, weiß ich und dass ich unmöglich bin auch. Spätestens, seit Hermine das mit mir gemacht hat.}

\enquote{Was mit dir gemacht?}, fragte sie nach. \enquote{Sie hat dich schon mal stimuliert?}

\enquote{Nein, Katharina. Sie hat mir, wie du eben, mit der flachen Hand\abs war das eben eine Retourkutsche?}

\enquote{Kann sein!}, sagte sie und schickte sich an, den Bereich zu verlassen.

Harry nahm das Buch mit und folgte ihr. Kurz vor dem Verlassen der Bibliothek zeigte er Madame Pince den Buchdeckel und ging weiter. Da sie jedes der Bücher in der Bibliothek kannte und mit einem Zauber belegt hatte, konnte man keine der Bücher mitnehmen, ohne sich vorher eingetragen zu haben. Aber da sie von diesen Büchern nichts wusste, konnte man sie so mitnehmen. Sie würden ihren Weg eh wieder selbstständig zurückfinden. Spätestens nach zwei Wochen waren sie wieder an ihrem Platz. Man musste sich also erneut in die Bibliothek aufmachen, falls man es länger brauchen sollte.

\enquote{Mit dem Reisen ohne Zeit könnte apparieren gemeint sein}, schlug Katharina vor.

\enquote{Aber das braucht doch Zeit. Du warst doch in der Stunde dabei, als\abs Nein, du kommst erst nächstes Jahr in den Kurs.}

\enquote{Mir ist schon klar, dass das Zeit braucht. Aber bedenke mal, wie alt das Buch ist. Es ist von Sechzehnhundert-noch-was. Ich glaube nicht, dass man damals schon gewusst hat, dass apparieren, wenn auch wenig, Zeit braucht.}

\enquote{Da könntest du recht haben.}

\enquote{Mal was anderes. Gibt es die DA überhaupt noch?}

\enquote{Ja}, antwortete er, ohne darüber nachzudenken.

\enquote{Darf ich dran teilnehmen?}

\enquote{Du bist schon die zweite Person aus eurem Haus, die mich das fragt.}

\enquote{Wer ist die andere?}, fragte sie nach.

\enquote{Das wirst du übermorgen um fünf herausfinden. Wir sind im alten Tränkezimmer im zweiten Stock.}

Katharina nickte und beide machten sich in verschiedenen Richtungen auf, um ihr Ziel zu erreichen.




\begin{kommentar}
Harry sieht sich die Erinnerungen von Severus an. Dort bemerkt er, dass eine Person ausgeblendet wurde. Zuerst denkt er, dass es daran liegt, dass er die Erinnerungen mit der Hand umgerührt hat. Aber dem ist nicht so. Erst im nächsten Teil erfährt er, dass es sich um seine Tante handelt.
\end{kommentar}
