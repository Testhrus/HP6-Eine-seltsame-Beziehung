\chapter{Aus grauer Vorzeit}


\kapitelvorwort{Denn tausend Jahre sind vor dir wie der Tag,\\ der gestern vergangen ist,\\ und wie eine Nachtwache.\\(Rulaman)}


Friedward ging mit seiner Frau Persope am Rande des Sees entlang. Sie war hoch-schwanger und ging entsprechend langsam.

\enquote{Wir sollten langsam zurück zum Schloss gehen, die Wehen kommen in kürzeren Abständen.}

Kurz darauf bekam sie einen Hustenanfall und ihr Mann stützte sie und klopfte ihr sachte auf den Rücken. Nachdem sie sich wieder beruhigt hatte, drehten die beiden um und gingen zum Schloss zurück. Einem kleinen Schmetterling, der sich auf einem Busch neben dem Weg niedergelassen hatte, flüsterte er etwas zu, worauf dieser Richtung Schloss davonflatterte. Im Schloss angekommen warteten bereits ein paar Hauselfen und zwei Hebammen, die schon vor ein paar Tagen eingetroffen waren, um die Geburt zu begleiten.

Für diese Zeit außergewöhnlich, begleitete ihr Mann sie in das Zimmer, in dem sie gebären sollte. Er stand an ihrem Kopfende, hielt ihre Hand und lächelte sie an. Beruhigend sprach er mit ihr und wischte immer wieder mit einem kühlen feuchten Tuch über ihre Stirn.

Während die Elfen im Hintergrund ihre Arbeit verrichteten und frisches heißes Wasser, sowie sterile Tücher brachten, kontrollierte eine Hebamme die Öffnung des Muttermundes und die andere überwachte die ganze Geburt und assistierte.

\begin{rueckblick}
\enquote{Sie werden Vierlinge bekommen}, sagte der Heiler im St. Helens. Persope war erstaunt und erzählte diese Neuigkeit ihrem Ehemann, als sie nach Hause kam. Dieser machte große Augen und starrte sie erst einmal eine Minute an.
\end{rueckblick}

\enquote{Pressen}, sagte die Hebamme und Persope presste und presste. Nach zehn Minuten war das erste Kind geboren. Ein Mädchen. Es wurde sofort einem der Elfen gegeben, der das Kind wusch und in warme Tücher bettete. Nach weiteren fünf Minuten kam der erste der zwei Jungen, der Zweite folgte eine Minute später. Es dauerte noch fünf Minuten, dann war auch das zweite Mädchen geboren. Jedes Kind wurde ebenfalls einem Elfen zum Waschen und Kleiden gegeben.

\enquote{Zwei Jungen und zwei Mädchen}, sagte die Hebamme. \enquote{Wie sollen sie heißen?}

\enquote{Die Erstgeborene soll \accentuate{Rowena} heißen}, sagte Persope.

\enquote{Die beiden Jungs \accentuate{Salazar} und \accentuate{Godric}}, sagte Friedward.

\enquote{Und die Jüngste: \accentuate{Helga}.}

Die Hebamme sprach einen einfachen Zauber über die Babys aus, die jeweils von einem der Elfen gehalten wurden und immer noch in der Reihenfolge der Geburt von der Mutter weg standen. Jetzt konnte man die kleinen nicht mehr verwechseln und die Elfen kamen einer nach dem anderen an das Bett der jungen Mutter, stiegen auf eine Empore, die dort für die Elfen stand, und gaben die Neugeborenen der Mutter. Diese nahm ihre vier Kinder nacheinander in den Arm und drückte sie.

Doch sie bekam wieder einen Hustenanfall. Die Elfen ließen die Kinder nach oben schweben und nahmen sie wieder in den Arm. Doch der Hustenanfall ließ nicht nach. Im Gegenteil, er wurde heftiger. Dann warf sie ihren Oberkörper nach hinten auf das Laken zurück und bekam keine Luft mehr. Sie rang nach Luft. Keiner konnte ihr helfen. Die Hebammen hatten weder eine Erklärung, noch einen Zauber dagegen. Ihr Ehemann, sowie die anwesenden Elfen versuchten alles. Doch auch sie waren erfolglos. Persope war gestorben.

Eine Stunde später saß er alleine in seinem Schloss. Seine Frau lag noch immer in dem Zimmer, in dem sie ihre Kinder geboren hatte. Die Hebammen, sowie die Elfen waren verschwunden. Einzig und alleine ein alter Elf, der der Familie schon viele Jahre diente und den seine Frau in die Ehe mitgebracht hatte, war noch anwesend. Er saß seinem Herrn gegenüber in einem Stuhl und hielt ein Glas mit Whisky in seiner Hand. Der Hausherr hatte seines auf einem Stuhl, der neben dem Sessel stand, abgestellt. Der Raum, in dem beide saßen, war dunkel. Die einzige Lichtquelle war ein Kamin in dem ein kleines Feuer brannte. Die vier jungen Kinder lagen in einzelnen Bettchen neben den beiden und schliefen.

\enquote{Ich kann mich nicht um sie kümmern}, sagte er und sah Gedankenversunken zu seinen Kindern. \enquote{So Leid es mir tut, aber ich werde sie in gute Hände geben müssen. Freunde\abs Gute Bekannte\abs Leute, denen ich vertrauen kann.}

\enquote{Ich könnte\abs}, doch Merowin verstummte.

Friedward sah ihn mit einem Blick an, der ihm sagte, was er nicht aussprechen wollte. \accentuate{Du bist zu alt. Du kannst deine Aufgaben schon nicht mehr richtig erledigen. Ich werde sie an vier Familien geben. Aber immer wieder nach ihnen sehen. Du kannst mitkommen. Jedes Mal, wenn ich meine Kinder besuchen werde.} Er pausierte kurz. \enquote{Ich werde meine Frau erst einmal begraben. Wo wir gerade vom Tod reden. Wie und vor allem wo möchtest du deine letzte Ruhe finden?}

\enquote{Ich gehöre zur Familie}, gab Merowin als einzige Antwort.

Friedward nickte und meinte: \enquote{Bei meiner Frau und \gst wenn es mal so weit ist \gst bei mir.}

Der Elf nickte.

\trenn

Nachdem er seine Frau beerdigt hatte, trug er Merowin auf, auf seine Kinder aufzupassen. Alle anderen Aufgaben seien unwichtig. Nur das Wohl seiner Kinder wäre jetzt noch wichtig. Das Schloss wäre unwichtig.

\enquote{Das Schloss werden wir, nachdem die Kinder versorgt sind, mit einem Haltbarkeitszauber belegen. Ich werde es wohl in nächster Zeit nicht mehr betreten. Was wirst du machen? Begleitest du mich? Dienst du deinem Volke? Gehst du in den Ruhestand?}

\enquote{Gebt mir etwas Zeit, Sir. Ich muss nachdenken.}

Friedward nickte. \enquote{Ich werde die Familien aufsuchen. Du kannst dann mitkommen, wenn ich sie morgen den Familien übergebe.}

Merowin nickte und kümmerte sich um die kleinen, während Friedward das Schloss verließ, um zu seinem ersten Ziel zu apparieren. Ein leises \geraeusch{Plopp} und er war weg.

Als er wieder auftauchte, musste er sich an einem Baum festhalten, damit er nicht umfiel. Kurz darauf erbrach er sich neben dem Baum.

\enquote{Es wird Zeit, dass da was anderes gefunden wird. Das ist ja ekelhaft.}

Er lief noch mehrere hundert Meter, bis er sein Ziel erreichte. Es war ein einfaches, aber gut-situiertes Landhaus. Er klopfte an und trat nach einer freundlichen Begrüßung ein.

\enquote{Friedward, schön dich zu sehen.}

Er stand in einer bäuerlichen Küche und umarmte gerade Molly, während die Tür aufging und ihr Mann William hereinkam. Friedward grinste ihn an, da ihn Molly immer noch im Arm hielt.

\enquote{Es scheint, dass deine Frau mich lieber hat}, scherzte er.

Molly ließ ihn los und knuffte auf seinen Oberarm. \enquote{Meinen Mann sehe ich öfter als dich, du Schlawiner.}

\enquote{Was willst du?}, fragte William.

\enquote{Ich möchte euch um einen großen Gefallen bitten.} Er lief zum Küchentisch und setzte sich. Die beiden folgten ihm. \enquote{Es geht um meinen Nachwuchs. Ich schaffe es nicht, mich um ihn zu kümmern. Ich möchte euch bitten, auf ihn aufzupassen\abs}

Dasselbe Szenario spielte sich bei den drei anderen Familien ab.

\enquote{\aabs Ich möchte euch bitten, auf ihn aufzupassen. Meine Frau ist gestorben und ich schaffe es nicht, da ich es ohne eine Frau\abs Ich möchte mich nicht mehr binden.}

Nach einem längeren Gespräch und der Zusage, dass sie sich um seinen Nachwuchs kümmern würde, verließ er sie wieder.

\enquote{Noch eine letzte Frage, wenn ihr es euch aussuchen könntet, würdet ihr euch um einen Jungen oder ein Mädchen kümmern?}

\enquote{Ein Mädchen}, antwortete Hiltrud.

Friedward grinste. Es ging genau auf. Zwei Jungen und zwei Mädchen.

Am nächsten Tag portete er achtmal, um seine Kinder schweren Herzens abzugeben. Jedes Mal nahm er seinen Elfen mit, der ihm folgte und die Übergabe überwachte.

\enquote{Und denkt dran. Zieht sie auf, als wäre es eure eigene Tochter. Sie muss aber immer wissen, dass ihre Mutter gestorben und ihr Vater auf Reisen ist. Er kann sich außerdem nicht um sie kümmern. Irgendwann wird der Zeitpunkt kommen, da wird sie erfahren, wer ihr Vater ist. Bis dahin werde ich als ein guter Bekannter ab und an vorbeikommen, damit wir einander kennenlernen. Ich werde nie ansprechen, was heute besprochen wurde.}

Dann verließ er die letzte Familie und kehrte mit seinem Elfen zurück zum Schloss.

Die nächsten Tage verbrachten beide damit, das Schloss zu konservieren. Dann hatte Merowin eine Idee, ein gemeinschaftliches Wappen der vier Familien zu erstellen und in der großen Empfangshalle anzubringen. So grübelten die zwei nach getaner Arbeit in der Halle und verwarfen immer wieder Vorschläge, bis am Ende ein Wappen gefunden wurde. Oben links ein goldener Greif auf rotem Hintergrund. Daneben eine silberne Schlange auf grünem Hintergrund. Unten rechts eine bronzene Meerjungfrau auf blauem Hintergrund und daneben einen schwarz-weißen Kraken auf gelbem Untergrund. In der Mitte der Wappen wurde ein \accentuate{H} platziert. Der Anfangsbuchstabe des Schlosses \gst \accentuate{Hogwarts}.

% Rowena, Salazar, Godric, Helga
Mit einem leisen \geraeusch{Plopp} verließen die beiden das Schloss und ihre Kinder, die bei den Familien in guten Händen waren. \accentuate{Mosley}, \accentuate{Slytherin}, \accentuate{Gryffindor} und \accentuate{Anecks}. Mit einer Träne tauchten beide wieder auf. In der Wüste Jordaniens, wo sie eine Weile bleiben wollten, bis Merowin alt genug wäre und wieder zurückreisen müsste, um seine letzte Ruhestätte zu finden. Elfen spürten meistens, wann ihre Zeit gekommen war und Merowin wusste schon seit drei Jahren, dass er noch zehn Jahre hatte, bevor er starb. Er reiste mit seinem Herrn um die halbe Welt und besuchte die abgelegensten Plätze. Diese Reise war für beide etwas. Für Friedward, um Abstand zu gewinnen und seine Frau zu vergessen; den Schmerz zu verarbeiten, und für Merowin, der seine letzten Jahre als Elf in Rente verbrachte. Er wollte nirgendwo anders sein als bei seinem Herrn. Also entschloss sich dieser, dass sie reisen würden. Merowin hatte soviel Spaß, wie zu seinen jüngsten Tagen als Elf. Und einmal im Monat, wenn Friedward für vier Tage weg war um seine Kinder zu besuchen, ruhte der Elf sich aus, falls er einmal keine Lust hatte, mitzukommen.

Als die Zeit reif war, kehrten die beiden für wenige Tage nach Hogwarts zurück, wo der Elf mit einem glücklichen Lächeln auf den Lippen im Kreise seiner Familie und seiner Verwandten starb. Friedward verließ die Ansammlung an Elfen, sobald Merowin gestorben war, da er spürte, sie würden ihn hinauswerfen. Dann verließ er Hogwarts wieder, um weitere zehn Jahre zu reisen. Er knüpfte Kontakte und erweiterte seine Künste im Bereich der Magie. Dann schaffte er es, das Apparieren für sich angenehmer zu machen.

Er spürte den Drang nach Hause zurückzukehren und tauchte wieder in Hogwarts auf. Seine Kinder hatten bereits geheiratet und lebten mit ihren Familien in ihrem eigenen kleinen Reich. So verließen Tage später vier Briefe Hogwarts. Einer an \accentuate{Rowena Ravenclaw}, einer an \accentuate{Salazar Slytherin} einer ging an \accentuate{Godric Gryffindor} und ein Brief erreichte \accentuate{Helga Hufflepuff}.

Das Wappen in der Großen Halle wurde verändert. Oben links war jetzt ein goldener Greif auf rotem Hintergrund, daneben eine silberne Schlange auf grünem Hintergrund. Unten rechts ein bronzener Adler auf blauem Hintergrund und daneben ein schwarz-weißer Dachs auf gelbem Untergrund.

Dann wartete Friedward auf die Gäste. Es dauerte ein paar Stunden, bis die Portschlüssel aktiv wurden und die Gäste mitbrachte. Jeder hatte einen kleinen silbernen Ring erhalten, den er sich anstecken musste, damit sich der Portschlüssel aktiviert. Zeitgleich trafen die Gäste ein und wurden von einem guten Freund der Familie erwartet. Das dachten sie zumindest, als sie Friedward am Tor sahen, der ihnen entgegenkam.

\enquote{Hallo meine Lieben, schön, dass ihr dem Ruf gefolgt seid.}

\enquote{Uns blieb nicht viel übrig. Der Brief war eindeutig.} Rowena stockte kurz, runzelte die Stirn und sagte dann: \enquote{Vater.}

Dann herrschte eine Weile Stille.

\enquote{Du bist wie immer die Klügste von allen. Kommt mit, ich erkläre es euch während des Essens.}

Stumm ging die Gruppe, von Friedward geführt, in das Schloss in die Große Halle zum Essen. Friedward ging um den runden Tisch herum und setzte sich der Tür gegenüber hin.

\enquote{Bitte, fangt an zu essen.}

\enquote{Und du, isst du nichts?}, fragte Salazar.

\enquote{Ich habe kurz vor eurer Ankunft etwas gegessen und esse noch etwas, wenn ihr euch beratet, bzw. das, was ich euch erzähle, sich setzt.}

Langsam und geduldig begannen seine vier Kinder zu essen, während Friedward zu erzählen begann.

\enquote{Ich fange ein paar Tage vor eurer Geburt an zu erzählen.} Da er keine Wiederworte hörte, fuhr er fort. \enquote{Eure Mutter, Persope, war mit euch Schwanger. Wir beide freuten uns auf euch. Vierlinge sind selten, aber wir hatten uns vorgenommen, das zu schaffen. Doch eure Mutter\abs} Er musste kurz unterbrechen, da sich die vier erstaunt ansahen. \enquote{Ja, ihr seid Geschwister. \gst Eure Mutter hatte sich eine schwere Lungenentzündung geholt, die nicht mehr geheilt werden konnte, da sie kurz vor der Geburt stand. Darum wollten wir uns danach kümmern. Also habe ich dafür gesorgt, dass sie immer warm angezogen war. Ihr Zustand war stabil. Dann wurdet ihr geboren. Zuerst du Rowena. Dann Salazar, Godric und schließlich Helga.} Er sah nacheinander seine vier Kinder an.

\enquote{Ihr seid Geschwister, aber auch, und das ist viel wichtiger, mächtige Hexen und Zauberer. Und deshalb habe ich euch zu mir gerufen. Einerseits, weil es Zeit ist, dass ihr mich kennenlernt. Andererseits, weil ich eine Aufgabe für euch habe. Die Situation unserer Art ist schlecht. Wir \gst das heißt, ihr \gst solltet eine Schule gründen und junge Hexen und Zauberer ausbilden.}

Alle Gabeln fielen auf den Tisch. Friedward nahm sich seine Gabel, steckte sie in geröstete Kartoffelscheiben und steckte sie danach in seinen Mund und kaute. Nachdem er seinen Bissen heruntergeschluckt hatte, fuhr er fort. Immer, wenn er eine kleine Pause machte, damit seine Kinder das gehörte verarbeiten konnten, nahm er einen Bissen vom reich gedeckten Tisch.

\enquote{Ich möchte, dass ihr euch der Ausbildung widmet. Junge Hexen und Zauberer brauchen eine fundierte Ausbildung. Und ihr hier seid dafür geeignet. Ihr seid die mächtigsten vier magisch begabten Personen dieser Zeit. Und, was noch wichtiger ist, ihr habt bereits euren Nachbarjungen und -mädchen etwas beigebracht. Dieses Schloss hier steht euch ab sofort zur Verfügung. \gst Nach dem Essen werden wir eine kleine Führung machen. Ich werde euch beim Umbau helfen, solltet ihr annehmen. Und ich werde euch die ersten Jahre als Hausmeister mit Rat und Tat zur Seite stehen. Ich werde euch aber nicht vorschreiben, wie ihr eure Aufgaben zu erledigen habt. \gst Lasst euch Zeit mit eurer Entscheidung. Ich will nichts hören, bevor wir nicht die Führung vollendet haben.}

Dann griff er endlich auch zu einem Messer und lud sich die restlichen Speisen auf seinen Teller und aß. Währenddessen saßen die vier stumm am Tisch und dachten nach.

\enquote{Dad?}

\enquote{Ja, Salazar.}

\enquote{Warum?}

\enquote{Werde genauer.}

\enquote{Warum sollen wir unterrichten?}

\enquote{Seht euch da draußen doch mal um. Die Lage ist katastrophal. Unsereins kann sich nicht richtig wehren, wenn wir von Muggeln angegriffen werden. Und vor allem, was passiert mit den ganzen magischen Kindern, die unter nicht-magischen Eltern aufwachsen. Sie müssen lernen, ihre Magie zu kontrollieren.}

\enquote{Es wird aber nicht jeder seine Kinder unterrichten lassen wollen. Viele wollen das selber machen.}

\enquote{Das geht natürlich weiterhin. Aber mit der Zeit, so hoffe ich, werden auch diese Leute merken, dass die Ausbildung hier besser ist.}

\enquote{Besser als wo?}

\enquote{Besser als bei den Eltern zu Hause. \gst Ich stelle mir das so vor. Alle magisch begabten Kinder werden in einer Liste geführt. Alle diese Kinder, die vor dem ersten September Elf werden, bekommen einen Brief zugeschickt, in dem sie eingeladen werden, die Schule zu besuchen. Das Schloss ist groß genug um die Schüler hier unterzubringen. In den Ferien werden sie nach Hause geschickt. Zumindest in den großen Ferien. Die restlichen Ferien können sie auch im Schloss verweilen.}

\enquote{Aber nicht jeder kann sich eine Ausbildung leisten}, warf Godric ein.

\enquote{Da kommen eure Familien und das Schloss mitsamt seinen Gütern ins Spiel. Die Kinder müssen nur Trankzutaten, Pergament und Tinte, Kessel und anderes Kleinmaterial kaufen, sowie eventuell, so vorhanden, die passenden Schulbücher \gst Oder normale Bücher, da es Schulbücher noch nicht gibt. \gst Der Rest, also die Unterkunft, die Verpflegung und andere Lehrmaterialien, wird von der Schule gestellt. Selbstverständlich auch die Krankenverpflegung.} Die Augen der vier wurden größer. \enquote{Wisst ihr, es soll eben nicht von der finanziellen Situation der Schüler abhängen, eine gute Ausbildung zu bekommen.}

\enquote{Aber, das viele Geld, was das kostet.}

\enquote{Das ist schon geregelt. Das Geld kommt zu hundert Prozent von mir \gst Besser gesagt, von einer Schulstiftung. In Gringotts ist bereits seit einiger Zeit ein entsprechendes Verlies mit dem Gold angelegt.}

\enquote{Mich hast du überzeugt, Dad}, sagte Helga. \enquote{Aber schauen wir uns noch die Räumlichkeiten an.}

Friedward zeigte die einzelnen Räumlichkeiten seinen Kindern und teilte ihnen mit, was für Fächer er sich vorstelle. \enquote{Am Anfang werdet es wohl nur ihr sein. Später, so hoffe ich, wird es weitere Lehrer geben.}

\enquote{Solange ich Zeit für meine Studien habe, von mir aus. Ich bin dabei}, sagte Salazar. \enquote{Wie sollen denn die Kinder untergebracht werden. Ich nehme nicht jeden}, sagte er noch.

\enquote{Wie wollt ihr denn die Auswahl treffen?}, fragte Friedward.

\enquote{Trennen wir sie doch nach Eigenschaften}, schlug Rowena vor.

\enquote{Und du nimmst dir die Besten?}, fragte Godric nach.

\enquote{Es gibt nicht die Besten. Es gibt Schüler mit stärker oder schwächer ausgeprägten Eigenschaften. \gst Ich bin dabei.}

Godric gab sich nach einem erneuten kleinen Disput geschlagen. \enquote{Da ich eh nichts anderes vorhabe, Ok.}

Friedward zeigte den Vieren noch, wo sie schlafen konnten und überließ es ihnen, das Schloss zu erkunden. Jetzt, Anfang März, dauerte es noch, bis die ersten Schüler kommen würden. Und für die Muggel-geborenen wurde vereinbart, dass der Brief persönlich überbracht werden soll, um direkt ein paar Fragen zu beantworten.

\trenn

Etwa eintausend Jahre später, Vernon Dursley stand gerade in seinem Garten und betrachtete die Ecke, in der über Nacht Pflanzen gewachsen waren.

\enquote{Petunia, kommst du mal?}

\enquote{Was ist denn, Vernon?} Petunia kam heraus und ihr Blick fiel sofort auf die Petunien, die in der Ecke gewachsen waren. \enquote{Oh Vernon, das ist so lieb von dir. Und sogar noch heute, an meinem Geburtstag.} Sie umarmte ihren Mann und gab ihm einen Kuss.

\enquote{Aber, das war ich nicht.}

\enquote{Nicht? Dann vergiss den Kuss. Aber wer war es dann?}

\enquote{Sag mal Petunia, du hast nicht etwa einen heimlichen Verehrer?}

\enquote{Nicht, dass ich wüsste. Aber die Blumen sind schön. Ich muss wieder in die Küche.} Damit verschwand sie wieder im Haus und lies einen nachdenklichen Vernon zurück. In der Küche angekommen, verrichtete sie weiterhin ihre Arbeit. Doch nach einer Weile hörte sie auf, sah noch einmal Richtung Fenster und dachte nach. \gedanke{Petunien \gst und das an meinem Geburtstag.} Sie begann leicht zu lächeln. \gedanke{Ich sollte mal wieder\abs}

Am Tag darauf erwachte Harry und war erst einmal verwirrt, als er die steinerne Decke sah. Dann nahm sein Ohr das Geklirr von Besteck und das Brechen von Zwieback wahr. Er richtete sich auf, doch er konnte die Person, die er sah, nicht mehr zuordnen, da ihm sofort schwarz vor Augen wurde und er nach hinten in das Bett zurücksackte. Wieder existierte er nur. Er roch nichts mehr, er fühlte nichts mehr, er sah nichts mehr und er hörte nichts mehr. Dann passierte etwas, was Harry noch nie erlebt hatte. Er schlug die Augen auf und lag in einem Bett. Grüne Satin-Bezüge über einer weichen Bettdecke und eine wertvolle Zimmereinrichtung. Er stand auf und sah an sich herab. Scheinbar hatte er mit seiner Kleidung geschlafen. Doch es war nicht seine Kleidung.

Er trat auf die Tür zu, öffnete sie und ging hindurch. Mehr neugierig als ängstlich ging er durch die ungewohnte und fremde Umgebung. Nach ein paar Ecken fand er eine Treppe. Er lief sie hinab. Eigenartigerweise kam ihm die Umgebung nun doch vertraut vor. Doch er konnte sie nicht genau zuordnen. Nicht so lange, bis eine Tür aufging und ein Mann herauskam. Er hatte blonde, lange Haare, welche zu einem Pferdeschwanz gebunden waren. Als er zu Harry blickte, sah er ihn ehrfürchtig, fast schon ängstlich an.

\enquote{Mylord}, nannte er ihn und ging dann durch die Große Halle in ein anderes Zimmer.

Harry blieb stehen und sah Lucius Malfoy nach. \gedanke{Er hatte Angst vor mir. Wieso hat Mister Malfoy Angst vor mir?}, fragte er sich. Er sah durch die Halle, entdeckte einen Spiegel und ging auf ihn zu. \gedanke{Und warum nennt er mich seinen Lord?} Als er in den Spiegel blickte, wusste er es. Die ersten Sekunden vergingen wie in Trance. Dann sickerte die Erkenntnis in sein Gehirn. Er war Voldemort. Jetzt könnte er Chaos verursachen. Doch was würde das Chaos für Auswirkungen haben?

Eine Tür ging auf und Bellatrix Lestrange schritt hindurch. Sofort zuckte seine Hand in der Versuchung seinen Zauberstab \gst Voldemorts Zauberstab \gst auf sie zu richten und sie zu bestrafen. Voldemorts Zauberstab. Er könnte ihn verstecken.

\enquote{Mylord}, wurde er auch von Bellatrix begrüßt.

Doch in ihrem Blick lag keine Angst. Dort fand er Hoffnung und Hingabe. Aber noch etwas entdeckte er. Ganz sachte näherte er sich ihrem Geist und drang in ihn ein. Als er Klarheit hatte, zog er sich zurück. Er fand seine Bestätigung. Sie verehrte ihn nicht nur, nein, sie liebte ihn. Das wäre doch etwas. Er könnte mit ihr\abs Und wenn er wieder zurück in seinem Körper wäre, dann würde er sie bestrafen.

\enquote{Komm mit}, sagte er, in der Hoffnung authentisch zu klingen.

Gehorsam folgte sie ihm. Er ging den Weg zurück zu seinem Zimmer. Davor angekommen drehte er sich kurz herum. \enquote{Warte kurz.} Er betrat sein Zimmer und verstaute seinen Zauberstab. Er versteckte ihn hinter seinem Nachtkästchen. Dann holte er Bellatrix herein. Kaum hatte sie den Raum betreten, stand er vor ihr und schob die Tür zu. Sie wich einen Schritt zurück und stand nun mit dem Rücken an der Tür. Harry konnte es nicht leugnen. Wenn er ihre Boshaftigkeit beiseite schob, sah sie ganz Attraktiv aus. Ihre Zähne einmal in der Farbe Weiß vorausgesetzt. Er nahm seine Hand hoch und fuhr ihr über die Lippen. Er konzentrierte sich auf einen Zahnreinigungszauber.

Bellatrix schloss ihre Augen und ein Kribbeln machte sich in ihrem Mund breit. Als sie einen Finger an ihrem Kinn spürte, der es ihr nach unten drückte, fing sie an zu lächeln und ihre Augen zu öffnen. Ihre jetzt weißen Zähne blitzten und ihr Herz begann zu klopfen.

Er löste seinen Griff von ihrem Kinn und umfasste jetzt mit beiden Händen ihre Hüfte. Langsam zog er sie zurück, worauf hin sie ihm willig folgte. Sie zeigte ihm ihr Lächeln und die neuen weißen Zähne. Er spürte, wie er begann die Kontrolle zu verlieren, doch er zwang sich, noch ein paar Sekunden durchzuhalten. Am Bett angekommen ließ er sich mit ihr zurückfallen und begann sie zu küssen. Ein paar Sekunden noch hielt er durch. Dann ließ er los.

Die Umgebung wechselte und er war wieder im Krankenflügel. Madame Pomfrey, Professor McGonagall und Professor Elber standen mit gezogenen Zauberstäben da. Ebenfalls Harry. Er ließ ihn los und griff um seinen Anhänger. Sofort stand er wieder in dem Zimmer und besah sich das Schauspiel.

Voldemort drückte Bellatrix von sich. \enquote{Was fällt dir ein}, schrie er.

Irritiert sah sie ihn an. Gerade eben war er doch noch so lieb.

Dann stutzte er und drehte seinen Kopf leicht. Er sah Harry. Dann wusste er, was Harry getan hatte. Und obwohl es ihm schwerfiel, bestrafte er Bellatrix nicht. Immerhin hatte er erfahren, dass Harry mittlerweile recht gut seine Magie beherrschte.

\enquote{Lass es gut sein, Bellatrix. Bleib einfach hier. Neben mir.} Den letzten Satz betonte er. Als Bellatrix neben ihm lag, glücklich, sah er noch einmal kurz zu Harry und lächelte ihn an. Dann schloss er selbst die Augen und entspannte. Harry zog es zurück und er dachte noch kurz nach. Voldemort dürfte die gewünschte Erinnerung an Ginny in der Kammer gar nicht haben, fiel ihm ein.

\gedanke{Warum ist das Salazar gar nicht aufgefallen?}, fragte er sich.

\stimme{Das habe ich ihn auch gefragt}, antwortete eine weibliche Stimme in seinem Kopf.

\gedanke{Agatha?}, fragte Harry nach.

\stimme{Ja, aber darüber reden wir später.}

Harry nickte innerlich. Dann war er wieder bei der Sache. Er ließ sein Amulett los und blickte in die Runde. \enquote{Habe ich mich aufgeführt?}, fragte er vorsichtig nach.

Die drei Professoren nickten nur, immer noch bereit ihn außer Gefecht zu setzen.

\enquote{Darf ich fragen, für wen Sie mich gehalten haben?}

\enquote{Für den Ungenannten}, antwortete Madame Pomfrey direkt aber zurückhaltend.

Harry nickte und hob seinen Zauberstab auf. \enquote{Das kann ich nur indirekt bestätigen. Scheinbar habe ich eine Weile mit ihm die Plätze getauscht.} Er schob seinen Stab ein, wurde aber noch immer sorgfältig beobachtet.

\enquote{Was machen wir jetzt mit ihm?}, fragte Professor McGonagall.

\enquote{Die für mich spannendere Frage ist: \inner{Was machen wir, wenn das noch mal passiert?}}, fragte Professor Elber.

\enquote{Du bist hier der Experte für diese Art von Magie. Schlag du was vor}, sagte Professor McGonagall. Elber senkte seinen Stab und schob ihn kurz darauf ein. Dann setzte er sich wieder auf sein Bett und ließ seine Beine herunterbaumeln. \enquote{Was machst du da?}, wurde er erneut gefragt.

\enquote{Ich denke nach.} Dann legte er sich auf das Bett und starrte an die Decke.

Professor McGonagall drehte sich zu Harry und sah ihn an. \enquote{Warum dürfen Sie Professor Dumbledore nicht bei seinem Vornamen anreden?}

\enquote{Wie? Ich darf sehr wohl. Allerdings nur, wenn wir alleine sind.}

Auch Professor McGonagall schob ihren Stab ein.

Nach einer Weile zog Madame Pomfrey nach und stand noch kurz da, um Harry zu beobachten. \enquote{Sie können von mir aus gehen}, sagte sie dann.

McGonagall nickte und verließ den Krankenflügel. Madame Pomfrey kümmerte sich schon um den nächsten Patienten, der bereits hereinkam und sich zur Untersuchung auf ein Bett legen musste.

Harry ging ebenfalls. Sein Lehrer würde sich schon melden. Dann fiel ihm wieder ein, dass er von der falschen Erinnerung ausgegangen war. Er suchte nicht die von Ginny in der Kammer, sondern die von Voldemort, in der er Hagrid beschuldigt hatte.

\gedanke{Aber wieso war das Agatha nicht aufgefallen?}, fragte er sich.

\stimme{Weil ich mich in deinen Gedanken verlaufen habe}, gab sie kleinlich zu.

\gedanke{Wie meinst du das?}

\stimme{Durch das Bild in unserem Raum und der Verbindung von Salazars Geist mit dem Bild, habe ich es mittlerweile geschafft auch eine Verbindung zu dir aufzubauen. Darüber war ich so erstaunt, dass ich mich einfach umgesehen habe und deine letzten Gedanken mitbekommen habe. Tut mir leid Harry, das war falsch. Das hätte ich nicht tun dürfen. Als ich es bemerkte, habe ich mich sofort zurückgezogen.}

\gedanke{Was hast du alles mitbekommen?}

\stimme{Nur die letzten paar Stunden.}

\gedanke{Gut. Dann weißt du jetzt, was ich suche.}

\stimme{Kannst du nicht einfach deine Erinnerungen nehmen?}

\gedanke{Nein, denn mir wurde nur ein Bild gezeigt. Ich brauche das Original.}

\stimme{Huch!}, sagte sie plötzlich.

\gedanke{Was ist los?}

\stimme{Ich habe gerade etwas bei dir bemerkt. Du hast einen Seelensplitter von ihm in dir.}

\gedanke{Richtig.}

\stimme{Dann brauchst du Voldemort gar nicht. Der Splitter ist so schwach, dass er sich nicht wehren kann. Es ist so, als würde er schlafen. Konzentriere dich auf die Erinnerung in seinem Geist und ziehe sie heraus.}

Harry blieb stehen.

\gedanke{Hilfst du mir?}

\stimme{Denke einfach nur an das, was du willst. Dann klappt es auch.}

Harry beschwor sich ein Glasröhrchen mit Korken herauf und suchte sich einen ruhigen Platz. Er entkorkte es und versuchte sein Glück. Er schloss seine Augen und konzentrierte sich auf den kleinen kalten Teil in sich in das er versuchte mit Legilimentik einzudringen. Doch er versagte. Er spürte nicht einmal einen Widerstand. Er spürte gar nichts. Bei jedem, bei dem er es versucht hatte und das waren eine Menge Mitschüler, hatte er Zugang zu Bildern erhalten. Er hielt sie bewusst Unscharf, da er nicht in deren Privatsphäre eindringen wollte.

Harry dachte nach. Nach einer gefühlten Ewigkeit hatte er eine Idee. \gedanke{Salazar? Agatha?}

\stimme{Ja}, sagten beide.

\gedanke{Ich habe mit Legilimentik keinen Erfolg.}

\stimme{Das haben wir bemerkt}, sagte Salazar.

\gedanke{Was habe ich sonst noch für Möglichkeiten?}

\stimme{Wenn du das nicht schaffst? Keine.}

Das gab Harry zu denken. Er holte Ginny ab und ging mit ihr spazieren.

\enquote{Harry Potter, Harry Potter!} Dobby kam auf ihn zu gerannt, als er gerade im Schloss unterwegs war.

\enquote{Dobby, schön dich zu sehen, was gibt es denn? Du bist ja ganz aufgeregt.}

\enquote{Dobby möchte Harry Potter um einen großen Gefallen bitten.}

\enquote{Gern, um was geht es denn?}

Ginny, mit der er gerade unterwegs war, blieb stehen und wartete.

\enquote{Dobby möchte gern Harry Potter als Traupaar-Führer.}

\enquote{Traupaar-Führer?}

\enquote{Ja, Harry Potter. Dobby und Winky werden bald heiraten.}

Harrys Gesichtszüge entgleisten. Er ging auf die Knie und nahm den kleinen Elfen in den Arm. \enquote{Das ist ja wunderbar, Dobby.} Dann stutze er. Er ließ Dobby los und sah ihn an. \enquote{Was ist meine Aufgabe dabei?}

\enquote{Dobby hat daran gedacht, Harry Potter.}

\enquote{Nenn mich endlich Harry, Dobby.}

\enquote{Gern, Sir Harry. Dobby hat Sir Harry Pergamente da gelassen. In Slytherins Räumen. Damit kann er sich auf die Aufgabe vorbereiten.}

\enquote{Lass mich erst einmal etwas darüber lesen. Ich weiß nichts über eure Art und eure Bräuche. \gst Wann brauchst du meine Entscheidung?}

\enquote{Ende des Schuljahres.}

Harry nickte. \enquote{Dann habe ich Zeit mir deine Informationen durchzusehen und mir meiner Aufgabe bewusst zu werden.}

\enquote{Das ist eine sehr verantwortungsvolle Aufgabe, Sir. \gst Dobby muss jetzt wieder arbeiten.} Damit verschwand er.

\enquote{Dobby will dich als Trauzeugen?}, fragte Ginny, die noch immer in der Nähe stand und daher alles mitbekam.

\enquote{Nein, er hat einen anderen Begriff verwendet. Ich weiß nicht, ob das einem Trauzeugen gleich kommt. Ich werde erst einmal darüber lesen.}

\enquote{Nimmst du mich mit?}

Harry nickte und nach dem Abendessen ging er mit Ginny zu Salazars Räumen, wo schon die Pergamente von Dobby auf dem Tisch lagen. Er nahm sie vom Tisch und begann zu lesen, während Ginny im Bücherregal ein Buch heraussuchte, es mitnahm, sich neben Harry auf das Sofa setzte und zu lesen begann. Während der nächsten Stunde, in der Harry in die Aufzeichnungen vertieft war, wurde Ginny immer müder, ihre Hände wurde schwerer und ihre Augen fielen zu. Ihr Kopf kippte leicht auf die Seite und kam auf Harrys Schulter zu liegen. Gedankenversunken nahm er sie in den Arm und las weiter, bis auch er fertig war; mit den Pergamenten und seinen Augen. Er schloss sie und legte seine Wange auf Ginnys Haar.

Unbewusst kuschelten sich beide aneinander. Das Ehepaar auf dem Bild über ihnen lächelte beide an. Danach nahm Salazar seine Frau in den Arm und schaute sie verliebt an.

Am nächsten Tag lief Harry wieder durch das Schloss. Er sollte sich heute im Pokalzimmer zu einer Unterrichtsstunde einfinden. Freudig betrat er das Zimmer und wurde bereits von Professor Elber erwartet. Dieser drückte ihm wortlos einen Eimer in die Hand. Harry stutzte. Danach nahm ihm sein Professor den Zauberstab aus dem Umhang und verließ mit den Worten: \enquote{Viel Spaß} den Raum und schloss hinter sich die Tür. Harry fragte sich, was das denn sollte. Er sah in den Eimer und entdeckte einen trockenen Lappen. Was darunter lag, sah er nicht, da er durch Rufe unterbrochen wurde.

\enquote{Putz mich}, hörte Harry.

Verwundert näherte er sich vorsichtig der Stimme, die aus einer der Vitrinen zu kommen schien.

\enquote{Putz mich}, hörte er erneut.

Er sah einen Pokal, aus dem ein Gesicht heraus schaute.

\enquote{Warum?}, fragte Harry ganz entgeistert.

\enquote{Du hast betrogen, deshalb.}

\enquote{Wann? Wo?}, fragte Harry. Er konnte sich nicht erinnern.

\enquote{Die Baumstümpfe.}

Jetzt wurde es Harry klar. Er hatte alle Baumstümpfe mit dem Zauberstab entfernt. Und nicht wie gefordert, die Hälfte ohne. Er sah in seinen Eimer und holte den ersten Putzlappen heraus. Darunter sah er weitere zwei, die er ebenfalls herausnahm und nun auf zwei Plastikfläschchen mit Schraubverschluss blickte. Auf einem stand  \accentuate{Wasser \gst Selbst auffüllend} und auf dem anderen \accentuate{Politur \gst Selbst auffüllend}.

\enquote{Putz mich}, hörte er erneut.

Also machte er sich ans Werk, tat etwas Politur auf einen der Lappen und fing an den Pokal zu reinigen, der ihn ansprach, indem er Politur auftrug und sie leicht einrieb. \enquote{Wie viele muss ich denn machen?}, fragte er, als er den ersten Pokal mit einem weiteren Lappen polierte.

\enquote{Solange, bis die Schuld abgetragen wurde}, sagte ein weiterer Pokal. \enquote{Ich bin der nächste.}

Harry kam zu ihm und bemerkte die Staubschicht auf ihm. Er sah seinen Politur-Lappen an und entschied, den Pokal erst einmal mit einem feuchten Tuch zu säubern\abs

Während seiner Strafaktion erfuhr er von jedem Pokal, den er reinigen musste, etwas über denjenigen oder diejenige, die ihn verdient hatte. Als er fertig war, schmerzten seine Arme und er konnte auf der Stelle einschlafen. Müde sank er zu Boden und wurde von dem weichen Boden aufgefangen.

\enquote{Was machen Sie hier?}, wurde er etliche Stunden später von Professor Snape unsanft geweckt.

\enquote{Polieren}, gab er matt zurück und kratze sich den Schlaf aus den Augen.

\enquote{Sieht mir aber nicht danach aus.}

\enquote{Musste die Pokale polieren\abs Habe mich bescheuert verhalten\abs Habe\abs nicht ganz sauber\abs gearbeitet.}

\enquote{Professor Elber?}, fragte Snape nach. Als ihn Harry fragend ansah, antwortete er: \enquote{Er hat was in der Richtung erwähnt.} Dann griff er in seinen Umhang und zog ein Pergament hervor. Er reichte es Harry und verschwand.

Als sich Harry das Pergament besah, fand er eine Entschuldigung für das späte Ausbleiben vor. Er konnte also zurück in dem Gemeinschaftsraum, ohne eine Strafe zu erhalten. Dann rappelte er sich auf und verdrückte sich. Und wieder überkam ihn dieser eigenartige Traum vom Anfang des Schuljahres, in dem er einen Pokal mit einem Dachs im Verlies der Lestranges sah. Dann dämmerte er in seinem Traum wieder weg.

Am nächsten Morgen wurde er von Professor Elber abgefangen. \enquote{Heute Abend bei Hagrid. Zweiter Versuch.} Dann lief er weiter und lies Harry stehen.

\enquote{Was war das denn jetzt?}, fragte Ron, der neben ihm stand.

\enquote{Ich habe beim letzten Training nicht ganz ehrlich gespielt. Darauf hin musste ich einen großen Teil der Pokale polieren. Er gibt mir wohl eine zweite Chance.}

Dann mussten sie schon zur nächsten Unterrichtsstunde.

\enquote{Harry ist dieses Jahr abweisender als sonst}, merkte Ron an.

\enquote{Er darf halt nichts über seine Extra-Stunden erzählen, genau wie du auch}, antwortete Hermine.

\enquote{Du hast ja recht. Auch wenn ich nicht verstehe, warum nicht wenigstens du mir etwas von dir erzählst. Immerhin bin ich dein Freund.}

\enquote{Du erzählst mir ja auch nichts. Und jetzt Ruhe, wir sind da.}

Der Unterricht bei Professor Flitwick verlief ruhig und nach dem Abendessen meldete sich Harry wie angewiesen bei Hagrid, der ihn in ein Waldstück führte und ihn wieder bei etwa zwanzig Baumstümpfen abstellte. Harry seufzte und setzt sich auf einen der Baumstümpfe. Die nächste viertel Stunde dachte er nach, wie er am besten vorgehen konnte. Während er seinen Gedanken nachhing, näherte sich von der Seite ein junges Testral-Männchen. Es schnupperte an seinem Haar und musste einmal durch seine Nüstern ausblasen, da Harrys Shampoo-Duft ihn in der Nase irritierte. Erst jetzt bemerkte Harry das Tier neben sich und begann es zu streicheln. Er erklärte dem jungen Männchen, was seine Aufgabe sei. Bis ihm bewusste wurde, dass das sinnlos war. Er sah wieder zu den Baumstümpfen, als er Bilder in seinem Kopf vernahm, Bilder, die ihn zeigten, wie er verschiedene Zauber auf die Stümpfe warf. Als er das Testral-Männchen ansah, nickte dieses nur einmal mit seinem Kopf und verließ ihn dann.

\gedanke{Na klar, so geht es. Immer auf andere Art und Weise. Einen Baumstumpf mit Zauberstab und Worten. Den nächsten mit Zauberstab und ohne Worte. Dann dasselbe nochmals ohne Zauberstab. Das wären dann vier Stümpfe mit demselben Zauber.} Also machte er sich ans Werk und entfernte so einen Baumstumpf nach dem anderen.

Als er fertig war, stand auch Hagrid schon wieder hinter ihm. \enquote{Gut gemacht Harry. Kannst geh’n.}

Harry war erleichtert. \enquote{Habe ich bestanden?}

\enquote{Denke schon. Hast wohl dies’mal alles richtig gemacht. Un’ nu geh mal.}

Auf dem Rückweg ging er einen Gang entlang, den er sonst selten wählte. Plötzlich blieb er stehen und sah auf eines der vielen Bilder, die alle gleich aussahen. Doch eines war leicht verändert. Es waren alles Landschaften mit Bäumen. Sein Unterbewusstsein musste den Unterschied aufgegriffen haben und mit den Markierungen auf der Karte, die er anfertigte um weitere Geheimgänge und Räume im Schoss zu erkunden, abgeglichen haben. Er sah sich das Bild genau an und wechselte zwischen mehreren hin und her. Nach mehrere Minuten entdeckte er ein fehlendes Blatt auf einem der Bilder. Mit der Hand strich er über den fehlenden Platz und hörte ein klackendes Geräusch. Der Bilderrahmen mitsamt dem Bild kam einige Zentimeter vor.

Harry schob am Rahmen das Bild beiseite und trat in den Gang dahinter. Mit gezücktem und leuchtendem Zauberstab ging er den Gang entlang. Ein klackendes Geräusch ließ ihn umsehen. Das Bild hinter ihm verschloss den Gang wieder. Harry setzte seinen Weg fort. Er versuchte sich den Plan des Schlosses vor seinem geistigen Auge vorzustellen. Laut diesem Plan dürfte er irgendwo in der Nähe der Großen Halle sein, dachte er, als er eine Tür sah. Leider hatte er die Karte des Rumtreibers nicht dabei, um auf ihr nachzusehen, wo er war.

So schob er vorsichtig die schwergängige Tür auf. Er entdeckte einen Raum, der ihm vertraut vorkam. Irgendwo hatte er ihn schon einmal gesehen. Stocksteif stand er plötzlich da, als er hinter sich eine bekannte Stimme hörte.

\enquote{Mister Potter.}

Vorsichtig drehte er sich um.

\enquote{Soso. Sie dringen also in meine Privatsphäre ein.}

\enquote{Nein, Professor. Das war keine Absicht. Ich war einfach neugierig}, stammelte er.

\enquote{Wie kommen Sie überhaupt hier herein?}

Der Gang schloss sich bereits und mit ihm das Bücherregal vor ihm, das den Gang mitsamt der Tür verschloss.

Harrys Konzentration war am Ende, so zog Professor Snape seinen Zauberstab und untersuchte Harrys Geist. Aber außer, dass er die Wahrheit gesagt hatte und diesen Gang durch Zufall entdeckt hatte, fand er nichts. \enquote{Ich hatte bisher keine Ahnung}, sagte er, als er seinen Stab wieder einsteckte, \enquote{dass sich hinter diesem Regal ein Gang befindet. Eigentlich hatte ich bisher noch nie das Bedürfnis, irgendetwas aus diesem Regal zu entnehmen.}

Harry machte das stutzig. Er schaute sich die Sachen im Regal genauer an. Könnte ein Schutzzauber darauf liegen, dass man das Regal unbeachtet ließ, wenn man nicht wusste, dass sich dahinter ein Geheimgang befand?

\enquote{Schutzzauber?}, fragte er in den Raum hinein, aber doch mehr zu sich selbst.

\enquote{Könnte sein}, antwortete Snape.

Harry ging an eines der Bücher und zog daran. Es klappte nur heraus und gab den Weg zum Gang dahinter wieder frei.

\enquote{Wie kamen sie darauf?}, fragte Snape.

\enquote{Es war das einzige, das nicht ins Gesamtbild passte}, antwortete Harry.

Snape nickte. \enquote{Wo endet der Gang?}

\enquote{Fünfter Stock. Im Gang, wo die vielen gleichen Bilder hängen. Eines der Bilder ist leicht anders. Das achte oder neunte, wenn man von Westen kommt. Bäume sind darauf abgebildet.}

Snape nickte erneut.

Harry versuchte die Stille zu überbrücken. \enquote{Wie kommen Sie mit meinen Rezepten voran?}, fragte er ablenkend.

\enquote{Sehr gut. Ich arbeite gerade an einem davon. Kommen Sie.}

Harry folgte seinem Zaubertränke-Lehrer an den Kessel mit dem Trank. Unter ihm brannte ein Feuer auf kleiner Flamme und ließ den Trank vor sich hin köcheln. Er sah auf das Pergament davor und danach in den Topf.

\enquote{Schritt vierzehn. Vor der Zugabe der Trient-Wurzel}, folgerte er.

\enquote{Gut erkannt. Es scheint, dass Ihre Kenntnisse in diesem Fach besser werden. Arbeiten Sie weiter daran. Ich muss mir langsam andere Sachen überlegen, wie ich Sie fertig machen kann.}

\enquote{Ich versaue einfach den nächsten Trank und nehme an, das wird er, oder?} Snape nickte. \enquote{Darf ich ihnen helfen?} Erneutes nicken. Harry sah wieder auf das Pergament und holte dann die entsprechenden Zutaten, um sie zu verarbeiten und dann in den Kessel zu geben. Es war ein eigenartiges Gefühl hier mit Snape zu stehen und Hand in Hand mit ihm zu arbeiten. Kein böses Wort, kein Streit, kein Hohn. Einfach nur zusammen an etwas arbeiten. Ab und an korrigiert Snape Harry, aber sonst arbeitete er sehr gewissenhaft, so gewissenhaft, dass sich sein Professor am Ende des Trankes genötigt sah, ihm dafür Punkte zu geben.

\trenn

Eine weitere, \accentuate{dunkle}, Stunde \VgddK hatte gerade begonnen, als Professor Elber anfing zu erzählen. \enquote{Wir werden uns heute mal den Imperius-Zauber vornehmen. Ich würde ihn Ihnen gern praktisch vorführen, um Ihnen die Auswirkungen zu zeigen, aber leider ist er illegal. Es gibt aber eine Reihe weiterer Zauber, die vom Ministerium nicht derart eingestuft wurden. Folglich kann ich Ihnen, sofern Sie sich zur Verfügung stellen, diesen Effekt zeigen. Zuvor aber einige andere Dinge.} Er trat durch den Raum und vor Ron. \enquote{Ich möchte Ihnen den Effekt des Imperius zunächst einmal beschreiben. Dazu werde ich Ihnen ein paar Fragen stellen. Bitte beantworten Sie diese wahrheitsgemäß. Ich verspreche Ihnen, es wird nichts Peinliches geben.} Ron nickte. \enquote{Der Imperius-Zauber hat den Zweck, einem Wesen den eigenen Willen aufzuzwingen. Dies ist aber nicht immer möglich. Es kommt zum einen auf den Charakter, oder genauer gesagt, den Willen der Person an, die unter den Zauber gestellt werden soll, auf den Wunsch der auslösenden Person und auf die Art des Befehls.} Er trat rückwärts und lehnte sich an seinen Schreibtisch. Er sah Ron wieder direkt an. \enquote{Wenn ich Ihnen einen Befehl geben würde, etwas zu kaufen, was Sie eh schon vorgehabt hatten, dann würden Sie keinen Sinn sehen, dagegen anzukämpfen. Also würde der Imperius bei Ihnen wirken. Wenn aber etwas gegen Ihre Überzeugung geht, würden Sie instinktiv versuchen, sich zu wehren und wären eventuell in der Lage, diesen zu brechen.} Ron nickte. \enquote{Ich gebe Ihnen einmal ein Beispiel, antworten Sie bitte ehrlich.} Und wieder kam ein Nicken. \enquote{Wenn ich Ihnen unter Imperius befehlen würde, Ihren Freund Harry zu ermorden, würden Sie das dann tun?}

\enquote{Nein Professor}, antwortete Ron entrüstet.

\enquote{Warum nicht?}

\enquote{Weil er mein Freund ist, ich könnte ihm nichts tun.}

\enquote{Es geht also gegen ihre Überzeugung?}

\enquote{Ja.}

\enquote{Gut}, fuhr Elber fort. \enquote{Gegenbeispiel. Wenn ich Ihnen sagen würde, dass Sie\abs} Er unterbrach sich kurz und schaute durch die Klasse. \enquote{\aabs sagen wir mal Mister Malfoy eine Ohrfeige geben sollen. Hätten Sie da etwas dagegen?}

\enquote{Nein Professor}, antwortete Ron ehrlich.

\enquote{Und wenn ich Ihnen Befehlen würde\abs}, wieder blickte er durch die Klasse, \enquote{\aabs Hermine hier zu küssen?} Er blickte zurück zu Ron und sah darüber hinweg, dass er leicht rosa anlief. \enquote{Ich meine, sie ist hübsch und für einen jungen Mann wie Sie ist es sicherlich nicht abstoßend.}

\enquote{Ich hätte nichts dagegen}, sagte Ron, der sich innerlich gestrafft hatte und daher mit fester Stimme sprach.

Nun war es Hermine, die ganz leicht rosa anlief.

Elber stieß sich vom Tisch ab und lief über das Podest. \enquote{Ich werde Ihnen nun demonstrieren, wie sich das anfühlt, wenn man unter den Imperius gesetzt wird. Es gibt einen Zauber, der dieses Gefühl hervorruft, ohne dass er weitere Auswirkungen hat. Sie werden also keinen Drang verspüren, irgendetwas zu tun.} Er zog seinen Zauberstab und fuhr in einer flüssigen Bewegung halbkreisartig über die Klasse.

Sofort fühlte sich jeder im Raum, als ob seine Gedanken zu verschwimmen begannen. Es war so, als ob einem das Denken abgenommen wurde. Jeder fühlte sich leicht und wie in einem Traum. Eine innere Stimme sagte ihnen, was sie zu tun hatten, doch bevor sie daran denken konnten, senkte Elber seinen Stab und beendete den Zauber.

\enquote{Sie haben jetzt einen Eindruck davon gewonnen, wie der Imperius sich bemerkbar macht. Dies spüren Sie allerdings nur, wenn sich jemand unerfahrenes an Ihrem Gehirn zu schaffen macht. Als Nächstes möchte ich Ihnen demonstrieren\abs}, machte er weiter und die Klasse setzte sich wieder gerade hin, da sie während des Rauschzustandes leicht auf die Sitzflächenkante gerutscht waren, \enquote{\aabs wie es sich anfühlt, wenn man unter den Imperius gesetzt wird. Wieder wird der Zauber nicht der Imperius sein, aber Sie werden einen Drang verspüren. Der Zauber, den ich verwenden werde, stellt wieder nur einen Teilaspekt dar.}

Professor Elber stieg die Treppen zu seinem Büro hoch, holte eine Kiste mit Tomaten heraus und nahm sie mit nach unten. \enquote{Jeder holt sich eine Frucht heraus}, sagte er, als er durch die Reihen lief. Als jeder eine Tomate vor sich liegen hatte, zauberte er noch Stoffservietten herbei. \enquote{Diese werden Sie danach brauchen, glauben Sie mir.} Auf den Gesichtern der Schüler begann sich gerade Staunen zu zeigen, als Elber schon anfing, den Zauber zu wirken.

In jedem der Schüler begann nun der Drang, die Tomate mit bloßen Händen zu zerreißen und essen zu wollen, stärker zu werden. Das wohlige, wolkige Gefühl blieb aber aus. Jeder Gedanke war klar im Kopf fassbar. Nach und nach stürzten sich die Schülerinnen und Schüler über die Tomate her, versuchten sie zu zerreißen und bissen ab, als sie bemerkten, dass das nicht klappte. Gierig bissen sie hinein, und sauten den Tisch vor sich ein. Nur eine Schülerin saß da und sah die Tomate mit wachsendem Interesse an. Dass auch Harry beherzt in die Tomate biss, wunderte Elber. Aber als er Harry grinsen sah, war er der Meinung, dass dieser einfach nur so zubiss, dem fremden Drang aber widerstand, oder es zumindest versuchte, da er seine Feinde so täuschen konnte.

Professor Elber brach den Zauber wieder ab und nach den ersten Schrecksekunden begannen die ersten, sich die Hände mit der Serviette zu putzen. Danach putzten sie die Tische, eventuell vorhandene Flecken und Spritzer aus der Kleidung entfernten sie wortlos mit ihren Stäben.

\enquote{Sehr gut, Sie haben nun einen Eindruck davon bekommen, wie es ist, unter fremdem Einfluss zu stehen.} Er wandte seinen Blick zu seiner Schülerin. \enquote{Sie haben sich als einzige gar nicht über die Tomate hergemacht. Wie kommt das?}, fragte er interessiert schauend.

Sie antwortete ihm: \enquote{Ich hasse Tomaten. Ich kann diese Früchte nicht ausstehen. Ihr Geschmack ist einfach eklig. Jedes Mal, wenn ich eine essen musste, konnte ich meinen Speisebrei nicht halten und musste über dem Esszimmertisch brechen}, sagte sie völlig normal.

\enquote{Gut, machen wir weiter. Dieses Mal sage ich Ihnen nicht, was\abs} Er schwang seinen Stab erneut. \enquote{\aabs dran kommt.}

Es dauerte keine zwei Sekunden, als die Schülerin aufstand und begann sich auszuziehen. Sie stand gerade in Unterwäsche da, als weitere Schüler aufstanden und begannen sich jetzt ebenfalls auszuziehen. Gerade als Astoria hinter sich griff und den Verschluss ihres BHs zu öffnen, sah Elber sie an und schnippte einmal mit den Fingern, den Blick auf sie fixiert. Astoria öffnete noch den Verschluss, als sie bemerkte, dass es nicht das war, was sie wollte. Schnell schloss sie ihn wieder. Leicht beschämt blickte sie ihren Professor an. Dieser gab ihr nur wortlos zu verstehen, sie möge sich setzen. Die stumme Frage, ob sie ihre Kleidung wieder anziehen dürfe, verneinte er mit leichtem Kopfschütteln. Jeden seiner Schüler, der so weit war, zu viel zu zeigen, fixierte er mit seinen Augen und schnippte danach einmal mit den Fingern. Als jeder Schüler und jede Schülerin nur noch in Unterwäsche da saß, brach er den Rest des Zaubers.

\enquote{Nun wissen auch Sie, wie es ist, unter diesem Einfluss zu stehen. Bei Ihnen waren die beiden Gefühle leider vertauscht. Scheinbar macht es Ihnen weniger aus, mehr Haut von Ihnen zu zeigen, als eine Tomate zu essen. \gst Das ist nichts, weswegen Sie sich zu schämen brauchen.} Das Mädchen, das bereits Schamgefühle zu spüren glaubte, war erleichtert. \enquote{Wenn es Ihnen nichts ausmacht, sich leicht bekleidet zu zeigen, heißt das nicht, dass sie sich gleich gegenüber jeder Person ausziehen würden. \gst Mir persönlich, würde es absolut nichts ausmachen, Professor McGonagall zu küssen, ich werde es aber nicht tun.} Das lockerte die mittlerweile leicht angespannt Stimmung wieder auf und die Schüler saßen jetzt entspannt auf ihren Plätzen.

Professor Elber drehte sich herum und sah kurz gehen die Wand, die sonst immer hinter ihm war. Als er wieder zur Klasse blickte, meinte er: \enquote{Ich habe noch ein letztes Experiment vor. Lassen Sie es mich Ihnen zuerst erklären, damit Sie es verstehen, was ich Ihnen damit zeigen möchte und Sie begreifen, warum ich dies von Ihnen \gst es ist keine Forderung, es ist vielmehr \gst wie soll ich mich ausdrücken \gst ich erwarte von Ihnen, dass sie Ihre Grenzen kennen \gst ich möchte Ihnen Ihre Grenze bezüglich des Imperius-Zaubers zeigen, damit Sie wissen, woran Sie arbeiten können. Im Buch zu diesem Fach wird am Ende der Stunde ein weiteres Kapitel zu sehen sein. Dies arbeiten Sie bitte für sich durch und ich erwarte am Ende des Schuljahres eine Steigerung Ihrer Grenze. Leider kann ich Ihnen dazu keine Anleitung geben. Üben Sie untereinander und finden Sie heraus, was Ihnen am besten hilft. Den entsprechenden Spruch finden Sie auch im Buch.} Er ließ der Klasse etwas Zeit, damit diese ihrem Professor folgen konnten. \enquote{Damit ist auch ein gewisses Opfer verbunden. Ich glaube aber, dass ich Sie gut genug kenne und Sie mich, dass Sie verstehen, warum ich das tun werde. Ich selber werde dabei nicht im Raum sein. Der Zauber ist so ausgelegt, dass er selbstständig abbricht, wenn Sie einen bestimmten Punkt erreicht haben. Dann wissen Sie auch gleichzeitig ihre Grenze.} Er verschwieg der Klasse allerdings ein paar wichtige Details, welche den Schülerinnen und Schülern erst später klar werden würden.

Nach einer Weile nickte die Klasse. Mit beiden Händen, die er mit den Handinnenflächen nach oben hob, stand die Klasse auf. Die Tische schwebten an die Decke und die Kleider der Schüler ordneten sich sauber hinter ihnen an. Der Unterricht würde in einer viertel Stunde beendet sein, das war jedem klar.

\enquote{Stellen Sie sich in einem Kreis auf. Abwechselnd Junge und Mädchen.} Als die Klasse kreisförmig im Raum stand, sagte er: \enquote{Drehen Sie sich um.} Jeder Schüler drehte sich und schaute nun vom Kreismittelpunkt weg. Elber ging durch den Raum zur Tür des Klassenzimmers. \enquote{Ich werde draußen warten und Ihnen die Anweisungen per Patronus übermitteln. Ich selber werde dem Rest nicht beiwohnen. Testen Sie Ihre Grenzen aus}, sagte er und verließ das Zimmer.

Wenige Sekunden später kamen viele kleine fliegende Insekten herein und verteilten sich im Raum. Vor jedem Schüler ein Insekt. Obwohl alle gleichzeitig sprachen, hörte jeder nur das Insekt vor ihm, mit der Stimme des Professors sprechen.

\enquote{Bitte entkleidet euch komplett, verdeckt eure Scham und dreht euch um. Setzt euch danach im Schneidersitz auf den weichen und warmen Boden und kämpft gegen den aufsteigenden Drang an.}

Zuerst standen die Schüler in einer Art Schockstarre da. Doch nach einer knappen halben Minute begann der erste sich zu entkleiden. Die anderen folgten ihm. Einige zuerst zögerlich, dann aber doch zunehmen schneller, da ihnen bewusst wurde, dass sie zu viel von sich denen gegenüber preisgeben würden, die schon saßen und sie daher beobachten konnten, denn zum Ausziehen der Unterwäsche musste man sich bücken, oder nacheinander beide Beine heben. Als alle, mit den Händen vor der primären Scham, auf dem Boden saßen und sich anblickten, bemerkten die Mädchen, dass es ihnen nichts ausmachte, den Jungs ihre Brüste zu zeigen.

Langsam begann der Drang, seine Hände wegzunehmen und an den Seiten des Körpers abzulegen, immer stärker zu werden. In gleichem Maße nahm der Drang, den Schneidersitz aufzulösen und die Beine auszustrecken, zu. Es dauerte knappe sieben Minuten, bis die erste Schülerin dem Drang nachgab, ihre Hände von ihrer Scham nahm und sich mit ausgestreckten Beinen und abgestützten Beinen leicht nach hinten lehnte.

%In der Geschwindigkeit, in der der Drang nachließ, kehrte die Erkenntnis ein, dass die anderen sie so nicht sahen. Die anderen sahen sie immer noch, wie sie im Schneidersitz und die Hände vor ihrer Scham dasaß. Sie konnte also nicht feststellen, wer bereits dem Drang erlegen war.
Nun wusste sie, dass sie sich anziehen und den Raum verlassen konnte.

Draußen traf sie auf ihren Lehrer. Langsam machte sich in ihr die Erkenntnis breit, dass sie sich bald nicht mehr an die genauen Details der erspähten Geschlechtsorgane der anderen erinnern würde.

\enquote{Warum wollten Sie nicht bei uns im Raum sein, Professor?}, fragte sie ihn.

\enquote{Auf den, der den Zauber ausspricht, hat er nicht dieselbe Wirkung, wie auf Sie}, antwortete er einfach.

Nach und nach kamen immer mehr Schüler aus dem Raum heraus, bis fünf Minuten vor dem Ende der Stunde der letzte Schüler den Raum verlassen hatte.

\enquote{Üben Sie kräftig}, sagte der Professor und verabschiedete sich von seinen Schülern.

\enquote{Das war eine spannende Stunde}, meinte Lavender. \enquote{Und wenn ich an Rons Gemächt denke\abs}, meinte sie, doch die Erinnerung daran wie es aussah, verblasste bereits. Nur nicht, dass sie es gesehen hatte.

Keiner der Schüler bemerkte wie der Professor mit einem dicken Grinsen im Gesicht durch das Schloss ging.

\enquote{Was haben Sie?}, fragte Professor Flitwick.

\enquote{Ich habe heute meine spezielle Stunde gehalten, über die wir geredet hatten.}

\enquote{Oh}, meinte Flitwick und grinste dann genauso breit. \enquote{Meinen Sie, die Schüler können Ihre Erinnerungen wieder zurückholen?}

\enquote{Ich habe Sie mit dem Vergessenszauber nicht daran gehindert. Wenn es eine oder einer bemerkt, dann kann sie oder er sich glücklich schätzen. \gst Wissen Sie, Filius, das ist es, was die Magie so spannend macht und mir die Stelle als Lehrer erträglich.}

Nun grinste Filius noch mehr. \enquote{Was werden die Schüler noch herausfinden?}

\enquote{Wenn sie gut sind, werden sie sich an jeden nackt erinnern können, hoffe ich mal.}

\enquote{Und Sie?}

\enquote{Ich war draußen. Ich habe schon genug damit zu kämpfen, die Magie um meine Schüler herum zu ignorieren.}

\enquote{Ach ja, das Phänomen. \gst Und bei Aurora?}

\enquote{Da konnte ich mich ein paar mal nicht konzentrieren.}

\enquote{Das heißt?}

\enquote{Nackt und in Falschfarben.}

\enquote{Und in Ihren Träumen?}

\enquote{Werden diese manchmal aufgehoben. \gst Ich brauche jedes Mal am Morgen danach eine viertel Stunde, um mit Okklumentik diese Bilder zu verdrängen. Sie würden mich sonst von meiner Arbeit abhalten.}

\enquote{Kann man das eigentlich lernen?}

Elber sah Flitwick an. \enquote{Ich würde es liebend gern abschalten können. Aber wenn Sie mich schon fragen\abs} Er berührte ihn kurz an seinem Kopf und sagte etwas auf Hebräisch. \enquote{Viel Spaß die nächsten vierundzwanzig Stunden}, meinte er noch, bevor er abbog.

Filius strich sich sauer über seinen Kopf, da es unangenehm war, wenn jemand über den eigenen Kopf strich und man das nicht erwartet hatte. Er verstand nicht, warum Elber das gemacht hatte, doch keine fünf Minuten später kam ihm die Erkenntnis, als er in das Lehrerzimmer eintrat.

Als er Elber nach Ablauf der Zeit in seinem Zimmer im Schloss besuchte, meinte er nur: \enquote{Jetzt weiß ich, warum Sie das als Fluch sehen. Die letzten drei Stunden hatte zwar ein Gewöhnungseffekt eingesetzt, aber dieser wird immer wieder aufgehoben oder abgeschwächt. Es ist einfach lästig.}




\begin{kommentar}
Vor über tausend Jahren gebar Friedwards Frau Persope Vierlinge, die sie Rowena, Salazar, Godric und Helga nannte. Damit ist schon mal alles gesagt. Und wenn man den ganzen kleinen Hinweisen folgt, wird man spätestens in den ersten Kapiteln des nächsten Teiles direkt mit der Nase darauf gestoßen, dass Friedward Frederick ist. Darum kennt er sich im Schloss auch so gut aus.
\end{kommentar}

\begin{kommentar}
Wieder zurück in der Gegenwart wachsen an Petunias Geburtstag Petunien. Jene Blumen, die Harry gepflanzt hatte. Das brachte Petunia zum Nachdenken. »Ich sollte mal wieder …« Der Satz, den sie nie vollendete, endet wie folgt: »Lilys Grab besuchen.« Warum, das wird klar, wenn sich Harry Snapes Erinnerungen ansieht, die er im nächsten Teil bekommt.
\end{kommentar}

\begin{kommentar}
Dann wacht Harry in der Krankenstation auf und fragte sich, warum er von Hexen und Zauberern mit ihren Zauberstäben bedroht wird. Aber schnell wird ihm klar, dass er mit Voldemort den Körper getauscht hat. Um zu wissen, wie es weitergeht, meint McGonagall: »Schlag du was vor.« Ein kleiner Satz, den einer der Geier aus dem Dschungelbuch gesagt hatte, als diese nicht wussten, was sie tun sollten.
\end{kommentar}

\begin{kommentar}
Nachdem Harry bei dem Entfernen der Baumstümpfe betrogen hatte, musste er die Pokale von Hogwarts polieren. Eine Arbeit, die Ron in den Büchern auch schon machen musste.
\end{kommentar}

\begin{kommentar}
Etwas später führt Elber einen Zauber aus, der den Effekt des Imperius-Zaubers nachahmen soll. Erst später wird Harry erfahren, wie man Zauber selber erstellt. Was Eber also hier macht, ist einen selber erstellten Zauber anzuwenden und diese Technik erst viel später anderen beizubringen.
\end{kommentar}
