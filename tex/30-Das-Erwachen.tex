\chapter{Das Erwachen}


Harry erwachte auf der Krankenstation und sah in Ginnys Augen. \enquote{Was ist los?}, fragte er sie, als er merkte, dass er auf der Krankenstation lag. Er fühlte sich von ihr angezogen.

\enquote{Du bist auf der Krankenstation}, sagte Ginny und lächelte ihn an. Dieses Lächeln war eindeutig nicht schwesterlich. Es war mehr als das.

\enquote{Was ist passiert?}, wollte Harry wissen.

\enquote{Du bist wieder einmal zusammengebrochen. Du lagst seit drei Wochen bewusstlos im Bett.} Und bevor Harry noch etwas sagen konnte. \enquote{Luna geht es gut. Sie ist bereits nach drei Tagen wieder aufgewacht. Wir waren schon in Sorge, dass ihr wieder eure Körper getauscht habt. Jetzt nachdem alles in Ordnung ist zwischen euch.}

\enquote{Seit drei Wochen}, sagte Harry. \enquote{Dann habe ich nur geträumt. Pansy und ich \gst}

\enquote{Hallo Harry}, hörte er plötzlich vom Nebenbett.

Harry erschrak. Es durchfuhr ihn wie ein Blitz. \gedanke{Pansy.} Er drehte seinen Kopf in die Richtung, aus der die Stimme kam, und sah in Pansys schwarze Augen.

Sie lag im Bett neben ihm. Ihr Haar, welches sie normalerweise zu einem Pferdeschwanz zusammengebunden hatte, lag nun offen auf dem Kissen.

\enquote{Hallo Pansy}, gab er zurück. Erst dann realisierte er, dass in seinen Worten kein Hass, keine Ablehnung war.

\enquote{Bist du wieder wach?}, fragte sie ihn. \enquote{Erinnerst du dich?}

Harry war verwirrt. \enquote{Woran sollte ich mich erinnern?}, fragte Harry.

\enquote{An deine Rede in der Großen Halle, in der du allen Leuten sagtest, dass du mich, deine Freundin, lieben würdest und unsere Trennung kurz danach}, sagte Pansy.

Harry riss die Augen auf. \enquote{Das war kein Traum?}, fragte er ungläubig. Er fragte mehr sich selbst, als die anderen. Pansy schüttelte nur den Kopf. Harry ließ seinen Kopf sinken und drehte ihn so, dass er die Decke betrachtete. Ausdruckslos starrte er sie an. Er schloss die Augen und traute sich erst dann zu fragen: \enquote{Dann haben wir zusammen \gst}

\enquote{Hm?}, fragte sie zurück. \enquote{Was haben wir zusammen?}

Harry atmete tief ein. \enquote{Dann haben wir uns in der Großen Halle vor allen Leuten geküsst? Haben vor Draco und Ron geknutscht.}

Sie gab nur ein einfaches \enquote{Ja}, als Antwort.

Harry wurde rot.

\enquote{Ich bin mir sicher, ihr habt noch eine Menge zu besprechen. Ich muss jetzt los zur nächsten Stunde. Bis später Schatz}, sagte Ginny und küsste Harry flüchtig auf den Mund. Eigentlich wollte sie seine Backe erwischen, aber Harry drehte sich gedankenverloren zu ihr und so traf sie seinen Mund. Erschrocken zog sie sich zurück, aber nur so weit, dass sie ihn nicht mehr berührte. Harry musste nur eine Schnute ziehen, um ihre Lippen wieder zu spüren. Er hob seinen Kopf kurz an und küsste sie erneut. Ginny gab nach und als sie sich von ihm löste, wurde sie rot im Gesicht.

\enquote{Wenn schon, dann richtig, Ginny}, war alles, was er herausbrachte. Sie lachte und verschwand aus der Krankenstation. Er wusste nicht mehr, seit wann sie wirklich zusammen waren und ob sie sauer auf Pansy oder ihn war, als sie sich öffentlich küssten. Er wusste nur, er war mit Ginny glücklich. Aber war er das wirklich? Mit Ginny zusammen? Seit wann? Und warum konnte er sich daran nicht mehr erinnern?

Nun waren Harry und Pansy ganz alleine. Sie warf ihre Bettdecke zurück und schlüpfte in ihre Pantoffeln. Ihr Nachthemd war hellgelb mit leichten kaum merkbaren weißen Verzierungen. Sie lief zu ihm hin und setze sich auf seine Bettkante.

\enquote{Harry}, fing sie an. \enquote{Ich glaube nicht, dass ich noch etwas für dich empfinde}, sagte sie.

Harry hatte das Gefühl, sie würde es so taktvoll sagen, wie sie auch konnte, ohne seine Gefühle zu verletzen, falls er doch noch etwas für sie empfinden sollte. \enquote{Mach dir darüber keine Sorgen, Pansy. Ich habe auch kein Verlangen mit dir \gst intim zu werden.}

Sie warf ihren Kopf zurück und lachte aus vollem Herzen. Harry konnte ihren Hals sehen. Sie sah wunderhübsch aus.

\enquote{Weißt du, dass wir diese Unterhaltung schon ein paar Mal geführt haben? Immer dann, wenn du kurz aufgewacht bist!}

\enquote{Nein. Ich kann mich an die letzten Wochen kaum erinnern. Ich glaube zu wissen, dass ich mit Ginny zusammen bin. Und wir zwei haben uns getrennt, richtig?}

Pansy nickte. \enquote{Wir haben uns kurz nach unserem Outing getrennt. Am nächsten Tag warst du mit Ginny zusammen.} Harry hörte aufmerksam zu. \enquote{Aber}, sie stupst ihn auf die Brust, \enquote{dass du mit Lavender geschlafen hast, während wir noch zusammen waren, nehme ich dir übel.}

\enquote{Ich habe dich auch noch lieb}, sagte Harry. \enquote{Willst du ein Bussi? Dann ist wieder alles gut zwischen uns.}

Die Flügeltüren der Krankenstation gingen auf und Draco Malfoy trat ein. Nachdem er sie sorgsam hinter sich verschlossen hatte und sichergestellt hatte, dass sie alleine waren, meinte er nur: \enquote{Guten Morgen Pansy, Harry.}

\enquote{Guten Morgen, Draco}, hörte sich Harry sagen. Es war eigenartig, dass ihn Draco Malfoy mit seinem Vornamen ansprach. Außer im Gemeinschaftsraum der Paare war er für ihn immer nur Potter gewesen. Er blickte zwischen Pansy und Draco hin und her. Draco setzte sich ebenfalls auf Harrys Bettkante neben Pansy und nahm ihre Hand in seine. Harry warf einen Blick darauf. \enquote{Wie hat es Maria aufgefasst?}, fragte er knapp.

\enquote{Sie hat mit mir Schluss gemacht, nachdem ich} und er stockte kurz, \enquote{Pansy mehrmals auf der Krankenstation getroffen hatte.}

Harry zog eine Augenbraue hoch.

\enquote{Nachdem sie gesehen hatte, wie Draco mich küsst}, vollende Pansy den Satz.

Harrys Augen weiteten sich ein weiteres Mal.

Draco fügte hinzu: \enquote{Pansy kam zwei Tage nach dir hier an. Ich habe sie seitdem jeden Tag besucht und vor eineinhalb Wochen oder so, da hat es gefunkt}

\enquote{Neros?}, fragte er Draco.

\enquote{Morgen, dachte ich mir}, sagte er.

Pansy sah ihn fragend an. \enquote{Was Draco?}

Die Tür öffnete sich und Madame Pomfrey kam herein. \enquote{Ah, Mister Potter. Wieder wach?}, fragte sie.

\enquote{Ja}, antwortete Harry. \enquote{Aber warum wieder?}

\enquote{Sie sind schon ein paar Mal aufgewacht und sind nach einigen Minuten wieder bewusstlos geworden.}

Leichte Panik stieg in Harry auf. \enquote{Wie kurz danach}, fragte er nach.

\enquote{So ca. eine viertel Stunde}, antwortete Madame Pomfrey.

Sie trat an sein Bett und Draco verabschiedete sich. \enquote{Muss noch Hausaufgaben machen. Und ihr beide klärt das}, sagte er, küsste Pansy auf den Mund und verschwand.

Madame Pomfrey untersuchte Harry und holte ihm danach ein Stärkungsmittel, welches er austrinken musste. Dann verschwand sie in ihrem Büro. Pansy saß noch immer auf seiner Bettkante und lächelte ihn an. Die Tür zu Madame Pomfreys Büro öffnete sich noch einmal und ihr Kopf schaute heraus. \enquote{Miss Parkinson?} Pansy drehte sich herum. \enquote{Sie können die Krankenstation verlassen.} Der Kopf zog sich zurück und die Tür wurde wieder verschlossen.

Pansy stand auf und fing an die Tasche zu öffnen, welche neben ihrem Bett stand. Sie holte eine Schulrobe, ein Hemd mit Krawatte, eine Hose, einen Schlüpfer und einen BH heraus. Dann zog sie ihr Nachthemd über den Kopf und begann sich anzuziehen.

\enquote{Äh Pansy?}, fragte Harry sie.

\enquote{Ja?}, antwortete sie ohne sich umzudrehen.

\enquote{Was sollen wir klären? Nicht dass es mir etwas ausmachen würde, aber hast du nicht etwas vergessen?}

Sie verschloss gerade ihren BH hinter ihrem Rücken und drehte sich dabei um. Sie hatte nichts als ihre Unterwäsche an. \enquote{Draco glaubt, wir sind noch zusammen. \gst Was sollte ich vergessen haben?}, fragte sie mit leicht unsicherem Gesichtsausdruck.

\enquote{Sag ihm, dass du mich die letzten eineinhalb Wochen mit ihm betrogen hast, das wird ihm gefallen und heute haben wir uns offiziell getrennt. \gst Die Vorhänge zurückzuziehen?}, meinte Harry.

Sie lachte nur. \enquote{Das stimmt, Draco wird das freuen. \gst Harry, wir haben uns nackt gesehen. Deine Nase war zwischen meinen \gst Nun ja. Ich verspüre keine Scham dir gegenüber, falls du das meinst.} Sie zog sich ihre Hose an und danach ihr Hemd samt Krawatte.

\enquote{Wie lange bin ich denn mit Ginny zusammen?}, fragte er sie.

\enquote{Etwa vier Wochen, wenn ich dein Koma mit einbeziehe.}

\gedanke{Vier Wochen}, dachte Harry nach. \enquote{Ich kann mich gar nicht daran erinnern. Ich muss Ginny fragen, wie\abs}

\enquote{Das wird ihr gar nicht gefallen. Liebst du sie überhaupt?}

\enquote{Ja. Schon länger. Ich war nur zu feige, es mir und vor allem ihr einzugestehen.}

\enquote{Dann hast du Ginny bereits betrogen, als du mit mir\abs}

Harry wurde rot. \enquote{Ich weiß nicht.}

Als sie gerade ihre Schulrobe aufnahm, öffnete sich wieder die Tür zum Krankenzimmer und Professor Dumbledore kam herein.

\enquote{Ah, Miss Parkinson, Harry}, begrüßte der Schulleiter die beiden. \enquote{Sie dürfen schon gehen, Miss Parkinson?}, fragte er Pansy.

\enquote{Ja Professor. Madame Pomfrey hatte es mir erlaubt}, sagte sie und zog sich ihre Schulrobe an.

Professor Dumbledore setze sich in der Zwischenzeit auf einen Stuhl neben Harrys Bett und sah Pansy gedankenverloren zu, wie sie ihre Schulrobe zuknöpfte.

Dann sah ihm Pansy in die Augen und blickte danach zu Harry. Sie kam auf ihn zu und gab ihm einen Kuss auf die Wange, nahe seinem Mundwinkel. \enquote{Mach’s gut, Harry. Und eine gute Genesung. Ich sage den anderen Slytherin Bescheid, dass du wach bist.} Dann sagte sie ihm leise ins Ohr: \enquote{Ich würde aber jederzeit wieder mit dir schlafen, falls du so niedergeschlagen und wir beide frei von Partnern sind.} Dann verließ sie den Krankenflügel.

Harry fragte sich, ob ihr das wieder herausgerutscht war. Mit rotem Kopf fing Harry Dumbledores Blick auf. Er konnte ihm nicht lange in die Augen schauen. Nicht nach dem, was gerade passiert war. Also legte er seinen Kopf wieder ins Kissen und entschloss sich, die Decke als passende Alternative anzunehmen. \gedanke{Hat sie das jetzt ernst gemeint? Oder war das Spaß? Sind das Nachwirkungen?} Dumbledore saß nur da und wartete. Harry würde sich schon wieder beruhigen. \enquote{Ist mit mir jetzt wieder alles in Ordnung?}, fragte Harry seinen Schulleiter.

\enquote{Ich nehme es an}, antwortete Dumbledore. \enquote{Es gibt keinerlei Interferenzen mehr und deine sexuelle Anziehungskraft auf die weibliche Bevölkerung ist auch nicht mehr vorhanden.}

\gedanke{Sexuelle Anziehungskraft}, dachte Harry erneut. Er entschloss sich, dem Schulleiter etwas zu sagen. \enquote{Albus}, fing er an.

\enquote{Ja, Harry}, antwortete Dumbledore.

\enquote{Als du mich zu Madame Pomfrey schicktest, um dieses Mittel abzuholen\abs}

Dumbledore antwortete: \enquote{Prophylaxis Reproducta finite.}

\enquote{Als du mich dorthin schicktest und Madame Pomfrey mir diesen Trank braute, sagte sie zu mir, dass es auch an ihnen nicht spurlos vorbeigeht.} Er sah nun Dumbledore direkt in die Augen. \enquote{Es muss für Professor McGonagall, Madame Pomfrey und all die anderen weiblichen Lehrer eine Qual gewesen sein, nicht über mich herzufallen}, sagte Harry.

Es hatte den Anschein, als ob Harry ins Schwarze getroffen hatte. Professor Dumbledore schien leicht verunsichert zu sein. \enquote{Na ja}, druckste er herum. \enquote{Es gab für die Lehrer \gst aber das darfst du niemandem erzählen}, Harry nickte, \enquote{einen Trank, um das Verlangen etwas zu lindern. Aber ich gebe dir recht. Es war nicht einfach für Professor McGonagall}, grinste Dumbledore.

\enquote{Albus?}, fragte Harry.

\enquote{Hmm}, gab Dumbledore zurück.

Harry schluckte. \enquote{Als ich bei Madame Pomfrey war um dieses Mittel zu holen und sie mir gestand, dass es auch an ihnen nicht spurlos vorbeiging, da kam mir der Gedanke, sie einfach zu küssen. Ich stellte mir vor, wie verlegen sie danach sein musste, oder nicht mehr von mir ablassen konnte und sich auf mich warf.} Er pausierte kurz. \enquote{Ich habe mich beherrscht, Albus. Ich habe es nicht getan, trotz des Reizes.}

Jetzt fing Dumbledore herzhaft zu lachen an. Er lachte so sehr, dass Madame Pomfrey die Tür ihres Büros öffnete, um zu sehen, was dort los sei. Harry merkte das und fing nun ebenfalls an zu lachen. Sie konnten sich einfach nicht mehr beherrschen.

\enquote{Sie scheinen sich ja gut zu amüsieren}, rief Madame Pomfrey aus ihrem Büro heraus. \enquote{Sie dürfen gehen, Mister Potter. Da es Ihnen scheinbar schon so gut geht.}

Harry und Dumbledore lachten noch eine Weile weiter, bevor Dumbledore sich zurückzog, damit Harry sich umziehen konnte. Zusammen gingen sie den Weg zur Großen Halle, um zu Abend zu essen. \gedanke{Das tat gut}, dachte Harry. \gedanke{Dumbledore so Lachen zu sehen.} Auf dem Weg dorthin sagte er noch: \enquote{Was würde eigentlich passieren\abs wäre eigentlich passiert, wenn ich mit einer Lehrerin?}

\enquote{Unter diesen Umständen gar nichts, wieso?}

\enquote{Ach, nur so ein Gedanke. Unter den Muggeln gibt es so etwas wie Schutzbefohlene. Da ist es verboten in so etwas wie eine Abhängigkeitssituation zu kommen. Welche Seite in meinem Falle allerdings die Abhängige wäre, müsste man klären.}

Jetzt wusste er, dass die kleine Schmuserei mit Professor Sinistra keine Folgen haben würde. Er setzte sich an einen freien Platz und begann zu essen. Seine Gedanken schweiften zurück und er musste schmunzeln.

\begin{rueckblick}
Harry bog um eine Ecke und stieß mit Sinistra zusammen. Reflexartig griff er zu und hatte sie um ihre Hüfte gepackt. Das machte es ihr sehr schwer, ihm zu widerstehen. Leider wurde durch seinen Zustand sein Verstand zurückgefahren und ihm entwich ein: \enquote{Aurora, tut mir leid.} Dies löste in ihr eine Aktion aus, die Harry vermeiden wollte. Sie zog ihn zu sich und küsste ihn. Mitten in der kleinen Knutscherei bekamen beide ihren vollen Verstand wieder zurück.

\enquote{Kein Wort zu niemand?}, fragte Harry.

\enquote{Richtig, Mister Potter.}

Beide ließen einander los und sammelten ihre Sachen vom Boden auf. Danach gingen sie ihre Wege und verhielten sich so, als ob das nie passiert wäre. Keine roten Backen oder schamhaftes Verhalten während der nächsten Zusammentreffen oder Unterrichtsstunden. Beide konnten dieses Ereignis gut verbergen.
\end{rueckblick}

\onelineback % Anderenfalls werden 2 Leerzeilen gesetzt
\trenn

Mitten in Snapes Unterricht brach Harry wieder zusammen. Das war bereits das dritte Mal, seit er in Dumbledores Büro umgefallen war. Er hatte seine Hausaufgaben abgegeben und war mitten im Brauen eines Trankes. Heute ging es darum, Schrumpfköpfe herzustellen. \gedanke{Ein eigenartiger Trank} dachte sich Harry. Doch viel bekam er nicht mehr mit, denn ihm wurde schwarz vor Augen. Er konnte sich nicht artikulieren und fühlte Nevilles Arm hinter seinem Rücken, als er zu Boden sackte. Dieses Mal konnte er nichts mehr hören, nur fühlen. Er fühlte, wie er auf den Boden gelegt wurde und danach schwebte. Er bekam jede Lageveränderung mit, bis er schließlich nichts mehr fühlte und wegdämmerte.

Neville kümmerte sich weiter um ihren gemeinsamen Trank, während Harry einfach, aber gut gepolstert, in einer Ecke liegen blieb. Madame Pomfrey konnte ihm nicht helfen, da sich die Magie mit ihm verbinden musste. Ein natürlicher Prozess.

Als er wieder erwachte, war die Stunde schon fast um. Snape bewertete Harrys Arbeit gar nicht und die von Neville konnte er nur schwer einschätzen. Aber da der Trank gut geworden war, vermied er es, sich mal wieder auszulassen. Er gab einfach keine Punkte.

Nach dem Ende der Stunde ging es Richtung See.

Professor Elber stand bereits mit den Füßen im Wasser als die Klasse ankam. Er hatte eine kurze Hose und ein kurzärmeliges Hemd an. Die gesamte Klasse baute sich am Rande des Sees auf und wartete gespannt, denn ihr Erscheinen musste ihm aufgefallen sein.

Die Wasseroberfläche bewegte sich und ein Meereslebewesen tauchte auf. Es war wieder Chwalla. Professor Elber begrüßte sie auf Meerisch, was Harry ohne Probleme verstand. \meerisch{Ich grüße euch, junge Königliche Hoheit.}

\meerisch{Ganz meinerseits}, gab sie zurück.

\meerisch{Wir sind vollständig anwesend, denke ich. Sie können sie einweisen}, redete Professor Elber weiter.

\meerisch{Gut, danke Professor.} Sie wandte sich der Gruppe von Schülern zu und begann in deren Sprache zu reden. \enquote{Ich freue mich, dass Sie sich heute einer großen Aufgabe widmen werden. Ich werde sie Ihnen erklären. Nachdem Sie Ihre Badesachen angezogen haben, werden Sie in kleinen Gruppen zu vier oder fünf versuchen in eine unserer bewachten Einrichtungen zu gelangen und den dort befindlichen Gegenstand zu entführen. Die Wachen wissen, dass sie in Kürze von Hexen und Zauberern besucht werden, die versuchen werden sie auszurauben. Für diese Aufgabe haben Sie vierzig Minuten Zeit. Zusätzlich fallen je zehn Minuten Schwimmweg an.}

Die Wasseroberfläche bewegte sich wieder und mehrere Meereslebewesen erschienen an der Wasseroberfläche. In ihren Händen hielten sie Netze, welche grüne algenartige Wasserpflanzen enthielten. Harry dachte, Dianthuskraut zu erkennen. \gedanke{Das konnte was werden, schon wieder Dianthuskraut.} Er hatte die letzten beiden Male noch gut im Gedächtnis. Besonders angenehm war das nicht.

\enquote{Worauf warten Sie? Ziehen Sie sich endlich um}, ermahnte Professor Elber die Klasse.

Die Meereslebewesen mit den Netzen kamen dem Ufer näher.

\enquote{Und wo sollen wir uns umziehen?}, fragte Susan Bones.

Währenddessen knöpfte Luna bereits ihre Schulrobe auf. Darunter sah man eine normale schwarze Hose, eine weiße Bluse und die übliche Ravenclaw-Krawatte. Sie löste den Krawattenknoten und zog sie ab. Dann nahm sie ihren Zauberstab und zauberte um sich herum eine kleine Bretterkabine, in der sie sich umzog.

\enquote{Nehmen Sie sich ein Beispiel an Ihrer Schulkameradin, sie hat es ziemlich schnell begriffen.}

Die Meereslebewesen mit den Netzen waren nun am Ufer angekommen und standen auf. Sie standen noch mit den Füßen im Wasser. Eine seltsame blaue Wasserblase befand sich um ihren Hals herum und lief mit einem dicken Schlauch den Rücken entlang herunter bis zur Wasseroberfläche. Damit mussten sie wohl atmen, denn es dauerte, bis alle fertig waren. Chwalla tauchte mehrmals kurz unter. Harry hatte bereits heute Morgen seine Badehose angezogen und konnte daher ohne Umkleide seine Schulrobe ausziehen. Professor Elber zog unterdessen seinen Zauberstab heraus und verzauberte eine kleine Einstiegsschneise in den See hinein. Wasserlianen kamen an die Oberfläche und bildeten so praktische Stufen. Die Wasseroberfläche färbte sich sehr leicht ins Rötliche.

Die Pflanzenträger öffneten ihre Netze und legten kleine Portionen von Dianthuskraut auf die flachen Steine, welche aus dem Wasser ragten. Eine gaben sie dem Professor. Danach verschwanden sie mit den leeren Netzen wieder unter Wasser. Es dauerte noch wenige Minuten, bis sich die Klasse restlos umgezogen hatte. Die Umkleiden verschwanden nacheinander wieder und alle standen in Badesachen da.

Professor Elber hob die Hand, in der er das Dianthuskraut hielt und fing an. \enquote{Sie sehen hier in meiner Hand, und ebenso auf den flachen Steinen dort, Dianthuskraut. Sobald Sie diese Pflanze geschluckt haben, sind Sie in der Lage eine Stunde lang unter Wasser zu atmen. Ihre Aufgabe ist klar. Schwimmen Sie zu den Ihnen zugewiesenen Stellen, die Ihnen Chwalla zeigen wird.} Chwalla schwamm näher an das Ufer heran. \enquote{Spätestens in einer Stunde müssen Sie zurück sein.}

\meerisch{Haben Sie Ihre Gruppen schon gebildet?} Die andern hielten sich ihre Ohren zu, denn Meerisch über Wasser war nicht zu ertragen. \enquote{Oh, Verzeichnung. Ich meinte natürlich: Haben Sie Ihre Gruppen schon gebildet?}, fragte sie.

\enquote{Ja}, antwortete die Klasse.

\enquote{Gut, dann finden Sie sich bitte zusammen, damit der Professor Ihnen entsprechend farbige Bändchen geben kann.}

Professor Elber zog wieder seinen Zauberstab und wartete, bis sich die Gruppen deutlich herauskristallisiert hatten. Er schwang seinen Zauberstab und die einzelnen Gruppen färbten sich.

\enquote{Was ist mit meiner Haut passiert?}, kamen schreie aus der Klasse, denn es hatten sich keine Bändchen an den Handgelenken gebildet. Die Hautfarbe der Gruppen hatte sich ins Blaue, ins Grüne oder anderen Farben verwandelt.

\enquote{Ups}, gab Professor Elber mit leichtem Lächeln zurück. \enquote{Bändchen sieht man nicht so gut unter Wasser. So ist es besser. War aber trotzdem nicht beabsichtigt. Manchmal macht die Magie Sachen, die praktischer sind, wenn man es nicht so genau nimmt.}

Leicht pikiert lief die Klasse nun ins Wasser und nahm sich eine Portion Dianthuskraut.

\enquote{Aber Professor}, beschwerte sich ein Schüler. \enquote{Wir können uns unter Wasser gar nicht unterhalten.}

\enquote{Das ist richtig}, entgegnete ihm Professor Elber. \enquote{Sie werden sich unter Wasser verständigen müssen. Es ist eine Prüfung, auf die sie sich nicht vorbereiten können. Viel Spaß.}

Chwalla erklärte ihnen, wohin sie schwimmen mussten, nachdem sie sich getrennt hatten. Dann tauchte sie unter und wartete. Jeder nahm das Kraut in den Mund und tauchte Sekunden später freiwillig ins Wasser, nachdem sich Kiemen bildeten und die Schüler über Wasser nicht mehr atmen konnten.

Chwalla schwamm voran. Unter Wasser war alles leicht bläulich. Je tiefer man kam, desto weniger sah man, doch die Wachen, welche Chwalla begleiteten, hatten kleine Leuchtquellen dabei, damit sie besser sehen konnten. Die Meereslebewesen brauchten sie nicht, sie sahen auch so genug. Es ging immer weiter hinunter in das Dunkle. Nach endlosen Minuten hielt Chwalla an und drehte sich um. Die Wächter schwammen näher und gaben einem Mitglied aus der Gruppe die Leuchtquelle. Es war eine kleine Kugel, die ein behagliches Licht ausströmte. Chwalla nahm von einer der Wachen einen Dreizack entgegen und stieß ihn in die Höhe. Kleine verschiedenfarbige Kugeln kamen aus der Spitze und breiten sich zu einer Straße aus, der man folgen musste. Jede Gruppe hatte eine andere Straße aus leuchtenden Punkten, denen sie folgen mussten. Das Spiel konnte also beginnen.

Neville wurde in Harrys Gruppe nun als Anführer bestimmt. Sie mussten sich mithilfe einer Zeichensprache verständigen, die sie vorher vereinbart hatten. Es waren nur wenige Zeichen, denn die Zeit reichte für mehr nicht aus. Dann folgten sie der Spur aus Licht, allen voran Neville. Als sie ihrem Ziel Nähe kamen, waren dort überall Lichter, damit sie besser sehen konnten und die Wachen nicht wussten, wann sie angreifen würden. Diesen Vorteil würden sie bei einem realen Angriff nicht haben. Neville schwamm hinter einen Busch, um sich in dessen Deckung heranzuschleichen. Die anderen folgten ihm.

Es musste genau geplant werden. Sie waren zu fünft, konnten aber nur vier Wachen zählen, also musste einer ins Innere schwimmen, während die anderen die Wachen ablenkten.

Nachdem der Weg frei war, schwamm Harry in das Innere, doch das war gar nicht so einfach, zuerst musste er nach unten schwimmen, und dort erwartete ihn dann ein Tintenfisch. Noch schlief er, aber seine Arme waren ihm im Weg. Er konnte unter Wasser keinen Zauberspruch sagen, also musste er sich konzentrieren. Er nahm seinen Zauberstab, zeigte auf die Arme des Fisches und ohne ein Wort zu sagen, bewegten sich die Arme des Tintenfisches vom Gang, den sie versperrten, weg. Dahinter konnte er in einigen Metern Entfernung eine Tür erkennen, doch er konnte nicht mehr dorthin schwimmen, denn der Tintenfisch wurde wach und griff mit seinen Armen nach ihm. In Panik schwamm er zurück und der Tintenfisch verfolgte ihn. Am Eingang des Versteckes angekommen, warteten schon die anderen auf ihn. In Schach gehalten von den Wachen. Sie hatten verloren und keine Zeit mehr für einen erneuten Angriff. Die Wache gab ihnen zu verstehen, dass sie zurück an die Wasseroberfläche schwimmen sollten und gab ihnen eine kleine schwarze Platte mit, welche sie abgeben sollten. Dies stand auf einem geschützten Zettel, den eine Wache ihnen zeigte.

Als Harrys Gruppe wieder mit dem Kopf über Wasser atmen konnte, sahen sie bereits am Ufer Tische in den Farben ihrer Gruppen. Ebenso viele Grills mit leckeren Würsten, sowie Professor Elber, der eine Grillzange in der Hand hielt und die Würste ab und an umdrehte. Neben ihm unterhielten sich angeregt Professor Dumbledore und Professor Flitwick, außerdem stand an jedem Tisch ein kleiner Elf und schien auf irgendetwas zu warten. Harry folgte den anderen an den Rand des Sees, um ins Trockene zu gelangen.

Sofort kam der Elf heran und fragte jeden, was er auf seinen Hotdog haben möchte. \enquote{Was möchten sie auf ihre Hotdogs haben Sir? Wir haben Kraut, Zwiebeln, Senf und Ketchup. Suchen Sie sich was aus.}

Harry bestellte seinen mit Senf und wenig Zwiebeln, Neville nahm seinen mit Ketchup und die anderen aus seiner Gruppe mit allem. Der Elf poppte hinter den Grill und nahm vom Tisch daneben ein Brötchen, schnitt es auf, nahm mit einer Zange eine Wurst von Grill und tat die bestellten Sachen darauf. Dann poppte er zurück und überreichte den ersten Hotdog. Harry konnte gar nicht so schnell schauen, wie der Elf ihnen fünf das Essen brachte. Außer ihnen war anscheinend noch niemand zurück. Sie liefen auf Professor Dumbledore und Professor Flitwick zu, um sich danebenzustellen und der Unterhaltung beizuwohnen. Dabei aßen sie ihre Hotdogs.

Plötzlich hörten sie vom Wasser ein Gekeuche und Gehuste. Die nächste Gruppe war gerade aufgetaucht. Und kurz darauf die nächste. Die beiden Elfen, welche bislang hinter ihren Tischen standen, liefen auf die Gruppen zu und nahmen die Bestellungen entgegen.

Harry ging ein paar Schritte zurück, um sich an seinen Tisch zu lehnen. Von hier konnte er der Unterhaltung ebenso gut beiwohnen. Das Schwimmen und Tauchen hatte ihn hungrig gemacht. Er sah sich um, aber es gab nichts zu Trinken.

Er sah den kleinen Elfen an und fragte ihn: \enquote{Gibt es auch was zu trinken?}

\enquote{Ja Sir}, antwortete der Elf. \enquote{Im Schloss! Was möchten Sie haben, Sir?}

\enquote{Sag mir erst, wie du heißt}, sagte Harry.

\enquote{Wonkes}, antwortete der kleine Elf.

\enquote{Dann bring mir bitte ein Glas Wasser, Wonkes}, sagte Harry mit leichtem Lächeln zu dem Elfen. Der verschwand und stand keine zwei Sekunden danach wieder mit einem Glas frischem kalten Wasser in der Hand neben ihm. Er reichte es ihm und Harry bedankte sich.

\enquote{Ah, das tat gut}, sagte Harry, als er sein Glas mit einem Zug ausgetrunken hatte. Die anderen drehten sich erstaunt um und fragten Harry, wo er denn das Glas herhabe. \enquote{Von Wonkes}, sagte Harry und zeigte auf den kleinen Elfen. Die anderen bestellten gleich etwas, denn das Schwimmen hatte auch sie durstig gemacht.

Neville gab seine schwarze Platte ab und machte sich mit den anderen auf den Weg zurück ins Schloss. Morgen würden sie darüber noch diskutieren.

\trenn

Es war schon spät und Harry entspannte sich von den Strapazen seines Quidditch-Trainings. Langsam wurden die Schüler im Gemeinschaftsraum weniger. Ginny saß alleine auf einem Sofa. Die Letzte ihrer Mitschülerinnen war vor wenigen Minuten aufgestanden und zu Bett gegangen. Harry stand also auf und setzte sich neben Ginny auf das Sofa. Vorsichtig begann er.

\enquote{Ginny, Schatz?}, fragte er zögerlich.

\enquote{Ja Harry, was ist?}

\enquote{Ich\abs ich weiß nichts mehr. Ich meine, wie und wann\abs}, er schluckte einmal, \enquote{seit wann sind wir zusammen?}

Ginny sah ihn komisch und mit leicht glasigen Augen an.

\enquote{Ich meine, ich liebe dich Ginny, aber ich habe komplett vergessen, wie wir\abs} Er sah betreten zu Boden. \enquote{Ich hätte dir meine Gefühle schon viel früher offenbaren sollen, aber ich war so ein Feigling. Und als du mit anderen Jungs\abs Na ja, ich war eifersüchtig und wusste nicht, ob du mich noch magst.}

\enquote{Natürlich mag ich dich.}

Jetzt war es Harry, der sie komisch ansah. Ginny bemerkte das und küsste ihn zur Bestätigung, dass es mehr als nur mögen war. Dies nahm ihm eine schwere Last von seiner Seele.

\enquote{Ich liebe dich, Harry. Es war nur besonders schwer seit Beginn des letzten Schuljahres. Und nachdem du mich als eine Schwester bezeichnet hast, da\abs}

\enquote{Das war dumm von mir, Ginny. Ich bemerkte deine Tränen und ruderte zurück, falls du nicht mehr für mich empfinden würdest.} Auch Harrys Augen wurden langsam feucht.

Ginny bemerkte dies lächelnd und küsste ihn erneut. Dann fing sie an. \enquote{Es begann vor fünf Wochen. Du hast dich gerade von deinen Strapazen erholt und hast mir gesagt, dass du nun bereit wärst. Ich habe dich natürlich komisch angesehen. Dann hast du meine Hände genommen und mir gesagt, dass du die ganze Zeit ein Idiot warst und dir schon länger klar war, dass du mich liebst. Ich habe deinem Werben schließlich nachgegeben, nach etwa anderthalb Wochen, wollte aber wissen, warum du dann mit Luna zusammen warst. Du hast mir auch etwas über Pansy erzählt. Du hast gemeint, du brauchst noch etwas Zeit, das zu verarbeiten, dann bist du ins Koma gefallen.}

Harry nickte. \enquote{Weißt du}, begann er, \enquote{das war eher eine Art Zufall. Ich weiß nicht genau, aber für Luna empfinde ich zwar etwas, aber nicht so stark und nicht das gleiche wie für dich. Es ist so, als ob meine Gefühle für sie anderen Ursprungs sind. Es sind nicht mehr die Gefühle für einen Partner, die ich einige Wochen lang hatte. Es ist vielmehr eine Art der seelischen Verbindung.}

\enquote{Du meinst eine platonische Liebe?}

\enquote{Nein, etwas anderes. Es ist eine Art Zuneigung, die sich im Laufe der Zeit gewandelt hat zu etwas, mit dem wir beide voll und ganz zufrieden sind.} Er senkte seinen Kopf, da er Ginny nicht länger in die Augen schauen konnte. \enquote{Weißt du, es ist mir unangenehm darüber zu reden, aber\abs} Er nahm sie ganz fest in die Hand, schaute sie an und sagte dann ganz schnell, bevor er seinen Blick wieder von ihr abwendete: \enquote{Wenn du mit jemanden schläfst und sich aufgrund einer besonderen seelischen Kombination nicht nur der Körper, sondern auch der Geist verbindet, dann kann man das nicht mit etwas vergleichen. Das körperliche Verlangen ihr gegenüber ist gleich null, aber\abs} Dann wandte er seinen Kopf ab und machte weiter. \enquote{Ich hatte einfach Angst, dass du mich nicht mehr haben wolltest, dass du der Meinung seist, du könntest mir nie mehr genug geben, dass ich dich immer mit Luna vergleichen würde. Ich dachte, du fühlst dich immer in ihrem Schatten.}

Das gab Ginny erst einmal zu denken. Als sie nach minutenlangem Nachdenken immer noch keine Reaktion zeigte, verabschiedete sich Harry von ihr und ging zu Bett.

\trenn

\enquote{Sie müssen besser aufpassen, Mister Potter. Diese Klatscher können einem ganz schön zusetzen. Ich behalte Sie über Nacht hier und dann können Sie morgen wieder neuen Taten entgegensehen. \gst Aber dieses Mal etwas vorsichtiger}, ermahnte ihn Madame Pomfrey.

Harry nickte und blieb im Bett liegen.

Gegenüber des Ganges war ein Sichtschutz aufgebaut und man hörte ein leises Atmen und Wimmern.

\enquote{Alles in Ordnung?}, fragte Harry vorsichtig, nachdem Madame Pomfrey den Raum verlassen hatte, um zu Abend zu essen.

\enquote{Ja}, kam zaghaft hinter dem Sichtschutz hervor.

\enquote{Wie heißt du?}, fragte Harry.

\enquote{Alina Meyers, Slytherin}, kam schwach von der anderen Seite. \enquote{Und du?}

\enquote{Harry Potter\abs}

\enquote{Gryffindor}, antwortete sie. \enquote{Du bist im sechsten Jahr, richtig? Ich bin im Zweiten.}

\enquote{Ja!}, antwortete Harry. \enquote{Wie siehst du aus? Damit ich weiß, ob du mir schon einmal aufgefallen bist.} Er hörte ein Kichern als Antwort. \enquote{Was ist?}, fragte er nach.

\enquote{Natürlich bin ich dir aufgefallen. Ich habe dich gleich nach deiner Ankunft umgerannt. Tut mir leid.} Harry überlegte kurz und kramte in seinem Gedächtnis. \enquote{Was ist?}, fragte Alina nach einiger Zeit.

\enquote{Nichts. Ich habe nur kurz überlegt, ob ich dich noch in Erinnerung habe. Aber ich glaube, ich weiß es jetzt wieder.}

Harry hatte durch seine verstärkten Bemühungen und der Wiederaufnahme seiner Ok"-klu"-men"-tik-Übungen das Bild gefunden und sah die Szene nun wieder deutlich vor sich.

\begin{rueckblick}
\enquote{Ist das nicht toll, Harry}, fragte Hermine.

Harry wollte gerade antworten, da stürmten zwei Schülerinnen um die Ecke und warfen ihn fast um.

\enquote{Oh, Entschuldigung. War keine Absicht}, sagten die beiden Mädchen und rannten davon.
\end{rueckblick}

\enquote{Lange braune Haare, graue Augen und leichte Sommersprossen auf der Nase}, sagte Harry dem Vorhang.

\geraeusch{Ieks.} \enquote{Das weißt du noch? Wie kannst du dir das alles merken?}

\enquote{Der Mensch kann sich viel mehr merken, als es uns bewusst ist. Wir müssen nur dafür Sorge tragen, dass wir Zugriff darauf haben. Ich habe mir sehr mühsam diese Technik erarbeiten müssen. Und es klappt nicht bei jedem.}

Das Mädchen nickte, doch Harry konnte es nicht sehen. Die Tür zur Krankenstation ging auf und Professor Elber kam herein. Er legte, als er Harry sah, einen Finger auf seine Lippen und ging leise in Richtung Büro. Er klopfte kurz an und trat dann ein.

\enquote{Madame Pomfrey ist beim Essen}, antwortete Harry ihm.

Nach einer Weile kam er wieder aus ihrem Büro heraus und hatte einen kleinen Tigel in der Hand. Er sah wieder zu Harry und legte abermals einen Finger auf seine Lippen. Er wollte gerade die Krankenstation verlassen, da kam ein Wimmern und Weinen durch den Vorhang. Professor Elber blieb stehen und ging zum verhüllten Bett.

\enquote{Alles in Ordnung?}, fragte er.

\enquote{Madame Pomfrey}, kam es abgehackt hinter dem Vorhang hervor.

\enquote{Die ist beim Essen}, antwortete Professor Elber.

Es herrschte kurz Ruhe, dann kam ein: \enquote{Hilfe.}

Professor Elber legte eine Hand an den Vorhang, schien sich aber doch eines Besseren zu besinnen und fragte, die Hand immer noch am Vorhang: \enquote{Darf ich Ihnen versuchen zu helfen?}

\enquote{Ja}, kam es leise zurück.

\enquote{Dazu müsste ich aber den Vorhang zur Seite schieben.}

\enquote{Wenn es sein muss.}

Er schob den Vorhang ein paar Zentimeter zur Seite und machte erst einmal einen halben Schritt zurück \gst und wurde bleich. Er schluckte ein paar Mal und trat dann hinter den Sichtschutz, zog sich einen Hocker heran und sprach mit Alina ganz leise. Dann hörte Harry etwas.

\enquote{Harry, wenn Sie wollen und Sie es verkraften, dann möchte Alina, dass Sie zu ihr kommen}, sagte Professor Elber.

\enquote{Ich würde gerne, Professor. Aber ich soll nicht aufstehen}, antwortete Harry.

\enquote{Kein Problem}, hörte er und ehe er noch einmal blinzeln konnte, bewegte sich sein Bett auf das von Alina zu, das Bett daneben zur Seite und tauschte mit seinem den Platz.

Als Harry in ihr Sichtfeld kam, verschlug es ihm fast die Sprache. Nachdem er sich gefasst hatte, sagte er zu Alina: \enquote{Du siehst immer noch hübsch aus.}

Sie errötete leicht. Professor Elber strich ihr über die Backe und meinte: \enquote{Nicht rot werden, das hinterlässt hässliche Flecken auf der Haut.}

Ihr Gesicht war wie früher und hatte sich kaum verändert, doch die Haut war nicht mehr rosig und zart, sondern ledrig und härter. Ihre Fingernägel wurden länger und die Anordnung hatte sich leicht verändert. Sie wurden mehr und mehr zu Klauen. Dann durchlief sie ein weiterer Schub. Ihre Augen veränderten sich in ein helles Gelb und die Haare wurde grauer.

\enquote{Harry wird heute bei Ihnen bleiben und Sie trösten.}

Dankbar nickte sie ihm zu und sah danach mit einem sehnsüchtigen Lächeln zu Harry.

\enquote{Was ist mit ihr passiert?}, fragte Harry.

\enquote{So weit bin ich noch nicht gekommen}, antwortete sein Professor.

Also erzählte Alina von ihrer Übungsstunde mit ihrer Freundin Bessi. Dem Üben des Schwebezaubers und des missglückten Färbezaubers auf der Feder. Sie erzählte, dass sie sich gegenüber standen und sie vom Zauber getroffen wurde. Professor Elber runzelte die Stirn. Er überlegte.

Plötzlich schrie Alina wieder auf und fasste sich an ihren Po. \enquote{Es tut weh}, schrie sie, als aus ihrem Hintern ein Schwanz mit einer laubförmigen Spitze zu wachsen schien. Als der Schwanz voll ausgebildet war, begann sie kurz zu fauchen.

Professor Elber legte seinen Kopf in seine Hände und sah gar nicht glücklich aus. \enquote{Weiß Madame Pomfrey was Ihnen fehlt?}, fragte er.

\enquote{Ja}, antwortete Alina.

\enquote{Hat sie es Ihnen gesagt?}

\enquote{Nein, Professor.} Dann nach einer kurzen Pause. \enquote{Was fehlt mir?}

Er hob seinen Kopf und sah Alina an. \enquote{Verkraften Sie es? Denn das, was ich Ihnen sage, könnte Ihnen mehr Angst einjagen\abs}

Alina nickte nur energisch mit dem Kopf. \enquote{Ich habe gesehen wie meine beiden Brüder starben. Einen habe ich mit sieben verloren. Den anderen kurz vor meiner Einschulung nach Hogwarts. Mich kann so leicht nichts schocken.}

Harry war beeindruckt, wie stark dieses Mädchen doch sein musste. Andererseits brauchte sie wohl jemanden, der sie tröstet oder ihr einfach nur Gesellschaft leistet.

\enquote{Also gut. Haben Sie schon einmal etwas von Harpyien gehört?}, fragte er.

Alina sah ihn an. Dann nahm sie den kleinen Handspiegel von ihrem Nachtkästchen und betrachtete ihr Gesicht. Nachdem sie den Spiegel zurückgelegt hatte, nahm sie ihren Schwanz in die Hand und betrachtete ihn, sowie ihre Klauen. \enquote{Ich werde zu einer!}, stellte sie mehr fest als dass es eine Frage war und schaute Harry und Professor Elber an.

Dieser nickte und meinte dann: \enquote{Ich weiß nicht wieso. Ich werde mit ihrer Mitschülerin sprechen. Sie soll uns an einem Dummy demonstrieren, was sie gemacht hat und dann sehen wir weiter. Ich habe so etwas noch nie erlebt. Ich habe schon über die irrsinnigsten Verwandlungen gelesen oder von ihnen gehört.} \gedanke{Oder sie selber erlebt.} \enquote{Aber so etwas ist mir neu. \gst Ich werde jetzt gehen und Ihre Mitschülerin mitbringen.} Er stand auf und verließ den Raum.

Harrys Bett rückte noch ein bisschen näher und Alina versuchte nach Harrys Hand zu greifen. Dieser ließ es zu und dachte bei sich: \gedanke{Wie die Klauen von Hedwig.}

\enquote{Woran denkst du, Harry?}, fragte Alina.

\enquote{An Hedwig. Ihre Klauen fühlen sich genauso an wie deine Hände.}

Alinas Augen wurden feucht.

\enquote{Nicht doch, Alina. Nicht weinen.}

\enquote{Fällt mir aber schwer.}

Ein paar Minuten später kam Madame Pomfrey herein und schaute in Richtung des Bettes in dem Harry liegen müsste. Dann fuhr ihr Kopf nach einem kleinen Räuspern herum und sie entdeckte ihn, wie er in seinem Bett neben Alina lag. Ihre Klaue in seiner Hand.

\enquote{Was machen Sie da, Mister Potter?}, fragte die Krankenschwester.

\enquote{Ich tröste Alina}, gab Harry knapp zurück.

\enquote{Wie kommen Sie dazu\abs}, begehrte Madame Pomfrey auf.

\enquote{Alina hat mich gebeten\abs}

\enquote{Mister Potter\abs} Harry fuhr lasziv mit seiner Zunge über seine Lippen und sah Madame Pomfrey dabei direkt an. Sie verstummte augenblicklich. \enquote{Also gut}, antwortete sie und ging in ihr Büro.

\enquote{Wie hast du das gemacht?}, wollte Alina wissen.

\enquote{Mein früherer Zustand der weiblichen Bevölkerung gegenüber, Alina}, sagte Harry.

\enquote{Oh, das.}

Es dauerte ca. eine halbe Stunde, da standen Professor Elber und Bessi vor den beiden Betten. Die Tür zur Krankenstation war gesichert und Madame Pomfrey stand mit gezogenem Zauberstab da. Professor Elber beschwor eine Feder hervor und legte sie auf ein Nachtkästchen, welches er in die Mitte des Raumes gezogen hatte. Bessi vollführte den Zauber und die Feder wurde in eine verwandelt, die von einer Harpyie stammte. Professor Elber und Madame Pomfrey beobachteten sie genau. Er legte die Feder beiseite und beschwor eine neue herauf. Dann vollzog er denselben Zauber wie Bessi. Die Feder verfärbte sich wunschgemäß. Nach einer neuen Feder nahm Professor Elber Bessis Zauberstab und versuchte es erneut. Und wieder wurde die Feder die einer Harpyie. An einer erneuten Feder durfte Bessi mit Professor Elbers Zauberstab es erneut versuchen. Dieses Mal klappte der Zauber.

Ein paar weitere Tests mit Bessis Zauberstab und verschiedene Zauber brachten als Ergebnis nur einen Defekt beim Färbezauber hervor. Sonst schien der Zauberstab ordnungsgemäß zu funktionieren.

\enquote{Behalten Sie ihn, Bessi, aber färben Sie damit nichts ein}, erklärte Professor Elber ihr. \enquote{Ich werde Mister Ollivander schreiben, er möge doch bitte herkommen. Oder noch besser, ich besuche ihn und versuche ihn mitzubringen.} Mit diesen Worten verließ er die Krankenstation. Madame Pomfrey ging ebenfalls durch ihr Büro in ein kleines Zimmer, in dem sie immer schlief, wenn sie solche Patienten hatte, um näher bei ihnen zu sein und ihnen schneller helfen zu können.

Nachdenklich setzte sich Bessi auf Harrys Bett, um Alina anzuschauen. Dicke Tränen liefen ihr übers Gesicht. \enquote{Es tut mir leid, Alina. Ich wollte das nicht.}

\enquote{Ist schon gut, Bessi. An dir liegt es nicht. Jemand hat deinen Stab manipuliert.}

Bessi legte sich auf die Seite und die beiden Mädchen unterhielten sich noch etwas. Doch viel zu schnell waren sie eingeschlafen. Harry zog Bessi vorsichtig zu sich unter die Decke, um sie vor dem Auskühlen zu schützen. Er versuchte das Bett zu verlassen, aber sein gebrochenes Bein schmerzte zu sehr. Als sich Bessi auch noch herumdrehte und sich an ihn schmiegte, ließ er den Versuch bleiben und legte sich resigniert zurück.

Als er am nächsten Morgen aufwachte, schmerzte sein Bein nicht mehr. Er stand vorsichtig auf und humpelte \gst ohne sein Bein zu sehr zu belasten \gst zum Nachbarbett. Dort legte er sich auf die Decke und schlief nochmals kurz ein.

Sanft wurde er durch ein Bussi auf eine Wange geküsst. Es war Bessi. \enquote{Danke}, sagte sie, \enquote{dass ich die Nacht bei dir verbringen durfte. Ich weiß nicht warum, aber bei dir habe ich mich wohlgefühlt. Jetzt weiß ich auch was Tamara meinte, als sie sagte: \inner{Mein Vertrauen zu ihm war einfach da.}} Sie schlug die Hand vor den Mund.

Harry lächelte sie an, nahm eine ihrer Hände in seine und sagte: \enquote{Ich werde ihr nichts davon erzählen.}

Bessi umarmte ihn erneut und verließ eilig die Krankenstation.

Nun hatte er das Herz von vier Slytherin-Mädchen gewonnen. Neben Katharina und Pansy, verstand er sich jetzt auch mit Alina und Bessi recht gut.

Harry zog sich gerade an, als Professor Elber mit Mr Ollivander und Bessi den Raum betrat.

\enquote{Ah, Mister Potter. Schön Sie wiederzusehen.}

\enquote{Ebenfalls, Mr Ollivander}, antwortete Harry.

Mr Ollivander nahm sich Bessis Zauberstab und setzte sich auf einen Stuhl, der durch einen Wink von ihm herankam und hinter ihm stoppte. Dann untersuchte er Bessis Zauberstab sehr genau und gründlich. Er nahm auch Alinas Zauberstab und Harrys als Referenz her. Dann vollführte er mit allen drei Zauberstäben denselben Zauber und nur bei Bessis Zauberstab kam die Veränderung der Feder in eine Harpyien-Feder. Mr Ollivander reichte ihr einen anderen Zauberstab, der ebenso elf Zoll lang war, aus Haselnussholz und im Kern ein Schwanzhaar eines Hippogreifes hatte.

Bessi vollführte den Zauber ohne Probleme und war zufrieden.

Mr Ollivander zog aus seiner Jacke eine Schachtel heraus, die wie eine Zauberstab-Schachtel aussah, aber etwas kleiner war und ziemlich zerschlissen aussah. Ein Gummiband hielt die Schachtel zusammen, welches er nun entfernte und die Schachtel öffnete. Er holte zwei Zauberstäbe heraus und legte einen auf einen kleinen Tisch, den er heranzog. Der Zauberstab wurde hinten unterlegt, sodass er genau auf die Feder zeigte. Bessi hatte gesagt, dass Alinas Zauberstab ebenfalls auf die Feder gezeigt hatte, als sie den verheerenden Zauber ausführte.

Mr Ollivander verlangte einen Dummy, den Madame Pomfrey normalerweise für ihre Erste-Hilfe-Kurse hatte. Er stellte den Dummy hinter dem Zauberstab auf und zielte mit Bessis altem Stab auf die Feder in der Mitte. Er sprach den Färbezauber und anstatt der Feder verwandelte sich der Dummy in eine Puppe mit dem Aussehen einer Harpyie. Nachdenklich sah Mr Ollivander den Zauberstab in seiner Hand an. Dann nahm er den zweiten Zauberstab aus seiner Schachtel und begann Bessis alten Zauberstab zu untersuchen. Nachdem er damit fertig war, holte er seinen zweiten auf dem Stuhl liegenden Zauberstab und untersuchte ihn ebenfalls.

\enquote{Kann ich den Zauberstab hier irgendwo einem Test unterziehen, ohne dass jemand Schaden davon trägt?}, fragte Mr Ollivander.

\enquote{Kommen Sie mit, Hogwarts hat einen eigenen Raum dafür. Wir sind hier bestens ausgerüstet}, sprach Professor Elber.

Bessi wurde zum Frühstück geschickt und versprach ihrer Freundin, ihr die Hausaufgaben zu bringen und den Stoff mitzuschreiben. Mr Ollivander und Professor Elber verließen die Krankenstation und Harry ging zum Frühstück, nachdem Madame Pomfrey ihn entlassen hatte.

Auf dem Weg dorthin fragte er sich, wie man einen Zauberstab verändern konnte, sodass er nur bei einem Zauber ein fehlerhaftes Verhalten zeigt.

\stimme{Das ist äußerst kompliziert, Harry. Das geht weit in den Bereich der schwarzen Magie}, meldete sich Salazar.

\gedanke{Könnte Voldemort es schaffen?}, fragte Harry.

\stimme{Ja, ihm dürfte es möglich sein. Nur frage ich mich was er damit\abs Wenn es ihm gelingt, Bessi unter den Imperius zu setzen, dann könnte er sie dazu veranlassen, die Zauberstäbe anderer Schüler zu entwenden und denselben Zauber an ihnen zu vollziehen und sie dann zurücklegen. \gst Das dürfte sie zwar umbringen, aber das macht ihm nichts aus. Dann hätte er eine Armee von Harpyien, die er unter seiner Kontrolle hat und dann kann er das Schloss aus dem Inneren angreifen.}
