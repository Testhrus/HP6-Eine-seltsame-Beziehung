\chapter{Wechselhaftes}


\enquote{In wenigen Tagen beginnen die Osterferien. Die Schüler, die sich bereits eingetragen haben, um hier ihre Ferien zu verbringen, tragen sich noch heute in einer der beiden neuen Aushänge ein, um entweder dieses Jahr Ostereier zu suchen, oder bei den Vorbereitungen zu helfen. Professor Sinistra wird dem Vorbereitungs-Team morgen Einweisungen geben. Tippen Sie einfach eine der beiden Listen freiwillig an und Ihr Name erscheint auf der entsprechenden Liste. Und jetzt wünsche ich Ihnen einen guten Appetit und eine angenehme Nacht}, schloss Professor McGonagall ihre Rede, setzte sich und begann zu essen.

\gedanke{Ostervorbereitungen!}, ging Harry durch den Kopf. \gedanke{Was mich da wohl erwartet?}

\enquote{Auf wessen Mist das wohl wieder gewachsen ist?}, fragte Hermine.

\enquote{Na auf wessen Wohl}, antwortete Ron. \enquote{Du hast sie doch gehört.}

\enquote{Das sagt nichts aus. Sie hat nur eine Ankündigung gemacht. Es könnte sonst wer sein. Ich denke nicht, dass es Sinistras Idee ist.}

\enquote{Auf jeden Fall hat sie sich bereit erklärt, die Organisatoren einzuweisen}, antwortete Harry.

Nach dem Essen trug er sich in die Liste der Vorbereiter ein. Er hatte früher genug Gelegenheiten, Eier zu suchen und einen Teil davon selber zu essen, bevor er sie an Dudley abgeben musste, oder der sie ihm wegnahm.

Hermine und Ron hingegen waren über die Ferien zu Hause, sodass sie sich nicht in eine der Listen eintrugen.

Harry streifte an seinem schulfreien Sonntag durchs Schloss, als er durch eine geschlossene Klassenzimmertür die Stimme seines Vgddk-Lehrers hörte. Er kam gerade von Adrian. Er hatte ihm mitgeteilt, dass es nach den Ferien losgehen würde. Er hatte sich mit der DA geeinigt, dass sie ab sofort in einem leeren Klassenzimmer üben würden. \enquote{Gut so, Draco. Weiter so. Du machst das sehr gut.} Vorsichtig näherte er sich der Tür und linste durch das Schlüsselloch. Er konnte nur die hintere Hälfte eines verschlissenen Umhangs erkennen. Aber es war eindeutig Professor Elber. Aber so sehr er sich auch anstrengte, er konnte nicht mehr sehen. Er hörte nur noch ein Stöhnen und Schnaufen. Dann einen Schrei, der von Draco kam, vermutete er. Sowie einen Plumps. \enquote{Sehr schön, Draco}, sagte Professor Elber und verschwand aus Harrys Sichtfeld. \enquote{Ruhe dich aus. Bald bist du so weit, dass du es länger beherrschst.}

Er spürte auf seinem Rücken eine Hand und erschrak. Es war Dumbledore. \enquote{Albus}, stammelte Harry.

\enquote{Na}, antwortete Dumbledore, \enquote{am lauschen?} Er drückte ihn sanft zur Seite und schaute ebenfalls durch das Schlüsselloch. Beide hörten nun, wie Professor Elber etwas aufhob und danach ein knarzendes Geräusch, als ob sich jemand auf ein Sofa oder in einen Sessel setzt.

\enquote{Ruhe dich aus. Du hast die Erholung notwendig.} Dann hörte er Schritte. Professor Dumbledore richtete sich auf und griff an den Türknauf. In dem Moment öffnete jemand von der anderen Seite die Tür und stieß fast mit Dumbledore zusammen. \enquote{Da drinnen ist leider besetzt}, antwortete er, als er einen halben Schritt zurückging und die Tür hinter sich schloss. Harry fiel das eigenartige Geräusch auf, das die Tür machte, als sie ins Schloss fiel. \enquote{Sie werden sich wohl einen anderen Ort suchen müssen.} Er schritt an Dumbledore vorbei und machte sich den Gang entlang in Richtung des großen Tores.

\enquote{Frederick}, rief ihm Dumbledore hinterher. \enquote{Wer ist da drin?}

\enquote{Niemand.}

Dumbledore legte seine Hand an die Tür, zog sie kurz zurück und nahm dann seinen Zauberstab heraus. Professor Elber war inzwischen verschwunden. Professor Dumbledore schwenkte seinen Zauberstab, doch die Tür öffnete sich nicht. Er dachte nach. Harry sah an seinem Gesicht, dass er überrascht war, obwohl er es nicht zeigte. Er sah Harry an. Dann schwenkte er wieder seinen Zauberstab und die Tür gab ein leises \geraeusch{Klack} von sich. Vorsichtig betraten sie den Raum.

Doch er war leer. Keine Menschenseele war zu sehen. \enquote{Hast du nicht auch was gehört?}, fragte er Harry.

\enquote{Doch, Albus. Er nannte ihn Draco. Aber wo ist er?} Sie sahen sich intensiv in dem leeren Raum um. \enquote{Glaubst du, er ist durch einen Geheimgang verschwunden?}

\enquote{Soweit ich weiß, gibt es zu diesem Zimmer keinen Geheimgang.}

Harry kam das Ganze komisch vor. Er lief zurück zur Tür und stellte sich bei offener Tür in den Türrahmen. Dann beugte er sich so weit herunter, dass er auf Höhe des Schlüsselloches war. Er bildete mit seinen Fingern einen Ring und sah hindurch. \enquote{Das sieht mir nicht so aus, als ob es der Raum ist, den ich zuerst gesehen habe.}

Dumbledore drehte sich um und ging nun ebenfalls in Harrys Stellung. Harry musste grinsen, als er seinen Schulleiter so stehen sah. \enquote{Du hast recht, Harry. Das ist ein anderer Raum.} Dumbledore schloss die Tür. Er blieb stumm einige Sekunden stehen. Er drehte sich zu Harry um und meinte dann: \enquote{Eigenartig. Na ja, ich gehe erst mal nach Hogsmeade. Sicherheitsüberprüfungen und Vorbereitungen für eure Apparierstunden.} Er ließ Harry alleine stehen.

Dieser machte sich auf den Weg zur Großen Halle, wo die Einweisung der Eierverstecker stattfinden sollte. Unterwegs richtete er noch eine der Messinglampen an der Wand, indem er sie absprengte und eine neue an die Wand zauberte.

\enquote{Schön, dass Sie alle da sind}, sagte Professor Sinistra, als die Schüler und Lehrer, die sich in die Liste eingetragen hatten, vor ihr standen. Die Türen der Großen Halle schlossen sich und eine große aufgerollte Pergamentrolle begann zu schweben, als Professor Sinistra mit ihrem Zauberstab auf sie zeigte.

Sie zeigte die Ländereien von Hogwarts und viele eingezeichnete rote, blaue, grüne, gelbe, violette, orange und rosa Punkte, die wohl die Verstecke der Eier darstellen sollte.

\enquote{Wie kommen Sie eigentlich darauf, Ostereier zu verstecken?}, fragte eine Ravenclaw aus Harrys Jahrgangsstufe. Er war sich nicht sicher, meinte aber, dass sie Helen hieß.

\enquote{Ich persönlich kannte den Brauch nicht, bis ich davon in unserem Lehrerzimmer gehört hatte. Dieser Brauch faszinierte mich derart, dass ich mich einzulesen begann. Bei den Muggeln werden die Süßigkeiten nur gesucht und können dann verspeist werden. Ich dachte mir, dass wir das Ganze etwas abwandeln. Bei uns wird es nicht nur eine Suche, sondern auch eine Art Rätsel. Die Schüler sollen die Eier nicht nur suchen und finden, sondern auch Zauber überwinden, um an sie heranzukommen.} Professor Sinistra lief auf das Pergament zu und zeigte mit ihrem Zauberstab auf die einzelnen Bereiche. \enquote{Diese Bereiche hier sind für die einzelnen Jahrgangsstufen. Die genauen Verstecke und Positionen habe ich auf extra Pergamenten festgehalten. Eure Aufgabe ist es, heute die Eier zu verstecken. Die Schüler und Lehrer, die suchen werden, werden eine dreiviertel Stunde in der Großen Halle festgehalten, bis die Versteckaktion beendet ist. Bis hierher Fragen?}, schloss Professor Sinistra.

Es dauerte eine Weile, doch dann schüttelten die ersten ihre Köpfe.

\enquote{Gut, dann kommen wir zur Verteilung der kleinen Gruppen. Die älteren unter ihnen werden die unteren Jahrgangsstufen beim Suchen überwachen und die Verstecke entsprechend präparieren. Dies machen wir so, weil die unteren Jahrgangsstufen nicht über die entsprechenden Zauber verfügen, die notwendig wären, um die entsprechenden Schutzzauber auszuführen. Ich teile Sie also den entsprechenden Gruppen zu.}

Auf dem Boden erschienen Kreise mit den entsprechenden Farben und kleine Tische daneben mit Pergamentrollen in den passenden Farben. Professor Sinistra teilte danach die ersten Gruppen ein, die die Jahrgangsstufen eins bis vier vorbereiten mussten. Die Schüler stellten sich in die genannten Kreise und warteten ab.

\enquote{Schauen Sie bitte schon einmal auf Ihre Pergamente, damit sie die Positionen und Zauber kennenlernen. Falls etwas unklar sein sollte; ich komme nachher vorbei.}

Die Schüler nahmen sich die Pergamente und betrachteten diese, während Professor Sinistra weitere Einteilungen vornahm. Die beiden Lehrer steckte sie in die schwarze Gruppe, um für ihre Kollegen die Eier zu verstecken. Nachdem alle bis auf Harry verteilt waren, kam Professor Sinistra auf ihn zu.

\enquote{Mister Potter?}, fragte sie leise. \enquote{Trauen sie sich zu in der Lehrergruppe zu arbeiten? Ich habe das Gefühl und vor allem einiges gehört, dass ihnen die entsprechenden Zauber keine Probleme bereiten dürften.}

Harry war erstaunt, ob der großen Verantwortung und des großen Vertrauens, das ihm da entgegengebracht wurden.

\enquote{Ich kann es versuchen, Professor. Aber, wieso trauen Sie mir das zu?}

\enquote{Wissen Sie, ich habe im Lehrerzimmer viele Dinge über Sie gehört, die mich vermuten lassen, dass Sie dazu in der Lage sind. Sie könnten es durchaus schaffen, dass einige meiner Kollegen und Kolleginnen zu knabbern haben. Außerdem habe ich vor, selber teilzunehmen. Ich habe ein paar Verstecke im Auge, die ich speziell für mich angelegt habe.} Sie übergab Harry ein Pergament. \enquote{Ich habe danach mein Gedächtnis verändert, damit ich während der Suche die Verstecke nicht kenne und auch suchen muss. Ich habe für mich einen eigenen Bereich festgelegt, den Sie bitte zusätzlich für mich präparieren.}

Harry machte große Augen und nickte schließlich. Das Pergament würde er hüten wie einen Augapfel. Zusätzlich nahm er sich vor, mit einigen zusätzlichen Zaubern, um die er Salazar bitten würde, die Verstecke zu sichern. Er stellte sich zu den beiden Lehrern und wartete ab.

\enquote{Stehst du nicht falsch?}, fragte ihn Zacharias, den er aus der DA kannte.

\enquote{Ich wurde hier hingestellt, also nein}, antwortete er.

Professor Sinistra fragte in die Runde, ob noch etwas unklar sein, doch alle verneinten, nachdem auch den letzten Gruppen Gelegenheit gegeben wurde die Pläne zu studieren. Das große Pergament wurde zusammen gerollt und zur Seite gelegt. Jeder der Schüler und Lehrer warf einen Sicherungszauber darauf, damit keiner vor Ablauf der Suchaktion spicken konnte.

Dann wurden sie in die Freiheit entlassen und die suchenden Schüler und Lehrer kamen in die Großen Halle. Die Türen wurden verschlossen und die Gruppe begann, die Eier und andere Süßigkeiten zu verstecken.

Harry lief mit Professor Flitwick und Professor McGonagall in das ihnen zugewiesene Gebiet und begann die Eier und andere Süßigkeiten zu verstecken. Das Gebiet war durch Bänder und andere Stangen abgesteckt, damit es zu keinen Missverständnissen kommen konnte. Harry belegte sein Versteck mit den geforderten Zaubern. Bei einigen hatte er Mühe und fragte Professor Flitwick, ob sie auch richtig ausgeführt wurden. Nach einer kurzen Kontrolle und einer entsprechenden Bestätigung machte er sich an das nächste Versteck.

Nachdem die Eier aus Harrys Gruppe versteckt waren, ging er zu seinem speziellen Bereich und begann die Sachen für Professor Sinistra zu verstecken. Er sah auf dem Pergament nach und legte diverse Schutzzauber über die Eier. Dann kam ihm eine Idee, wie er die Eier schützen konnte. Er fragte Salazar nach dem entsprechenden Zauber und legte ein streng kontrolliertes lebendiges Feuer über die Eier. Diese färbte er so, dass sie wie Schlangeneier aussahen und das Feuer nur aus seinem Versteck kam, wenn man die Eier ergreifen wollte. Ein anderes Versteck präparierte er so, dass man der Meinung war, dass ein Dementor die Sachen bewachen würde. Es war zwar nur ein einfacher Illusionszauber, aber Harry legte noch einen Kältezauber über das entsprechende Gebiet. Da das Gebiet am See lag, legte er ein Ei, geschützt durch einen Zauber, unter Wasser und vermerkte die Stelle auf dem Pergament. Nun hatte er eine Stelle mehr, als ihm vorgegeben war. Er würde es merken, wenn Professor Sinistra schummelte, da ein Versteck übrig blieb.

Die Gruppen trafen sich wie verabredet am Sammelpunkt und Harry wurde bestimmt, die Sucher zu holen, während die anderen wieder in ihre Bereiche zurückgingen und die Suche überwachten. Harry öffnete die Tür und die Sucher wurden von Harry und einer Karte, die bereits dort schwebte und nur farbige Bereiche zeigte, entsprechend den Jahrgängen eingeteilt. Harry ging mit Professor Sinistra und den anderen Lehrern zuerst in den Lehrerbereich, bevor er mit Professor Sinistra alleine in den für sie gesonderten Bereich zum Suchen ging. Da sie auch aus ihrem Gedächtnis die Positionen der Verstecke für die Lehrer entfernt hatte, begann sie dort mit der Suche. Allerdings hörte sie nach einem erfolgreich entdeckten Versteck auf.

Die Lehrer fanden die Verstecke relativ schnell \gst verglichen mit den Schülern \gst doch hatten sie zu tun, um an ihren Preis zu kommen. Den meisten Spaß hatte Dumbledore, meinte Harry. Er freute sich wie ein kleines Kind während der Suche und auch während der Erforschung der Schutzzauber. Harry war gespannt, ob Dumbledore es gleich merken würde, da er dieses Versteck schützte und ihn zusätzlich mit einem kleinen Illusionszauber, der die Position um einige Zentimeter verschob, belegt hatte. Dumbledore fiel prompt darauf rein und griff ins Gras, als er das Körbchen mit Eiern holen wollte.

Harry verkniff sich ein Lachen und schmunzelte still in sich hinein.

\enquote{Haben Sie noch einen Zauber darübergelegt?}, fragte ihn Professor Flitwick leise.

Harry sah zu ihm und nickte. Er ging in die Hocke und flüsterte in sein Ohr: \enquote{Ich habe einen einfachen Illusionszauber darübergelegt. So sieht es aus, als ob das Körbchen ein paar Zentimeter weiter entfernt sei.}

Professor Flitwick lächelte ebenfalls in sich hinein.

Als alle Stellen im Lehrerbereich gefunden wurden, kam Professor Sinistra auf Harry zu und sie gingen zusammen in den gesonderten Bereich. Dumbledore und McGonagall folgten ihnen, warteten außerhalb und beobachteten die beiden.

Professor Sinistra machte sich an die Arbeit und besah sich ihr Gebiet. Sorgfältig suchte sie entsprechende Stellen ab, an denen sie ein Versteck vermutete. Die ersten drei fand sie recht schnell und hatte nach wenigen Minuten die Schutzzauber entfernt. Es blieben nur noch drei übrig.

An einer Stelle blieb sie stehen und schüttelte sich. Sie lief bereits weiter, als sie sich anders besann und noch einmal umdrehte. Stirnrunzelnd und mit gezogenem Zauberstab näherte sie sich einem Busch, aus dem plötzlich ein Dementor hervorkam. Vor Schreck zuckte sie einige Schritte zurück und die anderen Lehrer zogen sofort ihre Zauberstäbe. Harry errichtete einen Schild zwischen sich und den anderen Lehrern, damit sie nicht eingreifen konnten.

\enquote{Mister Potter, was soll das, sehen Sie nicht, dass ein Dementor Professor Sinistra angreift?}

Doch Harry ignorierte seine Hauslehrerin.

Professor Sinistra konzentrierte sich auf den Dementor und warf einen nebelartigen Patronus auf den Dementoren, der aber durch die Erscheinung glitt und keinen Effekt hatte. Sie stutze erneut und brauchte ein paar Sekunden, bis sie begriff, dass es kein echter Dementor war. Sie überlegte kurz und warf dann einen Zauber auf das Abbild, das daraufhin verschwand. Dann sprach sie den Gegenzauber, der die Kälte auflöste und die Temperatur wieder wärmer werden ließ.

\enquote{Klever, Mister Potter. Ein Kältezauber und eine Illusion eines Dementoren. Ich dachte tatsächlich, dass es ein echter war. Das muss ich mir merken.}

Harry entfernte den Schild, den er aufgebaut hatte, und sah entschuldigend zu seiner Hauslehrerin. Diese zog nur eine Augenbraue hoch und zeigte ein kaum erkennbares Lächeln. Harry lächelte zurück. Dann sah er wieder zu Professor Sinistra, die sich dem Busch näherte und ein Nest entdeckte. Das zweite Versteck.

\enquote{Das sieht aus wie Schlangeneier}, sagte sie, ging in die Hocke und streckte ihre Hand nach den Eiern aus. Doch sie zog sie sofort wieder zurück und fiel auf ihren Hintern, da sie so sehr zurückschreckte, dass sie das Gleichgewicht verlor.

Eine kleine brennende Schlange kroch zwischen den Ästen hervor und legte sich schützend über die Eier. Züngelnd wartete sie auf weitere Aktionen.

\enquote{Mit so etwas hätte ich nicht gerechnet, Mister Potter. Erst ein Dementor, dann noch eine brennende Schlange.}

Jetzt begann sie zu grübeln und laut zu denken. \enquote{Ich nehme einfach mal an, dass es sich um keine Illusion handelt. Das hatten wir schon mal. Aber welcher Zauber reagiert so lebendig?}

Harry sah mal wieder zu seinen Lehrern, die Reaktionen wie Überraschung, Überlegen, oder Unkenntnis zeigten. Harry fühlte sich gut. Er hatte seinen Lehrern eine Herausforderung gestellt, da jeder nachzudenken schien. Er sah wieder zu Professor Sinistra, die inzwischen auf dem Boden saß und auch nachdachte.

\enquote{Versuche es mal mit Wasser}, meinte Professor Flitwick.

\enquote{Ok}, antwortete Professor Sinistra. Sie nahm ihren Zauberstab und warf einen Wasserschwall auf die Schlange. Doch die kleine Schlange zeigte sich davon unbeeindruckt.

Harry sah wieder zu seinen Lehrern. Dumbledores Mimik zeigte etwas, wie ein Gedanke, der Form annahm und zu einer Lösung führte. Sein Blick wanderte zu Harry. Dann gingen beide Augenbrauen nach oben und ein leises Lächeln umspielte seine Mundwinkel.

\enquote{Denk an unseren \accentuate{Todesser}-Lehrer}, sagte er.

Mit unverständigem Blick sah Professor Sinistra zuerst zu Dumbledore, dann zu Harry, der lächelte. Dann zeichnete sich auch auf Professor Sinistra Gesicht eine Erkenntnis ab. Sie brach den Fluch des lebendigen Feuers und konnte endlich ihren Schatz bergen.

\enquote{Das war vielleicht eine Herausforderung. Wie kommen Sie denn dazu, Dämonenfeuer anzuwenden? Und vor allem, woher kennen Sie den Zauber?}

\enquote{Lebendiges Feuer, Professor. Und ich hatte eine Eingebung.}

\enquote{Ich hoffe doch mal, das war das letzte Versteck.}

\enquote{Nein}, antwortete Harry. \enquote{Eines ist noch übrig.}

Professor Sinistra ging noch einmal das gesamte Gebiet ab, fand aber nichts. \enquote{Fehlt wirklich noch ein Versteck?}

\enquote{Ja, Sie sind allerdings noch nicht das gesamte Gebiet abgegangen}, gab er ihr als Hilfe.

Professor Sinistra runzelte erneut ihre Stirn und sah sich um. Nach einer ganzen Weile sagte sie: \enquote{Der See.} Sie ging an den Rand des Sees und fand nach einigen Minuten suchen ein kleines Nest aus Eiern und einem Osterhasen. Überlegend und sich ihr Kinn reibend stand sie da. Harry konnte sein Lachen kaum noch zurückhalten, blieb aber trotzdem stumm. Er hielt sich seinen Bauch, da er wusste, dass keinerlei Zauber über dem letzten Nest lag. Sie musste einfach nur ihre Hände in das Wasser tauchen und das Nest an sich nehmen. Das Nest war lediglich durch einen Zauber vor magischem Aufrufen geschützt.

Nach einigen Minuten und sämtlichen misslungenen Versuchen, legte sie ihren Kopf schief und steckte ihren Zauberstab ein. \enquote{Kann es sein, dass\abs}, sagte sie und ging in die Hocke. Sie griff vorsichtig in das Wasser und holte das Nest heraus. \enquote{Ich fasse es nicht, keinerlei Schutzzauber. Einfach nur reingreifen und herausholen.} Sie lachte Harry an. \enquote{Mister Potter, das hat Spaß gemacht. Ich bin beeindruckt. Solche Zauber hätte ich Ihnen gar nicht zugetraut. Vor allem ein Nest ohne Zauber zu belegen und mit der Erwartung, dass sie geschützt sein müssen, zu spielen. Respekt!}

\enquote{Dann können wir ja jetzt wieder zurück}, schlug Harry vor.

Die anderen nickten und zusammen ging es Richtung Schloss.

\trenn

Während dessen tauchte Lucius Malfoy an anderer Stelle aus dem Nichts in der Nokturngasse auf. Kurz darauf spürte er eine Hand auf seiner Schulter. Erschrocken drehte er sich um. \enquote{Frederick, ich habe jetzt keine Zeit für dich. Der\abs}

\enquote{\aabs Dunkle Lord hat einen Auftrag für dich. Ja ich weiß.}

\enquote{Woher?}

\enquote{Unwichtig. Wie viel Zeit gibt er dir?}

\enquote{Das wird so vier Stunden dauern.}

\enquote{Gut, dann haben wir Zeit genug. Komm mit.}

\enquote{Aber ich muss meine Aufgabe erfüllen.}

\enquote{Ja ja}, antwortete Frederick und beide verschwanden.

Sie tauchten in einer kleinen Eingangshalle wieder auf. Sie war schlicht eingerichtet.

\enquote{Deine Sachen findest du in dieser Tüte. Deine Frau und dein Sohn warten im Wohnzimmer, deine Tochter ist in Hogwarts. Beim nächsten Auftrag schaue ich, dass ihr euch treffen könnt. Ich bin in der Küche und arbeite.} Dann öffnete er eine Tür und verschwand im Nebenzimmer.

Lucius stand die nächste Minute starr da und wusste nicht, was er tun sollte. Zum Glück war er alleine. So konnte niemand sehen, wie er um Fassung rang. Er sah auf die Tüte und schaute hinein, nachdem er auf sie zugegangen war. Es schien alles da zu sein. Dann legte er sie wieder zurück und betrat das Wohnzimmer.

\enquote{Lucius}, rief seine Frau erfreut. Sie stand auf und nahm ihn in den Arm.

Dann war Draco dran. Entgegen seiner sonstigen Art stand auch er auf und umarmte seinen Vater. \enquote{Schön, dich wieder einmal zu sehen, Vater.}

Dieses Mal konnte er nichts sagen. Seine Maske fiel fast zusammen und er umarmte seinen Sohn zum ersten Mal nach langen Jahren herzlich. Seine Frau gesellte sich zu ihnen. \enquote{Das ist gefährlich, was ihr hier macht. Der Dunkle Lord wird euch finden und mich bestrafen, wenn er herausfindet, dass ich bei euch war.}

\enquote{Du weißt aber nicht, wo wir sind, oder?}

\enquote{Nein.}

\enquote{Gut. Dann kann der Dunkle Lord dir nichts anhaben. Du wurdest ja mehr oder weniger gegen deinen Willen hier hergebracht.}

Die Tür ging auf und ein kleines Mädchen stürmte auf die Gruppe zu. \enquote{Vater}, rief Tamara und hing schon an Lucius’ Umhang.

\enquote{Tamara}, rief er und hob seinen kleinen Engel hoch. \enquote{Ich dachte, du bist heute nicht da.}

\enquote{Dachte ich auch, da heute im Schloss einiges los war. Wir haben\abs}

\enquote{Na, was habt ihr gemacht.} Sie zog den Kopf ein. \enquote{Sag schon.} Sie schüttelte den Kopf. \enquote{Hast du Angst vor mir?} Ein leicht verschämtes Nicken kam zurück. Lucius sah betrübt in ihr Gesicht. \enquote{Und wenn ich dir verspreche, dass ich nichts sage, oder tue?}

\enquote{Na ja}, fing sie an. \enquote{Wie haben Ostereier gesucht.}

\enquote{Was für Eier?}

\enquote{Ostereier. Das ist ein Brauch der Muggel.} Lucius verzog sein Gesicht, worauf Tamara leiser und schüchterner fortfuhr. \enquote{Da ist es Brauch, zu einer bestimmten Zeit im Jahr bunte hartgekochte Eier, oder welche aus Schokolade \gst Es gibt sogar welche in Form eines Hasen \gst zu suchen. Auf Hogwarts hat man, nachdem man eines entdeckt hatte, noch einen Schutzzauber lösen müssen.}

\enquote{Hauptsache, es hat dir Spaß gemacht}, sagte er jetzt versöhnlicher. \enquote{Wie ist es euch so ergangen?}, fragte er seinen Sohn und seine Frau.

\enquote{Zuerst du, damit du hinterher mit mehr Freude wieder gehst}, sagte seine Frau energisch.

\enquote{So viel Feuer hätte ich dir gar nicht zugetraut. Respekt}, meinte ihr Mann.

Währenddessen stand Frederick in der Küche und dachte über Lucius’ Situation nach. \gedanke{Er ist zu bemitleiden. Einerseits liebt er seine Frau und seine Kinder, andererseits hat er sich dazu entschlossen den Lehren und Predigten von Voldemort zu folgen. Aber ich denke, er bemerkt so langsam, was ihm das einbringt. Er ist ein Gefangener in seinem eigenen Haus. Er kann sich nicht von Voldemort abwenden, ohne dass er alles verlieren würde. \gst Sicher, er mag Muggel nicht besonders. \gst Ich hoffe nur, dass er eine Entscheidung trifft\abs die richtige.}

\trenn

\enquote{Woher hast du eigentlich den Zauber für das Dämonenfeuer?}, fragte Dumbledore.

\enquote{Lebendiges Feuer}, korrigiert Harry. \enquote{Wie wäre es mit einer Tasse Tee und ein paar Keksen?}, fragte er frech nach.

\enquote{Gerne. Bei mir, oder bei dir?}

Damit hatte Harry nicht gerechnet. Er überlegte kurz und sagte dann: \enquote{Bei mir. Gemeinschaftsraum.}

Dumbledore nickte und zusammen gingen sie Richtung Gryffindorturm.

\enquote{Kreacher}, rief Harry und der Elf erschien und lief dann neben den beiden her. \enquote{Wir brauchen eine Kanne Tee mit zwei Tassen und etwas Gebäck. Sorge bitte, dass sie im Gemeinschaftsraum der Gryffindors bereitstehen. Wir gehen gerade dorthin.}

Kreacher verneigte sich und verschwand.

\enquote{Du hast deinen Elfen ja gut im Griff}, sagte Dumbledore.

\enquote{Ja, das denke ich auch. Er hat sich, seit ich ihn habe, sehr verändert. Seine Hetzparolen sind verschwunden und ich habe den Eindruck, dass er jetzt glücklicher ist. Ich bin sogar davon überzeugt, dass es nicht mehr lange dauert, dann ist er mir gegenüber vollkommen loyal.}

Am Porträt angekommen, öffnete sich das Gemälde, ohne dass Harry das Passwort sagen musste.

\enquote{Sie wird nachlässig}, sagte Dumbledore.

\enquote{Nein, Professor. Ich bin so weit, ihr das Passwort per Legilimentik zu übermitteln. Nachdem ich wieder meinen Okklumentikunterricht aufgenommen habe, hat sich diese Fähigkeit auch herausgebildet. Bei Gemälden funktioniert das recht gut. Auch bei Geistern klappt das; nicht immer, aber dennoch recht gut.}

\enquote{Erstaunlich. Ich bin beeindruckt.}

Plötzlich spürte Harry ein vorsichtiges Tasten an seinem Geist. Harry blockte sofort ab und schickte Dumbledore ein Schockbild. Dadurch irritiert und geschockt, brach die Verbindung ab. Harry grinste in sich hinein.

Beide traten ein und setzten sich in zwei Sessel, die gegenüber standen. Harry schenkte beiden eine Tasse Tee ein und nahm einen Schluck.

\enquote{Die Idee für das lebendige Feuer hatte ich von Professor Elber, als dieser den Vortrag in der Großen Halle hielt. Er hat damit herumgespielt, wie mit einem gut erzogenen Haustier. Als ich die Nester versteckte, fiel mir das ein. Der Zauber und wie man ihn anwendet, ist mir plötzlich eingefallen. Ich dachte mir, das wäre ein toller Zauber um Professor Sinistra zu testen.} Vorsichtig zog er sein Amulett hervor. \enquote{Dann habe ich dem Zauber den Schutz der Eier aufgetragen. Im zweiten Versteck habe ich als Schutz einen Kältezauber mit der Dementorenillusion darüber gelegt. Das hat Professor Sinistra schön verwirrt.} Er grinste Dumbledore an, als er ihr Gesicht Revue passieren ließ.

\enquote{Ich weiß zwar jetzt, dass und warum du ihn angewendet hast, aber nicht, woher du den Zauber hast.}

Harry überlegte, was er sagen sollte und durfte. Da sie fast alleine waren und die, die im Raum saßen, beschäftigt waren, oder weit weg saßen, konnte er mit Dumbledore eine Unterhaltung unter vier Augen führen.

\stimme{Sag ihm, dass das Amulett dir Zauber übermitteln kann, wenn du sie brauchst. Dumbledore weiß, dass es meines war. Sag ihm, ich habe die notwendigen Zauber auf das Amulett gelegt.}

Innerlich nickte Harry. \enquote{Den entsprechenden Zauber hat mir das Amulett mitgeteilt. Slytherin hat es so verzaubert, dass mir bei Bedarf die notwendigen Zauber einfallen.}

\enquote{Interessant}, sagte Dumbledore und nahm einen Schluck Tee.

Wieder spürte er ein vorsichtiges Kratzen an seinem Geist. Harry war das langsam lästig, aber trotzdem eine gute Übung. Er zog an seinem Gegenüber und warf ihn in den runden Raum mit den vielen Türen, die sich sofort drehten. Mit gleichgültigem Gesichtsausdruck sah er Dumbledore in die Augen und versuchte seinerseits in den Geist seines Schulleiters einzudringen. Er erwartete keinen großen Erfolg, da Dumbledore für seine Legilimentik und Okklumentik-Fähigkeiten bekannt war.

Harry fand sich in einem Raum wieder. Dumbledore war ein Teenager und neben ihm war noch ein weiterer junger Mann und ein Mädchen. Harry wusste bereits, dass Albus einen Bruder hatte und eine Schwester, die aber in jungen Jahren verstorben war.

\gedanke{Das muss Ariana sein}, ging Harry durch den Kopf. Es kostete ihn eine Menge Kraft, für Dumbledore das Bild der Türen aufrechtzuerhalten und gleichzeitig etwas über ihn zu erfahren. Nach einigen Minuten entschied er sich, die Bilder die er empfing einfach zurückzuwerfen. Das entlastete ihn.

Dadurch wieder verwirrt und aufgeschreckt, zog sich Dumbledore aus Harrys Geist zurück. Doch er blockte Harry nicht, sodass er weiterhin zu sah. Ein weiterer junger Mann betrat den Raum. Der junge Albus begrüßte ihn und nannte ihn Gellert.

\gedanke{Grindelwald. Dumbledore kannte Grindelwald}, ging Harry durch den Kopf.

Dumbledore schien nicht zu bemerken, dass Harry immer noch in seinem Geist unterwegs war, oder er ignorierte es und wollte, dass Harry es erfuhr. Doch Harry war es egal. Er wurde Zeuge eines Streites zwischen den beiden jungen Männern, dem ein Duell folgte. Alles ging so schnell, dass Harry gar nicht mitbekam, wer den tödlichen Zauber ausführte. Aber nachdem Ariana tot am Boden gelegen hatte, endete der Streit. Gellert verließ das Haus und Harry hatte den Eindruck, dass Dumbledore ihn nur noch einmal sehen würde. Während des berühmten Duells, in dem Dumbledore ihn entwaffnete.

Harry sah noch die Trauer der beiden jungen Dumbledores, bevor die Szene verblasste und Harry sich zurückzog.

\enquote{Ich habe ihren Tod nie verwunden}, sagte Dumbledore.

Da wusste Harry, dass er ihn gewähren ließ. Oder er redete sich den Schmerz nur von der Seele, hatte aber nicht die Kraft, Harry bei genau dieser Szene hinauszuwerfen. Er würde es wohl nie erfahren, war aber für die Information dankbar.

\enquote{Sie haben sie geliebt, richtig?}, fragte er.

\enquote{Wir sind hier alleine, Harry.}

\enquote{Sie hat dir viel bedeutet. Wie war sie?}

\enquote{Sie war eine ruhige Person, die ihre Magie aber nicht kontrollieren konnte. Muggeljungen misshandelten sie in jungen Jahren, als sie einen Zauber ausführte und sie wissen wollten, wie sie das gemacht hatte. Mein Vater wurde deshalb nach Askaban geschickt, da er sich an ihnen rächte, wegen dem, was sie ihr angetan hatten.} Wieder nahm er einen Schluck Tee. \enquote{Danach ist meine Mutter mit uns dreien weggezogen. Sie starb kurz nachdem ich die Schule beendet hatte. Als ältester Sohn fühlte ich mich verpflichtet, mich um sie zu kümmern. Ich habe mich deswegen mit meinem Bruder gestritten, der mich nur als den Wunderjungen mit den hohen Zielen gesehen hatte. Er würde sich um Ariana kümmern. Ich solle mit meinem Freund Gellert um die Welt ziehen, so wie ich es vorhatte. Er sah mir an, dass mich Mutters Tod um meine Pläne brachte. Doch ich wollte bei ihr sein, mich um sie kümmern. Also beendete Aberforth die Schule. Er hätte sie sonst abgebrochen.}

\enquote{Aberforth? Der Besitzer des Eberkopfes?}

\enquote{Ja.}

\enquote{Welche Pläne, Albus?}

\enquote{Pläne für eine bessere Gesellschaft. Die Herrschaft der Zauberer über die Muggel.} Zum Glück hatte Harry gerade seine Tasse abgestellt, um sich Tee nachzuschenken, sonst wäre sie ihm aus der Hand gefallen. \enquote{Doch Arianas Tod hielt mir den Spiegel vor Augen. Wir waren nicht besser. Das änderte meine Jungendansichten dramatisch und ich wurde der, der ich heute bin. Ein Muggelliebhaber, wie man überall spottet.}

So langsam begriff Harry, dass sein Schulleiter eine dunkle Vergangenheit hatte. \gedanke{Kein Licht ohne Schatten, oder Dunkelheit}, dachte er sich. \enquote{Und der Kontakt zu deinem Bruder? Hat er sich verbessert?}

\enquote{Nicht besonders.}

\enquote{Du solltest dich darum kümmern. Zumindest aussprechen, oder den Versuch unternehmen. Die Zeit verläuft schnell.}

\enquote{Sehr schmeichelhaft, Harry}, sagte Dumbledore. \enquote{Du hast eine tolle Art, einem zu sagen: \inner{Du wirst alt.}}

\enquote{So war das nicht gemeint. Wenn man so etwas auf die lange Bank schiebt, dann ist man tot, bevor man sich\abs Du weißt, wie ich das meine.}

Plötzlich hörten beide einen kräftigen Faustschlag auf einem Tisch. \enquote{Verdammt, wieso klappt das nicht?}, meckerte ein Schüler.

\enquote{Ich glaube, ich werde gebraucht.}

\enquote{Dann ist unsere Unterhaltung wohl beendet.}

Dumbledore nickte, stellte seinen Tee ab und stand auf. Er ging zu dem Schüler, setzte sich und fragte ihn, wo denn das Problem sei.

Harry rief nach Kreacher, nahm sich noch einen Keks und ließ das Geschirr zurückbringen.

\trenn

Die vier Stunden waren beinahe vorbei und Tamara war seit einer halben Stunde wieder in Hogwarts, als die Wohnzimmertür aufging und Frederick hereinkam. \enquote{Es wird Zeit.}

Lucius nickte und stand auf. Er verabschiedete sich von seiner Frau und seinem Sohn. Ihn nahm er noch einmal in den Arm und seine Frau küsste er noch einmal stürmisch. Das hatte er die letzten Monate am meisten vermisst. Dann trat er auf den Flur und nahm seine Tüte in die Hand.

\enquote{Hier, trink das}, sagte Frederick und reichte ihm einen Becher mit einer Flüssigkeit.

\enquote{Was ist das?}

\enquote{Trink es einfach.}

Lucius leerte den Becher und stellte ihn danach ab. Dann ging eine merkwürdige Veränderung in ihm vor. Die Erlebnisse der letzten vier Stunden begannen in chronologischer Reihenfolge zu verblassen. Als er sich an die erste viertel Stunde nicht mehr erinnern konnte, kamen die ersten Minuten wieder. Im gleichen Zug verschwammen die nächsten Minuten des erlebten.

\enquote{Was war das für ein Trank?}, fragte er.

\enquote{Einer, der verhindert, dass der Dunkle Lord deine Erinnerungen an die letzten vier Stunden durchstöbern kann. Er wird das sehen, was du tun solltest.}

Er wurde wieder an der Schulter gepackt, verschwand und wurde direkt vor Borgin und Burkes abgesetzt. Nichts deutete darauf hin, dass ihn jemand mitgenommen hat. Der alte Borgin kam gerade aus seinem Laden, als er Lucius Malfoy entdeckte.

\enquote{Ah, guten Tag Mister Malfoy. Schön, dass Sie da sind. Ich habe das bestellte Teil hier. Kommen Sie.}

Irritiert darüber, folgte er dem alten Herrn in seinen Laden. Dieser griff unter die Theke und holte eine kleine Schatulle hervor. Er öffnete sie und entnahm ein Pergament, das er Lucius gab. Dieser entrollte es und erkannte die Schriftzeichen darauf. Es war genau das, was der Dunkle Lord von ihm wollte. Dann fiel ihm ein, dass er in seiner Tüte keine Schriftrolle gesehen hatte.

\enquote{Nehmen Sie sie mit. Sie ist schon bezahlt. Ihre Frau war heute schon da.}

Das machte Lucius skeptisch. \enquote{Schon bezahlt, sagen Sie? Was ist mit Ihnen los? Normalerweise würden sie doch nochmal kassieren.}

Der alte Borgin zog darauf hin an seinem Ärmel und enthüllte drei feine goldene, schon im Verblassen begriffene Linien. \enquote{Unbrechbarer Schwur}, sagte er.

Lucius begriff. Er bedankte sich und verließ den Laden. Kurz bevor er disapparierte, entdeckte er eine Person, die ihn zu beobachten schien. Vor seinem Haus tauchte er wieder auf. Er lief die Einfahrt entlang zu seinem Anwesen und die Türen öffneten sich automatisch, als er näher trat. Zielstrebig ging er in den Salon, wo die anderen, vor allem Voldemort, schon warteten.

Er wollte gerade etwas sagen, als ihn der Dunkle Lord aufhielt, indem er seine Hand hob. \enquote{Warte noch einen Moment, Lucius.}

Dieser gehorchte und nickte nur. Kurz darauf kam die Person, von er glaubte, sie gesehen zu haben, herein. Er sah, wie sie Voldemort nur zunickte und ihren Platz einnahm.

\enquote{Nun Lucius, hast du alle meine Sachen, die ich dich zu besorgen beauftragt habe?}

\enquote{Ja, Mylord}, antwortete Lucius und kam näher. Einzeln legte er die Teile auf den Tisch und Voldemort besah sie sich genau. Lucius spürte, wie sein Geist sondiert wurde. Der Dunkle Lord fand genau das, was er zu finden suchte. Lucius’ Einkäufe. Verteilt über verschiedene Orte quer durch England.

\enquote{Du kannst dich setzen, Lucius}, sagte er. Lucius tat, wie ihm befohlen wurde, und nahm seinen Platz ein. \enquote{Spuren von deiner Familie?}, fragte er ihn.

\enquote{Meine Eltern sind schon gestorben, mein Lord. Andere Verwandte habe ich nicht}, gab er selbstbewusst zurück. Er wusste, dass das eine glatte Lüge war, aber die Überzeugung, mit der er sie wieder gab, überraschte ihn doch. Zum Glück nur innerlich.

Voldemort besah sich seine Gegenstände. \enquote{Ihr könnt gehen}, sagte er und nahm die einzelnen Stücke genauer unter die Lupe. Ein dunkler Stein, eine Rolle Pergament. Eine Flüssigkeit in einer gläsernen Viole. Einen kleinen Holzstab, ein Löffel aus einem dunkel schimmernden aber dennoch glänzenden und fliesendem Material. Klauen und andere Teile von toten Tieren und ein paar Ratten. Diese waren allerdings für seine Schlange. \enquote{Essen, Nagini}, flüsterte er und warf ihr die Ratten in den Schlund.

Lucius und ein paar andere gingen hinaus. Dort traf er auch auf Snape. Zusammen verschwanden sie in Lucius’ Arbeitszimmer.

\trenn

Die Ferien waren ruhig, da kaum Schüler im Schloss geblieben war. Harry saß mit Tamara auf dem Boden im Gemeinschaftsraum. Sie spielten Monopoly. Arabella hatte Harry dieses Spiel nach Weihnachten geschickt. \accentuate{Vielleicht kannst du es brauchen}, schrieb sie damals in ihrem Brief. Nach mehreren Partien hatte Tamara Blut geleckt. Immer wieder bat sie Harry um eine Partie. Da außer Hausaufgaben, die er immer wieder machte, nichts anstand, nahm er sich die Zeit. Zudem machte es ihm Spaß. Diese Partie ging an Tamara. Harry störte das nicht, da er die letzten beiden gewonnen hatte.

\enquote{Noch eine Runde, Harry}, forderte Tamara.

\enquote{Es ist schon spät. Wir sollten ins Bett gehen.}

\enquote{Oh}, jammerte Tamara. Sie räumte das Spiel in Rekordzeit in die Schachtel und krabbelte auf Harry zu. Sie schmiegte sich mit ihrem Rücken an seine Brust und legte seine Hand  auf ihren Bauch.

Müde schloss sie ihre Augen und meinte: \enquote{Weißt du, Harry. Du bist wie ein Bruder für mich geworden. Ich habe dich sehr gerne.}

\enquote{Was ist mit Draco?}, fragte er nach, als er sich mit ihr auf die Seite fallen ließ.

\enquote{Der war schon immer mein Bruder und wird es immer sein. Aber dich habe ich neu hinzugewonnen. Willst du mein Bruder sein?}

Harry dachte nach. Er wusste nicht viel über magische Bindungen, aber er war vorsichtig.

\enquote{Weißt du Tamara, auch ich empfinde so. Aber noch möchte ich diese Bindung nicht eingehen.}

Tamara stutzte, dann begriff sie. \enquote{Keine Angst, ich will dich nicht magisch an meine Familie binden. \gst Würde ich das, wenn du ja gesagt hättest?}, fragte sie mit geschlossenen Augen und wachen Ohren nach.

Harry strich ihr über ihr Haar und gab ihr einen Kuss auf ihre Schläfe. \enquote{Keine Ahnung, meine kleine. Aber sicher ist sicher.}

Ein sanftes Lächeln zeichnete sich auf ihrem Gesicht ab. Dann dämmerte sie weg. Harry strich ihr noch eine Weile durch ihr Haar. Dann musste er sie ins Bett bringen. Er überlegte, ob er es schaffen würde, sie in ihr Bett zu bringen. Aber er konnte nicht zu den Mädchenschlafsälen gelangen. In einem Raum der Jungen könnte er sie unterbringen. Doch was, wenn sie mitten in der Nacht aufwachen würde? Also nahm er sie hoch und trug sie in seinen Raum. Er legte sie in das Bett neben sich, zog ihr ihre Robe und ihre Schuhe aus und deckte sie danach zu. Dann richtete er sich selber her und stieg in sein Bett, um zu schlafen.

Mitten in der Nacht spürte er jemand in sein Bett krabbeln. Er öffnete erst ein Auge, danach das andere. Tamara kniete an seiner Bettkante und hatte die Hände vor seinem Körper in die weiche Matratze gestemmt. Harry drehte sich auf seinen Rücken und sah sie im aufkommenden Licht der Petroleumlampen an.

\enquote{Was ist? Geht es dir nicht gut?}

\enquote{Doch, Harry.} Sie gab ihm einen Kuss auf die Wange. \enquote{Danke.}

\enquote{Wofür?}

\enquote{Dass du mich ins Bett gebracht hast. \gst Aber warum bin ich hier bei dir?}

\enquote{Ich konnte dich nicht in dein Zimmer bringen. Weißt du, Jungs können nicht zu den Mädchen. Und in einem anderen Zimmer dachte ich, hättest du Angst, wenn du mitten in der Nacht aufwachst und nicht weißt, wo du bist.}

Tamara lächelte ihn an. \enquote{Darf ich bei dir\abs?}

\enquote{Geh zurück in dein Bett und schlafe.} Tamara zog einen Flunsch. \enquote{Es sieht nicht gut aus, wenn du bei mir schläfst.}

\enquote{Aber es sieht doch keiner.}

Nun hörten beide ein Glucksen. \enquote{Ich sehe es.}

Beide sahen in die Richtung, aus der der Satz kam.

\enquote{Myrte, was machst du denn hier?}

\enquote{Ich dachte, ich schaue mal nach, ob du auch schläfst.}

\enquote{Um dann was mit mir zu machen?}

Da Tamara kalt wurde, legte sie sich hin und zog etwas an Harrys Decke.

\enquote{He, solltest du nicht in dein Bett?}

Myrte kam auf die beiden zu geschwebt. \enquote{Es scheint, als ob du nichts zu sagen hättest. Du bist in der Unterzahl.} Auch Myrte näherte sich Harry und schmiegte sich an ihn. Zumindest hatte es den Anschein.

\enquote{Raus}, sagte er energisch.

\enquote{Nein}, sagten beide gleichzeitig.

Er überlegte, wie er die beiden aus seinem Bett werfen könnte. \gedanke{Tamara kann ich kitzeln. Aber was mache ich mit Myrte? Einen Zauber?}

Während er noch so überlegte, hatten sich Tamara und Myrte bereits an ihn geschmiegt und fingen an zu schlafen. Harry merkte nicht, wie er ins Reich der Träume hinüberglitt. Als er Stunden später wieder wach wurde, dachte er wieder darüber nach, wie er seine beiden Damen aus seinem Bett werfen könnte. Für ihn waren erst wenigen Minuten vergangen. Er schlug beide Augen auf und merkte, dass die morgendliche Helle bereits in den Raum schien. Von beiden Seiten gewärmt, lag Harry in seinem Bett. Er drehte seinen Kopf nach links und entdeckte Tamara, die mit dem Bauch an seine Seite gekuschelt war. Auch rechts war es warm. Harry drehte seinen Kopf. Dort lag Myrte. Die Bettdecke auf ihr, was eigenartig aussah, da er durch ihren Körper durch Sehen konnte. Aber ihm war auf seiner rechten Seite nicht kalt. Im Gegenteil, von dort kam die gleiche Wärme, wie von links. Er spürte keinerlei Unterschied. Beide atmeten ruhig und gleichmäßig.

\enquote{Aufstehen meine hübschen}, sagte er und bewegte sich in seinem Bett, damit sie aufwachen würden. Zuerst schlug Tamara ihre Augen auf und sah ihn mit sandigen Augen an. Harry zog seine Hände hervor und wischte ihr vorsichtig mit seinem Finger den Sand aus den Augen, nachdem er seine befreit hatte. Danach streckte er seine Hände nach oben, ballte sie zu Fäusten und streckte sich. Das weckte auch Myrte auf. Von ihr bekam er einen Kuss auf die Backe und dasselbe  Lächeln, das sie schon so oft gezeigt hatte, wenn sie ihn gesehen hatte.

\enquote{Du bist ganz warm, Myrte}, sagte Harry und versuchte einen Arm auf sie zu legen. Doch wie immer glitt er durch. Ihr schien das nichts auszumachen. Sie stieg hoch und durch die Decke hindurch.

\enquote{Wie kommt es, dass die Decke auf dir lag, als ich heute Morgen aufgewacht bin?}

Myrte sah ihn fragend an und hob nur ihre Schultern. \enquote{Weiß ich nicht.}

Tamara stieg aus dem Bett heraus und ging zur Zimmertür. \enquote{In einer viertel Stunde unten im Gemeinschaftsraum? Frühstücken?}

\enquote{Gerne. Ich bin da.}

Tamara verschwand aus dem Zimmer und ging.

\enquote{Darf ich mit zum Frühstücken?}, fragte Myrte.

\enquote{Von mir aus}, antwortete Harry.

Selig lächelnd machte sie eine Luftrolle rückwärts und sah ihn abwartend an.

Harry stieg aus dem Bett und verschwand im Bad. Als er wieder kam, wartete Myrte schwebend über seinem Bett. Sie schwebte im Schneidersitz und wartete.

Harry verschwand hinter einer Wand und zog sich um. Gerade als er seine Unterwäsche anhatte, kam Myrtes Kopf durch die Wand und Harry erschrak.

\enquote{Myrte, etwas mehr An- und Abstand bitte.}

Sie gluckste wieder und zog sich zurück. \enquote{Ich warte unten auf euch}, sagte sie und verschwand durch die Wand.

Harry zog sich an und ging danach nach unten. Tamara kam gerade herunter, als er die unterste Stufe erreicht hatte. Zu dritt gingen, beziehungsweise schwebten, sie zur Großen Halle.

Einige Schüler und Lehrer staunten, als die drei in der Großen Halle ankamen. Tamara und Harry beluden sich ihre Teller und Myrte schwebte immer mal wieder auf Mundhöhe durch die Speisen. Die Erinnerung daran machte sie glücklich. Nach einem kurzen Mahl verabschiedete sich Tamara von Harry und machte sich auf den Weg zum Slytherintisch, wo sie den Rest ihres Frühstücks einnahm und sich mit einem Schulkollegen ihres Bruders unterhielt.

Den Rest der Ferien passierte es Harry noch ein paar Mal, dass er mit einer, oder beiden, Mädels die Nacht verbringen musste. Eigentlich fand er es ja ganz angenehm, die Wärme einer Person zu spüren, aber andererseits wollte er einen gewissen Schein wahren. Auch führte er während seiner Ferien ein paar Unterhaltungen mit Draco. Es waren Unterhaltungen, die sie in ihren Träumen führten und somit nicht abgehört werden konnten. Die nur über ihre Vergangenheit gingen. Wie Harry aufgewachsen war und wie Draco aufgewachsen war. Beide verstanden einander nun etwas besser und die Feindschaft zwischen ihnen wich so langsam einer neutralen Einstellung zueinander. Während dieser Kontakte erzählte er Draco, wie er bei der Auswahlzeremonie nach Gryffindor geschickt wurde.

\trenn

Weder Ron, noch Harry oder einer der anderen Schüler wussten, wen sie als Apparierlehrer hatten. Selbst keiner der Lehrer oder Hagrid ließ etwas raus. Vielleicht wussten sie es auch gar nicht. Heute sollte es sein. Nach dem Frühstück würden sie ihre erste Stunde im Apparieren haben. Hermine hatte vermutlich bereits alles gelesen, was es über das Thema zu geben schien. Denn sie redete ununterbrochen über die verschiedenen Appariertechniken und deren Unterschiede.

\enquote{Warte doch einfach ab}, gab ihr Ron einen Dämpfer, \enquote{weißt du noch, als du bei Professor Elber deine Antworten gegeben hast, die waren nur halb richtig. Würde mich nicht wundern, wenn es hier ähnlich laufen würde.} Ron hoffte, sie dadurch zur Ruhe zu bringen, was auch einige Zeit half, aber auf ihrem Weg nach Hogsmeade fing sie wieder an.

Gedankenverlorenen ging Harry Ron und Hermine hinterher. Luna war schon unten. Sie bildete eine Ausnahme, da sie mit Harry verbunden war. Zwar musste sie die Prüfung erst nächstes Jahr machen und auch Auffrischungsstunden nehmen, durfte aber schon jetzt dabei sein, da beide noch immer im Körper des anderen waren.

In der Hogsmeader Stadthalle angekommen, erwartete sie neben den anderen Schülern nur Professor McGonagall. Doch sie machte nicht den Eindruck, dass sie auf Schüler wartete. Sie machte auf Harry einen unruhigen Eindruck. Immer wieder sah sie auf ihre Uhr und lief unruhig hin und her. Gerade wollte sie zur Klasse sprechen, als aus dem Nichts hinter ihr Professor Elber auftauchte. Als sich Professor McGonagall umdrehte, erschrak sie. \enquote{Frederick, wie kannst du mich so erschrecken?}

\enquote{Nachdem du mich dazu gezwungen hast die Stunden zu unterrichten, musste ich es dir doch irgendwie zurückzahlen.}

Professor Elber machte einen leicht mürrischen Eindruck. \enquote{Hermine, ich nehme an, dass sie bereits wieder alles über das Apparieren gelesen haben?}

\enquote{Ja}, antwortete sie.

\enquote{Gut \gst welche Strecke könnten sie beim Apparieren auf einmal, also ohne abzusetzen, überwinden? Wie lange brauchen sie dafür? Und warum ist das so?} Alle schauten wie immer auf Hermine, die sicherlich die richtigen Antworten geben würde. Doch sie schien still. Nach einer Weile fragte Professor Elber in die Runde \enquote{Jemand anderes?}

Lange Zeit herrschte in der Runde stille. Gelegentliche Blicke zu Professor McGonagall ergaben auch nichts Sinnvolles. Plötzlich meldete sich Neville. \enquote{Ja Neville. Was denken Sie?}

\enquote{Ich meine, es müssten so ca. 20.000 km sein.}

Professor Elber zeigte sich erstaunt. Aber Professor McGonagall machte ein Gesicht, das sagte, das stimmt nicht. \enquote{Und wie genau kommen sie darauf?}

\enquote{Na ja, der Umfang der Erde beträgt etwa 40.000 km. Und da ich zum gegenüberliegenden Punkt nur die Hälfte brauche, sind das eben 20.000 km.}

\enquote{Ein interessanter Ansatz. \gst Wäre der direkte Weg durch den Planetenkern nicht kürzer?}

\enquote{Also ich möchte mich nicht durch über 6.500 km Erde, Fels, Gestein und Metall apparieren.}

\enquote{Da ist was dran. Noch weitere Ideen?}, fragte Professor Elber in die Runde.

Vereinzelte Zwischenrufe von \enquote{100 km}, oder \enquote{400 km} konnte man hören.

\enquote{Noch jemand anderer Meinung?}

Stille.

\enquote{Dann löse ich mal auf. Neville hat recht. Theoretisch ist es möglich, etwa 20.000 km zu apparieren. Doch bevor ich das erkläre, hätte ich noch gerne gewusst, wie lange man für diese Strecke braucht.} Wiederum war nichts als Schweigen zu hören. Plötzlich schien Hermine eine Idee zu haben, denn sie suchte in ihrer Tasche etwas. Sie nahm ein Blatt Pergament heraus und zeichnete mit ihrer Feder darauf herum. Professor Elber näherte sich ihr und schaute ihr begeistert zu, als sie auf das Papier das Bild eines Taschenrechners malte.

Er wartete ab, bis sie fertig war. Dann nahm sie ihren Zauberstab und tippte mit murmelnden Worten mehrmals auf das Pergament. Nachdem sie ihren Zauberstab wieder eingesteckt hatte, begann sie auf den aufgemalten Tastenfeldern zu tippen, woraufhin im oberen Anzeigenfeld die unten angetippten Zahlen erschienen. Nachdem sie einige Rechenoperationen durchgeführt hatte, verkündete sie: \enquote{133ms}.

\enquote{Erstaunlich}, antwortete Professor Elber. \enquote{Toller Zauber. Den müssen sie mir mal bei Gelegenheit beibringen. Kann man sicher brauchen. Die Antwort ist übrigens richtig. Doch nun zu dem, was einige von Ihnen stutzen ließ und vermuten lässt, dass nur wenige 100 Kilometer möglich sind.}  Er stellte an die gesamte Runde nun die Frage: \enquote{Wer von Ihnen ist schon einmal appariert worden? Ich meine damit Seit-an-seit Apparition.} Einige der Schüler streckten die Hand in die Luft. \enquote{War das angenehm?}, fragte er.

\enquote{Nein}, kam es allgemein zurück.

% Anmerkung: Das Vereinigte Königreich wird im Süden vom Ärmelkanal, im Osten von der Nordsee und im Norden und Westen vom Atlantik begrenzt.
\enquote{Genau. Deswegen schaffen es die wenigsten überhaupt eine größere Strecke zurückzulegen. Wenn wir uns auf den Planeten begrenzen, dann sind die 20.000~km die maximale Strecke, die aber normalerweise schwer zu schaffen ist. Mit einigermaßen guter Kondition schaffen sie auf einmal in etwa die halbe Nord-Süd-Strecke durch England. Die längste Nord-Süd-Ausdehnung ist knapp 1000~km lang, während die Ost-West-Ausdehnung knapp 500 km erreicht. Der höchste Berg ist mit 1342~m der Ben Nevis in Schottland. Das sind ca. 500~km. Wir beschränken uns die ersten Male nur darauf, durch die Halle zu apparieren, danach durch Hogsmeade durch und schließlich bei ihrer Prüfung dann so ca. 100~km.} Er lief nun auf eine Tür an der Seite der Halle zu, öffnete sie und zog einen kleinen Wagen herein, auf dem etwas stand, das Harry an eine Lottomaschine erinnerte.

Nur waren im Inneren keine Kugeln. \enquote{Jeder von euch kommt einmal dem Auswähler näher, tippt ihn mit seinem Zauberstab an und sagt dabei seinen Namen.}

Nacheinander traten alle vor und tippten den Auswähler an. Es erschien jedes Mal im Inneren eine kleine Kugel.

Jeder stand wieder an seinem Platz. Professor Elber fuhr fort. \enquote{Wenn ihr später den ersten Versuch unternehmt, werdet ihr immer zu zweit sein. Aber zuvor werden wir ein wenig Theorie machen. Was müssen wir alles beim Auswählen eines Platzes, der zum Apparieren geeignet sein soll, beachten?}

Harry wusste, dass man nicht gesehen werden durfte.

\enquote{Er muss sicher sein}, antwortete Malfoy.

\enquote{Gut Draco. Was verstehen Sie darunter?}

\enquote{Ich darf nicht gesehen werden wie ich erscheine und mein Auftauchen sollte kein Argwohn erregen. Wenn ich zum Beispiel ohne erkennbaren Grund aus einer Seitengasse herauskomme, die keinen Ausgang hat, dann kann das die Muggel misstrauisch machen.}

\enquote{Gut erkannt, Draco. Was zählt noch dazu?}

Draco überlegte. Sofort schoss Parvatis Hand in die Höhe. Professor Elber blickte kurz zu ihr und danach wieder zu Malfoy. Harry dachte, Professor Elber wollte Draco sagen, \gedanke{Denken Sie nach Draco. Noch ein Punkt.}

\enquote{Man sollte darauf achten, dass man nicht von\abs äh\abs Überwachungskameras der Muggel gefilmt wird.}

\enquote{Sehr gut, Draco}, meinte Professor Elber. \enquote{Fünf Punkte für Slytherin. \gst Sie werden sicherlich die wichtigsten Regeln gelernt haben, oder sie haben darüber gelesen. \gst Ziel. Wille. Bedacht. \gst Dies ist, was sie überall lernen werden, wenn es um das Apparieren geht. \gst Bei mir werden sie noch zwei weitere Sachen lernen. \gst Direktheit und Bequemlichkeit. \gst Wir werden zuerst die Drei-Punkt-Regel lernen, bevor ich Ihnen die beiden anderen Punkte noch beibringen werde. Ich finde, das macht die Sache angenehmer. Bilden sie bitte einen Kreis. Minerva, stellst du dich bitte mir gegenüber?}

Die Schüler nahmen Aufstellung und Professor Elber und Professor McGonagall standen sich gegenüber.




\begin{kommentar}
Frederick bringt Draco in einem leeren Zimmer etwas bei. Harry und Dumbledore lauschen dabei etwas. Was nirgends erwähnt wird, oder zumindest später nur angedeutet wird, ist, dass Frederick Draco beibringt ein Animagus zu sein. Dracos Form ist dabei ein Drache. Ich glaube es wird viel später (eventuell auch im zweiten Teil) irgendwo erwähnt.
\end{kommentar}
