\chapter{Die Weihnachtsüberraschung}


Auf halbem Wege zum Schloss kam ihnen Hagrid entgegen.

\enquote{Oh, hallo Harry, hab dir noch was zu Weihnachten schenken wollen, konnte es aber nicht vorher bekommen}, sagte er, als er vor ihnen stand und Harry sein Weihnachtsgeschenk überreichte.

\enquote{Sie sind aber groß}, meinte Hermines Mutter.

\enquote{Er ist halb Riese und halb Mensch, Mutter}, sagte Hermine.

Die Augen ihrer Mutter weiteten sich und sie fragte: \enquote{Und wie groß ist so ein Riese normalerweise?}

\enquote{Och, nur so acht bis neun Meter groß, aber es gibt noch größere}, meinte Hagrid, als sei es das normalste auf der Welt, so groß zu sein.

Hermines Eltern waren sprachlos. Nicht nur, dass sie soeben einen Halbriesen gesehen hatte, der sich gut mit einem Schulkameraden ihrer Tochter verstand, nein, sie hörten auch noch von Riesen, die so hoch wie ein Haus werden konnten. Harry lachte und griff Hermines Vater unter seinem Arm und nahm ihn mit Richtung Schloss. Hermine nahm ihre Mutter und ging Harry hinterher.

Wieder zurück im Gemeinschaftsraum fingen sich beide so langsam wieder und kehrten zurück zur Normalität. Plötzlich poppte Dobby herbei und wollte aufräumen, da er nicht erwartet hatte, dass sich um diese Zeit jemand im Gemeinschaftsraum aufhalten würde. Die Feiertage bedeutete zusätzliche Arbeit für die Elfen, obwohl Professor Dumbledore Hermine gesagt hatte, dass er für die Aufgaben nur bezahlte Elfen herangezogen hatte und ihnen eine Sonderzulage gewährte.

\enquote{Oh, Verzeihung Harry Potter. Dobby wollte sie nicht stören.} Dann sah er Hermines Eltern an.

\enquote{Du störst uns nicht, Dobby}, meinte Harry. \enquote{Nicht im Geringsten.}

Hermines Eltern standen ihre Gesichter still.

\enquote{Das}, so sagte Hermine, \enquote{ist Dobby. Er ist ein Elf. Er kümmert sich hier um viele Sachen. Er räumt unter anderem auf und kocht.}

\enquote{Freut mich Sie kennenzulernen, Mister Dobby}, sagte Hermines Mutter ganz erstaunt.

\enquote{Das ist meine Mutter}, fügte Hermine hinzu, als sie Dobby fragend anschaute.

\enquote{Ah, Miss Hermines Mutter. Ich bin Dobby, der Hauself.} Er drehte sich wieder um und kümmerte sich um den Gemeinschaftsraum.

Einige Minuten saßen alle vier still im Raum und ließen ihre Gedanken schweifen.

\enquote{Das war wirklich eine tolle Idee von eurem Lehrer. Ich hätte nie gedacht, dass ich hier mal sein werde, geschweige denn Weihnachten hier feiern würde}, sagte Hermines Mutter.

\enquote{Wie kam das eigentlich, dass sie hierherkamen?}, fragte Harry.

\enquote{Na ja}, antwortete Mister Granger. \enquote{Das war so: Wir waren gerade zu Hause und aßen zu Abend, als eine Eule mit einem Brief im Schnabel an das Fenster klopfte. Wir dachten schon, es sei wieder Post von unserer Tochter. Sie schreibt uns immer mal wieder. Aber als wir dann den Namen Professor Elber \gst Lehrer im Fach Verteidigung gegen die dunklen Künste \gst auf der Rückseite lasen, wurde uns etwas mulmig. Wir hatten schon Angst um unsere kleine Miene.}

Harrys Augen wurden größer und er starrte Hermine an.

\enquote{Dad}, sagte Hermine.

\enquote{Miene?}, sagte Harry mit einem fragenden Ausdruck in seinem Gesicht, als er Hermine ansah.

Hermine errötete leicht, fasste sich aber dann recht schnell. \enquote{Wehe du sagst irgendjemanden etwas davon. Ich schwöre dir, ich verhexe dich, dass dich Madame Pomfrey die nächsten acht Wochen nicht mehr aus der Krankenstation entlässt.}

Harry grinst nur und meinte frech. \enquote{Mach’s ruhig. Du hältst es ja ohne mich eh nicht aus und besuchst mich jeden Tag.}

Jetzt lachten Hermines Eltern und auch Hermine konnte sich nicht mehr zurückhalten.

\enquote{Zurück zum Thema}, schlug Hermines Vater vor. \enquote{Wir öffneten also etwas unsicher den Brief und lasen.}

\begin{brief}
Sehr geehrte Familie Granger,

Ihre Tochter hatte sich in Hogwarts eingeschrieben, um die Weihnachtsferien hier zu verbringen. Ich nehme mir vor, dieses Fest für alle dagebliebenen unvergesslich zu gestalten. Zu diesem Zweck möchte ich Sie beide gerne zu einem kurzen Aufenthalt in Hogwarts einladen. Sollten Sie zustimmen, trennen Sie bitte den unteren Teil des Pergaments ab und geben ihn der Eule, die diesen Brief gebracht hat mit. Zu gegebener Zeit erhalten sie einen weiteren Brief mit exakten Informationen, wann und wo sie abgeholt werden.

Reden Sie mit niemandem darüber. Nicht einmal mit Ihrer Tochter.
\signumspace
Mit vielen Grüßen

Professor Frederick Elber

Verteidigung gegen die dunklen Künste
\end{brief}

\enquote{Damit waren unsere Sorgen vergessen und wir freuten uns natürlich, einmal Hogwarts sehen zu können. Wir trennten den unteren Teil des Pergaments ab und antworteten, dass wir kommen werden. Wir gaben den Brief der Eule, die daraufhin verschwand.}

\enquote{Und wie sind Sie schließlich angekommen?}, fragte Harry.

\enquote{Als es Zeit wurde, kam ein weiterer Brief. Er enthielt ein Schreiben mit Instruktionen und eine große ausfaltbare Scheibe. Auf die mussten wir uns zu gegebener Zeit einfach darauf stellen und kurz darauf waren wir in einem Dorf am Fuße Hogwarts. Wir sahen Professor Elber, der uns kurz erklärte, dass wir alle noch etwas Zeit hatten und uns ein wenig umsehen konnten. Glücklicherweise hatten wir noch ein bisschen Zauberergeld und so konnten wir uns ein paar Sachen kaufen. Später dann kamen diese merkwürdigen Kutschen an, wir stiegen ein und wurden hochgefahren, wo wir wie beschrieben vor den verschlossenen Türen der Großen Halle warteten.}

Harry musste nur grinsen und auch Hermine fühlte sich glücklich.

Plötzlich fragte ihre Mutter: \enquote{Und wie lange bist du schon mit deinem Freund zusammen?}, und schaute danach zu Harry.

Hermine errötete wieder. \enquote{Das ist nicht mein Freund.}

\enquote{Aber du hast uns doch erzählt, dass du mit jemandem zusammen bist.} Hermine druckste herum. Sie wollte Harry wohl nicht kränken. \enquote{Ist es Ron?}, fragte Harry.

Hermine schaute ihn nur an und nickte leicht.

\enquote{Ist schon in Ordnung, Hermine. Ich habe auch jemanden.}

Das hätte sie am allerwenigsten von Harry erwartet. \enquote{Wer?}, entfuhr es ihr ganz aufgeregt.

\enquote{Kannst du dir das nicht denken? Wir sind zusammen unter dem Baum im Hof gesessen. Dumbledore war auch dabei.}

Hermines Gesicht blieb buchstäblich stehen. Harry hätte, wer weiß was darum gegeben, es jetzt Fotografieren zu können.

\enquote{Du meinst doch nicht etwa Luna?}, fragte sie ganz erstaunt. Harry grinste nur und schaute sie an. \enquote{Ihr beide? Du und sie? \gst}

Harry grinste weiter. Plötzlich viel ihm der Gemeinschaftsraum ein. Sobald alle Eltern und Verwandten weg sein würden, würde es auch wohl für Ron und Hermine Zeit, in alles eingeweiht zu werden, Platz genug wäre ja noch. Nachdem Hermine wieder ihre Fassung gewonnen hatten, redeten die vier noch eine Weile über Gott und die Welt, bevor es an der Zeit war, zu Bett zu gehen.

Unruhige Träume durchzogen seine Nacht. Er fand sich wieder im Hause der Malfoys. Er wusste noch von Salazar, dass er sich im näheren Umkreis \gst wenige hundert Meter \gst um Voldemort herum frei bewegen konnte. Also begann er sich umzusehen. Er sah wieder den Raum mit dem Kamin und dem Feuer darin. Das große Familiengemälde mit den vier Malfoys, das über dem Kamin an einer sonst kahlen Steinwand hing. Drei davon sahen eigenartig verzerrt aus. Davor der Tisch, an dem die Todesser beratschlagten. Er drehte sich um und sah dieselbe Fensterreihe, mit den Bleiglasfenstern und den Figuren, die sich darin bewegten. Es waren Schlangen, die von einem Fenster in das andere schlängelten. Er sah das noch immer zerstörte Beistelltischchen und, als er sich nach links drehte, eine große, schwere Eichentüre. Er wollte sie öffnen, als er mit seiner Hand durch glitt. Also trat er nach kurzem Zögern einfach hindurch. Er ging eine große geschwungenen Treppe hinunter und sah eine offene Tür, aus der er Stimmen vernahm. Sie schienen sich nicht richtig zu unterhalten, sondern nur Buchstaben und Zahlen zu nennen. Und dann immer wieder: \enquote{Du bist dran!}, gefolgt von einem mechanischen Klacken.

Er wollte gerade durch den Türspalt schauen, als ihm wieder einfiel, er brauche ja nur durch die Tür zu gehen. Er sah als Erstes eine Menge Bücher in Holzregalen stehen. Die Zwischenwände waren reich verziert mit Holzschnitzereien und Intarsien aus Gold, Alabaster und Adamant. Von der Decke herunter hingen handgetriebene Lüster aus Gold und erhellten mit Kerzen den Raum. Er ging durch einen der Gänge auf die Stimmen zu. Vorsichtig blickte er um die Ecke. Er sah Lucius Malfoy vor einem Schachbrett sitzen. Die Person ihm gegenüber konnte er nicht erkennen. Neben ihm saß Draco und schaute den beiden zu. Mister Malfoy hatte die Ellenbogen auf dem wertvollen Tisch, die Hände ineinander verschränkt und legte seinen Kopf darauf ab. Nachdenklich betrachtete er das Brett. Dann nahm er eine Figur und bewegte sie. \enquote{Schach}, sagte er. Dann drückte er die Spieluhr, damit die Zeit für seinen Spielpartner ablief.

Dieser hatte, so sah Harry, nachdem er einige Schritte auf die beiden zugemacht hatte, ebenfalls seine Ellenbogen auf dem Tisch, seine Finger aber auf dem Tisch liegend verschränkt. Irgendwie kamen die Finger Harry bekannt vor. Er wusste nicht, wo er sie hintun sollte. Er war sich unsicher, ob er sich noch weiter bewegen sollte. Bisher war er auf niemanden getroffen und konnte durch Türen gehen. Aber was war, wenn er sich das nur eingebildet hatte und tatsächlich hier war. Im Hause der Malfoys. Ohne Zauberstab. Hier würde er niemals mehr lebend herauskommen.

Draco Malfoy schaute nun in seine Richtung. \enquote{Mum?}, fragte er.

Harry drehte sich erschrocken um. Wenige Millimeter hinter ihm stand Dracos Mutter.

\enquote{Kommst du bitte Draco? Deine Schwester ist im Wohnzimmer und möchte nicht alleine sein, wenn Tante Bellatrix kommt.}

Draco nickte und stand auf. Er ging zu seiner Mutter und verließ den Raum. Die Tür ließ er offen.

\enquote{Schach matt}, hörte Harry jetzt und das abermalige Klicken der Uhr. Diese Stimme kannte er genau. Er drehte sich herum und lief auf ihn zu. Er stellte sich an das Spielbrett und blickte seinem Lehrer in die Augen. Dieser sah Lucius an und meinte: \enquote{Du wirst besser, Lucius. Aber noch hast du mich nicht \gst noch eine Runde, oder lassen wir es für heute gut sein?}

Er sah seinem Lehrer für \VgddK in die Augen. Er konnte es nicht fassen, dass er hier in aller Ruhe saß und mit dem Vater seines schlimmsten Feindes Schach spielte.

\enquote{Lass es gut sein, Frederick. Wir sehen uns nächsten Monat.}

Elber nickte und meinte dann: \enquote{Du kannst dir das Spiel wie immer nochmals anschauen. Letztes Mal war eine Ausnahme, weil ich was versuchen wollte, das du noch nicht mitbekommen solltest. Ich werde es dir nächstes Mal zeigen.} Er wollte gerade aufstehen, als er einen fürchterlichen Schrei hörte.

\enquote{Tamara.} Sofort rannte Professor Elber aus der Bibliothek. Harry ihm hinterher. Als Professor Elber abrupt stehen blieb, rannte Harry förmlich durch ihn durch.

Er zog seinen Zauberstab und entwaffnete Bellatrix, die ihren Zauberstab auf ihre Nichte gerichtet hatte und diese gerade mit dem Cruciatus-Fluch belegte. Nachdem sie ihren Zauberstab verloren hatte, verstummten auch die Schreie der kleinen Tamara.

\enquote{Bist du verrückt geworden, Bellatrix? Deine eigene Nichte mit dem Cruciatus-Fluch zu belegen? Sie gehört zu deiner Familie. So etwas hätte ich dir nicht zugetraut. Zwar vieles, aber nicht das.}

\enquote{Sie ist eine Gryffindor und außerdem mit den Weasleys und diesem Potter befreundet. Sie hat es verdient}, spie sie heraus. Tamaras Mutter und Draco standen ganz verstört neben ihr. Dann ging Draco zu seiner kleinen Schwester hin und tröstete sie. Harry wusste, dass Draco nichts gegen seine Tante ausrichten konnte. Und ihre Mutter hatte wohl zu viel Angst, als dass sie ihrer Schwester Einhalt gebieten könnte.

Bellatrix wollte schon ihren Zauberstab holen, als sie Professor Elber mit einer gefluchten Ohrfeige davon abhielt. \enquote{Draco, nimm deine kleine Schwester und geh mit in ihr Zimmer. Wartet dort auf mich, bis ich komme.}

Draco nickte und trug seine noch immer zitternde und weinende Schwester hinaus. \enquote{Warte draußen, Narcissa. Das jetzt willst du nicht sehen.}

Er blickte ihr kurz aber intensiv in die Augen. Man konnte deutlich Angst in ihrem Blick lesen. Aber nicht die Angst vor ihrer Schwester, sondern Angst vor dem, was mit ihr gleich passieren würde. \enquote{Und}, fügte Elber hinzu, \enquote{leg' einen Schallschutzzauber über diesen Raum.}

Nachdem die Tür geschlossen wurde, steckte Elber seinen Zauberstab wieder ein und ging auf Bellatrix zu. \enquote{Wie kommst du dazu}, fragte er in einem normalen Tonfall, \enquote{deine Nichte zu foltern. Und erzähl mir nicht das Märchen von wegen falscher Umgang, falsches Haus. Warum hast du es wirklich gemacht? Wollte sie nicht dem Dunklen Lord dienen?}

Bei diesen Worten legte sich ein Schauer über Harrys Rücken. Er hatte seinen Lehrer Voldemort noch nie \accentuate{Den Dunklen Lord} nennen hören. \gedanke{War er ein Todesser?}, durchdrang ihn eine Überlegung.

Bellatrix schnappte sich ihren Zauberstab, als Elber gerade durchs Fenster sah und belegte ihn mit einem Fluch. Das hieß, sie versuchte es. Er schrie kurz vor Schmerzen auf, warf ihn aber zurück. So etwas hatte Harry noch nie gesehen. Der Zauberstab flog Bellatrix abermals aus der Hand.

Nun wurde Harry zunehmend unwohl. Denn was er jetzt sah, verschlug ihm die Sprache. Ihm wurde heiß und gleichzeitig kalt. Er wusste nicht mehr, ob er schwitzte, oder fror.

Die Haare seines Lehrers wurden weiß und er konnte in seinen Augen ein kurzes Funkeln erkennen. Seine Iris wurde rot und die Fingernägel wuchsen um mindestens drei Millimeter. Verunsichert ging Bellatrix einen Schritt zurück. Doch das half ihr nicht die Schmerzen, die gleich kommen würden, zu mildern. Elber streckte seine Hände aus und mit leicht offenem Mund und einem genießenden Blick in den Augen zogen blau-violette Blitze aus seinen Fingern heraus und direkt in Bellatrix' Körper. Diese schrie vor Schmerzen, fiel zu Boden und krümmte sich in eine Fötus-Haltung ein. Ihre Augen waren weit aufgerissen und Harry hatte den Eindruck, dass dies weitaus schlimmer als der Cruciatus-Fluch sein musste. Sie wechselte in eine Haltung mit gekrümmten Rücken und wieder zurück. Er hatte zwar keinen Geruchssinn, aber an den aufsteigenden Dämpfen und der Verfärbung ihrer Hände und ihres Gesichtes musste wohl ihr Fleisch an manchen Stellen sehr warm werden und teilweise sogar verbrennen.

Nach etwa zehn Sekunden beendete Elber seinen \gst was auch immer er da gerade getan hatte \gst und ging auf Bellatrix zu. \enquote{Jeder schmerzhafte Fluch, den dir der Dunkle Lord nun schenkt, wird wie ein Kitzeln für dich sein. Das war deine Strafe dafür, dass du deine Nichte gefoltert hast, Bellatrix. Ich sagte dir schon einmal, reize mich nicht.} Er zog wieder seinen Zauberstab und ließ sie wie eine leblose Puppe schweben, dann sich in der Luft drehen und setzte sie anschließend in einem Sessel ab. Dann sprach er mehrere Zauber, die Harry weder kannte, noch verstand. Die Hautrötungen verschwanden und auch der Dampf, welcher immer noch vereinzelt durch ihre Kleidung drang, verschwand. \enquote{Was bleibt, ist die Erinnerung, Bellatrix. Sonst nichts.}

Dann steckte er wieder seinen Zauberstab ein und legte eine Hand auf ihre Stirn. Sie erwachte, hatte aber keinen irren Blick mehr an sich. Sie wirkte irgendwie normal. Er gab ihr einen kurzen Kuss auf die Stirn und meinte: \enquote{Du hast jetzt eine halbe Stunde für dich, vorher wacht Sie nicht auf.} Dann drehte er sich um und verließ den Raum.

Harry sah Bellatrix nur an. Sie saß da, nahm sich ein Buch, das auf dem Tisch neben ihr lag und schlug es auf. Verträumt sah sie nun kurz in Harrys Richtung. Es schien, dass sie ihn betrachtete, aber ihr Blick war scheinbar abwesend. Dann widmete sie sich wieder ihrem Buch.

Ihr Verhalten war so merkwürdig, dass Harry sich keinen Reim darauf machen konnte. Er drehte sich schnell um und verließ mit langen Schritten den Raum, um nach seinem Lehrer Ausschau zu halten. Dieser unterhielt sich im nach oben gehen mit Dracos Mutter. \enquote{Sie braucht jetzt noch Ruhe, gib ihr eine halbe Stunde. Ich habe dafür gesorgt, dass keiner zu ihr kann.}

Dann waren die drei oben angekommen. Elber betrat Tamaras Zimmer und meinte: \enquote{Wir packen jetzt deine und Dracos Sachen, ihr kommt mit zu mir.}

Dracos und Tamaras Mutter lief eine einzelne Träne über ihr Gesicht.

\enquote{Mama!}, sagte Tamara. \enquote{Wir müssen gehen?}

\enquote{Es ist besser für euch. Bei Frederick seid ihr besser aufgehoben als hier. Ich kann euch nicht beschützen. Seht mich an, ich habe es nicht einmal geschafft, mich vor euch zu stellen, als meine Schwester dich\abs} doch Narcissa stockte. Sie konnte es nicht aussprechen.

\enquote{\aabs gefoltert hatte}, beendete ihre Tochter den Satz. Nun liefen noch mehr Tränen an Narcissa herunter.

Elber drehte sich zu Narcissa um, nahm sie kurz in den Arm und sagte dann zu ihr: \enquote{Du kannst jederzeit zu ihnen kommen. Du weißt, wo ich bin. Und du kannst auch ohne Zauberstab apparieren. Und wenn du es nicht mehr bewältigst, dann kommst du ganz. Du weißt, dass sie dich deswegen nicht foltern können?}

Sie nickte stumm und verließ den Raum. Es war alles zu viel für sie.

Harry musste schlucken. Jetzt verstand er Tamara, als sie sagte: \enquote{Draco hat immer weniger Freunde, beziehungsweise keine Freunde, und auch seine Familie macht ihm zu schaffen.}

\enquote{So}, sagte Harrys Lehrer und schlug vergnügt die Hände zusammen. \enquote{Wir packen! Das heißt, ich packe und ihr schaut zu.}

Abgelenkt und dadurch freudig, schaute Tamara ihn an und sprang vom Bett auf ihn zu. \enquote{Danke Frederick.}

\enquote{Aber für meine Patentochter mache ich das doch gerne.} Nachdem sie sich gedrückt hatten, hob Elber die Hände, schlug sie über dem Kopf zusammen, zog seinen Zauberstab und schwang ihn kunstvoll durch die Luft. In der Mitte des Raumes bildete sich eine kleine Kugel und sog das ganze Inventar samt Spielsachen und alles was sich im Raum befand ein. Er nahm die Kugel in die Hand und gab sie Tamara. \enquote{Jetzt machen wir das in Dracos Zimmer auch noch und dann apparieren wir zu mir.}

Vor Harry begann sich die Szene langsam aufzulösen. Er hatte den Eindruck, als würde er einschlafen.

Um drei Uhr morgens wachte er auf. Er erinnerte sich vage an seinen Traum. Er richtete sich im Bett auf, zog die Beine an und schlug seine Arme darum. Dann legte er seinen Kopf auf seine Knie und dachte angespannt nach. Er begann sich nur langsam zu erinnern. Er sah eine Person mit Lucius Malfoy in dessen Haus Schach spielen, sah, wie Draco die beiden beobachtete und hörte einen Schrei. Kurz darauf hatte er auch schon Tamara gesehen, zusammen gekrümmt am Boden. Draco nahm seine Schwester und verschwand. Dann lag Bellatrix, Tamaras Tante, am Boden und hatte Verbrennungen. Dann stand der Mann, der mit Lucius Schach gespielt hatte in Tamaras Zimmer, verließ es mit Draco und ihr und einer kleinen Kugel, die Tamara hielt und ging in Dracos Zimmer. Kurz darauf kamen die drei mit einer weiteren Kugel auf den Gang. Draco und Tamara hielten sich fest und die drei disapparierten.

\gedanke{Wenn ich doch nur das Gesicht wieder sehen könnte}, dachte sich Harry. \gedanke{Irgendwoher kenne ich die Person. Ich weiß nur nicht woher.}

Seine Hoden schmerzten, da sich seine Prellung bemerkbar machte. Er legte sich wieder zurück in sein Bett, zog die Bettdecke zu und wollte noch etwas darüber nachdenken. Doch schon schlummerte er wieder ein und vergaß weitere Details seines Traumes \gst oder war es eine Vision?

Am nächsten Morgen wurde es Zeit die Geschenke auszupacken. In der Mitte des Gemeinschaftsraumes stand ein raumhoher aber schmaler festlich geschmückter Weihnachtsbaum, der wie seine großen Brüder in der Großen Halle geschmückt war. Die restlichen verbliebenen Gryffindors packten ihre Geschenke aus und auch Hermines Eltern hatten für Harry eine Kleinigkeit eingepackt, die sie unten in Hogsmeade erstanden hatten. Es war ein \accentuate{sich Verändernder}, ein kleiner Ring, der sich der Größe seines Benutzers anpasste. Harry steckte ihn an seinen Finger und er verschmolz fast mit ihm. Harry konnte ihn kaum noch sehen und versuchte ihn wieder herunter zu bekommen. Dies gelang ihm auch gleich, worauf hin sich der Ring wieder in seine Ausgangsform zurückverwandelte.

\enquote{Was ist das?}, fragte er.

\enquote{Das}, antwortete Hermines Vater, \enquote{ist ein sogenannter \enquote{sich Verändernder}. Der Ring passt sich seinem Träger an und hat bei jedem eine andere Form und Gestalt. Er warnt einen vor Feinden und anderen komischen Gestalten.}

\enquote{Wir haben für uns auch gleich welche mitgenommen}, fügte Hermines Mutter hinzu. \enquote{Man weiß nie, wann man so etwas gebrauchen kann}, und lachte dabei.

Aus Harry kam nur ein knappes \enquote{Danke} heraus. Er hatte nicht erwartet, von Hermines Eltern etwas zu bekommen. Umso größer war die Überraschung.

Hermine packte noch ein weiteres Buch aus. Harry staunte nicht schlecht, da es über dunkle Künste handelte.

\enquote{Woher habt ihr denn das bekommen?}, fragte Hermine nach. \enquote{Ich glaube nicht, dass es das in Hogsmeade gab, oder bei \fab.}

\enquote{Wir haben es aus der Nokturngasse. Wir wussten nicht genau, was wir dir dieses Jahr schenken sollten, also haben wir deinen Lehrer gefragt. Diesen Professor Elber, als er uns unten in Hogsmeade empfing.}

\enquote{Wie seit ihr dahin\abs ?}, fragte Hermine weiter.

\enquote{Mit Flohpulver in den tropfenden Kessel und zurück.}

Hermines Augen weiteten sich und sie fiel ihren Eltern um den Hals. Auch die anderen hatten in der Zwischenzeit ihre Geschenke ausgepackt und unterhielten sich mit ihren Verwandten.

Plötzlich viel Harry noch ein weiteres Geschenk auf, welches in der Ecke lag. \enquote{Fehlt noch jemand ein Geschenk?}, rief Harry in die Runde.

Doch jeder verneinte und so stand Harry auf, um sich das Geschenk näher anzusehen. Er nahm es von der Wand an der lehnte und drehte es um. Es sah aus, wie ein Tisch ohne Beine. Jetzt entdeckte er ein Namensschild mit seinem Namen. Erstaunt öffnete er sein letztes Geschenk und zum Vorschein kam nur eine hölzerne Tischplatte und ein kleiner Brief. Die umstehenden und herum sitzenden Gryffindors machten nur komische Gesichter und fragten sich, wer Harry wohl so etwas schenken sollte. Harry konnte sich auch keinen Reim darauf machen und so öffnete er den beigelegten Brief.

\begin{brief}
Lieber Harry,

Ich wünsche dir frohe Weihnachten und alles Gute im neuen Jahr. Als kleines, oder besser gesagt großes, Weihnachtsgeschenk, habe ich dir das Mini-Quidditch-Spiel eingepackt. Viel Spaß beim Trainieren und pass' mir gut auf das Spiel auf, es ist wie du weißt sehr alt. Damit gehört es jetzt offiziell dir.
\signumspace
Grüße Arabella.
\end{brief}


\enquote{Es ist von Arabella}, sagte Harry. \enquote{Es ist das Mini-Quidditch-Spiel, welches wir bei meinem Geburtstag gespielt haben.}

Jetzt wurde die Aufmerksamkeit der anderen geweckt. Harry nahm das Spiel aus der aufgerissenen Verpackung heraus und legte es auf den Boden. Er freute sich, denn nun hatte er etwas für seine Mannschaft mit dem sie trainieren und Spielzüge planen konnten.

\enquote{Wie spielt man das?}, fragte Dean, der sich jetzt näherte.

Harry drehte seinen Kopf zu ihm und meinte nur: \enquote{Komm her und setz dich, dann erkläre ich es dir.}

Schnell waren vierzehn Leute für ein Spiel zusammen und Harry begann das Spiel.

Er zog seinen Zauberstab aus seinem Umhang und berührte mit der Spitze seines Zauberstabes das Spielbrett. \enquote{Spiel beginnen}, sagte er laut und deutlich.

Wie schon beim letzten Mal begannen sich zuerst die Stangen und danach die Ringe sowie die Tribünen aus dem Spielbrett herauszubilden.

\enquote{Wahnsinn}, hallte es durch den Gemeinschaftsraum.

Harry fing jetzt an zu erklären. \enquote{Jeder Spieler sucht sich eine Mannschaft und eine Position aus und sagt diese laut und deutlich. Ich führe es euch mal vor.} Er sagte: \enquote{Treiber, Panther von Loch Lumen.} Kurz darauf erschien auf dem Spielbrett eine kleine Figur, welche Harry sehr ähnlich sah. Sie trug die Uniform der Panther von Loch Lumen. \enquote{Ihr steuert eure Figuren mit Hilfe eurer Gedanken. Am Anfang braucht es ein paar Minuten, bis eure Figur auf euch reagiert, aber dann geht es mühelos.}

Alle Spieler sprachen nun durcheinander ihre Position und die Mannschaft. Harry verstand kein Wort, aber das Spielbrett hatte offenbar alles genau verstanden. Harry ließ jetzt seine Figur etwas steigen, um den anderen auf dem Spielbrett etwas mehr Platz zu geben und gleichzeitig zu zeigen, dass die Figur ohne Worte zu steuern war. Nach einigen Minuten bewegten sich alle Figuren in der Luft über dem Spielbrett. Hermine machte den Schiedsrichter und blies in die Pfeife, welche ebenfalls auf dem Spielbrett erschienen war. Jetzt gingen wieder die Klatscher in die Höhe und Harry gab seiner Figur zu verstehen, den Klatscher auf eine Figur der gegnerischen Mannschaft zu treiben. Einige Figuren fielen immer wieder von ihren Besen, oder konnten sich gerade noch so festhalten. Harry fand sich in der Rolle des Treibers gar nicht so schlecht. So konnte er alle Positionen einmal spielen und im Falle eines Falles jemanden ersetzen, bzw. vertreten.

Später, als es wieder Zeit für das Abendessen war, flanierte Harry mit Hermines Vater durch das Schloss. Unterwegs begegnete er dem kleinen Hufflepuff mit den Narben im Gesicht und dem einzelnen Auge. Mittlerweile hatten sich fast alle an diesen Anblick gewöhnt. Harry lief also mit  Mister Granger über das Gelände. Er wollte sich einmal mit ihm alleine Unterhalten. Harry sah immer wieder, wie sich einige Eltern mit Professor Dumbledore oder einem der anderen Lehrer unterhielten. Er musste grinsen und dachte an die Elternsprechtage in seiner Schule, als er noch mit Onkel Vernon und Tante Petunia zum Vorsprechen musste. Es war wohl für viele Eltern normal, die zum ersten Male ein magisch begabtes Kind zur Schule schickten, dass sie sich direkt mit einem Lehrer unterhielten. Im Gegensatz zu den sonst wohl üblichen Unterredungen an einer normalen Schule.

Als am nächsten Tag der Zeitpunkt der Abreise immer näher kam, begleiteten alle Schüler ihre Besucher nach Hogsmeade, um ihnen eine gute Reise zu wünschen. Die Plattformen erschienen wieder und die Besucher stiegen darauf, nur um kurz danach mit einem leisen \geraeusch{Fzzz} zu verschwinden.

Tags darauf war wieder einmal Samstag und Harry lotste Ron und Hermine nach dem Frühstück in den dritten Stock im Westflügel. Dort wartete bereits Luna, die noch an ihrem Nachtisch des Vortages kaute, welchen sie sich mitgenommen haben musste.

Ron bekam große Augen, als er Luna sah, und noch größere als Harry sie zur Begrüßung küsste. \enquote{Was? Ihr beide seid zusammen?}, fragte Ron.

\enquote{Ja}, antworteten beide unisono.

\enquote{Und nachher wird es wohl auch für den Rest der Schule offiziell gemacht werden. Bisher sind das ja nur Vermutungen und Spekulationen}, sagte Harry.

\enquote{Und um uns das zu sagen, mussten wir hierherkommen?}, fragte Ron.

\enquote{Nein}, antwortete Harry. \enquote{Ich weiß seit vorgestern Abend, dass du und Hermine ein Paar seid und \gst}

Doch Ron unterbrach ihn. \enquote{Du hast es ihm erzählt?}

\enquote{Ich konnte nicht anders. Ich war in einer misslichen Situation und \gst}, meinte Hermine.

\enquote{Missliche Situation?}, sagte Ron leicht indigniert. \enquote{Missliche\abs}, doch er wusste nicht mehr weiter.

\enquote{Ron, beruhige dich}, meinte Harry. \enquote{Was ich euch beiden jetzt zeigen werde, lässt dich schnell vergessen, dass Hermine wohl etwas zu früh geplaudert hat.}

\enquote{Na hoffentlich}, sagte Ron barsch.

\enquote{Aber du musst noch eines wissen. Nein, eigentlich sind es doch eher zwei Dinge. Erstens. Kein Wort zu Mister Filch, einem der Lehrer oder Dumbledore. Kein Wort zu irgendjemand außer einem der vielen anderen Pärchen der sechsten oder siebten Stufe. Klar?}, schloss Harry.

Ron sah Harry leicht komisch an: \enquote{Klar!}

\enquote{Und das Zweite! Fast alle Pärchen der sechsten oder siebten Stufe werden ab jetzt davon erfahren.} Leichte Panik stieg in Ron empor. \enquote{Keiner wird sich etwas anmerken lassen. Selbst Mal\aabs}, doch Harry konnte sich gerade noch bremsen.

\enquote{Selbst wer nicht?}, fragte Ron nach.

Doch Harry ignorierte ihn und meinte nur: \enquote{Versprichst du's?}

\enquote{Also gut. \gst Ja ich verspreche es.}

\enquote{Du auch Hermine?}

\enquote{Ja, ich verspreche es.}

\enquote{Na gut. Alles, was ihr ab jetzt zu sehen und zu hören bekommt, ist Top Secret.}

\enquote{Kommt mit}, und Harry lief mit Luna noch um die letzten Ecken bis sie vor das große Porträt kamen.

\zauber{Aqua Neros!}, sagten Harry und Luna unisono und das große Porträt schwenkte zur Seite.

Harry ließ Ron und Hermine den Vortritt und folgte dann Luna in das Loch. Das Porträt schloss sich und kurz darauf standen die vier im Gemeinschaftsraum der Paare.

Harry beeilte sich, um vor Hermine und Ron zu gelangen, und ihren Gesichtsausdruck aufzuschnappen. Er setzte sich auf ein leeres Sofa, Luna nahm neben ihm Platz und legte ihre Hand in seine. Stumm standen die beiden da und Harry musste schmunzeln. Ein ungewohnter Anblick. Doch als Ron Malfoy und Maria sah, fiel ihm die Kinnlade herunter. Draco konnte nicht anders und ihm entwich eine kleine Gehässigkeit.

\enquote{Aha, Granger und Weasley. Die beiden rötesten im ganzen Raum. Und das nicht nur von den Kopfhaaren her.} Und dann zu Harry gewandt: \enquote{Wird wohl Zeit für die beiden endlich zu Schubbern.}

Hier benahm sich Malfoy zumindest anständiger als draußen, denn ihm wurde wohl klar, dass er hier jede Menge Prügel und Flüche einstecken konnte und nachher keine Erklärung haben würde, woher und wie er die sich eingefangen haben konnte. Mit diesen Worten und Maria bei der Hand verließ er den Raum. Ron, immer noch sprachlos, sah nur Harry und Luna an.

\enquote{Dies, ist der Gemeinschaftsraum der Paare}, sagte Luna. \enquote{Nun gehört ihr auch dazu.}

Harry und Luna gaben Ron und Hermine zu verstehen, sie mögen sich doch bitte setzen, und zogen dann das gleiche Programm wie bei den anderen durch. Anschließend zeigten sie den beiden ihren Raum (es stand bereits ihr Name darauf) und zogen sich dann wieder in den großen Gemeinschaftsraum zurück. Es dauerte eine Weile, bis Hermine und Ron zurückkamen. Harry und Luna spielten gerade eine Runde Schach und dem aufmerksamen Zuschauer dürfte nicht entfallen sein, dass sich Harrys Spiel verbessert hatte.

\enquote{Aha, das ist also deine Übungspartnerin}, meinte Ron lapidar und setzte sich.

\enquote{Wieso sind hier überall nur Doppelbetten?}, fragte Hermine.

\enquote{Dies ist doch der Gemeinschaftsraum der Paare}, meinte Luna nur.

\enquote{Es ist ja nicht so, dass ihr unbedingt jedes Mal wenn ihr hier seid miteinander schlafen müsst. Die ersten paar Male lagen wir einfach nur nebeneinander und haben geschlafen.}

Hermine verzog leicht ihren Mund und machte ein Gesicht, das Harry sagte: \gedanke{Und dann habt ihr wohl miteinander geschlafen.}

Harry konnte nicht anders und meinte nur: \enquote{Genau.}

Hermine erschrak, so als hätte Harry ihre Gedanken gelesen.

\enquote{Hast du?}, fragte Hermine.

Harry antwortete nur: \enquote{Was genau? Deine Gedanken gelesen oder mit Luna geschlafen?}

Währenddessen öffnete sich das Porträt und weitere Paare kamen herein.

\enquote{Ach schau an. Mister Weasley und Miss Granger. Hab's doch gewusst. Seit wann sind die denn hier, Harry?}, fragte Donan.

\enquote{Seit etwa einer halben Stunde}, antwortete Harry.

\enquote{Tja, unserem Harry entgeht eben nichts}, grinste er und verschwand mit seiner Freundin um die Ecke, um sich zu duschen. Die beiden sahen nämlich sehr verschwitzt aus.

Die nächsten Tage war in der gesamten Schule nur vom Besuch der Verwandtschaft die Rede. Die Hausaufgaben wurden etwas vernachlässigt, da über den Jahreswechsel schulfrei war. Luna und Harry zeigten sich jetzt offiziell als Paar in der Schule und versteckten sich nicht länger. Die erste Zeit gab es einiges an Getuschel, welches sich aber bald wieder im Schulalltag verlor.

Harry las sich noch schnell die wenigen Seiten des schmalen Buches durch. Dann kopierte er sie, da er das Buch wieder abgeben musste. Er musste endlich den Test mit seinem Umhang machen.

Diese beiden Tage unterhielten sich Philips Eltern mit Professor Elber und Madame Pomfrey immer wieder und auch mit ihrem Sohn. Es dauerte eine Weile, bis sich die zwei für ihren Sohn entschieden, auch wenn sich wohl die restliche Verwandtschaft dagegen stellen würde. Seine Familie musste nicht an Hunger leiden und war gesellschaftlich recht gut angesehen. Man kam überein, dass die beiden es versuchen sollten. Demnächst sollte es losgehen.

Während dieser Zeit machte sich Harry daran, einen Trank zu brauen, der ihm die Gewissheit bringen sollte, ob er wirklich von den Peverells abstammte. Er stand also im Tränkelabor und gab gerade ein Haar zu. Dann schäumte es in seinem Kessel. Mit einer flachen Kelle schöpfte er den Schaum ab und warf ihn in das Waschbecken daneben. Dann, als der Trank sich wieder beruhigt hatte, gab er einen Tropfen Blut hinzu. Nach einigen Minuten des leise köcheln Lassens, siebte er ihn in einen Becher, belegte diesen mit einem Warmhaltezauber und verschloss den Becher selber magisch, damit er ihn nicht aus Versehen verschütten konnte.

In seinem Zimmer angekommen; er hatte seinen Arbeitsplatz sauber verlassen; holte er seinen Tarnumhang hervor und trank einen Schluck aus dem Becher. Den Rest goss er über den auf dem Boden ausgebreiteten Mantel. Nach einer knappen Minuten fingen sein Umhang, sowie er selbst, rot zu leuchten an. Dieses Leuchten hielt eine gute Minute. Dann verblasste es. Er schaute noch einmal in seinen Unterlagen nach, um sicherzugehen.

Jetzt hatte er Gewissheit. Er war einer der Nachfahren der Peverells. \gedanke{Nein}, korrigierte er sich selbst. \gedanke{Ich bin der Nachfahre einer der Peverells. Die beiden anderen Brüder hatten keine Nachkommen.} Er wusste nun etwas mehr über seine Vergangenheit. Aber es sagte ihm noch immer nichts Genaues.

\trenn

Morgens, kurz vor dem Frühstück, Hermine und Harry saßen bereits im Gemeinschaftsraum und warteten auf Ron und Ginny, kam Tamara Malfoy die Treppe herunter und fragte: \enquote{Harry, Hermine. Ich brauche eure Hilfe.}

Ron trat in den Raum und setzte sich auf die andere Seite des Sofas, einen Platz zwischen sich und Harry lassend. \enquote{Wobei?}, fragte er.

\enquote{Ihr müsst mir aber versprechen, nicht sauer oder wütend auf mich zu sein}, fügte sie ängstlich hinzu.

Die drei nickten und so fuhr Tamara mit ihrer Bitte fort. \enquote{Ihr kennt doch meinen Bruder!}

\enquote{Den Arsch}, bemerkte Ron.

Hermine trat ihm gegen sein Schienbein. \enquote{Ron!}

\enquote{Ok, ok.}

\enquote{Schon in Ordnung. Ich weiß ja, wie er auf andere wirkt.} Sie setzte sich zwischen Harry und Ron. Beschämt betrachtete sie ihre Finger. \enquote{Er ist eigentlich ganz lieb. Nur wenn er unter Leuten ist, dann muss er eine Maske aufsetzen. Wegen seines Vaters. Aber sagt das bitte niemandem weiter.} Die drei nickten etwas ungläubig. \enquote{Seit wir nicht mehr Zuhause wohnen, verliert er zunehmend den Halt. Seine eigenen Mitschüler aus seinem Haus distanzieren sich von ihm.}

\enquote{Aber er hat doch noch Crabbe und Goyle.}

\enquote{Die zählen nicht}, sagte Tamara nun zu Harry gewandt. \enquote{Die sind wie Hunde für ihn. Laufen ihm nach, wohin er auch geht. Aber selbst die beiden fangen an, sich von ihm abzukapseln.}

\enquote{Er hat's auch verdient \gst Au \gst Hermine.} Sie war ihm wieder gegen sein Schienbein getreten.

Jetzt stand Ron auf und setzte sich in einen Sessel neben Harry. \enquote{So kannst du mir wenigstens nicht mein Bein kaputt treten.}

Beide wechselten eigenartige Blicke, fiel Harry auf.

\enquote{Harry, ich möchte dich, und auch euch Ron und Hermine, nun um etwas bitten. Aber seid bitte nicht wütend auf mich.} Sie sah Harry mit klimpernden Wimpern an. Harry konnte diesem Blick nicht lange widerstehen, das hatte Tamara recht bald gelernt. Er nahm sie brüderlich in den Arm und wuschelte durch ihr Haar. \enquote{Harry}, sagte sie schüchtern.

\enquote{Ich versprech's.}

Dann sah sie Ron und Hermine an. Gerade als Hermine Ron über Tamara und Harry hinweg in die Seite knuffen wollte, versprach er es ebenfalls und auch Hermine stimmte zu.

\enquote{Ich möchte von euch, dass ihr beginnt, euch mit meinem Bruder anzufreunden.}

Harry musste einen Hustenanfall unterdrücken. Ron und Hermine schauten sie fragend an.

\enquote{Tamara}, sagte Hermine schließlich. \enquote{Wir verstehen uns mit deinem Bruder nicht sonderlich gut. Seit unserem ersten Tag hier, sind wir so etwas wie verfeindet}, sagte Hermine.

\enquote{Und aus diesem Grund sollt ihr anfangen, euch mit ihm anzufreunden. Wenn ihr das macht, dann gewinnt er vielleicht auch wieder Freunde aus seinem Haus}, sagte sie mit leicht feucht werdenden Augen.

\enquote{So einfach wie du dir das vorstellst, ist das aber nicht. Im Leben kann man sich nicht alles wünschen}, antwortete Hermine.

\enquote{Das weiß ich}, gab sie leicht verärgert zurück. \enquote{Nur weil ich eine Malfoy bin, heißt das noch lange nicht, dass ich nichts vom Leben weiß.}

\enquote{Vielleicht liegt es auch nur daran, dass ich ihn gekränkt habe}, antwortete Harry sanft.

\enquote{Wie das?}, fragte sie.

Harry überlegte kurz und sagte dann: \enquote{Ich erzähle dir die Geschichte. Aber sage sie nicht deinem Bruder, verstanden?}

Tamara nickte.

Dann nahm Harry sie auf seinen Schoß. Sie saß nun auf seinen Oberschenkeln, die Beine Richtung Rückenlehne; leicht angewinkelt. Harry hielt sie an ihrer Hüfte fest. Ihre Augen waren leicht feucht. Dahinter konnte er ihre strahlend dunkelgrauen Augen erkennen. Ihr fast schon goldenes Haar fiel über ihre Schulter und hörte in der Mitte ihrer Schulterblätter auf.

Mittlerweile war es im Gemeinschaftsraum leise geworden. Diejenigen, die noch da waren, hatten unterbrochen, was sie taten, um Harry zuzuhören.

\enquote{Vor meinem elften Geburtstag wusste ich nichts von Hexen oder Zauberern. Ich wusste nichts von der magischen Welt, deiner Welt \gst die mittlerweile auch zu meiner Welt geworden ist. Als ich meinen ersten Brief bekam, hatte ich keine Gelegenheit ihn zu lesen. Mein Onkel wunderte sich darüber, wer mir wohl schreiben würde. Gehässig wie immer, warf er ihn in den Kamin, als er das Wappen auf der Rückseite erkannte.} Er schluckte kurz. \enquote{Und jeden weiteren Brief, den die Eulen brachten. Er vernagelte sogar den Briefschlitz in unserer Haustür. Schließlich, als nichts half, sind wir kurz vor meinem Geburtstag\abs} Er unterbrach sich kurz und schaute ihr nun mit leicht schrägem Kopf in die Augen. \enquote{Weißt du, wann ich habe?}, fragte er sie.

Sie nickte. \enquote{Ende Juli}, sagte sie.

\enquote{Am 31.}, antwortete Harry, \enquote{sind wir, das heißt mein Onkel, meine Tante, mein Cousin und ich, in einen alten verlassenen Leuchtturm gezogen, damit die Briefe mich nicht finden würden. Es war am Tag vor meinem Geburtstag, als ich abends eine Torte in den staubigen Boden malte und mir kurz nach Mitternacht was wünschte, die Augen schloss und die imaginären Kerzen ausblies. Es war eine stürmische Nacht und plötzlich schlug etwas mit aller Gewalt gegen die Tür. Sie flog auf und ein großer, bärtiger Mann stand im Türrahmen. Er trat herein und schloss hinter sich die Tür.} Tamara schaute ihn mit entsetztem Blick an. \enquote{Der Riese\abs eigentlich Hagrid\abs kam auf mich zu.} Er spürte, wie sie ausatmete und ihre Anspannung verlor. \enquote{Er erzählte mir etwas über meine Eltern, dass sie nicht, wie ich bis dahin immer geglaubt hatte, bei einem Autounfall ums Leben gekommen seinen, sondern ermordet wurden.} Tamara bekam wieder große Augen. Sie rückte etwas näher an ihn heran. \enquote{Er überreichte mir den Brief von Hogwarts. Einen der Briefe, die ich schon seit langem hätte lesen wollen. Mein Onkel und meine Tante getrauten sich nicht, etwas zu sagen.}

Selbst Ron und Hermine, nebst Ginny, die sich kurz nachdem Harry zu erzählen begonnen hatte, setzte, lauschten gespannt Harrys Worten, denn bisher hatte er auch ihnen nichts Genaues darüber berichtet.

\enquote{Also las ich ihn. Ich bin dann am nächsten Tag mit Hagrid einkaufen gegangen. Du kannst dir das Gefühl gar nicht vorstellen. Doch ich musste noch einen Monat zurück zu meinen Verwandten. Dann, am Tag der Abreise, fuhr mich mein Onkel widerwillig zum Bahnhof und ließ mich zwischen Gleis neun und Gleis zehn stehen und sagte: \inner{Hier Junge, such dein Gleis. Viel Spaß.} Dann drehte er sich um und ging. Tja, nun stand ich da. Verloren und einsam. Wie in einem bösen Traum. Ich ging auf dem Bahnsteig entlang, bis ich ein Wort hörte: \inner{Muggel.} Sofort drehte ich mich zu der Frau hin und lief der Gruppe hinterher, die sie begleitete. Dann sah ich zwei Jungs in einer Mauer verschwinden. Ich ging zu der Frau hin und wollte wissen, wie man zum Gleis kommt. Sie erklärte es mir und ich rannte durch die Mauer und sah den Zug. Erleichtert fiel mir ein Stein vom Herzen.}

\enquote{Wer war die Frau?}, wollte Tamara wissen.

Harry sah zu Ron und antwortete: \enquote{Rons Mutter.}

Tamara drehte sich kurz um zu Ron, danach wieder zu Harry. Gedankenverloren sah er seinen besten Freund an. Bis ihn Tamara stupst und meinte: \enquote{Erzähl weiter.}

Harry sah Tamara wieder in die Augen. \enquote{Ich saß also im Zug in einem leeren Abteil und konnte mein Glück gar nicht fassen. Ein Jahr lang ohne Onkel und Tante und Cousin auf dem Weg nach Hogwarts. Dann ging die Abteiltür auf und Ron fragte mich, ob er sich zu mir setzen könne. Ich bejahte und wir stellten uns vor.} Jetzt lächelte Harry ganz leicht. \enquote{Ron war ziemlich aufgeregt, als ich mich vorgestellt habe.}

Tamara wischte sich das noch feuchte Gesicht ab und Harry fuhr mit seiner Geschichte fort.

\enquote{Ron erzählte mir von den vier Häusern in Hogwarts, von guten und von bösen Zauberern und Hexen und\abs dass fast alle schwarzen Magier aus Slytherin kamen. Ich wusste nun genau, dass ich dahin auf keinen Fall wollte. So langsam freundete ich mich mit Ron an. Die Zugfahrt war schließlich lang. Für eine kurze Weile kam Hermine zu uns, verschwand aber recht bald wieder. Dann fuhr der Zug in den Bahnhof ein und Hagrid brachte uns mit den Booten zum Schloss.}

Wieder pausierte Harry kurz.

\enquote{Als wir dann von Professor McGonagall begrüßt wurden und sie uns nochmals kurz verließ, stellte sich mir dein Bruder vor und bot mir seine Freundschaft an.} Tamara bekam große Augen. \enquote{Ron sagte mir, dass seine Familie seit Generationen in Slytherin ist. Das war wohl ein weiterer Grund, warum ich sagte, was ich zu ihm gesagt habe. Ich sagte in etwa: \enquote{Ich suche mir meine Freunde selber aus.} \gst Damit muss ich ihn wohl gekränkt haben. Es war vielleicht das erste Mal in seinem Leben, dass er nicht das bekam, was er wollte. Dann schritten wir in die Große Halle. Ich hatte Angst ohne Ende. Ich wusste nicht, was mich erwartete. Professor McGonagall setzte uns schließlich den sprechenden Hut auf. Draco kam noch vor mir dran. Er wurde nach Slytherin geschickt. Für mich ein Grund mehr, nichts mit ihm zu tun haben zu wollen.}

\enquote{Nur aufgrund des Hauses?}, fragte Tamara ungläubig. Es standen schon wieder Tränen in ihren Augen. Harry nahm ein Taschentuch heraus und trocknete ihr die Tränen. Mittlerweile hatte er den gesamten Gemeinschaftsraum in seinen Bann gezogen.

\enquote{Ich weiß, das war dumm von mir, aber ich hatte vor wenigen Stunden das erste Mal richtigen Kontakt mit der magischen Gemeinschaft. Ich wusste vorher sonst nichts.} Er senkte betrübt seinen Kopf.

Tamara kroch an ihn heran und legte ihren Kopf auf seine Schulter. Dann begann Harry weiterzuerzählen.

\enquote{Schließlich wurde auch mir der sprechende Hut aufgesetzt. Deinen Bruder hatte der Hut ja nach Slytherin geschickt, kaum dass er ihn aufgesetzt hatte. Bei anderen brauchte er ein bis zwei Sekunden, das ging sogar ziemlich schnell.}

Harry pausierte wieder kurz und sah danach an Ron und Hermine vorbei ins Leere.

\enquote{Dann saß ich auf dem Stuhl, den Hut auf meinem Kopf. Und ich hörte den Hut in meinem Geiste sprechen.}

Tamara setzte sich wieder auf und sah ihm in die Augen. Sie versuchte seinen Blick wiederzuerlangen, scheiterte aber.

Harry fuhr nun leiser fort. \enquote{Der sprechende Hut sagte zu mir: \inner{Oh ja, ich erkenne Mut und den Drang sich zu beweisen. Interessant. Im Köpfchen hast du es auch. Aber wo stecke ich dich hin.} Und ich dachte nur, nicht Slytherin. \inner{Nicht Slytherin}, hörte ich dann in meinem Kopf. \inner{Aber Slytherin könnte dir auf dem Weg zu wahrer Größe verhelfen}, sagte der Hut. Und ich dachte nur immer und immer wieder: \inner{Nicht Slytherin.} Dann sagte der Hut: \inner{Also gut, wenn du dir so sicher bist.} Und dann laut für alle hörbar: \enquote{Gryffindor.}}

Jetzt sah er Tamara wieder an. \enquote{Verstehst du? Der Hut gab mir die Wahl und ich habe mich entschieden. Ich hätte auch nach Slytherin gehen können.} Und dann nach einer kleinen Pause noch leiser zu Tamara: \enquote{Genau wie bei dir. Habe ich recht?}

Tamara drehte sich errötend weg von ihm, doch er nahm ihr Kinn sanft zwischen seine Finger und drehte ihren Kopf zu sich. Sie nickte nur stumm. Harry grinste. Danach küsste er ihre Stirn, stand auf und sagte: \enquote{Wird Zeit zu frühstücken.}

Jetzt war sie wieder bei klarem Verstand. Sie stand auf und sagte: \enquote{Warte Harry, ich muss noch was holen.} So schnell wie sie verschwand, so schnell war sie auch wieder da. Sie drückte ihm eine Rolle Pergament in die Hand und meinte: \enquote{Gib das Draco, gleich wenn du in die Große Halle kommst. Sei nett zu ihm und benutze seinen Vornamen. Sag ihm, dass du es in der Bibliothek gefunden hast und es wohl ihm gehöre.}

Harry und seine beiden Freunde sahen die kleine Tamara verständnislos an. \enquote{Du hast es mir versprochen}, sagte sie mit einem untrüglichen Wimpernschlag. Harry verlor die Kraft, nein zu sagen. \enquote{Gut. Ich mach's}, sagte er.

Tamara verließ den Gemeinschaftsraum durch das Porträtloch. Harry stand noch einige Sekunden da, bevor er sich wieder fasste und das Pergament zu lesen begann. \enquote{Das ist gut, was Mal\gst Draco da schreibt.} Und nach einer Weile fügte er hinzu: \enquote{Das ist sogar sehr gut.}

Hermine stellte sich neben ihn, um überflog ebenfalls den Zaubertrankaufsatz. \enquote{Es scheint}, sagte sie, \enquote{dass Draco bei Snape nicht nur sein Liebling ist, sondern er auch für seine Noten ordentlich arbeitet.}

Harry rollte das Pergament wieder zusammen und zog Ron und Hermine wie durch einen Zauber hinter sich her. Wenige Schritte vor der Großen Halle atmete er einmal tief durch und betrat diese. Er stoppte kurz, um nach Draco Ausschau zu halten, und Schritt dann zielsicher auf ihn zu. Dieser merkte es erst, als Harry fast schon hinter ihm stand. \enquote{Mal\gst Draco. Ich glaube, das ist deines. Du hast es gestern in der Bibliothek vergessen. Es war schon spät, also bringe ich es dir erst jetzt.} Er reichte ihm seine Pergamentrolle.

Dieser nahm sie und schaute kurz nach, was es denn sei. \enquote{Woher weißt du, dass es meine ist?}, fragte er ganz erstaunt und ohne Anflug von Hass oder Ärgernis.

\enquote{Ich kenne nach fünf Jahren Schule deine Handschrift eben.} Dann verließ er den Slytherintisch und setzte sich neben Tamara, um zu frühstücken.

Diese lächelte und winkte ihrem großen Bruder zu, nachdem sie sich kurz umgedreht hatte. Draco lächelte zurück.

\trenn

Bevor er an diesem Abend ins Bett ging, nahm Harry sein Amulett in die Hand und wickelt die Kette um seine Finger, damit er sie nicht verlieren konnte. Dann stieg er ins Bett, lies die Vorhänge offen und legte sich hin. Er fühlte die langsam aufkommende Wärme seines Amulettes und dämmerte nach einer halben Stunde weg. Harry träumte nun.

\begin{traum}
Er stand in Godrics Hollow und sah auf das Haus, welches er so oft in seinen Alpträumen gesehen hatte. Es war ein kleines Steinhaus. Zumindest sah es so aus. Er betrat den Vorgarten durch das Gartentor. Es quietschte leicht, als er es öffnete. \gedanke{Da dürfte etwas Öl angebracht sein}, dachte er sich.

Er lief langsam aber sicher den Weg mit den Steinplatten und dem gepflegten Garten Richtung Haustür. Zu seiner linken Seite sah er einen kleinen Obstbaum, er kannte die Sorte nicht, und darunter ein kleines Blumenbeet. Eine einzelne Figur stand zwischen den Blumen. Sie hatte Ähnlichkeit mit einem Gartenzwerg, obwohl sie ganz aus Stein war und keine Mütze aufhatte. Auf seiner rechten Seite sah er mehrere Tannen und Fichten und einen Weg, der hinter das Haus führte. Durch eine lichte Hecke sah er auf das Nachbargrundstück. Mittlerweile hatte er die Tür erreicht und seine Hand am Griff. Er öffnete die Haustür und ging hinein. Die Eingangshalle war klein, aber geschmackvoll eingerichtet. Dort standen mehrere kleine Tischchen mit Blumenvasen, die Pflanzen der Saison enthielten. Schmucke Verzierungen waren über den beiden Türen zu erkennen. Eine war auf der linken Seite in der Wand, eine andere war Harry gegenüber. Rechts von ihm sah er noch eine weitere. Er ging auf die linke Tür zu, da er Stimmen zu hören meinte.

Er nahm den Griff in die Hand, drückte ihn herunter und öffnete die Tür. Die Stimmen waren nun lauter. Auf dem Sofa, das inmitten des Raumes stand und das seitlich zum Kamin aufgestellt wurde, saßen zwei Personen. Harry kamen sie merkwürdig bekannt vor, er konnte sie aber nicht zuordnen. Die beiden sahen ihn an und baten ihn, sich zu setzen. Harry nahm ihnen gegenüber in einem gemütlichen grünen Ledersessel Platz. Das Sofa war ebenfalls in derselben Farbe gehalten. Hinter dem Sofa war ein Fenster, das die Obstbäume zeigte, die Harry bei seinem Gang durch den Garten gesehen hatte.

Langsam dämmerte ihm, wen er vor sich sah. Und mit einem Mal hatte er Gewissheit, denn die Frau sprach ihn an: \enquote{Hallo Harry, schön, dass du hier bist.}

\enquote{Mum, Dad}, war alles, was er hervorbrachte. Und als er sich endlich gefasst hatte, sagte er: \enquote{Ihr seid Tod, ihr könnt nicht hier sein.}

\enquote{Natürlich sind wir Tod, Harry}, sagte sein Vater. \enquote{Du siehst uns nur, weil du träumst und deine Magie es dir ermöglicht. Du hast es also geschafft.}

Mit großen Augen sah er die beiden an. \enquote{Nein\abs ich\abs bin nur\abs Er hat\abs bin nur mit\abs Amulett\abs eingeschlafen\abs wenn ich schlecht schlafen könnte, oder aufgeregt bin. Es würde mich beruhigen.}

\enquote{Dann bist du nur zufällig hier?}, fragte ihn sein Vater.

\enquote{Ja}, antwortete Harry. \enquote{Aber schön, dass ihr da seid.}

\enquote{Nun,} antwortete seine Mutter, \enquote{das hängt vom Standpunkt ab.} Und als sie Harrys fragendes Gesicht sah, fügte sie hinzu: \enquote{Es könnte genauso gutheißen: Schön, dass du da bist.}

Und Harrys Vater sagte: \enquote{Ob wir in deinen Träumen sind, oder du bei uns bist, kann man nicht genau sagen.}

\enquote{Harry, hör mir zu. Wir haben nicht viel Zeit. Es braucht lange, bis die Verbindung aufgebaut ist, und es kostet uns viel Kraft. Du kannst nicht jeden Abend mit uns Kontakt aufnehmen. Es gibt etwas, was wir dir sagen wollen. Wir wissen nur nicht, wie du es auffassen wirst}, sagte seine Mutter.

\enquote{Na ja, ich werde euch keine Vorwürfe machen}, antwortete Harry.

Seine Mutter hielt sich eine Hand auf ihren Bauch und nahm seines Vaters Hand in ihre andere.
\end{traum}



\begin{kommentar}
Während den Weihnachtsferien, an einem Dienstag, ist Elber im Malfoy Manor und spielt gerade mit Lucius Schach, als Bellatrix ihre Nichte zu foltern beginnt. Elber rennt sofort in den Salon und entwaffnet Bellatrix. Dann bestraft er sie, indem er sie mit blau-violetten Blitzen belegt. Jene Blitze, die der Imperator in Krieg der Sterne (Star Wars) auf Luke Skywalker legte. Eine nette kleine Anspielung, wie ich finde.
\end{kommentar}

\begin{kommentar}
Nachdem er Bellatrix in einen Sessel gesetzt hat, berührt er ihre Stirn und meint, dass sie eine halbe Stunde für sich hätte. Ein weiterer kleiner Hinweis darauf, dass Bellatrix eine gespaltene Persönlichkeit hat.
\end{kommentar}
