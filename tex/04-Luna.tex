\chapter{Luna}


Zurück im Schloss ging Harry erst mal ins Badezimmer, um danach mit Ron und Hermine zu Mittag zu essen. Während Ron duschte, sprach Harry einige Worte in seine schwarze Tagebuch-Scheibe. Dann legte er sie auf sein Tagebuch und betrachtete nach einen kurzen aufleuchten was dort geschrieben stand. Er musste morgen noch zu Snape. Der hatte ihm eine Strafarbeit gegeben, weil er wieder einen Trank versaut hatte. Das sollte ihn dazu bringen sich mehr anzustrengen. Harry war stinksauer, doch Hermine lenkte ihn auf dem Weg zur Großen Halle ab.

Während Ron, wie immer sehr begierig aufs Essen, und Hermine, wie immer in einem Buch vertieft, neben Harry saßen, dachte dieser über seine Begegnung mit Luna nach. \gedanke{Ob ich wohl Ron und Hermine etwas davon erzählen sollte?}, dachte er. Nein, das wäre noch zu früh und außerdem wusste er nicht, was sich zwischen ihm und Luna entwickeln würde oder ob sie nur jemanden zum Reden brauchte. Nachdem er mit dem Essen fertig war, gab er Ron und Hermine Bescheid. Er ging zu seinem Gemeinschaftsraum, in dem wenig los war, dann die Treppe hoch und in sein Zimmer, wo er seinen Besen aufbewahrte. Er nahm seine Quidditch-Sachen aus dem Schrank und begab sich hinunter auf das Feld, zog sich um und schwang sich auf seinen Besen. Dann ließ er den Schnatz aus seiner Tasche los, wartete ein paar Sekunden und machte sich auf die Suche. Harry fühlte sich wundervoll; wieder auf seinem Besen zu reiten und den Schnatz zu jagen. Es war herrlich, die sonnige Luft zu berühren, den Wind zu spüren, der ihm ins Gesicht blies. Er zog Schleifen und Kurven, stieg hoch und stürzte sich hinab um den Schnatz zu suchen und zu jagen. Nachdem er ihn mehrmals gefangen hatte und es langsam Abend wurde, packte er seine Sachen zusammen, duschte in den Umkleideräumen auf dem Quidditch-Feld und machte sich auf den Weg zurück ins Schloss. Vor dem Porträt der fetten Dame sagte er routinemäßig das Passwort. Bald würde das Quidditch-Training wieder beginnen.

Das Porträt schwenkte zur Seite und gab den Weg zum Gemeinschaftsraum der Gryffindors frei. Anders als am Morgen war nun viel Betrieb. Die Erstklässler saßen zusammen und einige von ihnen lasen Bücher. Er beschloss erst einmal seinen Besen zu verstauen. Nachdem er in seinem Zimmer angekommen war, den Besen in die Ecke gestellt und seine verschwitzen Quidditch-Klamotten in den Wäschekorb geschmissen hatte, zog er sich für das Abendessen um. Währenddessen herrschte im Gemeinschaftsraum der Gryffindors immer noch geschäftiges Treiben und ein Erstklässler spielte mit einem Sechstklässler gerade Schach.

Harry schaute ihnen eine Weile zu, um sich so die Zeit zu vertreiben bis Ron und Hermine kamen. Nachdem Harry eine viertel Stunde den beiden zugesehen hatte, kamen sie auch schon. Beide hatten ein seltsames Glänzen in den Augen. Sie nahmen ihn mit und gingen wie jeden Tag in die Große Halle zum Abendessen. Dort warf er Luna einen flüchtigen Blick zu, den sie lächelnd erwiderte. Harry setzte sich gegenüber von Hermine, die neben Ron saß. Irgendwie hatte er ein eigenartiges Gefühl während des gesamten Essens. Ron hatte sich noch nie alleine so dicht neben Hermine gesetzt. Näher als all die anderen Male, in denen sie Hermine in ihre Mitte nahmen.

Harry ließ sich nichts anmerken und begann zu essen. Heimlich schaute er immer wieder zu Luna, die seine Blicke erwiderte. Als er gerade auf dem Weg zu seinem Gemeinschaftsraum war, sprach ihn Luna von hinten an.

\enquote{Harry?}

Harry drehte sich um und sagte: \enquote{Ja Luna.}

Luna kam noch ein paar Schritte näher und nahm seine beiden Hände.

\enquote{Ich danke dir, dass du mich heute Morgen gehalten und getröstet hast.}

\enquote{Keine Ursache Luna, das habe ich gerne getan.}

\enquote{Harry Potter, Sir}, kam es plötzlich aus dem Dunklen. Harry erschrak und löste sich sofort von Luna. Still standen beide da und schauten in die Dunkelheit. Dann bewegte sich etwas und ein kleiner grau-häutiger Hauself erschien.

\enquote{Dobby \gst was machst du denn hier?}, fing Harry an, dem es sichtbar unangenehm war mit Luna erwischt worden zu sein.

\enquote{Dobby war gerade am Sauber machen}, sagte Dobby, \enquote{als Dobby Stimmen hörte. Würde Harry Potter und seine Schulkameradin bitte mitkommen. Dobby möchte ihnen etwas zeigen.}

Dobby dreht sich um und verschwand wieder im Dunkel der Gänge. Luna folgte ihm mit ihrem üblichen verträumten Gesichtsausdruck, ohne eine Sekunde darüber nachzudenken, was der kleine Hauself denn nun vorhatte. Nur Harry blieb stehen, nicht wissend was er denn tun solle.

Luna drehte sich kurz um und meinte, \enquote{Komm schon Harry, wird bestimmt lustig werden.}

Also trottete Harry etwas lustlos hinterher. Aber mit jedem Schritt stieg seine Spannung. Sie liefen durch das Treppenhaus und Harry sah zum ersten Mal, wie Dobby die Treppen hinauf lief. Für gewöhnlich apparierte er. Für einen Hauselfen waren die Stufen nämlich entschieden zu groß.

Als Dobby näher an die Stufen herantrat, passierte etwas, was Harry noch nie gesehen hatte. Die Stufen fingen auf der rechten Seite, dort wo Dobby sich näherte, an sich zu teilen. Aus jeder einzelnen Stufe wuchs eine kleine Stufe heraus, sodass sich die Stufen, wo der Hauself lief zwar nur noch halb so breit, aber auch nur noch halb so hoch waren.

Im dritten Stock im Westflügel angekommen, lief Dobby in den Gang hinein und bog nach links ab. Harry erinnerte sich an sein erstes Jahr in Hogwarts, wo im dritten Stock der Zugang zum Stein der Weisen bewacht worden war. Er war nicht oft dort, da es in diesem Teil des Schlosses scheinbar nichts Interessantes gab. Dobby bog rechts um eine Ecke, die in einer Sackgasse endete, und blieb vor einem Porträt stehen. Er dreht sich um und wartete bis beide nah genug waren.

\enquote{Dobby möchte ihnen beiden danken, dass sie sich entschlossen haben Dobby zu folgen.}

\enquote{Keine Ursache}, sagte Harry.

Luna blickte sich nur um. Sie schien sich wohl zu fühlen, soweit man das sagen konnte.

\enquote{Was Harry Potter und seine \gst äh\abs}

\enquote{Luna, Luna Lovegood}, antwortete Harry.

\enquote{Was Harry Potter und Luna Lovegood versprechen sollten, ist, niemandem etwas hiervon zu erzählen.}

\enquote{Ich verspreche es}, antworteten Harry und Luna fast gleichzeitig.

\enquote{Das hier ist der fünfte Gemeinschaftsraum}, sagte Dobby.

\enquote{Kommt}, er drehte sich um und sprach zu dem Porträt: \zauber{Aqua Neros} und das Porträt öffnete sich.

Dobby schritt voran und winkte die beiden herein. Luna und Harry betraten den Raum und das Porträt hinter ihnen versperrte wieder den Eingang. Nach ein paar dunklen Metern und einer Biegung gelangten sie in einen Raum, der um einiges größer als der Gemeinschaftsraum der Gryffindors aussah, war. Luna schaute sich nur verträumt um und lief umher.

\enquote{Was ist das hier, Dobby?}, fragte Harry.

\enquote{Das}, Dobby drehte sich zu Harry um, \enquote{ist der Gemeinschaftsraum der Paare}, antwortete Dobby.

Harry verschlug es die Sprache. Für ein paar Sekunden konnte er gar nichts mehr sagen. Dachte Dobby vielleicht Luna und er wären ein Paar?

\enquote{Aber Luna und ich sind kein\abs}, doch Dobby unterbrach ihn.

\enquote{Paar? Vielleicht nicht in dem Sinne wie Sie denken, Sir. Aber Dobby hat zwischen Ihnen und Miss Luna eine Verbindung gespürt, keine Liebesbeziehung wie zwischen den anderen Pärchen, die Dobby ab und an beobachtet hat.}

\enquote{Du spionierst uns allen hinterher?}, fragte Harry erstaunt.

\enquote{Nein}, antwortete Dobby, \enquote{ich treffe sie nur zufällig bei meiner Arbeit. Halte mich aber immer versteckt, sodass sie mich nicht sehen können.}

\enquote{Ja aber was sollen wir in dem Gemeinschaftsraum für Paare, wer weiß noch davon, \abs}

Dobby unterbrach ihn abermals. \enquote{Setzen Sie sich Harry Potter, Sir, und ich werde es ihnen erklären.}

Harry setzte sich und auch Luna nahm auf dem Sofa neben ihm Platz. Immer noch mit ihrem leicht verträumten Gesichtsausdruck.

\enquote{Harry und Luna müssen wissen, dass außer den Hauselfen und Ihnen beiden niemand hier im Schloss etwas über diesen Raum weiß. Nicht einmal die Lehrer hier, oder der Hausmeister. Dobby hat ihn bei seiner Einweisung durch die anderen Hauselfen hier gezeigt bekommen. Aber da hier nie einer ist, braucht er auch nicht sauber gemacht zu werden. Doch Dobby hat zwischen Ihnen und Miss Luna etwas gespürt und sich entschlossen, Ihnen diesen Ort zu zeigen.}

Dobby senkte den Kopf und sprach weiter. \enquote{Es wäre Dobby eine Ehre, wenn Sie und Miss Luna diesen Ort hier benutzen würden.}

Harry war perplex, so etwas hatte er nicht erwartet.

\enquote{Wie \gst benutzen? Und was ist mit den anderen Pärchen?}

Dobby sah wieder hoch zu Harry, schaute Luna an und sprach dann.

\enquote{Noch nicht. Für die anderen ist es zu früh. Dobby lässt es Harry Potter wissen, wenn er es für richtig hält} und er fügte hinzu, \enquote{sofern Harry Potter, Sir, nichts dagegen hat.}

\enquote{Nein Dobby}, antwortete Harry.

Dobby sprach weiter: \enquote{Vor vielen Jahrzehnten wurde dieser Ort von zwei Schülern hier errichtet} und er deutete auf das Bild über dem Kamin hinter ihm. \enquote{Er diente ihnen als Liebesnest. Doch nach ihrem siebten Jahr vergaßen sie, es den anderen Pärchen mitzuteilen.} Dobby nahm wieder seine Hand herunter und blickte nun zu Harry. \enquote{Nur Sechst- und Siebtklässler dürfen diesen Raum betreten. Fünftklässlern ist der Zutritt nur gestattet, wenn sie einen Sechst- oder Siebtklässler als Partner haben.}

\enquote{Und woher wissen dann die Hauselfen davon?}, fragte ihn Harry.

\enquote{Dazu wollte Dobby gleich kommen. Da sich die Hauselfen hier um alles kümmern, haben die beiden beschlossen ihnen die Aufgabe des Reinigens zu überlassen. Sie ließen ihnen immer dann eine Nachricht zukommen, wenn sie es für nötig hielten, dass der Raum gereinigt werden sollte.} Und Dobby fügte hinzu: \enquote{Sie und Miss Lovegood sind die ersten nach einer sehr langen Zeit, die diesen Raum wieder benutzen dürfen.}

\enquote{Danke Dobby}, sagte Harry \gst Luna lächelte Dobby nur an und sagte dann: \enquote{Ja, danke Dobby} und sah sich weiter um.

\enquote{Dobby muss Sie jetzt verlassen Harry Potter, Sir, er hat noch andere Arbeit}, sagte der Elf und verließ den Raum.

Harry und Luna schauten sich im Raum um als plötzlich Luna sagte: \enquote{Schau mal, hier ist ja ein Schaukelstuhl. Der sieht genau so aus wie die in unserem Gemeinschaftsraum.}

Harry fiel auf, dass er noch nie im Gemeinschaftsraum der Ravenclaws oder der Hufflepuffs gewesen war. Nur einmal im zweiten Schuljahr war er bei den Slytherins gewesen. Nun schaute er sich genauer um und bemerkte, dass wohl von jedem Gemeinschaftsraum etwas hier sein musste. Luna blickte in der Zwischenzeit zur Decke und sah, dass dort die vier Hausfahnen hingen.

\enquote{Schau mal da oben}, sagte sie zu Harry ohne ihn dabei anzusehen. Harry blickte nach oben und sah die Fahnen jetzt auch. \gedanke{Wahnsinn}, dachte er.

Nach einer Weile im Raum fiel sein Blick auf eine Art Ankündigungstafel. Er näherte sich und las laut vor.

\begin{brief}
Gemeinschaftsraum der Paare erbaut und erdacht im Jahre 1875

von Sardak Slyhoot (Slytherin) und Selvine Vertap (Hufflepuff)

letzte Benutzung im selbigen Jahr

Reaktivierung durch Harry Potter (Gryffindor)

und Luna Lovegood (Ravenclaw) im Jahre 1996
\end{brief}

Harry bekam ein komisches Gefühl seinen und Lunas Namen auf der Tafel zu sehen. Er entschloss sich zu gehen, als er einen schwebenden Brief im Ausgang bemerkte. Er war an ihn adressiert. Er drehte ihn um und las den Namen Dobby. Nachdem er ihn geöffnet hatte, las er Folgendes:

\begin{brief}
Hallo Harry Potter, Sir.

Ich vergaß ihnen mitzuteilen, dass sie bitte mit Miss Lovegood die heutige Nacht hier verbringen sollen, im gleichen Bett. Leider hatte Dobby keine Zeit mehr um umzukehren, so schicke ich ihnen den Brief auf diesem Weg.
\signumspace
Stets zu Diensten

Hauself Dobby
\end{brief}

Harry verschlug es die Sprache. \gedanke{Dachte Dobby, ich sollte mit Luna schlafen? Oder meinte er nur, wir sollen die Nacht miteinander verbringen; im selben Bett, schlafend?} Er hatte nicht bemerkt wie sich Luna ihm näherte und den Brief mitlas.

\enquote{Schön} sagte sie, \enquote{dann verbringen wir die Nacht eben gemeinsam hier}. Sie drehte sich wieder um und ging an einen Tisch mit Schachbrettmuster auf der Platte, setzte sich in einen Stuhl und fragte, \enquote{Harry, hast du Lust auf eine Runde Schach?}

Das riss Harry wieder los, er drehte sich um und fragte Luna \enquote{Was? Du kannst Schach spielen? Hab ich gar nicht gewusst.}

\enquote{Du weißt vieles von mir noch nicht, Harry}, sagte sie und bot ihm den Platz gegenüber an.

Harry machte sich auf den Weg zu ihr. In der Zwischenzeit holte Luna ihren Zauberstab aus der Tasche und tippte das Schachbrettmuster mit der Spitze ihres Zauberstabes an. Als die Figuren auf dem Brett erschienen, setzte sich Harry und Luna nahm einen weißen und einen schwarzen Bauern vom Spielbrett. Sie verschränkte hinter ihrem Rücken die Arme und tauschte die Figuren ein paar mal hin und her. Dann nahm sie ihre Arme wieder nach vorne und zeigte ihm beide Hände ausgestreckt zur Wahl der Farbe. Harry entschied sich für die linke Hand und durfte beginnen. Luna setzte die zwei fehlenden Bauern auf ihre Positionen. Da Luna die weißen Figuren bei sich hatte, fing das Spielbrett an leicht abzuheben und sich um 180 Grad zu drehen.

Nach einem 3:1 für Luna war es auch schon recht spät und die beiden entschieden sich ein Zimmer zu suchen. Harry holte nochmals Dobbys Brief aus der Tasche. Nur um sicherzugehen, dass sie auch wirklich das gleiche Bett benutzen sollen. Ihm war leicht flau. Sie nahmen das erstbeste Zimmer und öffneten die Tür. Harry ließ Luna den Vortritt.

An der rechten Wand im Zimmer stand ein großes doppeltes Himmelbett mit Vorhängen, die man um das Bett zuziehen konnte. Die Farbe der Vorhänge bestand aus einem leichten Creme-Ton und durch das Fenster an der gegenüberliegenden Seite schimmerte der Mond herein. Es war nicht besonders hell, aber man konnte genug erkennen. Gegenüber der Tür war ein Schrank. \gedanke{Darin verbergen sich wohl Anziehsachen}, dachte Harry. Er lief um das Bett herum und schlug die Bettdecke zurück. Darunter kamen Lunas und sein Schlafzeug zum Vorschein. Dobby hatte wie immer, wie alle Hauselfen, perfekte Arbeit geleistet. Er drehte sich um, setzte sich auf die Bettkante und begann seine Robe aufzuknöpfen. Luna tat es ihm gleich, nachdem sie sich auf die andere Seite des Bettes gesetzt hatte. Nachdem beide ihre Schlafsachen angehabt hatten, krabbelte Harry ins Bett, wo Luna schon lag. Beide lagen nun nebeneinander und betrachteten die verzauberte Decke des Himmelbettes. Sterne schimmerten und ab und an kam eine Sternschnuppe vorbei. Luna griff nach Harrys Hand, legte den Kopf zur Seite um Harry anzuschauen und sagte mit ihrem üblichen Gesichtsausdruck.

\enquote{Gute Nacht Harry.}

\enquote{Gute Nacht Luna}, antwortete Harry und machte die Augen zu. Er dachte an Dobby: \gedanke{Ich kann ihm vertrauen. Auch wenn ich noch nicht weiß, was er will}, dachte sich Harry und schlief ein.

Als es am nächsten Morgen langsam heller wurde und die ersten Sonnenstrahlen den Raum anfingen zu erhellen, waren beide mit ihren Körpern zueinander gedreht, immer noch die Hände ineinander gelegt. Luna öffnete ihre Augen und sagte: \enquote{Guten Morgen Harry.}

Harry war noch etwas schläfrig und öffnete seine Augen. Da lag sie, Luna Lovegood. Doch etwas an ihr war anders als sonst. Ihr verträumter Blick war verschwunden und sie hatte einen durchdringenden und musternden Ausdruck auf ihrem Gesicht. Nicht dass sie ihn anstarren würde, aber ihr verträumter Blick war verschwunden. Sie schaute sich jeden Zentimeter seines Gesichtes genau an. Dann passierte etwas, was Harry einen wohligen Schauer über seinen Rücken laufen ließ. Ihr Gesicht kam seinem näher und sie küsste ihn. Nicht auf die Backe wie am Morgen zuvor bei den Thestralen. Nein, sie küsste ihn direkt auf den Mund. Harry war erstaunt, aber nicht überrascht, irgendwie hatte er gespürt, dass sie ihn küssen würde. Es war ein eigenartiges Gefühl. Sie ließ seine Hand los und sagte: \enquote{Ich hoffe, es gibt wieder Toastbrot und etwas Speck.}

Und da war er wieder, der leicht verträumte Gesichtsausdruck den sie die ganze Zeit hatte. Sie stand auf und zog sich um. Harry dachte an Lunas Kuss. Es war so ein komisches Gefühl. Nicht zu vergleichen mit dem Kuss von Cho und den Schmetterlingen, die er in seinem Bauch gefühlt hatte, als er sie in seinem fünften Jahr ansah oder küsste. Etwas anderes war da. Etwas was ihn glücklich machte, ohne verliebt zu sein. Er konnte es nicht beschreiben. Er stand auf und begann sich umzuziehen. Luna war in der Zwischenzeit schon wieder im Gemeinschaftsraum der Paare und wartete.

Als Harry hereinkam, fragte sie ihn: \enquote{Sehen wir uns wieder? Hier? Vielleicht in zwei Wochen? Zur selben Zeit am selben Tag, an dem uns Dobby hierhergeführt hat?}

\enquote{Ja}, antwortete Harry knapp und fügte noch ein: \enquote{Gerne Luna}, hinzu.

Sie marschierten wieder Richtung Ausgang und bogen um die Ecke, als sie bemerkten, dass man, anders als bei den anderen Gemeinschaftsräumen, sehen konnte, was draußen passiert, ohne dass man hereinschauen konnte. Sie verließen den Raum und liefen noch ein Stück gemeinsam.

\enquote{Das behalten wir aber für uns, Luna}, sagte Harry.

\enquote{Ja natürlich. Mir würde das sowieso keiner glauben}, meinte Luna.

Ihre Wege trennten sich und jeder machte sich auf den Weg zum jeweiligen Gemeinschaftsraum. Als Harry bei seinem ankam, waren schon einige seiner Mitschüler auf. Ron und Hermine spielten gerade eine Partie Schach, um Hermines Spiel zu verbessern. Harry betrat den Raum und lief Richtung Jungenschlafsäle, um sich zu duschen.

\enquote{Wo warst du?}, fragte Ron.

\enquote{Draußen. Den Sonnenaufgang anschauen}, antwortete Harry.

Er traute sich nicht über die Nacht mit Luna zu reden.

\enquote{Jetzt? Sonntagmorgen? Du spinnst}, meinte Ron.

Harry zuckte mit den Schultern und begab sich nach oben.

\enquote{Manchmal ist er schon merkwürdig}, sagte Ron zu Hermine.

Harry kam in der Zwischenzeit oben an, zog seine Sachen aus und duschte wie schon am Morgen zuvor. Gerade als er das klare kühle Wasser über sein Gesicht fließen spürte, fiel ihm der eigenartige Zauberer ein, der ihm bei einem seiner morgendlichen Spaziergänge vorige Woche begegnet war.

\begin{rueckblick}
Er lief ein Stück neben ihm her, als er sagte: \enquote{Dementoren \gst Harry} und der Fremde schaute ihn an und fuhrt fort: \enquote{Dementoren kann man, wie sie sicherlich wissen, mit dem Patronus Zauber vertreiben. Und ich habe gehört, dass sie ihn schon erfolgreich angewendet haben.}

\enquote{Ja}, antwortete Harry wahrheitsgemäß.

\enquote{Sie müssen wissen, dass die effektivsten Gedanken die positivsten sind. Aber \gst und das wissen nicht viele, die positiven Gedanken, die man für sich behält und keinem anderen sagt, das sind die Besten.}

\enquote{Woher wollen sie das wissen?}, fragte Harry und schaute den fremden Mann an.

Der Fremde betrachtete Harry, dann den See hinter ihm, und zeigte auf den See.

\enquote{Da, schauen sie}, meinte er.

Harry drehte sich um, doch da war nichts. Als er den Fremden fragen wollte, was er denn gesehen habe, und sich zu ihm umdrehte, war dieser verschwunden. \gedanke{Wo ist er hin?}, wunderte sich Harry.
\end{rueckblick}

Mittlerweile war er mit Duschen fertig und lief mit einem Handtuch um seine Hüfte in seinen Schlafsaal, um sich umzuziehen. Nachdem er fertig war und in den Gemeinschaftsraum der Gryffindors zurückgekehrt war, nahm er Ron und Hermine mit, um mit ihnen zu Frühstücken. Auf dem Weg dorthin begegneten sie wieder Luna, die ihn nur verträumt anlächelte und weiter lief. Hermine schaute Harry etwas komisch an, sagte aber nichts. Das war Harry nur Recht. Er wollte sie nicht belügen. Aber er wollte auch nichts über seine gemeinsame Nacht mit Luna erzählen.

In der Großen Halle angekommen, waren schon einige beim Frühstücken und der Rest der Schule würde auch bald folgen. Harry nahm sich heute mal etwas Butter und Marmelade auf den Toast. Er wollte das schon immer mal versuchen, statt des sonst üblichen Speck mit Bohnen und Toast, dass er sonst hatte. Er fand, dass es nicht besonders übel schmeckte, und dachte sich: \gedanke{Das nimmst du jetzt jede zweite Woche zu dir. Und zwar immer sonntags, nachdem du mit Luna zusammen warst.} Während er sein Frühstück kaute und seinen Kürbissaft trank, fiel ihm die Bibliothek ein. Nach seinem Frühstück verabschiedete er sich von Ron und Hermine und sagte ihnen, sie würden ihn in der Bibliothek finden, falls sie ihn brauchen sollten.

Dort angekommen fragte er Madame Pince, wo er denn etwas über Dementoren und positive Gedanken zu deren Abwehr finden würde, da er vermutete, die gesuchte Lektüre würde sich im abgesperrten Teil der Bibliothek befinden. Madame Pince gab ihm durch ihren durchdringenden Blick zu verstehen, dass er das hätte besser nicht fragen sollen. Nach einigen Sekunden, die Harry viel zu lange vorkamen, besserte sich schließlich ihr Blick und sie sagte: \enquote{Ausnahmsweise Mister Potter, folgen sie mir}. Harry war erstaunt über so viel entgegenkommen. Madame Pince war normalerweise nicht so.

\enquote{Wissen Sie}, sagte sie, als sie mit Harry im Schlepp durch die Bibliothek lief, \enquote{da sie schon öfters gegen Dementoren gekämpft haben und die sie anscheinend für ein besonders lohnendes Opfer halten, gestatte ich ihnen einen kurzen Blick in das entsprechende Kapitel. Wenn sie mehr wissen wollen, fragen sie einen Lehrer und bringen mir eine schriftliche Erlaubnis.} Harry bejahte, und Madame Pince, bereits angekommen, öffnete den Zugang zum abgesperrten Bereich der Bibliothek. Sie schritt hindurch, dicht gefolgt von Harry, und bog ein- zweimal ab, um vor einem Regal haltzumachen. Sie griff eines der Bücher heraus und reichte es ihm. \enquote{Sie haben fünf Minuten und ich bleibe hier bei ihnen stehen.}

Harry nickte abermals und schlug das Buch auf. Im Inhaltsverzeichnis fand er unter der Rubrik Abwehr von Dementoren den Eintrag über positive Gedanken. Er schlug die entsprechende Seite auf und las. Nachdem er wieder zum Inhaltsverzeichnis zurückgekehrt war, bemerkte er ein Kapitel über die Fortpflanzung und die Aufzucht von Dementoren. Leider waren die fünf Minuten schon abgelaufen und Madame Pince forderte das Buch zurück. Harry wusste nicht, ob er froh oder niedergeschlagen sein sollte. Einerseits hätte es ihn schon interessiert, aber andererseits hatte er nie so richtig darüber nachgedacht, wie Dementoren ihre Jungen aufziehen, bzw. ob sie sich wirklich vermehren würden. Madame Pince begleitete Harry wieder in den normalen Teil der Bibliothek zurück, schloss ab und widmete sich wieder ihrer Arbeit.

Auf dem Rückweg zu ihrem Schreibtisch am Eingang der Bibliothek, schaute sie begierig in die einzelnen Reihen um irgendwelche Verstöße festzustellen. Denn sie hasste es, wenn in ihren Büchern Flecken oder Krümel zu finden waren. Zwar gehörten die Bücher der Schule und nicht Madame Pince, aber sie benahm sich so als seien es ihre. Und den Professoren schien das nichts auszumachen, denn sie konnten sich immer darauf verlassen, wenn sie mal ein Buch benötigten, dass es flecken- und krümmelfrei sei. Madame Pince machte zwar immer ein riesiges Theater, wenn sie wieder jemanden erwischte, der über ihren Büchern seine Brotzeit oder sein Getränk ausgebreitet hatte, aber Harry fand in seinem sechsten Jahr, dass sie gar nicht so übel sei; wenn man sich mit ihr arrangieren konnte. Sie schnauzte die Lehrer genauso an wie die Schüler, wenn die Bücher verschmutzt zurückgegeben wurden, und das machte sie in Harrys Augen noch sympathischer.

Harry musste grinsen, als er daran dachte, wie Madame Pince einmal Dumbledore zusammengeschrien hatte, weil auf dem Buchdeckel ein runder Getränkefleck zu sehen war. Zwar benötigte sie nur einen Schwenk mit ihrem Zauberstab, um das Buch zu säubern, aber sie erachtete das immer als unnötig und nicht ihre Aufgabe. Ihre Devise war: Die Bücher müssen sorgfältig behandelt werden und ordnungsgemäß abgeliefert werden.

Harry machte sich auf die Suche nach einem Buch über Zaubertränke, denn er musste für Snape noch einen Aufsatz schreiben. Er machte sich nicht allzu viele Hoffnungen, da er bei Snape sowieso keine guten Noten bekommen würde. Aber dieses Thema, das sie gerade durchnahmen, interessierte ihn doch und er entschied sich, sich dieses Mal mehr anzustrengen. \gedanke{Aber nicht um Professor Snape einen Gefallen zu tun}, dachte er sich, \gedanke{sondern weil ich es will.} Als er nur noch wenig Zeilen zu schreiben hatte, tauchten Ron und Hermine mit Ginny im Schlepptau auf.

\enquote{Hi Harry} sagte Ginny und setzte sich gegenüber Harry, nachdem sie ihn beim Frühstück nicht gesehen hatte. \enquote{Hi Ginny} sagte Harry, schaute auf, und für einige Zeit konnte er seinen Blick nicht von ihr lassen.

Zu allem Überfluss saß Hermine genau neben ihr und schaute ihn an. In diesem Licht, das durch die Bibliothek hereinschien, und in dieser neuen Robe hatte er Ginny noch nie gesehen. Er kannte sie zwar seit seinem ersten Schuljahr und hatte auch sonst regen Kontakt mit ihr, aber so hatte er sie noch nie gesehen. Und direkt daneben Hermine. Harry wusste nicht, was er machen oder denken sollte. Zwei bezaubernde junge Mädchen, die ihm gegenüber saßen und sein jugendliches Blut in Wallung brachten.

\enquote{Wie zwei Engel}, brach es aus ihm heraus. Als er merkte, dass er das nicht nur dachte, sondern den beiden direkt ins Gesicht sagte, senkte er beschämt seinen Blick. Ginny wurde sofort rot und drehte sich mit einem breiten Schmunzeln im Gesicht weg. Doch auch Hermine ging es nicht besser. Auch sie errötete, etwas weniger als Ginny aber dennoch. Peinlich berührt drehte er sich zur Seite. Leider in die Richtung, in der Ron saß, der ihn nur mit offenem Mund anstarrte. So etwas hätte er von Harry nie erwartet. Harry entschied, es sei das Beste seinen Aufsatz zu Ende zu Schreiben und dann seine Sachen in sein Zimmer zu bringen.

Später am Abend war er auf dem Weg zu den Kerkern. Es musste bei Snape nachsitzen. Er wusste, dass es nicht gerade angenehmen werden würde. Er klopfte an die Bürotür und sie schwang auf. In Professor Snapes Büro brannte im Kamin ein kleines Feuer. Snape saß hinter seinem Schreibtisch und korrigierte Hausaufgaben. \enquote{Setzen Sie sich, Potter} sagte er, ohne aufzuschauen. Harry lief zum Tisch und setzte sich auf den Stuhl. Dann wartete er. Snape korrigierte diese Hausaufgabe zu Ende und widmete sich dann Harry.

\enquote{Wie laufen ihre Okklumentik-Übungen?}, fragte er Harry. Harrys Gesicht versteinerte. \enquote{Nun?}, fragte Snape und nahm sich die nächste Hausaufgabe vor.

\enquote{Na ja, ich hatte mich in letzter Zeit nicht mehr darum bemüht. Nicht nachdem sie den Unterricht beendet hatten.}

Professor Snape sah auf. \enquote{Bitte? Ich dachte, sie führen sie trotzdem weiter!?} Jetzt war Harry vollkommen perplex. Professor Snape unterbrach seine Korrekturarbeiten, als Harry nicht mehr reagierte. \enquote{Ich war wütend und sauer. Ich wollte nicht, dass sie das sehen. Aber jetzt habe ich erkannt, dass sie es mehr als nur notwendig haben.} Snape legte die Hausaufgabe wieder auf den unerledigten Stapel und die Feder beiseite. Er ging um den Tisch herum und setzte sich Harry gegenüber. \enquote{Schließen Sie ihre Augen.} Widerwillig tat Harry, was von ihm verlangt wurde. \enquote{Was sehen sie?}, fragte Snape.

\enquote{Nichts. Es ist dunkel.}

\enquote{Gut. Stellen Sie sich vor, dass sie vor einer schwarzen Wand stehen. Stellen Sie sich ferner vor, dass hinter ihnen ebenfalls eine schwarze Wand ist. Sie fühlen sie in ihrem Rücken. Die Wand vor ihnen ist nur wenige Zentimeter von ihnen entfernt. Links und rechts ist frei, aber nur Dunkelheit. Stellen Sie sich vor, dass sie Müde werden. Sie sehen nur die schwarze Wand und stellen sich vor, dass sie Müde werden.}

\enquote{Verlassen Sie jetzt ihren Körper und sehen sie sich an, wie sie Müde werden. Ihre Gedanken verlassen sie.} Harry spürte, wie er langsam in einen leichten Dämmerschlaf glitt. Snapes Stimme schien eine beruhigende Wirkung auf ihn zu haben. Er hatte ihn mit keinerlei Aggressivität oder Hass angesprochen. Er sah sich, wie er zwischen zwei Wänden stand. Vollkommen entspannt. Sein Geist begann sich zu leeren.

\enquote{Öffnen Sie jetzt wieder ihre Augen.} Harry öffnete seine Augen und sah Snape direkt an. \enquote{Was denken sie?}

\enquote{Wenig. Ich dachte an\abs} Plötzlich kamen immer mehr Gedanken wieder in seinen Sinn.

\enquote{Professor, es kommen wieder mehr Gedanken in meinen Geist.}

\enquote{Gut, sehr gut, für das Erste mal. Machen sie weiter, wiederholen sie diese Übung.}

Harry nickte und schloss seine Augen. Er hörte noch wie Snape aufstand und sich hinter sein Pult setze, ein Pergament nahm und darauf herumkratzte.

Als er seine Augen wieder aufmachte, begann er nach einigen Sekunden kratzende Geräusche wahrzunehmen. Er drehte seinen Kopf und sagte: \enquote{Das ist eigenartig. Ich habe ein paar Sekunden keine Geräusche mehr wahrgenommen. Anders als beim ersten Mal.}

Snape nickte und sah auf seine Uhr. \enquote{Ihr Nachsitzen ist für heute vorbei. Üben Sie bis zum nächsten Termin. Ich werde ihnen schon Bescheid sagen. Dann werde ich versuchen wieder in ihren Geist einzudringen. Es ist also besser, sie üben abzuschalten.}

Dann wandte sich Professor Snape wieder seinen Hausaufgaben zu und beachtete Harry nicht mehr. Als Harry die Tür geöffnet hatte, hörte er noch: \enquote{Und kein Wort! Zu niemandem!} Harry verstand und nickte. \enquote{Verstanden Professor!} Auf dem Weg zu seinem Bett schwirrte sein Kopf. \gedanke{Warum tat Snape das?}

Aber die spannendere Frage war: Würde er es wiederholen, oder war das eine einmalige Sache?

\trenn

Kurz nach dem Mittagessen war es wieder so weit, dass sich das Gryffindor-Quidditch-Team, um die aktuelle Lage zu besprechen und zu trainieren, auf dem Quidditch-Feld traf. Nachdem alle umgezogen und mit ihren Besen bereit waren, begann das Training. Auf Katies Zeichen saßen alle auf und flogen hinaus auf das Feld. Katie holte ihren Zauberstab heraus und zeigte auf die Truhe unten auf dem Feld. Sie öffnete sich und der Schnatz, sowie die beiden Klatscher sprangen heraus. \zauber{Accio Quaffel} sprach Katie und steckte ihren Zauberstab sofort weg.

Der Quaffel bewegte sich auf sie zu und wurde sofort von den Treibern abgefangen. Diese versuchten nun den Ball in eine der drei runden Tore zu bekommen.

Nach dem Training und einer Dusche in den Duschräumen des Quid\-ditch-Team\-rau\-mes, ging Harry zurück in sein Zimmer, schaute auf seinen Stundenplan und entdeckte eine Änderung.

\enquote{Was?}, rief er, \enquote{Mittwochabend eine Doppelstunde Verteidigung gegen die dunklen Künste und donnerstagmorgens ebenso?}

Harry starrte seinen Stundenplan an. Warum hatte er erst in seiner vierten Woche Verteidigung gegen die dunklen Künste? Hatte Dumbledore jetzt doch jemanden gefunden? Oder konnte der Professor vorher nicht? Harry rannte hinunter in den Gemeinschaftsraum, um es Ron und Hermine zu erzählen. Doch die hatten auch ihre Stundenpläne in der Hand und schauten erstaunt.

\enquote{Harry, hast du es schon gesehen?}, fragte Hermine.

\enquote{Ja, gerade eben}, antwortete Harry.

\enquote{Was hat das zu bedeuten?}, fragte Hermine. \enquote{Erst gar kein Unterricht, nicht einmal eine Vertretung durch Snape wie im dritten Jahr, und jetzt gleich zwei Tage hintereinander Doppelstunden.}

\gedanke{Ich werde es noch bald genug herausfinden}, dachte Harry und machte sich daran in die Große Halle zum Abendessen zu gehen. Beim Hineingehen fiel ihm ein Mann auf, den er nur von hinten sah. Er war in der Hocke und lehnte seine Arme auf den Lehrertisch; direkt gegenüber von Dumbledore. Er schien sich mit ihm zu unterhalten. \gedanke{Irgendwie kommt er mir bekannt vor}, dachte Harry und begann sein Abendbrot einzunehmen. Als er gelegentlich wieder zum Lehrertisch herüberschaute, entdeckte er wie der unbekannte Mann aufstand und sich an den freien Platz an der Stirnseite des Lehrertisches aufseiten des Slytherin-Haustisches setzte. Es war ein Mann mittleren Altern, schätzungsweise 35. Er hatte einen Umhang an, der schmutzig aussah und nur vereinzelte Stellen eines goldenen Stoffes aufwies. Seine Hosen waren in dunklem grün gehalten und sein Oberteil war Kastanienbraun. Harry erkannte ihn wieder und dachte an den fremden Mann, den er einmal während einer seiner wenigen morgendlichen Spaziergänge getroffen hatte, als er mal nicht joggte. \gedanke{Komisch}, dachte er sich. \gedanke{Ist das der neue Professor in Verteidigung gegen die dunklen Künste?}

\trenn

Am Montag-Morgen hatte Harry, wie dieses Jahr üblich, Unterricht bei Professor McGonagall im Fach Verwandlung. So langsam trafen alle ein, aber McGonagall war nicht da. Harry schaute sich um, aber er fand sie nicht. Auch die Katzengestalt, die sie immer mal wieder annahm, sah er nirgends. \enquote{Das sieht ihr gar nicht ähnlich}, sagte Hermine, die genau hinter ihm stand. Sie setzten sich und als alle Schüler da waren, trat Professor McGonagall ein. Sie hatte den fremden Mann hinter sich, der ihr folgte. Nachdem sie ihr Pult erreicht hatte und sich dahinter gestellt hatte, blieb der Mann neben ihr stehen.

\enquote{Dies}, so sprach sie, \enquote{ist Professor Elber. Er wird sie in Verteidigung gegen die dunklen Künste unterrichten. Er ist hier, weil Professor Dumbledore meint, dass es an der Zeit ist, etwas in diesem Fach zu tun. Mehr als sie die vergangenen Jahre getan haben. Er hat noch nie unterrichtet und hat mich daher gebeten, einmal zuschauen zu dürfen. Sie werden ihn vor Mittwochabend in einigen anderen Klassenräumen antreffen.} Sie drehte sich zu ihm um und fragte ihn: \enquote{Möchtest du noch was sagen?} Harry hörte zum ersten Mal, dass Professor McGonagall einen anderen Lehrer duzte. Es war eigenartig.

\enquote{Ja, das würde ich gerne.} Er dreht sich zur Klasse und sprach weiter \enquote{Meinen Namen kennen sie ja bereits alle und außer der Tatsache, dass sie alle pünktlich zu meinem Unterricht erscheinen werden, gibt es momentan nichts zu sagen. \gst Ach ja, noch eines. Wenn sich mein Fach auf ihrem Stundenplan rot färbt, dann werfen sie einen Blick auf die Rückseite und lesen Sie sie bitte. Durch Antippen verschwindet die Schrift und die rote Markierung in meinem Fach}. Er dreht sich zu Professor McGonagall um und sagte zu ihr: \enquote{Du kannst anfangen, Minerva, ich bin fertig}.

Professor McGonagall hielt ihren Unterricht wie immer und man merkte, dass, anders als letztes Jahr, sie sich zwar genauso wenig anmerken ließ, dass jemand dabei war, der sie beobachtet, aber sie war wesentlich entspannter. \enquote{Heute}, so Professor McGonagall, \enquote{nehmen wir die Verwandlung von Gold in Lebewesen vor. Ein schwieriges Unterfangen. Zu diesem Zweck bekommen sie von mir je zwei Galeonen Wichtel-Gold, das nach dem Ende der Stunde verschwinden wird.}

\gedanke{Wichtel-Gold}, dachte Harry. \gedanke{So eines hat mir mal Ron gegeben. Doch nach kurzer Zeit war es verschwunden.}

Professor McGonagall öffnete eine ihrer Schubladen und nahm ein Säckchen heraus. Auf jeden Platz legte sie zwei Münzen. Als sie fertig war, nahm sie eine Münze aus dem Säckchen und warf es auf ihr Pult. Danach legte sie die Münze ebenfalls auf ihr Pult und fing an. \zauber{Aurumomorph chelys} Sie schwang ihren Zauberstab in einer eleganten Bewegung und das Goldstück verwandelte sich in eine kleine Schildkröte. Sie drehte sich wieder zur Klasse und sprach weiter.

\enquote{Auf diese Art und Weise, können sie ihr Geld verstecken, oder auch andere täuschen. Bitte versuchen Sie es alle.} Harry schaute zu Hermine, die sich sofort darüber hermachte. Funken sprühten aus ihrem Zauberstab und die Münze begannt sich zu verformen. Leider schaute das Ergebnis nicht sehr schön aus. Es war zwar eine Schildkröte, aber ihr Panzer sah immer noch wie eine Seite einer Münze aus. \enquote{Für diesen Zauber, wie für alle in meinem Unterricht, brauchen sie viel Konzentration. Fahren Sie fort.}

Die Stunde verlieft relativ ruhig und nach der Hälfte der Zeit hatte es Harry geschafft seine Münze in ein Tier zu verwandeln. Der Panzer war zwar nicht grün oder gräulich wie der einer echten Schildkröte, sondern hatte noch immer eine goldene Farbe, aber man konnte die Konturen und Muster auf dem Panzer erkennen. Ron machte sich auch erstaunlich gut, obwohl das Muster des Panzers nur leicht zu erkennen war. Hermine hatte es schon fast so weit, dass die Schildkröte nicht von einer echten zu unterscheiden war. Auch Parvati war recht gut. Nachdem Harry in Richtung Neville geblickt hatte, wurde er wieder ruhiger. Neville saß noch immer vor seiner Münze und versuchte sie umzuwandeln. Aber außer einem Kopf und einem Schwanz, war nicht viel zu sehen. McGonagall schaute ihn etwas mitleidig an und blickte danach zu Professor Elber. Dieser stand auf und ging zu Harrys erstaunen zu Neville. Als er an seinem Tisch angekommen war, machte er eine Geste die Neville andeutete, er hätte gerne seinen Zauberstab. Professor McGonagall schnürte ihre Lippen zusammen und ihre Augen weiteten sich, denn das hatte auch sie noch nicht gesehen. Neville gab Professor Elber seinen Zauberstab. Dieser tippte die Münze einmal an und sie verwandelte sich in eine perfekte Schildkröte. Er tippte die Schildkröte erneut an und an ihrer Stelle war wieder die Münze zu sehen. Dann ging er leicht in die Hocke, flüsterte Neville etwas ins Ohr und gab ihm seinen Zauberstab zurück. Neville nickte nur kurz und versuchte es noch einmal. Jetzt kamen wieder Funken aus Nevilles Zauberstab und die Münze begann sich zu verwandeln. Dieses Mal jedoch so weit, wie Harry bei seinem dritten Versuch war. Man konnte die Schildkröte erkennen, jedoch war sie vollkommen goldfarben und das Prägemuster der Münze war über ihren ganzen Körperteilen zu erkennen. Neville atmete sichtbar erleichtert auf, und Professor Elber und Professor McGonagall grinsten. Am Ende der Stunde waren fast alle so weit, dass sie ihre Münzen perfekt umwandeln konnten.

Nach der Stunde ging Harry noch zu Professor McGonagall, um sie wegen einer Berechtigung für die abgesperrte Sektion der Bibliothek zu fragen.

\enquote{Professor, kann ich sie was fragen?}

\enquote{Natürlich Mister Potter}, antwortete sie. Sie waren bereits alleine im Zimmer, als Harry fragte.

\enquote{Ich hätte gerne eine Berechtigung für die abgesperrte Sektion. Es geht um ein Buch über Dementoren, Madame Pince hat mich kurz lesen lassen\abs}

\enquote{Auf keinen Fall}, gab Professor McGonagall zurück und räumte ihre Sachen zusammen.

\enquote{Aber Professor}, unterbracht sie Harry.

\enquote{Nein Potter}, antwortete sie.

\trenn

Nachdem der restliche Tag einigermaßen ruhig verlief und Neville anscheinend jetzt bei vielen Fächern besser geworden zu sein schien, kam der Abend. Harry las noch einmal seinen Aufsatz durch, verbesserte noch ein paar Fehler und schrieb ihn noch einmal in Reinform ab. Dann ging er zu Bett. Der nächste Tag versprach nichts Gutes, da er kurz nach dem Frühstück Snape hatte. Wie die anderen bezog auch er seinen Platz im Kerker. Snape kam herein und sammelte die Aufsätze ein. Er tippte mit seinem Zauberstab auf die Tafel und die Zutaten für den nächsten Zaubertrank erschienen. Harry schrieb sie sich ab und fing an seinen Trank zu brauen. Währenddessen korrigierte Snape diverse Hausaufgaben. Immer wieder schaute er auf, suchte nach irgendjemanden, blickte zurück und strich ein paar Zeilen durch. Harrys Trank brodelte so vor sich hin, als er wieder aufblickte und Snapes Blick auffing.

Er fixierte ihn kurz und Snape blickte wieder auf seine Hausaufgaben zurück. \gedanke{Oh!}, dachte Harry. Das ist wohl meiner, aber außer ein paar scheinbar kurzen Bemerkungen und der Benotung passierte nichts. Harry befürchtete schon wieder ein D für Durchgefallen zu bekommen. Am Ende der Stunde füllte Harry seinen Trank in ein Glas ab, hängte einen Zettel mit seinem Namen dran, stellte es wie die anderen auf Snapes Tisch und nahm seine Hausaufgaben mit. \gedanke{Snape war heute einigermaßen ruhig}, dachte Harry. Als er den Kerker verlassen hatte, schaute er sich seine Hausaufgaben an. Er konnte nicht glauben, was er da las. Ein A - zum ersten Mal ein A, das hieß er hatte bei Snape eine Hausaufgabe bestanden, sie war annehmbar. Er schaute unten auf die Kommentare von Snape, die da lauteten. \accentuate{Es scheint, als ob sie doch lernfähig sind Potter. Zwar knapp aber doch noch ein A. Snape.}

Der Rest des Tages, dachte Harry, schien gerettet zu sein, denn die Stunden bei Professor Flitwick waren immer angenehm. Harrys Gedanken waren schon bei morgen Abend, wenn sie zum ersten mal Professor Elber hatten. Er wusste nichts über ihn und auch Hermine konnte nichts dazu sagen. Als er Hagrid fragte, meinte der nur: \enquote{Keine Ahnung, Professor Dumbledore hat ihn irgen'wo aufgetrieb'n. Ich hab' ihn sonst noch nie g'seh'n.} Er beschloss also zu warten bis es so weit war, da anscheinend niemand etwas über diesen ominösen Professor wusste.

Mittwochmorgens war mal wieder Wahrsagen bei Firenze, der sich seit diesem Jahr die Stunden mit Professor Trelawney zu teilen schien. Das waren jedes Mal eigenartige Stunden, mittwochs bei Firenze, in einem Zimmer, das wie ein Wald aussah, und freitags bei Professor Trelawney, bei der es immer nach Weihrauch und Räucherstäbchen roch und von der jeder genau wusste, dass sie nicht hellsehen konnte. Der Vormittag verlief genauso ruhig wie am Tag zuvor die Stunden bei Professor Flitwick. Endlich läutete die Schulglocke das Ende der Stunde ein und die Schüler packten ihre Sachen, um zum Mittagessen in die Große Halle zu gehen. Am Mittagstisch sitzend fragte Ron \enquote{Und, gippt es schm ws neus}, den Mund voller Essen. Und als er herunterschluckte \enquote{über diesen Professor Elber?}

\enquote{Nein}, antwortete Hermine, \enquote{aber das wirst du nachher schon erfahren}.

\trenn

\enquote{Herzlich willkommen zu einem neuen Schuljahr. Holen Sie bitte alle ihre Feder hervor}, sagte Professor Flitwick am Anfang der Stunde. Er wartete, bis alle Schüler dies getan hatten, und meinte dann: \enquote{Also, wir beginnen heute mit dem Wutschen und Wedeln.}

\enquote{Aber Professor}, sagte Hermine, \enquote{das hatten wir doch schon im ersten Schuljahr.}

Professor Flitwick drehte sich herum und meinte: \enquote{Dann können sie mir das sicher zeigen, Miss Granger.}

Hermine schluckte kurz, vollzog dann aber die Bewegung und sagte: \zauber{Wingardium Leviosa.} Ihre Feder begann zu schweben und flog durch den Raum.

\enquote{Gut, Miss Granger. Und jetzt wieder zurück.} Hermine lies ihre Feder sinken und auf ihren Platz zurück schweben.

\enquote{Ja ja, das war ganz eindrucksvoll, Miss Granger, aber das hatten sie ja bereits in ihrem ersten Jahr. Ja, sehr eindrucksvoll.} Er drehte sich grinsend weg und als er auf seinem Podest Platz genommen hat, sagte er: \enquote{Und jetzt Miss Granger, versuchen sie es ohne Worte.}

\enquote{Aber\abs} stammelte Hermine.

\enquote{Nur zu, Miss Granger.}

Hermine sah ihn unsicher an. Dann sah sie auf ihre Feder. Sie konzentrierte sich und dachte: \spruch{Wingardium Leviosa.} Doch nichts geschah.

Professor Flitwick grinste sie an. \enquote{Verzeihung, Miss Granger.} Nun lachte er lauter. \enquote{Das konnte nicht funktionieren. Ich hatte einen kleinen Zauber über das Klassenzimmer gelegt.} Er schwang seinen Zauberstab und meinte dann erneut. \enquote{Versuchen Sie es jetzt.}

Sie konzentrierte sich abermals und dachte: \spruch{Wingardium Leviosa.} Dieses Mal jedoch klappte es. Die Feder schwebte zaghaft und zittrig nach oben. Es kostete Hermine eine Menge Kraft.

\enquote{Sehr schön. Bitte versuchen Sie es jetzt alle.}

Kurz darauf schwebten ein paar Federn zittrig in die Höhe. Der Rest hatte damit noch Probleme.

\enquote{Wir werden die nächsten paar Male noch mit der Feder üben. Und damit auch dem letzten von ihnen klar ist, was wir hier machen, wir üben ungesagte Zauber.}

Den Rest der Stunde wurden die Federn immer und immer wieder durch den Raum schweben gelassen. Professor Flitwick erklärte immer wieder, worauf man achten musste, wenn man ungesagte Zauber ausführte, und so wurden die Schüler immer sicherer.

\trenn

\enquote{Der hat doch ’nen Knall}, sagte Adrian, ein Siebtklässler, als er den Gemeinschaftsraum betrat.

Harry drehte seinen Kopf zu ihm, um zu hören, wen oder was er meinte.

\enquote{Bringt uns \accentuate{lebendiges Feuer} bei. Gleich in der ersten Stunde}, fuhr er an seinen Kumpel, zwei Klassen unter ihm, fort. \enquote{Erst hat es ja noch Spaß gemacht, aber dann\abs ganz gewöhnliches Dämonenfeuer ist das gewesen.}

Harrys Kopf fuhr weiter herum, damit er beide gut sehen konnte. Hermine ließ ihr Buch fallen und Ron verschluckte sich fast an seinem Getränk.

\enquote{Erzählt uns was vom Pferd\abs}, maulte er weiter. \enquote{\inner{Am Montag werden wir uns ansehen, wann es nützlich ist und wann man es als Waffe einsetzen kann. Sie müssen wissen, wie sie diese Art von Feuer unter Kontrolle halten können, es selbst erzeugen und gegen ihre Gegner zurückwerfen können. Ich rede hier nicht vom Angriff}, sprach er. Dann hat er so eine Flamme erzeugt und uns alle daran fühlen lassen.}

\enquote{Wie hat es sich angefühlt?}, wollte Thomas, sein Kumpel wissen.

\enquote{Wie ein kleiner Herzschlag}, sagte Adrian ruhig. Er hatte sich in der Zwischenzeit beruhigt.

\enquote{Wieso bringt er euch so etwas bei?}, fragte Hermine.

\enquote{Weil er vielleicht ein verkappter Todesser, oder ein Sympathisant von Du-weißt-schon-wem ist?}

\enquote{Dann müsste sich Dumbledore aber schwer täuschen}, warf sie ein. \enquote{Wieso bringt er euch das bei? Dämonen- oder Teufelsfeuer ist gefährlich.}

\enquote{Eben nicht}, antwortete er. \enquote{Er hielt es in seiner Hand und manipulierte es, ohne Mühe. Er zeigte uns, dass auch das gefährliche Feuer zwei Seiten haben kann. Dass Magie nicht in zwei Seiten geteilt werden kann, sondern es verschiedene, mannigfaltige Abstufungen gibt.}

Mittlerweile lauschten alle Schüler im Raum Adrian zu.

\enquote{Er hat irgendwas Unheimliches an sich}, fuhr er fort. \enquote{Mal sehen, ob mich mein Gefühl täuscht. Denn der Unterricht selber war nicht schlecht, nur das Thema hat mich irritiert.}

\trenn

Harry saß in Dumbledores Büro.

\enquote{Nun Harry, was willst du von mir wissen?}

Harry senkte kurz seinen Kopf, sah dann in Dumbledores Augen und fing an, \enquote{Professor, \gst ich habe immer, wenn ich mein Amulett anfasse} und er zeigte es ihm kurz, \enquote{diese Vision, wie ich letztes Jahr den anderen den Patronus-Zauber lehre. Sie haben aber mal zu mir gesagt, dass das sehr fortgeschrittenen Magie sei, und die meisten Erwachsenen das nicht können. Und doch haben es viele von ihnen geschafft. Was hat das zu bedeuten Professor? Ist es, weil wir noch sehr jung sind? Liegt es an mir, weil die anderen Wissen, dass es möglich ist und sie dadurch auch hoffen, es zu können, oder liegt es vielleicht daran, wie ich es ihnen gezeigt habe?}

Dumbledore stellte seine Finger gegenüber und führte sie zu seinem Mund und seiner Nase. Er schaute Harry über seine halbmondförmige Brille an und atmete tief ein. Dann nahm er seine Hände herunter und stand auf. Er lief ein paar Schritte im Raum umher und drehte sich wieder zu Harry, der ihn die ganze Zeit über mit seinem Blick verfolgte. \enquote{Davon habe ich gehört. Eine erstaunliche Leistung, die dir da gelungen ist. Lass mich versuchen es dir zu erklären \gst Warum kannst du es?}

\enquote{Weil ich es schon einmal getan habe \gst Ich erinnere mich wieder an unser Gespräch darüber.}

\enquote{Ihr habt doch gelernt, dass Magie keine feststehende Größe ist. Sie ist ein lebendiges Wesen, das uns alle umhüllt, das uns durchdringt, uns mit jedem Baum, jedem Stein, jedem Lebewesen und anderen Dingen verbindet. Darum sind auch einige Zauberer in der Lage sich mit Tieren zu unterhalten.}

Harry staunte und erinnerte sich an sein zweites Schuljahr, als er die Schlange von Justin abhielt, indem er mit ihr sprach. Jetzt erinnerte er sich auch wieder, dass er Parsel sprechen konnte.

\enquote{Du musst wissen, dass ein Patronus, wie man dir schon gesagt hatte, nur von jemandem erzeugt werden kann, der sehr intensive positive Erinnerungen hat.}

Harry dachte an seine Nächte mit Luna, schmunzelte und nickte dann.

\enquote{Ich sehe, du hast einige sehr schöne Erinnerungen. Behalte sie für dich. Sie sind umso effektiver, wenn du sie mit niemandem, oder mit so wenig wie möglich teilst.}

Harry nickte abermals und wusste nun, dass er sehr wenig über Luna und sich erzählen würde.

\enquote{Aber, und das wissen nicht viele}, Professor Dumbledore setzte sich wieder und legte seine Hände auf die Armlehnen seines Sessels, \enquote{nur derjenige, oder diejenige, die auch traurige Erinnerungen, schreckliche Erinnerungen mit sich trägt, kann einen wirklich effektiven Patronus heraufbeschwören, der hunderte von Dementoren abhält. Und das ist nur, weil diejenigen am besten Wissen, wie sich Schmerz anfühlt. Wenn man nie einen so großen Schmerz gefühlt hat, hat man keine Ahnung davon hat, was einem ein Dementor alles anhaben kann.}

Harry begann zu verstehen. \enquote{Sie meinen, weil ich mit ansehen musste wie meine Mutter und Cedric Diggory starben und andere schreckliche Erlebnisse hatte, bin ich dazu in der Lage.}

\enquote{Ja}, sprach Dumbledore \enquote{und weil du weißt, dass du es kannst. Du hast ein intuitives Verständnis für die Macht entwickelt, die einem die Magie verleiht. Und ich habe bemerkt, dass du in dieser Schule nicht der einzige bist. Andere in deinem Jahrgang beginnen langsam zu verstehen, wie die ganzen Zusammenhänge funktionieren. Es gibt viele Zauberer und Hexen, die ihre Magie anwenden, ohne sich im Klaren über die Zusammenhänge zu sein. Um ehrlich zu sein, das ist ein ziemlich großer Teil.}

\trenn

Nachdem die Glocke den Beginn der Verteidigung gegen die dunklen Künste Stunde angekündigt hatte, standen Harry, Ron und Hermine auf, um ihr übliches Klassenzimmer aufzusuchen. Dort angekommen setzten sie sich und warteten bis der Professor aus seinem Büro kam. Es läutete ein zweites Mal, um die Schüler daran zu erinnern, dass in zwei Minuten der Unterricht begann. Es war still im Raum, denn die Tür zum Büro des Professors, die über eine Treppe zu erreichen war und direkt vor der Klasse in etwa 2 Meter 50 Höhe war, blieb verschlossen. Die Glocke läutete ein drittes Mal und die Tür zum Klassenzimmer schloss sich. Alle drehten sich um, in der Hoffnung der Professor würde auftauchen, aber nichts geschah. Langsam wurde es unruhig und noch immer fehlten zwei Schüler. Als dann die Tür aufging und die fehlenden Schüler zur Tür hereinstürmten, schrien diese kurz auf, und die Tür hinter ihnen fiel wieder zu. Sie rieben sich ihre Arme und setzten sich. Jetzt ging die Tür zum Büro des Professors auf, und ein gut gelaunter Professor Elber kam herein. Er schloss hinter sich die Tür und kam die Treppe herunter.

\enquote{Ich sagte ihnen schon zu Beginn der Woche bei Professor McGonagall, dass sie zu meinem Unterricht pünktlich erscheinen werden. Aber sie brauchen keine Angst zu haben, denn wie mir berichtet wurde, haben sie es ja bereits durchgenommen, wie man sich gegen virtuelle Schmerzen wehrt}. Dann sprach er weiter. \enquote{Dieses Jahr fangen wir mit den wirklich wichtigen Dingen an. Nachdem sie erst jetzt dieses Fach haben}, machte er weiter, \enquote{konnten sie keine Bücher mehr beschaffen. Deshalb werden sie von mir welche bekommen, die sie am Ende des Schuljahres hier wieder abliefern werden. Oder sie kaufen sie.} Er zeigte auf einen Bereich an der Seite des Zimmers auf dem \accentuate{Bücherrückgabe} stand. Er schwenkte seinen Zauberstab und wie aus dem Nichts tauchte an jedem Platz ein Buch auf. Pechschwarzer Einband und in silbernen Lettern stand darauf: \accentuate{Dunkle Künste für Anfänger} und darunter \accentuate{Band 1}.

\enquote{Diese Bücher enthalten die Grundlagen derjenigen Künste, die sie lernen, zu bekämpfen im Begriff sind. Es wird nur das jeweilige Kapitel, an dem wir arbeiten, und alle früheren sichtbar sein, um zu verhindern, dass sie sich durch Ausprobieren von Flüchen höherer Kapitel verletzen.}

Hermine hob ihre Hand und fragte: \enquote{Professor, lernen wir hier wirklich die dunklen Künste?}

Die ganze Klasse hielt den Atem an.

Professor Elber blickte sie an und sagte dann: \enquote{Ja, \gst was die Theorie anbelangt vorerst schon. Wir werden diese Zaubersprüche und Flüche erst später wirklich praktisch anwenden, denn nur wenn sie wissen, was der Gegner auf sie loslässt, wenn sie es an seinen Bewegungen und den Funken seines Zauberstabes erkennen, sind sie effektiv genug, sich dagegen zu wehren. Sie haben noch eine Menge Grundlagen zu lernen.} Er wanderte dabei durch die Klasse und fuhr mit seiner Erzählung fort. \enquote{Nur die Verteidigungszauber zu kennen, kann einem Probleme verursachen. Und lassen Sie mich eines klarstellen. Es gibt keine dunkle Magie. Genauso wenig gibt es eine weiße, oder helle Magie. Magie hat keine Farbe. Sie ist neutral, farblos. Nur zu welchem Zweck ich einen Zauber einsetze bestimmt, ob ich böse bin oder nicht.}

Er lief zu seinem Schreibtisch zurück und hob ein Tuch hoch, das zwei Pflanzen verdeckte. Die eine konnte man als eine Art Tulpe bezeichnen, die andere sah dürr und mager aus, mit kleinen Kapseln an den Enden der Zweige. \enquote{Miss Granger, kommen Sie bitte her}, sagte Professor Elber mit einem leicht forschen Ton. Hermine schluckte kurz, stand dann aber auf und ging zu Professor Elber. \enquote{Ich möchte, dass sie einen Feuerstrahl auf diese Pflanze halten.} Hermine schaute ihn etwas eigenartig an, holte dann aber ihren Zauberstab aus der Innentasche ihres Umhangs und brannte die Pflanze nieder. Außer einem Häufchen Asche blieb nichts mehr übrig. Die Pflanze war tot. \enquote{Und jetzt, nachdem sie alle gesehen haben, was dieser Zauber bewirkt, wenden sie ihn bei dieser Pflanze an, Miss Granger.}

Der Professor zeigte auf die andere Pflanze. Hermine tat wie geheißen und es ergoss sich ein Feuerstrahl über die andere Pflanze. Diese hier schien widerstandsfähiger zu sein. Doch plötzlich passierte etwas Unerwartetes. Die kleinen Kapseln sprangen auf und viele kleine Samen kamen heraus. Viele fielen auf den Boden und nur wenige wieder in die Erde. Hermine stoppte ihren Feuerregen und steckte ihren Zauberstab wieder ein. Die Pflanze war nun auch zu Asche geworden, aber im Gegensatz zur Tulpen-ähnlichen, begannen aus den Samen neue kleine Pflanzen zu sprießen. Zuerst kamen ein paar kleine grüne Blätter, die immer größer wurden, dann fing in der Mitte an ein Stängel zu wachsen, der am Ende in einer Blüte endete, die schöner als alle anderen Blüten war, die sie jemals gesehen hat. \enquote{Vielen Dank Miss Granger} sagte Professor Elber und deutete Hermine auf ihren Platz.

\enquote{Was sie gerade gesehen haben}, sagte er, als Hermine wieder zu ihrem Platz lief, um sich zu setzen, \enquote{war zweimal der exakt selbe Zauberspruch. Nur, einmal hat er geschadet und einmal hat er genutzt. Wir können also nicht sagen, dass dieser oder jene Zauber böse oder schlecht bzw. gut ist. Der Zauber war immer der gleiche. Es kommt nur darauf an, wie man ihn anwendet und einsetzt}, sprach Professor Elber. \enquote{Zwar gibt es viele Zauber, die unseren moralischen Vorstellungen nach als schlecht eingestuft werden, aber das sagt nichts über die Magie selber aus. Sie ist lediglich ein Werkzeug dessen wir uns bedienen. Denken Sie zum Beispiel an einem Hammer. Damit kann man sowohl gutes als auch böses machen.}

\enquote{Lest nun das erste Kapitel in eurem Buch. Ihr müsst euch erst die Grundlagen in diesem Kapitel aneignen, bevor wir weitermachen können.} Er ging hinter seinen Tisch und setzte sich in den Stuhl, der dort stand. Er betrachtete den Tisch und sein Blick fiel auf die Schublade, die er öffnete und der er ein Buch entnahm. Er legte es auf den Tisch und schlug es auf. Nach einem kurzen Blättern trübte sich seine Miene.
