\chapter{Besuch vom Ministerium}


\enquote{Nachdem Sie nun alle ihre Patroni beherrschen, ist es an der Zeit, denke ich, dass Sie lernen wollen, wie man mit ihnen Nachrichten versenden kann}, sagte Professor Elber, als die gesamte Klasse wieder vor ihm versammelt war. Alle Schüler nickten. Selbst die Slytherin waren in seinem Unterricht voll dabei. Vermutlich, weil es auch dunkle Künste gab. \enquote{Das ist ganz einfach. Zuerst die Theorie, dann die Praxis. Der erste Schritt ist, dass Sie sich Ihren Patronus erzeugen müssen. Das ist bei allen mittlerweile so. Sie beherrschen ihn. Der nächste Schritt ist etwas schwerer, aber dank Madame Pomfreys Bemühungen, Ihnen die erste Hilfe beizubringen, sind Sie auch da schon gut gerüstet. Sie müssen sich nämlich, nachdem Sie Ihren Patronus vor sich haben, mehrere Dinge vorstellen. Zum einen, die Zielperson, oder eine Gruppe von Personen, wenn diese an einem Ort sind. Zum anderen, die Nachricht. Locken Sie Ihren Patronus zu sich. Stellen Sie sich die Personen und die Nachricht vor und teilen sie dies Ihrem Patronus mit. Das geht auch mündlich, wenn Sie das Bild intensiv in Ihren Gedanken haben.} Der Professor lief im Zimmer umher. \enquote{Wir teilen die Gruppe auf. Nach Häusern getrennt, werden Sie eine Nachricht an Ihr Gegenüber schicken, das auf der anderen Seite der Mauer steht. Die Gryffindors gehen aus dem Klassenzimmer, über den Flur und danach in das Zimmer dahinter.} Professor Elber streckte seine Hand aus und zeigte auf den Raum dahinter. \enquote{Die Slytherin ziehen ihren Namen aus einer kleinen Vase und senden Ihnen dann die Nachricht. Diese müssen Sie sich anhören und dann zurücksenden. Natürlich mit Ihrem eigenen Patronus. Die Gryffindors werden nicht wissen, von wem der Patronus kommt, bis Sie ihn sehen und gegebenenfalls erkennen.}

Damit schickte er die Gryffindors zur Tür hinaus, über den Flur und in das andere Klassenzimmer. Harry wartete mit den anderen eine Weile, bis endlich die Patroni erschienen. Da die Schüler gleichmäßig im Raum verteilt waren, konnte es zu keinen Missverständnissen kommen, weil die Patroni eventuell die falsche Person ansprachen.

Harry wartete, bis sein Patronus ihn erreichte und seine Nachricht überbrachte. \enquote{Na Potter, klappt das auch mit dem Patronus?}, gab der kleine Vogel von sich, als er auf Harrys Knie landete, da er auf einem Tisch saß.

Harry nickte, was der Vogel aber nicht registrierte, da er sich bereits aufzulösen begann. Dann schalt er sich selber über sein Verhalten. Der Patronus war nur der Überbringer. Er konnte die Antwort nicht mitnehmen. \gedanke{Warum eigentlich nicht?}, dachte er sich. Dann erschuf er seinen Hirsch. Sachlich trug er ihm auf, was er zu übermitteln habe. Er sagte ihm exakt dasselbe, was er von dem kleinen Vogel hörte. Dann lief sein Patronus durch die Mauer.

Ein paar Minuten später, nachdem alle Patroni zurückgeschickt wurden und das Ganze noch einmal durchgemacht wurde, kam wieder ein Schwarm von Insekten herein und verkündete, dass die Schüler zurückkehren sollten.

Als alle wieder auf ihren Plätzen saßen, fragte Professor Elber: \enquote{Wie lange hält denn so ein Patronus?}

\enquote{Bei guter Pflege sicherlich ein paar Wochen}, antwortete Dean, was die halbe Klasse zum Lachen brachte. Selbst der Professor musste dabei lachen.

\enquote{Damit liegen Sie gar nicht mal so falsch}, antwortete er. \enquote{Die Lebensdauer hängt von der Menge an magischer Energie ab, die Sie ihm zuteilen. Und der Menge an Konzentration. Sie haben damit die Möglichkeit, eine Nachricht zu hinterlassen, falls Sie einen Ort verlassen und wissen, dass jemand kommt.}

\enquote{Aber wenn man sich nicht auf den Patronus konzentriert, dann vergeht er doch}, warf Hermine ein.

Professor Elber sah sie eine Weile an. \enquote{Das hieße dann, dass Ihr Patronus, wenn Sie ihm eine Nachricht übergeben und losgeschickt haben, nach hundert Metern zerfließt.}

Das war nachlässig von Hermine. Daran hatte sie nicht gedacht.

\enquote{Ihr Patronus hält, solange Sie eine Arbeit für ihn haben.} Dann läutete es. \enquote{Die Stunde ist beendet.}

\trenn

Stumm saß Harry seinem Ahnen gegenüber und betrachtete ihn. Er sah in das schmale Gesicht, die eingefallenen Wangen und den Bart, der Dumbledore Konkurrenz machen konnte. Nur sein Haupthaar hatte oben eine Glatze. Harry hatte es einmal gesehen, als Salazar seinen Hut abnahm und sein Haar über den Kopf strich. Es war früh Morgens und Harry kam gerade vom Laufen und Dumbledore und duschte sich danach. Er saß in seinem Sessel und sah Salazar nur an. Ihm war einfach nur nach Gesellschaft, ohne großartig zu reden.

\enquote{Warum reagiert dein Bild anders? Seid ihr nicht ein und dieselbe Person?}

\enquote{Nein. Es ist mein Ich, beim Zustand des Malens.}

Als er Stimmen und Schritte hörte, verblasste Salazar und verschwand. Das hatte ihm gutgetan. Einfach nur dazusitzen und zu schweigen, ihn nur anzusehen. Harry nahm sich vor, Dumbledore mal über die magischen Bilder auszufragen, besonders, wenn die Person auf den Bildern noch lebten.

\trenn

Harry war gerade auf dem Weg in den Innenhof, um sich mit seinen Mitschülern in das weiche Gras zu setzen, oder auch um ein paar Zauberspiele zu spielen. Eines davon war Zaubertennis, welches mit einem normalen Tennisball gespielt wurde. Allerdings gab es keine Schläger, sondern der Ball wurde durch gezielten Einsatz von Schilden des Protego-Zaubers abgelenkt. Professor Flitwick zeigte ihnen dieses Spiel, bei dem er ziemlich gut war. Vor allem hatte es den Vorteil, dass die Größe keine Rolle spielte. Professor Flitwick trat gerade gegen Hannah Abbott an, als ein Schrei die entspannte Atmosphäre durchbrach.

\enquote{Peeves!}, drang ein Schrei an die vielen Ohren im Hof. \enquote{Was habe ich dir gesagt, über dreimal denselben Streich in den ich hinein tappe?} Peeves gackerte. \enquote{Jetzt bist du fällig.}

Das war die Stimme von Professor Elber. Dann hörte man schnelle Schritte und kurz darauf Peeves, wie er quer über den Hof flog. Ihm hinterher ein wütender Professor. Ein kurzer Lichtblitz kam von links und dann erstarrte Peeves mit einem Grinsen mitten in der Luft. Er konnte nur noch seine Augen bewegen. Professor Elber hielt in seiner Bewegung inne, als er die Schüler und Lehrer im Hof sah.

\enquote{'Tschuldigung}, sagte er. Es braucht etwas, bis er sich gefasst hatte. Dann sagte er: \enquote{Wenn jemand Lust auf eine kleine Vorführung bezüglich Geister möchte, dann wäre jetzt ein passender Zeitpunkt, in den Rosenhof zu kommen.} Er schwang seinen Zauberstab und Peeves wurde wie ein Stück Eisen von einem Magneten hinter dem immer noch wütend blickenden Professor und seinem Zauberstab hergezogen.

Als er an eine Mauer kam, knallte Peeves dagegen, wurde ein paar Zentimeter zurückgeschleudert und wie an einem Gummiband durch den steinernen Bogen gezogen.

\enquote{Au!}, schrie Peeves. Dieser Aufprall hatte wohl sein Sprachzentrum ein wenig gelockert.

Harrys Interesse war nun geweckt. Er packte sein Buch in seinen Rucksack und ging den beiden hinterher. Scheinbar hatten die anderen dieselbe Idee, denn der Innenhof war kurz darauf wie ausgestorben. Als sie im Rosengarten ankamen, war Mister Filch gerade dabei, die Rosen zuzuschneiden. Er hatte eine Schere in der Hand und schnitt einige Zweige weg. Eine magische Schere schnitt auf der anderen Seite desselben Heckengewächses.

Erstaunt drehte sich der Hausmeister um, als er die Gruppe den Hof betreten sah. Peeves schwebte immer noch fluchend in der Luft dem Zauberstab hinterher.

\enquote{Hat Peeves was angestellt?}, fragte Mister Filch genüsslich.

\enquote{Ja}, bekam er als knappe Antwort. \enquote{Dieses Mal hat er es übertrieben. Er hat seine Chance verspielt, sich zu bessern. Das hatte damals nur eine Woche gehalten. Jetzt hat er den Bogen überspannt. Er wird dafür büßen.}

\enquote{Werden meine Rosen Schaden nehmen?}, fragte der Hausmeister ängstlich.

\enquote{Falls es so sein sollte, werde ich den Schaden reparieren und der Garten wird so sein wie vorher.} Er drehte sich kurz um sich selbst und besah sich den Garten. \enquote{Ich muss mich korrigieren. Der Garten wird komplett beschnitten sein, wenn ich hier fertig bin.}

Harrys Blick wanderte umher. Er hatte gar nicht bemerkt, wie sich Dumbledore neben ihn stellte und interessiert das Schauspiel beobachtete.

\enquote{Sie müssen ein paar Sachen wissen, bevor ich beginne}, begann Professor Elber. \enquote{Geister sind bereits tot. Also kann man an ihnen die unverzeihlichen Flüche anwenden, ohne bestraft zu werden. Außerdem spüren Geister keinen Schmerz. Trotzdem krümmt sich ihr Körper, wenn sie \gst \spruch{Crucio} \gst dem Folterfluch ausgesetzt sind.} Peeves Körper begann zu zucken und sich wie unter dem Crucius-Fluch hin und her zu bewegen, als Professor Elber den Fluch sprach. Und langsam begannen sich Laute aus Peeves heraus zu bilden. \enquote{Jetzt fängt Peeves an zu simulieren.} Er brach den Fluch und die Zuckungen hörten auf.

\enquote{Dies war also der Erste der drei Flüche. Kommen wir nun zum zweiten. Dem Befehlsfluch. Leider funktioniert dieser bei Geistern nicht, sodass ich zu Demonstrationszwecken einen anderen Fluch einsetzte, der dem Bereich der dunklen Künste zuzuordnen ist. \spruch{Anima tua, anima mea!}}, sprach er, auf Peeves erneut zeigend.

Peeves' Blick wurde glasig und Professor Elber löste die Bewegungsfessel. \enquote{Du nimmst dir jetzt diese Schere und beschneidest die Rosensträucher hier. Mister Filch wird dir sagen, was und wie.} Peeves nickte. \zauber{Anima tua!}

Peeves setzte sich in Bewegung, nahm die Schere, sie schwebte in seiner kleinen Geisterhand, flog auf Mister Filch zu und erwartete seine Befehle. Dieser war ganz begeistert und erklärte Peeves, was er zu tun hatte. Dann legte er los und schnitt in Rekordzeit die Rosen. Als er fertig war und die Schere fallen ließ, fiel der Fluch von ihm ab und Peeves schaute erstaunt um sich.

\enquote{Du}, schrie er boshaft zu Professor Elber. \enquote{Das wirst du mir büßen.} Er flog auf ihn zu.

\zauber{Avada Kedavra.}

Peeves wurde zunehmend langsamer und schien in der Luft wie versteinert. Er sah nun aus wie eine durchsichtige Steinstatue. Professor Elber schwang seinen Zauberstab und ein kleiner rotierender Wirbel erschien unter ihm. Die steinerne Hülle, die Peeves umgab, floss von ihm ab. Dann wurde auch Peeves, durch viele Flüche die er aussprach begleitet, in den Wirbel gezogen, der in sich kollabierte und mit einem leisen \geraeusch{Plopp} verschwand.

\enquote{Was hast du gemacht?}, fragte Professor McGonagall.

\enquote{Peeves entsorgt}, sagte er todernst, als er seinen Zauberstab einsteckte. \enquote{Er hat es eindeutig übertrieben.} Und dann mit einem gehässigen Lachen. \enquote{Er hat nur eine Chance, wiederzukommen. Er muss schwören, keine Streiche mehr zu spielen.}

\enquote{Das dürfte er, ohne mit der Wimper zu zucken tun.}

\enquote{Ja}, lachte er weiter, \enquote{aber wenn er das tut und einen Streich vorbereitet, wird er magisch dazu gezwungen, selber in die Falle zu tappen, wenn einer Gefahr läuft, Opfer des Streiches zu werden.}

Mister Filch lauschte den Ausführungen nur mit halbem Ohr. Er besah sich die Rosenschnitte und meinte dann anerkennend: \enquote{Zumindest hat Peeves noch etwas Gutes getan, bevor er gegangen wurde. Hähä.}

Da Harry nun etwas freihatte, wollte er es noch einmal versuchen, zu Salazars Bild zu gelangen. Also machte er sich auf den Weg zu den Kerkern. Ab und an traf er einen Slytherin, der ihn komisch ansah, aber noch war er in einem Bereich, der kein Aufsehen erregen sollte. Doch dann wurde es kritisch. Er schaffte es dennoch, indem er sich hinter Rüstungen drückte und in schattigen Bereichen aufhielt. Dann stand er wieder dicht vor dem Bild Salazar Slytherins.

\enquote{So, sind Sie wieder da?}, kam von dem Bild. \enquote{Da haben Sie mich aber schön hereingelegt. Ein Gryffindor, der sich für einen aus meinem Haus ausgibt}, spottete er Harry an.

\enquote{Ich habe nie behauptet, dass ich ein Slytherin bin, wenn hier einer das reininterpretiert hatte, dann warst das du.} In Harry baute sich eine gewisse Sicherheit auf, was das Bild anbelangte.

\enquote{Wie kommen Sie dazu, mich zu duzen, Sie unverschämter Schüler. Ich werde Snape rufen, damit er Sie von hier entfernt\abs}

Doch er verstummte, als Harry das Amulett hinter seiner Schulrobe hervorholte.

\enquote{Woher haben Sie das?}, fragte der Salazar im Bild ganz aufgeregt.

\enquote{Die Frage, die du mir stellen solltest, ist, was ich sehe, wenn ich es in der Hand halte. Aber das bereden wir besser im Inneren.}

\enquote{Ha, versuch es ruhig. Das Amulett zu haben, bedeutet nicht, eingelassen zu werden.}

Harry kam dem Bild näher und berührte mit dem Amulett den Bilderrahmen. Salazar verstummte und das Bild schwang auf. Ob Salazar generell stumm wurde, wenn das Bild aufschwang, oder vor Schreck, konnte Harry nicht sagen.

Er trat hinter die Schwelle in den dunklen Raum. Das Bild klappte hinter ihm zu und Harry stand im Dunklen. Er wartete, bis sich die Alarmanlage, wie ihm Salazar mitgeteilt hatte, bei neuen Personen beim ersten eintreten zeigte. Er stand still und horchte in die Dunkelheit hinein.

\parsel{Ein Eindringling}, hörte er.

\parsel{Ich bin kein Eindringling. Ich darf hier sein. Als Erbe und Nachfahre Slytherins bin ich hier, um meinen Besitz zu übernehmen.}

\parsel{Ein Nachfahre unsseress Herrn isst hier. Bereitet alless vor.} Und nach einer Weile sagte die Stimme: \parsel{Bereit. Jeder, der lichtempfindlich isst, hat ssich verkrümelt.}

Harry hatte ein komisches Gefühl und hielt den Zauberstab direkt vor sein Gesicht um ihn dann zu entzünden. Eine Schlange, die direkt vor seinem Gesicht mit geöffnetem Maul da stand, zuckte zurück.

\parsel{Hast du etwa vorgehabt, mich anzugreifen?}, fragte Harry.

\parsel{Nicht mehr}, antwortete die Schlange. \parsel{Das Amulett isst eindeutig. Und dass ess unss bei Licht bessehen ssagt, dassss ssie hier ssein dürfen, isst eindeutig. Wilkommen Meisster.} Die Schlange verzog sich.

\parsel{Danke}, antwortete Harry. Er trat einige Schritte durch eine Art Tunnel und sah in den dunklen Raum. Mit acht Bewegungen seines Zauberstabes öffnete er die Vorhänge der vier Fenster.

Er sah sich in dem kleinen Wohnzimmer um. Links war an der Wand eine verschlossene Tür. Daneben ein Kamin. \gedanke{Schlafzimmer oder Bad würde sich gut dahinter machen. Wegen des Kamins}, sinnierte Harry. Gerade aus waren vier Fenster, dessen Vorhänge vor kurzem noch geschlossen waren. Er sah hinaus und meinte, dass er auf einem Turm stand, doch eigentlich war er unter der Erde. \gedanke{Muss ein Zauber sein}, dachte er. Er drehte sich nach rechts und sah ein kleines Bücherregal mit Büchern verschiedenen Alters. Er trat um den Tisch und die Sessel herum näher heran und sah grob über die Auswahl an Büchern. \gedanke{Die müssen schon zu Salazars Zeiten alt gewesen sein. Die Zauberer müssen schon lange vor den Muggeln die Bücher erfunden haben. \gst Wann haben die nochmal die Bücher\abs? Nein, das war der Buchdruck}, überlegte Harry.

Er drehte sich herum und sah sich nun die Möbel an. Alte, aber stilvolle Möbel waren da. Drei Sessel und eine Doppelcouch waren um einen passenden Tisch herum gestellt. Der Tisch hatte die Höhe, um bequem Getränke abzustellen und die Füße hinauf zu legen. Für das Mittagessen war er allerdings zu niedrig.

\enquote{Willst du mir nicht den jungen Mann vorstellen, der da in unseren Räumen herumläuft?}, hörte er plötzlich.

Er drehte sich in die Richtung, aus der die Stimme kam, und sah ein Bild über dem Eingang hängen. Darauf waren zwei Personen abgebildet. Eine Frau um die sechzig, schätzte Harry, und Salazar in demselben Alter.

\enquote{Ich kenne den Jungen doch gar nicht. Er behauptet, mein Nachkomme zu sein. Und jetzt hat er sich auch noch gewaltsam Zugang verschafft.}

\enquote{Salazar Slytherin, der junge Mann ist hier hereingekommen und du hast mir immer wieder gesagt, dass hier keiner reinkommt, der es nicht verdient hat, oder hier sein darf. Wie kannst du so etwas behaupten?} Dann gab sie ihm eine Klaps auf den Hinterkopf.

Harry musste grinsen, als Salazar seine Frau böse ansah. Er setzte sich auf einen Sessel und schaute den beiden zu.

\enquote{Der Bengel hat sich als Slytherin ausgegeben, um mich in ein Gespräch zu locken.}

Seine Frau warf einen seitlichen Blick auf Harry, der nur schmunzelnd zusah.

\enquote{Also Mister\abs}

\enquote{Potter, Ma'am. Harry Potter.}

\enquote{Ach, nennen Sie mich Agatha, ich bin da nicht so kompliziert wie mein Mann.}

\enquote{Glauben Sie mir Ma'am, das lässt nach. Er wird sich noch ändern.}

\enquote{Aber sagen Sie mal Harry, wie sind Sie eigentlich an das Amulett gekommen?}

\enquote{Das war ein Geschenk. Ich habe erst später erfahren, was es mit dem Amulett auf sich hat, und dass ich, nachdem ich was sehe, wenn ich das Amulett festhalte, mitgeteilt bekommen habe, dass das nur ein Nachfahre Salazar Slytherins kann.} Agatha bekam große Augen. \enquote{Wissen Sie Agatha, zum damaligen Zeitpunkt habe ich von Slytherin nicht viel gehalten. Er war derjenige, der die Idee der Reinblütigkeit in die Welt gesetzt hat. Schwarzmagier kamen fast ausschließlich aus diesem Haus. Wissen Sie, ich bin unter Muggeln aufgewachsen.}

\enquote{Was?}, kreischte Salazar. \enquote{Das ist ja ungeheuerlich.}

\enquote{Klappe, Salazar. Mit dieser Idee hast du genug Ärger gestiftet.}

\enquote{Was weißt du schon? Diese Idee wird unsere Art schützen.}

\enquote{Irrtum, Salazar}, mischte sich Harry ein, sodass beide zu ihm sahen. \enquote{Diese genetische Inzucht schafft mehr Probleme, als sie löst. Krankheiten durch Jahrhunderte lange Inzucht und Degeneration der Magie sind die Folgen. Wir haben immer noch damit zu kämpfen, dass du diese Idee hattest. Sie mag vielleicht anfangs dazu geführt haben, dass eine Verbesserung zu sehen war, aber es hat sich später rausgestellt, dass das ein Fehler war. Seitdem stehst du bei den anderen drei Häusern auf schwerem Stand. Alle Schüler und Schülerinnen deines Hauses sind mehr oder weniger isoliert. Die anderen arbeiten freundschaftlich zusammen. Deine Idee hat sich so rapide durchgesetzt, dass wir die Folgen noch heute spüren. Du hast es zeitlebens nicht mehr geschafft, deinen Fehler zu korrigieren, obwohl du es versucht hast.}

\enquote{Woher willst du das wissen?}, fragte er ohne Anflug von Sarkasmus und Widerwillen.

Harry stutzte. \enquote{Hast du mir etwa was vorgemacht, um deinen Ruf zu schützen? Zuzutrauen wär’s dir. Slytherins sind ja bekanntlich verschlagen.}

Salazar zog seinen Kopf ein und murmelte: \enquote{Ja, vielleicht. Ich wusste ja nicht, dass du es weißt. Ich bin hier im Schloss schon ein paar mal angeeckt deswegen.} Und dann wieder normal. \enquote{Woher weißt du von meinem Sinneswandel? Das steht doch nirgendwo.}

\enquote{Ich bin auf ein älteres Ich von dir getroffen und dann habe ich noch von jemandem etwas über dich erfahren. Er scheint dich zu kennen. Ich bin mir nur nicht im Klaren darüber, woher er soviel über dich weiß.}

Salazar und seine Agatha sahen sich an.

\enquote{Wie auch immer, willkommen in unserem\abs äh deinem Heim.}

\enquote{Danke, Agatha. Wo geht es eigentlich hin, wenn ich durch die Türen gehe?} Jetzt bemerkte er auch den Aufgang, der neben der Tür beim Bücherregal, aber an der Wand mit dem Bild, war. Eine Wendeltreppe führte ein Stockwerk höher.

\enquote{Die Tür neben dem Kamin führt ins Bad und zur Toilette. Eine weitere Tür führt auf die Insel im großen See vor dem Schloss. Es ist eine magische Tür und sie lässt sich nur von hier öffnen. Wenn man sie verschließt, kann man von draußen nicht mehr rein. Die Tür neben dem Bücherregal führt zu den anderen Gemeinschaftsräumen und\abs zu den anderen Räumen der Schulgründer.}

\enquote{Was ist mit der Kammer?}, fragte Harry direkt an Salazar gewandt.

Der wurde sofort bleich. \enquote{Welche Kammer?}, stotterte er.

Harry zog eine Augenbraue hoch. \enquote{Die Kammer unter dem Schloss mit einem Basilisken darin, der seit vier Jahren tot ist und der bereits eine Schülerin auf dem Gewissen hat. Sie haust seit fünfzig Jahren auf einem Mädchenklo im dritten Stock hier im Schloss.} Salazar verlor jegliche Farbe. \enquote{Weißt du, ein mächtiger schwarzer Zauberer hat eine gute Freundin von mir dazu gebracht, die Kammer zu öffnen.}

\enquote{Wie überzeugt?}

\enquote{Mit einem Tagebuch. Es war ein Horkrux.}

Salazar bekam große Augen. \enquote{Erzähl weiter.}

\enquote{Wir, das heißt zwei meiner Freunde und ich, haben herausgefunden, wo der Eingang ist. Dort ist auch der Geist des Mädchens. Ich öffnete den Eingang und rutschte mit meinem Freund herunter, da meine andere Freundin versteinert wurde. Sie hatte mit einem Spiegel um die Ecke gesehen und so den Basilisken, wie sie herausgefunden hatte, gesehen. Unten angekommen bin ich dann weiter gegangen und habe die Kammer geöffnet.}

\enquote{Du kannst Parsel?}

\enquote{Sicher. Auf jeden Fall habe ich es geschafft, mithilfe eines Phönix’, der dem Basilisken die Augen ausgekratzt hatte, ihn mit Gryffindors Schwert zu töten. Dann habe ich den Horkrux mit einem Basiliskenzahn zerstört, den das Tier leider in meinem Arm versenkt hatte, während unseres Kampfes. Zum Glück haben mich die Tränen des Phönix geheilt. Es hat nie einer, außer meinen drei Freunden und den Lehrern, erfahren, was wirklich unten in der Kammer passiert ist.}

\enquote{Das war aber noch nicht alles?}, fragte Agatha Slytherin ehrlich interessiert nach.

\enquote{Nein und dabei soll es auch bleiben. Das ist das Wichtigste. Alles andere sind Details, die nicht so wichtig sind. Euch habe ich das als einzigen Außenstehenden erzählt.} Beide nickten. \enquote{Wohin geht denn der Aufgang? \gst Halt, führt der Gang jetzt auch zur Kammer?}

Agatha sah ihren Mann fragend an.

Dieser nickte. \enquote{Bei der Abzweigung Hufflepuff und Ravenclaw einfach gerade aus. Drücke dein Amulett gegen die Wand. Das ist eine Abkürzung. Der andere Weg ist farbig hinterlegt.}

\enquote{Die Wendeltreppe führt zu unserem\abs alten Schlafzimmer und zu weiteren Toiletten, sowie Gästezimmern, obwohl diese selten benutzt wurden. Nach unserer Rückkehr nach Hogwarts, ohne dass mein Mann wieder lehrte, hatten wir kaum noch Gäste.}

\enquote{Ihr kamt wieder zurück?}

\enquote{Ja, es wurde nicht sehr begeistert aufgenommen, aber in späteren Jahren haben wir hier wieder ein Zuhause gefunden. Kurz danach ist dieses Bild entstanden. Wir haben praktisch mitbekommen, wie wir gestorben sind. Die Elfen hatten den Auftrag, uns in der Familiengruft beizusetzen. Seitdem war es hier dunkel und wir haben geschlafen. Wir wissen nur noch, dass etwa tausend Jahre vergangen sind, bevor Sie aufgetaucht sind.}

Harry sah auf seine Uhr. \enquote{Es ist schon spät, ich sollte\abs}

\enquote{Hier schlafen. Rufen Sie einen Hauselfen, der wird Ihnen das Zimmer herrichten.}

\enquote{Ich denke, das werde ich selber schaffen}, sagte Harry.

\enquote{Wie?}, meinte Salazar. \enquote{Selber?}

\enquote{Na ja}, gab Harry zu, \enquote{ich habe ein eigenwilliges Verhältnis zu Hauselfen.} Und er fügte hinzu: \enquote{Aber das ist eine andere Geschichte.} Er stand auf und ging nach oben. Er öffnete die erste Tür und sah in das Bad. Dann schloss er die Tür und widmete sich der zweiten und sah jetzt in das Schlafzimmer der Slytherins. Es sah aus wie eine luxuriöse Version der Schlafzimmer, die er einmal gesehen hatte, nur konnte er sich beim besten Willen nicht mehr erinnern, wo. Er rief nach Kreacher und trug ihm auf, ihm seinen Schlafanzug zu bringen. Nachdem er wieder da war, sagte er noch: \enquote{Ich werde diese Nacht hier schlafen. Sag bitte Hermine und Ron, dass ich einen sicheren Schlafplatz heute habe und sie sich keine Sorgen machen müssen. Er ist aber in keinem der Gemeinschaftsräume.}

Kreacher nickte, verneigte sich und verschwand. Dann richtete sich Harry her und ging ins Bett. Er sah an die Decke und dachte nach. Über sich, über Salazar, dessen Frau, über Luna, Ginny und sich. Er dachte über alle nach, die ihm etwas bedeuteten. Langsam dämmerte er ins Reich der Träume.

Schon lange hatte er nicht mehr so gut geschlafen. Er hatte wieder an Erkenntnissen gewonnen. Er wusste nicht, was er davon seinen Freunden erzählen wollte. Er wusste nicht, ob sie es verstehen würden. Mit diesen Gedanken erwachte er und fand unten Kreacher mit einem Frühstück vor. Er versuchte, Kreacher zu überreden, mit ihm zu essen, doch er schaffte es nicht. Dann verschwand Kreacher mit den Resten in die Küche.

Jetzt meldete sich Agatha wieder: \enquote{Ich glaube, ich verstehe Sie Harry. Was war Ihr erster Kontakt mit einem Hauselfen?}

\enquote{Ein Elf, der mich warnte. Er diente einem dunklen Magier. Zumindest bin ich davon überzeugt. Er hat ihn nicht sehr gut behandelt.}

\enquote{Der Elf ist gestorben?}

\enquote{Nein, er wurde befreit.}

\enquote{Befreit?}

\enquote{Ja, ich habe seinen Meister\abs hereingelegt.}

Agatha schmunzelte. \enquote{Die Geschichte müssen Sie mir mal erzählen, Harry.}

Harry nickte und verließ die Räumlichkeiten durch die Gänge und tauchte im Gemeinschaftsraum der Gryffindors auf. Keiner entdeckte ihn, als er neben der Tür durch die offene Wand kam. Er holte seine Schulsachen und machte sich auf den Weg.

Jetzt hatte Harry wieder Unterricht im Fach \VgddK. Professor Elber war bereits im Klassenzimmer und lehnte sich an sein Pult. Die Klasse träufelte so nach und nach ein und jeder nahm seinen Platz ein. \enquote{Heute}, so fing er an, \enquote{werden wir uns mal ein wenig mit der zauberstabfreien Magie befassen. Und demnächst, nachdem ihr bei Professor Flitwick schon Zaubern ohne Worte durchgenommen habt, das Ganze auch ungesagt. Wir werden nicht viel schaffen, aber kleine Aufrufzauber oder ein einfaches Alohomora dürfte drin sein.}

Ein leises Raunen durchfuhr die Klasse. \enquote{Dies ermöglicht es Ihnen, aus einer misslichen Situation einen kleinen Vorteil zu erlangen und eventuell zu verschwinden}, fuhr Professor Elber seinen Vortrag fort. \enquote{Fangen wir mal mit was Einfachem an.} Er zog seinen Zauberstab und legte ihn auf seinen Tisch, setzte sich dahinter und sagte dann: \enquote{Nehmen Sie bitte alle Ihren Zauberstab heraus und legen Sie ihn vor sich auf den Tisch.}

Die ganze Klasse tat wie geheißen und alle nahmen ihre Zauberstäbe heraus und legten sie vor sich auf den Tisch. Professor Elber fasste sich mit drei Fingern an sein Kinn und schaute auf seinen vor ihm liegenden Zauberstab. Er strich an seinem Kinn entlang und sah danach hoch. \enquote{Erinnern Sie sich noch an ihre erste Stunde Flugunterricht bei Madame Hooch?}, fragte er.

Die ganze Klasse bejahte.

\enquote{Wie genau spielte sich das ab?} Er zeigte mit einem Finger auf einen seiner Schüler, worauf dieser anfing zu erzählen.

\enquote{Wir haben uns links neben unseren Besen aufgestellt, dann unsere rechte Hand über ihn ausgestreckt und danach \accentuate{Auf} gerufen.}

\enquote{Nun, das hier funktioniert ganz ähnlich}, fuhr Professor Elber fort. Er streckte seine Hand über seinen Zauberstab aus und rief: \enquote{Auf}. Sein Zauberstab hob vom Tisch ab und war kurz darauf in seiner Handinnenfläche, welche sich reflexartig schloss. Er drehte seine Hand und öffnete sie danach, nahm seinen Zauberstab aus seiner Hand und legte ihn wieder zurück auf den Tisch. Danach faltete er seine Hände und legte diese auf dem Tisch ab. \enquote{Habt ihr alle gut aufgepasst? Oder soll ich es nochmal vormachen?} Die Klasse stimmte einstimmig für nochmal vormachen. Professor Elber hob abermals seine Hand über seinem Zauberstab und rief: \enquote{Auf}. Der Zauberstab schnellte in seine Handfläche und seine Hand schloss sich. Er drehte abermals seine Hand und legte danach seinen Zauberstab auf den Tisch. Er stand auf und fuhr fort. \enquote{Wenn Sie mit Ihrem Besen gut umgehen können, brauchen Sie hier nur etwas mehr Konzentration und Ausdauer, denn die Besen sind dafür extra vorbereitet worden. Ihre Zauberstäbe hingegen nicht. Bitte, versuchen Sie es alle.}

Er lief durch die Reihen, während alle versuchten, ihren Zauberstab in ihre Hand zu rufen.

\enquote{Sobald es Ihnen einmal gelungen ist, legen Sie ihn bitte zurück und wiederholen die Übung.}

Die Stunde verflog rasch. Jeder schaffte es nach dem Ende der Stunde, seinen Zauberstab aufzurufen. Selbst vom Boden mit der Hand ganz nach unten gestreckt klappte es. Dann läutete es und die Stunde war beendet.

\enquote{Wo warst du letzte Nacht?}, fragte Ron Harry.

\enquote{In einem Bett\abs alleine}, fügte er hinzu. \enquote{Ich habe so gut geschlafen wie schon lange nicht mehr. Das Bett war sehr angenehm.}

\enquote{Wo hast du geschlafen? In \accentuate{unserem} Gemeinschaftsraum?}, fragte Hermine.

\enquote{Besser}, antwortete Harry.

Ron und Hermine sahen ihn angespannt und neugierig an. Er fiel mit ihnen etwas zurück.

\enquote{Ich war in}, doch plötzlich fiel ihm wieder ein, dass er noch nichts erzählen wollte, \enquote{einem Bett im Raum der Wünsche. Ich hatte eine Idee und wollte ein Zimmer. Eines, wie es von einem der Gründer bewohnt hätte sein können. Es sah Klasse aus.}

\enquote{Du hast was?}, fragte Ron aufgeregt. \enquote{Und, wie ist Gryffindors Zimmer so?}

\enquote{Als ob der Raum wüsste, wie Gryffindor so eingerichtet war. Das Zimmer entsprang nur Harrys Fantasie. Und überhaupt, wie kommst du auf Gryffindor? Es könnte auch Hufflepuff oder Ravenclaw sein.}

\enquote{Du vergisst Slytherin, Hermine}, fügte Harry hinzu.

\enquote{Damit macht man keine Witze, Harry}, beschwerte sich Ron.

\enquote{Wenn du meinst}, sagte Harry und lief schneller, um zur nächsten Stunde zu gelangen.

Ron und Hermine sahen sich nur verständnislos an.

\enquote{Willst du mir damit sagen}, fragte Hermine, als sie wieder mit Harry gleich auf war, \enquote{dass du ein Zimmer von Slytherin hast erscheinen lassen? Wie bist du denn da drauf gekommen?}

\enquote{Ich musste gerade an ihn denken. Ich habe beim Laufen das Amulett gespürt. Dann habe ich an ihn gedacht. Und so hat mir der Raum der Wünsche sein Zimmer erschaffen. Zumindest glaube ich das.}

\enquote{Ja aber\abs}

\enquote{Nichts aber. Ich bin einer der Nachkommen Slytherins. Also sein Erbe. Ich werde mich deswegen nicht verstecken. Außerdem habe ich interessante Dinge über ihn herausgefunden.} Und so erzählte Harry ihnen, was er wusste und mitteilen durfte. Bei Ron und Hermine waren die Geheimnisse sicher.

\trenn

Er schlich wieder einmal in die Küche, um sich mit Kreacher zu unterhalten. Je länger er mit Kreacher Zeit verbrachte und sich mit ihm unterhielt, desto besser verstanden sie sich. Auf einmal standen drei kleine Elfen vor ihm und schauten ihn kurz an. Danach verschwanden sie wieder. Harry wunderte sich darüber, dachte sich aber nichts Großes dabei.

\enquote{Und Kreacher, wie geht es dir?}, fragte ihn Harry.

\enquote{Kreacher geht es gut, Sir.}

\enquote{Wie lange hast du den Blacks gedient?}, fragte Harry den Elfen.

\enquote{Seit Kreachers Geburt.}

\enquote{Und wann war das?}

\enquote{Kreacher weiß es nicht genau, aber er erinnert sich an ein Weihnachtsfest im Jahre 1607. Er dürfte damals gerade ein Jahr alt gewesen sein.}
% 1996 minus 390 Jahre; Elfen können bis zu 400 Jahre alt werden

%1606
Harry rechnete nach. \gedanke{Das sind 390 Jahre.} \enquote{Und wie alt können Elfen werden?}

\enquote{Kreachers Art kann bis zu 400 Jahre alt werden.}

\enquote{Das heißt, du wirst bald sterben?}

\enquote{Ja Meis\gst Sir Harry}, antwortete der Elf.

Harry fragte weiter. \enquote{Was soll mit deinem Körper passieren, wenn du nicht mehr bist?}, fragte Harry nach.

Der Elf starrte ihn an. Harry sah, wie sein Kopf arbeitete. \enquote{Kreachers Vorfahren und Kollegen wurden alle geköpft und deren Köpfe wurden in der Großen Halle des Hauses aufgehängt.}

\enquote{Das hast du mir bereits erzählt, Kreacher. Aber, was soll mit \accentuate{dir} passieren, wenn du gestorben bist?}, fragte Harry erneut.

Kreacher dachte nach. Er dachte lange nach. Harry bekam in der Zwischenzeit von einem der drei Hauselfen, die ihn vorhin angesehen hatten, ein Glas Kürbissaft.

\enquote{Wenn Sir Harry damit einverstanden ist, dann würde Kreacher gerne\abs verbrannt werden.}

Harry nickte und verstand.

\enquote{Noch etwas anderes, Kreacher. Du kennst dich doch in meinem Haus\abs dem Haus der Blacks aus.}

Kreacher nickte.

\enquote{Hast du da ein Medaillon gesehen? Es ist etwa daumenlang und einen halben Handteller groß. Es ist rechteckig und hat eine Schlange auf dem Deckel, die mit einem S verworren ist.}

Kreacher fing an zu schluchzen und schnappte sich eine Pfanne, die ihm Harry wegnahm.

\enquote{Du wirst dich nicht bestrafen, Kreacher.} Der Elf nickte. \enquote{Setzt dich zu mir und erzähl.}

Also fing Kreacher an zu erzählen, wie er Regulus Black, seinen liebsten Meister nach Harry, einen Gefallen erfüllen sollte. Er wurde mit Voldemort losgeschickt auf eine Mission. Er war in einer Höhle und fuhr auf einem kleinen Boot über einen See zu einer Insel. Dort musste er eine Flüssigkeit trinken. Sie brannte furchtbar und er bekam Durst. Er musste alles trinken. Dann legte Voldemort ein Medaillon hinein und verschwand. Kreacher erzählte, wie er ans Wasser ging, um zu trinken. Inferi zogen ihn unter Wasser und versuchten ihn zu ertränken. In diesem Moment konnte sich Harry wieder an die Inferi im Wasser im See erinnern. Dann erzählte Kreacher, wie er heimkehrte. Harry verstand nicht. Erst als Kreacher deutlich wurde, verstand er, dass Kreacher einfach aus der Höhle apparierte. \gedanke{Voldemort hat einen Fehler begangen}, folgerte er. \gedanke{Einen sehr großen sogar.} Dann erzählte Kreacher weiter. Wie er von Regulus ausgefragt wurde und er ihn mit in die Höhle nehmen musste. Er musste Regulus versprechen, niemandem aus der Familie hiervon zu erzählen. Dann nahm er Kreacher das Versprechen ab, das Medaillon zu zerstören. Er Trank die Flüssigkeit, legte eine Fälschung hinein, ging voller Durst ans Wasser und wurde nach unten gezogen.

Kreacher weinte während er von Regulus erzählte und Harry musste ihn beruhigen. Er musste dieses Medaillon haben.

\enquote{Kreacher, kannst du mir das Medaillon\abs?} Er unterbrach sich. \enquote{Kannst du mich zum Grimmauldplatz mitnehmen?}

Kreacher nickte, nahm Harrys Hand und verschwand mit ihm. Er tauchte mit ihm im Salon auf, in dem auch die Vitrinen standen. Kreacher zeigte auf die Vitrine, in der das Medaillon lag. Harry kam dem Holzkasten mit den Glasscheiben näher und öffnete ihn. Dann nahm er das Medaillon heraus und hielt es an der Kette. Er schloss die Tür wieder und setzte sich auf den Boden. Kreacher saß ihm gegenüber.

\enquote{Von ihm geht eine Menge dunkler Magie aus}, erklärte Kreacher. \enquote{Kreacher konnte es nicht zerstören. Er hat jeden Zauber versucht, den er kann. Er hat versucht, es mit Messern und Gabeln zu zerstören und auch ins Feuer hat er es geworfen, doch nichts half. Kreacher hat sich oft dafür bestraft.}

\enquote{Kreacher, das wirst du nicht tun. Du wirst dich nicht mehr deswegen bestrafen.}

Kreacher nickte erleichtert. Die Strafen hatten seinem Körper schwer zugesetzt.

\enquote{Das ist keine normale schwarze Magie. Das ist mehr. Es lebt, es pulsiert}, sagte Harry, das Medaillon betrachtend. \enquote{Es ist etwas, was verhindert, dass Vold\abs naja du kennst ihn, Kreacher \gst dass er stirbt.}

Der Elf nickte. \enquote{Kreacher kennt einen solchen Zauber, den der Herr meint. Es gibt in der Bibliothek ein Buch darüber.}

\enquote{Bevor wir gehen holst du es mir bitte und tust es in meinen Koffer in Hogwarts. Ich habe Vorkehrungen getroffen, dass niemand ran kann. Er macht für dich eine Ausnahme. Es ist der Kleine.}

Kreacher nickte erneut. \enquote{Darf Kreacher etwas vorschlagen?} Harry nickte. \enquote{Gebt Kreacher das Medaillon in die Hand. Dann nehmt ihr ein Stofftuch und nehmt es damit entgegen und wickelt es ein. Es versucht nämlich diejenige Person zu beeinflussen, die es als Letztes in der Hand hatte. Auf Elfen hat es keine Auswirkung. Und da ich der Letzte bin, der es in Händen halten wird, kommt keiner von uns zu Schaden.}

Harry nickte und gab Kreacher das Medaillon. Dann zog er aus seiner Tasche ein Taschentuch und nahm das Medaillon mit dem Taschentuch von Kreacher entgegen. Er wickelte es in das Taschentuch ein und gab es Kreacher.

\enquote{Lege das zu dem Buch in meinen Koffer. Wird Zeit, dass wir gehen.}

Kreacher nickte, stand mit Harry auf und nahm ihn bei der Hand. Dann apparierte er in die Bibliothek des Hauses und ließ ein Buch zu sich heranschweben. Als er es in seiner Hand hatte, war er mit Harry auch schon verschwunden. Sie tauchten an derselben Stelle in Hogwarts wieder auf, von der sie verschwunden waren.

\enquote{Warum hier?}, fragte Harry.

\enquote{Elfen können zwar durch die Anti-Apparitionszauber der menschlichen Zauberer hindurch, müssen aber an dieselbe Stelle wieder zurück, bevor sie innerhalb weiterapparieren}, erklärte eine der Elfen aus der Küche, die gerade an beiden vorbeilief und nicht im Geringsten darüber erstaunt war, dass beide plötzlich auftauchten.

Harry nickte. \enquote{Danke dir.} Dann wandte er sich Kreacher zu. \enquote{Ich werde mich um das Medaillon kümmern. Darf ich es auspacken und untersuchen, oder hat das auch Auswirkungen auf mich?}

\enquote{Nein Sir. Nur ein körperlicher Kontakt wirkt sich aus. Es ergreift von einem Besitz.}

\enquote{Du kannst die Sachen aufräumen, Kreacher.}

Der Elf verschwand.

Nachdenklich verließ er die Küche. \gedanke{Wie bei Ginny in ihrem ersten Jahr. Das Tagebuch hat Besitz von ihr ergriffen. Dumbledore weiß vermutlich, was es ist. Wenn ich nicht herausfinde, wie man es zerstört, dann gehe ich zu ihm.}

Auf dem Weg zurück traf er auf Professor Elber, der an einem der vielen Leuchtkörper in Hogwarts scheinbar seinen Frust ausließ. Gerade lief er an einer Lampe vorbei, als er abrupt anhielt und die Lampe anstarrte. Dann zog er seinen Zauberstab und sprengte sie von der Wand weg. Verächtlich schnaufend trat er näher an die Wand und setzte seinen Zauberstab scheinbar in der Luft an und fuhr von unten nach oben die Kontur der Lampe nach. Sein Zauberstab näherte sich der Wand. Die Lampe baute sich, seinem Stab folgend, auf, verband sich am nächsten Punkt mit der Wand und bildete sich weiterhin aus dünner Luft zu einem massiven Messing-Gebilde aus. Als sich der Schirm ausgebildet hatte, stoppte die Bewegung des Stabes und die Lampe begann zu leuchten. Mit einem achtlos ausgeführten Schlenker verschwand auch die zerstörte Lampe und das zerbrochene Glas vom Boden vollständig.

\enquote{Darf ich Sie was fragen, Professor?}, fragte Harry, als er sich ihm näherte. Professor Elber hob eine Augenbraue. \gedanke{Ach ja: \enquote{Sie können mich alles fragen, was Sie wollen. Ich werde Ihre Fragen immer vertraulich behandeln. Sie bekommen nur nicht auf jede Frage eine Antwort}}, dachte Harry und begann: \enquote{Warum haben Sie gerade die Lampe zerstört?}

Jetzt lächelte sein Professor. \enquote{Wohin sind Sie unterwegs?}, fragte er.

\enquote{Zum Essen}, antwortete Harry.

\enquote{Gut, gehen wir ein Stück.} Damit setzte er sich in Bewegung und Harry lief neben ihm her. Nach einigen Schritten sagte er: \enquote{Ist Ihnen noch nie aufgefallen, dass einige Lampen in Hogwarts gelegentlicher Auffrischung bedürfen, andere hingegen nicht?}

Jetzt begann Harrys Hirn zu arbeiten. Doch als sie in der Großen Halle ankamen, konnte er ihn nichts mehr fragen, da sich ihre Wege trennten.

\trenn

Er saß wieder alleine im Gemeinschaftsraum. Es war Sonntagnachmittag und alle waren draußen. Doch Harry hatte keine Lust. Er hatte Ron, Hermine und den anderen gesagt, dass er nicht mitgehen würde. Diese Nacht war zu anstrengend für ihn gewesen. Er hatte wieder schlecht geschlafen und saß auf einem Sofa in der Ecke und dachte an die schrecklichen Träume und Visionen der letzten Nacht zurück.

\begin{rueckblick}
All dieses nur, weil er wieder einmal unter seinem Tarnumhang im Schloss herumgeschlichen war und die Punkte auf der Karte auszumachen versuchte. Zumeist waren es simple kleine Räume mit jeder Menge Plunder. \gedanke{Mister Filch hätte seine Freude daran}, dachte er noch bei sich. Einmal hatte er eine Passage entdeckt, die ihn vier Etagen auf einmal überbrücken ließ. Und ein anderes Mal ein geheimer Zugang zu einem der Vertrauensschüler-Bäder. Doch wenn er baden wollte, ging er in Salazars Räumen baden. Ron und Hermine hatten ihm dabei geholfen, die Punkt zu finden. Doch noch immer waren nicht alle Punkte gelöst.
\end{rueckblick}

Er sah, wie Voldemort jemanden ermordete, wie er versuchte, Informationen aus jemanden herauszuquetschen. Er war schrecklich müde und schlief ein. Als er erwachte, saß Luna ihm gegenüber. \enquote{Luna? Wie bist du hier hereingekommen.} Doch sie verschwand wieder. Zurück blieb nichts als leere Luft.

\gedanke{Na toll. Jetzt fantasiere ich auch noch}, dachte sich Harry. \gedanke{Ich sehe schon Gespenster.} Doch kaum hatte er das gedacht, schwebte schon Salazar Slytherin vor ihm im Raum. Immer noch müde schloss er die Augen und sagte nur matt. \enquote{Hallo Salazar. Bist du auch eine Einbildung?}

Dann durchfuhr ihn ein eiskalter Schauer. Harry schrak hoch und Salazar war verschwunden. \enquote{Hinter dir}, hörte er plötzlich. Er drehte sich um und sah in das Gesicht seines Ahnen. \enquote{Ich dachte, an so einem schönen Tag würdest du nach draußen gehen.}

\enquote{Ich bin müde, Salazar.} Und Harry drehte sich wieder um und ließ sich zurück auf das Sofa fallen. Ihn durchfuhr abermals ein eiskalter Schauer.

\enquote{Jetzt besser?}, fragte ihn sein Ahne mit schelmischem Grinsen im Gesicht.

\enquote{Etwas}, gab er matt zurück.

Dann begann er sich wieder daran zu erinnern.

\begin{rueckblick}
\enquote{Warum}, fragte Harry, \enquote{bist du ein Geist? Warum habe ich dich bisher nie gesehen?}

Salazar senkte seinen Kopf. \enquote{Als ich gestorben bin, entschied ich mich fürs Weitergehen. Aber vor kurzem hatte ich eine interessante Erscheinung.  Jemand sagte mir, dass du Probleme hast, dass du eine Aufgabe hast, eine Last. Und ich kann dir dabei helfen. Ich wurde gebraucht. Ich bin zurück, ja, aber nicht für ewig. Es braucht einige Zeit, bis ich länger in dieser Welt verweilen kann. Und nein, Harry. Ich kann dir keine Fragen über das Leben danach beantworten}, sagte er, als er Harrys nachdenkliches Gesicht sah. \enquote{Ich kann dir nichts über deine Mutter oder deinen Vater erzählen. Wenn ich in dieser Welt bin, weiß ich nichts mehr von der anderen. Und wenn mir doch etwas in den Sinn kommt, dann kann ich es nicht erzählen. Sobald ich anfangen möchte, vergesse ich es.}

\enquote{Das heißt, du kannst mir nichts über meine Eltern erzählen!}, folgerte er laut.

Salazar nickte.
\end{rueckblick}

Harry saß mit eingeknicktem Kopf da. Er zog seine Pantoffeln aus und zog die Beine zu sich heran auf das Sofa. Dann sah er Salazar wieder an.

\enquote{Schlaf etwas, mein Junge. Wenn du wieder aufgewacht bist, dann wird es dir besser gehen und du wirst dich mit deinen Freunden draußen Treffen}, sagte sein Urahn.

\enquote{Salazar?}

\enquote{Hmm!}

\enquote{Ich habe bisher mit keinem über dich gesprochen. Nur, dass ich dein Nachfahre bin.}

\enquote{Das ist gut so. Noch ist es dafür nicht an der Zeit.}

Salazar schwebte auf Harry zu und legte eine Hand auf seine Schulter.

\enquote{Ich wollte nicht, dass du\abs dass ich wieder von anderen\abs}

Doch dieses Mal war der Kontakt von Salazar herrlich warm. Die Wärme breitete sich in Harrys ganzem Körper aus und er spürte, wie er langsam müde und seine Glieder schwer wurden. Salazar verschwand und kurz bevor er nichts mehr wahrnahm, sah er undeutlich einen durchsichtigen, glühenden Umschlag in der Luft schweben.

Er dachte an gestern Abend, wie er die Unterhaltung über die Lampen hatte. \gedanke{Neue Lampen}, dachte er. \gedanke{Das ist es. Es ist ein Zauber. Und die neu hinzugekommenen Lampen wegen der Helligkeit sind mit einem anderen Zauber gemacht worden. Deshalb muss man sie ab und an auffrischen. Die Magie ist verbraucht.} Dann dämmerte er wieder weg und nahm nichts mehr wahr.

Erfrischt und gut ausgeruht wachte Harry nach einer halben Stunde wieder auf. Der Umschlag schwebte noch immer vor dem Sofa. Harry fragte sich, was er damit wohl anfangen könnte. Er versuchte ihn zu greifen, aber als er seine Hand danach ausstreckte wurde ihm klar, dass das so wohl nicht funktionieren würde und seine Hand griff ins Leere. Seine Nervensignale erreichten die Hand nicht mehr rechtzeitig. Er zog seinen Zauberstab und versuchte den Umschlag im Raum zu bewegen. Dann öffnete er den Umschlag auf magische Weise und ein Brief kam zum Vorschein. Nachdem er die wenigen Worte dort gelesen hatte, verschwand der Umschlag samt Brief.

Für Harry ging das viel zu schnell, aber der Spruch hatte sich schon in sein Unterbewusstsein gegraben. \gedanke{Mir wird er rechtzeitig wieder einfallen}, dachte er. Er ging in sein Zimmer, um sich ordentliche Schuhe anzuziehen. Dann kam ihm der Spruch wieder in den Sinn. \spruch{Ego Basiliskum per horam quinque iunctio tibi.}
% Ego Basiliskum   per   horam    quinque     iunctio        tibi
% Ich   Basilisk   für Stunden      fünf      Verbunden     mit dir  (du  tu, dir  tibi, es (ist)  est )

Er rannte förmlich, voll gesogen mit frischer Energie durchs Schloss und wäre fast mit Dumbledore zusammengestoßen.

\enquote{Entschuldigung, Albus} und als er merkte, was der hinter sich her schweben ließ, \enquote{gehen wir Schlitten fahren?} Harry biss sich auf die Zunge. Er wollte es nicht so klingen lassen und Dumbledore auffordern mit ihm Schlitten zu fahren.

Doch Dumbledore lachte nur und meinte: \enquote{Genau deswegen wollte ich zum Gryffindor-Turm. Du fehlst noch draußen. Gehen wir?}

Harry lachte seinen Schulleiter an und sagte nur: \enquote{Gerne! Aber nur, wenn ich ihn rauf Ziehen darf} und zeigte auf den Schlitten. Dumbledore lachte und ging neben Harry den Rest durchs Schloss und dann hinaus in die kalte, schneebedeckte Landschaft um Hogwarts. \gedanke{Das würde McGonagall wohl nie machen}, dachte er bei sich.

Draußen angekommen, sank der Schlitten in den Schnee und Dumbledore setzte sich hinter Harry auf den Schlitten, da der Weg zum See ständig abfallend war. Harry lachte die ganze Fahrt über. Bald hatte er das Gefühl, als ob er seine Gesichtsmuskeln danach nicht mehr bewegen konnte. Und auch Dumbledore schien hinter ihm sichtlich Spaß zu haben. Gerade fuhren sie an Hermine, Neville, Ron, Dean und einigen Ravenclaws und Hufflepuffs sowie zwei Slytherin vorbei die sich eine Schneeballschlacht lieferten. Ungläubig schaute die Gruppe sie an, als die beiden an ihnen vorbeifuhren. Unten angekommen, drehte der Schlitten um und fuhr noch schneller nach oben, als sie nach unten gefahren waren. Harry sah mit einem erstaunten Gesichtsausdruck auf die Gruppe, die sich eben noch mit Schneebällen beworfen hatten, als sie zusammenstanden und geschäftig miteinander diskutierten.

Harry hätte es sich denken können, als sie ein weiteres Mal auf ihrem Weg nach unten vorbeifuhren. Denn schon flogen Unmengen an Schneebällen auf sie zu, was dazu führte, dass der Schlitten kenterte und Harry mit Dumbledore im Schnee landete. Dumbledore zauberte noch eine kleine Schneewehe vor ihnen herbei, damit sie vor weiteren Angriffssalven geschützt waren, und begann sofort Schneebälle zu formen. Harry schaute ihn erstaunt an und begann gleich danach ebenfalls Bälle zu formen. Dann schossen sie zurück.

Es dauerte eine Weile bis Professor McGonagall angerannt kam und sie aufforderte, die Schlacht zu beenden.

\enquote{Bitte, bitte, hören Sie auf. Das ist nicht der richtige Ort für eine Schneeballschlacht.} Sie kam genau zwischen den Fronten zum Stehen. Dann traf sie ein Schneeball. \enquote{Mister Finnigan, ich muss doch sehr bitten}, sagte sie verärgert und streifte den Schnee ab.

\enquote{Entschuldigung Professor, aber Sie stehen im Weg}, antwortete er.

\enquote{Genau} und ein weiterer Schneeball traf sie und noch einer. Erschrocken drehte sie sich um und sah Dumbledore einen weiteren Schneeball aufnehmen und sich für einen Wurf vorbereiten. \enquote{Sie stehen einfach im Weg. Wenn Sie nicht getroffen werden wollen, dann gehen Sie aus der Schusslinie, Minerva.} Immer noch erschrocken schaute sie Dumbledore an, ungläubig, dass er sie gerade mit einem Schneeball beworfen hatte.

Doch das ließ Minerva McGonagall nicht auf sich sitzen. Sie zog ihren Zauberstab und formte einige Bälle. Dann steckte sie ihn weg und warf sie Richtung Harry und Dumbledore. Die beiden mussten sich ducken, denn Professor McGonagall konnte sehr gut und genau werfen und treffen. Als der Schauer vorüber war, war sie auch schon hinter dem Schutz, den Ron und Neville aufgebaut hatten, verschwunden und in Deckung gegangen. Nach einer halben Stunde mussten Harry und Dumbledore aufgeben, da sie eindeutig in der Unterzahl waren und ihnen langsam die Arme weh taten. Dumbledore nahm seinen Schlitten, setzte Professor McGonagall und sich darauf und fuhr mit ihr hinunter zum See. Die anderen folgten ihnen zu Fuß. Dort standen sie noch eine Weile schweigend da, bevor Professor Dumbledore mit Professor McGonagall zurück zum Schloss liefen. Harry hatte noch schwach in Erinnerung, dass sie ermahnt wurden kein Wort darüber zu erzählen, aber das war wohl eh nur eine halbherzige Drohung, da sich solche Sachen erfahrungsgemäß sehr schnell im Schloss verbreiteten. Und außerdem würden es sich die Schüler nicht nehmen lassen, das weiterzuerzählen.

\trenn

Professor Elber stand nach den Weihnachtsferien wieder vor seinem Pult und sagte: \enquote{Steht nun bitte alle auf und kommt hierher zu mir nach vorne.} Alle standen auf und kamen auf die Empore, auf der Professor Elber stand. \enquote{Und jetzt werfen Sie bitte ihre Zauberstäbe an die Wand da hinten, sodass sie auf dem Boden knapp davor zum Liegen kommen.} Viele Zauberstäbe flogen durch die Luft und auch Professor Elber warf seinen an die Wand. Es waren noch einige in der Luft, als sich plötzlich die Tür öffnete und Professor Dumbledore mit einem gut gekleideten Mann hereinkam.

\enquote{So\abs} fing Professor Elber gerade an, als die beiden hereinkamen.

\enquote{Ich hoffe doch, wir stören nicht}, meinte Professor Dumbledore, als er mit seinem Gast durch die Türe hereinkam und ihn ein Zauberstab knapp verfehlte.

\enquote{Nicht im Geringsten}, erwiderte Professor Elber. \enquote{Was können wir für euch tun?}

\enquote{Dieser Herr hier ist vom Zaubereiministerium und möchte einen Blick in unseren Unterricht werfen. Wir würden gerne ein bisschen zusehen.}

\enquote{Wollen \block{sie} meinen Unterricht diese Stunde führen?}, fragte Professor Elber freundlich. \enquote{Oder wollen sie vielleicht mitmachen? Oder schauen sie nur zu?}

\enquote{Oh nein, nein, ich beschränke mich auf’s zusehen}, meinte der Herr von Ministerium.

\enquote{Und sie, Albus?}, fragte Professor Elber.

\enquote{Ach, gerne. Ich mache gerne mit. Wird sicher lustig.}

Professor Dumbledore betrachtete die Zauberstäbe, die dort am Boden lagen und meinte dann nur lapidar. \enquote{Da muss ich wohl meinen dazulegen.} Ohne eine Antwort abzuwarten, griff er in seine Tasche und legte seinen Zauberstab dazu.

Dann ging er mit dem Mann vom Ministerium ebenfalls auf die Empore und stellte sich an die Seite.

\enquote{So}, fuhr Professor Elber fort. \enquote{Wir machen genau da weiter, wo wir gerade eben aufgehört haben. Nur mit dem Unterschied, dass die Distanz nun größer ist und wir statt dem \accentuate{Auf} nun ein \accentuate{Her} verwenden. Mit entsprechender Übung schaffen Sie das dann auch ohne direkte Worte. Ich fang’ mal wieder an.} Er streckte seine Hand in die ungefähre Richtung, in der der Zauberstab lag und rief dann: \enquote{Her.} Sein Zauberstab erhob sich und fand kurze Zeit später den Weg in seine Hand, die er schloss. \enquote{Bitte, jetzt sind Sie an der Reihe.}

Mehrere \accentuate{Her}’s erklangen durch den Raum und eine Menge Zauberstäbe erhoben sich, um kurz danach in der Hand ihres Besitzers zu landen, bei manchem erst nach dem zweiten oder dritten Versuch. Alle Schüler hatten ihren Zauberstab in der Hand, nur Professor Dumbledore stand noch ohne da.

\enquote{Beeindruckend}, meinte der Mann vom Ministerium. \enquote{Sehr beeindruckend. Unterrichten Sie schon lange hier?}, fragte er Professor Elber.

\enquote{Nein, das hier ist mein erstes Jahr; und vermutlich auch mein letztes. Ich bin nur hier, weil es zurzeit wohl keinen anderen gibt, der dieses Fach unterrichten möchte.}

\enquote{Ah ja}, kam es ihm vom Mann vom Ministerium entgegen. \enquote{Ich denke, ich habe hier genug gesehen. Wir können weiter gehen}, meinte der Mann zu Professor Dumbledore. Dieser ließ ihm den Vortritt und begleitete ihn Richtung Tür.

Jetzt bemerkte Professor Elber ein leises Kichern und Tuscheln in der Klasse. \enquote{Was ist denn so lustig?}, fragte er. Einer antwortete \enquote{Nur ein Zauberstab ist noch übrig und alle Schüler haben ihren. Und Sie haben Ihren auch noch in der Hand.}

Professor Elber betrachtete seinen Zauberstab, lächelte und sah dann zu Dumbledore, der seinen Blick auffing. Professor Elber hob seinen Zauberstab und zeigte auf ihn. Professor Dumbledore drehte sich um und da lag er noch, sein Zauberstab. Er lächelte zurück und zwinkerte Harry zu. Danach streckte er seine Hand aus und ohne ein Wort zu sagen, fand der Zauberstab den Weg in seine Hand. Professor Dumbledore verließ den Raum und schloss hinter sich die Tür.

Das leise Lachen war mittlerweile verstummt.

\enquote{Übt weiter}, meinte Professor Elber und setzte sich auf seinen Stuhl, um einen weiteren Eintrag ins Klassentagebuch zu machen.

