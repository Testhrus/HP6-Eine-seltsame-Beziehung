\chapter{Eine Squib}


Nachdem der Hogwarts-Express eingetroffen war, Harry sich von seinen Mitschülern verabschiedet hatte und von seinem Onkel abgeholt wurde, hievte er seinen Koffer in das Auto und schloss den Kofferraum. Er setzte sich hinten in das Auto auf den Rücksitz neben Hedwig, die in ihrem Käfig saß. Sein Onkel fuhr los und brachte ihn nach Hause. Er konnte nicht erwarten, dass sein Onkel viel mit ihm redete, wenn überhaupt. Nicht nachdem er ihn frisch von der Schule abgeholt hatte. Er wusste, dass ihn sein Onkel und seine Tante nicht besonders mochten. Und selbst Dudley, der in seinem Alter war, mied ihn, als hätte er eine ansteckende Krankheit.

\enquote{Ist mit Tante Petunia und Dudley irgendetwas?}, fragte Harry.

\enquote{Wieso? Was soll mit ihnen sein?}, sagte Onkel Vernon schnippisch. 

\enquote{Na ja, sonst sind sie und Dudley immer mitgekommen}, sagte Harry.

\enquote{Sie hat zu Hause zu tun. Dudley braucht neue Schulsachen}, antwortete Onkel Vernon ihm.

Harry hatte das Gefühl, jetzt nicht weiter mit Onkel Vernon reden zu können, also schwieg er. Zu Hause angekommen lud er wieder seine Schulsachen aus dem Auto, schleppte seinen Koffer in den Hausgang und holte Hedwig in ihrem Käfig nach. Onkel Vernon maulte ihn an, dass er seinen Koffer in sein Zimmer räumen sollte und verschwand in der Küche. Harry holte schnell einen Aufkleber aus dem Koffer, klebte ihn auf und verschloss den Koffer wieder sorgsam. Der Aufkleber sorgte dafür, dass der Koffer leichter wurde und nur einen Millimeter über dem Boden schwebte, denn Harry war noch keine siebzehn und dufte daher außerhalb der Schule nicht zaubern. Dann ließ er ihn in sein Zimmer schweben. Harry lächelte, als er seinen Koffer ohne Anstrengung aufgeräumt hatte. Er stellte Hedwigs Käfig auf dem Tisch ab und öffnete die Käfigtür und sein Zimmerfenster. Hedwig flog etwas im Zimmer umher und landete auf seinem Zimmertisch. Eigenartigerweise auf einer Vorrichtung, die genauso aussah, als wäre sie für Eulen gemacht worden. Harry nahm sich vor, es Vernon und Petunia gegenüber nicht zu erwähnen. Er vermutete, dass es eine Art Geburtstagsgeschenk seines Onkels und seiner Tante sei, sein erstes überhaupt, mal abgesehen von ein Paar Socken oder einem Kleiderbügel, aber er wollte nicht riskieren es wieder zu verlieren. Sie würden bestimmt sagen, sie hätten es vergessen. Sein Geburtstag war zwar erst in ein paar Wochen, aber sein Onkel und seine Tante nahmen es wohl nicht so genau.

Diese waren in den ersten Tagen wesentlich entspannter als in den Jahren zuvor, fiel Harry auf. Aber er machte sich darüber keine Gedanken. Er richtete wie gewöhnlich das Abendessen, spülte ab und ging dann früh zu Bett. Vielleicht lag es auch nur daran, dass sie ihn nur noch diesen Sommer und nächsten Sommer bis zu seinem Geburtstag ertragen mussten. Denn dann wird er siebzehn und kann in der Zaubererwelt tun, was er will. Er darf dann auch außerhalb der Schule zaubern. Doch wenn er das zu Hause tat, würde er bestimmt von seinem Onkel hinaus geworfen werden.

In der zweiten Woche, als Harry mit seinem Cousin, seinem Onkel und seiner Tante am Frühstückstisch saß, flatterte eine Eule herein und setzte sich auf den Fensterrahmen des offenen Küchenfensters. Onkel Vernon erschrak wie jedes Mal und schnauzte Harry an.

\enquote{Was will der blöde Vogel schon wieder?}

Harry stand auf, ohne seine Frage zu beantworten. Er war es leid, ihm jedes Mal sagen zu müssen, dass er es nicht wusste. Er nahm den Brief und staunte, als er nicht an ihn, sondern an Tante Petunia adressiert war.

\begin{brief}
Petunia Dursley,

Ligusterweg 4

Little Whinging

Surrey
\end{brief}

\enquote{Tante Petunia, der ist für dich.} Harry überreichte ihr den Brief.

Unsicher nahm sie ihn entgegen und stieß einen Schrei aus, als sie auf die Rückseite sah. Am Wappen hatte Harry erkannt, dass er von Hogwarts kam. 

\gedanke{Er musste wohl von Dumbledore sein}, dachte er. \gedanke{Bestimmt war das wieder ein Brief so wie letztes Mal.} 

Als Onkel Vernon fragte: \enquote{Was ist denn los?} und nach dem Brief greifen wollte, wich sie zurück und zog sich schnell ins Wohnzimmer zurück. Onkel Vernon stand auf und wollte sich zu ihr setzen.

\enquote{Bleib, wo du bist, Vernon, gehe keinen Schritt weiter}, rief Tante Petunia mit zitternder Stimme. Sie öffnete den Brief und begann zu lesen. Als sie fertig war, sagte sie zu Harry, wie immer betont ärgerlich, aber trotzdem mit einem Zittern in der Stimme: \enquote{Die ganzen sechs Wochen, die du bei uns bist, entfernst du dich nicht weiter als} und sie blickte kurz auf ihren Brief zurück, \enquote{als 1 Kilometer.} Tränen begannen über ihr Gesicht zu laufen, als sie das Wohnzimmer verließ und die Treppen nach oben stürmte.

\enquote{Da siehst du, was du wieder angerichtet hast. Nicht nur, dass du dauernd Briefe bekommst von deinen abnormen Freunden, nein, jetzt bekommt auch noch deine Tante einen und bricht in Tränen aus.} Er stürmte den Gang hinaus und die Treppe hoch. Er wollte wohl seine Frau beruhigen.

Dudley unterdessen schien das ganze wenig zu stören, denn er aß immer noch. Harry nahm sich einen Streifen Schinken und gab ihn der Eule. Die verschlang ihn genüsslich, drehte sich um und flog weg. Gerade in dem Moment kam eine zweite Eule angeflogen. Harry nahm ihr den Brief aus dem Schnabel und überreichte auch ihr einen Streifen Schinken, dieses Mal jedoch von Onkel Vernons Teller. Er schaute auf den Brief. Dieser war an ihn adressiert. Vorsichtig schaute er über seine Schulter, aber Dudley mampfte immer noch sein Frühstück. Harry drehte sich wieder zum Fenster und sah die Eule davonfliegen. Er öffnete seinen Brief und las.

\begin{brief}
Lieber Harry,

ich habe das Zaubereiministerium gebeten, dir die Erlaubnis zu geben auch in den Ferien zu zaubern (natürlich nur, ohne dass es ein Muggel erfährt) und zu trainieren. Nachdem Voldemort so ein immens großes Interesse an dir zeigt und die Dementoren dich und deinen Cousin letzten Sommer überfallen hatten, mache ich mir ernsthafte Sorgen.

Besonders die Tatsache, dass eine gewisse Dame euch letztes Jahr nicht ausreichend unterrichtet hat, hat das Ministerium nachdenklich gemacht.

Es überdenkt zurzeit meinen Vorschlag. Du wirst zu gegebener Zeit eine Eule mit einem blauen Umschlag und einem Siegel des Ministeriums erhalten. Sollte das Ministerium dem zustimmen, darfst du während der Ferien nicht nur deine gelernten Zauber zur Verteidigung einsetzen, sondern du bereitest dich besser vor und wirst kräftig üben, wann immer sich dir die Gelegenheit bietet. Da ich annehme, dass Voldemort oder einer seiner Todesser deine Briefe abfangen könnte, nimm bitte den blauen Umschlag vom Ministerium, um mit Ron und Hermine zu kommunizieren. Entsprechende Briefe und Umschläge für Ron und Hermine liegen bereit und warten nur noch auf das OK vom Ministerium. Lege einen in einem Umschlag verschlossenen und adressierten Brief in den blauen Umschlag vom Ministerium, klappe ihn zu und berühre das Siegel des Zaubereiministeriums mit deinem Zauberstab. Der Brief wird Ron oder Hermine dann zugestellt. Warte aber mit dem Üben, bis du vom Ministerium das OK bekommen hast.

Ron und Hermine wissen nichts von deiner Ausnahmegenehmigung des Ministeriums. Sie wissen nur von den blauen Umschlägen als Eulenersatz.
\signumspace
Albus Dumbledore
\signumspace
PS: Wir werden nächstes Schuljahr wohl kräftig üben müssen.
\end{brief}

Harry grinste und schob seinen Brief in die Hosentasche. Er fühle kurz an seinem Rücken, ob sein Zauberstab noch da war. Er trug ihn seit seinem ersten Ferientag ständig bei sich. Nur für den Fall.

Onkel Vernon kam mit hochrotem Kopf wieder durch die Küchentür und Harry kam es fast so vor, als ob kleine Rauchwolken seinen Ohren entstiegen. 

\enquote{Komm Dudley, gehen wir Einkaufen. Wir brauchen noch neue Schulbücher für Smeltings. Und du Freak räumst den Tisch und spülst ab.}

Onkel Vernon nahm Dudley mit und stieg in den Wagen ein. Seine Tante saß schon und sie fuhren davon. Jetzt fiel Harrys Blick in das Wohnzimmer. Petunia hatte ihren Brief fallen lassen. Er ging hin und schaute vorsichtshalber noch einmal aus dem Fenster. Nachdem er ihn aufgehoben hatte, begann er zu lesen.

\begin{brief}
Liebe Petunia Dursley,

Ich brauche ihnen wohl nicht zu sagen, wie sehr wir uns alle um Harry sorgen. Nach dem Angriff letzten Sommer auf ihn und seinen Cousin Dudley, hat sich hier einiges geändert. Harry schwebt in höchster Gefahr, wenn er sich zu sehr von seinem Haus entfernt. Sie wissen es sicher noch, dass bei Ihnen Harry ein sicheres Zuhause hat. In Ihrem Haus und der näheren Umgebung ist er während seiner Ferienzeit sicher. Die Todesser werden alles versuchen, ihren Neffen zu bekommen. Beschützen Sie ihn so gut es geht. Bleiben Sie weiterhin stark.

Der Bannkreis für Apparitionen wurde auf 2 Kilometer erweitert (Plötzliches magisches Auftauchen an einem Ort). Bitte geben Sie gut auf ihren Neffen Acht und passen sie auf ihn auf. Sie werden ihn nicht mehr lange sehen, falls Voldemort noch länger unter uns weilt.
\end{brief}

Am Freitag danach, nachdem alle mit dem Frühstück fertig waren, meinte Onkel Vernon: \enquote{Vorhin hat der Schneider angerufen. Wir sollen mit Dudley doch mal vorbeikommen, um zu sehen, ob seine neue Uniform denn passt. Und du, du weißt, was deine Aufgaben sind.}

\enquote{Ja Onkel Vernon}, antwortete Harry und begann den Tisch abzuräumen. Seine Gedanken kreisten immer noch um den Brief. Seine Tante hatte ihm gesagt: \gedanke{Die ganzen sechs Wochen, die du bei uns bist, entfernst du dich nicht weiter als 1 Kilometer}. Wieso war sie so besorgt um ihn? Wusste sie etwas, was sie ihm nicht sagte, oder sagen wollte? Hatte sie Onkel Vernon je etwas davon erzählt?

Onkel Vernon verschwand mit Tante Petunia und Dudley durch die Tür nach draußen. Er hörte, wie sie einstiegen und startete den Motor. Als er in Gedanken versunken durchs Fenster hinaus in den Garten blickte, sah er wie eine Eule heranflog. Sie hatte einen blauen Umschlag im Schnabel. Er nahm ihr den Umschlag ab, nachdem er ihr seinen Arm hinhielt und sie darauf landete und drehte ihn um. Auf der Rückseite war das Siegel des Ministeriums zu sehen. Er gab der Eule noch etwas zu fressen und schaute ihr zu, wie sie wieder davon flog. Er öffnete den Umschlag und fand im Inneren einen kleinen Zettel.

\begin{brief}
Ausnahmegenehmigung für Harry Potter

Sehr geehrter Mister Potter,

ihnen wurde auf Bitten ihres Schulleiters Professor Dumbledore und einer ausgiebigen und langwierigen Prüfung, eine Ausnahmegenehmigung erteilt, in den sonst für minderjährige Zauberer schulfreien Zeit ihre Künste auszuüben und zu verfeinern.

Entsprechende zu ihrer Verfügung stehende Zauber sind auf beigelegtem Faltblatt aufgeführt, beziehungsweise deren Sparten, falls ein ganzer Unterbereich darunter fallen sollte.

Ich möchte Sie zudem darauf hinweisen, dass diese Genehmigung jederzeit widerrufen werden kann. Ein entsprechendes Schreiben wird ihnen dann persönlich überreicht werden und bekommt damit Gültigkeit.

Seien sie außerdem gewarnt, dass Sie trotz dieser Genehmigung weiterhin unter Beobachtung durch das Ministerium stehen, sodass ein entsprechender Regelverstoß laut §12a des MZSGB\footnote{Minderjährigen Zauberer Strafgesetzbuch} eine sofortige Rücknahme der Genehmigung und ein entsprechendes Verfahren nach sich ziehen wird.

In der Hoffnung, dass es nicht dazu kommen wird
\signumspace
gez. Mafalda Hopfkirch,

Abteilung für unbefugte Zauberei,

Ministerium für die Vernunft gemäße Einschränkung der Zauberei Minderjähriger.
\end{brief}

Harry konnte sich ein Grinsen nicht verkneifen. \gedanke{Mafalda Hopfkirch, das war doch die Dame, die mich fast aus der Schule geworfen hätte}, dachte er.  Es lag noch eine Kopie der Regeln bei, die er kurz überflog und danach einsteckte. Vorsichtig sah er sich um und begann die restlichen Teller in die Spüle zu stellen. Er zog seinen Zauberstab und zeigte mit einer schwingenden Bewegung auf die Teller und die Spüle. Die Teller begannen sich in die Luft zu erheben; das Wasser fing an aus dem Hahn zu plätschern und die Spülbürste wurde von dem Wasserstrahl umspült. Die Teller näherten sich und auf jeden einzelnen tropfte eine kleine Menge Spülmittel, die aus einer ebenfalls schwebenden Flasche kam. Die Handtücher flogen von ihren Haken und trockneten die sauberen Teller ab. Nachdem das ganze Geschirr sauber gewaschen war, verstummte das Wasser aus dem Hahn; das Spülmittel und die Bürste nahmen wieder ihren Platz ein. Harry schwang erneut seinen Zauberstab und die Teller und Tassen, die Pfannen und das Besteck begannen sich erneut anzuheben. Die Schränke und Türen öffneten sich und die Dinge schwebten an ihren Platz. Die Schränke und Türchen schlossen sich mit einem satten \geraeusch{Bupp} und die Küche blitzte und funkelte, wie schon lange nicht mehr. Er schob seinen Zauberstab wieder ein und dachte: \gedanke{Das ist die einzig wahre Art Hausarbeit zu verrichten.}

Zufrieden mit seiner Arbeit ging er in sein Zimmer und packte erst einmal seinen Koffer aus. Danach setzte er sich an seinen Tisch, schlug ein Buch auf, nahm Tinte und Pergament und machte sich an die Hausaufgaben, die er zu lösen hatte. Durch das offene Fenster kam die beginnende vormittägliche Wärme in sein Zimmer und er spürte einen warmen Luftzug. Nach einer Weile hatte er einen Teil seiner Hausaufgaben für Professor Snape fertig. Das waren seiner Meinung nach die unangenehmsten und so hatte er sie als erstes erledigt. Zumindest den theoretischen Teil. Für den praktischen müsste er sich noch etwas überlegen, denn Zaubertränke brauen konnte er wohl schlecht in der Küche oder seinem Zimmer. Als das Auto seines Onkels die Einfahrt hereinfuhr, klappte er sein Buch zu und räumte die anderen Schulsachen weg. Er schaute aus seinem Fenster und bemerkte, wie Tante Petunia ein Prospekt und einige wichtige Dokumente auf ihrem Arm trug.

Er machte sich auf den Weg nach unten. Als er auf halbem Weg die Treppe heruntergegangen war, öffnete Onkel Vernon die Tür und ließ seine Frau und seinen Sohn herein. 

Onkel Vernon grinste Harry an. 

\gedanke{Das bedeutet nichts Gutes}, dachte Harry. Er ging die restlichen Stufen hinunter und schloss die Haustür, um dann auch in die Küche zu gehen. 

Mit zugekniffenen Augen und einem höchst befriedigtem Gesichtsausdruck sagte Vernon dann: \enquote{Wir fahren in den Urlaub.}

Harry wusste genau, wen er mit \accentuate{wir} meinte. 

\enquote{Und du}, fuhr sein Onkel fort, \enquote{wirst diese Zeit bei Miss Figg verbringen.}

Harry wollte schon anfangen zu grinsen, zwang dann aber doch seine Mundwinkel nach unten und senkte den Kopf.

\enquote{Du wirst dich anständig bei ihr aufführen. Und mach keinen Unsinn. Stell mir bloß nichts an}, sprach Onkel Vernon. Mit \accentuate{mach keinen Unsinn} meinte er Zaubern. Aber Harry hatte damit kein Problem. Glücklicherweise war es ihm ja erlaubt worden und Miss Figg war keine Muggel, sondern eine Squib. Er konnte ihr also ruhig im Haushalt helfen und dabei zaubern. \gedanke{Das werden lustige Ferien}, dachte Harry.

\enquote{Wir fahren nächsten Montag, also in drei Tagen. Du meldest dich derweil bei Miss Figg an und \gst sei mir ja anständig.}

Harry bejahte und verließ das Haus. Die Sonne fing an den Boden und die Luft weiter zu erwärmen. \gedanke{Vielleicht sollte ich doch anfangen Sport zu betreiben}, dachte er. Er machte sich auf den Weg zu Miss Figgs Wohnung. Seit dem Vorfall mit den Dementoren hatte er sie nicht mehr gesehen. Ihr Haus lag nur etwa 50~Meter von dem der Dursleys entfernt. Das sollte also noch innerhalb seines Bannkreises liegen, den Dumbledore in dem Brief an seine Tante erwähnt hatte. Bei Miss Figg angekommen klingelte er und nach kurzer Zeit öffnete Miss Figg ihre Haustüre.

\enquote{Harry}, sprach sie, \enquote{schön, dass du mich besuchen kommst. Komm herein.}

\enquote{Nein, nein Miss Figg. Ich habe keine Zeit. Ich wollte sie nur fragen, ob ich die paar Wochen, die mein Onkel und meine Tante mit Dudley im Urlaub sind, bei ihnen verbringen könnte.}

\enquote{Aber ja, Harry.}

Harry entwich nur ein: \enquote{Danke Miss Figg}, bevor er sich umdrehte und gehen wollte. Da fiel ihm ein, dass sie noch gar nicht wusste, wann er denn zu ihr kommen würde. Er fügte noch hinzu: \enquote{Montag früh.}

Wieder zu Hause bei seinem Onkel und zurück in der Küche sah Harry wie Tante Petunia ihre blitzblank geputzte Küche betrachtete.

\enquote{Harry, ausnahmsweise muss ich dich mal loben}, entfuhr es ihr.

Als er den erstaunten Blick seines Onkels auffing, musste er sich ein Lächeln verkneifen.

\enquote{Danke Tante Petunia, ich hatte auch genügend Zeit dazu.} 

Tante Petunia sah ihn an, als ob sie ihm nicht ganz glauben würde.

Am Montagmorgen, nachdem das Frühstück bereits um sechs Uhr stattgefunden hatte, räumte Harry gerade das Geschirr in die Spüle, als Onkel Vernon mit ein paar Koffern die Treppe herunterkam. \gedanke{Vermutlich lädt er sie in das Auto}, dachte Harry. Er war gerade dabei, mit einem feuchten Tuch den Tisch abzuputzen, als Onkel Vernon wieder hereinkam.

\enquote{Wir fahren jetzt los. Du räumst hier alles auf und machst dich dann auf den Weg zu Miss Figg.} Er drückte ihm einen Hausschlüssel in die Hand. \enquote{Den gibst du bei Miss Figg ab.} Harry bejahte.

Onkel Vernon drehte sich um, blieb kurz stehen und meinte dann: \enquote{Pass bloß auf, dass Miss Figg deine Eule nicht zu Gesicht bekommt.} Er verließ die Küche und ging durch den Flur. 

Dudley war bereits draußen, als seine Tante noch zu ihm sagte: \enquote{Nimm die Eulenstange mit.} Dann verließ auch sie die Wohnung und stieg ins Auto ein.

Harry ging ins Wohnzimmer und beobachtete wie Onkel Vernon in sein Auto einstieg, den Motor anließ und mit Tante Petunia und Dudley davon fuhr.

Harry grinste, zog seinen Zauberstab hervor und ließ den Abwasch sich selbst erledigen. Danach ging er in sein Zimmer, packte seinen Koffer und ließ ihn danach in den Flur schweben. Er zauberte kleine Räder an ein Ende des Koffers und machte ihn leichter, damit er nicht so schwer zu ziehen war.

Zurück in der Küche war das Geschirr bereits gespült. Harry räumte es, wie auch die Tage zuvor als er alleine abspülen musste, auf und machte sich auf den Weg zu Miss Figg.

\trenn

Harry bezog das Zimmer, das ihm Miss Figg für gewöhnlich gab, wenn er bei ihr über längere Zeit übernachtete. Sie trug Hedwig in ihrem Käfig die Stufen hinauf und Harry folgte ihr mit seinem Koffer, der leicht über dem Boden schwebte, denn er brauchte in Miss Figgs Haus keine Räder mehr. 

Sie öffnete Harrys Zimmerfenster und Hedwigs Käfigtür. Dann sagte sie: \enquote{Wenn du fertig bist, komm gleich zu mir. Wir müssen reden.}

Harry nickte und richtete sein Zimmer ein. Er öffnete seinen Koffer und stellte Hedwigs neue Stange auf dem kleinen Tisch in seinem Zimmer ab. Es war ein wenig kleiner als das, welches er bei den Dursleys bewohnte. Und dort war es das kleinste Zimmer im ganzen Haus. Er hatte es noch gut in Erinnerung. Damals durfte er bei Miss Figg nur wenig heraus. Aber seit letztem Sommer hatte sich die Einstellung Miss Figgs ihm gegenüber wohl verändert, denn die Einladung, nachher zu ihr in das Wohnzimmer zu kommen, sofern man es so nennen mochte, war ausgesprochen freundlich. Er verließ sein Zimmer und ging die Treppen hinab, um in das Wohnzimmer zu gelangen.

Dort saß sie bereits und wartete auf ihn.

\enquote{Harry, \gst es gibt ein paar wichtige Dinge, die du dir merken solltest}, sprach sie. \enquote{Erstens \gst vergiss nie deinen Zauberstab. Egal wo du auch hingehst. Zweitens \gst bleib immer in einem Umkreis von 1,5 Kilometern um das Haus deines Onkels und deiner Tante. Gehe nicht weiter, das wäre gefährlich. \gst Ach und übrigens, ich nehme an, du musst noch Hausaufgaben machen. Die kannst du hier im Wohnzimmer machen. Ich helfe dir auch gerne, wenn du willst. Und nenne mich in Zukunft Arabella.}

Harry wusste nicht, was er sagen sollte. So hatte er Miss Figg, oder Arabella, wie er sie jetzt nennen durfte, noch nie gesehen. Bisher war sie immer recht grob zu ihm gewesen und gab ihm das Gefühl, nicht so sehr willkommen zu sein. \gedanke{Das lag wohl daran, dass sie nicht wollte, dass die Dursleys glauben, ich würde mich hier wohlfühlen und mich nicht mehr zu ihr gehen lassen}, dachte Harry.

\enquote{Aber ich denke, sie sind eine Squib und können nicht zaubern?}, antwortete Harry.

\enquote{Das ist richtig}, lächelte sie, \enquote{aber ich habe ein Händchen für Tränke aller Art. Ob zur Abwehr von Krankheiten oder um die Sinne zu vernebeln.}

\enquote{Die habe ich schon fertig}, antwortete Harry leicht indigniert\footnote{empört, grimmig, entrüstet, unbeherrscht, ärgerlich, ungehalten}. \enquote{Aber Sie können sie gerne durchsehen. Vielleicht hilft mir das bei Snape.}

\enquote{Ach! Lehrt der immer noch?}, fragte Arabella.

\enquote{Ja! Sie kennen ihn?}, fragte Harry erstaunt.

\enquote{Du weißt doch, dass ich ihn kenne. Hast du schon vergessen, dass ich auch im Orden bin? Genau wie Severus!}

Harry kam sich blöd vor. Er hatte sie mehrere Male bei den Treffen gesehen, wo sie die neuesten Gerüchte aus der Muggelwelt erzählte. Er lief nach oben und holte seine Zauberbücher, seine Tinte und einige Rollen Pergament. Wieder unten angekommen, gab er Arabella seine Hausaufgaben von Snape und Sprout und machte sich an die anderen, die er noch zu erledigen hatte.

Gegen Mittag entschloss sich Harry nun endlich etwas für seine Figur zu tun und begann sich umzuziehen. Er sagte Arabella Bescheid, dass er noch ein wenig joggen ging und betonte dabei, dass er in der Nähe bleiben würde. Er verließ das Haus, kontrollierte, ob er noch seinen Zauberstab dabei hatte und fing an sich aufzuwärmen. Danach lief er den Block hinunter und zählte seine Schritte, damit er wusste, wann er wieder umkehren musste. Schweißgebadet kam er zurück und stellte sich unter die Dusche. \gedanke{Ab Morgen jogge ich nur noch frühmorgens, wenn es noch kühler ist}, dachte sich Harry. Er trocknete sich ab, zog sich wieder an und kam in das Esszimmer, wo Arabella gerade das Essen fertig hatte und es ihm auf seinen Platz stellte.

\enquote{Dein Aufsatz für Snape ist nicht schlecht, aber ich habe mir die Freiheit genommen, einige Anmerkungen dazuzuschreiben. Du findest sie auf dem Blatt darunter}, sagte sie.

Harry nickte und begann erst einmal zu essen. Während des Essens überkam ihm der Gedanke wie er denn nun in Form bleiben und seine Künste üben sollte. In Arabellas Haus ist nicht viel Platz. \gedanke{Darüber kann ich Morgen auch noch nachdenken}, dachte sich Harry und aß zu Ende. Arabella war eine gute Köchin und sie verwöhnte ihn mit den tollsten Dingen, die sich Harry vorstellen, oder manchmal auch nicht vorstellen, konnte.

Die erste Woche verlief relativ ruhig und Arabella hatte ihm inzwischen das du angeboten.

Harry stand schon bald auf und zog sich seine üblichen Laufsachen an. Es war Dienstag. Obwohl er keinen Trainingsanzug hatte, oder etwas, das als Sportkleidung durchging, hatte er eine normale kurze Hose und ein T-Shirt an. Er ging die Treppe hinunter und zur Tür hinaus, wärmte sich auf und begann einen anderen Weg zu laufen. Er kam an gepflegten Gärten vorbei, die schöner waren, als die seiner Tante. \gedanke{Vielleicht sollte ich in unserem Garten auch ein wenig graben}, dachte sich Harry. \gedanke{Aber nur nachts, wenn es keiner sieht, dass ich dazu keinen Spaten benutze.} Wieder bei Arabella angekommen, duschte sich Harry bevor er zum Frühstück gerufen wurde. Er zog sich schnell an und ging in die Küche um Arabella beim Decken des Tisches zu helfen.

\trenn

Als Harry am nächsten Morgen die Treppen heruntergegangen war, um noch was zu Trinken bevor er wieder loslief, fand er in der Küche einen kleinen Zettel auf dem Tisch.

\begin{brief}
Lieber Harry,

Ich bin kurz Einkaufen. Wir brauchen neue Lebensmittel. Tue nichts Unüberlegtes und bleib in der Nähe. Solltest du angegriffen werden, komm zurück zum Haus, oder das der Dursleys. Dort bist du sicher.
\signumspace
Grüße Arabella
\end{brief}

Harry schmunzelte. Er verließ das Haus, wärmte sich auf und machte sich auf den Weg. Vor lauter Aufregung über den überstürzten Aufbruch seines Onkels und seiner Tante nebst seines Neffen in den Urlaub, hatte er fast vergessen, dass er Morgen Geburtstag hatte. \gedanke{Geburtstag}, dachte Harry. \gedanke{Ich werde sechzehn Jahre alt.} Er lief mehrere Male um denselben Block, bevor er vor Arabellas Haus ankam und sich erst einmal duschte. Der Tag verlief wie immer recht ruhig und Harry erledigte wieder ein paar seiner Hausaufgaben. Er setzte sich ins Wohnzimmer und öffnete eines seiner Bücher um mit den Hausaufgaben von Professor Trelawney zu beginnen. Dieses Mal mussten sie etwas aus dem Rauch von Räucherstäbchen lesen. 

Als Arabella wieder nach Hause kam, roch das ganze Wohnzimmer und Teile der Küche nach Räucherstäbchen. Sie musste erst einmal ein Fenster öffnen, bevor sie überhaupt zu Wort kam. Wieder zurück im Wohnzimmer entdeckte sie Harry, der wie hypnotisiert in den Rauch der Stäbchen starrte. Sie gab ihm einen Schubs und weckte ihn so aus seiner Trance.

\enquote{Was hast du denn hier angestellt? Das halbe Haus riecht nach diesen verdammten Dingern.}

Harry wurde erst jetzt so richtig wach und entschuldigte sich. \enquote{Tut mir leid Arabella, aber das war die Hausaufgabe von Professor Trelawney. Wahrsagen. Ich hätte das wohl besser im Garten draußen gemacht.}

\enquote{Allerdings}, entgegnete ihm Arabella. \enquote{Das kriegen wir die nächsten Tage nicht wieder raus. Mach mir so etwas ja nicht noch einmal.}

Harry schluckte und senkte seinen Kopf. \enquote{Tut mir leid, kommt nicht wieder vor.} Dann fiel ihm etwas ein. Er zog seinen Zauberstab, murmelte etwas und lief im Wohnzimmer herum. Kleine gelbe Funken kamen aus seiner Spitze. Schnell war der Geruch verschwunden. \gedanke{Doch gut, dass Hermine darauf bestanden hat, mir ein paar Haushaltszauber beizubringen}, dachte er.

Nach dem Mittagessen setzte sich Harry an das nächste Fach. Kräuter- und Pflanzenkunde. Denn Arabella hatte bemerkt, dass er einen wichtigen Teil seiner Hausaufgaben vergessen hatte. Das war recht einfach und Arabella schaute ihm interessiert dabei zu. Sie verbesserte einige kleinere Fehler, die sie fand, bevor sie aufstand und das Abendessen herrichtete. Harry half ihr wie immer dabei. 

Nachdem es draußen dunkel geworden war, ging Harry nochmal raus, um zu dem Haus seines Onkels und seiner Tante zu gehen. Einige Zeit stand er vor dem Haus und sah es an. Die Luft war noch immer warm von den Sonnenstrahlen, die den ganzen Tag herab geschienen hatten. Nur der Schein der Straßenlaternen erhellte die Nacht. Die Sterne am Himmel funkelten durch die aufsteigende Wärme und Harry hatte fast den Eindruck, sie würden am Himmel tanzen. Er legte sich in Tante Petunias Garten und schaute einige Zeit in den Himmel. Eine einzelne Sternschnuppe flog hoch oben am Himmel vorbei und Harry wünschte sich etwas. Immerhin hatte er Morgen Geburtstag und er nahm nicht an, dass Arabella das wusste. Man könnte es ihr zwar gesagt haben, aber da war er sich nicht sicher. Da das Gras doch inzwischen etwas unangenehm war, entschied sich Harry nach einem prüfenden Blick in alle Richtungen es auf magische Art abzuschneiden. Er zog seinen Zauberstab und murmelte etwas. Viele kleine Funken sprühten aus seiner Spitze hervor und überdeckten die gesamte Rasenfläche bis auf den kleinsten Grashalm in der hintersten Ecke. Fast hatte es den Anschein, dass das Gras rückwärts wachsen würde. Als es die richtige Länge hatte, stoppte Harry den Zauber und legte sich wieder hin. Nachdem er etwa eine dreiviertel Stunde dort gelegen hatte, überkam ihn die Müdigkeit. Er entschloss sich aufzustehen und zurück zu Arabellas Haus zu laufen. Dort würde er ein gemütliches Bett vorfinden, in dem er schlafen konnte.

In seinem Bett, unter die Decke gekuschelt, schlief Harry friedlich ein und überlegte, was ihm wohl der Morgen bringen würde. \accentuate{Nicht viel}, dachte Harry. Nach dem üblichen morgendlichen Joggen und dem darauffolgenden Frühstück werden vielleicht ein paar Geburtstagsgrüße eintreffen. Ansonsten konnte er ja immer noch Hausaufgaben machen.

Am nächsten Morgen stand Harry wieder sehr früh auf, um draußen joggen zu gehen. Er zog sich an, ging die Treppen hinunter, öffnete die Haustüre und fing nach seinen üblichen Aufwärmübungen an zu laufen. Harry dachte nicht groß über seinen Geburtstag nach. \gedanke{Ein paar Glückwunschkarten von Ron und Hermine, vielleicht ein kleines Geschenk}, dachte sich Harry. \gedanke{Da ich nichts zu erwarten habe, laufe ich ein bisschen länger.} Wieder im Haus sagte er Arabella Bescheid und ging gleich die Treppe hoch um sich zu duschen. Er zog sich frische Sachen an und kam die Treppe herunter. Gerade wollte er den Gang entlang in die Küche laufen, als ihm Arabella entgegenkam und meinte.

\enquote{Harry, schau lieber mal nach deiner Eule. Als ich vorhin in deinem Zimmer war, schaute sie gar nicht gut aus.}

Harry drehte sich wieder um und ging die Treppen hoch. Arabella folgte ihm. Als er sein Zimmer erreichte, Hedwig intensiv angeschaut hatte und nichts Außergewöhnliches bemerkt hatte, drehte er sich zu Arabella um. 

\enquote{Ich sehe nichts Ungewöhnliches}, meinte er.

\enquote{Komisch, vorhin hatte sie richtig schlapp und Müde ausgesehen. Aber vielleicht habe ich mir das auch nur eingebildet. Ich habe hier nicht so oft Eulen. Tut mir leid}, sagte sie.

\enquote{Macht nichts. Hätte ja sein können.}

Harry ging wieder die Treppen hinunter und Arabella folgte ihm. Er ging in die Küche und sein Gesicht versteinerte.

\trenn

\accentuate{Einen Tag zuvor, an einer ganz anderen Stelle in England.}

Ein kleines Wohnzimmer, mit alten Möbeln aus Nussbaum war die Hauptkulisse der folgenden Szenen. Die Tapete an den Wänden war schon leicht vergilbt, da sie jahrzehntelang nicht gewechselt worden war. Seit der Geburt ihres Enkels hatte sie diesen Raum nicht mehr verändert. Die Möbel spiegelten die vergangenen Tage wider; alt aber wertvoll. Im hinteren Teil war ein Fenster, vor dem ein grobmaschiger, weißer Vorhang hing. In den Schränken mit Glasfront waren Gläser zu sehen: Champagnergläser, Cognacgläser, Weingläser und andere Gläser, die Muggel als eigentümlich bezeichnen würden. Sektflöten ohne Standteller, gläserne Pfeifen, die mit Flüssigkeit gefüllt werden konnten und beim Ziehen am Mundstück die Flüssigkeit als feinen Nebel abgaben um die Lungen zu reinigen. Direkt vor dem Fenster stand ein Sofa, welches mit einer Decke auf der Sitzfläche gegen durch Sitzen geschont wurde.

Im vorderen Teil hingegen waren helle und modernere Möbel zu sehen. Wäre eine Mauer durch den Raum gezogen, würde es nicht so Kontrastreich ausfallen. Der Tisch aus Buche, sowie die sechs Stühle drumherum, dienten als Esstisch für sämtliche Mahlzeiten.

Doch heute stand außer der üblichen Deko, einige schwebende Kugeln, frische Blumen in einer Vase, einem kleinen schmalen Läufer über dem Tisch, auch noch eine Geburtstagstorte mit Kerzen. Diese begannen gerade Feuer zu fangen, als von draußen Stimmen und Geräusche zu hören waren.

Die Haustür wurde geöffnet und man hörte zwei Paar Füße, die sich auf dem Schuhabstreifer abputzten.

\enquote{Augen zu}, sagte eine weibliche Stimme.

\enquote{Aber Grandma, bin ich dafür nicht schon ein wenig zu alt?}

\enquote{Nein, nun mach schon.} Kurz darauf stand der junge Mann vor der Tür und die alte Dame direkt hinter ihm.  Die Tür ging auf und beide traten hinein. \enquote{Augen auf.}

Neville öffnete seine Augen. Vor ihm stand eine Torte. Wie jedes Jahr. Und ein kleines verpacktes Geschenk. Doch kaum hatte sich die Tür hinter ihnen geschlossen und Neville die Kerzen aus gepustet, klopfte es.

\enquote{Machst du mal auf?}, bat ihn seine Oma. 

Neville nickte und öffnete die Tür. Erstaunt blieb er stehen und bewegte sich nicht. 

Dean, Seamus und Luna standen davor. \enquote{Alles Gute zum Geburtstag, Neville}, sagten die drei. \enquote{Dürfen wir hereinkommen?}

\enquote{Aber \gst klar doch, kommt rein.} Neville trat zur Seite und lies seine Freunde ein.

\enquote{Ron und Hermine haben mir ein Geschenk mitgegeben. Sie stecken noch in den Vorbereitungen für Harrys Feier}, sagte Luna und überreichte Neville ein kleines Geschenk.

Dankend nahm er es an und bat seine Gäste sich zu setzen. Dann sah er fragend seine Oma an.

\enquote{Du wirst nur einmal sechzehn, Neville}, sagte sie und verschwand. \enquote{Viel Spaß ihr vier.} Dann schloss sie die Tür hinter sich und verzog sich in die Küche. Während der Feier tauchte sie nur noch ein paar Mal auf um Nachschub an Getränken oder Nahrung zu liefern.

Nachdem alle ein Stück Kuchen gegessen hatten, fragte Dean vorsichtig nach. \enquote{Hast du deine Eltern heute schon besucht?}

\enquote{Ja}, antwortete Neville niedergeschlagen. \enquote{Keine Änderung. Sie erkennen niemanden.}

Luna legte eine Hand auf seinen Arm, was ihm ein Lächeln entlockte. Dean und Seasmus wechselten nur ein paar Blicke. Dann widmete sich Neville seinen Geschenken. Dean und Seasmus schenkten ihm ein Buch über seltene Pflanzen. Von seiner Oma bekam er dieses Jahr die Erlaubnis, sich um den Garten zu kümmern. Er konnte ihn dabei so gestalten wie er es wollte. Neville war begeistert. Nun konnte er mit Pflanzen und anderen Gewächsen experimentieren. Als er das Geschenk von Luna in Händen hielt, zögerte er kurz, öffnete es dann doch. Er wusste nicht, was ihn erwarten würde. Umso erstaunter war er, als er einen kleinen Pflanzenzögling fand.

\enquote{Das sind Lenkpflaumen. Dad hat mir einen Ableger gegeben. Wenn du dich um sie kümmerst, sind sie dir treu. Du kannst sie vor dein Haus pflanzen. Dann halten sie Feinde ab, die dir etwas Böses wollen. Zumindest warnen sie dich.}

Neville bedankte sich. Gerade wollte er fragen, was sie denn geplant hatten, als über dem Tisch eine kleine, leuchtende Kugel auftauchte. Diese wurde größer und verdampfte dann in einem Nebel. An seiner Stelle erschien ein Päckchen. Es senkte sich langsam auf den Tisch und bewegte sich auf Neville zu. Das Paket war in Geschenkpapier eingewickelt. Unter der Schleife war ein Brief.

Sorgfältig überprüfte Neville das Geschenk mit sämtlichen ihm bekannten Zaubern. Er achtete sorgfältig darauf, dass seine Großmutter davon nichts mitbekam. Er sollte zwar seinen Zauberstab immer bei sich tragen, aber das Zaubern, das hatte sie ihm nahe gelegt, sollte er sein lassen. Es sei denn es gäbe einen triftigen Grund dafür. Dann erst griff er nach dem Geschenk und zog den Brief heraus. Er entfaltete ihn und las halblaut vor:

\begin{brief}
Sehr geehrter Mister Longbottom,

vor etwa fünfzehn Jahren habe ich ihren Eltern ein Versprechen gegeben, ihnen zu ihrem sechzehnten Geburtstag dieses Geschenk zu überreichen. Ihre Eltern empfanden es für wichtig genug, es mir und nicht ihrer Großmutter anzuvertrauen. Ich selber habe keine Ahnung, worum es sich dabei handelt, da ich das Paket niemals geöffnet habe. Ein Brief im Inneren soll sie näher informieren, nachdem sie die Anweisungen ihrer Eltern, die ich hier wiedergeben möchte, gelesen haben.
\end{brief}

Dann wechselte die Schrift und er erkannte die Schrift seiner Mutter. Zumindest stimmte sie mit alten Briefen überein, die ihm seine Großmutter gezeigt hatte. 

\begin{brief}
Lieber Neville,

Es tut uns leid, dass wir dir dein Geschenk nicht persönlich überreichen können, aber unser Beruf als Auror ist nun mal gefährlich. Wir wollten aber, dass du es auf jeden Fall bekommst und auch nutzen kannst. Solltest du diese Zeilen lesen, sind wir vermutlich Tod. Packe dein Geschenk erst einmal aus und hole es heraus. Dann betrachte es mit deinen Freunden und ratet, was es sein könnte. Dann erst bitte liest du deinen Brief, der darunter (unter der Uhr) im Paket liegt.

Außerdem ist noch eine kleine Ausgabe davon eine Lage darunter. Diese ist für deine Hosentasche gedacht.
Alles Gute zum Geburtstag,
\signumspace
Mum und Dad
\end{brief}

Dann fand er noch zwei Unterschriften, die nur Mum und Dad lauteten. Er legte gerade den Brief zur Seite, als seine Oma hereinkam.

\enquote{Oh, bist du noch gar nicht fertig mit auspacken?}, fragte sie.

Neville schüttelte nur stumm den Kopf und sah auf das Päckchen, das vor ihm lag. Seine Großmutter setzte sich neben ihn hin und betrachtete es nun auch. Ihre Augen flogen über den Brief. Endlich nahm Neville es in die Hand und entfernte die Schleife. Dann entfernte er das Geschenkpapier. Heraus kam ein komplett schwarzes Pappkästchen, in dem z.B. Uhren verpackt waren. Vorsichtig drehte er es herum und besah sich die Verpackung. Dann begriff er, dass er nur den Deckel anheben musste, der bis zum Boden reichte.

Als der den Deckel hob und alle sahen, was darin lag, presste seine Großmutter ihre Hand vor den Mund. \enquote{Das habe ich seit Jahren vermisst.}

\enquote{Wie meinst du das?}

\enquote{Das hat Frank gehört. Eines Tages war es verschwunden. Ich wollte ihn noch fragen, wo er es hin hatte, aber dann war es auch schon zu Spät.}

Neville nahm die Sanduhr heraus. Das Glas wurde durch ein Gestänge aus grünem Glas an den Längsseiten und oben und unten durch einen Messingring gehalten. Die drei Glasstreben waren oben und unten eingedreht. Neville hatte das Gefühl, dass die Sanduhr selbst nichts wog. Er wollte sie gerade in der Hand wiegen, da bemerkte er, dass sie von selbst in der Luft stehen blieb. Waagerecht lag sie nun in der Luft. Der Sand im Inneren war auf beide Seiten der Uhr gleichmäßig verteilt. Doch er lag nicht auf dem Boden, sondern war an den Glasböden der Sanduhr. So, als ob es im Inneren der Uhr eine andere Schwerkraft geben würde, als außerhalb. So langsam fing der Sand an sich zu bewegen. Direkt auf die Verengung zu und, nachdem er durch war auf der anderen Seite auf den Sand zu.

\enquote{Ich würde sagen, das ist eine Sanduhr}, sagte Dean.

\enquote{Das ist zu offensichtlich, Dean. Hast du nicht gehört, was Neville vorgelesen hat?}, fragte Luna. \enquote{Wir sollen raten, was es ist. Eine Sanduhr kennt wohl jeder. Es muss eine besondere sein.}

\enquote{Hast du eine Ahnung, Granny?}

\enquote{Du hast den Brief doch vorgelesen. Du wirst schon dahinter kommen.} Dann stand sie auf und ging.

Nach einer Weile erfolglosen Ratens, gaben die vier auf und Neville nahm den Brief, der unter der Uhr lag heraus und begann ihn zu lesen.

\begin{brief}
Mein Sohn,

Diese Uhr wurde mir von meinem Vater vererbt und dieser hat sie von seinem Vater bekommen. Ich hoffe, dass auch du eines Tages einen Sohn oder eine Tochter hast, dem oder der du diese Uhr vererben kannst.

Diese Uhr hat mehrere Funktionen. Je nachdem, welche du gerade brauchst. Einmal zeigt sie dir die Zeit an. Jeder Person, auf die die Uhr geprägt ist, kann die aktuelle Zeit ablesen. Frag mich nicht, wie es funktioniert, aber wenn du die Uhr siehst, dann weißt du, wie spät es ist. Als weitere Funktion zeigt dir die Uhr an, ob ein Gespräch auch einen Wahrheitsgehalt besitzt, oder nicht. Je nachdem wie langsam oder schnell der Sand verrinnt. Je langsamer, desto mehr der Wahrheit entspricht das Gespräch.

Außerdem gibt es noch eine kleine Ausgabe. Hole sie jetzt bitte heraus.
\end{brief}

Neville kam dem nach und suchte in der Schachtel eine kleine Ausgabe der Eieruhr. Er nahm sie in die Hand und wog auch diese. Diese jedoch war etwas schwerer als die große Ausgabe. Da diese aber nichts wog, war das keine Kunst. Er schloss seine Faust und las weiter.

\begin{brief}
Trage diese Uhr immer bei dir. Sie wird dir mehr als nützlich sein. Sie hat ein paar außergewöhnliche Eigenschaften. Eine ist, dass nur du sie in deiner Tasche spüren kannst und sonst kein anderer. Man kann sie dir nicht wegnehmen. Zudem hast du immer die richtige Zeit, wenn du sie brauchst. Aber das Beste ist, man kann dir keinen fremden Willen aufzwingen. Ja, du verstehst mich richtig. Diese Uhr schützt dich vor einem Fluch, den man Imperius nennt. Damit können Todesser (ich hoffe, sie sind bereits seit langer Zeit Geschichte) einem den eigenen Willen aufzwingen. Das ist bei dir dann nicht mehr möglich.
\end{brief}

Neville staunte, sah sich die kleine Uhr noch einmal an und steckte sie dann in seine Tasche. Er fühlte zwar keine Veränderung, wusste aber sofort, wie spät es war. Neville lächelte leicht in sich hinein und verbrachte den Rest des Nachmittags mit seinen Gästen. Dean und Seamus verabschiedeten sich kurz nach dem Abendessen und machten sich auf die Heimreise. Luna blieb eine Stunde länger und unterhielt sich mit Neville über ihr Geschenk. Sie besprachen, wie er sie am besten zu Pflegen habe und wie sie zu vermehren seien. Neville erfuhr, dass sie eine der seltensten Pflanzen seien, die die Zaubererwelt kennt. Nur die wenigsten Pflanzenjäger würden sie allerdings als nützlich oder sinnvoll ansehen.

Mit einer herzlichen Umarmung verabschiedete sich Luna von Neville und machte sich ebenfalls auf die Heimreise.

\trenn

Während Neville und seine Gäste noch feierten, saßen die Todesser in einer großen, leeren und grauen Halle zusammen an einem langen Tisch. Voldemort sah in die Runde und sprach dann: \enquote{Wir sollten die Drachen auf unsere Seite ziehen. Sie sind mächtige Wesen, die dem Ministerium und Hogwarts eine Menge Schaden können.}

\enquote{Wenn ich darf, mein Lord?}, fragte Snape nach und fuhr nach einem Nicken Voldemorts fort. \enquote{Ich rate euch davon ab, die Drachen für eure Zwecke einzusetzen. Dies könnte zu Problemen führen, da einer der Drachen innerhalb unserer Reichweite vom Wildhüter Hagrid aufgezogen und dieser auf ihn geprägt ist. Außerdem ist einer der Drachenbändiger Charlie Weasley. Ein Freund von Hogwarts. Das könnte ein Problem darstellen.}

\enquote{Dann müssen wir diesen Wildhüter und Weasley aus dem Weg schaffen.}

\enquote{Das wird nicht einfach. Der Riese ist schwer zu bezwingen. Umbridge hat es letztes Jahr versucht und dieser Weasley kann sich sehr gut verteidigen. Das habe ich schon während seiner Schulzeit erlebt. Zudem hat er es mit Drachen zu tun, das stärkt auch.}

\enquote{Was schlägst du vor?}, fragte Voldemort.

\enquote{Wenn ich darf?}, fragte Yaxley. Voldemort nickte nur. \enquote{Es gibt Gerüchte über ein Artefakt, das Drachen schaden kann. Es dürfte in der Lage sein, sie zu vernichten. Diese Gerüchte besagen, dass dieser Gegenstand an einer Klippe sein soll. Ich meine auch mich zu erinnern, dass es in einem Buch mehr Informationen darüber gibt. Ich kann mich auf die Suche machen, mein Lord.}

\enquote{Tu das}, sagte Voldemort und dachte weiterhin nach. \enquote{Als Ausweichplan. \gst Nott, Macneir, ihr werdet mal sondieren, wie man Drachen schaden kann. Begebt euch nach Rumänien und sondiert die Lage. Berichtet mir dann.}

Die beiden nickten, standen auf und verließen den Raum. 




\begin{kommentar}
Tante Petunia bekommt einen Brief per Eule. Das ist schon mal per se ungewöhnlich. Aber ihre Reaktion darauf ist noch ungewöhnlicher. Hier habe ich bereits die Idee aufgegriffen, dass Petunia kein normaler Muggel ist, sondern eine Hexe und zusammen mit ihrer Schwester Lily Hogwarts besucht hat. Petunia hat sich später bereit erklärt, ihren Neffen zu schützen, auf die Magie weitgehend zu verzichten und ihn nicht besonders gut zu behandeln. Dies hatte allerdings zur Folge, dass sich keiner in der magischen Welt mehr daran erinnern konnte, dass Petunia in Hogwarts zur Schule gegangen war. Wie Dumbledore schon gesagt hat: Es gibt Magie, die nur durch Taten gewirkt wird. Dabei bezog er sich auf den Schutz Harrys vor Voldemort. Die Magie, die Harrys Tante wirkt, verstärkt den Schutz, den Harry durch seine Mutter bekommen hat zusätzlich.
\end{kommentar}

\begin{kommentar}
Etwas später, kurz bevor Harry zu Miss Figg geht, meint seine Tante, dass er die Eulenstange mitnehmen soll. Ein weiterer Hinweis, dass Petunia scheinbar weiß, dass Miss Figg eine Squib ist.
\end{kommentar}

\begin{kommentar}
Nevilles Geburtstag: Neville bekommt an seinem sechzehnten Geburtstag ein mysteriöses Päckchen. Dieses hatte Frederick Elber, von dem noch keiner weiß, dass er später auf Hogwarts unterrichten wird, aufbewahrt, damit es Neville rechtzeitig bekommt.
\end{kommentar}

\begin{kommentar}
Am Ende des Kapitels will Voldemort noch die Drachen auf seine Seite ziehen oder sie aus dem Verkehr bringen. Dabei soll ihm der Drachenstein helfen, der erst später im weiteren Verlauf genannt wird.
\end{kommentar}
